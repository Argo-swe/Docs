\par Le metriche di prodotto sono indicatori quantitativi e qualitativi utilizzati per valutare in modo obiettivo le caratteristiche del software. L'applicazione di queste metriche mira ad assicurare la conformità del software agli standard di qualità e ad aumentare il grado di soddisfazione del cliente.

\bgroup
\begin{adjustwidth}{-0.5cm}{-0.5cm}
	% MAX 12.5cm
	\begin{longtable}{|P{1.5cm}|P{4cm}|P{3.5cm}|>{\arraybackslash}P{3.5cm}|}
	  \hline
		\textbf{ID} & \textbf{Nome metrica} & \textbf{Valore tollerabile} & \textbf{Valore ambito} \\ 
		\hline
		\endfirsthead

		\hline
		\textbf{ID} & \textbf{Nome metrica} & \textbf{Valore tollerabile} & \textbf{Valore ambito} \\ 
		\hline
		\endhead

		\hline
		\multicolumn{4}{|r|}{{Continua nella prossima pagina}} \\ 
		\hline
		\endfoot

		\hline
		\endlastfoot
		
		M.PD.1 & Requisiti obbligatori soddisfatti & 100\% & 100\% \\
		\hline M.PD.2 & Requisiti desiderabili soddisfatti & - & - \\
		\hline M.PD.3 & Requisiti opzionali soddisfatti & - & - \\
		\hline M.PD.4 & Indice Gulpease & $\geq$ 50\% & $\geq$ 80\% \\
		\hline M.PD.5 & Browser supportati & 80\% & 100\% \\
		\hline M.PD.6 & Profondità (click necessari per reperire un'informazione) & 4-5 click & 3 click \\
		\hline M.PD.7 & Ampiezza (opzioni nel menu di navigazione principale) & 8 opzioni & 4 opzioni \\
		\hline M.PD.8 & Tempo di apprendimento & - & - \\
		\hline M.PD.9 & Tempo di risposta & - & - \\
		\hline M.PD.10 & Code coverage & $\geq$ 75\% & $\geq$ 90\% \\
		\hline M.PD.11 & Branch coverage & $\geq$ 70\% & $\geq$ 80\% \\
		\hline M.PD.12 & Statement coverage & $\geq$ 70\% & $\geq$ 80\% \\
		\hline M.PD.13 & Linee medie di codice per metodo & $\leq$ 40 & $\leq$ 20 \\
		\hline M.PD.14 & Complessità ciclomatica & $\leq$ 10 & $\leq$ 5 \\
		\hline M.PD.15 & Accoppiamento delle classi & $\leq$ 4 & $\leq$ 2 \\
		\hline M.PD.16 & Indice di manutenibilità & - & - \\
		\hline M.PD.17 & Percentuale di test superati & 80\% & 100\% \\
		\hline M.PD.18 & Accuratezza della risposta & $\geq$ 70\% & $\geq$ 90\% \\
    \hline M.PD.19 & Gestione degli errori & 70\% & 100\% \\
		\hline M.PD.20 & Impatto delle modifiche & $\leq$ 15\% & $\leq$ 5\% \\
    \end{longtable}
\end{adjustwidth}
\egroup


\subsubsection{Tracciamento delle metriche di prodotto}

\bgroup
\begin{adjustwidth}{-0.5cm}{-0.5cm}
	\centering
	% MAX 12.5cm
  \begin{longtable}{|P{3cm}|>{\raggedright}P{5.5cm}|>{\arraybackslash}P{4cm}|}
		\caption{Tracciamento metriche di prodotto}
  	\label{tab:tracciamento-metriche-prodotto} \\
    \hline
		\textbf{Caratteristica} & \textbf{Descrizione} & \textbf{Metriche}\\ 
		\hline
		\endfirsthead

		\caption[]{Tracciamento metriche di prodotto (continua)} \\
		\hline
		\textbf{Caratteristica} & \textbf{Caratteristica} & \textbf{Metriche}\\ 
		\hline
		\endhead

		\hline
		\multicolumn{3}{|r|}{{Continua nella prossima pagina}} \\ 
		\hline
		\endfoot

		\hline
		\endlastfoot

    Funzionalità & Il software deve implementare i requisiti riportati nel documento \AdR. & M.PD.1, M.PD.2, M.PD.3, M.PD.18 \\
		\hline Compatibilità & L'applicazione web deve essere compatibile con i seguenti browser:
		\begin{itemize}
			\item Mozilla Firefox;
			\item Google Chrome;
			\item Safari;
			\item Microsoft Edge;
			\item Opera.
		\end{itemize}
		& M.PD.5 \\
		\hline Usabilità & Il software deve facilitare l'interazione e il reperimento delle informazioni da parte dell'utente, senza provocare sovraccarico cognitivo. & M.PD.4, M.PD.6, M.PD.7, M.PD.8  \\
		\hline Efficienza & Il software deve fornire prestazioni adeguate in relazione alla quantità di risorse usate. & M.PD.9 \\
		\hline Affidabilità & Il software deve rispettare le specifiche tecniche di funzionamento nel tempo. & M.PD.17, M.PD.19 \\
		\hline Manutenibilità & Il software deve poter essere modificato senza richiedere uno sforzo eccessivo in termini di tempo e costi. & M.PD.10, M.PD.11, M.PD.12, M.PD.13, M.PD.14, M.PD.15, M.PD.16, M.PD.20 \\
  \end{longtable}
\end{adjustwidth}
\egroup
