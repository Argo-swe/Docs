\section{Verifica}

\par Il collaudo del software è un insieme di attività volte a garantire il soddisfacimento degli obiettivi di qualità. La fase di verifica può aiutare il team a identificare e risolvere con prontezza anomalie legate alle componenti software. Pertanto, il gruppo si impegna a eseguire i test contestualmente alle attività di sviluppo. Con questa procedura, il team si aspetta di ridurre l’impatto degli errori, garantendo il rispetto del budget e dei tempi previsti. Il gruppo ha individuato quattro classi di test finalizzate ad assicurare la correttezza, completezza e affidabilità del software:
\begin{itemize}
  \item \textbf{Test di unità}: attività di collaudo di singole \glossario{unità} del software;
  \item \textbf{Test di integrazione}: verificano che i diversi moduli, componenti o servizi utilizzati dall'applicazione funzionino in modo integrato;
  \item \textbf{Test di sistema}: controllano il comportamento del sistema nel suo complesso e verificano che l'applicazione funzioni secondo i requisiti specificati;
  \item \textbf{Test di accettazione}: sono test formali che precedono il rilascio del prodotto e valutano se l'applicazione è conforme alle aspettative del cliente.
\end{itemize}

\subsection{Test di unità}

\par Lo scopo dei test di unità è verificare il corretto funzionamento delle "unità software", ossia delle porzioni o segmenti (come una funzione, una classe o un componente) testabili in modo autonomo e isolato all'interno del sistema. Di seguito è riportato l'elenco dei test di unità:

\bgroup
\begin{adjustwidth}{-0.5cm}{-0.5cm}
	% MAX 12.5cm
 	\begin{longtable}{|P{1.5cm}|>{\raggedright}P{9.5cm}|>{\arraybackslash}P{1.5cm}|}
		\caption{Test di unità}
  	\label{tab:test-unita} \\
	  \hline
		\textbf{ID} & \textbf{Descrizione} & \textbf{Stato} \\ 
		\hline
		\endfirsthead

		\caption[]{Test di unità (continua)} \\
		\hline
		\textbf{ID} & \textbf{Descrizione} & \textbf{Stato} \\ 
		\hline
		\endhead

		\hline
		\multicolumn{3}{|r|}{{Continua nella prossima pagina}} \\ 
		\hline
		\endfoot

		\hline
		\endlastfoot

		% Esempio struttura test unità back-end
		%\hline TU & Verificare che il metodo "get_user" della classe "authentication_repository_adapter" soddisfi le seguenti condizioni:
		%\begin{itemize}
			%\item Il metodo deve restituire false se ...;
			%\item Il metodo deve restituire true se ...;
			%\item Il metodo deve sollevare un'eccezione se ...;
    %\end{itemize} & S \\
		
		TU & Verificare che il componente "AppLogo" soddisfi le seguenti condizioni:
    \begin{itemize}
      \item Il logo deve essere renderizzato con il percorso dell'immagine corretto;
			\item L'altezza e la larghezza dell'immagine devono essere impostate con valori predefiniti se non vengono passate come props;
			\item Il colore dell'immagine deve cambiare in base al tema selezionato.
    \end{itemize} & S \\
		\hline TU & Verificare che il componente "AppFooter" soddisfi le seguenti condizioni:
    \begin{itemize}
      \item Il footer deve essere renderizzato correttamente;
			\item All'interno del footer deve essere visualizzata la versione corretta dell'applicazione.
    \end{itemize} & S \\
		\hline TU & Verificare che il componente "AppMenu" soddisfi le seguenti condizioni:
    \begin{itemize}
      \item Il menu dell'Utente deve avere avere una o più opzioni;
			\item Il menu del Tecnico deve avere due o più opzioni.
    \end{itemize} & S \\
		\hline TU & Verificare che il componente "AppMenuItem" soddisfi le seguenti condizioni:
    \begin{itemize}
      \item La voce di menu deve contenere l'etichetta corretta;
			\item La voce di menu non deve essere attiva se non è stata selezionata;
			\item Una volta selezionata, la voce di menu deve essere attiva;
			\item La voce di menu può essere la radice di un menu;
			\item La voce di menu può contenere un sotto-menu;
			\item La voce di menu può aprire e chiudere un sotto-menu. 
    \end{itemize} & S \\
		\hline TU & Verificare che il componente "AppTopbar" soddisfi le seguenti condizioni:
    \begin{itemize}
      \item Il pulsante di login deve essere visibile se l'Utente non ha effettuato l'accesso;
			\item Il pulsante di logout deve essere visibile se l'Utente ha effettuato l'accesso;
			\item Quando viene cliccato il pulsante di logout, la funzione di logout deve essere chiamata;
			\item Il menu delle opzioni deve essere aperto al clic dell'apposito pulsante;
			\item Il menu delle opzioni deve essere chiuso quando l'Utente clicca al di fuori di esso.
    \end{itemize} & S \\
		\hline TU & Verificare che il componente "ConfigSidebar" soddisfi le seguenti condizioni:
    \begin{itemize}
      \item La sezione per modificare la scala deve essere visibile;
			\item La sezione per modificare il tema deve essere visibile;
			\item La sezione per modificare la lingua dell'interfaccia deve essere visibile;
			\item La scala deve essere diminuita al clic del'apposito pulsante;
			\item La scala deve essere aumentata al clic dell'apposito pulsante;
			\item Il pulsante per diminuire la scala deve essere disabilitato se la scala è al minimo;
			\item Il pulsante per aumentare la scala deve essere disabilitato se la scala è al massimo;
			\item Il contenuto del localStorage deve essere aggiornato al cambio del tema;
			\item Il contenuto del localStorage deve essere aggiornato al cambio della lingua.
    \end{itemize} & S \\
		\hline TU & Verificare che il componente "MenuSidebar" soddisfi le seguenti condizioni:
    \begin{itemize}
			\item Il pulsante per chiudere la sidebar non deve essere visibile su schermi di grandi dimensioni.
      \item Il pulsante per chiudere la sidebar deve essere visibile su schermi di piccole dimensioni.
    \end{itemize} & S \\
		\hline TU & Verificare che il componente "LoginDialog" soddisfi le seguenti condizioni:
    \begin{itemize}
      \item Il dialog deve essere chiuso al clic dell'apposito pulsante;
			\item La funzione "messageSuccess" deve essere chiamata se il login viene effettuato con successo;
			\item Una volta completato il login, il dialog deve essere chiuso;
			\item Il contenuto del localStorage deve essere aggiornato se il login viene effettuato con successo;
			\item Una volta completato il login, il form deve essere resettato;
			\item La funzione "messageError" deve essere chiamata se il login fallisce.
    \end{itemize} & S \\
		\hline TU & Verificare che il componente "DictPreview" soddisfi le seguenti condizioni:
    \begin{itemize}
      \item L'anteprima del dizionario dati non deve essere visibile se la variabile DetailsVisible è uguale a false;
			\item Se è visibile, l'anteprima deve mostrare i dati corretti;
			\item Lo stato di espansione dell'anteprima deve essere attivato/disattivato al clic dell'apposito pulsante;
			\item Quando viene cliccato il pulsante di chiusura, il componente deve inviare un segnale.
    \end{itemize} & S \\
		\hline TU & Verificare che il componente "ChatMessage" soddisfi le seguenti condizioni:
    \begin{itemize}
      \item Il messaggio inviato dall'Utente deve essere visualizzato correttamente;
			\item Il messaggio inviato dal ChatBOT deve essere visualizzato correttamente;
			\item I pulsanti di azione devono essere nascosti se il messaggio è stato inviato dall'Utente;
			\item La funzione "Copia negli appunti" deve essere chiamata con i parametri corretti al clic dell'apposito pulsante;
			\item Il pulsante di debug deve essere nascosto se l'Utente non ha effettuato il login;
			\item La funzione "Apri debug modal" deve essere chiamata con i parametri corretti al clic dell'apposito pulsante.
    \end{itemize} & S \\
		\hline TU & Verificare che il componente "DebugMessage" soddisfi le seguenti condizioni:
    \begin{itemize}
      \item Il componente deve visualizzare il messaggio di debug correttamente;
			\item Quando viene cliccato il pulsante "Scarica file", la funzione di download deve essere chiamata con i parametri corretti.
    \end{itemize} & S \\
		\hline TU & Verificare che il componente "ChatDeleteBtn" soddisfi le seguenti condizioni:
    \begin{itemize}
      \item Il pulsante deve essere disabilitato se non è presente alcun messaggio nella chat;
      \item Il pulsante deve essere disabilitato se il ChatBOT sta elaborando un nuovo messaggio;
      \item Quando viene cliccato, il pulsante deve inviare un segnale.
    \end{itemize} & S \\
		\hline TU & Verificare che il componente "CreateUpdateDictionaryModal" soddisfi le seguenti condizioni:
    \begin{itemize}
      \item Il form di creazione e modifica deve essere visualizzato correttamente;
			\item Il pulsante di submit deve essere disabilitato se il form è incompleto;
			\item La funzione "messageSuccess" deve essere chiamata se l'inserimento viene effettuato con successo;
			\item La funzione "messageError" deve essere chiamata se l'inserimento fallisce;
			\item Il pulsante di submit deve essere disabilitato se il formato del file non è valido;
			\item La funzione "messageSuccess" deve essere chiamata se l'aggiornamento dei metadati viene effettuato con successo;
			\item La funzione "messageError" deve essere chiamata se l'aggiornamento dei metadati fallisce;
			\item La funzione "messageSuccess" deve essere chiamata se l'aggiornamento del file viene effettuato con successo;
			\item La funzione "messageError" deve essere chiamata se l'aggiornamento del file fallisce;
			\item L'upload del file deve essere disabilitato se un file è già stato caricato;
			\item Il file selezionato deve essere rimosso al clic dell'apposito pulsante.
    \end{itemize} & S \\
		\hline TU & Verificare che il metodo "logout" della classe "auth.service" soddisfi le seguenti condizioni:
		\begin{itemize}
			\item Il metodo deve rimuovere il token di autenticazione dal localStorage;
			\item Il metodo deve inviare un segnale di notifica.
    \end{itemize} & S \\
		\hline TU & Verificare che il metodo "isLogged" della classe "auth.service" soddisfi le seguenti condizioni:
		\begin{itemize}
			\item Il metodo deve restituire false se il token di autenticazione non è presente;
			\item Il metodo deve restituire false se il token è scaduto;
			\item Il metodo deve restituire true se il token è valido;
			\item Il metodo deve restituire false se la data di scadenza non è definita.
    \end{itemize} & S \\
		\hline TU & Verificare che il metodo "downloadFile" della classe "utils.service" soddisfi le seguenti condizioni:
		\begin{itemize}
			\item Il metodo deve creare un collegamento per il download del file;
			\item Il metodo deve attivare il download cliccando sul collegamento.
    \end{itemize} & S \\
		\hline TU & Verificare che il metodo "stringToSnakeCase" della classe "utils.service" soddisfi le seguenti condizioni:
		\begin{itemize}
			\item Il metodo deve convertire una stringa in snake_case.
    \end{itemize} & S \\
		\hline TU & Verificare che il metodo "addCapitalizeValues" della classe "utils.service" soddisfi le seguenti condizioni:
		\begin{itemize}
			\item Il metodo deve aggiungere chiavi in maiuscolo a un determinato oggetto.
    \end{itemize} & S \\
		\hline TU & Verificare che il metodo "capitalizeString" della classe "utils.service" soddisfi le seguenti condizioni:
		\begin{itemize}
			\item Il metodo deve convertire in maiuscolo la prima lettera di una stringa.
    \end{itemize} & S \\
	\end{longtable}
\end{adjustwidth}
\egroup
\subsection{Test di integrazione}

\par La sezione relativa ai test di integrazione verrà aggiornata in seguito alla revisione \glossario{RTB}.
\subsection{Test di sistema}

\par I test di sistema devono assicurare una completa copertura dei requisiti concordati con la \glossario{Proponente} e/o specificati nel documento di \AdR. Di seguito è riportato l'elenco dei test di sistema:

\bgroup
\begin{adjustwidth}{-0.5cm}{-0.5cm}
	% MAX 12.5cm
 	\begin{longtable}{|P{1.5cm}|>{\raggedright}P{7cm}|P{2.5cm}|>{\arraybackslash}P{1.5cm}|}
	  \hline
		\textbf{ID} & \textbf{Descrizione} & \textbf{Requisito} & \textbf{Stato} \\ 
		\hline
		\endfirsthead

		\hline
		\textbf{ID} & \textbf{Descrizione} & \textbf{Requisito} & \textbf{Stato} \\ 
		\hline
		\endhead

		\hline
		\multicolumn{4}{|r|}{{Continua nella prossima pagina}} \\ 
		\hline
		\endfoot

		\hline
		\endlastfoot

		TS.1 & Verificare che l'Utente possa effettuare il login per accedere alla sezione del tecnico. & RF.O.1 & / \\ 
    \hline TS.- & Verificare che l'Utente possa inserire lo username in fase di autenticazione. & RF.O.1.1 & / \\ 
    \hline TS.- & Verificare che l'Utente possa inserire la password in fase di autenticazione. & RF.O.1.2 & / \\
		\hline TS.- & Verificare che il Tecnico possa inserire un nuovo \glossario{dizionario dati} nel sistema. & RF.O.13 & / \\ 
		\hline TS.- & Verificare che il Tecnico possa eliminare un \glossario{dizionario dati} dal sistema. & RF.O.18 & / \\ 
		\hline TS.- & Verificare che l'Utente visualizzi un errore nel caso in cui il nome del \glossario{dizionario dati} contenga caratteri non supportati. & RF.O.31.1 & / \\ 
		\hline TS.- & Verificare che il sistema restituisca un messaggio di errore qualora l'Utente inserisca un nome già esistente per il \glossario{dizionario dati}. & RF.O.31.2 & / \\ 
		\hline TS.- & Verificare che l'Utente visualizzi un errore nel caso in cui la descrizione del \glossario{dizionario dati} contenga caratteri non supportati. & RF.O.1.2 & / \\ 
		\hline TS.- & Verificare che il sistema restituisca un messaggio di errore qualora l'Utente inserisca un \glossario{dizionario dati} non conforme allo schema predefinito. & RF.O.1.2 & / \\ 
		\hline TS.- & Verificare che il sistema restituisca un messaggio di errore qualora l'Utente inserisca un file troppo pesante. & RF.O.1.2 & / \\ 
		\hline TS.- & Verificare che l'Utente visualizzi un errore nel caso in cui l'estensione del file caricato non sia supportata. & RF.O.1.2 & / \\ 
		\hline TS.- & Verificare che l'Utente possa selezionare un \glossario{dizionario dati} da utilizzare nel sistema. & RF.O.1.2 & / \\ 
		\hline TS.- & Verificare che l'Utente possa visualizzare il contenuto del \glossario{dizionario dati} selezionato. & RF.O.1.2 & / \\ 
		\hline TS.- & Verificare che l'Utente possa selezionare una lingua per la generazione del \glossario{prompt}. & RF.O.1.2 & / \\ 
		\hline TS.- & Verificare che l'Utente possa selezionare un DBMS per la generazione del \glossario{prompt}. & RF.O.1.2 & / \\ 
		\hline TS.- & Verificare che l'Utente possa inserire un messaggio nella maschera di richiesta all'interno della chat. & RF.O.1.2 & / \\ 
		\hline TS.- & Verificare che l'Utente non possa inserire una richiesta senza aver selezionato un \glossario{dizionario dati}. & RF.O.1.2 & / \\ 
		\hline TS.- & Verificare che l'Utente possa inviare un messaggio al ChatBOT e richiedere la generazione di un \glossario{prompt}. & RF.O.1.2 & / \\ 
		\hline TS.- & Verificare che il Tecnico possa effettuare il logout per terminare la sessione corrente. & RF.O.1.2 & / \\ 
		\hline TS.- & Verificare che l'Utente possa visualizzare il \glossario{prompt} generato. Il prompt deve contenere le seguenti informazioni:
		\begin{itemize}
			\item Legenda dei simboli utilizzati;
			\item Lista delle tabelle pertinenti. Per ogni tabella devono essere riportate le seguenti informazioni:
			\begin{itemize}
				\item Schema della tabella: composto dal nome della tabella e da una lista di nomi e tipi di colonne;
				\item Chiave primaria;
				\item Descrizione della tabella;
				\item Descrizione delle colonne della tabella;
				\item Chiavi esterne;
			\end{itemize}
			\item DBMS di riferimento;
			\item Lingua di riferimento;
			\item Richiesta il linguaggio naturale.
		\end{itemize}
	 	& RF.O.1.2 & / \\ 
		\hline TS.- & Verificare che l'Utente possa copiare il contenuto del \glossario{prompt} generato. & RF.O.1.2 & / \\ 
		\hline TS.- & Verificare che il sistema restituisca un avviso qualora l'Utente inserisca una richiesta ritenuta non idonea dal modello di \glossario{AI}. & RF.O.1.2 & / \\ 
		\hline TS.- & Verificare che il sistema restituisca un messaggio di errore nel caso in cui la generazione del \glossario{prompt} venga interrotta senza preavviso. & RF.O.1.2 & / \\ 
		\hline TS.- & Verificare che il sistema generi un \glossario{log} se la richiesta viene inviata dal Tecnico. & RF.O.1.2 & / \\ 
		\hline TS.- & Verificare che il Tecnico possa visualizzare il \glossario{log} relativo all'ultima richiesta inviata. Il log deve contenere le seguenti informazioni:
    \begin{itemize}
      \item Data e ora di generazione del \glossario{log};
      \item Richiesta in linguaggio naturale;
      \item Prima fase della generazione del \glossario{prompt} (lista delle tabelle considerate rilevanti dal modello):
      \begin{itemize}
        \item Nome della tabella;
        \item Punteggio assegnato alla tabella;
        \item Descrizione della tabella;
        \item Classifica di importanza dei termini presenti nella descrizione della tabella;
        \item Descrizione della colonna più rilevante;
        \item Classifica di importanza dei termini presenti nella descrizione della colonna;
      \end{itemize}
      \item Seconda fase della generazione del \glossario{prompt} (lista delle tabelle pertinenti):
      \begin{itemize}
        \item Spiegazione del motivo per cui una tabella viene inserita o meno nel \glossario{prompt}.
      \end{itemize}
    \end{itemize}
		& RF.O.1.2 & / \\ 
		\hline TS.- & Verificare che il Tecnico possa scaricare un file di \glossario{log} contenente il debug della generazione del \glossario{prompt}. & RF.O.1.2 & / \\ 
		\hline TS.- & Verificare che l'Utente possa modificare il nome di un \glossario{dizionario dati}. & RF.O.1.2 & / \\ 
		\hline TS.- & Verificare che l'Utente possa modificare la descrizione di un \glossario{dizionario dati}. & RF.O.1.2 & / \\
		\hline TS.- & Verificare che l'Utente possa sovrascrivere il file di configurazione di un \glossario{dizionario dati}. & RF.O.1.2 & / \\
		\hline TS.- & Verificare che l'Utente possa visualizzare il contenuto della chat. & RF.O.1.2 & / \\  
		\hline TS.- & Verificare che l'Utente possa visualizzare i singoli messaggi nella chat. & RF.O.1.2 & / \\  
		\hline TS.- & Verificare che l'Utente possa visualizzare il contenuto di un messaggio nella chat. & RF.O.1.2 & / \\  
		\hline TS.- & Verificare che l'Utente possa visualizzare il mittente di un messaggio nella chat. & RF.O.1.2 & / \\  
		\hline TS.- &  & RF.O.1.2 & / \\  
		\hline TS.- &  & RF.O.1.2 & / \\  
		\hline TS.- &  & RF.O.1.2 & / \\  
		\hline TS.- &  & RF.O.1.2 & / \\  
		\hline TS.- &  & RF.O.1.2 & / \\  
		\hline TS.- &  & RF.O.1.2 & / \\  
		\hline TS.- &  & RF.O.1.2 & / \\  
		\hline TS.- &  & RF.O.1.2 & / \\  
		\hline TS.- &  & RF.O.1.2 & / \\    
	\end{longtable}
\end{adjustwidth}
\egroup

\subsection{Test di accettazione}

\par L'obiettivo dei test di accettazione è verificare se il sistema soddisfa le aspettative del Committente e del Proponente. I test di accettazione determinano se il software è pronto per essere rilasciato, e pertanto richiedono un focus sul comportamento degli utenti. Di seguito è riportato l'elenco dei test di accettazione:

\bgroup
\begin{adjustwidth}{-0.5cm}{-0.5cm}
	% MAX 12.5cm
 	\begin{longtable}{|P{1.5cm}|>{\raggedright}P{9.5cm}|>{\arraybackslash}P{1.5cm}|}
    \caption{Test di accettazione}
  	\label{tab:test-accettazione} \\
	  \hline
		\textbf{ID} & \textbf{Descrizione} & \textbf{Stato} \\ 
		\hline
		\endfirsthead

    \caption[]{Test di accettazione (continua)} \\
		\hline
		\textbf{ID} & \textbf{Descrizione} & \textbf{Stato} \\ 
		\hline
		\endhead

		\hline
		\multicolumn{3}{|r|}{{Continua nella prossima pagina}} \\ 
		\hline
		\endfoot

		\hline
		\endlastfoot

		TA.1 & & N-D \\
		\hline TA.2 & & N-D \\
	\end{longtable}
\end{adjustwidth}
\egroup

\subsubsection{Tracciamento dei test di accettazione}

\bgroup
\begin{adjustwidth}{-0.5cm}{-0.5cm}
	\centering
	% MAX 12.5cm
  \begin{longtable}{|c|c|}
		\caption{Tracciamento test di accettazione}
  	\label{tab:tracciamento-test-accettazione} \\
    \hline
		\textbf{ID} & \textbf{Caso d'uso} \\ 
		\hline
		\endfirsthead

		\caption[]{Tracciamento test di accettazione (continua)} \\
		\hline
		\textbf{ID} & \textbf{Caso d'uso} \\ 
		\hline
		\endhead

		\hline
		\multicolumn{2}{|r|}{{Continua nella prossima pagina}} \\ 
		\hline
		\endfoot

		\hline
		\endlastfoot

    TA.1 & RF.O.1, RF.O.1.1, RF.O.1.2\\
		\hline TA.2 & R.F.O.2\\
		\hline TA.3 & RF.O.13\\
		\hline TA.4 & RF.O.29\\
		\hline TA.5 & RF.O.30\\
		\hline TA.6 & RF.O.20\\
		\hline TA.7 & RF.O.31, RF.O.31.1\\
		\hline TA.8 & RF.O.31, RF.O.31.2\\
		\hline TA.9 & RF.O.32\\
		\hline TA.10 & RF.O.33\\
		\hline TA.11 & RF.O.28\\
		\hline TA.12 & RF.O.34\\
		\hline TA.13 & RF.O.18\\
		\hline TA.14 & RF.O.21\\
		\hline TA.15 & RF.O.3\\
		\hline TA.16 & RF.O.4\\
		\hline TA.17 & RF.O.14, RF.O.14.1, RF.O.14.2, RF.O.14.3, RF.O.14.3.1, RF.O.14.3.2\\
		\hline TA.18 & RF.O.23\\
		\hline TA.19 & RF.O.6\\
		\hline TA.20 & RF.O.11\\
		\hline TA.21 & RF.O.35\\
		\hline TA.22 & RF.O.12\\
		\hline TA.23 & RF.O.8\\
		\hline TA.24 & RF.O.46\\
		\hline TA.25 & RF.O.22\\
		\hline TA.26 & RF.O.23\\
		\hline TA.27 & RF.O.24\\
		\hline TA.28 & RF.O.25, RF.O.25.1, RF.O.25.1.1, RF.O.25.1.2\\
		\hline TA.29 & RF.O.9, RF.O.9.1, RF.O.10, RF.O.10.1, RF.O.10.2, RF.O.10.3, RF.O.10.4, RF.O.10.5\\
		\hline TA.30 & RF.D.45\\
		\hline TA.31 & RF.D.52, RF.D.7\\
  \end{longtable}
\end{adjustwidth}
\egroup
\clearpage
\subsection{Checklist}

\par Le checklist sono strumenti che affiancano il team nell'attività di ispezione del codice e della documentazione, al fine di accertarsi che siano conformi alle specifiche e alle linee guida (pratiche e stili di codifica, coerenza della documentazione). È un metodo di analisi statica mirato a individuare gli errori più ricorrenti che possono manifestarsi nel prodotto in esame.

\clearpage
\subsubsection{Struttura della documentazione}

\bgroup
\begin{adjustwidth}{-0.5cm}{-0.5cm}
	% MAX 12.5cm
  \begin{longtable}{|P{4.5cm}|>{\raggedright\arraybackslash}P{9cm}|}
    \caption{Checklist - Struttura della documentazione}
  	\label{tab:check-struttura-documentazione} \\
    \hline
		\textbf{Titolo} & \textbf{Descrizione} \\ 
		\hline
		\endfirsthead

    \caption[]{Checklist - Struttura della documentazione (continua)} \\
		\hline
		\textbf{Titolo} & \textbf{Descrizione} \\ 
		\hline
		\endhead

		\hline
		\multicolumn{2}{|r|}{{Continua nella prossima pagina}} \\ 
		\hline
		\endfoot

		\hline
		\endlastfoot

    Riferimento a documenti soggetti al versionamento & Quando viene menzionato il contenuto di un documento soggetto al versionamento, il riferimento deve riportare, oltre al nome del documento, anche il numero di versione.\\
		\hline Riferimento a materiali online & Le risorse web, per loro stessa natura, sono mutevoli. Pertanto, il riferimento a materiali online deve riportare la data di ultimo accesso alla risorsa. Inoltre, il collegamento ipertestuale deve essere visibile direttamente come URL.\\
    \hline Didascalie ed etichette & Tutte le immagini e le tabelle devono essere corredate da una didascalia che ne descriva il contenuto, e da un'etichetta univoca che funga da riferimento globale.\\
    \hline Sezioni vuote o incomplete & Nessun documento deve contenere sezioni vuote o incomplete (ovvero sezioni il cui contenuto è descritto come "Todo").\\
    \hline Suddivisione indice & L'indice dei documenti deve essere suddiviso in tre sezioni:
    \begin{itemize}
      \item Indice generale;
      \item Elenco delle tabelle;
      \item Elenco delle figure.
    \end{itemize}\\
    \hline Occorrenze multiple di un termine nel \Gls & Quando un termine definito nel \Gls\ appare più volte all'interno di un documento, tutte le ricorrenze devono essere formattate in corsivo e marcate con una lettera \ped{G} in pedice (a meno di non compromettere la leggibilità). \\
    \hline Occorrenze multiple di un termine nel \Gls\ (verbali esterni) & Quando un termine definito nel \Gls\ appare più volte all'interno di un verbale esterno, solamente la prima ricorrenza deve essere formattata. \\
    \hline Introduzione \Gls\ (verbali esterni) & L'introduzione del \Gls\ nei verbali esterni deve essere diversa rispetto a quella degli altri documenti.\\
    \hline Punteggiatura elenchi puntati o numerati & La frase di introduzione di un elenco puntato o numerato deve terminare con i due punti. Le voci di un elenco, invece, devono finire con un punto se rappresentano la conclusione dell'elenco o sotto-elenco in questione, altrimenti con un punto e virgola.\\
    \hline Formato delle date & Tutte le date non incluse in un paragrafo discorsivo devono apparire nella forma "AAAA-MM-GG".\\
    \hline Indice di leggibilità & Le modifiche ai documenti devono rispettare la soglia di tollerabilità stabilita per l'Indice Gulpease.\\
    \hline Distribuzione verbali esterni & Nella distribuzione dei verbali esterni deve essere menzionata, oltre al gruppo fornitore e ai Committenti, anche la Proponente.\\
    \hline Ordinamento registro modifiche per data & Nel changelog, le modifiche devono essere ordinate dalla più recente alla più vecchia.\\
    \hline Ordinamento task per ID & I task devono essere disposti in ordine crescente sulla base del loro ID. \\
    \hline Menzione di un soggetto & Quando si menziona una persona, la formula da utilizzare è la seguente: "Nome Cognome". Per mantenere coerenza all’interno dei documenti e sfruttare i comandi \glossario{LaTeX}, il team ha adottato questa formula anche negli elenchi e nelle tabelle. Pertanto, i nomi non vengono disposti in ordine alfabetico, ma seguono l'ordinamento definito nel template globale. \\
  \end{longtable}
\end{adjustwidth}
\egroup


\subsubsection{Errori formali}

\bgroup
\begin{adjustwidth}{-0.5cm}{-0.5cm}
	% MAX 12.5cm
  \begin{longtable}{|P{4.5cm}|>{\raggedright\arraybackslash}P{9cm}|}
    \caption{Checklist - Errori formali}
  	\label{tab:check-errori-formali} \\
    \hline
		\textbf{Titolo} & \textbf{Descrizione} \\ 
		\hline
		\endfirsthead

    \caption[]{Checklist - Errori formali (continua)} \\
		\hline
		\textbf{Titolo} & \textbf{Descrizione} \\ 
		\hline
		\endhead

		\hline
		\multicolumn{2}{|r|}{{Continua nella prossima pagina}} \\ 
		\hline
		\endfoot

		\hline
		\endlastfoot

    Nomi dei ruoli di progetto & I nomi dei ruoli di progetto devono avere la lettera iniziale minuscola. \\
    \hline Proponente & Il termine Proponente deve iniziare con la lettera maiuscola e, in via preferenziale, essere declinato al femminile. Tuttavia, l'uso maschile di Proponente è ritenuto corretto.\\
    \hline Cliente e Committente & I termini Cliente e Committente richiedono l'iniziale maiuscola solamente quando si riferiscono a un attore specifico e non a un ruolo o entità astratta.\\
		\hline Repository & Il termine repository deve essere declinato al maschile.\\
    \hline IA e AI & Possono essere utilizzati entrambi gli acronimi (intelligenza artificiale o, come indicato nel capitolato, artificial intelligence).\\
    \hline Back-end/backend e front-end/frontend & Si possono utilizzare entrambe le forme, con o senza trattino.\\
    \hline Sintassi e ortografia & Il testo deve essere privo di errori di sintassi e ortografia. \\
    \hline \glossario{typo} & È essenziale limitare gli errori tipografici e le sviste, specialmente nella scrittura del codice. \\
    \hline Linguaggio & I documenti devono essere redatti in modo impersonale dal punto di vista della forma verbale. Inoltre, è opportuno adottare un linguaggio il più formale possibile, soprattutto nella stesura dei verbali esterni. \\
    \hline Versioni estese di abbreviazioni & Le versioni estese delle sigle devono rispettare la forma delle abbreviazioni (es.: AdR diventa Analisi dei requisiti, WoW diventa Way of Working). \\
		\hline Soggetto della frase & Il soggetto di un discorso deve sempre essere evidenziato nella sua introduzione. \\
		\hline D eufonica & La d eufonica deve essere inserita solo quando le due vocali sono uguali. \\
		\hline PoC & Nonostante la traduzione di Proof of Concept (PoC) sia "verifica teorica" o "prova di fattibilità", il termine PoC deve essere declinato al maschile. \\
		\hline Open-source e open source & Entrambe le forme sono accettate, con o senza trattino. \\
		\hline Consistenza nell'uso delle lettere maiuscole nei titoli & Nei titoli delle sezioni o dei paragrafi dei documenti, la lettera iniziale maiuscola è riservata solo alla prima parola, salvo disposizioni contrarie nelle \NdP. \\
		\hline Declinazione di termini provenienti dalla lingua inglese & I termini inglesi inseriti all'interno di un documento italiano non vanno declinati (salvo rare eccezioni, ad esempio embeddings), in quanto la lingua italiana non prevede la formazione del plurale tramite l'aggiunta della desinenza -s o -es. \\
		\hline Componente & Il termine componente può essere declinato sia al maschile che al femminile. \\
		\hline ChatBOT & Il termine ChatBOT deve seguire la convenzione utilizzata per la scrittura di ChatGPT ("Chat" + "GPT"). \\
		\hline Web-based e web based & Entrambe le forme sono corrette, con o senza trattino. \\
	\end{longtable}
\end{adjustwidth}
\egroup
\subsubsection{Analisi dei Requisiti}

\bgroup
\begin{adjustwidth}{-0.5cm}{-0.5cm}
	% MAX 12.5cm
  \begin{longtable}{|P{4.5cm}|>{\raggedright\arraybackslash}P{9cm}|}
    \caption{Checklist - Analisi dei Requisiti}
  	\label{tab:check-analisi-requisiti} \\
    \hline
		\textbf{Titolo} & \textbf{Descrizione} \\ 
		\hline
		\endfirsthead

    \caption[]{Checklist - Analisi dei Requisiti (continua)} \\
		\hline
		\textbf{Titolo} & \textbf{Descrizione} \\ 
		\hline
		\endhead

		\hline
		\multicolumn{2}{|r|}{{Continua nella prossima pagina}} \\ 
		\hline
		\endfoot

		\hline
		\endlastfoot

    Correlazione casi d'uso - requisiti & Ciascun caso d'uso dovrebbe essere associato a uno o più requisiti. \\
    \hline Ordinamento requisiti & I requisiti devono essere ordinati secondo la stessa disposizione dei casi d'uso. \\
		\hline Diagrammi dei casi d'uso & Le inclusioni e le estensioni di un caso d'uso dovrebbero essere rappresentate nello stesso diagramma \glossario{UML} del caso d'uso. Le relazioni di generalizzazione, invece, possono essere visualizzate, per motivi di spazio, in un altro diagramma.\\
    \hline Coerenza diagramma-descrizione UC & Il diagramma UML e la descrizione dei casi d'uso devono essere consistenti.\\
    \hline Completezza descrizione UC & La descrizione dei casi d'uso deve essere esaustiva, integrando le informazioni già riportate nel diagramma UML.\\
    \hline Distinzione tra requisiti funzionali e non funzionali & La separazione tra i requisiti funzionali e non funzionali deve essere chiara. \\
    \hline Tracciamento dei requisiti & Ogni requisito deve essere ricavato da almeno una fonte. Il tracciamento dei requisiti (composto dalle coppie requisito-fonti) deve essere privo di errori.\\
  \end{longtable}
\end{adjustwidth}
\egroup
\subsubsection{Codifica}

\bgroup
\begin{adjustwidth}{-0.5cm}{-0.5cm}
	% MAX 12.5cm
  \begin{longtable}{|P{4.5cm}|>{\raggedright\arraybackslash}P{9cm}|}
    \caption{Checklist - Codifica}
  	\label{tab:check-codifica} \\
    \hline
		\textbf{Titolo} & \textbf{Descrizione} \\ 
		\hline
		\endfirsthead

    \caption[]{Checklist - Codifica (continua)} \\
		\hline
		\textbf{Titolo} & \textbf{Descrizione} \\ 
		\hline
		\endhead

		\hline
		\multicolumn{2}{|r|}{{Continua nella prossima pagina}} \\ 
		\hline
		\endfoot

		\hline
		\endlastfoot

    Nomi esplicativi & I nomi di classi, metodi, attributi e variabili devono essere "parlanti", in quanto rappresentano la prima forma di documentazione del codice. \\
    \hline Regole di nomenclatura & Le regole specificate nelle \NdP\ devono essere rispettate. Tali regole possono includere:
    \begin{itemize}
      \item Uso di CamelCase (es.: firstName) o snake\_case (es.: first\_name);
      \item Prefissi o suffissi per tipo di dato;
      \item Convenzioni per le costanti.
    \end{itemize}\\
    \hline Header & Tutti i file devono contenere un header conforme alle regole definite nelle \NdP. \\
    \hline Numerosità dei commenti & Porzioni di codice o metodi rilevanti dovrebbero essere preceduti da un commento. Per contro, è opportuno evitare commenti superflui che non migliorano la leggibilità. \\
		\hline Commenti significativi & I commenti devono essere significativi, ossia devono fornire in modo diretto informazioni utili sul funzionamento del codice.\\
  \end{longtable}
\end{adjustwidth}
\egroup