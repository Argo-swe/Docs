\section{Verifica}

\par Il collaudo del software è un insieme di attività volte a garantire il soddisfacimento degli obiettivi di qualità. La fase di verifica può aiutare il team a identificare e risolvere con prontezza anomalie legate alle componenti software. Pertanto, il gruppo si impegna a eseguire i test contestualmente alle attività di sviluppo. Con questa procedura, il team si aspetta di ridurre l’impatto degli errori, garantendo il rispetto del budget e dei tempi previsti. Il gruppo ha individuato quattro classi di test finalizzate ad assicurare la correttezza, completezza e affidabilità del software:
\begin{itemize}
  \item \textbf{Test di unità}: attività di collaudo di singole \glossario{unità} del software;
  \item \textbf{Test di integrazione}: verificano che i diversi moduli, componenti o servizi utilizzati dall'applicazione funzionino in modo integrato;
  \item \textbf{Test di sistema}: controllano il comportamento del sistema nel suo complesso e verificano che l'applicazione funzioni secondo i requisiti specificati;
  \item \textbf{Test di accettazione}: sono test formali che precedono il rilascio del prodotto e valutano se l'applicazione è conforme alle aspettative del cliente.
\end{itemize}

\subsection{Test di unità}

\par Lo scopo dei test di unità è verificare il corretto funzionamento delle "unità software", ossia delle porzioni o segmenti (come una funzione, una classe o un componente) testabili in modo autonomo e isolato all'interno del sistema. Di seguito è riportato l'elenco dei test di unità:

\bgroup
\begin{adjustwidth}{-0.5cm}{-0.5cm}
	% MAX 12.5cm
 	\begin{longtable}{|P{1.5cm}|>{\raggedright}P{9.5cm}|>{\arraybackslash}P{1.5cm}|}
		\caption{Test di unità}
  	\label{tab:test-unita} \\
	  \hline
		\textbf{ID} & \textbf{Descrizione} & \textbf{Stato} \\
		\hline
		\endfirsthead

		\caption[]{Test di unità (continua)} \\
		\hline
		\textbf{ID} & \textbf{Descrizione} & \textbf{Stato} \\
		\hline
		\endhead

		\hline
		\multicolumn{3}{|r|}{{Continua nella prossima pagina}} \\
		\hline
		\endfoot

		\hline
		\endlastfoot

		%json_file_adapter
		\hline TU.1 & Verificare che il metodo "create" della classe "FileFactory" soddisfi le seguenti condizioni:
		\begin{itemize}
			\item Deve creare correttamente un'istanza di "JsonFileAdapter" quando il tipo di file specificato nella configurazione è "json";
			\item Deve sollevare un'eccezione "ValueError" se il tipo di file specificato nella configurazione è sconosciuto.
		\end{itemize} & S \\

		%json_schema_validator_adapter
		\hline TU.2 & Verificare che il metodo "validate" della classe "JsonSchemaValidatorAdapter" soddisfi le seguenti condizioni:
		\begin{itemize}
			\item Deve restituire "True" se lo schema è valido secondo il modello di confronto per la validazione;
			\item Deve restituire "False" se lo schema non è valido secondo il modello di confronto per la validazione;
			\item Deve sollevare un'eccezione se non viene trovato il file da validare.
		\end{itemize} & S \\

		%sql_alchemy_authentication_repository_adapter
		\hline TU.3 & Verificare che il metodo "get\_user\_by\_username" della classe "SqlAlchemyAuthenticationRepositoryAdapter" soddisfi le seguenti condizioni:
		\begin{itemize}
			\item Deve recuperare correttamente un utente dal database in base allo username specificato, verificando che la query venga eseguita sul modello "Admins" applicando il filtro appropriato;
			\item Deve restituire "None" se non esiste alcun utente associato allo username fornito.
		\end{itemize} & S \\

		%sql_alchemy_db_manager_factory
		\hline TU.4 & Verificare che il metodo "create\_authentication\_repository" della classe "SqlAlchemyDbManagerFactory" soddisfi le seguenti condizioni:
		\begin{itemize}
			\item Deve creare correttamente un'istanza di "SqlAlchemyAuthenticationRepositoryAdapter";
			\item Deve sollevare un'eccezione se si verifica un errore durante la creazione dell'istanza.
		\end{itemize} & S \\

		\hline TU.5 & Verificare che il metodo "create\_dictionary\_repository" della classe "SqlAlchemyDbManagerFactory" soddisfi le seguenti condizioni:
		\begin{itemize}
			\item Deve creare correttamente un'istanza di "SqlAlchemyDictionaryRepositoryAdapter";
			\item Deve sollevare un'eccezione se si verifica un errore durante la creazione dell'istanza;
			\item Deve creare una nuova istanza ad ogni chiamata, verificando che più chiamate al metodo restituiscano istanze distinte.
		\end{itemize} & S \\

		\hline TU.6 & Verificare che l'inizializzazione della classe "SqlAlchemyDbManagerFactory" soddisfi le seguenti condizioni:
		\begin{itemize}
			\item Deve chiamare il metodo "create\_all" per garantire che tutte le tabelle del database siano create.
		\end{itemize} & S \\

		%sql_alchemy_dictionary_respository_adapter
		\hline TU.7 & Verificare che il metodo "create\_dictionary" della classe "SqlAlchemyDictionaryRepositoryAdapter" soddisfi le seguenti condizioni:
		\begin{itemize}
			\item Deve creare un nuovo dizionario e salvarlo nel database, verificando che le proprietà del dizionario siano impostate correttamente;
			\item Deve chiamare le operazioni di add, commit e refresh durante l'inserimento;
			\item Deve sollevare un'eccezione se si verifica un errore durante il commit del dizionario.
		\end{itemize} & S \\

		\hline TU.8 & Verificare che il metodo "update\_dictionary" della classe "SqlAlchemyDictionaryRepositoryAdapter" soddisfi le seguenti condizioni:
		\begin{itemize}
			\item Deve aggiornare correttamente le proprietà di un dizionario esistente e salvare le modifiche nel database;
			\item Deve chiamare le operazioni di commit e refresh durante l'aggiornamento;
			\item Deve sollevare un'eccezione se si verifica un errore durante il commit delle modifiche;
			\item Deve sollevare un'eccezione "AttributeError" se si tenta di aggiornare un dizionario inesistente.
		\end{itemize} & S \\

		\hline TU.9 & Verificare che il metodo "delete\_dictionary" della classe "SqlAlchemyDictionaryRepositoryAdapter" soddisfi le seguenti condizioni:
		\begin{itemize}
			\item Deve eliminare correttamente un dizionario dal database;
			\item Deve chiamare le operazioni di delete e commit durante l'eliminazione;
			\item Non deve chiamare le operazioni di delete e commit se si tenta di eliminare un dizionario inesistente;
			\item Deve sollevare un'eccezione se si verifica un errore durante l'eliminazione del dizionario.
		\end{itemize} & S \\

		\hline TU.10 & Verificare che il metodo "get\_all\_dictionaries" della classe "SqlAlchemyDictionaryRepositoryAdapter" soddisfi le seguenti condizioni:
		\begin{itemize}
			\item Deve restituire correttamente tutti i dizionari presenti nel database, verificando che la query venga eseguita sul modello "Dictionaries";
			\item Deve chiamare il metodo "all" durante il recupero dei dizionari.
		\end{itemize} & S \\

		\hline TU.11 & Verificare che il metodo "get\_dictionary\_by\_id" della classe "SqlAlchemyDictionaryRepositoryAdapter" soddisfi le seguenti condizioni:
		\begin{itemize}
			\item Deve restituire correttamente un dizionario in base all'ID fornito, verificando che la query venga eseguita sul modello "Dictionaries" applicando il filtro appropriato.
		\end{itemize} & S \\

		\hline TU.12 & Verificare che il metodo "get\_dictionary\_by\_name" della classe "SqlAlchemyDictionaryRepositoryAdapter" soddisfi le seguenti condizioni:
		\begin{itemize}
			\item Deve restituire correttamente un dizionario in base al nome fornito, verificando che la query venga eseguita sul modello "Dictionaries" applicando il filtro appropriato.
		\end{itemize} & S \\

		%db_manager_factory
		\hline TU.13 & Verificare che il metodo "create" della classe "DbManagerFactory" soddisfi le seguenti condizioni:
		\begin{itemize}
			\item Deve creare correttamente un'istanza di "SqlAlchemyDbManagerFactory" quando il tipo di database manager specificato nella configurazione è "sqlalchemy";
			\item Deve sollevare un'eccezione "ValueError" se il tipo di database manager specificato nella configurazione è sconosciuto.
			\item Deve sollevare un'eccezione "ValueError" se manca la chiave "db\_manager" nella configurazione.
		\end{itemize} & S \\

		%txtai_prompt_manager_adapter
		\hline TU.14 & Verificare che il metodo "prompt\_generator" della classe "TxtaiPromptManagerAdapter" soddisfi le seguenti condizioni:
		\begin{itemize}
			\item Deve generare correttamente il \glossario{prompt} includendo la richiesta dell'Utente, i metadati dello schema, la lingua e il \glossario{DBMS};
			\item Deve restituire un prompt vuoto se non vengono trovati risultati rilevanti;
			\item Deve generare correttamente il \glossario{log} se la modalità di \glossario{debug} è attiva;
			\item Deve gestire correttamente il caso in cui la richiesta dell'Utente è vuota;
			\item Deve gestire correttamente gli errori durante l'estrazione dei metadati dallo schema.
		\end{itemize} & S \\

		\hline TU.15 & Verificare che il metodo "get\_tuples" della classe "TxtaiPromptManagerAdapter" soddisfi le seguenti condizioni:
		\begin{itemize}
			\item Deve formulare correttamente la query \glossario{SQL} ed eseguire la ricerca restituendo le tuple corrispondenti;
			\item Deve attivare la registrazione del log quando richiesto e restituire il contenuto del log generato;
			\item Deve restituire una lista vuota se la ricerca semantica non produce risultati.
		\end{itemize} & S \\

		\hline TU.16 & Verificare che il metodo "get\_relevant\_tuples" della classe "TxtaiPromptManagerAdapter" soddisfi le seguenti condizioni:
		\begin{itemize}
			\item Deve mantenere le tuple con punteggi elevati, indipendentemente dalla distanza di punteggio;
			\item Deve mantenere o scartare le tuple in base alla distanza di punteggio rispetto alle tuple precedenti;
			\item Deve generare correttamente il log se la modalità di debug è attiva;
			\item Deve gestire correttamente una lista vuota di tuple.
		\end{itemize} & S \\

		%txtai_index_manager_adapter
		\hline TU.17 & Verificare che il metodo "get\_embeddings" della classe "TxtaiIndexManagerAdapter" soddisfi le seguenti condizioni:
		\begin{itemize}
			\item Deve restituire un'istanza della classe "Embeddings".
		\end{itemize} & S \\

		\hline TU.18 & Verificare che il metodo "create\_or\_load\_index" della classe "TxtaiIndexManagerAdapter" soddisfi le seguenti condizioni:
		\begin{itemize}
			\item Deve ripristinare un \glossario{indice} e restituire "False" se il dizionario ha già un indice associato;
			\item Deve creare un nuovo indice e restituire "True" se il dizionario non ha un indice associato.
		\end{itemize} & S \\

		\hline TU.19 & Verificare che il metodo "create\_index" della classe "TxtaiIndexManagerAdapter" soddisfi le seguenti condizioni:
		\begin{itemize}
			\item Deve creare un indice utilizzando i documenti estratti e salvarlo se "save\_index" è impostato a "True";
			\item Deve creare un indice senza salvarlo se "save\_index" è impostato a "False";
			\item Deve creare un indice anche se la lista dei documenti è vuota e salvarlo se "save\_index" è impostato a "True";
			\item Deve sollevare un'eccezione se si verifica un errore durante l'\glossario{indicizzazione}.
		\end{itemize} & S \\

		\hline TU.20 & Verificare che il metodo "save\_index" della classe "TxtaiIndexManagerAdapter" soddisfi le seguenti condizioni:
		\begin{itemize}
			\item Deve salvare l'indice utilizzando il percorso corretto;
			\item Deve sollevare un'eccezione se si verifica un errore durante il salvataggio dell'indice.
		\end{itemize} & S \\

		\hline TU.21 & Verificare che il metodo "load\_index" della classe "TxtaiIndexManagerAdapter" soddisfi le seguenti condizioni:
		\begin{itemize}
			\item Deve ripristinare l'indice utilizzando il percorso corretto.
		\end{itemize} & S \\

		\hline TU.22 & Verificare che il metodo "delete\_index" della classe "TxtaiIndexManagerAdapter" soddisfi le seguenti condizioni:
		\begin{itemize}
			\item Deve eliminare correttamente la directory dell'indice;
			\item Non deve tentare di eliminare la directory dell'indice se quest'ultima non esiste.
		\end{itemize} & S \\

		%txtai_embeddings_manager_factory
		\hline TU.23 & Verificare che il metodo "create\_index\_manager" della classe "TxtaiEmbeddingsManagerFactory" soddisfi le seguenti condizioni:
		\begin{itemize}
			\item Deve creare correttamente un'istanza di "TxtaiIndexManagerAdapter";
			\item Deve creare correttamente un'istanza di "TxtaiIndexManagerAdapter" anche se manca la chiave "txtai" nella configurazione.
		\end{itemize} & S \\

		\hline TU.24 & Verificare che il metodo "create\_prompt\_manager\_with\_dependencies" della classe "TxtaiEmbeddingsManagerFactory" soddisfi le seguenti condizioni:
		\begin{itemize}
			\item Deve creare un'istanza di "TxtaiPromptManagerAdapter" quando viene fornita un'istanza di "TxtaiIndexManagerAdapter" come dipendenza.
		\end{itemize} & S \\

		\hline TU.25 & Verificare che il metodo "create\_prompt\_manager" della classe "TxtaiEmbeddingsManagerFactory" soddisfi le seguenti condizioni:
		\begin{itemize}
			\item Deve creare un'istanza di "TxtaiPromptManagerAdapter", gestendo internamente la creazione di un'istanza di "TxtaiIndexManagerAdapter".
		\end{itemize} & S \\

		%txtai_search_algorithm_adapter
		\hline TU.26 & Verificare che il metodo "semantic\_search" della classe "TxtaiSearchAlgorithmAdapter" soddisfi le seguenti condizioni:
		\begin{itemize}
			\item Deve eseguire correttamente una ricerca semantica tramite l'index manager e restituire i risultati pertinenti;
			\item Deve generare il log corretto quando il logging è attivato, includendo i dettagli sull'importanza dei termini;
			\item Deve restituire i risultati della ricerca e il contenuto del log in modo appropriato.
		\end{itemize} & S \\

		\hline TU.27 & Verificare che il metodo "search\_filtering" della classe "TxtaiSearchAlgorithmAdapter" soddisfi le seguenti condizioni:
		\begin{itemize}
			\item Deve filtrare correttamente i risultati della ricerca basandosi sui punteggi ("max\_score");
			\item Non deve generare log quando il logging è disattivato;
			\item Deve restituire solo le tuple rilevanti senza contenuto di log quando il logging è disattivato.
		\end{itemize} & S \\

		\hline TU.28 & Verificare che il metodo "semantic\_search\_log" della classe "TxtaiSearchAlgorithmAdapter" soddisfi le seguenti condizioni:
		\begin{itemize}
			\item Deve generare correttamente il log per una ricerca semantica, includendo i punteggi e le spiegazioni dei termini rilevanti;
			\item Deve gestire correttamente la generazione del log quando non vengono trovati risultati.
		\end{itemize} & S \\

		\hline TU.29 & Verificare che il metodo "search\_filtering\_log" della classe "TxtaiSearchAlgorithmAdapter" soddisfi le seguenti condizioni:
		\begin{itemize}
			\item Deve generare un log che indichi quali tabelle vengono mantenute in base ai punteggi e quali vengono scartate;
			\item Deve indicare correttamente il motivo per cui alcune tabelle vengono scartate (es. differenza di punteggio > 0.2).
		\end{itemize} & S \\

		\hline TU.30 & Verificare che il metodo "get\_debug\_header" della classe "TxtaiSearchAlgorithmAdapter" soddisfi le seguenti condizioni:
		\begin{itemize}
			\item Deve generare correttamente un'intestazione di debug con il livello di log e il sistema specificato;
			\item Deve formattare l'intestazione secondo lo schema "[SystemName] [LogLevel]".
		\end{itemize} & S \\

		%json_file_adapter
		\hline TU.31 & Verificare che il metodo "save" della classe "JsonFileAdapter" soddisfi le seguenti condizioni:
		\begin{itemize}
			\item Deve aprire un file con permessi di scrittura;
			\item Deve eseguire almeno un'operazione di scrittura sul file aperto.
		\end{itemize} & S \\

		\hline TU.32 & Verificare che il metodo "load" della classe "JsonFileAdapter" soddisfi le seguenti condizioni:
		\begin{itemize}
			\item Deve restituire il percorso del file.
		\end{itemize} & S \\

		\hline TU.33 & Verificare che il metodo "delete" della classe "JsonFileAdapter" soddisfi le seguenti condizioni:
		\begin{itemize}
			\item Deve controllare l'esistenza del file;
			\item Deve eseguire l'operazione di eliminazione del file.
		\end{itemize} & S \\

		\hline TU.34 & Verificare che il metodo "get\_preview" della classe "JsonFileAdapter" soddisfi le seguenti condizioni:
		\begin{itemize}
			\item Deve restituire un oggetto che rappresenti lo schema del file;
			\item Deve restituire "None" se lo schema del file non è presente.
		\end{itemize} & S \\

		\hline TU.35 & Verificare che il metodo "extract\_index\_metadata" della classe "JsonFileAdapter" soddisfi le seguenti condizioni:
		\begin{itemize}
			\item Deve restituire i metadati del dizionario per l'indicizzazione.
		\end{itemize} & S \\

		\hline TU.36 & Verificare che il metodo "extract\_schema\_metadata" della classe "JsonFileAdapter" soddisfi le seguenti condizioni:
		\begin{itemize}
			\item Deve restituire i metadati del dizionario in forma di prompt.
		\end{itemize} & S \\

		%componenti front-end
		\hline TU.37 & Verificare che il componente "AppLogo" soddisfi le seguenti condizioni:
    \begin{itemize}
      \item Il logo deve essere renderizzato con il percorso dell'immagine corretto;
			\item L'altezza e la larghezza dell'immagine devono essere impostate con valori predefiniti se non vengono passate come props;
			\item Il colore dell'immagine deve cambiare in base al tema selezionato.
    \end{itemize} & S \\

		\hline TU.38 & Verificare che il componente "AppMenu" soddisfi le seguenti condizioni:
    \begin{itemize}
      \item Il menu dell'Utente deve avere avere una o più opzioni;
			\item Il menu del Tecnico deve avere due o più opzioni.
    \end{itemize} & S \\

		\hline TU.39 & Verificare che il componente "AppMenuItem" soddisfi le seguenti condizioni:
    \begin{itemize}
      \item La voce di menu deve contenere l'etichetta corretta;
			\item La voce di menu non deve essere attiva se non è stata selezionata;
			\item Una volta selezionata, la voce di menu deve essere attiva;
			\item La voce di menu può essere la radice di un menu;
			\item La voce di menu può contenere un sotto-menu;
			\item La voce di menu può aprire o chiudere un sotto-menu;
			\item L'icona "submenu-toggler" deve essere visibile se la voce di menu contiene un sotto-menu.
    \end{itemize} & S \\

		\hline TU.40 & Verificare che il componente "AppTopbar" soddisfi le seguenti condizioni:
    \begin{itemize}
      \item Il pulsante di login deve essere visibile se l'Utente non ha effettuato l'accesso;
			\item Il pulsante di logout deve essere visibile se l'Utente ha effettuato l'accesso;
			\item Quando viene cliccato il pulsante di logout, la funzione di logout deve essere chiamata;
			\item L'option menu deve essere aperto al clic dell'apposito pulsante;
			\item L'option menu deve essere chiuso quando l'Utente clicca al di fuori di esso.
    \end{itemize} & S \\

		\hline TU.41 & Verificare che il componente "AppFooter" soddisfi le seguenti condizioni:
    \begin{itemize}
      \item Il footer deve essere renderizzato correttamente;
			\item All'interno del footer deve essere visualizzata la versione corretta dell'applicazione.
    \end{itemize} & S \\

		\hline TU.42 & Verificare che il componente "ConfigSidebar" soddisfi le seguenti condizioni:
    \begin{itemize}
      \item La sezione per modificare la scala deve essere visibile;
			\item La sezione per modificare il tema deve essere visibile;
			\item La sezione per modificare la lingua dell'interfaccia deve essere visibile;
			\item La scala deve essere diminuita al clic del'apposito pulsante;
			\item La scala deve essere aumentata al clic dell'apposito pulsante;
			\item Il pulsante per diminuire la scala deve essere disabilitato se la scala è al minimo;
			\item Il pulsante per aumentare la scala deve essere disabilitato se la scala è al massimo;
			\item Il contenuto del localStorage deve essere aggiornato al cambio del tema;
			\item Il contenuto del localStorage deve essere aggiornato al cambio della lingua.
    \end{itemize} & S \\

		\hline TU.43 & Verificare che il componente "MenuSidebar" soddisfi le seguenti condizioni:
    \begin{itemize}
			\item Il pulsante per chiudere la sidebar deve essere nascosto su schermi di grandi dimensioni;
      \item Il pulsante per chiudere la sidebar deve essere visibile su schermi di piccole dimensioni.
    \end{itemize} & S \\

		\hline TU.44 & Verificare che il componente "LoginDialog" soddisfi le seguenti condizioni:
    \begin{itemize}
      \item Il dialog deve essere chiuso al clic dell'apposito pulsante;
			\item La funzione "messageSuccess" deve essere chiamata se il login viene effettuato con successo;
			\item Una volta completato il login, il dialog deve essere chiuso;
			\item Il contenuto del localStorage deve essere aggiornato se il login viene effettuato con successo;
			\item Una volta completato il login, il form deve essere resettato;
			\item La funzione "messageError" deve essere chiamata se il login fallisce.
    \end{itemize} & S \\

		\hline TU.45 & Verificare che il componente "DictPreview" soddisfi le seguenti condizioni:
    \begin{itemize}
      \item L'anteprima del dizionario dati deve essere nascosta se la variabile detailsVisible è uguale a "false";
			\item Se è visibile, l'anteprima deve mostrare i dati corretti;
			\item Lo stato di espansione dell'anteprima deve essere attivato/disattivato al clic dell'apposito pulsante;
			\item Quando viene cliccato il pulsante di chiusura, il componente deve inviare un segnale.
    \end{itemize} & S \\

		\hline TU.46 & Verificare che il componente "ChatMessage" soddisfi le seguenti condizioni:
    \begin{itemize}
      \item Il messaggio inviato dall'Utente deve essere visualizzato correttamente;
			\item Il messaggio inviato dal ChatBOT deve essere visualizzato correttamente;
			\item Se il messaggio inviato dal ChatBOT è "null", il sistema deve mostrare un avviso all'Utente;
			\item I pulsanti di azione devono essere nascosti se il messaggio è stato inviato dall'Utente;
			\item La funzione "Copia negli appunti" deve essere chiamata con i parametri corretti al clic dell'apposito pulsante;
			\item Il pulsante di debug deve essere nascosto se l'Utente non ha effettuato il login;
			\item La funzione "Apri debug modal" deve essere chiamata con i parametri corretti al clic dell'apposito pulsante.
    \end{itemize} & S \\

		\hline TU.47 & Verificare che il componente "DebugMessage" soddisfi le seguenti condizioni:
    \begin{itemize}
      \item Il componente deve visualizzare il messaggio di debug correttamente;
			\item Quando viene cliccato il pulsante "Scarica file", la funzione di download deve essere chiamata con i parametri corretti.
    \end{itemize} & S \\

		\hline TU.48 & Verificare che il componente "ChatDeleteBtn" soddisfi le seguenti condizioni:
    \begin{itemize}
      \item Il pulsante deve essere disabilitato se non è presente alcun messaggio nella chat;
      \item Il pulsante deve essere disabilitato se il ChatBOT sta elaborando un nuovo messaggio;
      \item Quando viene cliccato, il pulsante deve inviare un segnale.
    \end{itemize} & S \\

		\hline TU.49 & Verificare che il componente "CreateUpdateDictionaryModal" soddisfi le seguenti condizioni:
    \begin{itemize}
      \item Il form di creazione e modifica deve essere visualizzato correttamente;
			\item Il pulsante di invio deve essere disabilitato se il form è incompleto;
			\item La funzione "messageSuccess" deve essere chiamata se l'inserimento viene effettuato con successo;
			\item La funzione "messageError" deve essere chiamata se l'inserimento fallisce;
			\item Il pulsante di invio deve essere disabilitato se il nome del dizionario dati contiene caratteri non supportati;
			\item Il pulsante di invio deve essere disabilitato se il formato del file non è valido;
			\item La funzione "messageSuccess" deve essere chiamata se l'aggiornamento dei metadati viene effettuato con successo;
			\item La funzione "messageError" deve essere chiamata se l'aggiornamento dei metadati fallisce;
			\item La funzione "messageSuccess" deve essere chiamata se l'aggiornamento del file viene effettuato con successo;
			\item La funzione "messageError" deve essere chiamata se l'aggiornamento del file fallisce;
			\item L'upload del file deve essere disabilitato se un file è già stato caricato;
			\item Il file selezionato deve essere rimosso al clic dell'apposito pulsante.
    \end{itemize} & S \\

		\hline TU.50 & Verificare che il componente "AppLayout" soddisfi le seguenti condizioni:
    \begin{itemize}
      \item Il layout dell'applicazione deve essere renderizzato correttamente;
      \item Il wrapper globale deve contenere le classi corrette sulla base delle impostazioni e dello stato del layout.
    \end{itemize} & S \\

		\hline TU.51 & Verificare che il componente "DictionariesListView" soddisfi le seguenti condizioni:
    \begin{itemize}
      \item La lista dei dizionari dati deve essere visualizzata correttamente;
			\item Se la lista dei dizionari è vuota, il sistema deve mostrare un messaggio esplicativo;
			\item I risultati della ricerca dei dizionari per nome devono essere visualizzati correttamente;
			\item Se la ricerca dei dizionari per nome non produce risultati, il sistema deve mostrare un messaggio esplicativo;
			\item I risultati della ricerca dei dizionari per descrizione devono essere visualizzati correttamente;
			\item Se la ricerca dei dizionari per descrizione non produce risultati, il sistema deve mostrare un messaggio esplicativo;
			\item La funzione "Apri dictionary modal" deve essere chiamata con i parametri corretti al clic del pulsante di inserimento;
			\item La funzione "Apri dictionary modal" deve essere chiamata con i parametri corretti al clic del pulsante di modifica dei metadati;
			\item La funzione "Apri dictionary modal" deve essere chiamata con i parametri corretti al clic del pulsante di modifica del file;
			\item Quando viene cliccato il pulsante "Scarica file", la funzione di download deve essere chiamata con i parametri corretti;
			\item La funzione "messageSuccess" deve essere chiamata se l'eliminazione viene effettuata con successo;
			\item La funzione "messageError" deve essere chiamata se l'eliminazione fallisce.
    \end{itemize} & S \\

		\hline TU.52 & Verificare che il componente "ChatView" soddisfi le seguenti condizioni:
    \begin{itemize}
      \item Il contenuto della chat deve essere renderizzato correttamente;
			\item Il form di selezione pre-richiesta deve essere attivato/disattivato al click dell'apposito pulsante;
			\item Nella chat deve essere visualizzato il dizionario dati attivo con la relativa estensione;
			\item Quando un dizionario viene selezionato, il contenuto del localStorage deve essere aggiornato;
			\item Il pulsante di invio deve essere disattivato se non è stato selezionato alcun dizionario;
			\item Il pulsante di invio deve essere disattivato se non è stata inserita alcuna richiesta;
			\item La richiesta deve poter essere inviata premendo il tasto "Invio";
			\item La funzione di generazione del prompt deve essere chiamata con i parametri corretti al clic del pulsante di invio;
			\item Quando il ChatBOT ritorna una risposta, il contenuto del sessionStorage deve essere aggiornato;
			\item Quando viene inviata una richiesta, il campo di testo deve essere resettato;
			\item La funzione "messageError" deve essere chiamata se la generazione del prompt fallisce;
			\item Il pulsante "Scroll to bottom" deve essere visualizzato quando l'Utente scorre i messaggi;
			\item L'ultimo messaggio della chat deve essere visualizzato se viene cliccato il pulsante "Scroll to bottom".
    \end{itemize} & S \\

		\hline TU.53 & Verificare che il metodo "logout" della classe "auth.service" soddisfi le seguenti condizioni:
		\begin{itemize}
			\item Il metodo deve rimuovere il token di autenticazione dal localStorage;
			\item Il metodo deve inviare un segnale di notifica.
    \end{itemize} & S \\

		\hline TU.54 & Verificare che il metodo "isLogged" della classe "auth.service" soddisfi le seguenti condizioni:
		\begin{itemize}
			\item Il metodo deve restituire "false" se il token di autenticazione non è presente;
			\item Il metodo deve restituire "false" se il token è scaduto;
			\item Il metodo deve restituire "true" se il token è valido;
			\item Il metodo deve restituire "false" se la data di scadenza non è definita.
    \end{itemize} & S \\

		\hline TU.55 & Verificare che il metodo "downloadFile" della classe "utils.service" soddisfi le seguenti condizioni:
		\begin{itemize}
			\item Il metodo deve creare un collegamento per il download del file;
			\item Il metodo deve attivare il download cliccando sul collegamento.
    \end{itemize} & S \\

		\hline TU.56 & Verificare che il metodo "stringToSnakeCase" della classe "utils.service" soddisfi le seguenti condizioni:
		\begin{itemize}
			\item Il metodo deve convertire una stringa in snake\_case.
    \end{itemize} & S \\

		\hline TU.57 & Verificare che il metodo "addCapitalizeValues" della classe "utils.service" soddisfi le seguenti condizioni:
		\begin{itemize}
			\item Il metodo deve aggiungere chiavi in maiuscolo a un determinato oggetto.
    \end{itemize} & S \\

		\hline TU.58 & Verificare che il metodo "capitalizeString" della classe "utils.service" soddisfi le seguenti condizioni:
		\begin{itemize}
			\item Il metodo deve convertire in maiuscolo la prima lettera di una stringa.
    \end{itemize} & S \\
	\end{longtable}
\end{adjustwidth}
\egroup

\subsection{Test di integrazione}

\par La sezione relativa ai test di integrazione verrà aggiornata durante la progettazione di dettaglio.
\subsection{Test di sistema}

\par I test di sistema devono assicurare una completa copertura dei requisiti concordati con la \glossario{Proponente} e/o specificati nel documento di \AdR. Di seguito è riportato l'elenco dei test di sistema:

\bgroup
\begin{adjustwidth}{-0.5cm}{-0.5cm}
	% MAX 12.5cm
 	\begin{longtable}{|P{1.5cm}|>{\raggedright}P{9.5cm}|>{\arraybackslash}P{1.5cm}|}
		\caption{Test di sistema}
  	\label{tab:test-sistema} \\
	  \hline
		\textbf{ID} & \textbf{Descrizione} & \textbf{Stato} \\ 
		\hline
		\endfirsthead

		\caption[]{Test di sistema (continua)} \\
		\hline
		\textbf{ID} & \textbf{Descrizione} & \textbf{Stato} \\ 
		\hline
		\endhead

		\hline
		\multicolumn{3}{|r|}{{Continua nella prossima pagina}} \\ 
		\hline
		\endfoot

		\hline
		\endlastfoot

		TS.1 & Verificare che il Tecnico possa effettuare il login. & N-D \\
		\hline TS.2 & Verificare che il Tecnico visualizzi un errore qualora inserisca delle credenziali errate in fase di autenticazione. & N-D \\
		\hline TS.3 & Verificare che il Tecnico possa inserire un nuovo \glossario{dizionario dati} nel sistema. & N-D \\ 
		\hline TS.4 & Verificare che il Tecnico possa modificare il nome di un dizionario dati. & N-D \\ 
		\hline TS.5 & Verificare che il Tecnico possa modificare la descrizione di un dizionario dati. & N-D \\
		\hline TS.6 & Verificare che il Tecnico possa modificare il file di configurazione di un dizionario dati. & N-D \\
		\hline TS.7 & Verificare che il Tecnico visualizzi un errore nel caso in cui il nome del dizionario dati contenga caratteri non supportati. & N-D \\ 
		\hline TS.8 & Verificare che il sistema restituisca un messaggio di errore qualora il Tecnico inserisca un nome già esistente per il dizionario dati. & N-D \\ 
		\hline TS.9 & Verificare che il Tecnico visualizzi un errore nel caso in cui la descrizione del dizionario dati contenga caratteri non supportati. & N-D \\ 
		\hline TS.10 & Verificare che il sistema restituisca un messaggio di errore qualora il Tecnico inserisca un dizionario dati con una dimensione superiore a 1 MB. & N-D \\
		\hline TS.11 & Verificare che il Tecnico visualizzi un errore nel caso in cui l'estensione del file caricato sia diversa da .json. & N-D \\ 
		\hline TS.12 & Verificare che il sistema restituisca un messaggio di errore qualora il Tecnico inserisca un dizionario dati non conforme allo schema predefinito. & N-D \\  
		\hline TS.13 & Verificare che il Tecnico possa eliminare un dizionario dati dal sistema. & N-D \\ 
		\hline TS.14 & Verificare che il sistema restituisca un messaggio di errore nel caso in cui l'eliminazione del dizionario dati abbia esito negativo. & N-D \\  
		\hline TS.15 & Verificare che il Tecnico possa scaricare un dizionario dati. & N-D \\
		\hline TS.16 & Verificare che l'Utente possa accedere alla chat e inserire un messaggio nella maschera di richiesta. & N-D \\   
		\hline TS.17 & Verificare che l'Utente possa selezionare un dizionario dati e renderlo operativo nel sistema. & N-D \\ 
		\hline TS.18 & Verificare che l'Utente possa visualizzare il contenuto del dizionario dati selezionato. Il sistema deve mostrare le seguenti informazioni:
		\begin{itemize}
			\item Nome del database;
			\item Descrizione del database;
			\item Lista delle tabelle del database. Per ciascuna tabella devono essere riportate le seguenti informazioni:
				\begin{itemize}
					\item Nome della tabella;
					\item Descrizione della tabella.
				\end{itemize}
		\end{itemize} & N-D \\  
		\hline TS.19 & Verificare che l'Utente possa inviare una richiesta al ChatBOT e ottenere il \glossario{prompt} risultante. & N-D \\ 
		\hline TS.20 & Verificare che il sistema restituisca un avviso qualora l'Utente inserisca una richiesta ritenuta non idonea dal modello di \glossario{AI}. & N-D \\
		\hline TS.21 & Verificare che il sistema restituisca un messaggio di errore nel caso in cui la generazione del prompt venga interrotta senza preavviso. & N-D \\ 
		\hline TS.22 & Verificare che l'Utente possa visualizzare correttamente il prompt generato. Il prompt deve contenere le seguenti informazioni:
		\begin{itemize}
			\item Lista delle tabelle pertinenti. Ogni tabella deve essere corredata da:
			\begin{itemize}
				\item Schema della tabella: composto dal nome della tabella e da una lista di colonne, a loro volte organizzate per nome e tipo (es.: integer, string);
				\item Chiave primaria;
				\item Descrizione della tabella;
				\item Descrizione delle colonne della tabella;
				\item Chiavi esterne;
			\end{itemize}
			\item DBMS di riferimento;
			\item Lingua di riferimento;
			\item Richiesta in linguaggio naturale.
		\end{itemize} & N-D \\
		\hline TS.23 & Verificare che il Tecnico possa effettuare il logout per terminare la sessione corrente. & N-D \\ 
		\hline TS.24 & Verificare che l'Utente possa copiare il contenuto del prompt generato. & N-D \\ 
		\hline TS.25 & Verificare che il sistema generi un \glossario{log} se la richiesta viene inviata dal Tecnico. & N-D \\ 
		\hline TS.26 & Verificare che il Tecnico possa copiare il messaggio di \glossario{debug} relativo alla richiesta inviata. & N-D \\ 
		\hline TS.27 & Verificare che il Tecnico possa scaricare un file di log contenente il debug del prompt. & N-D \\
		\hline TS.28 & Verificare che il sistema restituisca un messaggio di errore nel caso in cui il download di un file fallisca. & N-D \\  
		\hline TS.29 & Verificare che l'Utente possa visualizzare correttamente il contenuto della chat. & N-D \\  
		\hline TS.30 & Verificare che il Tecnico possa visualizzare i dizionari dati con le relative informazioni:
		\begin{itemize}
			\item Nome del dizionario dati;
			\item Estensione del file;
			\item Dimensione del file;
			\item Descrizione del dizionario dati;
			\item Data di ultimo aggiornamento.
		\end{itemize} & N-D \\  
		\hline TS.31 & Verificare che il sistema di generazione del \glossario{prompt} supporti richieste in lingue diverse dall'inglese. Di seguito sono riportate le lingue che devono essere verificate: 
		\begin{itemize}
			\item Italiano;
			\item Francese;
			\item Tedesco;
			\item Spagnolo. 
		\end{itemize}
		& N-D \\  
	\end{longtable}
\end{adjustwidth}
\egroup

\clearpage
\subsubsection{Tracciamento dei test di sistema}

\bgroup
\begin{adjustwidth}{-0.5cm}{-0.5cm}
	\centering
	% MAX 12.5cm
  \begin{longtable}{|c|c|}
		\caption{Tracciamento test di sistema}
  	\label{tab:tracciamento-test-sistema} \\
    \hline
		\textbf{ID} & \textbf{Requisito} \\ 
		\hline
		\endfirsthead

		\caption[]{Tracciamento test di sistema (continua)} \\
		\hline
		\textbf{ID} & \textbf{Requisito} \\ 
		\hline
		\endhead

		\hline
		\multicolumn{2}{|r|}{{Continua nella prossima pagina}} \\ 
		\hline
		\endfoot

		\hline
		\endlastfoot

    TS.1 & RF.O.1, RF.O.1.1, RF.O.1.2\\
		\hline TS.2 & R.F.O.2\\
		\hline TS.3 & RF.O.13, RF.O.15, RF.O.16, RF.O.17\\
		\hline TS.4 & RF.O.29\\
		\hline TS.5 & RF.O.30\\
		\hline TS.6 & RF.O.15, RF.O.20\\
		\hline TS.7 & RF.O.31, RF.O.31.1\\
		\hline TS.8 & RF.O.31, RF.O.31.2\\
		\hline TS.9 & RF.O.32\\
		\hline TS.10 & RF.O.33\\
		\hline TS.11 & RF.O.28\\
		\hline TS.12 & RF.O.34\\
		\hline TS.13 & RF.O.18\\
		\hline TS.14 & RF.O.21\\
		\hline TS.15 & RF.O.37\\
		\hline TS.16 & RF.O.3\\
		\hline TS.17 & RF.O.4\\
		\hline TS.18 & RF.O.14, RF.O.14.1, RF.O.14.2, RF.O.14.3, RF.O.14.3.1, RF.O.14.3.1.1, RF.O.14.3.1.2\\
		\hline TS.19 & RF.O.5\\
		\hline TS.20 & RF.O.6\\
		\hline TS.21 & RF.O.11\\
		\hline TS.22 & RF.O.35\\
		\hline TS.23 & RF.O.12\\
		\hline TS.24 & RF.O.8\\
		\hline TS.25 & RF.O.46\\
		\hline TS.26 & RF.O.38\\
		\hline TS.27 & RF.O.23\\
		\hline TS.28 & RF.O.24\\
		\hline TS.29 & RF.O.25, RF.O.25.1, RF.O.25.1.1, RF.O.25.1.2\\
		\hline TS.30 & RF.O.9, RF.O.9.1, RF.O.10, RF.O.10.1, RF.O.10.2, RF.O.10.3, RF.O.10.4, RF.O.10.5\\
		\hline TS.31 & RF.D.52, RF.D.7\\
  \end{longtable}
\end{adjustwidth}
\egroup
\subsection{Test di accettazione}

\par L'obiettivo dei test di accettazione è verificare se il sistema soddisfa le aspettative del Committente e del Proponente. I test di accettazione determinano se il software è pronto per essere rilasciato, e pertanto richiedono un focus sul comportamento degli utenti. Di seguito è riportato l'elenco dei test di accettazione:

\bgroup
\begin{adjustwidth}{-0.5cm}{-0.5cm}
	% MAX 12.5cm
 	\begin{longtable}{|P{1.5cm}|>{\raggedright}P{9.5cm}|>{\arraybackslash}P{1.5cm}|}
    \caption{Test di accettazione}
  	\label{tab:test-accettazione} \\
	  \hline
		\textbf{ID} & \textbf{Descrizione} & \textbf{Stato} \\ 
		\hline
		\endfirsthead

    \caption[]{Test di accettazione (continua)} \\
		\hline
		\textbf{ID} & \textbf{Descrizione} & \textbf{Stato} \\ 
		\hline
		\endhead

		\hline
		\multicolumn{3}{|r|}{{Continua nella prossima pagina}} \\ 
		\hline
		\endfoot

		\hline
		\endlastfoot

		TA.1 & Verificare che l'Utente possa effettuare il login alla sezione del tecnico:
		\begin{enumerate}
			\item Avviare la procedura di autenticazione (da qualsiasi pagina);
			\item Inserire uno username;
			\item Inserire una password;
			\item Richiedere l'accesso;
			\item Visualizzare un messaggio di conferma una volta effettuato il login.
		\end{enumerate}
		& N-D \\
		\hline TA.2 & Verificare che il Tecnico possa salvare un nuovo \glossario{dizionario dati} nel sistema:
		\begin{enumerate}
			\item Accedere alla pagina di gestione dei \glossario{dizionari dati};
			\item Avviare la procedura di inserimento di un nuovo dizionario dati;
			\item Inserire il nome del dizionario dati;
			\item Inserire la descrizione del dizionario dati;
			\item Caricare il file .json scelto come dizionario dati;
			\item Confermare l'inserimento;
			\item Visualizzare un feedback una volta che il dizionario è stato salvato.
		\end{enumerate}
		& N-D \\
		\hline TA.3 & Verificare che l'Utenta possa visualizzare la lista dei dizionari caricati nel sistema:
		\begin{enumerate}
			\item Visualizzare la lista dei dizionari disponibili;
			\item Visualizzare le caratteristiche di ciascun dizionario presente nella lista.
		\end{enumerate}
		& N-D \\
		\hline TA.4 & Verificare che il Tecnico possa eliminare un dizionario dati dal sistema:
		\begin{enumerate}
			\item Accedere alla pagina di gestione dei dizionari;
			\item Visualizzare la lista dei dizionari dati caricati nel sistema;
			\item Visualizzare le caratteristiche di ciascun dizionario presente nella lista;
			\item Scegliere un dizionario dati tra quelli disponibili;
			\item Richiedere l'eliminazione del dizionario dati;
			\item Confermare la decisione;
			\item Visualizzare un feedback una volta che il dizionario è stato eliminato.
		\end{enumerate}
		& N-D \\
		\hline TA.5 & Verificare che il Tecnico possa modificare il nome di un dizionario dati:
		\begin{enumerate}
			\item Accedere alla pagina di gestione dei dizionari;
			\item Visualizzare la lista dei dizionari dati caricati nel sistema;
			\item Visualizzare le caratteristiche di ciascun dizionario presente nella lista;
			\item Scegliere un dizionario dati tra quelli disponibili;
			\item Modificare il nome del dizionario dati;
			\item Confermare la modifica;
			\item Visualizzare un feedback positivo una volta effettuata la modifica.					
		\end{enumerate}
		& N-D \\
		\hline TA.6 & Verificare che il Tecnico possa modificare la descrizione di un dizionario dati:
		\begin{enumerate}
			\item Accedere alla pagina di gestione dei dizionari;
			\item Visualizzare la lista dei dizionari dati caricati nel sistema;
			\item Visualizzare le caratteristiche di ciascun dizionario presente nella lista;
			\item Scegliere un dizionario dati tra quelli disponibili;
			\item Modificare la descrizione del dizionario dati;
			\item Confermare la modifica;
			\item Visualizzare un feedback positivo una volta effettuata la modifica.										
		\end{enumerate}
		& N-D \\
		\hline TA.7 & Verificare che il Tecnico possa modificare il file di configurazione di un dizionario dati:
		\begin{enumerate}
			\item Accedere alla pagina di gestione dei dizionari;
			\item Visualizzare la lista dei dizionari dati caricati nel sistema;
			\item Visualizzare le caratteristiche di ciascun dizionario presente nella lista;
			\item Scegliere un dizionario dati tra quelli disponibili;
			\item Caricare un nuovo file \glossario{JSON};
			\item Confermare la sostituzione;
			\item Visualizzare un feedback positivo una volta effettuata la modifica.									
		\end{enumerate}
		& N-D \\
		\hline TA.8 & Verificare che il Tecnico possa effettuare correttamente il logout. & N-D \\
		\hline TA.9 & Verificare che l'Utente possa effettuare una ricerca tra i dizionari dati:
		\begin{enumerate}
			\item Visualizzare la lista dei dizionari disponibili;
			\item Richiedere al sistema di effettuare una ricerca (per nome o per descrizione);
			\item Visualizzare i risultati della ricerca.				
		\end{enumerate}
		& N-D \\
		\hline TA.10 & Verificare che l'Utente possa selezionare un \glossario{dizionario dati} da utilizzare nell'applicazione. & N-D \\
		\hline TA.11 & Verificare che l'Utente possa visualizzare il contenuto del dizionario dati selezionato. & N-D \\
		\hline TA.12 & Verificare che l'Utente possa ottenere un \glossario{prompt} in risposta a un'interrogazione in linguaggio naturale:
		\begin{enumerate}
			\item Accedere alla chat;
			\item Inserire una richiesta in linguaggio naturale;
			\item Inviare il messaggio;
			\item Visualizzare il prompt generato dal sistema.
		\end{enumerate}
		& N-D \\
		\hline TA.13 & Verificare che l'Utente possa copiare il prompt generato:
		\begin{enumerate}
			\item Accedere alla chat;
			\item Visualizzare il prompt generato;
			\item Richiedere al sistema di copiare il contenuto del prompt;
			\item Visualizzare un messaggio di conferma una volta che il prompt è stato copiato negli appunti.				
		\end{enumerate}
		& N-D \\
		\hline TA.14 & Verificare che l'Utente possa visualizzare il contenuto della chat: 
		\begin{enumerate}
			\item Visualizzare la lista dei messaggi;
			\item Visualizzare un singolo messaggio nella lista;
			\item Visualizzare il mittente del messaggio;
			\item Visualizzare il contenuto del messaggio.						
		\end{enumerate}
		& N-D \\
		\hline TA.15 & Verificare che l'Utente possa eliminare la cronologia della chat. & N-D \\
		\hline TA.16 & Verificare che il Tecnico possa visualizzare il \glossario{log} elaborato durante la generazione del prompt. & N-D \\
		\hline TA.17 & Verificare che il Tecnico possa scaricare il file di \glossario{log}. & N-D \\
		\hline TA.18 & Verificare che il ChatBOT restituisca un avviso se la richiesta inserita dall'Utente non ha prodotto risultati rilevanti. & N-D \\
	\end{longtable}
\end{adjustwidth}
\egroup

\clearpage
\subsubsection{Tracciamento dei test di accettazione}

\bgroup
\begin{adjustwidth}{-0.5cm}{-0.5cm}
	\centering
	% MAX 12.5cm
  \begin{longtable}{|c|c|}
		\caption{Tracciamento test di accettazione}
  	\label{tab:tracciamento-test-accettazione} \\
    \hline
		\textbf{ID} & \textbf{Caso d'uso} \\ 
		\hline
		\endfirsthead

		\caption[]{Tracciamento test di accettazione (continua)} \\
		\hline
		\textbf{ID} & \textbf{Caso d'uso} \\ 
		\hline
		\endhead

		\hline
		\multicolumn{2}{|r|}{{Continua nella prossima pagina}} \\ 
		\hline
		\endfoot

		\hline
		\endlastfoot

    TA.1 & UC1, UC1.1, UC1.2\\
		\hline TA.2 & UC13, UC15, UC16, UC17\\
		\hline TA.3 & UC9, UC9.1, UC10, UC10.1, UC10.2, UC10.3, UC10.4, UC10.5\\
		\hline TA.4 & UC18\\
		\hline TA.5 & UC29\\
		\hline TA.6 & UC30\\
		\hline TA.7 & UC20\\
		\hline TA.8 & UC12\\
		\hline TA.9 & UC36\\
		\hline TA.10 & UC4\\
		\hline TA.11 & UC14, UC14.1, UC14.2, UC14.3, UC14.3.1, UC14.3.1.1, UC14.3.1.2\\
		\hline TA.12 & UC5\\
		\hline TA.13 & UC8\\
		\hline TA.14 & UC25, UC25.1, UC25.1.1, UC25.1.2\\
		\hline TA.15 & UC27\\
		\hline TA.16 & UC22\\
		\hline TA.17 & UC23\\
		\hline TA.18 & UC6\\
  \end{longtable}
\end{adjustwidth}
\egroup
\subsection{Checklist}

\par Le checklist sono strumenti che affiancano il team nell'attività di ispezione del codice e della documentazione, al fine di accertarsi che siano conformi alle specifiche e alle linee guida (pratiche e stili di codifica, coerenza della documentazione). È un metodo di analisi statica mirato a individuare gli errori più ricorrenti che possono manifestarsi nel prodotto in esame.

\clearpage
\subsubsection{Struttura della documentazione}

\bgroup
\begin{adjustwidth}{-0.5cm}{-0.5cm}
	% MAX 12.5cm
  \begin{longtable}{|P{4.5cm}|>{\raggedright\arraybackslash}P{9cm}|}
    \caption{Checklist - Struttura della documentazione}
  	\label{tab:check-struttura-documentazione} \\
    \hline
		\textbf{Titolo} & \textbf{Descrizione} \\ 
		\hline
		\endfirsthead

    \caption[]{Checklist - Struttura della documentazione (continua)} \\
		\hline
		\textbf{Titolo} & \textbf{Descrizione} \\ 
		\hline
		\endhead

		\hline
		\multicolumn{2}{|r|}{{Continua nella prossima pagina}} \\ 
		\hline
		\endfoot

		\hline
		\endlastfoot

    Riferimento a documenti soggetti al versionamento & Quando viene menzionato il contenuto di un documento soggetto al versionamento, il riferimento deve riportare, oltre al nome del documento, anche il numero di versione.\\
		\hline Riferimento a materiali online & Le risorse web, per loro stessa natura, sono mutevoli. Pertanto, il riferimento a materiali online deve riportare la data di ultimo accesso alla risorsa. Inoltre, il collegamento ipertestuale deve essere visibile direttamente come URL.\\
    \hline Didascalie ed etichette & Tutte le immagini e le tabelle devono essere corredate da una didascalia che ne descriva il contenuto, e da un'etichetta univoca che funga da riferimento globale.\\
    \hline Sezioni vuote o incomplete & Nessun documento deve contenere sezioni vuote o incomplete (ovvero sezioni il cui contenuto è descritto come "Todo").\\
    \hline Suddivisione indice & L'indice dei documenti deve essere suddiviso in tre sezioni:
    \begin{itemize}
      \item Indice generale;
      \item Elenco delle tabelle;
      \item Elenco delle figure.
    \end{itemize}\\
    \hline Occorrenze multiple di un termine nel \Gls & Quando un termine definito nel \Gls\ appare più volte all'interno di un documento, tutte le ricorrenze devono essere formattate in corsivo e marcate con una lettera \ped{G} in pedice (a meno di non compromettere la leggibilità). \\
    \hline Occorrenze multiple di un termine nel \Gls\ (verbali esterni) & Quando un termine definito nel \Gls\ appare più volte all'interno di un verbale esterno, solamente la prima ricorrenza deve essere formattata. \\
    \hline Introduzione \Gls\ (verbali esterni) & L'introduzione del \Gls\ nei verbali esterni deve essere diversa rispetto a quella degli altri documenti.\\
    \hline Punteggiatura elenchi puntati o numerati & La frase di introduzione di un elenco puntato o numerato deve terminare con i due punti. Le voci di un elenco, invece, devono finire con un punto se rappresentano la conclusione dell'elenco o sotto-elenco in questione, altrimenti con un punto e virgola.\\
    \hline Formato delle date & Tutte le date non incluse in un paragrafo discorsivo devono apparire nella forma "AAAA-MM-GG".\\
    \hline Indice di leggibilità & Le modifiche ai documenti devono rispettare la soglia di tollerabilità stabilita per l'Indice Gulpease.\\
    \hline Distribuzione verbali esterni & Nella distribuzione dei verbali esterni deve essere menzionata, oltre al gruppo fornitore e ai Committenti, anche la Proponente.\\
    \hline Ordinamento registro modifiche per data & Nel changelog, le modifiche devono essere ordinate dalla più recente alla più vecchia.\\
    \hline Ordinamento task per ID & I task devono essere disposti in ordine crescente sulla base del loro ID. \\
    \hline Menzione di un soggetto & Quando si menziona una persona, la formula da utilizzare è la seguente: "Nome Cognome". Per mantenere coerenza all’interno dei documenti e sfruttare i comandi \glossario{LaTeX}, il team ha adottato questa formula anche negli elenchi e nelle tabelle. Pertanto, i nomi non vengono disposti in ordine alfabetico, ma seguono l'ordinamento definito nel template globale. \\
  \end{longtable}
\end{adjustwidth}
\egroup

