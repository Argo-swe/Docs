\section{Introduzione}

\subsection{Scopo del documento}
\par Lo scopo del \PdQ\ è delineare un insieme di indici di valutazione e validazione del progetto, assieme a delle metriche di qualità che il prodotto deve rispettare. Gli obiettivi di qualità devono essere chiari, quantificabili e conformi ai requisiti e alle aspettative del cliente. I parametri vengono fissati dal team sulla base di standard qualitativi e dell'esperienza acquisita nell'arco dello svolgimento del progetto. In linea con la dinamicità del \PdQ, i range possono essere aggiustati o migliorati. Per tale motivo, viene fornito un cruscotto di valutazione della qualità, che monitora la capacità del team di rispettare le metriche stabilite durante il progetto.

\subsection{Riferimenti}

\subsubsection{Riferimenti normativi}
\begin{itemize}
  \item \NormeDiProgetto;
  \item Capitolato C9 - ChatSQL (Zucchetti S.p.A.): \\ \href{https://www.math.unipd.it/~tullio/IS-1/2023/Progetto/C9.pdf}{https://www.math.unipd.it/~tullio/IS-1/2023/Progetto/C9.pdf} \\ (Ultimo accesso: 2024-04-11);
  \item Slide PD2 - Corso di Ingegneria del Software - Regolamento del Progetto Didattico: \\ \href{https://www.math.unipd.it/~tullio/IS-1/2023/Dispense/PD2.pdf}{https://www.math.unipd.it/~tullio/IS-1/2023/Dispense/PD2.pdf} \\ (Ultimo accesso: 2024-04-11).
\end{itemize}

\subsubsection{Riferimenti informativi}
\begin{itemize}
  \item \AnalisiDeiRequisiti;
  \item Slide T7 - Corso di Ingegneria del Software - Qualità del software \\ \href{https://www.math.unipd.it/~tullio/IS-1/2023/Dispense/T7.pdf}{https://www.math.unipd.it/~tullio/IS-1/2023/Dispense/T7.pdf}  \\ (Ultimo accesso: 2024-07-01);
  \item Slide T8 - Corso di Ingegneria del Software - Qualità di processo \\ \href{https://www.math.unipd.it/~tullio/IS-1/2023/Dispense/T8.pdf}{https://www.math.unipd.it/~tullio/IS-1/2023/Dispense/T8.pdf}  \\ (Ultimo accesso: 2024-07-01);
  \item Slide T10 - Corso di Ingegneria del Software - Verifica e validazione: analisi statica \\ \href{https://www.math.unipd.it/~tullio/IS-1/2023/Dispense/T10.pdf}{https://www.math.unipd.it/~tullio/IS-1/2023/Dispense/T10.pdf}  \\ (Ultimo accesso: 2024-07-01);
  \item Slide T11 - Corso di Ingegneria del Software - Verifica e validazione: analisi dinamica \\ \href{https://www.math.unipd.it/~tullio/IS-1/2023/Dispense/T11.pdf}{https://www.math.unipd.it/~tullio/IS-1/2023/Dispense/T11.pdf}  \\ (Ultimo accesso: 2024-07-01);
  \item Standard ISO/IEC 9126: \\ \href{https://en.wikipedia.org/wiki/ISO/IEC_9126}{https://en.wikipedia.org/wiki/ISO/IEC\_9126}  \\ (Ultimo accesso: 2024-07-01);
  \item Panoramica generale sul test del software: \\ \href{https://www.ibm.com/it-it/topics/software-testing}{https://www.ibm.com/it-it/topics/software-testing}  \\ (Ultimo accesso: 2024-06-26);
  \item Tipologie di test del software: \\ \href{https://www.atlassian.com/it/continuous-delivery/software-testing/types-of-software-testing}{\nolinkurl{https://www.atlassian.com/it/continuous-delivery/software-testing/types-of-\-software-testing}}  \\ (Ultimo accesso: 2024-06-26);
  \item Test di unità: \\ \href{https://en.wikipedia.org/wiki/Unit_testing}{https://en.wikipedia.org/wiki/Unit\_testing}  \\ (Ultimo accesso: 2024-06-26);
  \item Test di integrazione: \\ \href{https://en.wikipedia.org/wiki/Integration_testing}{https://en.wikipedia.org/wiki/Integration\_testing}  \\ (Ultimo accesso: 2024-06-26);
  \item Test di sistema: \\ \href{https://vitolavecchia.altervista.org/differenza-tra-system-testing-e-system-integration-testing}{\nolinkurl{https://vitolavecchia.altervista.org/differenza-tra-system-testing-e-system-\-integration-testing}}  \\ (Ultimo accesso: 2024-06-26);
  \item Test di accettazione: \\ \href{https://vitolavecchia.altervista.org/tipologie-testing-software-test-di-accettazione}{\nolinkurl{https://vitolavecchia.altervista.org/tipologie-testing-software-test-di-accetta\-zione}}  \\ (Ultimo accesso: 2024-06-26);
  \item \Glossario;
  \item \PianoDiProgetto;
  \item Verbali interni ed esterni.
\end{itemize}

\subsection{Glossario} 
\GlossarioIntroduzione

\subsection{Note organizzative}

\par Il \PdQ\ è un documento dinamico; pertanto, la sua struttura e il suo contenuto sono soggetti a costanti aggiornamenti e miglioramenti.
