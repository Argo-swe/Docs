\subsection{Test di accettazione}

\par L'obiettivo dei test di accettazione è verificare se il sistema soddisfa le aspettative del Committente e del Proponente. I test di accettazione determinano se il software è pronto per essere rilasciato, e pertanto richiedono un focus sul comportamento degli utenti. Di seguito è riportato l'elenco dei test di accettazione:

\bgroup
\begin{adjustwidth}{-0.5cm}{-0.5cm}
	% MAX 12.5cm
 	\begin{longtable}{|P{1.5cm}|>{\raggedright}P{9.5cm}|>{\arraybackslash}P{1.5cm}|}
    \caption{Test di accettazione}
  	\label{tab:test-accettazione} \\
	  \hline
		\textbf{ID} & \textbf{Descrizione} & \textbf{Stato} \\ 
		\hline
		\endfirsthead

    \caption[]{Test di accettazione (continua)} \\
		\hline
		\textbf{ID} & \textbf{Descrizione} & \textbf{Stato} \\ 
		\hline
		\endhead

		\hline
		\multicolumn{3}{|r|}{{Continua nella prossima pagina}} \\ 
		\hline
		\endfoot

		\hline
		\endlastfoot

		TA.1 & & N-D \\
		\hline TA.2 & & N-D \\
	\end{longtable}
\end{adjustwidth}
\egroup

\subsubsection{Tracciamento dei test di accettazione}

\bgroup
\begin{adjustwidth}{-0.5cm}{-0.5cm}
	\centering
	% MAX 12.5cm
  \begin{longtable}{|c|c|}
		\caption{Tracciamento test di accettazione}
  	\label{tab:tracciamento-test-accettazione} \\
    \hline
		\textbf{ID} & \textbf{Caso d'uso} \\ 
		\hline
		\endfirsthead

		\caption[]{Tracciamento test di accettazione (continua)} \\
		\hline
		\textbf{ID} & \textbf{Caso d'uso} \\ 
		\hline
		\endhead

		\hline
		\multicolumn{2}{|r|}{{Continua nella prossima pagina}} \\ 
		\hline
		\endfoot

		\hline
		\endlastfoot

    TA.1 & RF.O.1, RF.O.1.1, RF.O.1.2\\
		\hline TA.2 & R.F.O.2\\
		\hline TA.3 & RF.O.13\\
		\hline TA.4 & RF.O.29\\
		\hline TA.5 & RF.O.30\\
		\hline TA.6 & RF.O.20\\
		\hline TA.7 & RF.O.31, RF.O.31.1\\
		\hline TA.8 & RF.O.31, RF.O.31.2\\
		\hline TA.9 & RF.O.32\\
		\hline TA.10 & RF.O.33\\
		\hline TA.11 & RF.O.28\\
		\hline TA.12 & RF.O.34\\
		\hline TA.13 & RF.O.18\\
		\hline TA.14 & RF.O.21\\
		\hline TA.15 & RF.O.3\\
		\hline TA.16 & RF.O.4\\
		\hline TA.17 & RF.O.14, RF.O.14.1, RF.O.14.2, RF.O.14.3, RF.O.14.3.1, RF.O.14.3.2\\
		\hline TA.18 & RF.O.23\\
		\hline TA.19 & RF.O.6\\
		\hline TA.20 & RF.O.11\\
		\hline TA.21 & RF.O.35\\
		\hline TA.22 & RF.O.12\\
		\hline TA.23 & RF.O.8\\
		\hline TA.24 & RF.O.46\\
		\hline TA.25 & RF.O.22\\
		\hline TA.26 & RF.O.23\\
		\hline TA.27 & RF.O.24\\
		\hline TA.28 & RF.O.25, RF.O.25.1, RF.O.25.1.1, RF.O.25.1.2\\
		\hline TA.29 & RF.O.9, RF.O.9.1, RF.O.10, RF.O.10.1, RF.O.10.2, RF.O.10.3, RF.O.10.4, RF.O.10.5\\
		\hline TA.30 & RF.D.45\\
		\hline TA.31 & RF.D.52, RF.D.7\\
  \end{longtable}
\end{adjustwidth}
\egroup