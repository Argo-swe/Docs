\subsubsection{Struttura documentazione}

\bgroup
\begin{adjustwidth}{-0.5cm}{-0.5cm}
	\centering
	% MAX 12.5cm
  \begin{longtable}{|c|>{\raggedright}|}
    \hline
		\textbf{Titolo} & \textbf{Descrizione} \\ 
		\hline
		\endfirsthead

		\hline
		\textbf{Titolo} & \textbf{Descrizione} \\ 
		\hline
		\endhead

		\hline
		\multicolumn{2}{|r|}{{Continua nella prossima pagina}} \\ 
		\hline
		\endfoot

		\hline
		\endlastfoot

    Riferimento a documenti soggetti al versionamento & Quando viene menzionato il contenuto di un documento soggetto al versionamento, il riferimento deve riportare, oltre al nome del documento, anche il numero di versione.\\
		\hline Riferimento a materiali online & Le risorse web, per loro stessa natura, sono mutevoli. Pertanto, il riferimento a materiali online deve riportare la data di ultimo accesso alla risorsa. Inoltre, il collegamento ipertestuale deve essere visibile direttamente come URL.\\
    \hline Didascalie ed etichette & Tutte le immagini e le tabelle devono essere corredate da una didascalia che ne descriva il contenuto, e da un'etichetta univoca che funga da riferimento globale.\\
    \hline Sezioni vuote o incomplete & Nessun documento deve contenere sezioni vuote o incomplete (ovvero sezioni il cui contenuto è descritto come "Todo").\\
    \hline Suddivisione indice & L'indice dei documenti deve essere suddiviso in tre sezioni:
    \begin{itemize}
      \item Indice generale;
      \item Elenco delle tabelle;
      \item Elenco delle figure.
    \end{itemize}\\
    \hline Occorrenze multiple di un termine nel \Gls & Quando un termine definito nel \Gls\ appare più volte all'interno di un documento, tutte le ricorrenze devono essere formattate in corsivo e marcate con una lettera \ped{G} in pedice. \\
    \hline Occorrenze multiple di un termine nel \Gls (verbali esterni) & Quando un termine definito nel \Gls\ appare più volte all'interno di un verbale esterno, solamente la prima ricorrenza deve essere formattata in corsivo e marcata con una lettera \ped{G} in pedice. \\
    \hline Punteggiatura elenchi puntati o numerati & La frase di introduzione di un elenco puntato o numerato deve terminare con i due punti. Le voci di un elenco, invece, devono finire con un punto se rappresentano la conclusione dell'elenco o sotto-elenco in questione, altrimenti con un punto e virgola.\\
    \hline Formato delle date & Tutte le date non incluse in un paragrafo discorsivo devono apparire nella forma "AAAA-MM-GG".\\
    \hline & \\
    \hline & \\
    \hline & \\
    \hline & \\
    \hline & \\
    \hline & \\
    \hline & \\
  \end{longtable}
\end{adjustwidth}
\egroup

