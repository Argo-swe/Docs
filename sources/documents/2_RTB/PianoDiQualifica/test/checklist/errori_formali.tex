\subsubsection{Errori formali}

\bgroup
\begin{adjustwidth}{-0.5cm}{-0.5cm}
	% MAX 12.5cm
  \begin{longtable}{|P{4.5cm}|>{\raggedright\arraybackslash}P{9cm}|}
    \caption{Checklist - Errori formali}
  	\label{tab:check-errori-formali} \\
    \hline
		\textbf{Titolo} & \textbf{Descrizione} \\ 
		\hline
		\endfirsthead

    \caption[]{Checklist - Errori formali (continua)} \\
		\hline
		\textbf{Titolo} & \textbf{Descrizione} \\ 
		\hline
		\endhead

		\hline
		\multicolumn{2}{|r|}{{Continua nella prossima pagina}} \\ 
		\hline
		\endfoot

		\hline
		\endlastfoot

    Nomi dei ruoli di progetto & I nomi dei ruoli di progetto devono avere la lettera iniziale minuscola. \\
    \hline Proponente & Il termine Proponente deve iniziare con la lettera maiuscola e, in via preferenziale, essere declinato al femminile. Tuttavia, l'uso maschile di Proponente è considerato corretto.\\
    \hline Cliente e Committente & I termini Cliente e Committente richiedono l'iniziale maiuscola solamente quando si riferiscono a un attore specifico e non ad un ruolo o entità astratta.\\
		\hline Repository & Il termine repository deve essere declinato al maschile.\\
    \hline IA e AI & Possono essere utilizzati entrambi gli acronimi (intelligenza artificiale o, come indicato nel capitolato, artificial intelligence).\\
    \hline Back-end e backend & Si possono utilizzare entrambe le forme, con o senza trattino. Lo stesso vale per "front-end" e "frontend".\\
    \hline Sintassi e ortografia & Il testo deve essere privo di errori di sintassi e ortografia. \\
    \hline \glossario{typo} & È essenziale limitare gli errori tipografici e le sviste, specialmente nella scrittura del codice. \\
    \hline Linguaggio & I documenti devono essere redatti in modo impersonale dal punto di vista della forma verbale. Inoltre, è opportuno adottare un linguaggio il più formale possibile, soprattutto nella stesura dei verbali esterni. \\
    \hline Versioni estese di abbreviazioni & Le versioni estese delle sigle devono rispettare la forma delle abbreviazioni (es.: AdR diventa Analisi dei requisiti, WoW diventa Way of Working). \\
		\hline Soggetto della frase & Il soggetto di un discorso deve sempre essere evidenziato nella sua introduzione. \\
		\hline D eufonica & La d eufonica deve essere inserita solo quando le due vocali sono uguali. \\
  \end{longtable}
\end{adjustwidth}
\egroup