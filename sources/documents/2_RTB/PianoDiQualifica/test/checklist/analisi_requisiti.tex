\subsubsection{Analisi dei Requisiti}

\bgroup
\begin{adjustwidth}{-0.5cm}{-0.5cm}
	% MAX 12.5cm
  \begin{longtable}{|P{4.5cm}|>{\raggedright\arraybackslash}P{9cm}|}
    \caption{Checklist - Analisi dei Requisiti}
  	\label{tab:check-analisi-requisiti} \\
    \hline
		\textbf{Titolo} & \textbf{Descrizione} \\ 
		\hline
		\endfirsthead

    \caption[]{Checklist - Analisi dei Requisiti (continua)} \\
		\hline
		\textbf{Titolo} & \textbf{Descrizione} \\ 
		\hline
		\endhead

		\hline
		\multicolumn{2}{|r|}{{Continua nella prossima pagina}} \\ 
		\hline
		\endfoot

		\hline
		\endlastfoot

    Correlazione casi d'uso - requisiti & Ciascun caso d'uso dovrebbe essere associato a uno o più requisiti. \\
    \hline Ordinamento requisiti & I requisiti devono essere ordinati secondo la stessa disposizione dei casi d'uso. \\
		\hline Diagrammi dei casi d'uso & Le inclusioni e le estensioni di un caso d'uso dovrebbero essere rappresentate nello stesso diagramma \glossario{UML} del caso d'uso. Le relazioni di generalizzazione, invece, possono essere visualizzate, per motivi di spazio, in un altro diagramma.\\
    \hline Coerenza diagramma-descrizione UC & Il diagramma UML e la descrizione dei casi d'uso devono essere consistenti.\\
    \hline Completezza descrizione UC & La descrizione dei casi d'uso deve essere esaustiva, integrando le informazioni già riportate nel diagramma UML.\\
    \hline Distinzione tra requisiti funzionali e non funzionali & La separazione tra i requisiti funzionali e non funzionali deve essere chiara. \\
    \hline Tracciamento dei requisiti & Ogni requisito deve essere ricavato da almeno una fonte. Il tracciamento dei requisiti (composto dalle coppie requisito-fonti) deve essere privo di errori.\\
  \end{longtable}
\end{adjustwidth}
\egroup