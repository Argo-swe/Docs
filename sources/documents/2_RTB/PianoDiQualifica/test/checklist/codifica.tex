\subsubsection{Codifica}

\bgroup
\begin{adjustwidth}{-0.5cm}{-0.5cm}
	% MAX 12.5cm
  \begin{longtable}{|P{4.5cm}|>{\raggedright\arraybackslash}P{9cm}|}
    \caption{Checklist - Codifica}
  	\label{tab:check-codifica} \\
    \hline
		\textbf{Titolo} & \textbf{Descrizione} \\ 
		\hline
		\endfirsthead

    \caption[]{Checklist - Codifica (continua)} \\
		\hline
		\textbf{Titolo} & \textbf{Descrizione} \\ 
		\hline
		\endhead

		\hline
		\multicolumn{2}{|r|}{{Continua nella prossima pagina}} \\ 
		\hline
		\endfoot

		\hline
		\endlastfoot

    Nomi esplicativi & I nomi di classi, metodi, attributi e variabili devono essere "parlanti", in quanto rappresentano la prima forma di documentazione del codice. \\
    \hline Regole di nomenclatura & Le regole specificate nelle \NdP\ devono essere rispettate. Tali regole possono includere:
    \begin{itemize}
      \item Uso di CamelCase (es.: firstName) o snake\_case (es.: first\_name);
      \item Prefissi o suffissi per tipo di dato;
      \item Convenzioni per le costanti.
    \end{itemize}\\
    \hline Header & Tutti i file devono contenere un header conforme alle regole definite nelle \NdP. \\
    \hline Numerosità dei commenti & Porzioni di codice o metodi rilevanti dovrebbero essere preceduti da un commento. Per contro, è opportuno evitare commenti superflui che non migliorano la leggibilità. \\
		\hline Commenti significativi & I commenti devono essere significativi, ossia devono fornire in modo diretto informazioni utili sul funzionamento del codice.\\
  \end{longtable}
\end{adjustwidth}
\egroup