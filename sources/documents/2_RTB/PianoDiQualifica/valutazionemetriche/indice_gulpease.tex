\subsection{M.PD.4 - Indice Gulpease}
\begin{figure}[H]
    \centering
    \includegraphics[width=\textwidth]{assets/indice_gulpease.pdf}
    \caption{M.PD.4 - Indice Gulpease}
\end{figure}

\par Il grafico riporta la media degli Indici Gulpease di tutti i documenti, sia interni che esterni. Il valore medio oscilla tra 55 e 70; ciò significa che i documenti sono comprensibili anche per chi possiede una licenza media. I valori più bassi sono stati rilevati nella fase centrale della \glossario{RTB}, poiché il team ha apportato modifiche sostanziali alla documentazione senza applicare un controllo rigoroso sulla leggibilità. Tuttavia, i valori misurati rientrano nella soglia di tollerabilità. Per quanto concerne i singoli documenti, l’AdR\ ha evidenziato un indice di leggibilità inferiore rispetto agli altri, in quanto contiene frasi lunghe, specialmente nelle tabelle dei requisiti. L’obiettivo del gruppo è di rielaborare, ove possibile, le porzioni più prolisse, riformulando il discorso e/o spezzando le frasi. Di seguito è riportato l’Indice Gulpease dei singoli documenti (per i verbali viene menzionato il valore medio).