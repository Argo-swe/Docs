\section{C}


\vspace{2em}
\subsection*{Capitolato}
Il capitolato è un documento contrattuale che definisce le specifiche e le condizioni di un progetto, un'appalto o un contratto. È di solito emesso da un'organizzazione o da un cliente per descrivere i requisiti, gli obiettivi e le condizioni contrattuali che devono essere soddisfatte dal fornitore o dall'appaltatore. Il capitolato fornisce dettagli sullo scopo del progetto, sulle funzionalità richieste, sui vincoli tecnici ecc. È utilizzato come base per la negoziazione e l'esecuzione del contratto tra le parti coinvolte nel progetto.


\vspace{2em}
\subsection*{Casi d'uso}
I casi d'uso sono una tecnica utilizzata nell'ingegneria del software per descrivere le interazioni tra un sistema e gli \glossario{attori} esterni o interni ad esso. Ogni caso d'uso rappresenta uno scenario completo di interazione che descrive come un attore utilizza il sistema per raggiungere un obiettivo specifico. Ogni caso d'uso è composto da una serie di passi o azioni che il sistema compie in risposta alle azioni dell'attore. I casi d'uso sono spesso rappresentati graficamente mediante diagrammi dei casi d'uso e sono utilizzati come strumento per definire e analizzare i requisiti del sistema, nonché per guidare il processo di progettazione e sviluppo del software.


\vspace{2em}
\subsection*{Commit}
Un commit è un'operazione in un sistema di controllo versione che salva le modifiche apportate al codice sorgente o ai documenti in un repository. Rappresenta una "istantanea" dello stato del progetto in un determinato momento, accompagnata da un messaggio che descrive le modifiche effettuate. Il commit permette di tracciare la storia delle modifiche, facilitando il lavoro collaborativo e il mantenimento del codice, consentendo di tornare a versioni precedenti se necessario.
