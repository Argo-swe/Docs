\section{B}

\vspace{2em}
\subsection*{Base di dati}
Una base di dati è una raccolta organizzata di dati strutturati, generalmente memorizzata e accessibile elettronicamente da un sistema informatico. Le basi di dati sono progettate per consentire una gestione efficiente dei dati, facilitando operazioni di inserimento, aggiornamento, eliminazione e interrogazione. Utilizzano sistemi di gestione delle basi di dati (DBMS) per garantire integrità, sicurezza e disponibilità dei dati. Le basi di dati possono essere di diversi tipi, tra cui relazionali, NoSQL, gerarchiche e a grafo, ognuna adatta a specifiche esigenze applicative.
\par Vedi \glossario{database}

\vspace{2em}
\subsection*{Best practices}
Le best practices, o migliori pratiche, sono metodologie, processi o tecniche riconosciute come standard di riferimento per l'esecuzione di determinate attività o per il raggiungimento di obiettivi specifici. Sono sviluppate attraverso l'esperienza, l'analisi dei risultati e il confronto con altre organizzazioni o professionisti del settore e sono considerate efficaci e efficienti nel fornire risultati desiderati. Le best practices sono utilizzate in vari campi, tra cui lo sviluppo del software, la gestione dei progetti, la sicurezza informatica, la gestione delle risorse umane e altro ancora, e sono adattate alle esigenze specifiche di ciascuna organizzazione o situazione.
