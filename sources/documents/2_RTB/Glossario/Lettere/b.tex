\section{B}

\vspace{2em}
\subsection*{Backend}
\par Il backend di un'applicazione software è la componente che gestisce la logica di business, l'elaborazione dei dati e l'interazione con il database. Non è visibile agli utenti finali, ma supporta tutte le funzionalità dell'interfaccia utente. Il backend include server, database, API e servizi che orchestrano le operazioni principali. La progettazione del backend è cruciale per le prestazioni, la scalabilità e la sicurezza dell'applicazione. Tecnologie comuni per il backend includono Python, Java e Node.js.

\vspace{2em}
\subsection*{Backlog}
\par Il backlog è un elenco ordinato di attività, funzionalità, miglioramenti e bug che devono essere sviluppati e risolti in un progetto software. Funge da strumento di pianificazione per il team di sviluppo, che seleziona gli elementi da implementare in ciascun ciclo di lavoro o sprint. Il backlog è dinamico e viene aggiornato regolarmente in base ai feedback degli stakeholder e alle esigenze del progetto. Elementi tipici del backlog includono user story, task tecnici e bug fix.

\vspace{2em}
\subsection*{Base di dati}
%Una base di dati è una raccolta organizzata di dati strutturati, generalmente memorizzata e accessibile elettronicamente da un sistema informatico. Le basi di dati sono progettate per consentire una gestione efficiente dei dati, facilitando operazioni di inserimento, aggiornamento, eliminazione e interrogazione. Utilizzano sistemi di gestione delle basi di dati (DBMS) per garantire integrità, sicurezza e disponibilità dei dati. Le basi di dati possono essere di diversi tipi, tra cui relazionali, NoSQL, gerarchiche e a grafo, ognuna adatta a specifiche esigenze applicative.
\par Vedi \glossario{database}.

\vspace{2em}
\subsection*{Benchmark}
\par Un benchmark è un processo di misurazione delle prestazioni di un sistema o componente software rispetto a uno standard o a sistemi concorrenti. Viene utilizzato per valutare l'efficienza, la velocità, la scalabilità e altri attributi di qualità. I benchmark aiutano a identificare i punti deboli e a confrontare diverse soluzioni per determinare quella più adeguata. Esempi di benchmark includono test di velocità dei database, misure di throughput delle reti e valutazioni delle prestazioni delle applicazioni web.

\vspace{2em}
\subsection*{Best practice}
\par Le best practice, o migliori pratiche, sono metodologie, processi e tecniche riconosciute come standard di riferimento per l'esecuzione di determinate attività o per il raggiungimento di obiettivi specifici. Sono sviluppate attraverso l'esperienza, l'analisi dei risultati e il confronto con altre organizzazioni o professionisti del settore e sono considerate efficaci ed efficienti nel fornire i risultati desiderati.

\vspace{2em}
\subsection*{Bootstrap}
\par Bootstrap è un framework front-end open-source per lo sviluppo rapido di interfacce web responsive e mobili. Creato da Twitter, offre una vasta gamma di componenti predefiniti, come griglie, pulsanti, moduli e menu di navigazione, che possono essere facilmente personalizzati e integrati. Bootstrap utilizza HTML, CSS e JavaScript per garantire un design coerente e adattabile su diverse piattaforme e dispositivi. È ampiamente utilizzato per ottimizzare lo sviluppo e garantire la qualità visiva delle applicazioni web.

\vspace{2em}
\subsection*{Branch}
\par Un branch rappresenta una linea di sviluppo indipendente. I rami fungono da astrazione per il processo di modifica/stage/commit. Vengono considerati come una directory di lavoro, un'area di staging e una cronologia del progetto completamente nuove. I commit vengono registrati nella cronologia del ramo corrente, il che si traduce in un fork nella cronologia del progetto.

\vspace{2em}
\subsection*{Bug}
\par Un bug è un'anomalia che porta al malfunzionamento di un software, per esempio producendo un risultato inatteso o non valido, tipicamente dovuto a un errore nella scrittura del codice sorgente di un programma. Un bug, in sostanza, è un difetto software di tipo funzionale, nonostante a volte il termine venga usato per indicare anche falle di sicurezza (vulnerabilità) o per evidenziare inadeguatezze prestazionali.

\vspace{2em}
\subsection*{Build}
\par Una build è un processo di trasformazione del codice sorgente in un artefatto eseguibile. Talvolta lo stesso termine viene utilizzato per indicare non tanto il processo quanto il risultato finale, ovvero l'artefatto ottenuto o il corrispondente codice numerico (o alfanumerico). Tale codice permette di distinguere l'artefatto da quelli ottenuti in precedenza nell'ambito dell'evoluzione di un prodotto software.
