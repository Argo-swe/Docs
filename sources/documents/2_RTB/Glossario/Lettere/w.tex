\section{W}



\vspace{2em}
\subsection*{Web app}
Una web app, abbreviazione di "applicazione web", è un'applicazione software che viene eseguita su un server web e viene accessa dagli utenti attraverso un browser web su una rete Internet. Le web app sono progettate per funzionare su diverse piattaforme e dispositivi senza la necessità di essere installate localmente sul dispositivo dell'utente. Possono essere sviluppate utilizzando linguaggi di programmazione web come HTML, CSS e JavaScript, e possono offrire una vasta gamma di funzionalità, tra cui la gestione dei dati, l'interazione utente, la collaborazione in tempo reale e l'integrazione con altri servizi web. Le web app sono popolari per la loro facilità di distribuzione, aggiornamento e accesso da parte degli utenti da qualsiasi luogo e dispositivo connesso a Internet.

\vspace{2em}
\subsection*{WoW : Way of working}
Il WoW, acronimo di "Way of Working", è un insieme di pratiche, processi e metodologie adottati da un team o un'organizzazione per organizzare e gestire il proprio lavoro. Il WoW definisce le modalità operative che guidano il modo in cui i membri del team collaborano, comunicano, pianificano e eseguono le attività. Include procedure per la gestione del progetto, l'organizzazione del lavoro, la comunicazione interna, la risoluzione dei problemi e altro ancora. Un efficace WoW è progettato per massimizzare l'efficienza, la produttività e la qualità del lavoro svolto dal team.