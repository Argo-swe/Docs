\section{W}

\vspace{2em}
\subsection*{Way of Working}
Il WoW, acronimo di "Way of Working", è un insieme di pratiche, processi e metodologie adottati da un team o un'organizzazione per organizzare e gestire il proprio lavoro. Il WoW definisce le modalità operative che guidano il modo in cui i membri del team collaborano, comunicano, pianificano e eseguono le attività. Include procedure per la gestione del progetto, l'organizzazione del lavoro, la comunicazione interna, la risoluzione dei problemi e altro ancora. Un efficace WoW è progettato per massimizzare l'efficienza, la produttività e la qualità del lavoro svolto dal team.

\vspace{2em}
\subsection*{Web App}
Una web app, abbreviazione di "applicazione web", è un'applicazione software che viene eseguita su un server web e viene accessa dagli utenti attraverso un browser web su una rete Internet. Le web app sono progettate per funzionare su diverse piattaforme e dispositivi senza la necessità di essere installate localmente sul dispositivo dell'utente.

\vspace{2em}
\subsection*{Workflow}
Il Workflow è una sequenza di operazioni, attività o compiti necessari per completare un processo, spesso automatizzato e gestito tramite software per migliorare l'efficienza e la produttività.

\vspace{2em}
\subsection*{WoW}
\par Vedi \glossario{Way of working}
