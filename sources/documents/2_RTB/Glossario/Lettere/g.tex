\section{G}

\vspace{2em}
\subsection*{GHCR}

\vspace{2em}
\subsection*{Git}
Git è un sistema di controllo versione distribuito utilizzato principalmente per il coordinamento del lavoro tra programmatori nello sviluppo di software. Permette di tracciare le modifiche nel codice sorgente durante lo sviluppo del progetto, facilitando la collaborazione e la gestione delle versioni. Con Git, ogni sviluppatore ha una copia completa del repository di codice, inclusa la cronologia completa delle modifiche, il che consente di lavorare in modalità offline e di integrare le modifiche in modo efficiente. Git è noto per la sua velocità, flessibilità e capacità di gestire ramificazioni e fusioni complesse.


\vspace{2em}
\subsection*{GitHub}
GitHub è una piattaforma di hosting per progetti software basati su \glossario{Git}. Fornisce strumenti per la gestione del codice sorgente, la collaborazione tra sviluppatori e la pubblicazione di progetti open source. Oltre al controllo versione tramite Git, GitHub offre funzionalità come problemi, richieste pull, gestione delle attività e integrazioni con strumenti di sviluppo e servizi di continuos integration. È ampiamente utilizzato dalla comunità open source e dalle aziende per lo sviluppo e la condivisione di software.

\vspace{2em}
\subsection*{Google Docs}
Google Docs è un'applicazione web basata su cloud per la creazione, la modifica e la condivisione di documenti di testo, fogli di calcolo e presentazioni. Consente a più utenti di collaborare in tempo reale sugli stessi documenti, consentendo la modifica simultanea da parte di più persone e fornendo strumenti per commentare, revisionare e formattare il testo. Google Docs sincronizza automaticamente i documenti tra tutti i dispositivi collegati e offre funzionalità di condivisione avanzate.

\vspace{2em}
\subsection*{Google Sheets}
Google Sheets è un'applicazione web basata su cloud per la creazione, l'organizzazione e l'analisi di fogli di calcolo. Offre funzionalità simili a Microsoft Excel, consentendo agli utenti di inserire dati, eseguire calcoli, creare grafici e tabelle pivot. Come Google Docs, Google Sheets supporta la collaborazione in tempo reale, consentendo a più utenti di lavorare sugli stessi fogli di calcolo contemporaneamente e sincronizzando automaticamente le modifiche tra i dispositivi.
