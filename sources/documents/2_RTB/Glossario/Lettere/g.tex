\section{G}

\vspace{2em}
\subsection*{GGUF}
\par GGUF (GPT-Generated Unified Format) è un formato introdotto per semplificare il processo di archiviazione ed elaborazione di modelli linguistici di grandi dimensioni come GPT. Questo formato viene spesso impiegato in contesti che coinvolgono i modelli LLaMA (Large Language Model Meta AI).

\vspace{2em}
\subsection*{GHCR}
\par Il registro dei contenitori è un servizio che permette agli utenti di archiviare le immagini di un contenitore Docker all'interno di un'organizzazione o di un account personale; consente inoltre di associare un'immagine a un repository. Questo servizio è spesso integrato nei flussi di lavoro DevOps. Pubblicando un'immagine nel registro dei contenitori di GitHub, il team di sviluppo può ridurre significativamente il tempo di esecuzione di un workflow.

\vspace{2em}
\subsection*{Git}
\par Git è un sistema di controllo di versione distribuito utilizzato principalmente per il coordinamento del lavoro tra programmatori nello sviluppo software. Consente di tracciare le modifiche nel codice sorgente durante lo sviluppo del progetto, facilitando la collaborazione e la gestione delle versioni. Ogni sviluppatore possiede una copia completa del repository, inclusa la cronologia completa delle modifiche, il che consente di lavorare in modalità offline e di integrare i cambiamenti in modo efficiente. Git è noto per la sua velocità, flessibilità e capacità di gestire ramificazioni e fusioni complesse.

\vspace{2em}
\subsection*{GitHub}
\par GitHub è una piattaforma di hosting per progetti software basati su \glossario{Git}. Fornisce strumenti per la gestione del codice sorgente, la collaborazione tra sviluppatori e la pubblicazione di progetti open source. Oltre al controllo di versione, GitHub offre funzionalità come issue, pull request, gestione delle attività e integrazioni con strumenti di sviluppo e servizi di \glossario{continuos integration}. È ampiamente utilizzato dalla comunità open source e dalle aziende per lo sviluppo e la condivisione di software.

\vspace{2em}
\subsection*{Google Docs}
\par Google Docs è un'applicazione web-based di elaborazione testuale offerta da Google all'interno della suite Google Workspace (precedentemente G Suite). Consente agli utenti di creare, modificare e condividere documenti online in tempo reale. Le funzionalità disponibili includono collaborazione simultanea, tracciamento delle revisioni, integrazione con altri servizi Google (come Google Drive e Google Sheets), e la possibilità di integrare componenti aggiuntivi. Google Docs supporta diversi formati di file, inclusi .docx, .pdf e .odt.

\vspace{2em}
\subsection*{Google Drive}
\par Google Drive è un servizio cloud di archiviazione e gestione documentale offerto da Google. Consente agli utenti di memorizzare, sincronizzare e condividere file attraverso dispositivi collegati a Internet. Offre 15 GB di spazio gratuito, espandibile a pagamento. Google Drive supporta la collaborazione in tempo reale, permettendo agli utenti di modificare documenti, fogli di calcolo e presentazioni simultaneamente. Integra funzionalità avanzate di ricerca e organizzazione dei file, migliorando l'accessibilità e la gestione dei documenti. È parte integrante di Google Workspace, agevolando l'integrazione con altre applicazioni come Google Docs e Google Sheets.

\vspace{2em}
\subsection*{Google Moduli}
\par Google Moduli (Google Forms) è uno strumento di Google che consente agli utenti di creare moduli, sondaggi e quiz online. Parte della suite Google Workspace, offre funzionalità per la raccolta di dati, l'integrazione con Google Sheets per l'analisi delle risposte, e la personalizzazione dei moduli con vari tipi di domande, logiche ramificate e temi. È utilizzato in diversi contesti, tra cui istruzione, ricerca di mercato e gestione di eventi, per raccogliere e organizzare informazioni in modo intuitivo.

\vspace{2em}
\subsection*{Google Sheets}
\par Google Sheets è un'applicazione web di fogli di calcolo inclusa nella suite Google Workspace. Consente agli utenti di creare, modificare e collaborare su fogli di calcolo in tempo reale da qualsiasi dispositivo connesso a Internet. Supporta funzioni avanzate come formule complesse, grafici dinamici e tabelle pivot. La caratteristica principale è la collaborazione simultanea, permettendo a gruppi di persone di lavorare sullo stesso documento contemporaneamente. Google Sheets integra anche strumenti di automazione tramite Google Apps Script e offre opzioni avanzate per la gestione delle revisioni e dei permessi di accesso, rendendolo uno strumento essenziale per la gestione dati e la produttività team-oriented.
