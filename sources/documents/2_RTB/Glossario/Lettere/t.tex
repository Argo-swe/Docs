\section{T}

\vspace{2em}
\subsection*{Template}
\par Un template è un modello predefinito utilizzato per creare documenti, pagine web o altre risorse digitali. Fornisce una struttura di base che può essere personalizzata per soddisfare requisiti specifici, agevolando la creazione di contenuti consistenti e professionali. I template sono ampiamente utilizzati in design grafico, sviluppo web e produzione di documenti.

\vspace{2em}
\subsection*{Testing}
\par Il testing è un processo di verifica finalizzato a identificare eventuali difetti e valutare il corretto funzionamento del software rispetto alle specifiche. Include test di unità, test di integrazione e test di sistema. Il testing è essenziale per garantire la qualità del software, riducendo il rischio di errori e migliorando la stabilità del prodotto finale.

\vspace{2em}
\subsection*{Ticket}
\par Un ticket è un elemento utilizzato nei sistemi di tracciamento delle attività per registrare un compito (task, bug, story) da completare. I ticket contengono informazioni dettagliate sull'attività, inclusi lo stato corrente, la priorità, la descrizione, i tempi di risoluzione previsti, nonché eventuali commenti aggiuntivi o riferimenti all'ambiente di sviluppo. I ticket vengono utilizzati per monitorare lo stato e il progresso delle attività nel tempo.

\vspace{2em}
\subsection*{Token}
\par Nel contesto dei modelli di intelligenza artificiale, un token rappresenta l'unità di testo più piccola che il modello elabora durante il processo di analisi del linguaggio naturale. Un token può essere una singola parola, un carattere, o anche una sotto-unità di parole (come un prefisso o un suffisso). Nei modelli basati sui transformers, i token vengono codificati come vettori numerici attraverso embedding che catturano il significato semantico e sintattico del testo. L'uso dei token permette al modello di comprendere e generare il linguaggio naturale, semplificando task complessi come traduzione, analisi del sentiment e generazione di testo appropriato.

\vspace{2em}
\subsection*{Trello}
\par Trello è una piattaforma di gestione di progetto basata su board virtuali e liste. Consente agli utenti di organizzare e monitorare il lavoro in modo visuale, suddividendo i progetti in schede (cards) che possono essere spostate tra colonne rappresentanti stati diversi. Trello offre funzionalità per la collaborazione in tempo reale, assegnazione di compiti, aggiunta di note o commenti e integrazione con altri strumenti. Può essere utilizzato per la gestione di progetti personali e attività di sviluppo software in ambito professionale.

\vspace{2em}
\subsection*{txtai}
\par Txtai è una libreria open-source per l'elaborazione del linguaggio naturale (NLP) che si concentra sulla ricerca semantica all'interno di grandi raccolte di "documenti". Utilizza tecniche di apprendimento automatico per l'elaborazione del testo e l'estrazione di informazioni, consentendo agli utenti di eseguire ricerche semantiche e identificare relazioni tra le frasi. Txtai offre funzionalità avanzate di ricerca e analisi, ottimizzando l'esplorazione e la comprensione di grandi volumi di dati testuali.

\vspace{2em}
\subsection*{Typo}
\par Il termine typo si riferisce a un errore di battitura o a un refuso. Typo è l'abbreviazione di typographical error e viene spesso utilizzato in ambito informatico per indicare un errore commesso, per fretta o per distrazione, nella scrittura del codice.
