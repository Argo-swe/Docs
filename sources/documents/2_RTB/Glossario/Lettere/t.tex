\section{T}

\vspace{2em}
\subsection*{Template}
Un template è un modello predefinito utilizzato per creare documenti, pagine web o altre risorse digitali. Fornisce una struttura di base che può essere personalizzata per soddisfare requisiti specifici, facilitando la creazione di contenuti consistenti e professionali. I template sono ampiamente utilizzati in design grafico, sviluppo web e produzione di documenti, riducendo il tempo di sviluppo e migliorando l'efficienza nella creazione di contenuti visivi e testuali.

\vspace{2em}
\subsection*{Testing}
Il testing software è il processo di valutazione del software per identificare difetti e verificare che funzioni correttamente secondo le specifiche. Include test di unità, test di integrazione e test di sistema, garantendo che il software sia robusto, sicuro e conforme ai requisiti utente. Il testing è cruciale per garantire la qualità del software e la soddisfazione del cliente, riducendo il rischio di errori e migliorando la stabilità e le performance del prodotto finale.

\vspace{2em}
\subsection*{Ticket}
Un ticket è un elemento di tracciamento utilizzato nei sistemi di gestione delle issue o dei problemi per registrare una richiesta di assistenza, un problema o un compito da completare. I ticket contengono informazioni dettagliate sulla richiesta, inclusi il suo stato corrente, la priorità, la descrizione del problema, i tempi di risoluzione previsti e assegnati, nonché eventuali commenti aggiuntivi o aggiornamenti relativi alla richiesta. I ticket vengono utilizzati per tenere traccia dei compiti da svolgere, assegnare responsabilità ai membri del team e monitorare lo stato e il progresso delle attività nel tempo.

\vspace{2em}
\subsection*{Token}

\paragraph*{Definizione 1:}
Un token è un identificatore univoco il cui scopo è certificare la proprietà di un bene digitale o consentire l'accesso a un servizio, come ad esempio le API di GitHub;

\paragraph*{Definizione 2:}
Un token è un'unità di testo che rappresenta un segmento (parole o caratteri) di una stringa. I modelli di intelligenza artificiale elaborano i messaggi convertendoli in una sequenza di token per generare risposte adeguate.

\vspace{2em}
\subsection*{Trello}
Trello è una piattaforma di gestione progetti basata su board virtuali e liste. Consente agli utenti di organizzare e monitorare il lavoro in modo visuale, suddividendo i progetti in schede (cards) che possono essere spostate tra colonne rappresentanti stati diversi. Trello offre funzionalità per la collaborazione in tempo reale, assegnazione di compiti, aggiunta di note e commenti, e integrazioni con altri strumenti di produttività. È ampiamente utilizzato per la gestione di progetti personali, team di lavoro, attività di sviluppo software e molto altro.

\vspace{2em}
\subsection*{txtai}
Txtai è una libreria open-source per l'elaborazione del linguaggio naturale (NLP) che si concentra sulla ricerca e l'analisi di testo in grandi raccolte di documenti. Utilizza tecniche di apprendimento automatico per l'elaborazione del testo e l'estrazione di informazioni, consentendo agli utenti di eseguire ricerche semantiche, generare riassunti automatici, identificare relazioni tra documenti e molto altro. Txtai offre funzionalità avanzate per l'analisi del testo, rendendo più facile e veloce esplorare e comprendere grandi volumi di dati testuali.
