\section{F}

\vspace{2em}
\subsection*{FAISS}
\par FAISS (Facebook AI Similarity Search) è una libreria open-source sviluppata da Facebook AI Research per agevolare l'indicizzazione e la ricerca efficiente di grandi set di vettori. Utilizzato prevalentemente per l'implementazione di sistemi di ricerca basati su similarità, FAISS supporta operazioni come la ricerca dei vicini più prossimi (Nearest Neighbor Search) e la classificazione su vasta scala. La libreria è ottimizzata per alte prestazioni su GPU e CPU, rendendola ideale per applicazioni di machine learning e deep learning che richiedono un rapido accesso e confronto di rappresentazioni vettoriali.

\vspace{2em}
\subsection*{FastAPI}
\par FastAPI è un framework web moderno e ad alte prestazioni per la creazione di API con Python. Uno dei vantaggi di FastAPI è la generazione automatica della documentazione interattiva delle API. Inoltre, FastAPI sfrutta le annotazioni di tipo di Python per definire e validare automaticamente i dati in input e in output. Questo consente al team di sviluppo di mantenere un controllo rigoroso sui tipi di dati, riducendo il tasso di errori.

\vspace{2em}
\subsection*{Feature branch}
\par Un feature branch è una branca (branch) del codice sorgente creata per sviluppare una specifica funzionalità o feature in modo isolato rispetto al resto del codice. Questo approccio consente agli sviluppatori di lavorare su nuove caratteristiche senza influenzare il codice stabile del branch principale (spesso chiamato master o main). Una volta completata e testata una funzionalità, il feature branch viene unito (merged) nel ramo principale attraverso una pull request, facilitando la revisione del codice e la gestione delle versioni.

\vspace{2em}
\subsection*{Flask}
\par Flask è un microframework per lo sviluppo di applicazioni web in Python. È progettato per essere leggero e modulare, consentendo agli sviluppatori di scegliere e configurare solo i componenti necessari per il progetto. Flask offre funzionalità di base come il routing delle URL e la gestione delle richieste, e può essere esteso con numerosi plugin per aggiungere funzionalità avanzate. La sua semplicità e flessibilità lo rendono una scelta popolare sia per piccoli progetti che per applicazioni web più complesse.

\vspace{2em}
\subsection*{Fornitore}
\par Un fornitore, nel contesto dello sviluppo software, è un'azienda o un individuo che fornisce prodotti, servizi o soluzioni tecnologiche a un'altra organizzazione. I fornitori possono offrire una vasta gamma di servizi, inclusi sviluppo software, consulenza tecnologica, supporto e manutenzione. La scelta del fornitore è cruciale per garantire che le soluzioni tecnologiche adottate soddisfino i requisiti dell'azienda cliente in termini di qualità, costo, tempistiche e supporto tecnico.

\vspace{2em}
\subsection*{Framework}
\par Un framework è una struttura di supporto utilizzata nello sviluppo software per fornire una base standardizzata su cui costruire e sviluppare applicazioni. Consiste in un insieme di librerie, strumenti e best practice che ottimizzano il processo di sviluppo, promuovendo la coerenza e la riusabilità del codice. I framework possono essere progettati per specifici linguaggi di programmazione o per tipologie particolari di applicazioni, quali web, desktop o mobile.

\vspace{2em}
\subsection*{Frontend}
\par Il frontend è la porzione di un'applicazione software o di un sito web con cui l'utente interagisce direttamente. Comprende tutti gli elementi visivi, come layout, bottoni, immagini, testi, moduli e interfacce utente, ed è implementato utilizzando tecnologie come HTML, CSS e JavaScript. Gli sviluppatori frontend si concentrano sull'esperienza utente (UX) e sull'interfaccia utente (UI), assicurando che l'applicazione sia intuitiva, reattiva e visivamente attraente. Il frontend comunica con il backend per recuperare e visualizzare i dati.
