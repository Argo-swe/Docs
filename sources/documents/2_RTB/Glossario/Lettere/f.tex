\section{F}
\vspace{2em}
\subsection*{FastAPI}
FastAPI è un framework web moderno e ad alte prestazioni per la creazione di API con Python. Uno dei vantaggi di FastAPI è la generazione automatica della documentazione interattiva delle API. Inoltre, FastAPI sfrutta le annotazioni di tipo di Python per definire e validare automaticamente i dati in input e in output. Questo consente al team di sviluppo di mantenere un controllo rigoroso sui tipi di dati, riducendo il tasso di errori.

\vspace{2em}
\subsection*{Flask}
Flask è un microframework per lo sviluppo di applicazioni web in Python. È progettato per essere leggero e modulare, consentendo agli sviluppatori di scegliere e configurare solo i componenti necessari per il proprio progetto. Flask offre funzionalità di base come il routing delle URL e la gestione delle richieste, e può essere esteso con numerosi plugin per aggiungere funzionalità avanzate. La sua semplicità e flessibilità lo rendono una scelta popolare sia per piccoli progetti che per applicazioni web più complesse.


\vspace{2em}
\subsection*{Framework}
Un framework è una struttura di supporto utilizzata nello sviluppo software per fornire una base standardizzata su cui costruire e sviluppare applicazioni. Consiste in un insieme di librerie, strumenti e best practice che facilitano e accelerano il processo di sviluppo, promuovendo la coerenza e la riusabilità del codice. I framework possono essere specifici per determinati linguaggi di programmazione o tipi di applicazioni, come applicazioni web, applicazioni desktop o mobile.
