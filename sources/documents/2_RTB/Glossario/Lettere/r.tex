\section{R}

\vspace{2em}
\subsection*{Ramo base}
\par Il ramo base, anche noto come "branch principale" o "branch master" in alcuni contesti, è il ramo predefinito all'interno di un sistema di controllo versione come \glossario{Git}. È il ramo principale del progetto che rappresenta lo stato stabile e funzionante del codice sorgente. Nel ramo base vengono generalmente integrati i cambiamenti provenienti da altri rami di sviluppo dopo essere stati testati e approvati. Il ramo base è spesso utilizzato per rilasciare versioni stabili del software e viene mantenuto protetto da modifiche dirette non autorizzate. È considerato il punto di riferimento principale per lo sviluppo del progetto e viene utilizzato come base per la creazione di nuovi rami di sviluppo o funzionalità.

\vspace{2em}
\subsection*{React}
\par React è una libreria front-end e open-source utilizzata per lo sviluppo di interfacce utente web e native. Le interfacce vengono costruite a partire da singole porzioni di codice (modulari, riutilizzabili e scritte in JavaScript o TypeScript), chiamate componenti.

\vspace{2em}
\subsection*{Release}
\par Una release è una versione specifica di un software o di un prodotto che viene distribuita al pubblico o agli utenti finali. Rappresenta un punto di riferimento nello sviluppo del software in cui le nuove funzionalità sono implementate, i bug sono stati corretti e il codice è stato testato e valutato come stabile e pronto per l'uso. Le release vengono numerate o denominate in modo sequenziale (ad esempio, versione 1.0, versione 2.0, ecc.) e spesso vengono accompagnate da note di rilascio che descrivono le modifiche apportate, i requisiti di sistema e altre informazioni pertinenti. Le release sono importanti per il ciclo di vita del software e consentono agli utenti di accedere alle nuove funzionalità e alle correzioni di bug in modo organizzato e controllato.

\vspace{2em}
\subsection*{Report}
\par Un report è uno strumento informativo che può essere impiegato per fornire una panoramica generale sull’andamento di un progetto o per monitorare i risultati di analisi, ricerche e processi di elaborazione dei dati.

\vspace{2em}
\subsection*{Repository}
\par Un repository è un archivio o una collezione di file digitali o di dati, organizzati in modo strutturato e gestiti tramite un sistema di versionamento. È comunemente utilizzato nello sviluppo software per archiviare e gestire il codice sorgente, ma può anche contenere documentazione, risorse multimediali, file di configurazione e altro ancora. I repository forniscono un ambiente centralizzato per la collaborazione, il controllo delle versioni e la gestione dei cambiamenti, consentendo a più utenti di lavorare contemporaneamente sugli stessi file senza rischio di sovrascrittura o perdita di dati. Sistemi di controllo versione come Git sono ampiamente utilizzati per gestire i repository e tracciare le modifiche nel tempo.

\vspace{2em}
\subsection*{Requisito}
\par Un requisito è una specifica o una condizione che deve essere soddisfatta o posseduta da un sistema, un prodotto o un servizio per soddisfare determinati obiettivi o esigenze. I requisiti possono essere di diversi tipi, tra cui requisiti funzionali che descrivono le funzionalità che il sistema deve fornire, requisiti non funzionali che definiscono le qualità del sistema come prestazioni, sicurezza o usabilità, e requisiti di vincolo che rappresentano le limitazioni o le restrizioni sul sistema. La raccolta, l'analisi e la gestione dei requisiti sono fondamentali nel processo di sviluppo del software per garantire che il prodotto finale soddisfi le aspettative degli utenti e le esigenze del cliente.

\vspace{2em}
\subsection*{Retrospettiva}
\par La retrospettiva è una pratica utilizzata nei processi agili di sviluppo del software, come \glossario{Scrum} e \glossario{Agile}, per riflettere sulle attività svolte durante un periodo specifico di lavoro, identificare i punti di forza e di debolezza e individuare opportunità di miglioramento. Durante una retrospettiva, il team si riunisce per analizzare ciò che è stato fatto, discutere delle sfide affrontate e delle lezioni apprese, e collaborare per identificare azioni concrete da intraprendere per migliorare il processo di lavoro. Le retrospettive possono assumere varie forme, tra cui riunioni strutturate, sessioni di brainstorming o survey online, e vengono solitamente condotte alla fine di ciascuna iterazione o sprint di lavoro per garantire un miglioramento continuo e adattativo.

\vspace{2em}
\subsection*{Ricerca semantica}
\par La ricerca semantica è un approccio alla ricerca dell'informazione che mira a comprendere il significato del testo oltre alla corrispondenza delle parole chiave. Utilizza tecniche di analisi del linguaggio naturale (NLP) e di rappresentazione semantica per comprendere il contesto e le relazioni tra i concetti nei documenti. Questo approccio consente di ottenere risultati di ricerca più pertinenti e accurati, in quanto tiene conto del significato implicito e della rilevanza semantica, piuttosto che basarsi solo sulla presenza di parole chiave.

\vspace{2em}
\subsection*{RTB}
\par Requirements and Technology Baseline (RTB): Il termine "Requirements and Technology Baseline" si riferisce alla linea base di requisiti e tecnologie stabilite all'inizio di un progetto. Questa fase definisce i requisiti funzionali e non funzionali del sistema, insieme alle tecnologie chiave da impiegare per soddisfare tali requisiti. La RTB fornisce una struttura chiara per il design e lo sviluppo del prodotto, assicurando coerenza e allineamento con gli obiettivi aziendali e le esigenze degli utenti finali. È un pilastro fondamentale nel garantire che il prodotto finale sia robusto, scalabile e in grado di rispondere efficacemente alle sfide tecniche e alle aspettative degli stakeholder.
