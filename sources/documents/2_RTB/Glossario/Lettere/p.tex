\section{P}

\vspace{2em}
\subsection*{Parser}
\par Un parser è un componente software che analizza una sequenza di dati di input secondo una determinata grammatica o sintassi e produce una rappresentazione strutturata o un'altra forma di output. È comunemente utilizzato nel campo dell'informatica per analizzare e interpretare il codice sorgente di linguaggi di programmazione, il markup di documenti o altri formati di dati strutturati. Un parser suddivide il testo di input in token o simboli atomici e applica regole grammaticali per costruire una rappresentazione intermedia della struttura del testo.

\vspace{2em}
\subsection*{PB}
\par Vedi \glossario{Product Baseline}.

\vspace{2em}
\subsection*{PDCA}
\par PDCA (Plan-Do-Check-Act) è un ciclo di gestione iterativo utilizzato per il controllo continuo della qualità e il miglioramento dei processi. Consiste in quattro fasi: Pianificazione (Plan), dove vengono identificati problemi e obiettivi; Esecuzione (Do), in cui vengono implementate le soluzioni; Verifica (Check), dove vengono valutati i risultati rispetto agli obiettivi prefissati; e Azione (Act), dove vengono standardizzate le soluzioni efficaci o vengono intraprese ulteriori modifiche per migliorare il processo. Il ciclo PDCA promuove un approccio sistematico per il miglioramento continuo.

\vspace{2em}
\subsection*{Performance}
\par La performance nel contesto dello sviluppo software si riferisce all'efficienza con cui un sistema o un'applicazione esegue le sue funzioni. Gli aspetti chiave includono la velocità di esecuzione, il tempo di risposta, l'uso delle risorse (CPU, memoria, disco), la scalabilità e la capacità di gestire carichi di lavoro elevati. La performance è cruciale per garantire una buona esperienza utente e può essere ottimizzata attraverso varie tecniche di programmazione, testing e tuning delle risorse. Gli strumenti di monitoraggio e profiling sono utilizzati per identificare e risolvere i colli di bottiglia delle prestazioni.

\vspace{2em}
\subsection*{Persistenti}
\par Nel contesto del software, i dati persistenti sono quelli che continuano a esistere anche dopo la chiusura dell'applicazione che li ha creati. Questi dati vengono tipicamente memorizzati su supporti di archiviazione non volatili come dischi rigidi o SSD, utilizzando database, file di log o altri meccanismi di archiviazione. La persistenza è una caratteristica fondamentale per applicazioni che necessitano di conservare informazioni a lungo termine, garantendo che i dati possano essere recuperati e utilizzati in future sessioni di lavoro.

\vspace{2em}
\subsection*{PoC}
\par Vedi \glossario{Proof of Concept}.

\vspace{2em}
\subsection*{Processo}
\par In ambito informatico e gestionale, un processo rappresenta una serie organizzata di attività interconnesse finalizzate alla trasformazione di input in output. Questo può includere procedure standardizzate, strumenti e risorse per raggiungere obiettivi specifici di business o tecnologici. L'efficace gestione dei processi è cruciale per migliorare l'efficienza operativa, la qualità del prodotto e la soddisfazione del cliente, essendo fondamentale in settori come lo sviluppo software, la produzione e la gestione dei servizi.

\vspace{2em}
\subsection*{Prodotto software}
\par Un prodotto software è un'applicazione informatica o sistema sviluppato per soddisfare specifiche esigenze utente. Include il ciclo completo di sviluppo software, dalla progettazione alla realizzazione, testing e manutenzione. I prodotti software variano da applicazioni desktop a sistemi embedded e servizi cloud, rivestendo un ruolo cruciale in tutti gli aspetti della vita moderna, dall'automazione industriale alla gestione delle informazioni e al divertimento personale.

\vspace{2em}
\subsection*{Product backlog}
\par Il product backlog è un elenco ordinato delle attività che il team deve svolgere, non soltanto compiti legati alle funzionalità che il prodotto deve offrire, ma anche task di natura organizzativa, logistica e di redazione dei documenti. Il Product Backlog viene costantemente rivisto e riordinato in base alle richieste e alle aspettative del cliente. All’interno del framework Scrum, il product backlog è considerato un "artefatto".

\vspace{2em}
\subsection*{Product Baseline}
\par La Product Baseline è la seconda revisione dello stato di avanzamento del progetto. La presentazione della PB deve illustrare le scelte architetturali del team, in coerenza con la Technology Baseline, riportando i diagrammi delle classi e di sequenza, e i design pattern adottati.

\vspace{2em}
\subsection*{Progettazione}
\par La progettazione nel contesto informatico è il processo di definizione architetturale e dettaglio delle componenti di un sistema software. Include l'analisi dei requisiti, la creazione di modelli, l'implementazione di strutture dati e algoritmi, assicurando che il sistema soddisfi gli obiettivi funzionali e di performance prefissati. Una progettazione ben eseguita è essenziale per garantire la scalabilità, la manutenibilità e la sicurezza del software, contribuendo alla sua efficace implementazione e evoluzione nel tempo.

\vspace{2em}
\subsection*{Prompt Engineering}
\par Il Prompt Engineering è un approccio metodologico che si concentra sull'ottimizzazione e sulla progettazione dei prompt utilizzati nelle interfacce utente, nei sistemi di intelligenza artificiale e nelle applicazioni di dialogo uomo-macchina. Questo approccio considera i prompt come un elemento cruciale per guidare e influenzare il comportamento degli utenti o dei sistemi, e si propone di progettare prompt efficaci, chiari e coinvolgenti per massimizzare l'interazione e il coinvolgimento degli utenti. Il Prompt Engineering coinvolge la comprensione del contesto di utilizzo, la progettazione linguistica, l'analisi del feedback degli utenti e l'ottimizzazione continua dei prompt per migliorare l'esperienza complessiva degli utenti.

\vspace{2em}
\subsection*{Proof of Concept}
\par Il PoC (Proof of Concept) è l’allestimento di una demo prototipale, ovvero la dimostrazione pratica dei funzionamenti di base di un applicativo software, finalizzata a dimostrare la fattibilità dei requisiti e la validità delle tecnologie individuate.

\vspace{2em}
\subsection*{Proponente}
\par La proponente è la parte interessata che propone un progetto, un'idea o un'iniziativa. Nel contesto dell'ingegneria del software, la proponente è colui che avvia il processo di sviluppo di un sistema software, solitamente identificando la necessità o l'opportunità di un nuovo prodotto o servizio. La proponente può essere un individuo, un'organizzazione o un'azienda che ha un interesse diretto nel successo del progetto. Per ChatSQL la Proponente è stata l'azienda Zucchetti s.p.a.

\vspace{2em}
\subsection*{Pull requests}
\par Le pull requests sono una caratteristica dei sistemi di controllo versione distribuiti come \glossario{Git}, utilizzate per richiedere l'integrazione di modifiche o aggiunte da parte di un collaboratore in un \glossario{repository} principale. Una pull request rappresenta una richiesta di revisione del codice, in cui il collaboratore propone le sue modifiche e chiede ai membri del team o ai revisori di esaminare, discutere e approvare le modifiche prima di essere integrate nel repository principale. Le pull requests sono spesso accompagnate da commenti, descrizioni e discussioni che facilitano la collaborazione e la comunicazione tra i membri del team. Dopo aver ricevuto il consenso e l'approvazione, le modifiche proposte nella pull request vengono fuse (merged) nel repository principale, aggiornando così il codice sorgente con le nuove modifiche.

\vspace{2em}
\subsection*{Python}
\par Python è un linguaggio di programmazione ad alto livello, interpretato e orientato agli oggetti, noto per la sua sintassi chiara e leggibile. È versatile e può essere utilizzato in una vasta gamma di applicazioni, tra cui sviluppo web, analisi dati, automazione di compiti, intelligenza artificiale e molto altro. Python è apprezzato per la sua facilità d'uso, la vasta libreria standard e la ricca comunità di sviluppatori che contribuiscono a un'ampia gamma di librerie e framework.
