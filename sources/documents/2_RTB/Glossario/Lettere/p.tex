\section{P}


\vspace{2em}
\subsection*{Parser}
Un parser è un componente software che analizza una sequenza di dati di input secondo una determinata grammatica o sintassi e produce una rappresentazione strutturata o un'altra forma di output. È comunemente utilizzato nel campo dell'informatica per analizzare e interpretare il codice sorgente di linguaggi di programmazione, il markup di documenti o altri formati di dati strutturati. Un parser suddivide il testo di input in token o simboli atomici e applica regole grammaticali per costruire una rappresentazione intermedia della struttura del testo. 

\vspace{2em}
\subsection*{Prompt}
TODO

\vspace{2em}
\subsection*{Prompt Engineering}
Il Prompt Engineering è un approccio metodologico che si concentra sull'ottimizzazione e sulla progettazione dei prompt utilizzati nelle interfacce utente, nei sistemi di intelligenza artificiale e nelle applicazioni di dialogo uomo-macchina. Questo approccio considera i prompt come un elemento cruciale per guidare e influenzare il comportamento degli utenti o dei sistemi, e si propone di progettare prompt efficaci, chiari e coinvolgenti per massimizzare l'interazione e il coinvolgimento degli utenti. Il Prompt Engineering coinvolge la comprensione del contesto di utilizzo, la progettazione linguistica, l'analisi del feedback degli utenti e l'ottimizzazione continua dei prompt per migliorare l'esperienza complessiva degli utenti.

\vspace{2em}
\subsection*{Proponente}
La proponente è la parte interessata che propone un progetto, un'idea o un'iniziativa. Nel contesto dell'ingegneria del software, la proponente è colui che avvia il processo di sviluppo di un sistema software, solitamente identificando la necessità o l'opportunità di un nuovo prodotto o servizio. La proponente può essere un individuo, un'organizzazione o un'azienda che ha un interesse diretto nel successo del progetto. Per ChatSQL la Proponente è stata l'azienda Zucchetti s.p.a.  

\vspace{2em}
\subsection*{Pull requests}
Le pull requests sono una caratteristica dei sistemi di controllo versione distribuiti come \glossario{Git}, utilizzate per richiedere l'integrazione di modifiche o aggiunte da parte di un collaboratore in un \glossario{repository} principale. Una pull request rappresenta una richiesta di revisione del codice, in cui il collaboratore propone le sue modifiche e chiede ai membri del team o ai revisori di esaminare, discutere e approvare le modifiche prima di essere integrate nel repository principale. Le pull requests sono spesso accompagnate da commenti, descrizioni e discussioni che facilitano la collaborazione e la comunicazione tra i membri del team. Dopo aver ricevuto il consenso e l'approvazione, le modifiche proposte nella pull request vengono fuse (merged) nel repository principale, aggiornando così il codice sorgente con le nuove modifiche.

\vspace{2em}
\subsection*{Python}
Python è un linguaggio di programmazione ad alto livello, interpretato e orientato agli oggetti, noto per la sua sintassi chiara e leggibile. È versatile e può essere utilizzato in una vasta gamma di applicazioni, tra cui sviluppo web, analisi dati, automazione di compiti, intelligenza artificiale e molto altro. Python è apprezzato per la sua facilità d'uso, la vasta libreria standard e la ricca comunità di sviluppatori che contribuiscono a un'ampia gamma di librerie e framework.
