\section{J}

\vspace{2em}
\subsection*{Jira}
\par Jira è una piattaforma di gestione dei progetti sviluppata da Atlassian, utilizzata per tracciare le attività, gestire i progetti e coordinare il lavoro collaborativo. È ampiamente utilizzato nelle industrie del software, ma può essere adattato a una varietà di settori e casi d'uso. Jira offre funzionalità per la creazione e l'assegnazione di attività, il monitoraggio dello stato dei progetti, la gestione delle risorse e la collaborazione tra i membri del team.

\vspace{2em}
\subsection*{Jira Software}
\par Jira Software è una versione specifica di Jira progettata per i programmatori e i team di sviluppo agile. Include funzionalità per la pianificazione e la gestione dei progetti software, come il backlog degli sprint, le board Kanban e Scrum, e il tracciamento dei task. Jira Software è integrato con altri strumenti di sviluppo software come \glossario{GitHub} per una gestione completa del ciclo di sviluppo.

\vspace{2em}
\subsection*{Jira Work Management}
\par Jira Work Management è uno strumento progettato per l'utilizzo in contesti non strettamente associati allo sviluppo software. Offre funzionalità per la gestione dei progetti, la pianificazione delle attività, la collaborazione e la comunicazione tra i membri del team.

\vspace{2em}
\subsection*{JSON}
\par JSON (JavaScript Object Notation) è un formato di interscambio dati leggero e strutturato, utilizzato principalmente per trasmettere dati tra un server e un client web. È basato su un sottoinsieme del linguaggio di programmazione JavaScript e rappresenta i dati come coppie chiave-valore e array. JSON non dipende dal linguaggio di programmazione, il che lo rende ideale per l'integrazione di sistemi eterogenei. La sua sintassi e la semplicità lo rendono una scelta popolare per le API web e la configurazione dei file.

\vspace{2em}
\subsection*{JSX}
\par JSX (JavaScript XML) è una sintassi di estensione per JavaScript spesso affiancata alla libreria React per la creazione di interfacce utente. Consente agli sviluppatori di scrivere codice simile all'HTML all'interno di file JavaScript, agevolando la creazione di componenti React. JSX rende il codice più leggibile e facile da comprendere, definendo l'aspetto e il comportamento dei componenti in modo combinato.
