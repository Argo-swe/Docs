\section{J}

\vspace{2em}
\subsection*{Jira}
Jira è una piattaforma di gestione dei progetti sviluppata da Atlassian, utilizzata per tracciare le attività, gestire i progetti e coordinare il lavoro di squadra. È ampiamente utilizzato nelle industrie del software, ma può essere adattato per una varietà di settori e casi d'uso. Jira offre funzionalità per la creazione e l'assegnazione di attività, il monitoraggio dello stato dei progetti, la gestione delle risorse e la collaborazione tra i membri del team.

\vspace{2em}
\subsection*{Jira Software}
Jira Software è una versione specifica di \glossario{Jira} progettata per gli sviluppatori di software e le squadre di sviluppo agile. Include funzionalità specifiche per la pianificazione e la gestione dei progetti software, come il backlog degli sprint, le board Kanban e Scrum, e la tracciabilità delle issue. Jira Software è integrato con altri strumenti di sviluppo software come \glossario{GitHub} per una gestione completa del ciclo di sviluppo.

\vspace{2em}
\subsection*{Jira Work Management}
Jira Work Management è una versione di Jira progettata per team non tecnici o per l'uso in contesti non legati allo sviluppo software. Offre funzionalità per la gestione dei progetti, la pianificazione delle attività, la collaborazione e la comunicazione tra i membri del team.

\vspace{2em}
\subsection*{JSON}
JSON (JavaScript Object Notation) è un formato di interscambio dati leggero e facile da leggere, utilizzato principalmente per trasmettere dati tra un server e un client web. È basato su un sottoinsieme del linguaggio di programmazione JavaScript e rappresenta i dati come coppie chiave-valore e array. JSON è indipendente dalla lingua di programmazione, il che lo rende ideale per l'integrazione di sistemi eterogenei. La sua sintassi semplice e la leggerezza lo rendono una scelta popolare per le API web e la configurazione dei file.

\vspace{2em}
\subsection*{JSX}
JSX (JavaScript XML) è una sintassi di estensione per JavaScript utilizzata principalmente con la libreria React per la creazione di interfacce utente. Consente agli sviluppatori di scrivere strutture simili a HTML all'interno del codice JavaScript, facilitando la creazione di componenti React in modo dichiarativo. JSX rende il codice più leggibile e facile da comprendere, permettendo di definire l'aspetto e il comportamento dei componenti in un unico luogo. Durante la fase di build, il codice JSX viene trasformato in chiamate a funzioni JavaScript standard.
