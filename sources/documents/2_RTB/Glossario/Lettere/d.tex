\section{D}

\vspace{2em}
\subsection*{Database}
Un database è una raccolta organizzata di dati strutturati, che possono essere memorizzati e accessibili elettronicamente da un sistema informatico. I database sono progettati per consentire una gestione efficiente dei dati, facilitando operazioni di inserimento, aggiornamento, eliminazione e interrogazione. Utilizzano sistemi di gestione di database (DBMS) per garantire l'integrità, la sicurezza e la disponibilità dei dati. I database possono essere di diversi tipi, inclusi relazionali, NoSQL, gerarchici e a grafo, ognuno adatto a specifiche esigenze applicative.
TODO

\vspace{2em}
\subsection*{Dizionario dati}
Un dizionario dati è una raccolta di metadati che descrive la struttura, le caratteristiche e le relazioni dei dati all'interno di un database o di un sistema informativo. Include informazioni come i nomi dei campi, i tipi di dati, le dimensioni, le restrizioni e le relazioni tra tabelle. Il dizionario dati funge da riferimento per sviluppatori e amministratori di database, facilitando la comprensione e la gestione dei dati, assicurando coerenza e integrità nel sistema.
TODO