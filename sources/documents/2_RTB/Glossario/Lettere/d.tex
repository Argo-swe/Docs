\section{D}

\vspace{2em}
\subsection*{Database}
Un database è una raccolta organizzata di dati strutturati, che possono essere memorizzati e accessibili elettronicamente da un sistema informatico. I database sono progettati per consentire una gestione efficiente dei dati, facilitando operazioni di inserimento, aggiornamento, eliminazione e interrogazione. Utilizzano sistemi di gestione di database (DBMS) per garantire l'integrità, la sicurezza e la disponibilità dei dati.

\vspace{2em}
\subsection*{Debug}
Il debug è il processo di individuazione, analisi e risoluzione dei problemi o dei bug in un software o in un sistema informatico. Consiste nell'identificare le cause dei malfunzionamenti, tracciare i problemi e correggere gli errori al fine di ripristinare il corretto funzionamento del sistema. 

\vspace{2em}
\subsection*{Debugging}
\par Vedi \glossario{Debug}
%Il debugging è il processo di ricerca e risoluzione dei bug o dei malfunzionamenti in un software o in un sistema informatico. Coinvolge l'analisi dei sintomi dei problemi, la ricerca delle cause sottostanti e l'implementazione di soluzioni per correggere gli errori. Il debugging può richiedere l'uso di strumenti specializzati, la revisione del codice sorgente e il testing approfondito per garantire che il software funzioni correttamente senza errori.

\vspace{2em}
\subsection*{Dizionario dati}
Un dizionario dati è una raccolta di metadati che descrive la struttura, le caratteristiche e le relazioni dei dati all'interno di un \glossario{database}. Include informazioni come: i nomi delle tabelle e le loro descrizioni, i nomi dei campi e le loro descrizioni, i tipi di dati, le dimensioni e le relazioni tra tabelle. Questo strumento in ChatSQL serve agli utenti per comprendere quali tipi di richieste possono essere fatte al sistema e ai tecnici per comprendere il debug del sistema.

\vspace{2em}
\subsection*{Diagramma di Gantt}
Il diagramma di Gantt è uno strumento di visualizzazione utilizzato nella gestione dei progetti per rappresentare la pianificazione delle attività nel tempo. È composto da una serie di barre orizzontali che rappresentano le attività del progetto e la loro durata prevista, disposte lungo un asse temporale. Il diagramma di Gantt consente di visualizzare facilmente la sequenza delle attività, le dipendenze tra di esse e lo stato di avanzamento del progetto nel tempo.

\vspace{2em}
\subsection*{Docker}
Docker è una piattaforma open-source per la creazione, la distribuzione e l'esecuzione di applicazioni in contenitori leggeri e portabili. I contenitori Docker consentono agli sviluppatori di eseguire le proprie applicazioni e le relative dipendenze in un ambiente isolato, garantendo che esse funzionino in modo consistente su qualsiasi sistema operativo o infrastruttura. Docker facilita la distribuzione delle applicazioni, permettendo agli sviluppatori di racchiudere il proprio codice insieme a tutte le librerie e le dipendenze necessarie, eliminando così i problemi di compatibilità tra ambienti di sviluppo e produzione.

\vspace{2em}
\subsection*{DTO}
DTO è l'acronimo di Data Transfer Object, un design pattern utilizzato per trasferire dati tra sottosistemi di un'applicazione software. I DTO sono oggetti che non dovrebbero contenere alcuna logica di business, in quanto si occupano di archiviare, recuperare, serializzare e deserializzare i dati in rete.
