\section{D}

\vspace{2em}
\subsection*{Database}
Un database è una raccolta organizzata di dati strutturati, che possono essere memorizzati e accessibili elettronicamente da un sistema informatico. I database sono progettati per consentire una gestione efficiente dei dati, facilitando operazioni di inserimento, aggiornamento, eliminazione e interrogazione. Utilizzano sistemi di gestione di database (DBMS) per garantire l'integrità, la sicurezza e la disponibilità dei dati. I database possono essere di diversi tipi, inclusi relazionali, NoSQL, gerarchici e a grafo, ognuno adatto a specifiche esigenze applicative.

\vspace{2em}
\subsection*{Debug}
Il debug è il processo di individuazione, analisi e risoluzione dei problemi o dei bug in un software o in un sistema informatico. Consiste nell'identificare le cause dei malfunzionamenti, tracciare i problemi e correggere gli errori al fine di ripristinare il corretto funzionamento del sistema. Il debug può coinvolgere l'uso di strumenti di monitoraggio, log, testing e analisi del codice per individuare e risolvere le anomalie nel software.

\vspace{2em}
\subsection*{Debugging}
Il debugging è il processo di ricerca e risoluzione dei bug o dei malfunzionamenti in un software o in un sistema informatico. Coinvolge l'analisi dei sintomi dei problemi, la ricerca delle cause sottostanti e l'implementazione di soluzioni per correggere gli errori. Il debugging può richiedere l'uso di strumenti specializzati, la revisione del codice sorgente e il testing approfondito per garantire che il software funzioni correttamente senza errori.

\vspace{2em}
\subsection*{Dizionario dati}
Un dizionario dati è una raccolta di metadati che descrive la struttura, le caratteristiche e le relazioni dei dati all'interno di un database o di un sistema informativo. Include informazioni come i nomi dei campi, i tipi di dati, le dimensioni, le restrizioni e le relazioni tra tabelle. Il dizionario dati funge da riferimento per sviluppatori e amministratori di database, facilitando la comprensione e la gestione dei dati, assicurando coerenza e integrità nel sistema.

\vspace{2em}
\subsection*{Diagramma di Gantt}
Il diagramma di Gantt è uno strumento di visualizzazione utilizzato nella gestione dei progetti per rappresentare la pianificazione delle attività nel tempo. È composto da una serie di barre orizzontali che rappresentano le attività del progetto e la loro durata prevista, disposte lungo un asse temporale. Il diagramma di Gantt consente di visualizzare facilmente la sequenza delle attività, le dipendenze tra di esse e lo stato di avanzamento del progetto nel tempo. È ampiamente utilizzato per pianificare, monitorare e comunicare lo svolgimento dei progetti in vari settori, inclusi l'ingegneria, la costruzione, lo sviluppo del software e altro ancora.

\vspace{2em}
\subsection*{Docker}
Docker è una piattaforma open-source per la creazione, la distribuzione e l'esecuzione di applicazioni in contenitori leggeri e portabili. I contenitori Docker consentono agli sviluppatori di isolare le proprie applicazioni e le relative dipendenze in un ambiente autonomo, garantendo che esse funzionino in modo consistente su qualsiasi sistema operativo o infrastruttura. Docker facilita la distribuzione delle applicazioni, permettendo agli sviluppatori di impacchettare il proprio codice insieme a tutte le librerie e le dipendenze necessarie, eliminando così i problemi di compatibilità tra ambienti di sviluppo e produzione. Docker è ampiamente utilizzato nello sviluppo del software moderno, nell'implementazione di microservizi, nel deployment di applicazioni cloud e in molte altre aree.