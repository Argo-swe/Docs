\section{L}

\vspace{2em}
\subsection*{LaTeX}
\par LaTeX è un sistema di preparazione di testi basato su markup, utilizzato per la produzione di documenti di alta qualità, in particolare nel campo accademico, scientifico e tecnico. LaTeX si basa su un insieme di macro e comandi che consentono agli autori di concentrarsi sul contenuto del documento senza preoccuparsi della sua formattazione, assicurando uno stile coerente e professionale.

\vspace{2em}
\subsection*{Large Language Model}
\par Large Language Model (abbreviato LLM) è un tipo di modello di intelligenza artificiale progettato per comprendere e generare il linguaggio naturale in maniera avanzata. Questi modelli sono addestrati su grandi quantità di dati e sono in grado di eseguire una varietà di compiti, tra cui generazione di testo, traduzione automatica, risposta alle domande e molto altro. Gli LLM hanno dimostrato risultati sorprendenti in molte applicazioni, ma richiedono risorse computazionali significative per l'addestramento e l'esecuzione.

\vspace{2em}
\subsection*{LLM}
\par Vedi \glossario{Large Language Model}.

\vspace{2em}
\subsection*{LM Studio}
\par LM Studio è un'applicazione desktop che semplifica il processo di testing di modelli linguistici di grandi dimensioni (LLM) locali e open source. La piattaforma mette a disposizione un server API, in esecuzione su localhost, tramite il quale interrogare i modelli disponibili. Le richieste e le risposte seguono il formato API di OpenAI.

\vspace{2em}
\subsection*{Log}
\par Un log, o registro, è un file o un insieme di dati utilizzato per registrare eventi, azioni o informazioni rilevanti in un sistema o applicazione software. I log vengono utilizzati per scopi di monitoraggio, diagnosi, audit e tracciamento delle attività. Possono contenere informazioni come errori, avvisi, operazioni eseguite e altro ancora. I log sono spesso utilizzati nel contesto dello sviluppo software per comprendere il comportamento del sistema, individuare e risolvere i problemi, tracciare le modifiche e garantire la conformità agli standard e alle normative.
