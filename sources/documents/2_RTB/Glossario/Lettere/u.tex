\section{U}

\vspace{2em}
\subsection*{UML}
\par In ingegneria del software, UML (Unified Modeling Language) è un linguaggio di modellazione che consente di rappresentare un sistema secondo tre aspetti principali, per ciascuno dei quali sono disponibili diagrammi specifici:
\begin{itemize}
  \item Modello funzionale: descrive il sistema dal punto di vista dell'utente; corrisponde all'analisi dei requisiti e utilizza i diagrammi dei casi d'uso;
  \item Modello a oggetti: rappresenta la struttura del sistema sulla base del paradigma orientato agli oggetti;
  \item Modello dinamico: descrive il comportamento degli oggetti del sistema.
\end{itemize}

\vspace{2em}
\subsection*{Use case}
\par Vedi \glossario{Casi d'uso}.
