\section{I}

\vspace{2em}
\subsection*{IA}
\par Vedi \glossario{Intelligenza Artificiale}.

\vspace{2em}
\subsection*{IEC}
\par L'IEC (International Electrotechnical Commission) è un'organizzazione internazionale che sviluppa standard per il settore dell'elettronica, dell'elettricità e delle tecnologie correlate. I suoi standard influenzano direttamente il settore del software attraverso normative che regolano la sicurezza, l'affidabilità e le prestazioni dei sistemi informatici e dei dispositivi elettronici.
Ad esempio, l'IEC 61508 definisce i requisiti per i sistemi di sicurezza nel controllo dei processi industriali, inclusi i software embedded utilizzati nei dispositivi di automazione. Questo standard assicura che il software risponda a rigorosi criteri di sicurezza e affidabilità per prevenire incidenti e guasti critici.
Un altro esempio è l'IEC 62304, che stabilisce i requisiti per il ciclo di vita del software per dispositivi medici. Questo standard regola lo sviluppo, la validazione e la manutenzione del software medico per garantire la sicurezza e l'efficacia dei dispositivi utilizzati in ambito sanitario.
Grazie agli standard IEC, le aziende nel settore dello sviluppo software possono migliorare la qualità dei loro prodotti, rispettare normative internazionali e aumentare la fiducia dei clienti nella sicurezza e nella performance dei loro sistemi informatici ed elettronici.

\vspace{2em}
\subsection*{Indice}
\par Vedi \glossario{Indice vettoriale}.

\vspace{2em}
\subsection*{Indice vettoriale}
\par Un indice vettoriale è una struttura dati utilizzata per archiviare e recuperare in modo efficiente dati vettoriali, consentendo ricerche rapide di similarità.

\vspace{2em}
\subsection*{Indicizzazione}
\par L'indicizzazione è una componente chiave dei database vettoriali. Contribuisce ad aumentare la velocità del processo di ricerca per similarità con un compromesso minimo in termini di precisione della ricerca.

\vspace{2em}
\subsection*{Information Technology}
\par Con il termine Information Technology (IT) si fa riferimento a un insieme di ambiti correlati che comprendono sistemi informatici, software, linguaggi di programmazione, tecnologie per la trasmissione, elaborazione e archiviazione di dati.

\vspace{2em}
\subsection*{Integrazione continua}
\par Vedi \glossario{Continuous Integration}.

\vspace{2em}
\subsection*{Intelligenza Artificiale}
\par L'intelligenza artificiale (IA) è un campo di ricerca che studia e sviluppa metodi per consentire a macchine e sistemi informatici di svolgere compiti che normalmente sarebbero appannaggio dell'intelligenza umana. Questi compiti includono l'apprendimento, la risoluzione di problemi, la comprensione del linguaggio naturale e la capacità di adattarsi al contesto.

\vspace{2em}
\subsection*{ISO}
\par ISO (International Organization for Standardization) è un'organizzazione indipendente che sviluppa e pubblica standard internazionali in vari settori, tra cui tecnologia, sicurezza, ambiente e gestione della qualità. I suoi standard forniscono linee guida e specifiche tecniche per garantire la qualità, l'affidabilità e l'efficienza dei prodotti, servizi e processi in tutto il mondo. ISO 9001, ad esempio, stabilisce i requisiti per i sistemi di gestione della qualità, mentre ISO 27001 riguarda la gestione della sicurezza delle informazioni. L'adozione degli standard ISO aiuta le organizzazioni a ridurre i rischi e soddisfare le aspettative dei clienti.

\vspace{2em}
\subsection*{Issue Tracking System}
\par Un sistema di tracciamento delle attività è un'applicazione software utilizzata per registrare, monitorare e gestire i task durante lo sviluppo software. Fornisce strumenti per la creazione di issue, l'assegnazione delle attività, il tracciamento dello stato e la comunicazione. Gli issue tracking system aiutano a mantenere traccia delle richieste di funzionalità, dei bug e di altri compiti, consentendo ai team di lavorare in modo collaborativo e organizzato.

\vspace{2em}
\subsection*{ITS}
\par Vedi \glossario{Issue Tracking System}.
