\section{I}

\vspace{2em}
\subsection*{IA}
\par Vedi \glossario{Intelligenza Artificiale}.

\vspace{2em}
\subsection*{IEC}
International Electrotechnical Commission (IEC): Organizzazione internazionale di normazione che prepara e pubblica norme internazionali per tutte le tecnologie elettriche, elettroniche e correlate.
TODO

\vspace{2em}
\subsection*{Indice}
\par Vedi \glossario{Indice vettoriale}.

\vspace{2em}
\subsection*{Indice vettoriale}
\par Un indice vettoriale è una struttura dati utilizzata per archiviare e recuperare in modo efficiente dati vettoriali, consentendo ricerche rapide di similarità.

\vspace{2em}
\subsection*{Indicizzazione}
L'indicizzazione è una componente chiave dei database vettoriali. Contribuisce ad aumentare la velocità del processo di ricerca per similarità con un compromesso minimo in termini di precisione della ricerca.

\vspace{2em}
\subsection*{Information Technology}
\par Con il termine Information Technology (IT) si fa riferimento a un insieme di ambiti correlati che comprendono sistemi informatici, software, linguaggi di programmazione, tecnologie per la trasmissione, elaborazione e archiviazione di dati.

\vspace{2em}
\subsection*{Integrazione continua}
\par Vedi \glossario{Continuous Integration}.

\vspace{2em}
\subsection*{Intelligenza Artificiale}
\par L'intelligenza artificiale (IA) è un campo di ricerca che studia e sviluppa metodi per consentire a macchine e sistemi informatici di svolgere compiti che normalmente sarebbero appannaggio dell'intelligenza umana. Questi compiti includono l'apprendimento, la risoluzione di problemi, la comprensione del linguaggio naturale e la capacità di adattarsi al contesto.

\vspace{2em}
\subsection*{ISO}
\par ISO, acronimo di International Organization for Standardization, è un'organizzazione internazionale che promuove lo sviluppo e la diffusione di standard volontari per varie industrie e settori, compreso l'ingegneria, la tecnologia dell'informazione, la sanità, l'agricoltura e altro ancora. Gli standard ISO sono sviluppati da gruppi di lavoro tecnici composti da esperti del settore provenienti da tutto il mondo e sono destinati a migliorare la qualità, l'efficienza e la sicurezza dei prodotti, dei servizi e dei processi. Gli standard ISO sono riconosciuti a livello internazionale e vengono utilizzati come linee guida per la conformità, l'interoperabilità e la competitività sul mercato globale.
TODO

\vspace{2em}
\subsection*{Issue Tracking System}
\par Un sistema di tracciamento delle issue è un'applicazione software utilizzata per registrare, monitorare e gestire le issue o i problemi durante lo sviluppo di software o altri progetti. Fornisce strumenti per la creazione di nuove issue, l'assegnazione a membri del team, il tracciamento dello stato e la comunicazione sulle soluzioni proposte e implementate. Gli issue tracking system aiutano a mantenere traccia delle richieste di funzionalità, dei bug e di altri compiti, consentendo ai team di lavorare in modo collaborativo e organizzato.

\vspace{2em}
\subsection*{ITS}
\par Vedi \glossario{Issue Tracking System}.
