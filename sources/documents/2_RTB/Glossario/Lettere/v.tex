\section{V}

\vspace{2em}
\subsection*{Validazione}
La validazione è il processo di verifica e conferma che un sistema, un prodotto o un processo soddisfi determinati requisiti specificati o standard. Può riguardare la conformità alle specifiche tecniche, la correttezza dei dati, la sicurezza, l'usabilità o altri criteri di qualità. La validazione può essere eseguita mediante test, ispezioni, revisioni o altre attività di controllo per garantire che il risultato finale sia affidabile e conforme alle aspettative.

\vspace{2em}
\subsection*{Verifica}
La verifica è il processo di valutazione e conferma che un sistema, un prodotto o un processo rispetti determinati requisiti o standard specificati. Si concentra sull'esame e sulla verifica delle caratteristiche, delle funzionalità o delle prestazioni per garantire la correttezza e l'adeguatezza del risultato finale. La verifica può coinvolgere attività come l'analisi dei requisiti, il testing del software, l'ispezione del codice o altre attività di revisione per assicurare la qualità e la conformità del prodotto o del processo.

\vspace{2em}
\subsection*{Versionamento}
Il versionamento è il processo di gestione e tracciamento delle varie versioni di un determinato documento, file o sistema software nel corso del tempo. Consiste nel mantenere un registro delle modifiche apportate al documento o al codice sorgente, indicando chi ha apportato la modifica, quando è stata effettuata e quale sia la natura della modifica stessa. Il versionamento è essenziale per tenere traccia delle modifiche, garantire la collaborazione tra più autori e ripristinare versioni precedenti in caso di necessità. I sistemi di controllo versione come \glossario{Git} sono ampiamente utilizzati per implementare il versionamento nel contesto dello sviluppo del software.

\vspace{2em}
\subsection*{Vettore}
TODO

\vspace{2em}
\subsection*{Vue.js}
Vue.js è un framework JavaScript progressivo utilizzato per la creazione di interfacce utente reactive e dinamiche. È progettato per essere incrementale e può essere integrato gradualmente in progetti esistenti senza richiedere una riscrittura completa del codice. Vue.js offre funzionalità per la creazione di componenti riutilizzabili, la gestione dello stato dell'applicazione e il reattivo aggiornamento dell'interfaccia utente in risposta ai cambiamenti dei dati.
