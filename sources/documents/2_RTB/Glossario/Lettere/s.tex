\section{S}

\vspace{2em}
\subsection*{Scrum}
Scrum è un framework di sviluppo agile utilizzato per gestire progetti complessi, in particolare nel contesto dello sviluppo software. Si basa su principi di trasparenza, ispezione e adattamento, e promuove il lavoro collaborativo, l'autorganizzazione del team e la consegna incrementale di prodotti di valore. In Scrum, il lavoro è organizzato in cicli di sviluppo chiamati sprint, durante i quali il team si impegna a consegnare un insieme di funzionalità completate e testate.

\vspace{2em}
\subsection*{Scrum Meeting}
Il framework Scrum prevede quattro occasioni formali, o meeting, per valutare lo stato di avanzamento delle iterazioni: sprint planning, daily scrum, sprint review e sprint retrospective.

\vspace{2em}
\subsection*{Sentence Similarity}
La Sentence Similarity, o similarità tra frasi, è una misura che quantifica il grado di somiglianza tra due frasi o segmenti di testo. Questo concetto è ampiamente utilizzato nell'ambito del Natural Language Processing (NLP) per valutare quanto due frasi siano semanticamente simili o correlate. Le tecniche per calcolare la similarità tra frasi possono variare a seconda del contesto e degli obiettivi specifici, ma spesso coinvolgono l'uso di modelli di \glossario{embedding} del linguaggio naturale per rappresentare le frasi in spazi vettoriali e misurare la distanza o la similarità tra i loro vettori rappresentativi.

\vspace{2em}
\subsection*{Sentence Transformers}
Sentence Transformers sono modelli di elaborazione del linguaggio naturale (NLP) progettati per codificare frasi e paragrafi in spazi vettoriali continui, rappresentando il significato semantico delle frasi. Utilizzano tecniche di apprendimento automatico basate su trasformatori per catturare le relazioni semantiche tra le parole e le frasi. I Sentence Transformers sono ampiamente utilizzati per compiti come la ricerca semantica, il clustering di testo, il riassunto automatico e l'analisi della similarità del testo.

\vspace{2em}
\subsection*{SPICE}
TODO

\vspace{2em}
\subsection*{Spreadsheet}
Uno spreadsheet, o foglio di calcolo, è un'applicazione software utilizzata per organizzare, elaborare e analizzare dati in forma tabellare. Lo spreadsheet è composto da una griglia di celle organizzate in righe e colonne, in cui è possibile inserire numeri, testo e formule matematiche per eseguire calcoli e analisi dei dati. Le celle dello spreadsheet possono essere formattate e personalizzate in base alle esigenze dell'utente, e possono essere utilizzate per creare grafici, report e visualizzazioni dei dati.

\vspace{2em}
\subsection*{Sprint}
Lo Sprint è un periodo di tempo definito e limitato all'interno di uno sviluppo \glossario{Agile} durante il quale viene svolto un lavoro specifico e concreto. Solitamente ha una durata fissa, tipicamente da una a quattro settimane, durante le quali il team di sviluppo si impegna a completare un insieme di attività pianificate. Gli Sprint sono caratterizzati da obiettivi chiari e misurabili e terminano con la consegna di un incremento di lavoro funzionante. Durante uno Sprint, il lavoro viene suddiviso in attività più piccole, chiamate backlog items, e il progresso viene monitorato attraverso riunioni quotidiane chiamate Daily Scrum o Stand-up.

\vspace{2em}
\subsection*{Sprint Backlog}
TODO

\vspace{2em}
\subsection*{SQL}
TODO

\vspace{2em}
\subsection*{SQLite}
SQLite è una libreria di gestione del database relazionale incorporata nella maggior parte dei linguaggi di programmazione. È un database leggero e autosufficiente che non richiede un processo separato per funzionare ed è progettato per essere incorporato direttamente nelle applicazioni. SQLite supporta la maggior parte delle funzionalità dei database SQL standard, tra cui la creazione di tabelle, l'inserimento, l'aggiornamento e la query dei dati, nonché l'implementazione di vincoli di integrità dei dati. È ampiamente utilizzato in applicazioni mobile, desktop e web per gestire dati locali o di piccole dimensioni.

\vspace{2em}
\subsection*{Staging Area}
TODO

\vspace{2em}
\subsection*{Stakeholders}
Gli stakeholders, o portatori di interesse, sono individui, gruppi o organizzazioni che hanno un interesse diretto o indiretto in un progetto, un'azienda o un'iniziativa e possono essere influenzati o influenzare il suo successo o fallimento. Gli stakeholders possono includere clienti, utenti finali, investitori, dipendenti, partner commerciali, organizzazioni governative, gruppi di interesse pubblico e altro ancora. È importante coinvolgere gli stakeholders nel processo decisionale e nella pianificazione del progetto per garantire che le loro esigenze, aspettative e preoccupazioni siano prese in considerazione e gestite in modo appropriato.

\vspace{2em}
\subsection*{Story}
TODO

\vspace{2em}
\subsection*{Streamlit}
Streamlit è un framework open-source per la creazione rapida di applicazioni web interattive in Python. Si concentra sulla semplicità e sulla facilità d'uso, consentendo agli sviluppatori di creare e distribuire applicazioni web complesse utilizzando solo poche righe di codice Python. Streamlit offre componenti predefiniti per la visualizzazione dei dati, la gestione degli input utente e la creazione di interfacce utente intuitive, consentendo agli sviluppatori di concentrarsi sullo sviluppo della logica dell'applicazione senza doversi preoccupare della complessità dell'infrastruttura web.

\vspace{2em}
\subsection*{Sviluppatore}
TODO
