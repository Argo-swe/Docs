\section{E}

\vspace{2em}
\subsection*{Embedding}
Nel contesto dell'elaborazione del linguaggio naturale (NLP), l'embedding si riferisce alla rappresentazione numerica di parole o frasi in uno spazio vettoriale continuo. Gli embedding sono creati utilizzando modelli di apprendimento automatico che assegnano a ciascuna parola o frase un vettore numerico denso, in modo che parole semanticamente simili siano rappresentate da vettori vicini nello spazio vettoriale. Questa rappresentazione vettoriale consente ai modelli di NLP di comprendere il significato e la relazione tra le parole in modo più efficace e di eseguire compiti come il raggruppamento semantico, la traduzione automatica e l'analisi del sentiment. Gli embedding sono spesso addestrati su larghe collezioni di testo e possono essere utilizzati come input per reti neurali e altri modelli di apprendimento automatico.
