\section{A}

\vspace{2em}
\subsection*{Abbreviazioni}
\par Le abbreviazioni, conosciute anche come acronimi o sigle, sono forme ridotte di parole o frasi utilizzate per semplificare la scrittura e la lettura. Spesso impiegate nel codice e nella documentazione, contribuiscono a migliorare l'indice di leggibilità del testo e dei programmi.

\vspace{2em}
\subsection*{Agile}
\par Agile è una metodologia di sviluppo software che enfatizza l'adattabilità, la collaborazione e la trasparenza. Promuove la gestione iterativa di un progetto, con cicli di sviluppo brevi chiamati \glossario{sprint}, durante i quali il team, dopo aver suddiviso i task per ruolo, lavora in modo organico per portare a termine le attività prestabilite. L'Agile valorizza la comunicazione continua con il cliente e la capacità di rispondere rapidamente ai cambiamenti, garantendo che il prodotto finale soddisfi le esigenze e le aspettative degli utenti.

\vspace{2em}
\subsection*{AI}
\par Vedi \glossario{Intelligenza Artificiale}

\vspace{2em}
\subsection*{API}
\par API, acronimo di Application Programming Interface, è un insieme di regole, protocolli e strumenti che consente a diverse applicazioni software di comunicare tra loro. Le API definiscono i metodi e le strutture dati che le applicazioni possono utilizzare per interagire e scambiare informazioni tra loro in modo standardizzato e sicuro. Le API possono essere utilizzate per accedere alle funzionalità di un sistema operativo, di un software o di un servizio web, consentendo agli sviluppatori di integrare tali funzionalità nelle proprie applicazioni senza dover conoscere i dettagli interni del sistema sottostante.

\vspace{2em}
\subsection*{Architettura}
\par Un'architettura software è una struttura che definisce le decisioni strategiche e le soluzioni per soddisfare i requisiti e raggiungere gli obiettivi di qualità, come ad esempio affidabilità, prestazioni, scalabilità, manutenibilità, usabilità e flessibilità. Un'architettura definisce in modo chiaro gli elementi del sistema, le loro relazioni e attributi, e fornisce una visione globale dell'intero sistema.

\vspace{2em}
\subsection*{Attori}
\par Nel contesto dei \glossario{casi d'uso}, gli attori rappresentano le entità interne o esterne che interagiscono con il sistema. Gli attori possono essere persone, altri sistemi software o componenti hardware. Essi sono definiti in base ai ruoli che svolgono nel contesto del sistema e possono essere coinvolti nell'innesco o nell'esecuzione dei casi d'uso. Identificare gli attori è fondamentale per comprendere le interazioni tra il sistema e il suo ambiente e per definire correttamente i requisiti e le funzionalità del sistema stesso.
