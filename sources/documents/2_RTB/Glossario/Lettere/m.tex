\section{M}
\vspace{2em}
\subsection*{Markup language}
Un markup language è un linguaggio di codifica utilizzato per annotare un documento in modo che sia distinguibile sia dal testo normale sia dal software che lo elabora. Esempi comuni includono HTML (HyperText Markup Language), utilizzato per creare pagine web, e XML (eXtensible Markup Language), utilizzato per memorizzare e trasportare dati. I markup languages definiscono una struttura e una sintassi specifiche per etichettare il contenuto, facilitando l'organizzazione, la visualizzazione e la manipolazione delle informazioni.

\vspace{2em}
\subsection*{Merge}
Il merge è il processo di combinazione di due o più rami di codice sorgente in un singolo ramo. Questo è un'operazione comune nei sistemi di controllo versione, come Git, dove gli sviluppatori lavorano su diverse branche per sviluppare funzionalità o correggere bug in modo isolato. Una volta completati, i cambiamenti vengono uniti nel ramo principale (main o master). Il merge può essere automatico o richiedere intervento manuale in caso di conflitti, dove due cambiamenti sono stati apportati allo stesso pezzo di codice.

\vspace{2em}
\subsection*{Metadati}
I metadati sono dati che descrivono altri dati, fornendo informazioni contestuali che facilitano la comprensione, l'organizzazione e la gestione delle informazioni. Esempi comuni di metadati includono autori, date di creazione, dimensioni dei file e parole chiave. Nei database, i metadati possono descrivere la struttura delle tabelle, le colonne e le relazioni tra dati. Nei sistemi di gestione dei contenuti, i metadati migliorano la ricerca e il recupero delle informazioni, permettendo una gestione più efficiente delle risorse digitali.

\vspace{2em}
\subsection*{Metriche}
Le metriche sono misurazioni quantitative utilizzate per valutare e valutare vari aspetti di un sistema, un processo o un prodotto. Nel contesto dello sviluppo del software, le metriche possono essere utilizzate per valutare la qualità del codice, le prestazioni del sistema, l'efficacia del processo di sviluppo e altro ancora. Le metriche forniscono informazioni utili per comprendere lo stato attuale, identificare le aree di miglioramento e prendere decisioni informate per ottimizzare il lavoro svolto. È importante selezionare e utilizzare le metriche in modo oculato, assicurandosi che siano rilevanti per gli obiettivi e i requisiti specifici del progetto o del processo in questione.

%\vspace{2em}
%\subsection*{Metriche di prodotto e qualità}

%Le metriche di prodotto e qualità sono misurazioni utilizzate per valutare le caratteristiche e le prestazioni di un prodotto software. Queste metriche possono includere la complessità del codice, la copertura dei test, la facilità d'uso dell'interfaccia utente, la manutenibilità del software e altro ancora. Le metriche di prodotto e qualità sono fondamentali per garantire che il software soddisfi gli standard di qualità e le esigenze degli utenti.

%\vspace{2em}
%\subsection*{Metriche di processo e progetto}

%Le metriche di processo e progetto sono misurazioni utilizzate per valutare l'efficacia e l'efficienza del processo di sviluppo del software e del progetto nel suo complesso. Queste metriche possono includere la produttività del team, il rispetto delle scadenze, la gestione del budget, il controllo delle modifiche e altro ancora. Le metriche di processo e progetto sono importanti per monitorare il progresso del progetto, identificare eventuali problemi e prendere provvedimenti correttivi tempestivi.

%\vspace{2em}
%\subsection*{Metriche di gestione dei rischi}

%Le metriche di gestione dei rischi sono misurazioni utilizzate per valutare e monitorare i rischi associati a un progetto software. Queste metriche possono includere la probabilità di occorrenza di un rischio, l'impatto potenziale di un rischio sul progetto, la gravità dei rischi identificati e l'efficacia delle strategie di mitigazione dei rischi. Le metriche di gestione dei rischi sono importanti per identificare e gestire proattivamente i rischi durante tutto il ciclo di vita del progetto.

%\vspace{2em}
%\subsection*{Metriche di documentazione}

%Le metriche di documentazione sono misurazioni utilizzate per valutare la completezza, la correttezza e la chiarezza della documentazione associata a un progetto software. Queste metriche possono includere la presenza di documenti necessari, la conformità alla struttura e al formato richiesti, la comprensibilità del contenuto e altro ancora. Le metriche di documentazione sono importanti per garantire che tutte le informazioni rilevanti siano documentate in modo adeguato e accessibile a tutti i membri del team e agli stakeholder del progetto.

\vspace{2em}
\subsection*{Modello}
\par Vedi \glossario{LLM}

\vspace{2em}
\subsection*{MVC}
Il Model-view-controller (MVC) è un pattern architetturale molto diffuso nell’ambito della programmazione orientata agli oggetti e dello sviluppo web. Si tratta di un modello di progettazione software in grado di separare la logica di presentazione dalla logica di business. La logica del programma è divisa in tre elementi interconnessi: il modello (rappresentazione interna e gestione dei dati), la vista (esposizione dei dati all’utente) e il controller (collegamento tra model e view).
