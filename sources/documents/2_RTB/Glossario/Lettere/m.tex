\section{M}
\vspace{2em}
\subsection*{Metriche}
Le metriche sono misurazioni quantitative utilizzate per valutare e valutare vari aspetti di un sistema, un processo o un prodotto. Nel contesto dello sviluppo del software, le metriche possono essere utilizzate per valutare la qualità del codice, le prestazioni del sistema, l'efficacia del processo di sviluppo e altro ancora. Le metriche forniscono informazioni utili per comprendere lo stato attuale, identificare le aree di miglioramento e prendere decisioni informate per ottimizzare il lavoro svolto. È importante selezionare e utilizzare le metriche in modo oculato, assicurandosi che siano rilevanti per gli obiettivi e i requisiti specifici del progetto o del processo in questione.

%\vspace{2em}
%\subsection*{Metriche di prodotto e qualità}

%Le metriche di prodotto e qualità sono misurazioni utilizzate per valutare le caratteristiche e le prestazioni di un prodotto software. Queste metriche possono includere la complessità del codice, la copertura dei test, la facilità d'uso dell'interfaccia utente, la manutenibilità del software e altro ancora. Le metriche di prodotto e qualità sono fondamentali per garantire che il software soddisfi gli standard di qualità e le esigenze degli utenti.

%\vspace{2em}
%\subsection*{Metriche di processo e progetto}

%Le metriche di processo e progetto sono misurazioni utilizzate per valutare l'efficacia e l'efficienza del processo di sviluppo del software e del progetto nel suo complesso. Queste metriche possono includere la produttività del team, il rispetto delle scadenze, la gestione del budget, il controllo delle modifiche e altro ancora. Le metriche di processo e progetto sono importanti per monitorare il progresso del progetto, identificare eventuali problemi e prendere provvedimenti correttivi tempestivi.

%\vspace{2em}
%\subsection*{Metriche di gestione dei rischi}

%Le metriche di gestione dei rischi sono misurazioni utilizzate per valutare e monitorare i rischi associati a un progetto software. Queste metriche possono includere la probabilità di occorrenza di un rischio, l'impatto potenziale di un rischio sul progetto, la gravità dei rischi identificati e l'efficacia delle strategie di mitigazione dei rischi. Le metriche di gestione dei rischi sono importanti per identificare e gestire proattivamente i rischi durante tutto il ciclo di vita del progetto.

%\vspace{2em}
%\subsection*{Metriche di documentazione}

%Le metriche di documentazione sono misurazioni utilizzate per valutare la completezza, la correttezza e la chiarezza della documentazione associata a un progetto software. Queste metriche possono includere la presenza di documenti necessari, la conformità alla struttura e al formato richiesti, la comprensibilità del contenuto e altro ancora. Le metriche di documentazione sono importanti per garantire che tutte le informazioni rilevanti siano documentate in modo adeguato e accessibile a tutti i membri del team e agli stakeholder del progetto.
