\section{Preventivo}
\label{sec:preventivo}

\par Il preventivo viene formulato tenendo in considerazione il costo orario di ciascun ruolo, il budget stimato e le attività pianificate per il periodo corrispondente. Ogni \glossario{sprint} è corredato da:
\begin{itemize}
    \item Un preventivo orario in forma tabellare;
    \item Un istogramma della distribuzione oraria per la coppia risorsa-ruolo;
    \item Un preventivo economico in forma tabellare;
    \item Un areogramma della distribuzione oraria per ruolo.
\end{itemize}

\vspace{0.5\baselineskip}
\par Il rendimento complessivo per ciascun componente è di 92 ore, ripartite equamente nei ruoli di progetto, per un totale di 646 ore produttive. L'uniformità nella distribuzione dei ruoli tra i membri del team viene mantenuta procedendo a rotazione, affinché ogni risorsa possa esplorare tutte le mansioni. Al fine di migliorare la leggibilità e la compattezza delle tabelle, i ruoli di progetto sono identificati dalle seguenti \glossario{abbreviazioni}:
\begin{itemize}
    \item \textbf{Re}: \Responsabile[U];
    \item \textbf{Am}: \Amministratore[U];
    \item \textbf{An}: \Analista[U];
    \item \textbf{Pt}: \Progettista[U];
    \item \textbf{Pr}: \Programmatore[U];
    \item \textbf{Ve}: \Verificatore[U].
\end{itemize}

%\subsection{RTB}
%TODO
\subsubsection{Sprint 1: da 2024-04-03 a 2024-04-19}
\begin{minipage}{\textwidth}
Di seguito è riportata la distribuzione delle ore per ciascun membro del team, accumulate in totali per persona e per ruolo:
\begin{table}[H]
  \begin{tabularx}{\textwidth}{|c|*{6}{>{\centering}X|}c|}
    \hline
    \multicolumn{8}{|c|}{\textbf{Preventivo orario}} \\
    \hline
    \textbf{Membro del team} & \textbf{Re} & \textbf{Am} & \textbf{An} & \textbf{Pt} & \textbf{Pr} & \textbf{Ve} & \textbf{Totale per persona} \\
    \hline
    Cavalli Riccardo & 7 & 0 & 0 & 0 & 0 & 0 & 7 \\
    \hline
    Pianon Raul & 0 & 0 & 0 & 0 & 0 & 8 & 8 \\
    \hline
    Dall'Amico Martina & 0 & 0 & 9 & 0 & 0 & 0 & 9 \\
    \hline
    Cristo Marco & 0 & 0 & 9 & 0 & 0 & 0 & 9 \\
    \hline
    Lewental Sebastiano & 0 & 0 & 9 & 0 & 0 & 0 & 9 \\
    \hline
    Zecchinato Mattia & 0 & 0 & 0 & 7 & 0 & 0 & 7 \\
    \hline
    Stocco Tommaso & 0 & 6 & 0 & 0 & 0 & 0 & 6 \\
    \hline
    \textbf{Totale per ruolo} & 7 & 6 & 27 & 7 & 0 & 8 & \textbf{55} \\
    \hline
  \end{tabularx}
  \caption{Sprint 1 - Preventivo orario}
\end{table}
\end{minipage}

\begin{figure}[H]
  \centering
  \includegraphics[width=0.90\textwidth]{assets/Preventivo/Sprint-1/distribuzione_ore_risorsa_ruolo.pdf}
  \caption{Sprint 1 - Istogramma della distribuzione oraria per la coppia risorsa-ruolo}
\end{figure}

\begin{figure}[H]
  \centering
  \includegraphics[width=0.90\textwidth]{assets/Preventivo/Sprint-1/distribuzione_ore_ruolo.pdf}
  \caption{Sprint 1 - Areogramma della distribuzione oraria per ruolo}
\end{figure}

Di seguito è riportato il preventivo economico del primo \glossario{sprint}:
\begin{table}[H]
  \centering
  \begin{tabular}{|c|c|c|}
    \hline
    \multicolumn{3}{|c|}{\textbf{Preventivo economico}} \\
    \hline
    \textbf{Ruolo} & \textbf{Ore per ruolo} & \textbf{Costo (in \texteuro)} \\
    \hline
    Responsabile & 7 & 210,00 \\
    \hline
    Amministratore & 6 & 120,00 \\
    \hline
    Analista & 27 & 675,00 \\
    \hline
    Progettista & 7 & 175,00 \\
    \hline
    Programmatore & 0 & 0,00 \\
    \hline
    Verificatore & 8 & 120,00 \\
    \hline
    \textbf{Totale} & 55 & \textbf{1.300,00} \\
    \hline
  \end{tabular}
  \caption{Sprint 1 - Preventivo economico}
\end{table}
\subsection{Secondo sprint}

\begin{minipage}{\textwidth}
Di seguito è riportata la distribuzione delle ore per ciascun membro del team, accumulate in totali per persona e per ruolo:
\begin{table}[H]
  \begin{tabularx}{\textwidth}{|c|*{6}{>{\centering}X|}c|}
    \hline
    \multicolumn{8}{|c|}{\textbf{Consuntivo orario}} \\
    \hline
    \textbf{Membro del team} & \textbf{Re} & \textbf{Am} & \textbf{An} & \textbf{Pt} & \textbf{Pr} & \textbf{Ve} & \textbf{Totale per persona} \\
    \hline
    Cavalli Riccardo & 1 & 6 & 0 & 0 & 0 & 0 & 7 \\
    \hline
    Pianon Raul & 7 & 0 & 0 & 0 & 0 & 0 & 7 \\
    \hline
    Dall'Amico Martina & 0 & 0 & 0 & 0 & 0 & 5 & 5 \\
    \hline
    Cristo Marco & 0 & 0 & 0 & 3 & 5 & 0 & 8 \\
    \hline
    Lewental Sebastiano & 0 & 0 & 0 & 6 & 0 & 0 & 6 \\
    \hline
    Zecchinato Mattia & 0 & 0 & 0 & 0 & 0 & 6 & 6 \\
    \hline
    Stocco Tommaso & 0 & 0 & 6 & 0 & 0 & 0 & 6 \\
    \hline
    \textbf{Totale per ruolo} & 8 & 6 & 6 & 9 & 5 & 11 & \textbf{45} \\
    \hline
  \end{tabularx}
  \caption{Sprint 2 - Consuntivo orario}
\end{table}
\end{minipage}

\begin{figure}[H]
  \centering
  \includegraphics[width=0.90\textwidth]{assets/Consuntivo/Sprint-2/distribuzione_ore_risorsa_ruolo.pdf}
  \caption{Sprint 2 - Istogramma della distribuzione oraria per la coppia risorsa-ruolo}
\end{figure}

\begin{figure}[H]
  \centering
  \includegraphics[width=0.90\textwidth]{assets/Consuntivo/Sprint-2/distribuzione_ore_ruolo.pdf}
  \caption{Sprint 2 - Areogramma della distribuzione oraria per ruolo}
\end{figure}

\begin{minipage}{\textwidth}
Di seguito è riportato il consuntivo economico del secondo \glossario{sprint}:
\begin{table}[H]
\begin{adjustwidth}{-0.5cm}{-0.5cm}
  \centering
  \begin{tabular}{|P{2.9cm}|P{2.3cm}|P{2.5cm}|P{2.3cm}|>{\arraybackslash}P{2.5cm}|}
    \hline
    \multicolumn{5}{|c|}{\textbf{Consuntivo economico}} \\
    \hline
    \textbf{Ruolo} & \textbf{Ore per ruolo} & \textbf{Delta ore preventivo - consuntivo} & \textbf{Costo (in \texteuro)} & \textbf{Delta costo preventivo - consuntivo (in \texteuro)} \\
    \hline
    Responsabile & 8 & -1 & 240,00 & -30,00 \\
    \hline
    Amministratore & 6 & +1 & 120,00 & +20,00 \\
    \hline
    Analista & 6 & +1 & 150,00 & +25,00 \\
    \hline
    Progettista & 9 & -2 & 225,00 & -50,00 \\
    \hline
    Programmatore & 5 & +3 & 75,00 & +45,00 \\
    \hline
    Verificatore & 11 & +1 & 160,00 & +15,00 \\
    \hline
    \textbf{Totale} & \textbf{45} & +3 & \textbf{975,00} & +25,00 \\
    \hline
    \textbf{Restante} & 542 & / & 10.870,00 & / \\
    \hline
    \textbf{Sprint pregressi} & 587 & / & 11.845,00 & / \\
    \hline
  \end{tabular}
  \caption{Sprint 2 - Consuntivo economico}
\end{adjustwidth}
\end{table}
\end{minipage}

\begin{figure}[H]
  \centering
  \includegraphics[width=0.90\textwidth]{assets/Consuntivo/Sprint-2/copertura_oraria.pdf}
  \caption{Sprint 2 - Areogramma del tempo speso (in ore) rispetto al totale}
\end{figure}

\begin{figure}[H]
  \centering
  \includegraphics[width=0.90\textwidth]{assets/Consuntivo/Sprint-2/budget_speso.pdf}
  \caption{Sprint 2 - Areogramma del budget speso rispetto al totale}
\end{figure}

\begin{minipage}{\textwidth}
  Di seguito sono riportate le ore rimanenti per la coppia risorsa-ruolo:
  \begin{table}[H]
    \begin{tabularx}{\textwidth}{|c|*{6}{>{\centering}X|}c|}
      \hline
      \multicolumn{8}{|c|}{\textbf{Ore rimanenti per la coppia risorsa-ruolo}} \\
      \hline
      \textbf{Membro del team} & \textbf{Re} & \textbf{Am} & \textbf{An} & \textbf{Pt} & \textbf{Pr} & \textbf{Ve} & \textbf{Totale per persona} \\
      \hline
      Cavalli Riccardo & 1 & 2 & 9 & 23 & 22 & 20 & 77 \\
      \hline
      Pianon Raul & 2 & 8 & 9 & 23 & 22 & 12 & 76 \\
      \hline
      Dall'Amico Martina & 9 & 8 & 1 & 23 & 22 & 15 & 78 \\
      \hline
      Cristo Marco & 9 & 8 & 2 & 20 & 17 & 20 & 76 \\
      \hline
      Lewental Sebastiano & 9 & 8 & 2 & 17 & 22 & 20 & 78 \\
      \hline
      Zecchinato Mattia & 9 & 8 & 8 & 17 & 22 & 20 & 79 \\
      \hline
      Stocco Tommaso & 9 & 2 & 3 & 23 & 22 & 20 & 79 \\
      \hline
      \textbf{Totale per ruolo} & 48 & 44 & 34 & 146 & 149 & 121 & \textbf{542} \\
      \hline
    \end{tabularx}
    \caption{Sprint 2 - Ore rimanenti per la coppia risorsa-ruolo}
  \end{table}
\end{minipage}

\subsubsection{Revisione delle attività}

Nel corso del secondo \glossario{sprint}, il team ha svolto le seguenti attività:
\begin{itemize}
    \item Stesura verbali interni;
    \item Stesura verbali esterni;
    \item Stesura iniziale del \PdQ;
    \item Aggiornamento del \PdP\ con l'aggiunta dei preventivi e consuntivi dei primi due \glossario{sprint}, e la stesura della sezione riguardante l'analisi dei rischi;
    \item Aggiornamento delle \NdP\ con l'inserimento delle sezioni riguardanti \glossario{Jira}, l'integrazione con \glossario{GitHub} e la tabella delle attività pianificate (Todo) nei verbali;
    \item Inserimento dei termini nel \Gls;
    \item Conversione del documento di \AdR\ in \glossario{LateX};
    \item Miglioramento della struttura dei verbali interni;
    \item Creazione di un template per la stesura degli appunti in Google Docs;
    \item Aggiornamento del documento di \AdR\ con la descrizione dei \glossario{casi d'uso};
    \item Definizione di un file di configurazione per automatizzare il caricamento dei pdf;
    \item Studio delle tecnologie back-end;
    \item Composizione preliminare del \glossario{prompt};
    \item Studio delle tecnologie per lo sviluppo della \glossario{web app};
    \item Sviluppo di un prototipo per la parte funzionale del prodotto.
\end{itemize}

\subsubsection{Retrospettiva}

\par Di seguito sono riportati i risultati del questionario di valutazione dello \glossario{sprint}, realizzato dal Responsabile in carica per ottimizzare la fase di \glossario{retrospettiva}:
\begin{itemize}
  \item Organizzazione dello \glossario{sprint}\ - Valutazione: 8,5;
  \item Conduzione dei meeting interni - Valutazione: 8,5;
  \item Conduzione dei meeting esterni - Valutazione: 8,5;
  \item Impegno e partecipazione dei singoli membri - Valutazione: 7,5;
  \item Non tutti i membri del team erano a conoscenza delle proprie mansioni;
  \item La numerosità delle riunioni è adeguata;
  \item Le riunioni sono state organizzate quasi sempre con il giusto preavviso;
  \item Da migliorare significativamente il rapporto ore spese/ore produttive;
  \item Da migliorare la definizione del lavoro di progettista e programmatore;
  \item Da migliorare la distribuzione delle risorse al ruolo di amministratore;
  \item Da definire meglio il ruolo del progettista;
  \item Proposta di divisione della stesura del \PdQ\ in sotto-task.
\end{itemize}

\vspace{0.5\baselineskip}
\par A seguire le \textbf{analisi a posteriori} del secondo \glossario{sprint}:
\begin{itemize}
  \item Durante lo \glossario{sprint} sono state evidenziate delle difficoltà nella divisione delle attività e delle ore distribuite ai ruoli di progettista, analista e programmatore;
  \item A tali difficoltà si andrà incontro dando una migliore definizione delle attività da svolgere e una distribuzione più equa al ruolo di amministratore, a partire dal prossimo \glossario{sprint};
  \item Viene inoltre richiesto di dividere la stesura del \PdQ in sotto-task per permettere di lavorare in segmenti più piccoli, in modo da poter strutturare e redarre più velocemente le metriche;
  \item La migrazione da \glossario{GitHub} a \glossario{Jira} non è stata immediata, ma ci si è resi conto essere una soluzione più conveniente e utile per lo sviluppo di un cruscotto delle attività più organizzato.
  \item Dopo l'incontro con la Proponente è stata proposta una presentazione sulla tecnologia \glossario{Streamlit} come strumento di sviluppo dell'applicativo.
\end{itemize}

\subsubsection{Aggiornamento pianificazione e preventivo}
\par Il team ha definito un piano d'azione per migliorare l'organizzazione e la produttività del prossimo \glossario{sprint}:
\begin{itemize}
  \item Destinare ore di ruoli diversi ai membri, in modo da coprire eventuali zone di ettesa o inattività;
  \item Dividere la stesura dei documenti in sotto-task specifiche;
  \item Distribuzione più efficiente delle risorse, assegnando più persone ad un determinato ruolo quando ci si aspetta che questo lo possa prevedere;
  \item Utilizzo di \glossario{Jira} come \glossario{ITS}, mantenendo \glossario{GitHub} per il sistema di versionamento;
  \item Fissato un incontro in presenza per permettere l'allineamento dei membri nel ruolo entrante.
\end{itemize}

\paragraph*{Pianificazione futura:}
\par Come riportato nell'analisi a posteriori, il team ha deciso di assegnare ai vari membri ore di ruoli diversi per redistribuire le responsabilità a ridurre eventuali tempi di rallentamento dati dalla mancanza di risorse per un ruolo. Inoltre si propone lo studio della tecnologia \glossario{Streamlit} da presentare alla Proponente in modo da evidenziarne i pro e i contro.

\paragraph*{Preventivo "a finire" (\sezione{sec:stima_temporale}):}
\par Riguardo il ruolo di amministratore sono state riscontrate delle difficoltà nell'assegnazione delle risorse, in quanto un membro si è ritrovato a gestire un carico eccessivo di lavoro, che ha portato ad un'ora di eccesso rispetto alle ore preventivate. Discorso simile è applicabile al ruolo di programmatore, il quale interfacciandosi con una libreria sconosciuta, ha dovuto documentarsi prima di iniziare una sezione di test e aggiornare il gruppo sul funzionamento della tecnologia. Infine il ruolo di progettista ha delle ore in difetto a causa di una scarsa definizione delle attività da svolgere. Azioni correttive per queste differenze verranno attivate dal prossimo sprint come già menzionato nel piano d'azione.

\paragraph*{Gestione dei rischi (\sezione{sec:analisi_rischi}):}


\subsubsection{Sprint 3: da 2024-05-06 a 2024-05-21}
\par Dato che il \PdQ\ non ha registrato avanzamenti significativi negli \glossario{sprint} precedenti, il team ha fissato come obiettivo primario l'individuazione delle metriche di qualità. Inoltre, il gruppo ha programmato il \glossario{benchmark} dei modelli e l'implementazione delle funzionalità back-end correlate alla libreria \glossario{txtai}. Contestualmente, verrà avviato lo sviluppo della web app con \glossario{Streamlit} e verranno configurati \glossario{framework} back-end e front-end alternativi. Infine, il team ha deciso di riorganizzare la struttura del \PdP.

\paragraph{Obiettivi}
\begin{itemize}
  \item Aggiornamento delle \NdP\ con le modalità di integrazione Jira - Github;
  \item Stesura verbali interni ed esterni;
  \item Formalizzazione delle metriche di qualità nelle \NdP;
  \item Inserimento nel \PdQ\ delle soglie di tolleranza stabilite per ciascuna metrica;
  \item Stesura delle sezioni incomplete nel documento di \AdR\ ed espansione dei casi e sottocasi d'uso;
  \item Elaborazione di una relazione sul framework Streamlit (pro e contro, confronto con alternative);
  \item Pianificazione e preventivo dello sprint 4;
  \item Miglioramento della struttura del \PdP;
  \item Creazione della prima bozza dell'interfaccia grafica;
  \item Inizio sviluppo della web app;
  \item Creazione di un doppio indice per la ricerca semantica e la generazione del prompt;
  \item Benchmark dei modelli;
  \item Configurazione di framework back-end e front-end alternativi (Flask, Django, Vue.js, Next.js).
\end{itemize}

\vspace{0.5\baselineskip}
\par [Inserire Gantt]
\subsection{Sprint 4: da 2024-05-22 a 2024-06-03}
\begin{minipage}{\textwidth}
Di seguito è riportata la distribuzione delle ore per ciascun membro del team, accumulate in totali per persona e per ruolo:
\begin{table}[H]
  \begin{tabularx}{\textwidth}{|c|*{6}{>{\centering}X|}c|}
    \hline
    \multicolumn{8}{|c|}{\textbf{Preventivo orario}} \\
    \hline
    \textbf{Membro del team} & \textbf{Re} & \textbf{Am} & \textbf{An} & \textbf{Pt} & \textbf{Pr} & \textbf{Ve} & \textbf{Totale per persona} \\
    \hline
    Riccardo Cavalli & 0 & 0 & 0 & 0 & 3 & 6 & 9 \\
    \hline
    Raul Pianon & 0 & 0 & 6 & 0 & 0 & 2 & 8 \\
    \hline
    Martina Dall'Amico & 0 & 0 & 0 & 6 & 2 & 0 & 8 \\
    \hline
    Marco Cristo & 6 & 0 & 0 & 0 & 0 & 2 & 8 \\
    \hline
    Sebastiano Lewental & 0 & 7 & 0 & 0 & 0 & 0 & 7 \\
    \hline
    Mattia Zecchinato & 0 & 3 & 0 & 3 & 0 & 0 & 6 \\
    \hline
    Tommaso Stocco & 0 & 0 & 0 & 0 & 8 & 0 & 8 \\
    \hline
    \textbf{Totale per ruolo} & 6 & 10 & 6 & 9 & 13 & 10 & \textbf{54} \\
    \hline
  \end{tabularx}
  \caption{Sprint 4 - Preventivo orario}
\end{table}
\end{minipage}

\begin{figure}[H]
  \centering
  \includegraphics[width=0.90\textwidth]{assets/Preventivo/Sprint-4/distribuzione_ore_risorsa_ruolo.pdf}
  \caption{Sprint 4 - Istogramma della distribuzione oraria per la coppia risorsa-ruolo}
\end{figure}

\begin{figure}[H]
  \centering
  \includegraphics[width=0.90\textwidth]{assets/Preventivo/Sprint-4/distribuzione_ore_ruolo.pdf}
  \caption{Sprint 4 - Areogramma della distribuzione oraria per ruolo}
\end{figure}

\begin{minipage}{\textwidth}
Di seguito è riportato il preventivo economico del quarto \glossario{sprint}:
\begin{table}[H]
  \centering
  \begin{tabular}{|c|c|c|}
    \hline
    \multicolumn{3}{|c|}{\textbf{Preventivo economico}} \\
    \hline
    \textbf{Ruolo} & \textbf{Ore per ruolo} & \textbf{Costo (in \texteuro)} \\
    \hline
    \Responsabile[U]{} & 6 & 180,00 \\
    \hline
    \Amministratore[U]{} & 10 & 200,00 \\
    \hline
    \Analista[U]{} & 6 & 150,00 \\
    \hline
    \Progettista[U]{} & 9 & 225,00 \\
    \hline
    \Programmatore[U]{} & 13 & 195,00 \\
    \hline
    \Verificatore[U]{} & 10 & 150,00 \\
    \hline
    \textbf{Totale} & 54 & \textbf{1.100,00} \\
    \hline
  \end{tabular}
  \caption{Sprint 4 - Preventivo economico}
\end{table}
\end{minipage}
\subsection{Quinto sprint}

\begin{minipage}{\textwidth}
  Di seguito è riportata la distribuzione delle ore per ciascun membro del team, accumulate in totali per persona e per ruolo:
  \begin{table}[H]
    \begin{tabularx}{\textwidth}{|c|*{6}{>{\centering}X|}c|}
      \hline
      \multicolumn{8}{|c|}{\textbf{Consuntivo orario}} \\
      \hline
      \textbf{Membro del team} & \textbf{Re} & \textbf{Am} & \textbf{An} & \textbf{Pt} & \textbf{Pr} & \textbf{Ve} & \textbf{Totale per persona} \\
      \hline
      Cavalli Riccardo & 0 & 1 & 2 & 0 & 2 & 1 & 6 \\ \hline
      Pianon Raul & 0 & 0 & 0 & 0 & 5 & 0 & 5 \\ \hline
      Dall’Amico Martina & 3 & 0 & 0 & 0 & 0 & 3 & 6 \\ \hline
      Cristo Marco & 1 & 0 & 0 & 0 & 3 & 2 & 6 \\ \hline
      Lewental Sebastiano & 0 & 0 & 0 & 0 & 3 & 2 & 5 \\ \hline
      Zecchinato Mattia & 0 & 1 & 0 & 0 & 4 & 0 & 5 \\ \hline
      Stocco Tommaso & 0 & 0 & 0 & 0 & 3 & 0 & 3 \\ \hline
      \textbf{Totale ore per ruolo} & 4 & 2 & 2 & 0 & 20 & 8 & \textbf{36} \\
      \hline
    \end{tabularx}
    \caption{Sprint 5 - Consuntivo orario}
  \end{table}
  \end{minipage}
  
  \begin{figure}[H]
    \centering
    \includegraphics[width=0.90\textwidth]{assets/Consuntivo/Sprint-5/distribuzione_ore_risorsa_ruolo.pdf}
    \caption{Sprint 5 - Istogramma della distribuzione oraria per la coppia risorsa-ruolo}
  \end{figure}
  
  \begin{figure}[H]
    \centering
    \includegraphics[width=0.90\textwidth]{assets/Consuntivo/Sprint-5/distribuzione_ore_ruolo.pdf}
    \caption{Sprint 5 - Areogramma della distribuzione oraria per ruolo}
  \end{figure}
  
  \begin{minipage}{\textwidth}
  Di seguito è riportato il consuntivo economico del quinto \glossario{sprint}:
  \begin{table}[H]
  \begin{adjustwidth}{-0.5cm}{-0.5cm}
    \centering
    \begin{tabular}{|P{2.9cm}|P{2.3cm}|P{2.5cm}|P{2.3cm}|>{\arraybackslash}P{2.5cm}|}
      \hline
      \multicolumn{5}{|c|}{\textbf{Consuntivo economico}} \\
      \hline
      \textbf{Ruolo} & \textbf{Ore per ruolo} & \textbf{Delta ore preventivo - consuntivo} & \textbf{Costo (in \texteuro)} & \textbf{Delta costo preventivo - consuntivo (in \texteuro)} \\
      \hline
      Responsabile & 4 & 0 & 120,00 & 0,00 \\ \hline
      Amministratore & 2 & 1 & 40,00 & 20,00 \\ \hline
      Analista & 2 & 1 & 50,00  & 25,00 \\ \hline
      Progettista &  0 & 0 & 0,00 & 0,00 \\ \hline
      Programmatore & 20 & 2 & 300,00 & 30,00 \\ \hline
      Verificatore & 8 & -3 & 120,00 & -45,00 \\ \hline
      \textbf{Totale} & \textbf{36} & 1 & \textbf{630,00} & 30,00 \\ \hline
    \textbf{Restante} & 410 & / & 8.115,00 & / \\ \hline
      \textbf{Sprint pregressi} & 200 & / & 4.275,00 & / \\ \hline
    \end{tabular}
    \caption{Sprint 5 - Consuntivo economico}
  \end{adjustwidth}
  \end{table}
  \end{minipage}
  
  \begin{figure}[H]
    \centering
    \includegraphics[width=0.90\textwidth]{assets/Consuntivo/Sprint-5/copertura_oraria.pdf}
    \caption{Sprint 5 - Areogramma del tempo speso (in ore) rispetto al totale}
  \end{figure}
  
  \begin{figure}[H]
    \centering
    \includegraphics[width=0.90\textwidth]{assets/Consuntivo/Sprint-5/budget_speso.pdf}
    \caption{Sprint 5 - Areogramma del budget speso rispetto al totale}
  \end{figure}
  
  \begin{minipage}{\textwidth}
    Di seguito sono riportate le ore rimanenti per la coppia risorsa-ruolo:
    \begin{table}[H]
      \begin{tabularx}{\textwidth}{|c|*{6}{>{\centering}X|}c|}
        \hline
        \multicolumn{8}{|c|}{\textbf{Ore rimanenti per la coppia risorsa-ruolo}} \\
        \hline
        \textbf{Membro del team} & \textbf{Re} & \textbf{Am} & \textbf{An} & \textbf{Pt} & \textbf{Pr} & \textbf{Ve} & \textbf{Totale per persona} \\
        \hline
        Cavalli Riccardo & 0 & 1 & 7 & 14 & 16 & 15 & 53 \\ \hline
        Pianon Raul & 2 & 10 & 2 & 20 & 10 & 12 & 56 \\ \hline
        Dall’Amico Martina & 6 & 2 & 1 & 14 & 22 & 13 & 58 \\ \hline
        Cristo Marco & 2 & 10 & 2 & 17 & 10 & 15 & 56 \\ \hline
        Lewental Sebastiano & 9 & 4 & 2 & 11 & 18 & 15 & 59 \\ \hline
        Zecchinato Mattia & 9 & 8 & 3 & 11 & 16 & 15 & 62 \\ \hline
        Stocco Tommaso & 5 & 4 & 3 & 20 & 13 & 19 & 64 \\ \hline
        \textbf{Totale ore per ruolo} & 33 & 40 & 20 & 107 & 106 & 104 & \textbf{410} \\ 
        \hline
      \end{tabularx}
      \caption{Sprint 5 - Ore rimanenti per la coppia risorsa-ruolo}
    \end{table}
  \end{minipage}

\subsubsection{Revisione delle attività}

Nell'arco del quinto \glossario{sprint}, il team ha svolto le seguenti attività:
\begin{itemize}
  \item Stesura del consuntivo dello sprint 4;
  \item Completamento della formalizzazione della dashboard \glossario{Google Sheets} nelle \NdP;
  \item Completamento della spiegazione delle funzionalità e specifiche dei vari documenti nelle \NdP;
  \item Stesura del verbale interno del 03/06;
  \item Stesura del preventivo e della pianificazione dello sprint 5 nel \PdP;
  \item Ampliamento delle \AdR con grafici e aggiunta casi d’uso;
  \item Configurazione di Flask con Vue.js;
  \item Integrazione tra \glossario{back-end} e \glossario{front-end};
  \item Connessione al \glossario{database} con Flask;
  \item Implementazione delle funzionalità di Login e Logout con FastAPI nel \glossario{back-end};
  \item Gestione delle operazioni \glossario{CRUD} per il \glossario{dizionario dati} nel backend;
  \item Ultimi approfondimenti sul funzionamento di \glossario{txtai};
  \item Sviluppo del \glossario{front-end} per il ChatBOT;
  \item Progettazione e sviluppo dei casi d’uso nel \glossario{front-end};
  \item Modifica del \glossario{dizionario dati}, inclusa la descrizione delle tabelle e delle chiavi esterne;
  \item Aggiornamento del prompt generator dopo l'incontro con la \glossario{Proponente};
  \item Riorganizzazione del file di \glossario{log}.
  \item Test del dizionario dati in italiano;
  \item Aggiunta di test con modelli locali.
\end{itemize}

\subsubsection{Retrospettiva}

\par Di seguito sono riportati i risultati del questionario di valutazione dello \glossario{sprint}:
\begin{itemize}
  \item Organizzazione dello \glossario{sprint}\ - Valutazione: 8,2;
  \item Conduzione dei meeting interni - Valutazione: 8,2;
  \item Conduzione dei meeting esterni - Valutazione: 8,8;
  \item Impegno e partecipazione dei singoli membri - Valutazione: 7,5;
  \item La quasi totalità dei membri del team era a conoscenza delle proprie mansioni;
  \item La numerosità delle riunioni è risultata adeguata;
  \item Le riunioni sono state organizzate sempre con il giusto preavviso;
  \item Il rapporto ore spese/ore produttive ha avuto un miglioramento significativo e un bilanciamento equo tra i membri del gruppo;
  \item La produttività generale ha raggiunto una buona soglia;
  \item Alcuni membri del team hanno messo in luce la necessità di una frequanza maggiore di aggiornamento del \glossario{repository} di ChatSql.
\end{itemize}

\vspace{0.5\baselineskip}
\par A seguire le \textbf{analisi a posteriori} del quinto \glossario{sprint}:
\begin{itemize}
  \item Miglioramento dell'Efficienza: Il team ha riscontrato un miglioramento nel rapporto tra ore produttive e ore di orologio spese per portare a termine i task stabiliti. Questo è dovuto alla migliore organizzazione che, con l'aumentare degli sprint e dell'esperienza, sta maturando. Inoltre, il minor bisogno di imparare nuove tecnologie e l'applicazione delle conoscenze acquisite in precedenza hanno contribuito a questo miglioramento;
  \item Adozione delle Nuove Tecnologie: Il passaggio alle nuove tecnologie è stato repentino ma non complicato. Negli sprint precedenti, tutte le ipotesi sulle varie tecnologie erano state valutate e approfondite, inclusi Vue.js e Flask. Questo ha permesso al team di adattarsi rapidamente e di integrare efficacemente le nuove tecnologie nel progetto;
  \item Incontro con la Proponente: Durante lo sprint, è stato organizzato un incontro con la Proponente, durante il quale è stata mostrata una demo del progetto. I riscontri ricevuti sono stati positivi e costruttivi. Dall'incontro sono emersi nuovi casi d'uso e nuovi test da sviluppare, che assicureranno un miglioramento del prodotto finale;
  \item Aggiornamento del repository: un'area di miglioramento individuata riguarda l'aggiornamento del \glossario{repository} ChatSQL. Per mantenere il team allineato rispetto al lavoro dei programmatori, è necessario aggiornare l'ambiente condiviso con maggiore frequenza. Pertanto, si è stabilito di applicare con più costanza la pratica di \glossario{continuous integration}, invece di effettuare commit con minore frequenza e maggiore volume. Con questo approccio, il team ritiene di poter ottimizzare sia la verifica che lo sviluppo, riducendo il rischio di conflitti;
  \item Comprensione dei Casi d'Uso: Una difficoltà riscontrata è stata la comprensione del legame tra i casi d'uso. Questo aspetto è stato approfondito durante un incontro con il Professor Cardin, in cui è stato esaminato un caso d'uso specifico relativo alla generazione del prompt. L'incontro ha permesso di chiarire i ragionamenti e le motivazioni alla base della struttura attuale e ha fornito indicazioni sui miglioramenti da apportare nel prossimo sprint.
\end{itemize}

In conclusione, lo sprint 5 ha visto significativi progressi in termini di efficienza e adozione delle tecnologie, nonché preziosi feedback e direzioni per migliorare ulteriormente il progetto.


\subsubsection{Aggiornamento pianificazione e preventivo}
\par Il team ha definito un piano d'azione per migliorare l'organizzazione e la produttività del prossimo \glossario{sprint}:
\begin{itemize}
  \item Riduzione delle ore produttive per tutti i membri del gruppo, al fine di lasciare spazio alla preparazione degli esami universitari e mantenimento di sprint più corti con task mirate in previsione dell’ RTB;
  \item Revisione completa dei documenti esistenti, con particolare attenzione alla verifica della coerenza tra di essi;
  \item Miglioramento della comunicazione tra i programmatori di backend e frontend per garantire un maggiore allineamento e coordinamento delle attività;
  \item Aggiornamento continuo dell'Analisi dei Requisiti (AdR) per riflettere accuratamente i requisiti funzionali e non funzionali del progetto;
  \item Espansione del Piano di Qualifica (PdQ), includendo grafici delle metriche, variazioni di costo, stabilità dei requisiti e frequenza di merge delle pull request;
  \item Sviluppo del frontend, con l'implementazione della pagina di gestione dei dizionari dati, la visualizzazione della struttura del dizionario dati e la visualizzazione del log;
  \item Completamento dell'integrazione tra backend e frontend per assicurare una funzionalità completa e senza interruzioni;
  \item Espansione e integrazione delle metriche di valutazione nel PdQ per un monitoraggio più efficace delle prestazioni del progetto;
  \item Revisione dei requisiti funzionali nell'AdR per garantire che siano aggiornati e accurati, in linea con l'evoluzione del progetto.
\end{itemize}

\paragraph*{Pianificazione futura:}

Con l'avvicinarsi della scadenza per la RTB, la pianificazione futura prevede un'attenzione particolare alle ore di verificatore, che saranno fondamentali per il controllo e la revisione di tutti i documenti redatti fino ad ora. In particolare, sarà effettuata una revisione completa del Glossario e del \PdP, con l'obiettivo di garantire la massima coerenza e coesione tra tutti i documenti prodotti. Questa attività di verifica è essenziale per assicurare che ogni parte della documentazione sia accurata, completa e in linea con gli standard richiesti.
Per quanto riguarda lo sviluppo di ChatSQL, le attività rimanenti nel backlog saranno affrontate nel prossimo sprint. In particolare, ci concentreremo sull'implementazione delle parti mancanti del \glossario{front-end}, come la gestione dei \glossario{log}, la visualizzazione del \glossario{dizionario dati} e il completamento del chatbot. 
In sintesi, la pianificazione futura si focalizzerà sulla rigorosa verifica dei documenti e sul completamento delle componenti chiave del front-end di ChatSQL, assicurando che tutti gli obiettivi prefissati per la RTB siano raggiunti in modo efficace e puntuale.



\paragraph*{Preventivo "a finire" (\sezione{sec:stima_temporale}):}
La revisione \glossario{RTB} (inizialmente prevista per la prima settimana di giugno) è stata posticipata a causa del cambio tecnologico effettuato durante lo \glossario{sprint}. Il team ha fissato la RTB per il periodo che va dal 2024-07-01 al 2024-07-15. In origine, il piano prevedeva l'utilizzo di tecnologie differenti per la realizzazione delle funzionalità del sistema. Tuttavia, dopo un'attenta valutazione delle esigenze del progetto e delle opportunità offerte dalle nuove tecnologie, si è deciso di adottare Flask per il back-end e Vue.js per il front-end. Il passaggio a Flask e Vue.js ha comportato una ridefinizione delle attività e la necessità di acquisire nuove competenze, richiedendo un periodo di adattamento per il team. Di conseguenza, la data prevista per la RTB è stata estesa, al fine di garantire il raggiungimento degli obiettivi senza pregiudicare la sessione di esami. Nonostante il posticipo, non c'è stata alcuna variazione nel budget previsto per il progetto.


\paragraph*{Gestione dei rischi (\sezione{sec:analisi_rischi}):}
\par Nel corso del quinto \glossario{sprint}, il rischio di rallentamento è stato mitigato correttamente; tuttavia, nei prossimi sprint tale rischio potrebbe affiorare nuovamente e sommarsi ad altri:
\begin{itemize}
  \item \textbf{Rischi relativi a rallentamenti}: Uno dei rischi riscontrati dal team è stato il rallentamento dovuto alla sessione di esami universitari, che ha comportato la necessità di coordinare sia lo studio personale che l'avanzamento del progetto. La concomitanza delle scadenze accademiche con le attività di sviluppo ha rallentato il flusso di lavoro; tuttavia, questo rischio era stato preventivato con largo anticipo. Per mitigarne l'impatto, sono state adottate le seguenti contromisure: negli sprint precedenti, il carico di lavoro è stato intensificato, in modo tale da anticipare e completare una parte significativa delle attività prima dell'inizio della sessione di esami. In fase di preventivo del quinto \glossario{sprint}, le ore produttive sono state ridotte, permettendo al team di dedicare il tempo necessario alla preparazione degli esami. Inoltre, sono state pianificate le attività con scadenze più flessibili. Infine, è stato promossa la comunicazione asincrona e l'organizzazione di riunioni brevi e mirate, per ottimizzare il tempo a disposizione di ciascun membro del team.
\end{itemize}

\vspace{0.5\baselineskip}
\par Durante il quinto \glossario{sprint}, alcune contromisure si sono rivelate utili a mitigare i rischi individuati in fase di pianificazione. In particolare:
\begin{itemize}
  \item \textbf{Sottostima delle risorse necessarie per un'attività:} Durante lo \glossario{sprint} è stato attuato un cambio di tecnologie e inizialmente, dato che quest'ultime erano già state approfondite precedentemente il gruppo aveva preventivato che il passaggio sarebbe stato più graduale e l'integrazione tra \glossario{back-end} e \glossario{front-end} avrebbe richiesto meno risorse. Quando il team si è accorto il rischio è stato gestito correttamente dividendo il lavoro tra più persone e lasciando allo sprint successivo le task con una priorità più bassa e concentrando le risorse nello sviluppo del Proof of Concept;
  \item \textbf{Rischi relativi al preventivo:} Al termine del quarto \glossario{sprint} il gruppo ha ridistribuire le ore a disposizione a favore dell'amministratore. Questo è stato un modo per anticipare e arginare un possibile rischio futuro in quanto nel corso del tempo il team sarebbe potuto uscire dai limiti del budget staibiliti nel preventivo;
  \item \textbf{Rischi relativi alla rotazione dei ruoli:} Ruotare i ruoli è stato più semplice con l'avanzare degli \glossario{sprint} perchè quasi tutti i membri del gruppo avevano già affrontato il ruolo assegnatogli durante gli \glossario{sprint} precedenti. Inoltre come contromisura è stata applicato quello che era stato deciso nel quarto \glossario{sprint} ovvero che al cambio di mansioni fossero le persone con il ruolo uscente a dover stilare le task del ruolo per lo \glossario{sprint} successivo. Un'altra contromuisura applicata è stata la divisione delle mansioni dei componenti del team in 2 o 3 ruoli diversi in modo da tenere aggiornato il gruppo.
\end{itemize}


\subsubsection{Sprint 6: da 2024-06-15 a 2024-06-24}
\begin{minipage}{\textwidth}
Di seguito è riportata la distribuzione delle ore per ciascun membro del team, accumulate in totali per persona e per ruolo:
\begin{table}[H]
  \begin{tabularx}{\textwidth}{|c|*{6}{>{\centering}X|}c|}
    \hline
    \multicolumn{8}{|c|}{\textbf{Preventivo orario}} \\
    \hline
    \textbf{Membro del team} & \textbf{Re} & \textbf{Am} & \textbf{An} & \textbf{Pt} & \textbf{Pr} & \textbf{Ve} & \textbf{Totale per persona} \\
    \hline
    Cavalli Riccardo & 0 & 1 & 3 & 0 & 0 & 0 & 4 \\ 
    \hline
    Pianon Raul & 0 & 3 & 1 & 0 & 0 & 0 & 4 \\ 
    \hline
    Dall’Amico Martina & 2 & 0 & 0 & 0 & 2 & 0 & 4 \\ 
    \hline
    Cristo Marco & 0 & 1 & 2 & 0 & 0 & 1 & 4 \\ 
    \hline
    Lewental Sebastiano & 1 & 0 & 0 & 0 & 3 & 0 & 4 \\ 
    \hline
    Zecchinato Mattia & 0 & 1 & 0 & 0 & 2 & 2 & 5 \\ 
    \hline
    Stocco Tommaso & 0 & 1 & 0 & 0 & 2 & 2 & 5 \\ 
    \hline
    \textbf{Totale ore per ruolo} & 3 & 8 & 6 & 0 & 9 & 4 & \textbf{30} \\ 
    \hline
  \end{tabularx}
  \caption{Sprint 6 - Preventivo orario}
\end{table}
\end{minipage}

\begin{figure}[H]
  \centering
  \includegraphics[width=0.90\textwidth]{assets/Preventivo/Sprint-6/distribuzione_ore_risorsa_ruolo.pdf}
  \caption{6 - Istogramma della distribuzione oraria per la coppia risorsa-ruolo}
\end{figure}

\begin{figure}[H]
  \centering
  \includegraphics[width=0.90\textwidth]{assets/Preventivo/Sprint-6/distribuzione_ore_ruolo.pdf}
  \caption{Sprint 6 - Areogramma della distribuzione oraria per ruolo}
\end{figure}

\begin{minipage}{\textwidth}
Di seguito è riportato il preventivo economico del sesto \glossario{sprint}:
\begin{table}[H]
  \centering
  \begin{tabular}{|c|c|c|}
    \hline
    \multicolumn{3}{|c|}{\textbf{Preventivo economico}} \\
    \hline
    \textbf{Ruolo} & \textbf{Ore per ruolo} & \textbf{Costo (in \texteuro)} \\
    \hline
    Responsabile & 3 & 90,00 \\ 
    \hline
    Amministratore & 8 & 160,00 \\ 
    \hline
    Analista & 6 & 150,00 \\ 
    \hline
    Progettista & 0 & 0,00 \\ 
    \hline
    Programmatore & 9 & 135,00 \\ 
    \hline
    Verificatore & 4 & 60,00 \\ 
    \hline
    \textbf{Totale} & 30 & \textbf{595,00} \\ 
    \hline
  \end{tabular}
  \caption{Sprint 6 - Preventivo economico}
\end{table}
\end{minipage}
\subsection{Settimo sprint}

\begin{minipage}{\textwidth}
  Di seguito è riportata la distribuzione delle ore per ciascun membro del team, accumulate in totali per persona e per ruolo:
  \begin{table}[H]
    \begin{tabularx}{\textwidth}{|c|*{6}{>{\centering}X|}c|}
      \hline
      \multicolumn{8}{|c|}{\textbf{Consuntivo orario}} \\
      \hline
      \textbf{Membro del team} & \textbf{Re} & \textbf{Am} & \textbf{An} & \textbf{Pt} & \textbf{Pr} & \textbf{Ve} & \textbf{Totale per persona} \\
      \hline
      Cavalli Riccardo & 0 & 0 & 0 & 0 & 3 & 2 & 5 \\
      \hline
      Pianon Raul & 0 & 3 & 0 & 0 & 0 & 0 & 3 \\
      \hline
      Dall’Amico Martina & 0 & 2 & 0 & 0 & 0 & 1 & 3 \\
      \hline
      Cristo Marco & 0 & 2 & 0 & 0 & 0 & 1 & 3 \\
      \hline
      Lewental Sebastiano & 2 & 0 & 0 & 0 & 1 & 1 & 4 \\
      \hline
      Zecchinato Mattia & 2 & 3 & 0 & 0 & 0 & 0 & 5 \\
      \hline
      Stocco Tommaso & 0 & 4 & 0 & 0 & 0 & 0 & 4 \\
      \hline
      \textbf{Totale ore per ruolo} & 4 & 12 & 0 & 0 & 7 & 4 & \textbf{27} \\
      \hline
    \end{tabularx}
    \caption{Sprint 7 - Consuntivo orario}
  \end{table}
  \end{minipage}

  \begin{figure}[H]
    \centering
    \includegraphics[width=0.90\textwidth]{assets/Consuntivo/Sprint-7/distribuzione_ore_risorsa_ruolo.pdf}
    \caption{Sprint 7 - Istogramma della distribuzione oraria per la coppia risorsa-ruolo}
  \end{figure}

  \begin{figure}[H]
    \centering
    \includegraphics[width=0.90\textwidth]{assets/Consuntivo/Sprint-7/distribuzione_ore_ruolo.pdf}
    \caption{Sprint 7 - Areogramma della distribuzione oraria per ruolo}
  \end{figure}

  \begin{minipage}{\textwidth}
  Di seguito è riportato il consuntivo economico del settimo \glossario{sprint}:
  \begin{table}[H]
  \begin{adjustwidth}{-0.5cm}{-0.5cm}
    \centering
    \begin{tabular}{|P{2.9cm}|P{2.3cm}|P{2.5cm}|P{2.3cm}|>{\arraybackslash}P{2.5cm}|}
      \hline
      \multicolumn{5}{|c|}{\textbf{Consuntivo economico}} \\
      \hline
      \textbf{Ruolo} & \textbf{Ore per ruolo} & \textbf{Delta ore preventivo - consuntivo} & \textbf{Costo (in \texteuro)} & \textbf{Delta costo preventivo - consuntivo (in \texteuro)} \\
      \hline
      Responsabile & 4 & 0 & 120,00 & 0,00 \\ \hline
      Amministratore & 12 & -4 & 240,00 & -80,00 \\ \hline
      Analista & 0 & 1 & 0,00 & 25,00 \\ \hline
      Progettista & 0 & 0 & 0,00 & 0,00 \\ \hline
      Programmatore & 7 & 0 & 105,00 & 0,00 \\ \hline
      Verificatore & 4 & 4 & 60,00 & 60,00 \\ \hline
      \textbf{Totale} & \textbf{27} & 1 & \textbf{525,00} & 5,00 \\ \hline
      \textbf{Restante} & 353 & / & 6.995,00 & / \\ \hline
      \textbf{Sprint pregressi} & 266 & / & 5.500,00 & / \\ \hline
    \end{tabular}
    \caption{Sprint 7 - Consuntivo economico}
  \end{adjustwidth}
  \end{table}
  \end{minipage}

  \begin{figure}[H]
    \centering
    \includegraphics[width=0.90\textwidth]{assets/Consuntivo/Sprint-7/copertura_oraria.pdf}
    \caption{Sprint 7 - Areogramma del tempo speso (in ore) rispetto al totale}
  \end{figure}

  \begin{figure}[H]
    \centering
    \includegraphics[width=0.90\textwidth]{assets/Consuntivo/Sprint-7/budget_speso.pdf}
    \caption{Sprint 7 - Areogramma del budget speso rispetto al totale}
  \end{figure}

  \begin{minipage}{\textwidth}
    Di seguito sono riportate le ore rimanenti per la coppia risorsa-ruolo:
    \begin{table}[H]
      \begin{tabularx}{\textwidth}{|c|*{6}{>{\centering}X|}c|}
        \hline
        \multicolumn{8}{|c|}{\textbf{Ore rimanenti per la coppia risorsa-ruolo}} \\
        \hline
        \textbf{Membro del team} & \textbf{Re} & \textbf{Am} & \textbf{An} & \textbf{Pt} & \textbf{Pr} & \textbf{Ve} & \textbf{Totale per persona} \\
        \hline
        Cavalli Riccardo & 0 & 0 & 4 & 14 & 13 & 13 & 44 \\
        \hline
        Pianon Raul & 2 & 4 & 1 & 20 & 10 & 12 & 49 \\
        \hline
        Dall’Amico Martina & 5 & 2 & 1 & 14 & 16 & 12 & 50 \\
        \hline
        Cristo Marco & 2 & 7 & 1 & 17 & 10 & 13 & 50 \\
        \hline
        Lewental Sebastiano & 4 & 4 & 2 & 11 & 16 & 14 & 51 \\
        \hline
        Zecchinato Mattia & 7 & 3 & 3 & 11 & 14 & 14 & 52 \\
        \hline
        Stocco Tommaso & 5 & 0 & 3 & 20 & 9 & 18 & 55 \\
        \hline
        \textbf{Totale ore per ruolo} & 25 & 21 & 15 & 107 & 89 & 96 & \textbf{353} \\
        \hline
      \end{tabularx}
      \caption{Sprint 7 - Ore rimanenti per la coppia risorsa-ruolo}
    \end{table}
  \end{minipage}

\subsubsection{Revisione delle attività}

Nell'arco del settimo \glossario{sprint}, il team ha svolto le seguenti attività:
\begin{itemize}
  \item Revisione consuntivi \PdP;
  \item Aggiunta metriche alle \NdP;
  \item Completamento sezioni incomplete \NdP;
  \item Stesura sezione test nel \PdQ;
  \item Stesura dei verbali interni;
  \item Gestione stato Utente/Tecnico a \glossario{front-end};
  \item Creazione interfaccia Chat a \glossario{front-end};
  \item Implementazione funzioni per generare il \glossario{prompt} e mostrarlo a \glossario{front-end}.
\end{itemize}

\subsubsection{Retrospettiva}

\par Di seguito sono riportati i risultati del questionario di valutazione dello \glossario{sprint}:
\begin{itemize}
  \item Organizzazione dello \glossario{sprint}\ - Valutazione: 8;
  \item Conduzione dei meeting interni - Valutazione: 8;
  \item Conduzione dei meeting esterni - Valutazione: 8;
  \item Impegno e partecipazione dei singoli membri - Valutazione: 3;
  \item La quasi totalità dei membri del team era a conoscenza delle proprie mansioni;
  \item La numerosità delle riunioni è risultata adeguata per tutti i membri del gruppo;
  \item Le riunioni sono state organizzate quasi sempre con il giusto preavviso;
  \item Il rapporto ore spese/ore produttive si sta notevolmente equilibrando;
  \item La produttività è stata ragioneole considerando le criticità della sessione;
  \item È diffusa l'idea di programmare incontri in presenza più frequenti.
\end{itemize}

\vspace{0.5\baselineskip}
\par A seguire le \textbf{analisi a posteriori} del settimo \glossario{sprint}:
\begin{itemize}
  \item La sessione di esami ha impattato sulla quantità di lavoro svolto, come previsto.
\end{itemize}

\subsubsection{Aggiornamento pianificazione e preventivo}
\par Il team ha definito un piano d'azione per migliorare l'organizzazione e la produttività del prossimo \glossario{sprint}:
\begin{itemize}
  \item Continuare con la pianificazione di un incontri in presenza;
\end{itemize}

\paragraph*{Pianificazione futura:}
\par Si è deciso di mantenere moderata la quantità di ore assegnate a ciascun membro del team, in modo da continuare a garantire un equilibrio tra lavoro e studio.

\paragraph*{Preventivo "a finire" (\sezione{sec:stima_temporale}):}
\par L'avanzamento dello stato prodotto porta a stabilire la prossimità della revisione RTB, che avverrà tuttavia dopo la sessione di esami.

\paragraph*{Gestione dei rischi (\sezione{sec:analisi_rischi}):}
\par Nel corso del settimo \glossario{sprint}, il seguente rischio si è presentato ed è stato gestito correttamente:
\begin{itemize}
  \item \textbf{Rischi relativi a rallentamenti:} durante la sessione di esame, sono state ridotte le ore assegnate ad ogni membro per favorire lo studio; essendo un riscio previsto, la data di consegna era stata pianificata in modo adeguato.
\end{itemize}

\subsection{Sprint 8: da 2024-07-03 a 2024-07-10}
\par In preparazione alla revisione \glossario{RTB}, il team ha fissato come obiettivo principale l’aggiornamento e la revisione della documentazione prodotta fino allo sprint corrente. Per quanto riguarda i singoli documenti, l’attività prioritaria è la stesura finale del cruscotto di valutazione della qualità nel \PdQ. Inoltre, il gruppo presenterà il \glossario{PoC} alla Proponente.

\subsubsection{Obiettivi}
\begin{itemize}
  \item Aggiornamento della documentazione in preparazione alla prima fase della revisione \glossario{RTB};
  \item Ultimazione del \glossario{PoC};
  \item Revisione delle metriche di processo e di prodotto;
  \item Suddivisione delle metriche in base all'obiettivo.
\end{itemize}

\begin{figure}[H]
  \centering
  \includegraphics[width=0.90\textwidth]{assets/Pianificazione/Sprint-8/gantt.png}
  \caption{Sprint 8 - Diagramma di Gantt}\label{fig:sprint-8-gantt}
\end{figure}

\subsection{Nono sprint}

\begin{minipage}{\textwidth}
  Di seguito è riportata la distribuzione delle ore per ciascun membro del team, accumulate in totali per persona e per ruolo:
  \begin{table}[H]
    \begin{tabularx}{\textwidth}{|c|*{6}{>{\centering}X|}c|}
      \hline
      \multicolumn{8}{|c|}{\textbf{Consuntivo orario}} \\
      \hline
      \textbf{Membro del team} & \textbf{Re} & \textbf{Am} & \textbf{An} & \textbf{Pt} & \textbf{Pr} & \textbf{Ve} & \textbf{Totale per persona} \\
      \hline
      Riccardo Cavalli & 0 & 0 & 0 & 2 & 1 & 2 & 5 \\
      \hline
      Raul Pianon & 0 & 2 & 0 & 1 & 0 & 1 & 4 \\
      \hline
      Martina Dall'Amico & 1 & 0 & 0 & 0 & 0 & 3 & 4 \\
      \hline
      Marco Cristo & 0 & 2 & 0 & 0 & 0 & 2 & 4 \\
      \hline
      Sebastiano Lewental & 2 & 0 & 0 & 0 & 0 & 2 & 4 \\
      \hline
      Mattia Zecchinato & 0 & 0 & 0 & 2 & 0 & 2 & 4 \\
      \hline
      Tommaso Stocco & 0 & 0 & 0 & 0 & 1 & 3 & 4 \\
      \hline
      \textbf{Totale ore per ruolo} & 3 & 4 & 0 & 6 & 1 & 15 & \textbf{29} \\
      \hline
    \end{tabularx}
    \caption{Sprint 9 - Consuntivo orario}
  \end{table}
  \end{minipage}

  \begin{figure}[H]
    \centering
    \includegraphics[width=0.90\textwidth]{assets/Consuntivo/Sprint-9/distribuzione_ore_risorsa_ruolo.pdf}
    \caption{Sprint 9 - Istogramma della distribuzione oraria per la coppia risorsa-ruolo}
  \end{figure}

  \begin{figure}[H]
    \centering
    \includegraphics[width=0.90\textwidth]{assets/Consuntivo/Sprint-9/distribuzione_ore_ruolo.pdf}
    \caption{Sprint 9 - Areogramma della distribuzione oraria per ruolo}
  \end{figure}

  \begin{minipage}{\textwidth}
  Di seguito è riportato il consuntivo economico del nono \glossario{sprint}:
  \begin{table}[H]
  \begin{adjustwidth}{-0.5cm}{-0.5cm}
    \centering
    \begin{tabular}{|P{2.9cm}|P{2.3cm}|P{2.5cm}|P{2.3cm}|>{\arraybackslash}P{2.5cm}|}
      \hline
      \multicolumn{5}{|c|}{\textbf{Consuntivo economico}} \\
      \hline
      \textbf{Ruolo} & \textbf{Ore per ruolo} & \textbf{Delta ore preventivo - consuntivo} & \textbf{Costo (in \texteuro)} & \textbf{Delta costo preventivo - consuntivo (in \texteuro)} \\
      \hline
      \Responsabile[U]{} & 3 & 0 & 90,00 & 0,00 \\ \hline
      \Amministratore[U]{} & 4 & 0 & 80,00 & 0,00 \\ \hline
      \Analista[U]{} & 0 & 0 & 0,00 & 0,00 \\ \hline
      \Progettista[U]{} & 6 & -1 & 150,00 & -25,00 \\ \hline
      \Programmatore[U]{} & 1 & 0 & 15,00 & 0,00 \\ \hline
      \Verificatore[U]{} & 15 & 0 & 225,00 & 0,00 \\ \hline
      \textbf{Totale} & \textbf{29} & -1 & \textbf{560,00} & -25,00 \\ \hline
      \textbf{Restante} & 282 & / & 5.710,00 & / \\ \hline
      \textbf{Sprint pregressi} & 335 & / & 6.750,00 & / \\ \hline
    \end{tabular}
    \caption{Sprint 9 - Consuntivo economico}
  \end{adjustwidth}
  \end{table}
  \end{minipage}

  \begin{figure}[H]
    \centering
    \includegraphics[width=0.90\textwidth]{assets/Consuntivo/Sprint-9/copertura_oraria.pdf}
    \caption{Sprint 9 - Areogramma del tempo speso (in ore) rispetto al totale}
  \end{figure}

  \begin{figure}[H]
    \centering
    \includegraphics[width=0.90\textwidth]{assets/Consuntivo/Sprint-9/budget_speso.pdf}
    \caption{Sprint 9 - Areogramma del budget speso rispetto al totale}
  \end{figure}

  \begin{minipage}{\textwidth}
    Di seguito sono riportate le ore rimanenti per la coppia risorsa-ruolo:
    \begin{table}[H]
      \begin{tabularx}{\textwidth}{|c|*{6}{>{\centering}X|}c|}
        \hline
        \multicolumn{8}{|c|}{\textbf{Ore rimanenti per la coppia risorsa-ruolo}} \\
        \hline
        \textbf{Membro del team} & \textbf{Re} & \textbf{Am} & \textbf{An} & \textbf{Pt} & \textbf{Pr} & \textbf{Ve} & \textbf{Totale per persona} \\
        \hline
        Riccardo Cavalli & 0 & 1 & 3 & 12 & 10 & 9 & 35 \\
        \hline
        Raul Pianon & 2 & 1 & 1 & 19 & 10 & 6 & 39 \\
        \hline
        Martina Dall'Amico & 2 & 1 & 1 & 14 & 16 & 8 & 42 \\
        \hline
        Marco Cristo & 2 & 4 & 0 & 17 & 10 & 7 & 40 \\
        \hline
        Sebastiano Lewental & 3 & 4 & 1 & 11 & 11 & 11 & 41 \\
        \hline
        Mattia Zecchinato & 5 & 2 & 3 & 9 & 11 & 11 & 41 \\
        \hline
        Tommaso Stocco & 5 & 0 & 3 & 19 & 9 & 8 & 44 \\
        \hline
        \textbf{Totale ore per ruolo} & 19 & 13 & 12 & 101 & 77 & 60 & \textbf{282} \\
        \hline
      \end{tabularx}
      \caption{Sprint 9 - Ore rimanenti per la coppia risorsa-ruolo}
    \end{table}
  \end{minipage}

\subsubsection{Revisione delle attività}

Nell'arco del nono \glossario{sprint}, il team ha svolto le seguenti attività:
\begin{itemize}
  \item Test su database locale;
  \item Aggiornamento delle \NdP;
  \item Estensione sezioni delle \NdP;
  \item Stesura iniziale del documento di Specifiche Tecniche;
  \item Aggiornamento del \PdP;
  \item Revisione delle \NdP;
  \item Revisione del \PdP;
  \item Stesura e aggiornamento dei verbali interni;
\end{itemize}

\subsubsection{Retrospettiva}

\par Di seguito sono riportati i risultati del questionario di valutazione dello \glossario{sprint}:
\begin{itemize}
  \item Organizzazione dello sprint - Valutazione: 8;
  \item Conduzione dei meeting interni - Valutazione: 8;
  \item Conduzione dei meeting esterni - Valutazione: 8;
  \item Impegno e partecipazione dei singoli membri - Valutazione: 2.5;
  \item Tutti i membri del gruppo sapevano cosa fare nel loro ruolo;
  \item La numerosità delle riunioni è risultata adeguata per quasi tutti i membri del gruppo;
  \item Le riunioni sono state organizzate sempre con il giusto preavviso;
  \item Il rapporto ore spese/ore produttive è sbilanciato a causa della sessione d'esami;
  \item La produttività è stata comunque ragionevole considerando le criticità della sessione;
\end{itemize}

\vspace{0.5\baselineskip}
\par A seguire le \textbf{analisi a posteriori} del nono \glossario{sprint}:
\begin{itemize}
  \item Durante questo \glossario{sprint} il team si è concentrato sul completamento delle attività mancanti in vista dell'incontro con il Professor Riccardo Cardin per la revisione \RTB. Dal prossimo \glossario{sprint}, il gruppo si concentrerà sulla conclusione delle attività rimaste per il completamento della documentazione in vista della seconda parte della revisione. In seguito a ciò, si inizieranno anche i lavori per la progettazione logica e architetturale del prodotto.
  \item La sessione d'esami ha impattato notevolmente sulla quantità di lavoro svolto, come previsto.
\end{itemize}

\subsubsection{Aggiornamento pianificazione e preventivo}
\par Il team ha definito un piano d'azione per migliorare l'organizzazione e la produttività del prossimo \glossario{sprint}:
\begin{itemize}
  \item Il team ha optato per l'organizzazione di \glossario{sprint} della durata di due settimane.
\end{itemize}

\paragraph*{Pianificazione futura:}
\par Conclusa la fase di \RTB e terminata la sessione estiva d'esami, il gruppo ha deciso di tornare alla programmazione di \glossario{sprint} di due settimane, così da favorire un progresso più flessibile e meno restrittivo; in questo modo sarà possibile mantenere un focus maggiore sulla progettazione, garantendo comunque il tempo necessario per un'accurata e completa redazione della documentazione.

\paragraph*{Preventivo "a finire" (\sezione{sec:stima_temporale}):}
\par Effettuata la prima presentazione della revisione RTB, il gruppo rimane in attesa dell'esito da parte del Committente, preparando nel frattempo la documentazione necessaria ad affrontare la PB. Si mantiene un carico di lavoro equilibrato, tenendo sempre in considerazione il periodo corrente di sessione d'esami.

\paragraph*{Gestione dei rischi (\sezione{sec:analisi_rischi}):}
\par Durante il nono \glossario{sprint}, i seguenti rischi sono stati gestiti con successo:
\begin{itemize}
  \item \textbf{RO1 - Periodi di rallentamento}: Durante l'ultima fase della sessione di esami, le ore di lavoro assegnate a ciascun membro sono state calcolate in base agli impegni universitari, pianificando le attività di conseguenza. Una volta conclusa la sessione, il gruppo ha organizzato delle riunioni dedicate allo svolgimento delle attività, incrementando così la produttività.
\end{itemize}
\subsection{Sprint 10: da 2024-07-22 a 2024-08-04}
\par Durante lo sprint, il gruppo si occuperà di verificare gli ultimi documenti da rilasciare e, in seguito, di avviare la progettazione architetturale del software.

\subsubsection{Obiettivi}
\begin{itemize}
  \item Aggiornamento delle convenzioni per lo stile di codifica nelle \NdP;
  \item Verifica e rilascio dei seguenti documenti:
  \begin{itemize}
    \item \NdP;
    \item \PdP;
    \item \AdR{} (post colloquio \RTB);
    \item \PdQ.
  \end{itemize}
  \item Avvio della progettazione architetturale;
  \item Individuazione di stili e pattern architetturali per il back-end e il front-end;
  \item Prima stesura del documento di \ST{}.
\end{itemize}

\begin{figure}[H]
  \centering
  \includegraphics[width=0.90\textwidth]{assets/Pianificazione/Sprint-10/gantt.png}
  \caption{Sprint 10 - Diagramma di Gantt}\label{fig:sprint-10-gantt}
\end{figure}


