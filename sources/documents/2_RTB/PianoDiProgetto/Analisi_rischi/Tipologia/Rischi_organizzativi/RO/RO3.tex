\subsubsection{RO3: Sottostima delle risorse necessarie per un'attività}
\begin{itemize}
    \item \textbf{Probabilità:} Alta;
    \item \textbf{Grado di criticità:} Alto;
    \item \textbf{Descrizione:} La complessità del progetto e l'inesperienza del gruppo potrebbero portare a una sottostima delle risorse, economiche e temporali, necessarie per completare determinate attività;
    \item \textbf{Stragie di rilevamento:} In caso di valutazioni errate o difficoltà nella gestione del carico di lavoro, ciascun componente deve informare tempestivamente il resto del gruppo. La discussione delle criticità può avvenire sia nella chat Telegram che durante le riunioni interne. Con questa procedura, il team ritiene di poter rilevare i problemi celermente e di poter mitigare gli impatti negativi sui task successivi;
    \item \textbf{Contromisure:} Se lo sforzo produttivo necessario per completare un task risulta sottostimato, l'attività viene suddivisa in sotto-task. Ciascuna sotto-attività può essere assegnata a membri diversi del team, al fine di portare a termine l’incarico col minor ritardo possibile. Inoltre, chi ha esperienza si impegna a fornire assistenza per minimizzare l'impatto sui task successivi. In alternativa, un’attività non particolarmente urgente può essere suddivisa in sotto-task con priorità diversa. Così facendo il team può prolungare la finestra temporale di esecuzione del task, concentrandosi sulle sotto-attività a priorità più alta.
\end{itemize}
