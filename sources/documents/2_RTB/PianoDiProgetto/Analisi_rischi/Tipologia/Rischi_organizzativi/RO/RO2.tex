\subsubsection{RO2: Collaborazione}
\begin{itemize}
    \item \textbf{Probabilità}: Bassa;
    \item \textbf{Grado di criticità}: Alto;
    \item \textbf{Descrizione}: Idee, metodologie e tempistiche di lavoro potrebbero non essere allineate tra i vari membri del gruppo, con conseguenti divergenze da sanare per evitare conflitti interni;
    \item \textbf{Stragie di rilevamento}: Il \Responsabile{} in carica ha il compito di monitorare le attività, specialmente quelle assegnate a più membri del team. Inoltre, il \Responsabile{} deve mantenere un dialogo costante con il gruppo al fine di rilevare potenziali divergenze di pensiero;
    \item \textbf{Contromisure}: Prima di pianificare attività sensibili o intraprendere azioni correttive, il \Responsabile{} deve raccogliere i feedback del team. In caso di divergenze, le discussioni avverranno durante le riunioni, in modo tale che ciascun componente possa esporre la sua opinione in modo chiaro e costruttivo. Per ridurre al minimo i possibili contrasti, il gruppo si impegna a definire un \WoW\ condiviso e a formalizzarlo nelle \NdP.
\end{itemize}
