\noindent\begin{minipage}{\textwidth}
\subsubsection{RO5: Rischi relativi al preventivo}
    
\bgroup
\begin{adjustwidth}{-0.5cm}{-0.5cm}
    % MAX 12.5cm
    \begin{longtable}{P{4.5cm}|>{\justifying \arraybackslash}P{9cm}}

        \textbf{Probabilità} & Media \\
        \hline
        \textbf{Grado di criticità} & Alto \\
        \hline
        \textbf{Descrizione} & Forte variazione tra preventivo e consuntivo, con relativo aumento
        dei costi. \\
        \hline
        \textbf{Strategie di rilevamento} &  Controllo periodico dello stato di avanzamento delle
        attività, rendicontazione delle ore tramite una tabella condivisa che riesca ad indicare le ore produttive svolte nell'attività. \\
        \hline
        \textbf{Contromisure} & Preventivare le attività tenendo conto di una fase di allentamento. Facendo ciò si cercano di anticipare e prevenire eventuali imprevisti, rimanendo più laschi sul preventivo iniziale. 
    \end{longtable}
\end{adjustwidth}
\egroup
\end{minipage}