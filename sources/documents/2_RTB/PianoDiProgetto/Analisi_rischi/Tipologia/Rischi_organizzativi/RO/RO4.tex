\subsubsection{RO4: Rotazione dei ruoli}
\begin{itemize}
    \item \textbf{Probabilità}: Alta;
    \item \textbf{Grado di criticità}: Alto;
    \item \textbf{Descrizione}: Ciascun componente del gruppo è tenuto ad assumere, a rotazione, tutti i ruoli previsti dal regolamento del progetto. Questa pratica differisce dall'approccio abitualmente adottato nella realizzazione di progetti didattici. Pertanto, esiste la possibilità che i membri del team riscontrino delle difficoltà all’inizio di ogni sprint. L’impatto di tale rischio può risultare negativo sul lungo periodo, quando ciascun componente dovrà prendere in carico il lavoro (sempre più voluminoso) svolto da altri membri del gruppo;
    \item \textbf{Stragie di rilevamento}: Confronto costante attraverso i canali di comunicazione stabiliti nelle \NdP. Per rilevare eventuali rallentamenti dovuti alla rotazione dei ruoli, il \Responsabile{} può consultare la tabella di rendicontazione delle ore;
    \item \textbf{Contromisure}: Nelle \NdP\ sono elencati i compiti assegnati a ciascun ruolo in base al regolamento del progetto. Durante le riunioni di pianificazione e retrospettiva, le mansioni generali (definite dal \WoW) vengono integrate con le attività specifiche relative allo sprint corrente. In aggiunta, il team organizza brevi riunioni dedicate alla rotazione dei ruoli, alle quali partecipano unicamente i membri interessati.
\end{itemize}
