\subsubsection{RO4: Rischi relativi alla rotazione dei ruoli}
\begin{itemize}
    \item \textbf{Probabilità:} Alta;
    \item \textbf{Grado di criticità:} Alto;
    \item \textbf{Descrizione:} Ciascun componente del gruppo deve assumere, a turno, tutti i
    ruoli del progetto. Come previsto, alcuni ruoli rappresentano un’assoluta novità rispetto all’approccio comunemente adottato per la realizzazione di progetti
    didattici. C’è un’alta probabilità che i membri del gruppo possano trovarsi in
    difficoltà nello svolgere compiti inediti, soprattutto quelli di natura amministrativa e gestionale fondamentali per monitorare e garantire l’avanzamento del
    progetto;
    \item \textbf{Stragie di rilevamento:} Confronto continuo attraverso i canali di comunicazione stabiliti nel way of working;
    \item \textbf{Contromisure:} Organizzazione di riunioni interne all’inizio di ogni sprint per
    anticipare e risolvere eventuali dubbi e ostacoli nelle attività critiche. I membri
    più esperti si impegnano a condividere le loro competenze con il resto del team. Infine il documento di \NdP\ permette di informare i membri entranti nel nuovo ruolo su cosa questo si fondi.
\end{itemize}
