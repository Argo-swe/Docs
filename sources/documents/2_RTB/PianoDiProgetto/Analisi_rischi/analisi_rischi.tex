\section{Analisi dei rischi}
\label{sec:analisi_rischi}

\subsection{Ultimo aggiornamento: 2024-07-25}

\par In questa sezione vengono esaminate le situazioni problematiche che possono verificarsi durante lo svolgimento del progetto. Ciascun rischio è corredato da:
\begin{itemize}
  \item Probabilità di occorrenza;
  \item Grado di criticità;
  \item Descrizione;
  \item Strategie di rilevamento;
  \item Contromisure.
\end{itemize}

\vspace{0.5\baselineskip}
\par La notazione utilizzata per identificare i rischi è riportata nelle \NormeDiProgetto\ (§Gestione dei rischi). L'analisi dei rischi influenza la pratica di gestione dei rischi, che è documentata nel consuntivo di periodo (\sezione{sec:consuntivo}). La gestione dei rischi fornisce un riscontro in merito a:
\begin{itemize}
  \item Occorrenza effettiva di un rischio;
  \item Attuazione delle misure di mitigazione previste;
  \item Valutazione dell'impatto del rischio. 
\end{itemize}

\vspace{0.5\baselineskip}
\par Dalla gestione dei rischi deriva una valutazione critica dell’efficacia delle misure di mitigazione previste e attuate, che porta all’aggiornamento e al miglioramento dell’analisi iniziale.

\subsection{Rischi tecnologici}

\subsubsection{RT1: Scarso know-how tecnologico}
\begin{itemize}
    \item \textbf{Probabilità:} Alta;
    \item \textbf{Grado di criticità:} Alto;
    \item \textbf{Descrizione:} Utilizzo di tecnologie sconosciute. Nessun membro del gruppo ha
    esperienze pregresse con gli strumenti e le librerie suggerite dalla Proponente; di conseguenza, l’avanzamento del progetto rischia di subire rallentamenti dovuti a fasi di apprendimento delle nuove tecnologie.
    Con questo si intendono anche fasi di esplorazione di tecnologie che possono risultare più vantaggiose rispetto a quelle utilizzate al momento;
    \item \textbf{Stragie di rilevamento:} Il primo passo consiste in una serie di incontri, di breve durata, atti a valutare le competenze tecniche e l'esperienza del team su ciascuna tecnologia. Inoltre, è necessaria un'analisi collaborativa per stimare la curva di apprendimento. A questo si aggiunge poi una valutazione delle risorse disponibili, specialmente quelle temporali.
    \begin{itemize}
        \item Nel corso dello \glossario{sprint}, il responsabile si impegna a monitorare constantemente le attività e raccogliere feedback dai singoli membri;
        \item Come strategia di rilevamento, il team ha introdotto anche la \glossario{continuous integration}. Tale pratica prevede un allineamento frequente con l'ambiente condiviso (\glossario{GitHub}) e consente al gruppo di individuare eventuali difficoltà nell'uso delle tecnolgoie e prevenire la propagazione degli errori. 
    \end{itemize}
    \item \textbf{Contromisure:} Considerando l'inesperienza del gruppo e la continua evoluzione delle tecnologie proposte, il rischio tecnologico non può essere totalmente scongiurato. Tuttavia, il team intende lavorare per mitigare i problemi ed evitare rallentamenti sfavorevoli. La prima contromisura prevede lo studio individuale da parte di un gruppo ristretto di risorse, così da non dover sospendere le attività in corso e non pregiudicare l'avanzamento dello \glossario{sprint}. Una volta ultimato l'approfondimento di una determinata tecnologia, i membri interessati terranno un workshop per uniformare le conoscenze del gruppo. Di seguito sono elencate altre misure di mitigazione:
    \begin{itemize}
        \item 
    \end{itemize}
    \par In caso di difficoltà, il team valuterà la possibilità di impiegare tecnologie alternative, o di supporto, che possano incrementare l'\glossario{efficienza} del progetto.
\end{itemize}

\newpage
\subsubsection{RT2: Malfunzionamenti hardware}
\begin{itemize}
    \item \textbf{Probabilità}: Bassa;
    \item \textbf{Grado di criticità}: Basso;
    \item \textbf{Descrizione:} Possibili malfunzionamenti hardware delle macchine dei membri del team;
    \item \textbf{Stragie di rilevamento}: Controllo ripetuto da parte di ciascun componente del gruppo sulla propria strumentazione e segnalazione tempestiva in casi di guasto;
    \item \textbf{Contromisure}: Svolgimento del lavoro con \glossario{integrazione continua} su sistemi di versionamento, al fine di avere un ambiente condiviso e limitare la perdita di informazioni.  
\end{itemize}

\subsection{Rischi organizzativi}

\subsubsection{RO1: Rischi relativi a rallentamenti}
\begin{itemize}
    \item \textbf{Probabilità:} Alta;
    \item \textbf{Grado di criticità:} Basso;
    \item \textbf{Descrizione:} Durante il corso del progetto saranno presenti periodi di rallentamento dovuti a fattori esterni (giorni festivi, impegni studenteschi) o altri al momento non prevedibili;
    \item \textbf{Stragie di rilevamento:} Monitoraggio continuo delle ore dedicate dai membri del gruppo; controllo del calendario per verificare la vicinanza a date particolari (tra cui quelle indicate in tabella);
    \item \textbf{Contromisure:} La stima della data di consegna è stata adattata tenendo conto di questo rischio.
\end{itemize}

\noindent\begin{minipage}{\textwidth}
\bgroup
\begin{adjustwidth}{-0.5cm}{-0.5cm}
    \begin{longtable}{|P{4cm}|P{4cm}|>{\arraybackslash}P{4cm}|}
        \hline
        \textbf{Periodo} & \textbf{Da} & \textbf{A} \\
        \hline
        Pasquale & 2024-03-29 & 2024-04-02 \\
        \hline
        Ponte 25 Aprile & 2024-04-25 & 2024-04-28 \\
        \hline
        Sessione estiva & 2024-06-17 & 2024-07-20 \\
        \hline 
        Sessione autunnale & 2024-08-19 & Termine progetto \\
        \hline
    \end{longtable}
\end{adjustwidth}
\egroup
\end{minipage}


\subsection{Rischi di natura personale}

\subsubsection{RP1: Questioni personali}
\begin{itemize}
    \item \textbf{Probabilità:} Bassa;
    \item \textbf{Grado di criticità:} Alto;
    \item \textbf{Descrizione:} La necessità di un periodo di fermo per questioni personali, seppur giustificate, potrebbe condurre il team a una fase di stallo;
    \item \textbf{Stragie di rilevamento:} Comunicazione tempestiva da parte dei soggetti interessati;
    \item \textbf{Contromisure:} Il team si impegna a ridistribuire i task assegnati ai membri indisponbili previa individuazione e proroga di attività ritenute meno urgenti. In caso di necessità, il gruppo attuerà un'opportuna riallocazione delle risorse temporali. I soggetti interessati saranno tenuti ad allineare le proprie ore produttive con il resto del gruppo.
\end{itemize}