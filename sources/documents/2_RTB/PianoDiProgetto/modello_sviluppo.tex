\section{Modello di sviluppo}
\label{sec:modello_sviluppo}

Il gruppo ha optato per l'adozione del modello \glossario{Agile} poiché consente larga flessibilità e reattività di fronte ai cambiamenti, accogliendo nuove esigenze anche nelle fasi avanzate dello sviluppo. Per questo motivo, sono stati selezionati degli eventi chiave del framework \glossario{Scrum}, ragionando in periodi a tempo determinato, generalmente di due settimane, detti \glossario{sprint}, organizzati come segue:
\begin{itemize}
\item \textbf{Sprint Planning}: all'inizio di ogni sprint, il team seleziona, dal \glossario{backlog}, un elenco ordinato di task da portare a termine entro la fine dell'iterazione. Le attività possono essere a loro volta suddivise in sotto-task individuali o condivisi tra più membri del gruppo;
\item \textbf{Sprint Review}: Al termine di ciascuno sprint, il team illustra al cliente il lavoro svolto nel corso dell'iterazione, comunicando eventuali dubbi e difficoltà, raccogliendo feedback costruttivi e identificando le attività di maggior valore per la \glossario{Proponente};
\item \textbf{Sprint Retrospective}: Il team valuta a posteriori lo \glossario{sprint} e definisce un piano d'azione per migliorare l'organizzazione e la produttività dei periodi immediatamente successivi.
\end{itemize}
A seguito di ciò, il responsabile del gruppo stabilisce un'opportuna rotazione dei ruoli per lo sprint successivo, tenendo in considerazione il \glossario{product backlog} e la disponibilità oraria dei singoli membri.

Questo approccio consente inoltre di interagire in modo collaborativo e trasparente con il cliente, assicurando che i suoi feedback, ottenuti a seguito dello svolgimento di opportune riunioni, vengano integrati rapidamente, migliorando così la qualità e la conformità del software sviluppato.
La metodologia impiegata promuove quindi un ambiente di sviluppo continuo e iterativo, permettendo al team di analizzare regolarmente i processi adottati e di apportare miglioramenti costanti.
