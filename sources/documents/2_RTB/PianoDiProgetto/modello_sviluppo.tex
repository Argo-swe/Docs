\section{Modello di sviluppo}
\label{sec:modello_sviluppo}

Il gruppo ha optato per l'adozione del modello \glossario{Agile} poiché consente larga flessibilità e reattività di fronte ai cambiamenti, accogliendo nuove esigenze anche nelle fasi avanzate dello sviluppo. Per questo motivo, sono stati selezionati degli eventi chiave del framework \glossario{Scrum}, ragionando in periodi a tempo determinato, generalmente di due settimane, detti \glossario{sprint}, organizzati come segue:
\begin{itemize}
    \item \textbf{Sprint Planning}: All'inizio di ogni sprint, il team pianifica il lavoro che sarà completato e la durata dello  sprint stesso, suddividendolo in task individuali o condivise tra più membri a scadenza fissata;
    \item \textbf{Sprint Review}: A metà e al termine di ogni sprint, il team rivede il lavoro portato a termine e si raccolgono le considerazioni di ciascun membro a riguardo;
    \item \textbf{Sprint Retrospective}: Il team riflette sullo sprint precedente e identifica i modi per migliorare in futuro.
\end{itemize}
A seguito di ciò, il Responsabile del gruppo stabilisce un'opportuna rotazione dei ruoli assegnati a ciascun membro del team per lo sprint successivo.

Questo approccio permette inoltre di collaborare strettamente con i clienti, assicurando che i loro feedback, ottenuti a seguito dello svolgimento di opportune riunioni, vengano integrati rapidamente, migliorando così la qualità e la pertinenza del software sviluppato. 
La metodologia impiegata promuove quindi un ambiente di sviluppo continuo e iterativo, permettendo al team di riflettere regolarmente sui processi adottati e di apportare miglioramenti costanti.