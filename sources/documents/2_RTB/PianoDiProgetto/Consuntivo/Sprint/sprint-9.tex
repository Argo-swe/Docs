\subsection{Nono sprint}

\begin{minipage}{\textwidth}
  Di seguito è riportata la distribuzione delle ore per ciascun membro del team, accumulate in totali per persona e per ruolo:
  \begin{table}[H]
    \begin{tabularx}{\textwidth}{|c|*{6}{>{\centering}X|}c|}
      \hline
      \multicolumn{8}{|c|}{\textbf{Consuntivo orario}} \\
      \hline
      \textbf{Membro del team} & \textbf{Re} & \textbf{Am} & \textbf{An} & \textbf{Pt} & \textbf{Pr} & \textbf{Ve} & \textbf{Totale per persona} \\
      \hline
      Cavalli Riccardo & 0 & 0 & 0 & 2 & 1 & 2 & 5 \\
      \hline
      Pianon Raul & 0 & 2 & 0 & 1 & 0 & 1 & 4 \\
      \hline
      Dall’Amico Martina & 1 & 0 & 0 & 0 & 0 & 3 & 4 \\
      \hline
      Cristo Marco & 0 & 2 & 0 & 0 & 0 & 2 & 4 \\
      \hline
      Lewental Sebastiano & 2 & 0 & 0 & 0 & 0 & 2 & 4 \\
      \hline
      Zecchinato Mattia & 0 & 0 & 0 & 2 & 0 & 2 & 4 \\
      \hline
      Stocco Tommaso & 0 & 0 & 0 & 0 & 1 & 3 & 4 \\
      \hline
      \textbf{Totale ore per ruolo} & 3 & 4 & 0 & 6 & 1 & 15 & \textbf{29} \\
      \hline
    \end{tabularx}
    \caption{Sprint 9 - Consuntivo orario}
  \end{table}
  \end{minipage}

  \begin{figure}[H]
    \centering
    \includegraphics[width=0.90\textwidth]{assets/Consuntivo/Sprint-9/distribuzione_ore_risorsa_ruolo.pdf}
    \caption{Sprint 9 - Istogramma della distribuzione oraria per la coppia risorsa-ruolo}
  \end{figure}

  \begin{figure}[H]
    \centering
    \includegraphics[width=0.90\textwidth]{assets/Consuntivo/Sprint-9/distribuzione_ore_ruolo.pdf}
    \caption{Sprint 9 - Areogramma della distribuzione oraria per ruolo}
  \end{figure}

  \begin{minipage}{\textwidth}
  Di seguito è riportato il consuntivo economico del nono \glossario{sprint}:
  \begin{table}[H]
  \begin{adjustwidth}{-0.5cm}{-0.5cm}
    \centering
    \begin{tabular}{|P{2.9cm}|P{2.3cm}|P{2.5cm}|P{2.3cm}|>{\arraybackslash}P{2.5cm}|}
      \hline
      \multicolumn{5}{|c|}{\textbf{Consuntivo economico}} \\
      \hline
      \textbf{Ruolo} & \textbf{Ore per ruolo} & \textbf{Delta ore preventivo - consuntivo} & \textbf{Costo (in \texteuro)} & \textbf{Delta costo preventivo - consuntivo (in \texteuro)} \\
      \hline
      Responsabile & 3 & 0 & 90,00 & 0,00 \\ \hline
      Amministratore & 4 & 0 & 80,00 & 0,00 \\ \hline
      Analista & 0 & 0 & 0,00 & 0,00 \\ \hline
      Progettista & 6 & -1 & 150,00 & -25,00 \\ \hline
      Programmatore & 1 & 0 & 15,00 & 0,00 \\ \hline
      Verificatore & 15 & 0 & 225,00 & 0,00 \\ \hline
      \textbf{Totale} & \textbf{29} & -1 & \textbf{560,00} & -25,00 \\ \hline
      \textbf{Restante} & 296 & / & 5.940,00 & / \\ \hline
      \textbf{Sprint pregressi} & 321 & / & 6.520,00 & / \\ \hline
    \end{tabular}
    \caption{Sprint 9 - Consuntivo economico}
  \end{adjustwidth}
  \end{table}
  \end{minipage}

  \begin{figure}[H]
    \centering
    \includegraphics[width=0.90\textwidth]{assets/Consuntivo/Sprint-9/copertura_oraria.pdf}
    \caption{Sprint 9 - Areogramma del tempo speso (in ore) rispetto al totale}
  \end{figure}

  \begin{figure}[H]
    \centering
    \includegraphics[width=0.90\textwidth]{assets/Consuntivo/Sprint-9/budget_speso.pdf}
    \caption{Sprint 9 - Areogramma del budget speso rispetto al totale}
  \end{figure}

  \begin{minipage}{\textwidth}
    Di seguito sono riportate le ore rimanenti per la coppia risorsa-ruolo:
    \begin{table}[H]
      \begin{tabularx}{\textwidth}{|c|*{6}{>{\centering}X|}c|}
        \hline
        \multicolumn{8}{|c|}{\textbf{Ore rimanenti per la coppia risorsa-ruolo}} \\
        \hline
        \textbf{Membro del team} & \textbf{Re} & \textbf{Am} & \textbf{An} & \textbf{Pt} & \textbf{Pr} & \textbf{Ve} & \textbf{Totale per persona} \\
        \hline
        Cavalli Riccardo & 0 & 0 & 4 & 12 & 10 & 9 & 35 \\
        \hline
        Pianon Raul & 2 & 1 & 1 & 19 & 9 & 8 & 40 \\
        \hline
        Dall’Amico Martina & 2 & 1 & 1 & 14 & 16 & 812 & 42 \\
        \hline
        Cristo Marco & 2 & 4 & 1 & 17 & 10 & 8 & 42 \\
        \hline
        Lewental Sebastiano & 3 & 4 & 1 & 11 & 14 & 10 & 43 \\
        \hline
        Zecchinato Mattia & 5 & 2 & 3 & 9 & 13 & 12 & 44 \\
        \hline
        Stocco Tommaso & 5 & 0 & 3 & 19 & 9 & 12 & 48 \\
        \hline
        \textbf{Totale ore per ruolo} & 19 & 13 & 14 & 101 & 82 & 67 & \textbf{296} \\
        \hline
      \end{tabularx}
      \caption{Sprint 9 - Ore rimanenti per la coppia risorsa-ruolo}
    \end{table}
  \end{minipage}

\subsubsection{Revisione delle attività}

Nell'arco del nono \glossario{sprint}, il team ha svolto le seguenti attività:
\begin{itemize}
  \item TODO
\end{itemize}

\subsubsection{Retrospettiva}

\par Di seguito sono riportati i risultati del questionario di valutazione dello \glossario{sprint}:
\begin{itemize}
  \item TODO
\end{itemize}

\vspace{0.5\baselineskip}
\par A seguire le \textbf{analisi a posteriori} del nono \glossario{sprint}:
\begin{itemize}
  \item TODO
\end{itemize}

\subsubsection{Aggiornamento pianificazione e preventivo}
\par Il team ha definito un piano d'azione per migliorare l'organizzazione e la produttività del prossimo \glossario{sprint}:
\begin{itemize}
  \item TODO
\end{itemize}

\paragraph*{Pianificazione futura:}
\par TODO: passaggio pianificazione a 2 sett dopo RTB

\paragraph*{Preventivo "a finire" (\sezione{sec:stima_temporale}):}
\par Effettuata la prima presentazione della revisione RTB, il gruppo rimane in attesa dell'esito da parte della Committente, preparando nel frattempo la documentazione necessaria ad affrontare la seconda fase PB. Si mantiene un carico di lavoro equilibrato, tenendo sempre in considerazione il periodo corrente di sessione d'esami.

\paragraph*{Gestione dei rischi (\sezione{sec:analisi_rischi}):}
\par Nel corso del nono \glossario{sprint}, il seguente rischio si è presentato ed è stato gestito correttamente:
\begin{itemize}
  \item \textbf{Rischi relativi a rallentamenti}: TODO durante la sessione di esame, sono state ridotte le ore assegnate ad ogni membro per favorire lo studio; essendo un riscio previsto, la data di consegna era stata pianificata in modo adeguato.
\end{itemize}

\vspace{0.5\baselineskip}
\par Durante il nono \glossario{sprint}, alcune contromisure si sono rivelate utili a mitigare i rischi individuati in fase di pianificazione. In particolare:
\begin{itemize}
  \item TODO
\end{itemize}
