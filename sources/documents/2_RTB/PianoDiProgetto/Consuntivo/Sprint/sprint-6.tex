\subsection{Sesto sprint}

\begin{minipage}{\textwidth}
  Di seguito è riportata la distribuzione delle ore per ciascun membro del team, accumulate in totali per persona e per ruolo:
  \begin{table}[H]
    \begin{tabularx}{\textwidth}{|c|*{6}{>{\centering}X|}c|}
      \hline
      \multicolumn{8}{|c|}{\textbf{Consuntivo orario}} \\
      \hline
      \textbf{Membro del team} & \textbf{Re} & \textbf{Am} & \textbf{An} & \textbf{Pt} & \textbf{Pr} & \textbf{Ve} & \textbf{Totale per persona} \\
      \hline
      Riccardo Cavalli & 0 & 1 & 4 & 0 & 0 & 0 & 5 \\ 
      \hline
      Raul Pianon & 0 & 3 & 1 & 0 & 0 & 1 & 5 \\ 
      \hline
      Martina Dall'Amico & 1 & 0 & 0 & 0 & 3 & 1 & 5 \\ 
      \hline
      Marco Cristo & 0 & 1 & 2 & 0 & 0 & 2 & 5 \\ 
      \hline
      Sebastiano Lewental & 3 & 0 & 0 & 0 & 2 & 0 & 5 \\ 
      \hline
      Mattia Zecchinato & 0 & 2 & 0 & 0 & 3 & 1 & 6 \\ 
      \hline
      Tommaso Stocco & 0 & 0 & 0 & 0 & 4 & 2 & 6 \\ 
      \hline
      \textbf{Totale ore per ruolo} & 4 & 7 & 7 & 0 & 12 & 7 & \textbf{37} \\
      \hline
    \end{tabularx}
    \caption{Sprint 6 - Consuntivo orario}
  \end{table}
  \end{minipage}

  \begin{figure}[H]
    \centering
    \includegraphics[width=0.90\textwidth]{assets/Consuntivo/Sprint-6/distribuzione_ore_risorsa_ruolo.pdf}
    \caption{Sprint 6 - Istogramma della distribuzione oraria per la coppia risorsa-ruolo}
  \end{figure}

  \begin{figure}[H]
    \centering
    \includegraphics[width=0.90\textwidth]{assets/Consuntivo/Sprint-6/distribuzione_ore_ruolo.pdf}
    \caption{Sprint 6 - Areogramma della distribuzione oraria per ruolo}
  \end{figure}

  \begin{minipage}{\textwidth}
  Di seguito è riportato il consuntivo economico del sesto \glossario{sprint}:
  \begin{table}[H]
  \begin{adjustwidth}{-0.5cm}{-0.5cm}
    \centering
    \begin{tabular}{|P{2.9cm}|P{2.3cm}|P{2.5cm}|P{2.3cm}|>{\arraybackslash}P{2.5cm}|}
      \hline
      \multicolumn{5}{|c|}{\textbf{Consuntivo economico}} \\
      \hline
      \textbf{Ruolo} & \textbf{Ore per ruolo} & \textbf{Delta ore preventivo - consuntivo} & \textbf{Costo (in \texteuro)} & \textbf{Delta costo preventivo - consuntivo (in \texteuro)} \\
      \hline
      Responsabile & 4 & -1 & 120,00 & -30,00 \\ \hline
      Amministratore & 7 & 1 & 140,00 & 20,00 \\ \hline
      Analista & 7 & 0 & 175,00 & 0,00 \\ \hline
      Progettista & 0 & 0 & 0,00 & 0,00 \\ \hline
      Programmatore & 12 & 0 & 180,00 & 0,00 \\ \hline
      Verificatore & 7 & 0 & 105,00 & 0,00 \\ \hline
      \textbf{Totale} & \textbf{37} & 0 & \textbf{720,00} & -10,00 \\ \hline
      \textbf{Restante} & 366 & / & 7.290,00 & / \\ \hline
      \textbf{Sprint pregressi} & 243 & / & 5.010,00 & / \\ \hline
    \end{tabular}
    \caption{Sprint 6 - Consuntivo economico}
  \end{adjustwidth}
  \end{table}
  \end{minipage}

  \begin{figure}[H]
    \centering
    \includegraphics[width=0.90\textwidth]{assets/Consuntivo/Sprint-6/copertura_oraria.pdf}
    \caption{Sprint 6 - Areogramma del tempo speso (in ore) rispetto al totale}
  \end{figure}

  \begin{figure}[H]
    \centering
    \includegraphics[width=0.90\textwidth]{assets/Consuntivo/Sprint-6/budget_speso.pdf}
    \caption{Sprint 6 - Areogramma del budget speso rispetto al totale}
  \end{figure}

  \begin{minipage}{\textwidth}
    Di seguito sono riportate le ore rimanenti per la coppia risorsa-ruolo:
    \begin{table}[H]
      \begin{tabularx}{\textwidth}{|c|*{6}{>{\centering}X|}c|}
        \hline
        \multicolumn{8}{|c|}{\textbf{Ore rimanenti per la coppia risorsa-ruolo}} \\
        \hline
        \textbf{Membro del team} & \textbf{Re} & \textbf{Am} & \textbf{An} & \textbf{Pt} & \textbf{Pr} & \textbf{Ve} & \textbf{Totale per persona} \\
        \hline
        Riccardo Cavalli & 0 & 0 & 3 & 14 & 16 & 15 & 48 \\ 
        \hline
        Raul Pianon & 2 & 7 & 1 & 20 & 10 & 10 & 50 \\ 
        \hline
        Martina Dall'Amico & 5 & 2 & 1 & 14 & 19 & 12 & 53 \\ 
        \hline
        Marco Cristo & 2 & 9 & 0 & 17 & 10 & 13 & 51 \\ 
        \hline
        Sebastiano Lewental & 6 & 4 & 2 & 11 & 14 & 16 & 53 \\ 
        \hline
        Mattia Zecchinato & 9 & 6 & 3 & 11 & 12 & 13 & 54 \\ 
        \hline
        Tommaso Stocco & 5 & 4 & 3 & 20 & 9 & 14 & 55 \\ 
        \hline
        \textbf{Totale ore per ruolo} & 29 & 33 & 13 & 107 & 91 & 93 & \textbf{366} \\ 
        \hline
      \end{tabularx}
      \caption{Sprint 6 - Ore rimanenti per la coppia risorsa-ruolo}
    \end{table}
  \end{minipage}

\subsubsection{Revisione delle attività}

Nell'arco del sesto \glossario{sprint}, il team ha svolto le seguenti attività:
\begin{itemize}
    \item Stesura dei verbali interni ed esterni;
    \item Miglioramento della struttura dei \glossario{casi d'uso} nell’\AdR;
    \item Revisione dei grafici nell’\AdR;
    \item Ampliamento dei casi d’uso con la gestione del \glossario{debug} e degli errori;
    \item Aggiornamento del \PdP; (stesura delle sezioni incomplete e revi-sione dei consuntivi precedenti);
    \item Stesura della dashboard di valutazione della qualità nel Piano di Qualifica;
    \item Inserimento dei grafici nel Piano di Qualifica;
    \item Creazione componente di login funzionante;
    \item Sviluppo delle pagine \glossario{front-end} (chat, gestione dei \glossario{dizionari dati} e debug).
\end{itemize}

\subsubsection{Retrospettiva}

\par Di seguito sono riportati i risultati del questionario di valutazione dello \glossario{sprint}:
\begin{itemize}
  \item Organizzazione dello \glossario{sprint}\ - Valutazione: 8;
  \item Conduzione dei meeting interni - Valutazione: 8;
  \item Conduzione dei meeting esterni - Valutazione: 8;
  \item Impegno e partecipazione dei singoli membri - Valutazione: 7;
  \item La quasi totalità dei membri del team era a conoscenza delle proprie mansioni;
  \item La numerosità delle riunioni è risultata adeguata per tutti i membri del gruppo;
  \item Le riunioni sono state organizzate quasi sempre con il giusto preavviso;
  \item Il rapporto ore spese/ore produttive si sta notevolmente equilibrando;
  \item La produttività è stata ragioneole considerando le criticità dell'imminente sessione;
  \item È diffusa l'idea di programmare incontri in presenza più frequenti.
\end{itemize}

\vspace{0.5\baselineskip}
\par A seguire le \textbf{analisi a posteriori} del sesto \glossario{sprint}:
\begin{itemize}
  \item In seguito ad un incontro in presenza avvenuto durante questo \glossario{sprint}, il team ha notato quanto più efficace risulti questa modalità di lavoro rispetto a quella a distanza, evidenziata dalla quantità di \glossario{pull request} avvenute e risolte in tale occasione;
  \item Le precedenti tecnologie individuate per la parte di \glossario{backend} sono risultate troppo complesse per lo sviluppo in essere, quindi è stata decisa un'ulteriore modifica della libreria, col passaggio a \glossario{FastAPI};
  \item La sessione di esami ha impattato sulla quantità di lavoro svolto, come previsto.
\end{itemize}

\subsubsection{Aggiornamento pianificazione e preventivo}
\par Il team ha definito un piano d'azione per migliorare l'organizzazione e la produttività del prossimo \glossario{sprint}:
\begin{itemize}
  \item Pianificare incontri in presenza con frequenza maggiore, di almeno un incontro ogni due settimane;
\end{itemize}

\paragraph*{Pianificazione futura:}
\par Come riportato nell'analisi a posteriori, il team ha deciso di aumentare il numero di riunioni in presenza.
Inoltre si è deciso di mantenere moderata la quantità di ore assegnate a ciascun membro del team, in modo da continuare a garantire un equilibrio tra lavoro e studio.

\paragraph*{Preventivo "a finire" (\sezione{sec:stima_temporale}):}
\par L'avanzamento dello stato prodotto porta a stabilire la prossimità della revisione RTB, che avverrà tuttavia dopo la sessione di esami.

\paragraph*{Gestione dei rischi (\sezione{sec:analisi_rischi}):}
\par Nel corso del sesto \glossario{sprint}, alcune contromisure si sono rivelate insufficienti a mitigare i rischi emersi:
\begin{itemize}
  \item \textbf{RO4 - Rotazione dei ruoli}: nonostante il team avesse previsto di affiancare alcuni membri esperti ai nuovi programmatori, la concomitanza degli esami ha impedito l'attuazione di questa misura di mitigazione. Il rischio è stato gestito attraverso l’organizzazione di un incontro in presenza, durante il quale il gruppo ha completato le attività con la priorità più alta. Nella \sezione{sec:analisi_rischi}, il rischio relativo alla rotazione dei ruoli è stato aggiornato, con gli incontri in presenza considerati come possibili contromisure.
\end{itemize}

\par Di seguito sono elencati i rischi gestiti con successo:
\begin{itemize}
  \item \textbf{RO1 - Periodi di rallentamento}: uno dei rischi riscontrati dal team è stato il rallentamento dovuto alla sessione di esami universitari, che ha comportato la necessità di coordinare sia lo studio personale che l'avanzamento del progetto. La concomitanza delle scadenze accademiche con le attività di sviluppo ha rallentato il flusso di lavoro; tuttavia, questo rischio era stato preventivato con largo anticipo. Per mitigarne l'impatto, sono state adottate le seguenti contromisure: negli sprint precedenti, il carico di lavoro è stato intensificato, in modo tale da anticipare e completare una parte significativa delle attività. In fase di preventivo del sesto \glossario{sprint}, le ore produttive sono state ridotte, permettendo al team di dedicare il tempo necessario alla sessione di esami. Inoltre, sono state pianificate le attività con scadenze più flessibili. Infine, è stato promossa la comunicazione asincrona e l'organizzazione di riunioni brevi e mirate, per ottimizzare il tempo a disposizione di ciascun membro del team;
  \item \textbf{RT3 - Malfunzionamenti software}: le anomalie software sono state trattate nel canale dedicato su Discord e risolte attraverso l'apertura di issue su GitHub.
  \item \textbf{RT1 - Scarso know-how tecnologico}: attraverso il processo di integrazione continua, il team ha potuto allinearsi con le soluzioni sviluppate nello sprint precedente.
\end{itemize}
