\subsection{Quarto sprint}

\begin{minipage}{\textwidth}
  Di seguito è riportata la distribuzione delle ore per ciascun membro del team, accumulate in totali per persona e per ruolo:
  \begin{table}[H]
    \begin{tabularx}{\textwidth}{|c|*{6}{>{\centering}X|}c|}
      \hline
      \multicolumn{8}{|c|}{\textbf{Consuntivo orario}} \\
      \hline
      \textbf{Membro del team} & \textbf{Re} & \textbf{Am} & \textbf{An} & \textbf{Pt} & \textbf{Pr} & \textbf{Ve} & \textbf{Totale per persona} \\
      \hline
      Cavalli Riccardo & 0 & 0 & 0 & 0 & 3 & 5 & 8 \\ 
      \hline
      Pianon Raul & 0 & 0 & 7 & 0 & 0 & 1 & 8 \\ 
      \hline
      Dall’Amico Martina & 0 & 0 & 0 & 6 & 1 & 0 & 7 \\ 
      \hline
      Cristo Marco & 6 & 0 & 0 & 0 & 0 & 2 & 8 \\ 
      \hline
      Lewental Sebastiano & 0 & 6 & 0 & 0 & 0 & 1 & 7 \\ 
      \hline
      Zecchinato Mattia & 0 & 0 & 0 & 3 & 3 & 0 & 6 \\ 
      \hline
      Stocco Tommaso & 0 & 0 & 0 & 0 & 7 & 0 & 7 \\ 
      \hline
      \textbf{Totale ore per ruolo} & 6 & 6 & 7 & 9 & 14 & 9 & \textbf{51} \\
      \hline
    \end{tabularx}
    \caption{Sprint 4 - Consuntivo orario}
  \end{table}
  \end{minipage}
  
  \begin{figure}[H]
    \centering
    \includegraphics[width=0.90\textwidth]{assets/Consuntivo/Sprint-4/distribuzione_ore_risorsa_ruolo.pdf}
    \caption{Sprint 4 - Istogramma della distribuzione oraria per la coppia risorsa-ruolo}
  \end{figure}
  
  \begin{figure}[H]
    \centering
    \includegraphics[width=0.90\textwidth]{assets/Consuntivo/Sprint-4/distribuzione_ore_ruolo.pdf}
    \caption{Sprint 4 - Areogramma della distribuzione oraria per ruolo}
  \end{figure}
  
  \begin{minipage}{\textwidth}
  Di seguito è riportato il consuntivo economico del quarto \glossario{sprint}:
  \begin{table}[H]
  \begin{adjustwidth}{-0.5cm}{-0.5cm}
    \centering
    \begin{tabular}{|P{2.9cm}|P{2.3cm}|P{2.5cm}|P{2.3cm}|>{\arraybackslash}P{2.5cm}|}
      \hline
      \multicolumn{5}{|c|}{\textbf{Consuntivo economico}} \\
      \hline
      \textbf{Ruolo} & \textbf{Ore per ruolo} & \textbf{Delta ore preventivo - consuntivo} & \textbf{Costo (in \texteuro)} & \textbf{Delta costo preventivo - consuntivo (in \texteuro)} \\
      \hline
      Responsabile & 6 & 0 & 180,00 & 0,00 \\ \hline
      Amministratore & 6 & 4 & 120,00 & 80,00 \\ \hline
      Analista & 7 & -1 & 175,00 & -25,00 \\ \hline
      Progettista & 9 & 0 & 225,00 & 0,00 \\ \hline
      Programmatore & 14 & -1 & 210,00 & -15,00 \\ \hline
      Verificatore & 9 & 1 & 135,00 & 15,00 \\ \hline
      \textbf{Totale} & \textbf{51} & 3 & \textbf{1.045,00} & 55,00 \\ \hline
      \textbf{Restante} & 444 & / & 8.710,00 & / \\ \hline
      \textbf{Sprint pregressi} & 149 & / & 3.230,00 & / \\ \hline
    \end{tabular}
    \caption{Sprint 4 - Consuntivo economico}
  \end{adjustwidth}
  \end{table}
  \end{minipage}
  
  \begin{figure}[H]
    \centering
    \includegraphics[width=0.90\textwidth]{assets/Consuntivo/Sprint-4/copertura_oraria.pdf}
    \caption{Sprint 4 - Areogramma del tempo speso (in ore) rispetto al totale}
  \end{figure}
  
  \begin{figure}[H]
    \centering
    \includegraphics[width=0.90\textwidth]{assets/Consuntivo/Sprint-4/budget_speso.pdf}
    \caption{Sprint 4 - Areogramma del budget speso rispetto al totale}
  \end{figure}
  
  \begin{minipage}{\textwidth}
    Di seguito sono riportate le ore rimanenti per la coppia risorsa-ruolo:
    \begin{table}[H]
      \begin{tabularx}{\textwidth}{|c|*{6}{>{\centering}X|}c|}
        \hline
        \multicolumn{8}{|c|}{\textbf{Ore rimanenti per la coppia risorsa-ruolo}} \\
        \hline
        \textbf{Membro del team} & \textbf{Re} & \textbf{Am} & \textbf{An} & \textbf{Pt} & \textbf{Pr} & \textbf{Ve} & \textbf{Totale per persona} \\
        \hline
        Cavalli Riccardo & 0 & 2 & 9 & 14 & 18 & 16 & 59 \\ 
        \hline
        Pianon Raul & 2 & 10 & 2 & 20 & 15 & 12 & 61 \\ 
        \hline
        Dall’Amico Martina & 9 & 2 & 1 & 14 & 22 & 16 & 64 \\ 
        \hline
        Cristo Marco & 3 & 10 & 2 & 17 & 13 & 17 & 62 \\ 
        \hline
        Lewental Sebastiano & 9 & 4 & 2 & 11 & 21 & 17 & 64 \\ 
        \hline
        Zecchinato Mattia & 9 & 9 & 3 & 11 & 20 & 15 & 67 \\ 
        \hline
        Stocco Tommaso & 5 & 4 & 3 & 20 & 16 & 19 & 67 \\ 
        \hline
        \textbf{Totale ore per ruolo} & 37 & 41 & 22 & 107 & 125 & 112 & \textbf{444} \\ 
        \hline
      \end{tabularx}
      \caption{Sprint 4 - Ore rimanenti per la coppia risorsa-ruolo}
    \end{table}
  \end{minipage}

\subsubsection{Revisione delle attività}

Nell'arco del quarto \glossario{sprint}, il team ha svolto le seguenti attività:
\begin{itemize}
    \item Stesura verbali interni ed esterni;
    \item Revisione dei consuntivi pregressi all'interno del \PdP;
    \item Rielaborazione completa dei \glossario{casi d'uso} nell'\AdR;
    \item Estensione dei casi d'uso nell'\AdR\ in accordo con il team di programmatori;
    \item Creazione e connessione al database;
    \item Sviluppo dell'interfaccia di login tramite \glossario{Streamlit};
    \item \glossario{Dockerizzazione} dell'ambiente di sviluppo;
    \item Definizione dei test di correttezza del \glossario{prompt};
    \item Scelta definitiva del modello \glossario{LLM};
    \item Creazione della lista di selezione del dizionario dati;
    \item Sviluppo back-end della funzionalità di debug per il profilo Tecnico;
    \item Riorganizzazione del repository ChatSQL per integrare i moduli e rimuovere i file ridondanti;
    \item Refactoring completo del codice con rimozione dei file obsoleti.
\end{itemize}

\subsubsection{Retrospettiva}

\par Di seguito sono riportati i risultati del questionario di valutazione dello \glossario{sprint}:
\begin{itemize}
  \item Organizzazione dello \glossario{sprint}\ - Valutazione: 7;
  \item Conduzione dei meeting interni - Valutazione: 8;
  \item Conduzione dei meeting esterni - Valutazione: 8;
  \item Impegno e partecipazione dei singoli membri - Valutazione: 9;
  \item La quasi totalità dei membri del team era a conoscenza delle proprie mansioni;
  \item La numerosità delle riunioni è risultata adeguata per circa la metà dei membri; inoltre, il team preferirebbe organizzare più incontri informali tra programmatori;
  \item Le riunioni sono state organizzate quasi sempre con il giusto preavviso;
  \item Il rapporto ore spese/ore produttive è risultato meno equilibrato rispetto allo \glossario{sprint} precedente;
  \item La produttività generale ha raggiunto una buona soglia;
  \item Alcuni membri del team ritengono opportuno avere più dettagli anche sulle task meno prossime.
\end{itemize}

\vspace{0.5\baselineskip}
\par A seguire le \textbf{analisi a posteriori} del quarto \glossario{sprint}:
\begin{itemize}
  \item La sola pianificazione iniziale delle attività non ha permesso ai membri del team di poter svolgere le proprie task a inizio \glossario{sprint}, richiedendo qualche giorno di stallo prima di potersi dedicare alle attività assegnate. Di conseguenza, il gruppo ha ritenuto opportuno, a partire dal prossimo sprint, che i ruoli uscenti spiegassero, tramite riunioni ad hoc, il lavoro che i ruoli entranti dovranno svolgere;
  \item Come conseguenza del punto precedente, il team ha stabilito di affidare l'assegnazione delle task per i ruoli entranti ai membri che non ricopreranno più lo stesso ruolo. In questo modo risulterà possibile delineare più puntualmente le attività imminenti;
  \item Anche se i risultati dei questionari di valutazione riguardo le riunioni mostrano una sufficiente adeguatezza organizzativa, il responsabile entrante ha ritenuto opportuno aumentare il margine di preavviso per minimizzare le assenze durante i meeting, comunicando il calendario delle riunioni future entro la fine dello sprint in corso;
  \item Nonostante il team abbia organizzato una quantità apprezzabile di incontri tra programmatori e altri ruoli correlati durante la seconda metà dello sprint, lo stesso non si può dire per la prima settimana. Ciò ha portato a dei ritardi nella prosecuzione delle attività, mantenendo comunque alto l'impegno individuale e, di conseguenza, inficiando sul rapporto tra le ore spese e quelle produttive. Preso atto di ciò, il gruppo ha prontamente affrontato il problema e ha incrementato il numero di riunioni, recuperando così lo stallo iniziale.
\end{itemize}

\subsubsection{Aggiornamento pianificazione e preventivo}
\par Il team ha definito un piano d'azione per migliorare l'organizzazione e la produttività del prossimo \glossario{sprint}:
\begin{itemize}
  \item Aumentare il preavviso per i meeting futuri;
  \item I ruoli uscenti devono affiancare, ove possibile, il responsabile nella definizione delle task per i ruoli entranti;
  \item Incrementare il numero di riunioni informali;
  \item Effettuare una separazione più netta tra \glossario{front-end} e \glossario{back-end};
\end{itemize}

\paragraph*{Pianificazione futura:}

\paragraph*{Preventivo "a finire" (\sezione{sec:stima_temporale}):}
\par Data la numerosità delle risorse assegnate originariamente al ruolo di progettista, il team ha concordato la necessità di ridistribuire tali risorse a favore dell'amministratore, il cui impegno orario è stato ritenuto sottostimato. Questa ripartizione ha portato a un lieve aumento delle ore produttive totali per i ruoli di programmatore e di verificatore. 
La ripartizione è avvenuta come segue:
\begin{itemize}
  \item Le ore complessive assegnate al ruolo di progettista sono diminuite da 161 a 140;
  \item Le ore complessive assegnate al ruolo di amministratore sono aumentate da 56 a 70;
  \item Le ore complessive assegnate al ruolo di programmatore sono aumentate da 154 a 161;
  \item Le ore complessive assegnate al ruolo di verificatore sono aumentate da 140 a 147;
\end{itemize}
Il monte ore individuali è di conseguenza aumentato da 91 a 92. Nonostante l'incremento delle ore totali per ruolo, la riallocazione delle risorse ha comportato un risparmio di €35 rispetto al budget previsto per il progetto.

\paragraph*{Gestione dei rischi (\sezione{sec:analisi_rischi}):}
\par Nel corso del quarto \glossario{sprint}, il seguente rischio non è stato gestito con successo:
\begin{itemize}
  \item \textbf{Rischi relativi alla rotazione dei ruoli:} Diversi membri del team hanno incontrato degli ostacoli a seguito della rotazione dei ruoli. Una volta individuata tale problematica, dopo una serie di confronti interni, è stata attuata la contromisura descritta nella \sezione{sec:analisi_rischi}. L'efficacia del metodo di mitigazione ha subito però una riduzione a causa della sua applicazione tardiva, causando un periodo di rallentamento iniziale. La gestione del rischio deve quindi essere modificata, per aggiungere ulteriori controlli nelle strategie di rilevamento.
\end{itemize}