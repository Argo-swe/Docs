\subsection{Settimo sprint}

\begin{minipage}{\textwidth}
  Di seguito è riportata la distribuzione delle ore per ciascun membro del team, accumulate in totali per persona e per ruolo:
  \begin{table}[H]
    \begin{tabularx}{\textwidth}{|c|*{6}{>{\centering}X|}c|}
      \hline
      \multicolumn{8}{|c|}{\textbf{Consuntivo orario}} \\
      \hline
      \textbf{Membro del team} & \textbf{Re} & \textbf{Am} & \textbf{An} & \textbf{Pt} & \textbf{Pr} & \textbf{Ve} & \textbf{Totale per persona} \\
      \hline
      Cavalli Riccardo & 0 & 0 & 0 & 0 & 3 & 2 & 5 \\
      \hline
      Pianon Raul & 0 & 3 & 0 & 0 & 0 & 0 & 3 \\
      \hline
      Dall’Amico Martina & 0 & 2 & 0 & 0 & 0 & 1 & 3 \\
      \hline
      Cristo Marco & 0 & 2 & 0 & 0 & 0 & 1 & 3 \\
      \hline
      Lewental Sebastiano & 2 & 0 & 0 & 0 & 1 & 1 & 4 \\
      \hline
      Zecchinato Mattia & 2 & 3 & 0 & 0 & 0 & 0 & 5 \\
      \hline
      Stocco Tommaso & 0 & 4 & 0 & 0 & 0 & 0 & 4 \\
      \hline
      \textbf{Totale ore per ruolo} & 4 & 12 & 0 & 0 & 7 & 4 & \textbf{27} \\
      \hline
    \end{tabularx}
    \caption{Sprint 7 - Consuntivo orario}
  \end{table}
  \end{minipage}

  \begin{figure}[H]
    \centering
    \includegraphics[width=0.90\textwidth]{assets/Consuntivo/Sprint-7/distribuzione_ore_risorsa_ruolo.pdf}
    \caption{Sprint 7 - Istogramma della distribuzione oraria per la coppia risorsa-ruolo}
  \end{figure}

  \begin{figure}[H]
    \centering
    \includegraphics[width=0.90\textwidth]{assets/Consuntivo/Sprint-7/distribuzione_ore_ruolo.pdf}
    \caption{Sprint 7 - Areogramma della distribuzione oraria per ruolo}
  \end{figure}

  \begin{minipage}{\textwidth}
  Di seguito è riportato il consuntivo economico del settimo \glossario{sprint}:
  \begin{table}[H]
  \begin{adjustwidth}{-0.5cm}{-0.5cm}
    \centering
    \begin{tabular}{|P{2.9cm}|P{2.3cm}|P{2.5cm}|P{2.3cm}|>{\arraybackslash}P{2.5cm}|}
      \hline
      \multicolumn{5}{|c|}{\textbf{Consuntivo economico}} \\
      \hline
      \textbf{Ruolo} & \textbf{Ore per ruolo} & \textbf{Delta ore preventivo - consuntivo} & \textbf{Costo (in \texteuro)} & \textbf{Delta costo preventivo - consuntivo (in \texteuro)} \\
      \hline
      Responsabile & 4 & 0 & 120,00 & 0,00 \\ \hline
      Amministratore & 12 & -4 & 240,00 & -80,00 \\ \hline
      Analista & 0 & 1 & 0,00 & 25,00 \\ \hline
      Progettista & 0 & 0 & 0,00 & 0,00 \\ \hline
      Programmatore & 7 & 0 & 105,00 & 0,00 \\ \hline
      Verificatore & 4 & 4 & 60,00 & 60,00 \\ \hline
      \textbf{Totale} & \textbf{27} & 1 & \textbf{525,00} & 5,00 \\ \hline
      \textbf{Restante} & 353 & / & 6.995,00 & / \\ \hline
      \textbf{Sprint pregressi} & 266 & / & 5.500,00 & / \\ \hline
    \end{tabular}
    \caption{Sprint 7 - Consuntivo economico}
  \end{adjustwidth}
  \end{table}
  \end{minipage}

  \begin{figure}[H]
    \centering
    \includegraphics[width=0.90\textwidth]{assets/Consuntivo/Sprint-7/copertura_oraria.pdf}
    \caption{Sprint 7 - Areogramma del tempo speso (in ore) rispetto al totale}
  \end{figure}

  \begin{figure}[H]
    \centering
    \includegraphics[width=0.90\textwidth]{assets/Consuntivo/Sprint-7/budget_speso.pdf}
    \caption{Sprint 7 - Areogramma del budget speso rispetto al totale}
  \end{figure}

  \begin{minipage}{\textwidth}
    Di seguito sono riportate le ore rimanenti per la coppia risorsa-ruolo:
    \begin{table}[H]
      \begin{tabularx}{\textwidth}{|c|*{6}{>{\centering}X|}c|}
        \hline
        \multicolumn{8}{|c|}{\textbf{Ore rimanenti per la coppia risorsa-ruolo}} \\
        \hline
        \textbf{Membro del team} & \textbf{Re} & \textbf{Am} & \textbf{An} & \textbf{Pt} & \textbf{Pr} & \textbf{Ve} & \textbf{Totale per persona} \\
        \hline
        Cavalli Riccardo & 0 & 0 & 4 & 14 & 13 & 13 & 44 \\
        \hline
        Pianon Raul & 2 & 4 & 1 & 20 & 10 & 12 & 49 \\
        \hline
        Dall’Amico Martina & 5 & 2 & 1 & 14 & 16 & 12 & 50 \\
        \hline
        Cristo Marco & 2 & 7 & 1 & 17 & 10 & 13 & 50 \\
        \hline
        Lewental Sebastiano & 4 & 4 & 2 & 11 & 16 & 14 & 51 \\
        \hline
        Zecchinato Mattia & 7 & 3 & 3 & 11 & 14 & 14 & 52 \\
        \hline
        Stocco Tommaso & 5 & 0 & 3 & 20 & 9 & 18 & 55 \\
        \hline
        \textbf{Totale ore per ruolo} & 25 & 21 & 15 & 107 & 89 & 96 & \textbf{353} \\
        \hline
      \end{tabularx}
      \caption{Sprint 7 - Ore rimanenti per la coppia risorsa-ruolo}
    \end{table}
  \end{minipage}

\subsubsection{Revisione delle attività}

Nell'arco del settimo \glossario{sprint}, il team ha svolto le seguenti attività:
\begin{itemize}
  \item Revisione consuntivi \PdP;
  \item Aggiunta metriche alle \NdP;
  \item Completamento sezioni incomplete \NdP;
  \item Stesura sezione test nel \PdQ;
  \item Stesura dei verbali interni;
  \item Gestione stato Utente/Tecnico a \glossario{front-end};
  \item Creazione interfaccia Chat a \glossario{front-end};
  \item Implementazione funzioni per generare il \glossario{prompt} e mostrarlo a \glossario{front-end}.
\end{itemize}

\subsubsection{Retrospettiva}

\par Di seguito sono riportati i risultati del questionario di valutazione dello \glossario{sprint}:
\begin{itemize}
  \item Organizzazione dello \glossario{sprint}\ - Valutazione: 8;
  \item Conduzione dei meeting interni - Valutazione: 8;
  \item Conduzione dei meeting esterni - Valutazione: 8;
  \item Impegno e partecipazione dei singoli membri - Valutazione: 3;
  \item La quasi totalità dei membri del team era a conoscenza delle proprie mansioni;
  \item La numerosità delle riunioni è risultata adeguata per tutti i membri del gruppo;
  \item Le riunioni sono state organizzate quasi sempre con il giusto preavviso;
  \item Il rapporto ore spese/ore produttive si sta notevolmente equilibrando;
  \item La produttività è stata ragioneole considerando le criticità della sessione;
  \item È diffusa l'idea di programmare incontri in presenza più frequenti.
\end{itemize}

\vspace{0.5\baselineskip}
\par A seguire le \textbf{analisi a posteriori} del settimo \glossario{sprint}:
\begin{itemize}
  \item La sessione di esami ha impattato sulla quantità di lavoro svolto, come previsto.
\end{itemize}

\subsubsection{Aggiornamento pianificazione e preventivo}
\par Il team ha definito un piano d'azione per migliorare l'organizzazione e la produttività del prossimo \glossario{sprint}:
\begin{itemize}
  \item Continuare con la pianificazione di un incontri in presenza;
\end{itemize}

\paragraph*{Pianificazione futura:}
\par Si è deciso di mantenere moderata la quantità di ore assegnate a ciascun membro del team, in modo da continuare a garantire un equilibrio tra lavoro e studio.

\paragraph*{Preventivo "a finire" (\sezione{sec:stima_temporale}):}
\par L'avanzamento dello stato prodotto porta a stabilire la prossimità della revisione RTB, che avverrà tuttavia dopo la sessione di esami.

\paragraph*{Gestione dei rischi (\sezione{sec:analisi_rischi}):}
\par Nel corso del settimo \glossario{sprint}, il seguente rischio si è presentato ed è stato gestito correttamente:
\begin{itemize}
  \item \textbf{Rischi relativi a rallentamenti:} durante la sessione di esame, sono state ridotte le ore assegnate ad ogni membro per favorire lo studio; essendo un riscio previsto, la data di consegna era stata pianificata in modo adeguato.
\end{itemize}
