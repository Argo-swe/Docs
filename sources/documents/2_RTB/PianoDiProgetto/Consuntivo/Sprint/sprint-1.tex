\subsection{Primo sprint}

\begin{minipage}{\textwidth}
Di seguito è riportata la distribuzione delle ore per ciascun membro del team, accumulate in totali per persona e per ruolo:
\begin{table}[H]
  \begin{tabularx}{\textwidth}{|c|*{6}{>{\centering}X|}c|}
    \hline
    \multicolumn{8}{|c|}{\textbf{Consuntivo orario}} \\
    \hline
    \textbf{Membro del team} & \textbf{Re} & \textbf{Am} & \textbf{An} & \textbf{Pt} & \textbf{Pr} & \textbf{Ve} & \textbf{Totale per persona} \\
    \hline
    Cavalli Riccardo & 7 & 0 & 0 & 0 & 0 & 0 & 7 \\
    \hline
    Pianon Raul & 0 & 0 & 0 & 0 & 0 & 8 & 8 \\
    \hline
    Dall'Amico Martina & 0 & 0 & 8 & 0 & 0 & 0 & 8 \\
    \hline
    Cristo Marco & 0 & 0 & 7 & 0 & 0 & 0 & 7 \\
    \hline
    Lewental Sebastiano & 0 & 0 & 7 & 0 & 0 & 0 & 7 \\
    \hline
    Zecchinato Mattia & 0 & 0 & 1 & 6 & 0 & 0 & 7 \\
    \hline
    Stocco Tommaso & 0 & 6 & 0 & 0 & 0 & 0 & 6 \\
    \hline
    \textbf{Totale per ruolo} & 7 & 6 & 23 & 6 & 0 & 8 & \textbf{50} \\
    \hline
  \end{tabularx}
  \caption{Sprint 1 - Consuntivo orario}
\end{table}
\end{minipage}

\begin{figure}[H]
  \centering
  \includegraphics[width=0.90\textwidth]{assets/Consuntivo/Sprint-1/distribuzione_ore_risorsa_ruolo.pdf}
  \caption{Sprint 1 - Istogramma della distribuzione oraria per la coppia risorsa-ruolo}
\end{figure}

\begin{figure}[H]
  \centering
  \includegraphics[width=0.90\textwidth]{assets/Consuntivo/Sprint-1/distribuzione_ore_ruolo.pdf}
  \caption{Sprint 1 - Areogramma della distribuzione oraria per ruolo}
\end{figure}

\begin{minipage}{\textwidth}
Di seguito è riportato il consuntivo economico del primo \glossario{sprint}:
\begin{table}[H]
\begin{adjustwidth}{-0.5cm}{-0.5cm}
  \centering
  \begin{tabular}{|P{2.9cm}|P{2.3cm}|P{2.5cm}|P{2.3cm}|>{\arraybackslash}P{2.5cm}|}
    \hline
    \multicolumn{5}{|c|}{\textbf{Consuntivo economico}} \\
    \hline
    \textbf{Ruolo} & \textbf{Ore per ruolo} & \textbf{Delta ore preventivo - consuntivo} & \textbf{Costo (in \texteuro)} & \textbf{Delta costo preventivo - consuntivo (in \texteuro)} \\
    \hline
    Responsabile & 7 & 0 & 210,00 & 0,00 \\
    \hline
    Amministratore & 6 & 0 & 120,00 & 0,00 \\
    \hline
    Analista & 23 & 4 & 575,00 & 100,00 \\
    \hline
    Progettista & 6 & 1 & 150,00 & 25,00 \\
    \hline
    Programmatore & 0 & 0 & 0,00 & 0,00 \\
    \hline
    Verificatore & 8 & 0 & 120,00 & 0,00 \\
    \hline
    \textbf{Totale} & \textbf{50} & 5 & \textbf{1.175,00} & 125,00 \\
    \hline
    \textbf{Restante} & 587 & / & 11.845,00 & / \\
    \hline
    \textbf{Sprint pregressi} & 0 & / & 0,00 & / \\
    \hline
  \end{tabular}
  \caption{Sprint 1 - Consuntivo economico}
\end{adjustwidth}
\end{table}
\end{minipage}

\begin{figure}[H]
  \centering
  \includegraphics[width=0.90\textwidth]{assets/Consuntivo/Sprint-1/copertura_oraria.pdf}
  \caption{Sprint 1 - Areogramma del tempo speso (in ore) rispetto al totale}
\end{figure}

\begin{figure}[H]
  \centering
  \includegraphics[width=0.90\textwidth]{assets/Consuntivo/Sprint-1/budget_speso.pdf}
  \caption{Sprint 1 - Areogramma del budget speso rispetto al totale}
\end{figure}


\begin{minipage}{\textwidth}
  Di seguito sono riportate le ore rimanenti per la coppia risorsa-ruolo:
  \begin{table}[H]
    \begin{tabularx}{\textwidth}{|c|*{6}{>{\centering}X|}c|}
      \hline
      \multicolumn{8}{|c|}{\textbf{Ore rimanenti per la coppia risorsa-ruolo}} \\
      \hline
      \textbf{Membro del team} & \textbf{Re} & \textbf{Am} & \textbf{An} & \textbf{Pt} & \textbf{Pr} & \textbf{Ve} & \textbf{Totale per persona} \\
      \hline
      Cavalli Riccardo & 2 & 8 & 9 & 23 & 22 & 20 & 84 \\
      \hline
      Pianon Raul & 9 & 8 & 9 & 23 & 22 & 12 & 83 \\
      \hline
      Dall'Amico Martina & 9 & 8 & 1 & 23 & 22 & 20 & 83 \\
      \hline
      Cristo Marco & 9 & 8 & 2 & 23 & 22 & 20 & 84 \\
      \hline
      Lewental Sebastiano & 9 & 8 & 2 & 23 & 22 & 20 & 84 \\
      \hline
      Zecchinato Mattia & 9 & 8 & 8 & 17 & 22 & 20 & 84 \\
      \hline
      Stocco Tommaso & 9 & 2 & 9 & 23 & 22 & 20 & 85 \\
      \hline
      \textbf{Totale per ruolo} & 56 & 50 & 40 & 155 & 154 & 132 & \textbf{587} \\
      \hline
    \end{tabularx}
    \caption{Sprint 1 - Ore rimanenti per la coppia risorsa-ruolo}
  \end{table}
\end{minipage}

\subsubsection{Revisione delle attività}

Nel corso del primo \glossario{sprint}, il team ha svolto le seguenti attività:
\begin{itemize}
  \item Stesura iniziale del \PdP;
  \item Scelta del modello di sviluppo;
  \item Pianificazione e preventivo del secondo \glossario{sprint};
  \item \AdR\ e definizione dei primi \glossario{casi d'uso};
  \item Individuazione degli \glossario{attori} coinvolti nel sistema e delle loro caratteristiche;
  \item Analisi dei rischi tecnologici e organizzativi;
  \item Progettazione del \glossario{dizionario dati};
  \item Raccolta dei termini da inserire nel \Gls;
  \item Miglioramento del \WoW\ e automazione delle procedure con relativa stesura delle \NdP.
\end{itemize}

\subsubsection{Retrospettiva}

\par Di seguito sono riportati i risultati del questionario di valutazione dello \glossario{sprint}, realizzato dal responsabile in carica per supportare la fase di \glossario{retrospettiva}:
\begin{itemize}
  \item Organizzazione dello \glossario{sprint} - Valutazione: 7,5;
  \item Conduzione dei meeting interni - Valutazione: 7,5;
  \item Conduzione dei meeting esterni - Valutazione: 8;
  \item Impegno e partecipazione dei singoli membri - Valutazione: 7;
  \item Non tutti i membri del team erano a conoscenza delle proprie mansioni;
  \item La numerosità delle riunioni è adeguata;
  \item Le riunioni sono state organizzate quasi sempre con il giusto preavviso;
  \item Da migliorare il rapporto ore spese/ore produttive.
\end{itemize}

\par A seguire le analisi a posteriori effettuate durante la riunione di \glossario{retrospettiva}:
\begin{itemize}
  \item Le valutazioni raccolte dal team hanno evidenziato una pianificazione idonea ma non sempre esaustiva. Inoltre, il gruppo ha ritenuto di aver sovrastimato il carico di lavoro per alcuni ruoli, in particolare l'analista e il progettista, a cui sono state assegnate 34 ore produttive totali. Considerando l'inesperienza nell'approccio alle prime fasi del progetto e una pianificazione non dettagliata, sono state registrate delle discrepanze tra preventivo e consuntivo. Sebbene l'analisi dei requisiti sia un vincolo fondamentale per la realizzazione del \glossario{PoC}, il team non ha ritenuto necessaria, a posteriori, una divisione così netta delle ore per ruolo. Si è quindi optato per una distribuzione più omogenea nei prossimi \glossario{sprint}.
  \item Inoltre, il passaggio da un'attività precedente a quella successiva non è sempre stato immediato, e questo ha portato a delle fasi, seppur brevi, di stallo. Questa situazione, insieme alla considerazione precedente, ha portato il team a performare al di sotto delle aspettative.
  \item Il team ha riscontrato un feedback positivo per quanto riguarda l'organizzazione e la conduzione delle riunioni. Al contrario, sono stati rilevati come aspetti da migliorare la coesione interna e il rendimento delle singole risorse. Si è ritenuto inoltre necessario proseguire sulla strada dei micro-gruppi, suddividendo il lavoro in team più piccoli capaci di auto-coordinarsi e lavorare in sincronia.
\end{itemize}


\subsubsection{Aggiornamento pianificazione e preventivo}
TODO

