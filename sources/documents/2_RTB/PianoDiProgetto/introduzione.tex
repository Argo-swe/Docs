\section{Introduzione}

\subsection{Scopo del documento}
TODO 
\subsection{Scopo del prodotto}
Lo scopo del prodotto è lo sviluppo di un'applicazione che consenta di scrivere un testo in linguaggio naturale per ricevere un prompt che copiato e incollato su LLM, restituisca codice correttamente generato in SQL per la formulazione di query che possano interrogare un database. 
Un Utente Generico può eseguire l'accesso all'applicazione per visualizzare il dizionario dati disponibile per la ricerca e inserire la richiesta in linguaggio naturale. 
L'applicazione restituirà il prompt copiabile. 
L'Utente Generico può poi eseguire un nuovo login per ricevere permessi da Utente Tecnico: questi ha la possibilità fare modifiche, aggiunte o rimozioni al dizionario dati e visualizzare tramite una finestra di debug il comportamento del modello al ricevimento della richiesta in linguaggio naturale. 
Questo permette di avere una visione dei termini selezionati nel dizionario dati dalla frase, in modo da poter capire se il linguaggio ha raccolto informazioni poco utili o in eccessiva quantità.
\subsection{Riferimenti}
TODO
\subsubsection{Riferimenti normativi}
\begin{itemize}
  \item \href{https://github.com/Argo-swe/Argo-swe.github.io/tree/main/2 - RTB}{Norme di Progetto \textbf{v\VersioneNP}};
  \item Capitolato C9 - ChatSQL (Zucchetti S.p.A.): \\ \href{https://www.math.unipd.it/~tullio/IS-1/2023/Progetto/C9.pdf}{https://www.math.unipd.it/~tullio/IS-1/2023/Progetto/C9.pdf} \\ (Ultimo accesso: 2024-04-11);
  \item Slide PD2 - Corso di Ingegneria del Software - Regolamento del Progetto Didattico: \\ \href{https://www.math.unipd.it/~tullio/IS-1/2023/Dispense/PD2.pdf}{https://www.math.unipd.it/~tullio/IS-1/2023/Dispense/PD2.pdf} \\ (Ultimo accesso: 2024-04-11).
\end{itemize}

\subsubsection{Riferimenti informativi}
\begin{itemize}
  \item Slide T2 - Corso di Ingegneria del Software - Processi di ciclo di vita del Software: \\ \href{https://www.math.unipd.it/~tullio/IS-1/2023/Dispense/T2.pdf}{https://www.math.unipd.it/~tullio/IS-1/2023/Dispense/T2.pdf}  \\ (Ultimo accesso: 2024-04-11);
  \item Slide T3 - Corso di Ingegneria del Software - Modelli di sviluppo del Software: \\ \href{https://www.math.unipd.it/~tullio/IS-1/2023/Dispense/T3.pdf}{https://www.math.unipd.it/~tullio/IS-1/2023/Dispense/T3.pdf}  \\ (Ultimo accesso: 2024-04-11);
  \item Slide T4 - Corso di Ingegneria del Software - Gestione di Progetto \\ \href{https://www.math.unipd.it/~tullio/IS-1/2023/Dispense/T4.pdf}{https://www.math.unipd.it/~tullio/IS-1/2023/Dispense/T4.pdf}  \\ (Ultimo accesso: 2024-04-11);
  \item Verbali interni ed esterni.
\end{itemize}

\subsection{Glossario} 
\GlossarioIntroduzione

\subsection{Note organizzative}
TODO