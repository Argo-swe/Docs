\section{Introduzione}
\label{sec:introduzione}

\subsection{Scopo del documento}
Il \PdP\ è un documento essenziale per la gestione e l'organizzazione delle attività necessarie allo sviluppo di un applicativo software. Il suo scopo principale è tracciare la pianificazione dei task, distribuire i compiti tra i membri del team e fornire una visione d'insieme sia a livello amministrativo che operativo.

Per pianificare il lavoro in modo chiaro e puntuale, garantendo il controllo dei costi e la gestione dei rischi, gli argomenti trattati da questo documento sono stati suddivisi come segue:
\begin{itemize}
  \item L'identificazione dei rischi potenziali e la pianificazione di misure di mitigazione per minimizzare l'impatto;
  \item La definizione del modello di sviluppo adottato;
  \item Le stime temporali ed economiche, attraverso un calendario di massima, una stima preliminare dei costi e un preventivo "a finire";
  \item La pianificazione dettagliata delle attività da svolgere;
  \item La formulazione di un preventivo in base al risultato delle stime temporali e dei costi;
  \item La revisione e la \glossario{retrospettiva} delle attività svolte.
\end{itemize}
\subsection{Scopo del prodotto}
Lo scopo del prodotto è lo sviluppo di un'applicazione che consenta di scrivere un testo in linguaggio naturale per ricevere un \glossario{prompt} che, una volta copiato e incollato su \glossario{LLM}, restituisca una query SQL per interrogare un database. 
Un Utente può visualizzare i \glossario{dizionari dati} disponibili nel sistema e inserire una richiesta in linguaggio naturale; l'applicazione restituirà il prompt generato. 
L'Utente può effettuare il login per acquisire il ruolo di Tecnico; quest'ultimo ha la possibilità di modificare, aggiungere o rimuovere un dizionario dati e visualizzare, tramite una finestra di \glossario{debug}, il processo di generazione del prompt. 
Il debug fornisce una panoramica dei termini estratti dal dizionario dati, consentendo di valutare se il modello ha raccolto informazioni rilevanti o eccessive.
\subsection{Riferimenti}
Il presente documento si basa su normative elaborate dal team, dall'ente proponente o da entità esterne, oltre a includere materiali informativi. Tali riferimenti sono elencati di seguito.
\subsubsection{Riferimenti normativi}
\begin{itemize}
  \item \NormeDiProgetto;
  \item Capitolato C9 - ChatSQL:
  \begin{itemize}
    \item \href{https://www.math.unipd.it/~tullio/IS-1/2023/Progetto/C9.pdf}{https://www.math.unipd.it/\textasciitilde tullio/IS-1/2023/Progetto/C9.pdf} \\ (Ultimo accesso: 2024-07-02);
    \item \href{https://www.math.unipd.it/~tullio/IS-1/2023/Progetto/C9.pdf}{https://www.math.unipd.it/\textasciitilde tullio/IS-1/2023/Progetto/C9p.pdf} \\ (Ultimo accesso: 2024-07-02).
  \end{itemize}
  \item Regolamento del progetto didattico:\\ \href{https://www.math.unipd.it/~tullio/IS-1/2023/Dispense/PD2.pdf}{https://www.math.unipd.it/\textasciitilde tullio/IS-1/2023/Dispense/PD2.pdf}.
\end{itemize}

\subsubsection{Riferimenti informativi}
\begin{itemize}
  \item Slide T2 - Corso di Ingegneria del Software - Processi di ciclo di vita del Software: \\ \href{https://www.math.unipd.it/~tullio/IS-1/2023/Dispense/T2.pdf}{https://www.math.unipd.it/\textasciitilde tullio/IS-1/2023/Dispense/T2.pdf}  \\ (Ultimo accesso: 2024-04-11);
  \item Slide T3 - Corso di Ingegneria del Software - Modelli di sviluppo del Software: \\ \href{https://www.math.unipd.it/~tullio/IS-1/2023/Dispense/T3.pdf}{https://www.math.unipd.it/\textasciitilde tullio/IS-1/2023/Dispense/T3.pdf}  \\ (Ultimo accesso: 2024-04-11);
  \item Slide T4 - Corso di Ingegneria del Software - Gestione di Progetto \\ \href{https://www.math.unipd.it/~tullio/IS-1/2023/Dispense/T4.pdf}{https://www.math.unipd.it/\textasciitilde tullio/IS-1/2023/Dispense/T4.pdf}  \\ (Ultimo accesso: 2024-04-11);
  \item Introduzione a Jira \\ \href{https://www.atlassian.com/it/software/jira/guides/getting-started/introduction#what-is-jira-software}{https://www.atlassian.com/it/software/jira/guides/getting-started/introducti \- on\#what-is-jira-software}  \\ (Ultimo accesso: 2024-05-06);
  \item Metodologia agile \\ \href{https://it.wikipedia.org/wiki/Metodologia_agile}{https://it.wikipedia.org/wiki/Metodologia\_agile}  \\ (Ultimo accesso: 2024-05-15);
  \item Retrospettiva di uno sprint - Punti di discussione \\ \href{https://www.humanwareonline.com/project-management/center/retrospettiva-sprint-scrum}{https://www.humanwareonline.com/project-management/center/retrospettiva-sprint-scrum}  \\ (Ultimo accesso: 2024-05-15);
  \item Scrum Agile: 4 tipologie di meeting \\ \href{https://www.coachingroup.it/blog/scrum-agile-4-tipologie-di-meeting-per-un-progetto-vincente}{https://www.coachingroup.it/blog/scrum-agile-4-tipologie-di-meeting-per-un-progetto-vincente}  \\ (Ultimo accesso: 2024-06-10);
  \item Cos'è il debito tecnico e come affrontarlo in modo Agile \\ \href{https://www.productheroes.it/cosa-e-debito-tecnico-agile}{https://www.productheroes.it/cosa-e-debito-tecnico-agile}  \\ (Ultimo accesso: 2024-06-10);
  \item Integrazione continua \\ \href{https://www.atlassian.com/it/agile/software-development/continuous-integration}{https://www.atlassian.com/it/agile/software-development/continuous-integr \- ation}  \\ (Ultimo accesso: 2024-07-02);
  \item \PianoDiQualifica;
  \item \Glossario;
  \item Verbali interni:
  \begin{itemize}
    \item 2024-04-03;
    \item 2024-04-10;
    \item 2024-04-16;
    \item 2024-04-20;
    \item 2024-04-25;
    \item 2024-05-02;
    \item 2024-05-07;
    \item 2024-05-16;
    \item 2024-05-23;
    \item 2024-05-28;
    \item 2024-06-03;
    \item 2024-06-14;
    \item 2024-06-22;
    \item 2024-07-06;
    \item 2024-07-10;
    \item 2024-07-18.
  \end{itemize}
  \item Verbali esterni:
  \begin{itemize}
    \item 2024-04-09;
    \item 2024-05-06;
    \item 2024-05-22;
    \item 2024-06-07;
    \item 2024-07-09.
  \end{itemize}
\end{itemize}

\subsection{Glossario} 
\GlossarioIntroduzione

\subsection{Note organizzative}
Il presente documento viene costantemente aggiornato e raffinato per riflettere lo stato attuale del progetto e i suoi incrementi, permettendo di mantenere il focus sulle attività e impedendo al contempo il riproporsi di problematiche già affrontate.
