\subsubsubsection{Indice Gulpease}
\begin{itemize}
    \item \textbf{Notazione specifica}: M.PD.4;
    \item \textbf{Nome}: Indice Gulpease;
    \item \textbf{Descrizione}: Questa metrica misura l'indice di leggibilità di un testo in lingua italiana. I valori sono compresi tra 0 e 100, dove 100 indica una leggibilità più alta mentre 0 una leggibilità più bassa. I punteggi si dividono in:
    \begin{itemize}
        \item punteggi inferiori a 80: difficili da leggere per chi possiede la licenza elementare;
        \item punteggi inferiori a 60: difficili da leggere per chi possiede la licenza media;
        \item punteggi inferiori a 40: difficili da leggere per chi possiede un diploma superiore.
    \end{itemize}
    \item \textbf{Caratteristica di riferimento}: Usabilità, comprensibilità;
    \item \textbf{Motivo}: Garantire che i documenti di progetto siano comprensbili per la maggior parte dei lettori;
    \item \textbf{Misurazione}:
    \[
        \text{Indice Gulpease} = 89 + \frac{{300 \cdot {{N_f}} - 10 \cdot {{N_l}}}}{{{{N_p}}}}
    \]
    dove:
    \begin{itemize}
        \item $N_{f}$: Numero totale di frasi;
        \item $N_{l}$: Numero totale di lettere;
        \item $N_{p}$: Numero totale di parole.
    \end{itemize}
\end{itemize}
