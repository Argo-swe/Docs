\subsubsubsection{Linee medie di codice per metodo}
\begin{itemize}
    \item \textbf{Notazione specifica}: M.PD.14;
    \item \textbf{Nome}: Linee medie di codice per metodo;
    \item \textbf{Descrizione}: Questa metrica misura la lunghezza media, in termini di linee di codice, dei metodi o funzioni all'interno del codice sorgente;
    \item \textbf{Caratteristiche}: Manutenibilità, testabilità, modificabilità, comprensibilità;
    \item \textbf{Motivo}: Verificare che non vi sia mancanza di modularità e chiarezza nel codice. Funzioni e metodi più corti sono generalmente preferibili perché risultano più semplici da comprendere, testare e mantenere; 
    \item \textbf{Misurazione}: Il calcolo viene effettuato con strumenti di analisi statica.
\end{itemize}
