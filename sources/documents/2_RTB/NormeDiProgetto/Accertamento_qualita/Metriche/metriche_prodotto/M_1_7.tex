\subsubsubsection{Core size}
\begin{itemize}
    \item \textbf{Notazione specifica}: M.1.7;
    \item \textbf{Nome}: Core size;
    \item \textbf{Descrizione}: I core file sono file altamente interconnessi da una catena di dipendenze cicliche, i quali sono maggiormente propensi ad avere difetti. Core size è la percentuale di file con una o più dipendenze che hanno un alto fan-in (numero di moduli che dipendono da essi) ed un alto fan-out (numero di moduli da cui dipendono). Questi file sono critici per la stabilità e la manutenibilità del sistema poiché la loro complessità e interconnessione li rendono più vulnerabili a difetti e più difficili da modificare senza introdurre errori;
    \item \textbf{Caratteristica di riferimento}: Affidabilità, Manutenibilità;
    \item \textbf{Motivo}: Identificare i file critici nel sistema che sono suscettibili a difetti a causa della loro alta interconnessione e complessità. Monitorare il core size aiuta a prevenire la formazione di "nodi critici" che possono influire negativamente sulla qualità del software;
    \item \textbf{Misurazione}: Il calcolo di questa metrica viene effettuato attraverso la libreria networkx per analizzare le dipendenze.
\end{itemize}
