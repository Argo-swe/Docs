\subsubsubsection{Completezza descrittiva}
\begin{itemize}
    \item \textbf{Notazione specifica}: M.1.11;
    \item \textbf{Nome}: Completezza descrittiva;
    \item \textbf{Descrizione}: La completezza descrittiva è una misura della qualità e dell'esaustività dei commenti e della documentazione all'interno del codice sorgente;
    \item \textbf{Caratteristica di riferimento}: Manutenibilità;
    \item \textbf{Motivo}: Garantire che il codice sia ben documentato è cruciale per la manutenibilità e la facilità di comprensione del software. Una documentazione completa aiuta gli sviluppatori a comprendere meglio il funzionamento e la logica del codice, riducendo il tempo necessario per la manutenzione e l'aggiornamento;
    \item \textbf{Misurazione}: La completezza descrittiva è ottenuta tramite questa formula:
    \[
    \text{Completezza descrittiva} = \frac{L_{com}}{L_{t}} \times 100
    \]
    Dove:
    \begin{itemize}
        \item $L_{com}$: Numero di linee di commento;
        \item $L_{t}$: Numero totale delle righe del codice.
    \end{itemize}
\end{itemize}
