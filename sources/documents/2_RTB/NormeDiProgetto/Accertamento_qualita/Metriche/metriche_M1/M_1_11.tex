\subsubsubsection{Indice di manutenibilità}
\begin{itemize}
    \item \textbf{Notazione specifica}: M.1.11
    \item \textbf{Nome}: Indice di manutenibilità
    \item \textbf{Descrizione}: L'indice di manutenibilità è una misura che riflette quanto sia facile mantenere, comprendere, modificare e correggere il codice sorgente;
    \item \textbf{Caratteristica di riferimento}: Manutenibilità;
    \item \textbf{Motivo}: Fornire una misura quantitativa della manutenibilità del software, aiutando a identificare aree del codice che potrebbero richiedere rifattorizzazione o miglioramenti per ridurre i costi di manutenzione e migliorare la qualità del software;
    \item \textbf{Misurazione}: Il calcolo dell'indice di manutenibilità è effettuato utilizzando strumenti di analisi del codice come SonarQube, che integra l'analisi della complessità ciclomatica, la dimensione del codice e la documentazione.
    %Un esempio di formula utilizzata per calcolare l'indice di manutenibilità è:
    %\begin{verbatim}
    %MI = 171 - 5.2 * ln(LOC) - 0.23 * CC - 16.2 * ln(commenti)
    %\end{verbatim}
    %dove:
    %\begin{itemize}
    %%    \item \textbf{LOC}: Linee di codice
      %  \item \textbf{CC}: Complessità ciclomatica
       % \item \textbf{commenti}: Numero di linee di commento
    %\end{itemize}
    %\item \textbf{Range}: 0 - 100
    %\item \textbf{Spiegazione}: Un range ottimale per l'indice di manutenibilità dovrebbe essere mantenuto tra 70 e 100. Un valore superiore a 70 indica che il codice è facilmente manutenibile, mentre un valore inferiore suggerisce che il codice potrebbe essere difficile da mantenere e potrebbe beneficiare di una rifattorizzazione per migliorare la leggibilità e la facilità di manutenzione. Mantenere questa metrica entro il range suggerito aiuta a garantire che il codice sia sostenibile a lungo termine.
\end{itemize}
