\subsubsubsection{Linee medie di codice per metodo}
\begin{itemize}
    \item \textbf{Notazione specifica}: M.1.12;
    \item \textbf{Nome}: Linee medie di codice per metodo;
    \item \textbf{Descrizione}: Le linee medie di codice per metodo è una misura che indica la lunghezza media, in termini di linee di codice, dei metodi o funzioni all'interno del codice sorgente;
    \item \textbf{Caratteristica di riferimento}: Manutenibilità;
    \item \textbf{Motivo}: Valutare la leggibilità e la manutenibilità del codice. Metodi più corti sono generalmente preferibili perché tendono a essere più semplici da comprendere, testare e mantenere. Identificare metodi eccessivamente lunghi può aiutare a focalizzare gli sforzi di rifattorizzazione;
    \item \textbf{Misurazione}: Il calcolo di questa metrica è effettuato con lo strumento SonarQube.
    %\item \textbf{Range}: 5 - 20 linee
    %\item \textbf{Spiegazione}: Un range ottimale per le linee medie di codice per metodo dovrebbe essere mantenuto tra 5 e 20 linee. Metodi più corti tendono ad essere più leggibili e manutenibili. Se la media supera le 20 linee, può indicare che i metodi sono troppo lunghi e complessi, rendendo il codice più difficile da comprendere e mantenere. Mantenere questa metrica entro il range suggerito aiuta a garantire una buona manutenibilità e leggibilità del codice.
\end{itemize}
