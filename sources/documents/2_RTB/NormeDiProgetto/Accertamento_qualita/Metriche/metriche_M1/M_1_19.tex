\subsubsubsection{McCabe’s Cyclomatic Number}
\begin{itemize}
    \item \textbf{Notazione specifica}: M.1.19;
    \item \textbf{Nome}: McCabe’s Cyclomatic Number;
    \item \textbf{Descrizione}: McCabe’s Cyclomatic Number è una metrica che valuta la complessità ciclomatica di un programma software. Misura il numero di cammini linearmente indipendenti attraverso il grafo di controllo di flusso di un programma, contando il numero di decisioni logiche indipendenti presenti nel codice;
    \item \textbf{Caratteristica di riferimento}: Manutenibilità, Complessità;
    \item \textbf{Motivo}: McCabe’s Cyclomatic Number fornisce un'indicazione della complessità strutturale del codice sorgente, consentendo di valutare la sua manutenibilità. Un valore più alto indica una maggiore complessità, che può rendere il codice più difficile da comprendere, testare e mantenere nel tempo;
    \item \textbf{Misurazione}: TODO.
    %La complessità ciclomatica è calcolata contando il numero di regioni fondamentali (nodi di decisione) nel grafo di controllo del flusso del programma e aggiungendo 1 al risultato finale. Questo valore rappresenta McCabe’s Cyclomatic Number.
    %\item \textbf{Range}: Dipende dal contesto dell'applicazione e dalle linee guida specifiche, ma valori superiori a 10 indicano una complessità significativa del codice.
    %\item \textbf{Spiegazione}: McCabe’s Cyclomatic Number è utile per identificare le aree di potenziale rischio nel codice e guidare la manutenzione e il refactoring per migliorarne la qualità e la manutenibilità.
\end{itemize}
