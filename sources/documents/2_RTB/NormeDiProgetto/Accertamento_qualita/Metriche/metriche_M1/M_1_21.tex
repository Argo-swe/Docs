\subsubsubsection{Requisiti desiderabili soddisfatti}
\begin{itemize}
    \item \textbf{Notazione specifica}: M.1.21;
    \item \textbf{Nome}: Requisiti desiderabili soddisfatti;
    \item \textbf{Descrizione}: Questa metrica indica la percentuale di requisiti desiderabili soddisfatti dal software rispetto al totale dei requisiti desiderabili definiti nel documento di specifica dei requisiti;
    \item \textbf{Caratteristica di riferimento}: Funzionalità, Qualità;
    \item \textbf{Motivo}: Misura il grado di adempimento dei requisiti desiderabili, fornendo una valutazione della completezza del software rispetto alle richieste aggiuntive specificate;
    \item \textbf{Misurazione}: La percentuale di requisiti desiderabili soddisfatti è calcolata utilizzando la seguente formula:
    \[
    \text{Requisiti desiderabili soddisfatti (\%)} = \frac{R_{d}}{T_{d}} \times 100
    \]
    dove:
    \begin{itemize}
        \item $R_{o}$: Numero di requisiti desiderabili soddisfatti;
        \item $T_{d}$: Numero totale dei requisiti desiderabili.
    \end{itemize}
    
    %\item \textbf{Range}: Il range ideale per questa metrica è dal 100% (tutti i requisiti desiderabili soddisfatti) al 0% (nessun requisito desiderabile soddisfatto).
    %\item \textbf{Spiegazione}: Una percentuale più alta di requisiti desiderabili soddisfatti indica una maggiore capacità del software di rispondere a richieste aggiuntive o di migliorare le funzionalità oltre le specifiche di base.
\end{itemize}
