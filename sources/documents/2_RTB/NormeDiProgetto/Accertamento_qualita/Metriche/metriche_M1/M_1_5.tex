\subsubsubsection{Gestione delle operazioni non permesse}
\begin{itemize}
    \item \textbf{Notazione specifica}: M.1.5;
    \item \textbf{Nome}:  Gestione delle operazioni non permesse;
    \item \textbf{Descrizione}: Appresenta la percentuale di operazioni non consentite o non gestite correttamente dal software durante l'esecuzione;
    \item \textbf{Caratteristica di riferimento}: Affidabilità e robustezza;
    \item \textbf{Motivo}: La gestione delle operazioni non permesse è stata introdotta per valutare la robustezza e l'affidabilità del software nell'affrontare input imprevisti o non validi. Misura quanto bene il sistema sia in grado di gestire eccezioni e situazioni anomale senza interrompere o compromettere il normale flusso di esecuzione;
    \item \textbf{Misurazione}: Viene calcolata come:
    \[
    \text{Gestione delle operazioni non permesse (\%)} = \frac{N_{np}}{N_{t}} \times 100
    \]
    
    dove:
    \begin{itemize}
        \item $N_{np}$: rappresenta il numero di operazioni non permesse o non gestite correttamente;
        \item $N_{t}$: rappresenta il numero totale di operazioni eseguite durante il test.
    \end{itemize}
\end{itemize}