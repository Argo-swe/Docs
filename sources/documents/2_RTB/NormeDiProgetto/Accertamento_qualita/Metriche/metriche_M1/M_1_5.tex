\subsubsection{Functional Size Measurement}
\begin{itemize}
    \item \textbf{Notazione specifica}: M.1.5
    \item \textbf{Nome}: Functional Size Measurement
    \item \textbf{Descrizione}: È una metrica utilizzata per quantificare la dimensione funzionale di un sistema software in base alle sue funzionalità o requisiti funzionali;
    \item \textbf{Caratteristica di riferimento}: Dimensionamento;
    \item \textbf{Motivo}: È stata introdotta per valutare la dimensione funzionale del software in modo da poter stimare il lavoro richiesto per lo sviluppo, il testing e la manutenzione del sistema;
    \item \textbf{Misurazione}: La complessità totale del software può essere calcolata utilizzando la seguente formula:
    \[
    \text{Functional Size Measurement} = \frac{\sum_{i=1}^{n} P_i}{n}
    \]
    Dove:
    \begin{itemize}
        \item $n$: numero totale di funzioni nel software.
        \item $P_i$: punteggio assegnato alla funzione $i$, con $i$ che varia da 1 a $n$. Ogni $P_i$ può essere compreso tra 1 e 5, inclusi.
    \end{itemize}
    %\item \textbf{Range}: 2-3 punteggio di complessità;
    %\item \textbf{Modifiche nel tempo}: 
    %\item \textbf{Azioni Correttive}:
    %\begin{itemize}
       % \item Verificare la corretta identificazione e categorizzazione delle funzioni all'interno del sistema per garantire una misurazione accurata della dimensione funzionale.
        %\item Effettuare una revisione periodica della classificazione delle funzioni per adattarsi a eventuali cambiamenti nei requisiti del software.
        %\item Utilizzare strumenti e metodologie standardizzati per il calcolo della dimensione funzionale al fine di garantire coerenza e riproducibilità.
        %\item Formare adeguatamente il personale coinvolto nella misurazione della dimensione funzionale per assicurare una comprensione uniforme e un'applicazione corretta delle procedure.
    %\end{itemize}
    %\item \textbf{Obiettivi di Miglioramento}:
    %\begin{itemize}
        %\item Ottimizzare la precisione della misurazione della dimensione funzionale riducendo al minimo gli errori di categorizzazione e valutazione delle funzioni.
        %\item Mantenere un sistema di classificazione delle funzioni flessibile e aggiornato per riflettere accuratamente l'evoluzione dei requisiti del software.
        %\item Promuovere l'adozione diffusa di standard e best practice per il calcolo della dimensione funzionale all'interno dell'organizzazione.
        %\item Continuare a migliorare le competenze del personale riguardo alla misurazione della dimensione funzionale attraverso la formazione e l'apprendimento continuo.
    %\end{itemize}
\end{itemize}
