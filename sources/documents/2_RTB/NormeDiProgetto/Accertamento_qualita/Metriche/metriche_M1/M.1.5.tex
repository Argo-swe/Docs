\subsubsection{Functional Size Measurement}
\begin{itemize}
    \item \textbf{Notazione specifica}:M.1.5
    \item \textbf{Nome}: Functional Size Measurement;
    \item \textbf{Descrizione}: è una metrica utilizzata per quantificare la dimensione funzionale di un sistema software in base alle sue funzionalità o requisiti funzionali. Misura la complessità e l'entità delle funzionalità implementate nel software, indipendentemente dall'implementazione tecnica o dal linguaggio di programmazione utilizzato;
    \item \textbf{Caratteristica di riferimento}: si concentra sulla caratteristica di "Sizing", nella quale si valuta la dimensione e la complessità del software.E’ definita nello standard  ISOG /IECG 14143;
    \item \textbf{Motivo}: è stata introdotta per valutare la dimensione funzionale del software in modo da poter stimare il lavoro richiesto per lo sviluppo, il testing e la manutenzione del sistema;
    \item \textbf{Misurazione}: la complessità totale del software può essere calcolata utilizzando la seguente formula:
    \[
    \text{Functional Size Measurement} = \frac{\sum_{i=1}^{n} P_i}{n}
    \]
    
    Dove:
    \begin{itemize}
        \item $n$: numero totale di funzioni nel software.
        \item $P_i$: punteggio assegnato alla funzione $i$, con $i$ che varia da 1 a $n$. Ogni $P_i$ può essere compreso tra 1 e 5, inclusi.
    \end{itemize}

    \item \textbf{Range}: 2-3 punteggio di complessità;
    %\item \textbf{Modifiche nel tempo}:  
\end{itemize}