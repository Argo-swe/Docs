\subsubsubsection{Code coverage}
\begin{itemize}
    \item \textbf{Notazione specifica}: M.1.1;
    \item \textbf{Nome}: Code coverage;
    \item \textbf{Descrizione}: Serve per valutare la percentuale di codice sorgente di un'applicazione software che è stata eseguita durante l'esecuzione dei test automatizzati. Indica quindi la quantità di codice che viene testata rispetto alla totalità del codice sorgente. Una copertura topologica del test del 100\% di tipo code coverage garantisce di aver eseguito almeno una volta tutte le istruzioni, ma non tutti i rami;
    \item \textbf{Caratteristica di riferimento}: Affidabilità e manutenibilità del software;
    \item \textbf{Motivo}: È stata introdotta per valutare l'efficacia dei test automatizzati nel garantire la correttezza e l'affidabilità del software. Una copertura del codice elevata suggerisce una maggiore confidenza nella stabilità e nella qualità del software;
    \item \textbf{Misurazione}: In Python è stato utilizzato lo strumento Coverage.py che permette di eseguire test del codice sorgente e generare report sulla copertura del codice.

\end{itemize}
