\subsubsubsection{Modified condition/decision coverage}
\begin{itemize}
    \item \textbf{Notazione specifica}: M.1.2
    \item \textbf{Nome}: Modified condition/decision coverage
    \item \textbf{Descrizione}: È una combinazione delle metriche di function coverage (copertura delle funzioni chiamate) e branch coverage (copertura dei \glossario{branch} delle strutture di controllo). Questa metrica richiede che ogni punto di entrata o uscita in un programma sia invocato almeno una volta e che per ogni decisione condizionale vengano considerati tutti i possibili esiti. La versione modified richiede inoltre che entrambe le coperture siano soddisfatte, ed in particolare che ogni condizione influenzi gli esiti condizionali indipendentemente;
    \item \textbf{Caratteristica di riferimento}: Affidabilità e correttezza del software;
    \item \textbf{Motivo}: Per garantire una copertura completa e accurata delle condizioni e delle decisioni nel codice sorgente. Ciò aiuta a ridurre il rischio di errori di logica nel software e a garantire che tutte le possibili combinazioni di risultati di condizioni siano testate in modo esaustivo;
    \item \textbf{Misurazione}: I valori si prendono dallo strumento di analisi per Python \glossario{Coverage.py};
    %\item \textbf{Range}: >80\%
    %\item \textbf{Modifiche nel tempo}: 
    %\item \textbf{Azioni Correttive}:
    %\begin{itemize}
        %\item Identificare le decisioni condizionali non completamente coperte e sviluppare test specifici per coprire tutte le combinazioni di condizioni.
        %\item Migliorare i casi di test esistenti per garantire che ogni condizione influenzi l'esito della decisione in modo indipendente.
        %\item Eseguire revisione del codice per individuare condizioni complesse che potrebbero richiedere una maggiore attenzione nei test.
        %\item Implementare tecniche di test-driven development (TDD) per migliorare la copertura delle condizioni e delle decisioni durante lo sviluppo.
    %\end{itemize}
    %\item \textbf{Obiettivi di Miglioramento}:
    %\begin{itemize}
        %\item Aumentare la copertura delle condizioni e delle decisioni fino a raggiungere e mantenere un livello accettabile (>80\%).
        %\item Garantire che tutte le nuove funzionalità e le modifiche al codice siano accompagnate da test adeguati per la copertura delle condizioni e delle decisioni.
        %\item Ridurre il numero di errori di logica nel software migliorando la qualità complessiva del codice.
        %\item Promuovere una cultura di test continuo e rigoroso all'interno del team di sviluppo per mantenere elevata la copertura delle condizioni e delle decisioni nel tempo.
    %\end{itemize}
\end{itemize}
