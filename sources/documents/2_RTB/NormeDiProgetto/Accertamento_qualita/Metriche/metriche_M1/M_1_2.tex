\subsubsubsection{Test eseguiti su totali}
\begin{itemize}
    \item \textbf{Notazione specifica}: M.1.2;
    \item \textbf{Nome}: Test eseguiti su totali;
    \item \textbf{Descrizione}: Questa metrica indica la percentuale di test effettivamente eseguiti rispetto al numero totale di test pianificati per il testing descritti nel Piano di Qualifica;
    \item \textbf{Caratteristica di riferimento}: Efficacia;
    \item \textbf{Motivo}: È stata introdotta per valutare l'efficacia dell'esecuzione dei test pianificati durante il processo di sviluppo del software. Misura quanto bene il team rispetti il piano di test stabilito e se vengono eseguite tutte le attività di testing pianificate;
    \item \textbf{Misurazione}: La metrica viene calcolata utilizzando la seguente formula:
    \[
    \text{Test eseguiti su totali (\%)} = \frac{N_{te}}{N_{tt}} \times 100
    \]
    dove:
    \begin{itemize}
        \item $N_{te}$ rappresenta il numero di test effettivamente eseguiti;
        \item $N_{tt}$ rappresenta il numero totale di test pianificati.
    \end{itemize}
   
\end{itemize}
