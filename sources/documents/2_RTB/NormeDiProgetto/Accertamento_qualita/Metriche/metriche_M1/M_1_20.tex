\subsubsubsection{Requisiti obbligatori soddisfatti}
\begin{itemize}
    \item \textbf{Notazione specifica}: M.1.20
    \item \textbf{Nome}: Requisiti obbligatori soddisfatti
    \item \textbf{Descrizione}: Questa metrica indica la percentuale di requisiti obbligatori soddisfatti dal software rispetto al totale dei requisiti obbligatori definiti nel documento di specifica dei requisiti.
    \item \textbf{Caratteristica di riferimento}: Funzionalità, Qualità;
    \item \textbf{Motivo}: Misura il grado di adempimento dei requisiti essenziali per il funzionamento del sistema, fornendo una valutazione della completezza del software rispetto alle specifiche stabilite.
    \item \textbf{Misurazione}: La percentuale di requisiti obbligatori soddisfatti può essere calcolata utilizzando la seguente formula:
    \[ Requisiti obbligatori soddisfatti= \frac{Ro}{To} \times 100\% \]
    Dove:
    \begin{itemize}
        \item \textbf{Ro}: Requisiti obbligatori soddisfatti;
        \item \textbf{To}: Totale dei requisiti obbligatori.
    \end{itemize}
    %\item \textbf{Range}: Il range ideale per questa metrica è dal 100% (tutti i requisiti obbligatori soddisfatti) al 0% (nessun requisito obbligatorio soddisfatto).
    %\item \textbf{Spiegazione}: Una percentuale più alta di requisiti obbligatori soddisfatti indica una maggiore aderenza del software alle specifiche, garantendo un maggiore grado di completezza e funzionalità.
\end{itemize}
