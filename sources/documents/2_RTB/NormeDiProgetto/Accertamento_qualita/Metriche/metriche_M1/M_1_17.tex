\subsubsection{Tempo di risposta}
\begin{itemize}
    \item \textbf{Notazione specifica}: M.1.17
    \item \textbf{Nome}: Tempo di risposta
    \item \textbf{Descrizione}: Fa parte delle metriche esterne di comportamento rispetto al tempo necessario; serve per misurare l’efficienza del prodotto. Il tempo di risposta indica il tempo medio in cui il sistema risponde ad un comando immesso dall’utente;
    \item \textbf{Caratteristica di riferimento}: Efficienza-Comportamento;
    \item \textbf{Motivo}: Misurare e migliorare il tempo medio in cui il sistema risponde ad un comando immesso dall’utente;
    \item \textbf{Misurazione}: Il tempo di risposta (M) è calcolato moltiplicando il tempo (T) che intercorre tra l’immissione del comando da parte dell’operatore e la presentazione della risposta del sistema: \( M = T \).
    %\item \textbf{Range}: Dipende dal contesto dell'applicazione e dalle aspettative degli utenti. In generale, un tempo di risposta inferiore a 2 secondi è considerato accettabile per molte applicazioni interattive.
\end{itemize}
