\subsubsubsection{Accessibilità}
\begin{itemize}
    \item \textbf{Notazione specifica}: M.1.16;
    \item \textbf{Nome}: Accessibilità;
    \item \textbf{Descrizione}: L'accessibilità è una misura che valuta la facilità con cui gli utenti, inclusi coloro con disabilità, possono interagire con il sistema software;
    \item \textbf{Caratteristica di riferimento}: Usabilità, Accessibilità;
    \item \textbf{Motivo}: Garantire che il sistema sia accessibile a tutti gli utenti, indipendentemente dalle loro capacità fisiche o cognitive, è fondamentale per fornire un'esperienza utente inclusiva e soddisfacente;
    \item \textbf{Misurazione}: L'accessibilità può essere valutata utilizzando strumenti automatizzati di valutazione dell'accessibilità come Axe, che analizza l'accessibilità dell'interfaccia utente rispetto alle linee guida WCAG (Web Content Accessibility Guidelines).
    %\item \textbf{Range}: Conforme / Non conforme agli standard di accessibilità
    %\item \textbf{Spiegazione}: Un sistema è considerato conforme agli standard di accessibilità se rispetta le linee guida e i criteri stabiliti dagli standard di accessibilità pertinenti (ad esempio, WCAG). Mantenere l'accessibilità conforme agli standard aiuta a garantire che il sistema sia accessibile a una vasta gamma di utenti, migliorando l'esperienza utente complessiva e dimostrando un impegno per l'inclusione e l'equità.
\end{itemize}
