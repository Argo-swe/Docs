\subsubsubsection{Requisiti opzionali soddisfatti}
\begin{itemize}
    \item \textbf{Notazione specifica}: M.1.16;
    \item \textbf{Nome}: Requisiti opzionali soddisfatti;
    \item \textbf{Descrizione}: Questa metrica indica la percentuale di requisiti opzionali soddisfatti dal software rispetto al totale dei requisiti opzionali definiti nel documento di specifica dei requisiti;
    \item \textbf{Caratteristica di riferimento}: Funzionalità, Qualità;
    \item \textbf{Motivo}: Misura il grado di adempimento dei requisiti opzionali, fornendo una valutazione della completezza del software rispetto alle richieste supplementari specificate;
    \item \textbf{Misurazione}: La percentuale di requisiti opzionali soddisfatti è calcolata utilizzando la seguente formula:
    \[
    \text{Requisiti opzionali soddisfatti (\%)} = \frac{R_{o}}{T_{o}} \times 100
    \]
    dove:
    \begin{itemize}
        \item $R_{o}$: Numero di requisiti opzionali soddisfatti;
        \item $T_{o}$: Numero totale dei requisiti opzionali.
    \end{itemize}
\end{itemize}
