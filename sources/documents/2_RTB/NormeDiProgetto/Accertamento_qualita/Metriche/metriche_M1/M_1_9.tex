\subsubsubsection{Linee medie di codice per metodo}
\begin{itemize}
    \item \textbf{Notazione specifica}: M.1.9;
    \item \textbf{Nome}: Linee medie di codice per metodo;
    \item \textbf{Descrizione}: Le linee medie di codice per metodo è una misura che indica la lunghezza media, in termini di linee di codice, dei metodi o funzioni all'interno del codice sorgente;
    \item \textbf{Caratteristica di riferimento}: Manutenibilità;
    \item \textbf{Motivo}: Valutare la leggibilità e la manutenibilità del codice. Metodi più corti sono generalmente preferibili perché tendono a essere più semplici da comprendere, testare e mantenere. Identificare metodi eccessivamente lunghi può aiutare a focalizzare gli sforzi di rifattorizzazione;
    \item \textbf{Misurazione}: Il calcolo di questa metrica è effettuato con lo strumento SonarQube.
\end{itemize}
