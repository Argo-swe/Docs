\subsubsection{Test superati}
\begin{itemize}
    \item \textbf{Notazione specifica}:
    \item \textbf{Nome}: Test superati;
    \item \textbf{Descrizione}:questa metrica indica la percentuale di test che sono stati superati con successo rispetto al numero totale di test eseguiti;
    \item \textbf{Caratteristica di riferimento}: TODO
    \item \textbf{Motivo}: è stata introdotta per valutare l'efficacia dei test eseguiti nel rilevare eventuali difetti nel software. Misura quanto bene i test eseguiti stiano identificando e segnalando le eventuali anomalie nel comportamento del software;
    \item \textbf{Misurazione}: la metrica viene calcolata utilizzando la seguente formula:
        \[
        \text{Test eseguiti su totali (\%)} = \frac{N_{ts}}{N_{te}} \times 100
        \]
        
        Dove:
        \begin{itemize}
            \item $N_{ts}$ rappresenta il numero totale di test che sono stati superati con successo.
            \item $N_{te}$ rappresenta il numero di test effettivamente eseguiti.
        \end{itemize}
    \item \textbf{Range}: 90-100\%
    %\item \textbf{Modifiche nel tempo}:  
\end{itemize}