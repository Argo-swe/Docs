\subsubsubsection{Test eseguiti su totali}
\begin{itemize}
    \item \textbf{Notazione specifica}: M.1.3
    \item \textbf{Nome}: Test eseguiti su totali
    \item \textbf{Descrizione}: Questa metrica indica la percentuale di test effettivamente eseguiti rispetto al numero totale di test pianificati per il testing descritti nel Piano di Qualifica;
    \item \textbf{Caratteristica di riferimento}: Efficacia;
    \item \textbf{Motivo}: È stata introdotta per valutare l'efficacia dell'esecuzione dei test pianificati durante il processo di sviluppo del software. Misura quanto bene il team di sviluppo e di testing rispetti il piano di test stabilito e se vengono eseguite tutte le attività di testing pianificate;
    \item \textbf{Misurazione}: La metrica viene calcolata utilizzando la seguente formula:
    \[
    \text{Test eseguiti su totali (\%)} = \frac{N_{te}}{N_{tt}} \times 100
    \]
    dove:
    \begin{itemize}
        \item $N_{te}$ rappresenta il numero di test effettivamente eseguiti.
        \item $N_{tt}$ rappresenta il numero totale di test pianificati.
    \end{itemize}
    %\item \textbf{Range}: 90-100\%
    %\item \textbf{Modifiche nel tempo}: 
    %\item \textbf{Azioni Correttive}:
    %\begin{itemize}
        %\item Analizzare i test non eseguiti per identificare le cause (es. mancanza di tempo, risorse insufficienti, problemi tecnici) e risolvere i problemi.
        %\item Migliorare la pianificazione dei test per assicurare che tutti i test necessari possano essere eseguiti entro le scadenze stabilite.
        %\item Automatizzare il più possibile l'esecuzione dei test per aumentare la copertura e ridurre il carico di lavoro manuale.
        %\item Monitorare regolarmente l'avanzamento dell'esecuzione dei test per intervenire tempestivamente in caso di ritardi.
    %\end{itemize}
    %\item \textbf{Obiettivi di Miglioramento}:
    %\begin{itemize}
        %\item Aumentare la percentuale di test eseguiti su totali fino a raggiungere e mantenere un livello del 100\%.
        %\item Garantire che tutti i test pianificati siano eseguiti in modo completo e accurato.
        %\item Migliorare l'affidabilità del processo di testing assicurando che tutte le funzionalità del software siano adeguatamente testate.
        %\item Stabilire una cultura di test rigorosa e disciplinata all'interno del team di sviluppo per mantenere elevata l'efficacia dei test nel tempo.
    %\end{itemize}
\end{itemize}
