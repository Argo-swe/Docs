\subsubsubsection{Test superati}
\begin{itemize}
    \item \textbf{Notazione specifica}: M.1.3;
    \item \textbf{Nome}: Test superati;
    \item \textbf{Descrizione}: Questa metrica indica la percentuale di test che sono stati superati con successo rispetto al numero totale di test eseguiti;
    \item \textbf{Caratteristica di riferimento}: Efficacia dei test nel rilevare difetti;
    \item \textbf{Motivo}: È stata introdotta per valutare l'efficacia dei test eseguiti nel rilevare eventuali difetti nel software. Misura quanto bene i test eseguiti stiano identificando e segnalando le eventuali anomalie nel comportamento del software;
    \item \textbf{Misurazione}: La metrica viene calcolata utilizzando la seguente formula:
    \[
    \text{Test superati (\%)} = \frac{N_{ts}}{N_{te}} \times 100
    \]
    dove:
    \begin{itemize}
        \item $N_{ts}$ rappresenta il numero totale di test che sono stati superati con successo;
        \item $N_{te}$ rappresenta il numero di test effettivamente eseguiti.
    \end{itemize}
    
\end{itemize}
