\subsubsection{Accuratezza rispetto alle attese}
\begin{itemize}
    \item \textbf{Notazione specifica}: M.1.6
    \item \textbf{Nome}: Accuratezza rispetto alle attese;
    \item \textbf{Descrizione}: rappresenta la percentuale di risultati dei test che rispettano quanto previsto. Misura l'aderenza dei risultati dei test alle aspettative stabilite durante la fase di pianificazione dei test;
    \item \textbf{Caratteristica di riferimento}:affidabilità e conformità del software ai requisiti specificati;
    \item \textbf{Motivo}: l'accuratezza rispetto alle attese è stata introdotta per valutare quanto bene i risultati dei test corrispondano alle aspettative previste durante la fase di pianificazione dei test. Una percentuale alta indica una maggiore affidabilità del software nel produrre risultati conformi alle aspettative;
    \item \textbf{Misurazione}: viene calcolata utilizzando la seguente formula:
    \[
    (1 - \frac{N_{rd}}{N_{te}}) \times 100
    \]
    
    Dove:
    \begin{itemize}
        \item $N_{rd}$: numero di test che producono risultati discordanti;
        \item $N_{te}$: numero di test eseguiti.
    \end{itemize} 
    \item \textbf{Range}: 80-100\%
    %\item \textbf{Modifiche nel tempo}:  
\end{itemize}