\subsubsection{Test eseguiti su totali}
\begin{itemize}
    \item \textbf{Notazione specifica}: M.1.3
    \item \textbf{Nome}: Test eseguiti su totali;
    \item \textbf{Descrizione}:Questa metrica indica la percentuale di test effettivamente eseguiti rispetto al numero totale di test pianificati per il testing descritti nel PdQ(g). 
    \item \textbf{Caratteristica di riferimento}: TODO
    \item \textbf{Motivo}: è stata introdotta per valutare l'efficacia dell'esecuzione dei test pianificati durante il processo di sviluppo del software. Misura quanto bene il team di sviluppo e di testing rispetti il piano di test stabilito e se vengono eseguite tutte le attività di testing pianificate;
    \textbf{Misurazione:} la metrica viene calcolata utilizzando la seguente formula:
        \[
        \text{Test eseguiti su totali (\%)} = \frac{N_{te}}{N_{tt}} \times 100
        \]
        dove:
        \begin{itemize}
            \item $N_{te}$ rappresenta il numero di test effettivamente eseguiti.
            \item $N_{tt}$ rappresenta il numero totale di test pianificati.
        \end{itemize}
    \item \textbf{Range}: 90- 100\%
    %\item \textbf{Modifiche nel tempo}:  
\end{itemize}