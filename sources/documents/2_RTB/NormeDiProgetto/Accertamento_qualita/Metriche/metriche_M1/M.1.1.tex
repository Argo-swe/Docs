\subsubsection{Code coverage}
\begin{itemize}
    \item \textbf{Notazione specifica}: M.1.1
    \item \textbf{Nome}:Code coverage;
    \item \textbf{Descrizione}: serve per valutare la percentuale di codice sorgente di un'applicazione software che è stata eseguita durante l'esecuzione dei test automatizzati. Indica quindi la quantità di codice che viene testata rispetto alla totalità del codice sorgente.Una copertura topologica del test del 100\% di tipo code coverage garantisce di aver eseguito
    almeno una volta tutte le istruzioni, ma non tutti i rami;
    \item \textbf{Caratteristica di riferimento}: affidabilità e manutenibilità del software;
    \item \textbf{Motivo}: è stata introdotta per valutare l'efficacia dei test automatizzati nel garantire la correttezza e l'affidabilità del software. Una copertura del codice elevata suggerisce una maggiore confidenza nella stabilità e nella qualità del software;
    \item \textbf{Misurazione}: in python è stato utilizzato lo strumento coverage.py che permette di eseguire test dell codice sorgente e generare report sulla copertura del codice;
    \item \textbf{Range}: 80-100\%;
    \item \textbf{Modifiche nel tempo}: per il POC verrà accettato un range inferiore fino al 80\% mentre nello sviluppo del prodotto finale la code coverage dovrà essere del 100\%.Il range non ha quasi margine di esclusione poiché il codice da sviluppare sarà limitato e ridurre il rischio di bug è una nostra priorità. 
\end{itemize}