\subsubsubsection{Completezza descrittiva}
\begin{itemize}
    \item \textbf{Notazione specifica}: M.1.14
    \item \textbf{Nome}: Completezza descrittiva
    \item \textbf{Descrizione}: La completezza descrittiva è una misura della qualità e dell'esaustività dei commenti e della documentazione all'interno del codice sorgente;
    \item \textbf{Caratteristica di riferimento}: Manutenibilità;
    \item \textbf{Motivo}: Garantire che il codice sia ben documentato è cruciale per la manutenibilità e la facilità di comprensione del software. Una documentazione completa aiuta gli sviluppatori a comprendere meglio il funzionamento e la logica del codice, riducendo il tempo necessario per la manutenzione e l'aggiornamento;
    \item \textbf{Misurazione}: La completezza descrittiva può essere misurata calcolando il rapporto tra il numero di linee di commento e il numero totale di linee di codice. La formula per il calcolo è:
    \begin{equation}
    \text{Completezza descrittiva} = \left( \frac{\text{Lcom}}{\text{Lt}} \right) \times 100
    \end{equation}
    Dove:
    \begin{itemize}
        \item \textbf{Lcom}: Numero di linee di commento;
        \item \textbf{Lt}: Numero totale delle righe del codice.
    \end{itemize}
    %\item \textbf{Range}: 20\% - 40\%
    %\item \textbf{Spiegazione}: Un range ottimale per la completezza descrittiva dovrebbe essere mantenuto tra il 20\% e il 40\%. Un valore inferiore al 20\% indica che il codice potrebbe non essere adeguatamente documentato, rendendo difficile la comprensione e la manutenzione. Un valore superiore al 40\% potrebbe indicare un eccesso di commenti che potrebbe sovraccaricare il codice. Mantenere questa metrica entro il range suggerito aiuta a garantire una documentazione equilibrata e utile.
\end{itemize}
