\subsubsubsection{Fallimento dei test}
\begin{itemize}
    \item \textbf{Notazione specifica}: M.1.4;
    \item \textbf{Nome}: Fallimento dei test;
    \item \textbf{Descrizione}: Rappresenta la percentuale di test che non superano con successo i criteri di accettazione o di conformità definiti durante la fase di pianificazione dei test. Misura il grado di non conformità del software rispetto alle aspettative stabilite;
    \item \textbf{Caratteristica di riferimento}: Affidabilità e conformità del software ai requisiti specificati;
    \item \textbf{Motivo}: Il fallimento dei test è stato introdotto per valutare il numero e la percentuale di test che non superano con successo i criteri di accettazione o di conformità definiti durante la pianificazione dei test. Identifica i punti deboli del software e delle attività di testing, consentendo di concentrare gli sforzi di correzione e miglioramento dove sono più necessari;
    \item \textbf{Misurazione}: Viene calcolata utilizzando la seguente formula:
    \[
    \text{Fallimento dei test (\%)} = \frac{N_{fr}}{N_{te}} \times 100
    \]
    dove:
    \begin{itemize}
        \item $N_{fr}$: Numero di fallimenti ricevuti;
        \item $N_{te}$: Numero di test eseguiti.
    \end{itemize}
    \end{itemize}