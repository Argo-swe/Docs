\subsubsubsection{Test superati}
\begin{itemize}
    \item \textbf{Notazione specifica}: M.1.4;
    \item \textbf{Nome}: Test superati;
    \item \textbf{Descrizione}: Questa metrica indica la percentuale di test che sono stati superati con successo rispetto al numero totale di test eseguiti;
    \item \textbf{Caratteristica di riferimento}: Efficacia dei test nel rilevare difetti;
    \item \textbf{Motivo}: È stata introdotta per valutare l'efficacia dei test eseguiti nel rilevare eventuali difetti nel software. Misura quanto bene i test eseguiti stiano identificando e segnalando le eventuali anomalie nel comportamento del software;
    \item \textbf{Misurazione}: La metrica viene calcolata utilizzando la seguente formula:
    \[
    \text{Test superati (\%)} = \frac{N_{ts}}{N_{te}} \times 100
    \]
    dove:
    \begin{itemize}
        \item $N_{ts}$ rappresenta il numero totale di test che sono stati superati con successo;
        \item $N_{te}$ rappresenta il numero di test effettivamente eseguiti.
    \end{itemize}
    %\item \textbf{Range}: 90-100\%
    %\item \textbf{Modifiche nel tempo}: 
    %\item \textbf{Azioni Correttive}:
    %\begin{itemize}
        %\item Analizzare i test falliti per identificare le cause (es. errori nel codice, problemi di configurazione) e implementare le correzioni necessarie.
        %\item Rivedere e migliorare i casi di test per coprire meglio le funzionalità critiche del software e ridurre i falsi positivi.
        %\item Automatizzare i test, ove possibile, per aumentare la copertura e ridurre il rischio di errori manuali.
        %\item Condurre revisioni periodiche dei test per aggiornare i test esistenti e svilupparne di nuovi in base alle modifiche del software.
    %\end{itemize}
    %\item \textbf{Obiettivi di Miglioramento}:
    %\begin{itemize}
        %\item Aumentare la percentuale di test superati fino a raggiungere e mantenere un livello superiore al 95\%.
        %\item Garantire che i test coprano tutte le funzionalità critiche e rilevino efficacemente i difetti nel software.
        %\item Ridurre il numero di difetti rilevati durante il testing migliorando la qualità del software attraverso pratiche di sviluppo rigorose.
        %\item Promuovere una cultura di testing continuo e accurato all'interno del team di sviluppo per mantenere elevata l'efficacia dei test nel tempo.
    %\end{itemize}
\end{itemize}
