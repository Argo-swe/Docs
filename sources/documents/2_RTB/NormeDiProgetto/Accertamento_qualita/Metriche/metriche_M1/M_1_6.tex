\subsubsubsection{Numero di parametri per funzione}
\begin{itemize}
    \item \textbf{Notazione specifica}: M.1.6;
    \item \textbf{Nome}: Numero di parametri per funzione;
    \item \textbf{Descrizione}: Misura il numero medio di parametri passati alle funzioni nel codice sorgente del software;
    \item \textbf{Caratteristica di riferimento}: Manutenibilità e complessità del software;
    \item \textbf{Motivo}: Il numero di parametri per funzione è stato introdotto per valutare la complessità delle funzioni all'interno del software. Un numero elevato di parametri può indicare una scarsa progettazione modulare e può rendere il codice difficile da comprendere, testare e mantenere. Questa metrica aiuta a identificare funzioni che potrebbero beneficiare di una rifattorizzazione per migliorare la qualità del codice;
    \item \textbf{Misurazione}: La metrica viene calcolata utilizzando la seguente formula:
    \[
    \text{Numero di Parametri per Funzione} = \frac{N_{p}}{N_{f}}
    \]
    dove:
    \begin{itemize}
        \item $N_{p}$: Somma del numero di parametri di tutte le funzioni nel codice;
        \item $N_{f}$: Numero totale di funzioni nel codice.
    \end{itemize}
\end{itemize}
