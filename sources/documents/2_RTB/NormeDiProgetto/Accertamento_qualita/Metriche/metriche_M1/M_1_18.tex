\subsubsection{Efficienza dell'installazione}
\begin{itemize}
    \item \textbf{Notazione specifica}: M.1.18
    \item \textbf{Nome}: Efficienza dell'installazione
    \item \textbf{Descrizione}: L'efficienza dell'installazione è una metrica che valuta la facilità e la rapidità con cui il software può essere installato e configurato il sistema;
    \item \textbf{Caratteristica di riferimento}: Efficienza;
    \item \textbf{Motivo}: Un'installazione efficiente è cruciale per garantire che gli utenti possano iniziare a utilizzare il software rapidamente e senza difficoltà. Ridurre il tempo e lo sforzo necessari per installare il software migliora l'esperienza utente complessiva e riduce la possibilità di errori durante l'installazione;
    \item \textbf{Misurazione}: TODO.
    %\item \textbf{Range}: Dipende dal contesto dell'applicazione e dalle aspettative degli utenti. In generale, un processo di installazione che richiede meno di 5 minuti e viene completato senza la necessità di interventi aggiuntivi è considerato efficiente.
    %\item \textbf{Spiegazione}: Un processo di installazione efficiente è essenziale per garantire una rapida adozione del software da parte degli utenti. Mantenere un'installazione rapida e senza problemi aiuta a migliorare l'esperienza utente e la soddisfazione complessiva degli utenti con il software.
\end{itemize}
