\subsubsubsection{Accuratezza della risposta}
\begin{itemize}
    \item \textbf{Notazione specifica}: M.1.13;
    \item \textbf{Nome}: Accuratezza della risposta;
    \item \textbf{Descrizione}: L'accuratezza della risposta è una misura della correttezza e precisione con cui il sistema risponde a una interrogazione in linguaggio naturale. La metrica viene valutata in base a a dei test descritti nel \glossario{Dizionario dati};
    \item \textbf{Caratteristica di riferimento}: Funzionalità, Affidabilità;
    \item \textbf{Motivo}: Garantire che il sistema fornisca risposte corrette e affidabili è cruciale per la soddisfazione dell'utente e il funzionamento efficiente del software. Misurare l'accuratezza delle risposte aiuta a identificare e correggere errori nel sistema, migliorando la qualità complessiva del software;
    \item \textbf{Misurazione}: L'accuratezza della risposta è misurata in base alla distanza della risposta ottenuta dopo la generazione di un prompt su frasi preselezionate e la risposta attesa scritta dall'attore tecnico nel dizionario dati.Questa percentuale l'abbiamo calcolata come :
    \[
    \text{Accuratezza della risposta} = \frac{R_{c}}{R_{a}} \times 100
    \]
    Dove:
    \begin{itemize}
        \item $R_{c}$: Risposta corretta;
        \item $N_{a}$: Risposta attesa.
    \end{itemize}
 \end{itemize}
