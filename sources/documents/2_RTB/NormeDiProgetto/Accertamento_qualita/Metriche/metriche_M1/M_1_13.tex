\subsubsubsection{Accuratezza della risposta}
\begin{itemize}
    \item \textbf{Notazione specifica}: M.1.13
    \item \textbf{Nome}: Accuratezza della risposta
    \item \textbf{Descrizione}: L'accuratezza della risposta è una misura della correttezza e precisione con cui un sistema software risponde a una richiesta o esegue una funzione. Questa metrica valuta la percentuale di risposte corrette fornite dal sistema rispetto al numero totale di richieste.
    \item \textbf{Caratteristica di riferimento}: Funzionalità, Affidabilità (ISO/IEC 9126)
    \item \textbf{Motivo}: Garantire che il sistema fornisca risposte corrette e affidabili è cruciale per la soddisfazione dell'utente e il funzionamento efficiente del software. Misurare l'accuratezza delle risposte aiuta a identificare e correggere errori nel sistema, migliorando la qualità complessiva del software.
    \item \textbf{Misurazione}: L'accuratezza della risposta può essere misurata calcolando il rapporto tra il numero di risposte corrette e il numero totale di richieste. La formula per il calcolo è:
    \begin{equation}
    \text{Accuratezza della risposta} = \left( \frac{\text{Rc}}{\text{Rt}} \right) \times 100
    \end{equation}
    Dove:
    \begin{itemize}
        \item \textbf{Rc}: Numero di risposte corrette.
        \item \textbf{Rt}: Numero totale delle richieste.
    \end{itemize}
    %\item \textbf{Range}: 95\% - 100\%
    %\item \textbf{Spiegazione}: Un range ottimale per l'accuratezza della risposta dovrebbe essere mantenuto tra il 95\% e il 100\%. Un valore superiore al 95\% indica che il sistema fornisce risposte per la maggior parte corrette, mentre valori inferiori suggeriscono che ci sono errori significativi da correggere. Mantenere questa metrica entro il range suggerito aiuta a garantire un alto livello di qualità e affidabilità del sistema.
\end{itemize}
