\subsubsubsection{Requisiti opzionali soddisfatti}
\begin{itemize}
    \item \textbf{Notazione specifica}: M.1.22
    \item \textbf{Nome}: Requisiti opzionali soddisfatti
    \item \textbf{Descrizione}: Questa metrica indica la percentuale di requisiti opzionali soddisfatti dal software rispetto al totale dei requisiti opzionali definiti nel documento di specifica dei requisiti.
    \item \textbf{Caratteristica di riferimento}: Funzionalità, Qualità;
    \item \textbf{Motivo}: Misura il grado di adempimento dei requisiti opzionali, fornendo una valutazione della completezza del software rispetto alle richieste supplementari specificate.
    \item \textbf{Misurazione}: La percentuale di requisiti opzionali soddisfatti può essere calcolata utilizzando la seguente formula:
    \[ ROS = \frac{Ro}{To} \times 100\% \]
    Dove:
    \begin{itemize}
        \item \textbf{Ro}: Numero di requisiti opzionali soddisfatti;
        \item \textbf{To}: Numero totale dei requisiti opzionali.
    \end{itemize}
    %\item \textbf{Range}: Il range ideale per questa metrica è dal 100% (tutti i requisiti opzionali soddisfatti) al 0% (nessun requisito opzionale soddisfatto).
    %\item \textbf{Spiegazione}: Una percentuale più alta di requisiti opzionali soddisfatti indica una maggiore capacità del software di soddisfare richieste aggiuntive che non sono considerate essenziali ma che possono migliorare l'esperienza dell'utente o fornire funzionalità extra.
\end{itemize}
