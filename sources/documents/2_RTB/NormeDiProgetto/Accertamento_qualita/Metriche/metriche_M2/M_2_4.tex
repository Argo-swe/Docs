\subsubsubsection{Variazione temporale}
\begin{itemize}
    \item \textbf{Notazione specifica}: M.2.4;
    \item \textbf{Nome}: Variazione temporale;
    \item \textbf{Descrizione}: Questa metrica misura la variazione tra tempo pianificato e tempo effettivo per completare un progetto o una fase di un progetto;
    \item \textbf{Caratteristica di riferimento}: Gestione del progetto, Adempimento delle scadenze;
    \item \textbf{Motivo}: Valutare la capacità del progetto di rispettare le tempistiche programmate, identificando eventuali ritardi e migliorando la gestione del tempo;
    \item \textbf{Misurazione}: La variazione temporale è calcolata utilizzando la seguente formula:
    \[
        \text{Variazione temporale (\%)} =\frac{T_p - T_e}{T_p} \times 100
    \]
    dove:
    \begin{itemize}
        \item $T_{p}$: Tempo pianificato;
        \item $T_{e}$: Tempo effettivo.
    \end{itemize}
    %\item \textbf{Range}: Il range ideale per questa metrica è dal 100\% (tempo effettivo uguale al tempo pianificato) a valori inferiori (tempo effettivo superiore al tempo pianificato).
    %\item \textbf{Spiegazione}: Una percentuale più alta indica una migliore efficienza nell'uso del tempo, suggerendo che il progetto è stato completato entro il tempo previsto. Valori inferiori a 100\% possono indicare ritardi o inefficienze nella gestione del tempo.
\end{itemize}
