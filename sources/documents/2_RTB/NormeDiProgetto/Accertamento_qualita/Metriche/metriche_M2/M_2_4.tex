\subsubsubsection{Efficienza temporale}
\begin{itemize}
    \item \textbf{Notazione specifica}: M.2.4
    \item \textbf{Nome}: Efficienza temporale
    \item \textbf{Descrizione}: Questa metrica misura l'efficienza con cui il tempo viene utilizzato per completare un progetto o una fase di un progetto rispetto al tempo pianificato. È un indicatore della capacità del team di rispettare le scadenze e di ottimizzare l'uso del tempo disponibile.
    \item \textbf{Caratteristica di riferimento}: Gestione del progetto, Adempimento delle scadenze
    \item \textbf{Motivo}: Valutare la capacità del progetto di rispettare le tempistiche programmate, identificando eventuali ritardi e migliorando la gestione del tempo.
    \item \textbf{Misurazione}: L'efficienza temporale (\textbf{ET}) può essere calcolata utilizzando la seguente formula:
    \[ ET = \frac{T_p}{T_e} \times 100\% \]
    dove:
    \begin{itemize}
        \item \textbf{T_p}: Tempo pianificato
        \item \textbf{T_e}: Tempo effettivo
    \end{itemize}
    \item \textbf{Range}: Il range ideale per questa metrica è dal 100\% (tempo effettivo uguale al tempo pianificato) a valori inferiori (tempo effettivo superiore al tempo pianificato).
    \item \textbf{Spiegazione}: Una percentuale più alta indica una migliore efficienza nell'uso del tempo, suggerendo che il progetto è stato completato entro il tempo previsto. Valori inferiori a 100\% possono indicare ritardi o inefficienze nella gestione del tempo.
\end{itemize}
