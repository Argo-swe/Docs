\subsubsubsection{Frequenza di merge}
\begin{itemize}
    \item \textbf{Notazione specifica}: M.2.6;
    \item \textbf{Nome}: Frequenza di merge;
    \item \textbf{Descrizione}: Questa metrica misura la frequenza con cui vengono effettuati i merge di branch nel repository del progetto. Indica la frequenza con cui il codice viene integrato nel branch principale dopo aver completato una determinata funzionalità o correzione;
    \item \textbf{Caratteristica di riferimento}: Integrazione continua, Efficienza dello sviluppo;
    \item \textbf{Motivo}: Valutare la frequenza di merge aiuta a comprendere la rapidità con cui le nuove funzionalità o correzioni vengono integrate nel codice principale del progetto, favorendo un flusso di lavoro continuo e una collaborazione efficace tra i membri del team;
    \item \textbf{Misurazione}: La frequenza di \glossario{merge} è calcolata come il numero totale di merge effettuati.
   
   
   
    %\item \textbf{Range}: Il range ideale per questa metrica può variare a seconda delle pratiche di sviluppo del progetto, ma generalmente una frequenza di merge regolare indica una buona pratica di integrazione continua.
    %\item \textbf{Spiegazione}: Una frequenza di merge regolare suggerisce che il team sta seguendo un flusso di lavoro collaborativo e integrando il codice in modo tempestivo, migliorando l'efficienza dello sviluppo e riducendo i rischi di conflitti di merge e divergenze nel codice.
\end{itemize}
