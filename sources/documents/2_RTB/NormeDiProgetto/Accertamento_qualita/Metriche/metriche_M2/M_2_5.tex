\subsubsubsection{Velocità di verifica dopo una pull request}
\begin{itemize}
    \item \textbf{Notazione specifica}: M.2.5;
    \item \textbf{Nome}: Velocità di verifica dopo una \glossario{pull request};
    \item \textbf{Descrizione}: Questa metrica misura il tempo medio impiegato per verificare e approvare una \glossario{pull request} dopo che è stata sottomessa. È un indicatore della rapidità e dell'efficienza del processo di revisione del codice;
    \item \textbf{Caratteristica di riferimento}: Efficienza, Manutenibilità;
    \item \textbf{Motivo}: Valutare la rapidità con cui il team di sviluppo riesce a verificare e integrare le modifiche proposte, migliorando il flusso di lavoro e riducendo i tempi di attesa;
    \item \textbf{Misurazione}: La velocità di verifica dopo una pull request è calcolata utilizzando la seguente formula:
    \[
        \text{Velocità di verifica dopo una pull request} =\frac{1}{N} \sum_{i=1}^{N} T_i 
    \]
    dove:
    \begin{itemize}
        \item $N$: Numero di PR verificate;
        \item $T_{i}$: Tempo impiegato per verificare la \(i\)-esima PR.
    \end{itemize}
    %\item \textbf{Range}: Il range ideale per questa metrica può variare a seconda del progetto, ma generalmente un tempo medio di verifica inferiore a 24 ore è considerato efficiente.
    %\item \textbf{Spiegazione}: Un tempo medio di verifica più basso indica una maggiore efficienza nel processo di revisione del codice, suggerendo un flusso di lavoro più snello e una più rapida integrazione delle modifiche.
\end{itemize}
