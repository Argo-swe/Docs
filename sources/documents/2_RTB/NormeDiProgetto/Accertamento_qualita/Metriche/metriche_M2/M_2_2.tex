\subsubsection{M.2.2 Variazione pianificazione task completati}
\begin{itemize}
    \item \textbf{Notazione specifica}: M.2.2
    \item \textbf{Nome}: Variazione pianificazione task completati
    \item \textbf{Descrizione}: Questa metrica misura la variazione tra il numero di task pianificati e quelli effettivamente completati entro un periodo di tempo specifico. È un indicatore della capacità del team di rispettare le scadenze e le pianificazioni stabilite.
    \item \textbf{Caratteristica di riferimento}: Gestione del progetto, Adempimento delle scadenze
    \item \textbf{Motivo}: Valutare l'efficacia della pianificazione del progetto e l'aderenza del team alle tempistiche programmate, identificando eventuali deviazioni che potrebbero richiedere interventi correttivi.
    \item \textbf{Misurazione}: La variazione pianificazione task completati (\textbf{VPTC}) può essere calcolata utilizzando la seguente formula:
    \[ VPTC = \frac{T_p - T_c}{T_p} \times 100\% \]
    dove:
    \begin{itemize}
        \item \textbf{T_p}: Numero di task pianificati
        \item \textbf{T_c}: Numero di task completati
    \end{itemize}
    \item \textbf{Range}: Il range ideale per questa metrica è dal 0\% (nessuna variazione, tutti i task completati come pianificato) al 100\% (tutti i task pianificati non completati).
    \item \textbf{Spiegazione}: Una percentuale più bassa indica una maggiore aderenza del team alla pianificazione originale, suggerendo un'efficace gestione del progetto e capacità di rispettare le scadenze.
\end{itemize}
