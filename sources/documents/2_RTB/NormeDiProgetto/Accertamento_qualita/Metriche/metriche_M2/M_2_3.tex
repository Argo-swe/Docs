\subsubsubsection{Variazione di costo}
\begin{itemize}
    \item \textbf{Notazione specifica}: M.2.3;
    \item \textbf{Nome}: Variazione di costo;
    \item \textbf{Descrizione}: Questa metrica misura la variazione tra il costo pianificato e il costo effettivo di un progetto o di una fase di un progetto. È un indicatore dell'efficacia della gestione dei costi e della capacità del team di mantenere il progetto entro il budget stabilito;
    \item \textbf{Caratteristica di riferimento}: Gestione del progetto, Controllo dei costi;
    \item \textbf{Motivo}: Valutare la precisione delle stime dei costi iniziali e la capacità del progetto di rimanere entro il budget, identificando eventuali sforamenti che potrebbero richiedere interventi correttivi;
    \item \textbf{Misurazione}: La variazione di costo è calcolata utilizzando la seguente formula:
    \[
        \text{Variazione di costo (\%)} =\frac{C_p - C_a}{C_p} \times 100
    \]
    dove:
    \begin{itemize}
        \item $C_{p}$: Costo pianificato;
        \item $C_{a}$: Costo attuale (effettivo).
    \end{itemize}
    %\item \textbf{Range}: Il range ideale per questa metrica è dal 0\% (nessuna variazione, costo effettivo uguale al costo pianificato) al 100\% (costo effettivo doppio rispetto al costo pianificato).
    %\item \textbf{Spiegazione}: Una percentuale più bassa indica una migliore gestione dei costi del progetto, suggerendo che il progetto è stato completato entro il budget previsto. Deviazioni significative possono indicare problemi di stima dei costi o di controllo del budget.
\end{itemize}
