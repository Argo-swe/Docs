\subsubsubsection{Percentuale di metriche soddisfatte}
\begin{itemize}
    \item \textbf{Notazione specifica}: M.2.1;
    \item \textbf{Nome}: Percentuale di metriche soddisfatte;
    \item \textbf{Descrizione}: Questa metrica misura la percentuale di metriche che soddisfano i criteri di accettazione specificati rispetto al totale delle metriche definite;
    \item \textbf{Caratteristica di Riferimento}: Qualità del prodotto;
    \item \textbf{Motivo}: Valutare il grado di conformità del prodotto agli standard e requisiti di qualità;
    \item \textbf{Misurazione}: La percentuale di requisiti opzionali soddisfatti è calcolata utilizzando la seguente formula:
    \[
        \text{Percentuale di metriche soddisfatte (\%)} =\frac{M_{s}}{M_{t}} \times 100 
    \]
    dove:
    \begin{itemize}
        \item $M_{s}$: Numero di metriche soddisfatte;
        \item $M_{t}$: Numero totale di metriche.
    \end{itemize}
    
    %\item \textbf{Range}: Il range ideale per questa metrica è dal 100\% (tutti i requisiti opzionali soddisfatti) al 0\% (nessun requisito opzionale soddisfatto).
    %\item \textbf{Spiegazione}: Una percentuale più alta di requisiti opzionali soddisfatti indica una maggiore capacità del software di soddisfare richieste aggiuntive che non sono considerate essenziali ma che possono migliorare l'esperienza dell'utente o fornire funzionalità extra.
\end{itemize}
