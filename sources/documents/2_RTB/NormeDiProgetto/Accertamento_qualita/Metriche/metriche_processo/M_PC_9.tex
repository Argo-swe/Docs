\subsubsubsection{Frequenza di merge delle pull request}
\begin{itemize}
    \item \textbf{Notazione specifica}: M.PC.9;
    \item \textbf{Nome}: Frequenza di merge delle pull request;
    \item \textbf{Descrizione}: Questa metrica misura la frequenza con cui le pull request vengono approvate e unite al ramo base;
    \item \textbf{Motivo}: Valutare la capacità del team di integrare in modo efficiente le nuove funzionalità, modifiche e correzioni nell’ambiente condiviso, favorendo un flusso di lavoro continuo e una collaborazione  stretta tra i membri. La chiusura delle pull request include il processo di revisione del codice, l’innesto delle modifiche richieste dal verificatore e l’approvazione finale;
    \item \textbf{Misurazione}:
    \[
        \text{Frequenza di merge delle pull request} = \frac{N_{pr}}{T} 
    \]
    dove:
    \begin{itemize}
        \item $N_{pr}$: Numero totale di pull request approvate e unite al ramo base;
        \item $T$: Periodo di tempo considerato (in giorni).
    \end{itemize}
\end{itemize}
