\subsubsection{M.3.4 Rischi non previsti divisi per errori successi}
\begin{itemize}
    \item \textbf{Notazione specifica}: M.3.4
    \item \textbf{Nome}: Rischi non previsti divisi per errori successi
    \item \textbf{Descrizione}: Questa metrica rappresenta il rapporto tra il numero di rischi non previsti che emergono durante lo sviluppo del software e il numero di errori di sistema successi. Gli errori di sistema successi sono quelli che sono stati risolti e corretti con successo durante il processo di sviluppo.
    \item \textbf{Caratteristica di riferimento}: Gestione dei rischi, Affidabilità
    \item \textbf{Motivo}: Questa metrica fornisce una misura della proporzione di rischi non previsti rispetto agli errori di sistema che sono stati risolti con successo. Aiuta a valutare l'efficacia della gestione dei rischi nel ridurre l'incidenza degli errori nel sistema.
    \item \textbf{Misurazione}: Il rapporto tra il numero di rischi non previsti (\( RN_{sprint} \)) e il numero di errori di sistema successi (\( ES_{sprint} \)) durante lo stesso periodo.
    \item \textbf{Formula di Misurazione}:
    \[
    RNR_{sprint} = \frac{RN_{sprint}}{ES_{sprint}}
    \]
    \item \textbf{Range}: Il range ideale per questa metrica è tendente a zero, il che indica una bassa incidenza di rischi non previsti rispetto agli errori di sistema successi.
    \item \textbf{Spiegazione}: Un valore prossimo a zero per questa metrica indica che il numero di rischi non previsti è relativamente basso rispetto al numero di errori di sistema che sono stati correttamente identificati e risolti, suggerendo una gestione efficace dei rischi nel processo di sviluppo del software.
\end{itemize}
