\subsubsubsection{M.3.3 Rischi non previsti in assoluto per sprint}
\begin{itemize}
    \item \textbf{Notazione specifica}: M.3.3
    \item \textbf{Nome}: Rischi non previsti in assoluto per sprint
    \item \textbf{Descrizione}: Questa metrica misura il rapporto tra il numero effettivo di rischi non previsti che emergono durante lo sviluppo del software in ciascuno sprint e il numero massimo previsto di rischi per lo stesso periodo.
    \item \textbf{Caratteristica di riferimento}: Gestione dei rischi
    \item \textbf{Motivo}: Monitorare i rischi non previsti consente di valutare l'efficacia del processo di identificazione e gestione dei rischi, nonché la capacità del team di rispondere in modo tempestivo e efficace a situazioni impreviste.
    \item \textbf{Misurazione}: Il rapporto tra il numero effettivo di rischi non previsti (\(NR_{sprint}\)) e il numero massimo previsto di rischi (\(NR_{max}\)) per lo sprint.
    \item \textbf{Formula di Misurazione}:
    \[
    RNP_{sprint} = \frac{NR_{sprint}}{NR_{max}}
    \]
    Dove:
    \begin{itemize}
        \item \(RNP_{sprint}\) rappresenta il rapporto tra il numero effettivo di rischi non previsti durante lo sprint e il numero massimo previsto di rischi per lo stesso periodo.
        \item \(NR_{sprint}\) è il numero effettivo di rischi non previsti durante lo sprint.
        \item \(NR_{max}\) è il numero massimo previsto di rischi per lo sprint.
    \end{itemize}
    \item \textbf{Range}: Il range ideale per questa metrica è < 1, il che significa che il numero effettivo di rischi non previsti è inferiore al numero massimo previsto di rischi per lo sprint.
    \item \textbf{Spiegazione}: Un valore inferiore a 1 indica che il numero effettivo di rischi non previsti è inferiore al numero massimo previsto, evidenziando una pianificazione efficace e una gestione proattiva dei rischi durante lo sprint.
\end{itemize}
