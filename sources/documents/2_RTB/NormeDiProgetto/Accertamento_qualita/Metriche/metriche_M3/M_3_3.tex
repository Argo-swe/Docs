\subsubsubsection{Rischi non previsti in assoluto per sprint}
\begin{itemize}
    \item \textbf{Notazione specifica}: M.3.3;
    \item \textbf{Nome}: Rischi non previsti in assoluto per sprint;
    \item \textbf{Descrizione}: Questa metrica misura il rapporto tra il numero effettivo di rischi non previsti che emergono durante lo sviluppo del software in ciascuno sprint e il numero massimo previsto di rischi per lo stesso periodo;
    \item \textbf{Caratteristica di riferimento}: Gestione dei rischi;
    \item \textbf{Motivo}: Monitorare i rischi non previsti consente di valutare l'efficacia del processo di identificazione e gestione dei rischi, nonché la capacità del team di rispondere in modo tempestivo e efficace a situazioni impreviste;
    \item \textbf{Misurazione}: Il rapporto tra il numero effettivo di rischi non previsti e il numero massimo previsto di rischi per lo sprint secondo le formula:
    \[
        \text{Rischi non previsti in assoluto per sprint} =\frac{NR_{sprint}}{NR_{max}} 
    \]
    dove:
    \begin{itemize}
        \item $NR_{sprint}$: Numero effettivo di rischi non previsti durante lo sprint;
        \item $NR_{max}$: Numero massimo previsto di rischi per lo sprint.
    \end{itemize}

    TODO -> in assoluto o in %
    %\item \textbf{Range}: Il range ideale per questa metrica è < 1, il che significa che il numero effettivo di rischi non previsti è inferiore al numero massimo previsto di rischi per lo sprint.
    %\item \textbf{Spiegazione}: Un valore inferiore a 1 indica che il numero effettivo di rischi non previsti è inferiore al numero massimo previsto, evidenziando una pianificazione efficace e una gestione proattiva dei rischi durante lo sprint.
\end{itemize}
