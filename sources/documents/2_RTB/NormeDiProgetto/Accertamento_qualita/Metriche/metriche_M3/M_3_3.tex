\subsubsubsection{Efficienza delle contromisure}
\begin{itemize}
    \item \textbf{Notazione specifica}: M.3.3;
    \item \textbf{Nome}: Efficienza delle contromisure;
    \item \textbf{Descrizione}: Idicatore utile per valutare l'efficacia delle azioni correttive intraprese per mitigare i rischi;
    \item \textbf{Caratteristica di riferimento}: Gestione dei rischi;
    \item \textbf{Motivo}: Questa metrica permette di misurare quanto le contromisure adottate siano state in grado di ridurre o eliminare i rischi identificati e quante si sono rivelate incomplete per valutare se un rischio ha bisogno di modifiche correttive;
    \item \textbf{Misurazione}: Il rapporto dei rischi è calcolato con questa formula:
    \[
        \text{Efficienza delle contromisure} =\frac{1}{N} \sum_{i=1}^{N} Xi
    \]
    dove:
    \begin{itemize}
        \item $X$: Valore compreso tra [0,1] corrispondente alla valutazione a posteriori dell’efficienza delle mitigazioni nei confronti del rischio;
        \item $N$: Numero di rischi mitigati.
    \end{itemize}
\end{itemize}