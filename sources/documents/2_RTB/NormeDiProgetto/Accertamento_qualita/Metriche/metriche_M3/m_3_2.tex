\subsubsubsection{M.3.2 Metriche di gestione degli errori}
\begin{itemize}
    \item \textbf{Notazione specifica}: M.3.2
    \item \textbf{Nome}: Metriche di gestione degli errori
    \item \textbf{Descrizione}: Questa metrica valuta l'efficacia del sistema di gestione degli errori all'interno del software. Include l'identificazione, la registrazione, la correzione e il monitoraggio degli errori nel sistema.
    \item \textbf{Caratteristica di riferimento}: Affidabilità, Manutenibilità
    \item \textbf{Motivo}: Valutare le metriche di gestione degli errori aiuta a identificare la capacità del sistema di gestire e correggere efficacemente gli errori che possono verificarsi durante l'esecuzione del software, garantendo un'esperienza utente affidabile e riducendo i tempi di inattività.
    \item \textbf{Misurazione}: Le metriche specifiche per valutare la gestione degli errori possono includere il numero di errori identificati, il tempo medio per risolvere un errore, la frequenza di aggiornamento della documentazione sugli errori, tra gli altri.
    \item \textbf{Range}: Il range ideale per questa metrica può variare in base al tipo e alla complessità del software, ma generalmente indica una gestione proattiva ed efficiente degli errori.
    \item \textbf{Formula di Misurazione}: La valutazione della metrica di gestione degli errori (\( ME \)) può essere rappresentata come segue:
    \[
    ME = \begin{cases} 
          3 & \text{se la criticità dell'errore è bassa} \\
          2 & \text{se la criticità dell'errore è media} \\
          1 & \text{se la criticità dell'errore è alta} \\
       \end{cases}
    \]
    Dove:
    \begin{itemize}
        \item \( ME \) rappresenta il punteggio assegnato alla gestione dell'errore.
        \item La criticità dell'errore viene valutata con un punteggio da 1 a 3, dove 1 rappresenta un errore ad alta criticità che compromette l'avanzamento del progetto, 2 rappresenta un errore di criticità media e 3 rappresenta un errore di bassa criticità che non compromette significativamente l'avanzamento del progetto.
    \end{itemize}
    \item \textbf{Spiegazione}: Le metriche di gestione degli errori forniscono al team di sviluppo una panoramica delle prestazioni nel rilevare, registrare, risolvere e monitorare gli errori nel software, permettendo al team di prendere decisioni informate e di migliorare continuamente il processo di gestione degli errori.
\end{itemize}

