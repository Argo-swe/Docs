\subsubsubsection{M.3.1 Livello di controllo dei guasti}
\begin{itemize}
    \item \textbf{Notazione specifica}: M.3.1
    \item \textbf{Nome}: Livello di controllo dei guasti
    \item \textbf{Descrizione}: Questa metrica valuta l'efficacia del sistema di controllo e gestione dei guasti all'interno del software. Include l'identificazione, la registrazione, la risoluzione e il monitoraggio dei guasti nel sistema.
    \item \textbf{Caratteristica di riferimento}: Affidabilità, Manutenibilità
    \item \textbf{Motivo}: Valutare il livello di controllo dei guasti aiuta a identificare la capacità del sistema di gestire e risolvere efficacemente i guasti che possono verificarsi durante l'uso del software, garantendo un'esperienza utente affidabile e riducendo i tempi di inattività.
    \item \textbf{Misurazione}: Il livello di controllo dei guasti (\textbf{LCG}) può essere misurato come il rapporto tra il numero di guasti identificati e risolti correttamente rispetto al totale dei guasti riscontrati durante un periodo di tempo specifico.
    \item \textbf{Formula}:
    \[ LCG = \frac{NG}{NT} \times 100\% \]
    dove:
    \begin{itemize}
        \item \( LCG \) rappresenta il Livello di controllo dei guasti, espresso come percentuale.
        \item \( NG \) è il numero di guasti identificati e risolti correttamente durante un periodo di tempo specifico.
        \item \( NT \) è il numero totale di guasti riscontrati durante lo stesso periodo di tempo.
    \end{itemize}
    \item \textbf{Range}: Il range ideale per questa metrica è dal 0\% (nessun guasto identificato e risolto) al 100\% (tutti i guasti identificati e risolti correttamente).
    \item \textbf{Spiegazione}: Un livello più alto di controllo dei guasti indica una maggiore capacità del sistema di identificare, registrare e risolvere i guasti in modo tempestivo ed efficiente, migliorando l'affidabilità e la manutenibilità complessiva del software.
\end{itemize}
