\subsubsubsection{Livello di controllo dei guasti}
\begin{itemize}
    \item \textbf{Notazione specifica}: M.3.1;
    \item \textbf{Nome}: Livello di controllo dei guasti;
    \item \textbf{Descrizione}: Questa metrica valuta l'efficacia del sistema di controllo e gestione dei guasti all'interno del software;
    \item \textbf{Caratteristica di riferimento}: Affidabilità, Manutenibilità;
    \item \textbf{Motivo}: Valutare il livello di controllo dei guasti aiuta a identificare la capacità del sistema di gestire e risolvere efficacemente i guasti che possono verificarsi durante l'uso del software, garantendo un'esperienza utente affidabile e riducendo i tempi di inattività;
    \item \textbf{Misurazione}: Il livello di controllo dei guasti è misurato con la seguente formula: 
    \[
        \text{Livello di controllo dei guasti (\%)} =\frac{L_{g}}{N_{t}} \times 100 
    \]
    dove:
    \begin{itemize}
        \item $L_{g}$: Numero di guasti identificati e risolti correttamente;
        \item $N_{t}$: Numero totale di guasti riscontrati.
    \end{itemize}
    
    %\item \textbf{Range}: Il range ideale per questa metrica è dal 0\% (nessun guasto identificato e risolto) al 100\% (tutti i guasti identificati e risolti correttamente).
    %\item \textbf{Spiegazione}: Un livello più alto di controllo dei guasti indica una maggiore capacità del sistema di identificare, registrare e risolvere i guasti in modo tempestivo ed efficiente, migliorando l'affidabilità e la manutenibilità complessiva del software.
\end{itemize}
