\subsubsubsection{Vocaboli inseriti nel glossario ad ogni sprint}
\begin{itemize}
    \item \textbf{Notazione specifica}: M.4.3;
    \item \textbf{Nome}: Vocaboli inseriti nel glossario ad ogni sprint;
    \item \textbf{Descrizione}: Questa metrica misura il numero di nuove parole che vengono introdotte nel glossario del progetto ad ogni sprint. Il glossario del progetto include tutti i termini tecnici, acronimi e altri concetti specifici utilizzati nella documentazione e nelle comunicazioni del team;
    \item \textbf{Caratteristica di riferimento}: Comunicazione, comprensione del linguaggio tecnico e accessibilità;
    \item \textbf{Motivo}: Monitorare il tasso di crescita del glossario del progetto aiuta a valutare la complessità e l'evoluzione del linguaggio tecnico utilizzato all'interno del team e può evidenziare la necessità di formazione aggiuntiva o di chiarimenti sui concetti;
    \item \textbf{Misurazione}: La metrica misura il numero di nuove parole introdotte nel glossario secondo la formula:
    \[
        \text{Parole inserite nel glossario ad ogni sprint (\%)} =\frac{P_{a}}{P_{t}} \times 100 
    \]
    dove:
    \begin{itemize}
        \item $P_{a}$: Numero di vocaboli aggiunti nello sprint;
        \item $P_{t}$: Numero di vocaboli totali.
    \end{itemize}
    
    
    %\item \textbf{Range}: N/A (il valore ottimale dipende dalle esigenze specifiche del progetto e dalla sua complessità)
    %\item \textbf{Spiegazione}: Un aumento significativo nel numero di parole introdotte potrebbe indicare una crescente complessità del progetto o una comunicazione inefficiente, mentre una diminuzione potrebbe indicare una maggiore stabilità o un'efficace condivisione della conoscenza.
\end{itemize}
