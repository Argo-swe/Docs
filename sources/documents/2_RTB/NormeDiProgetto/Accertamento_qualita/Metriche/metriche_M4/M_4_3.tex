\subsubsubsection{M.4.4 Completezza della documentazione}
\begin{itemize}
    \item \textbf{Notazione specifica}: M.4.4
    \item \textbf{Nome}: Completezza della documentazione
    \item \textbf{Descrizione}: Questa metrica valuta quanto la documentazione copra tutti gli aspetti del software, inclusi requisiti, progettazione, implementazione, test e manutenzione. Può essere misurata in base alla presenza di documenti richiesti e alla copertura dei contenuti.
    \item \textbf{Caratteristica di riferimento}: Qualità della documentazione
    \item \textbf{Motivo}: Assicurarsi che la documentazione fornisca una visione completa e accurata del software, essenziale per il suo utilizzo e la sua manutenzione.
    \item \textbf{Misurazione}: La completezza della documentazione può essere misurata utilizzando la seguente formula:
    \[ \text{Completezza della documentazione (\%)} = \frac{\text{Numero di documenti presenti}}{\text{Numero totale di documenti attesi}} \times 100\% \]
    \item \textbf{Range}: Ideale: >95\%
    \item \textbf{Spiegazione}: Una documentazione completa è essenziale per garantire la comprensione e la gestione efficace del software. Un valore del 95\% o superiore indica una documentazione completa che copre tutti gli aspetti significativi del software.
\end{itemize}
