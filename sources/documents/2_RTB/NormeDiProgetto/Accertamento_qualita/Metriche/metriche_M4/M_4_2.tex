\subsubsubsection{M.4.2 Indice Gunning Fog}
\begin{itemize}
    \item \textbf{Notazione specifica}: M.4.2
    \item \textbf{Nome}: Indice Gunning Fog
    \item \textbf{Descrizione}: Per valutare la complessità e la leggibilità di un testo, si utilizza l'Indice Gunning Fog. Questa metrica tiene conto della lunghezza delle parole e delle frasi all'interno del testo.
    \item \textbf{Formula}:
    \[ 0.4 \left( \frac{{\text{{numero totale di parole}}}}{{\text{{numero totale di frasi}}}} + 100 \left( \frac{{\text{{numero di parole con più di tre sillabe}}}}{{\text{{numero totale di parole}}}} \right) \right) \]
    \item \textbf{Caratteristica di riferimento}: Complessità e leggibilità del testo
    \item \textbf{Motivo}: L'Indice Gunning Fog restituisce un valore che indica il grado di difficoltà di lettura del testo, fondamentale per valutare la comprensibilità del contenuto da parte del pubblico.
    \item \textbf{Misurazione}: Utilizza la formula sopra indicata, dove il numero totale di parole, frasi e parole con più di tre sillabe è calcolato all'interno del testo.
    \item \textbf{Range di valori e interpretazione}: 
    \begin{itemize}
        \item Valori intorno a 6: testi con una leggibilità molto semplice, adatti a un vasto pubblico, incluso chi ha completato solo la scuola elementare.
        \item Valori compresi tra 7 e 9: testi con una leggibilità media, adatti a persone con un livello di istruzione medio, come chi ha completato la scuola media.
        \item Valori superiori a 10: testi con una leggibilità elevata, adatti a lettori con un alto livello di istruzione, come chi ha completato la scuola superiore o universitaria.
    \end{itemize}
\end{itemize}
