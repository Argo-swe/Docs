\subsubsubsection{Completezza della documentazione}
\begin{itemize}
    \item \textbf{Notazione specifica}: M.4.2;
    \item \textbf{Nome}: Completezza della documentazione;
    \item \textbf{Descrizione}: Questa metrica valuta quanto la documentazione copra tutti gli aspetti del software, inclusi requisiti, progettazione, implementazione, test e manutenzione;
    \item \textbf{Caratteristica di riferimento}: Qualità della documentazione;
    \item \textbf{Motivo}: Assicurarsi che la documentazione fornisca una visione completa e accurata del software, essenziale per il suo utilizzo e la sua manutenzione;
    \item \textbf{Misurazione}: La completezza della documentazione è misurata utilizzando la seguente formula:
    \[
        \text{Completezza della documentazione (\%)} =\frac{{N_p}}{{N_a}} \times 100 
    \]
    dove:
    \begin{itemize}
        \item $N_{p}$: Numero di documenti totali presenti;
        \item $N_{a}$: Numero di documenti attesi.
    \end{itemize}
    %\item \textbf{Range}: Ideale: >95\%
    %\item \textbf{Spiegazione}: Una documentazione completa è essenziale per garantire la comprensione e la gestione efficace del software. Un valore del 95\% o superiore indica una documentazione completa che copre tutti gli aspetti significativi del software.
\end{itemize}
