\subsubsection{M.4.1 Indice Gulpease}
\begin{itemize}
    \item \textbf{Notazione specifica}: M.4.1
    \item \textbf{Nome}: Indice Gulpease
    \item \textbf{Descrizione}: L'Indice Gulpease è una metrica di leggibilità del testo di un documento. È basato sulla lunghezza delle parole e sulla lunghezza delle frasi. La formula è la seguente: \[ 89+\frac{300 \cdot \text{{numero delle frasi}} - 10 \cdot \text{{numero delle lettere}}}{\text{{numero delle parole}}} \]
    \item \textbf{Caratteristica di riferimento}: Qualità del testo
    \item \textbf{Motivo}: L'Indice Gulpease fornisce una valutazione della leggibilità di un testo, importante per garantire la comprensibilità e l'accessibilità delle informazioni contenute nel documento.
    \item \textbf{Misurazione}: Per garantire la leggibilità della documentazione in lingua italiana si applica il calcolo dell’indice di leggibilità Gulpease sui testi, considerando la lunghezza in lettere di parole e frasi.
    \[ 89 + \left( \frac{{300 \cdot \text{{numero delle frasi}} - 10 \cdot \text{{numero delle lettere}}}}{{\text{{numero delle parole}}}} \right) \]
    \item \textbf{Range}: Il range ideale per questa metrica è tra 0 e 100, dove valori più alti indicano una maggiore leggibilità del testo.
    \item \textbf{Spiegazione}: La formula per il calcolo dell’indice di Gulpease restituisce un punteggio in centesimi che determina il grado di comprensibilità rapportato al livello di istruzione del lettore:
    \begin{itemize}
        \item punteggi intorno a 0: testi a leggibilità più bassa;
        \item punteggi minori di 40: testi di difficile comprensione per lettori in possesso di un diploma di scuola superiore;
        \item punteggi minori di 60: testi di difficile comprensione per lettori in possesso di licenza di scuola media;
        \item punteggi minori di 80: testi di difficile comprensione per lettori in possesso di licenza elementare;
        \item punteggi intorno a 100: i testi con maggiore livello di comprensione.
    \end{itemize}
\end{itemize}
