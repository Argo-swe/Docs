\subsubsubsection{Indice Gulpease}
\begin{itemize}
    \item \textbf{Notazione specifica}: M.4.1;
    \item \textbf{Nome}: Indice Gulpease;
    \item \textbf{Descrizione}: L'Indice Gulpease è una metrica di leggibilità di un testo scritto nella lingua italiana. È basato sulla lunghezza delle parole e sulla lunghezza delle frasi;
    \item \textbf{Caratteristica di riferimento}: Qualità del testo;
    \item \textbf{Motivo}: L'Indice Gulpease fornisce una valutazione della leggibilità di un testo, importante per garantire la comprensibilità e l'accessibilità delle informazioni contenute nel documento;
    \item \textbf{Misurazione}: Per garantire la leggibilità della documentazione si applica questa formula:
    \[
        \text{Indice Gulpease} =89 + \frac{{300 \cdot {{N_f}} - 10 \cdot {{N_l}}}}{{{{N_p}}}}
    \]
    dove:
    \begin{itemize}
        \item $N_{f}$: Numero totale di frasi;
        \item $N_{l}$: Numero totale di lettere;
        \item $N_{p}$: Numero totale di parole.
        
    \end{itemize}
    %\item \textbf{Range}: Il range ideale per questa metrica è tra 0 e 100, dove valori più alti indicano una maggiore leggibilità del testo.
    \item \textbf{Interpretazione risultato}: La formula per il calcolo dell’indice di Gulpease restituisce un punteggio in centesimi che determina il grado di comprensibilità rapportato al livello di istruzione del lettore:
    \begin{itemize}
        \item punteggi intorno a 0: testi a leggibilità più bassa;
        \item punteggi minori di 40: testi di difficile comprensione per lettori in possesso di un diploma di scuola superiore;
        \item punteggi minori di 60: testi di difficile comprensione per lettori in possesso di licenza di scuola media;
        \item punteggi minori di 80: testi di difficile comprensione per lettori in possesso di licenza elementare;
        \item punteggi intorno a 100: i testi con maggiore livello di comprensione.
    \end{itemize}
\end{itemize}
