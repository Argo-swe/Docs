\subsection{Formazione}
\subsubsection{Scopo}
\par Il processo di formazione ha lo scopo di fornire al team le competenze necessarie per svolgere le attività del progetto e, soprattutto, di mantenere aggiornate le conoscenze all'interno del gruppo. La fornitura, lo sviluppo e la manutenzione dei prodotti software dipendono in gran parte da personale esperto e qualificato.

\subsubsection{Implementazione del processo}
\par Il team deve condurre una revisione dei requisiti del progetto per identificare le risorse e le competenze richieste. Inoltre, è necessario sviluppare un piano di formazione che includa le seguenti attività:
\begin{itemize}
  \item \textbf{Formazione individuale}: prevede lo studio della documentazione relativa alle tecnologie e agli strumenti utilizzati. Alla formazione teorica segue un approccio pratico di tipo "learning by doing";
  \item \textbf{Workshop} (formazione collaborativa): sessioni in cui i membri più esperti condividono le loro conoscenze per uniformare le competenze all'interno del gruppo.
\end{itemize}

\subsubsection{Materiale di formazione}
\par Il materiale di formazione utilizzato dal gruppo comprende:
\begin{itemize}
  \item Documentazione ufficiale delle tecnologie o degli strumenti utilizzati;
  \item Forum e community online;
  \item Manuali di formazione redatti dal team;
  \item Presentazioni multimediali;
  \item Esempi pratici (es.: programmi già sviluppati).
\end{itemize}

\vspace{0.5\baselineskip}
\par Il materiale di formazione è organizzato in una sezione dedicata su Google Drive. Per le risorse web, viene fornito l'indirizzo di destinazione delle pagine.

\subsubsection{Strumenti}
\IntroStrumenti
\begin{itemize}
  \item Discord;
  \item Google Meet;
  \item Google Drive;
  \item Google Docs;
  \item Google Slides;
  \item GitHub;
  \item Visual Studio Code;
  \item Jira.
\end{itemize}