\subsection{Gestione}

\subsubsection{Descrizione}
\par Il processo di gestione include i task che vengono eseguiti dal responsabile per il coordinamento del processo.
\par Tra le attività principali del processo di gestione si possono distinguere:
\begin{itemize}
  \item Pianificazione;
  \item Esecuzione e controllo;
  \item Valutazione e approvazione.
\end{itemize}

\subsubsection{Pianificazione}
\par Questa fase include l'assegnazione di ruoli e attività. Ogni attività è corredata da:
\begin{itemize}
  \item Uno o più membri incaricati di svolgere l'attività;
  \item Data di inizio e fine;
  \item Priorità;
  \item Uno o più verificatori incaricati di valutare il risultato;
  \item Eventuali rischi associati all'attività.
\end{itemize}

\vspace{0.5\baselineskip}
\par Una volta pianificata un'attività, il responsabile e l'amministratore hanno il compito di aprire uno o più ticket su \glossario{Jira}, seguendo le linee guida riportate nella \sezione{gestione-configurazione}.

\subsubsection{Esecuzione e controllo}
\par Il responsabile provvede a far eseguire e mantenere il risultato della pianificazione, analizzando e risolvendo i problemi sorti durante l'avanzamento. Problemi e soluzioni saranno documentate.
\par Il responsabile inoltre si occupa di comunicare con gli \glossario{stakeholder}.

\subsubsection{Valutazione e approvazione}
Il responsabile assicura la soddisfazione dei requisiti del software o la completezza e correttezza della documentazione durante e alla fine dell'esecuzione dei rispettivi processi.

\subsubsection{Ruoli}
\par Il progetto didattico prevede l'assegnazione dei seguenti ruoli, con una rotazione costante e bilanciata che va considerata nella pianificazione.

\subsubsubsection{Responsabile}
\par Il responsabile è il ruolo che coordina le attività dell'intero gruppo internamente e gestisce i contatti con la Proponente e i Committenti come riferimento unico per tutto il gruppo.
\par La responsabilità del ruolo sono:
\begin{itemize}
  \item Pianificazione dello sprint (task);
  \item Sviluppo del preventivo dello sprint;
  \item Redazione e cura del \PdP;
  \item Controllo delle attività del gruppo;
  \item Valutazione e gestione dei rischi;
  \item Preparazione dell'ordine del giorno e moderazione delle riunioni;
  \item Sviluppo del consuntivo dello sprint.
\end{itemize}

\subsubsubsection{Amministratore}
\par L'amministratore è il ruolo legato alla gestione della configurazione e manutenzione dell'ambiente di sviluppo comune.
\par Le responsabilità del ruolo sono:
\begin{itemize}
  \item Attività legate al processo di gestione della configurazione;
  \item Cura delle \NdP\ (non come unico redattore, ma gestore delle modifiche da attuare);
  \item Misurazione delle metriche di qualità e aggiornamento del \PdQ;
\end{itemize}

\subsubsubsection{Analista}
\par L'analista è il ruolo legato all'ingegneria dei requisiti.
\par Le responsabilità dell'analista sono:
\begin{itemize}
  \item Svolgimento delle attività descritte nella \sezione{analisi-requisiti};
  \item Redazione del documento di \AdR.
\end{itemize}

\subsubsubsection{Progettista}
\par Il progettista definisce soluzioni architetturali e implementative per lo sviluppo del prodotto.
\par Le responsabilità del progettista sono:
\begin{itemize}
  \item Svolgimento delle attività descritte nella \sezione{progettazione};
  \item Redazione del documento Specifica Tecnica.
\end{itemize}

\subsubsubsection{Programmatore}
\par Il programmatore implementa l'architettura definita producendo codice che ne soddisfa le necessità. Inoltre si occupa di implementare test di unità e integrazione per la verifica del codice scritto
\par Le responsabilità del ruolo sono:
\begin{itemize}
  \item Attività descritte nella \sezione{codifica} e nella \sezione{testing-codice}.
\end{itemize}

\subsubsubsection{Verificatore}
\par Il verificatore controlla il risultato del lavoro degli altri ruoli accertando la qualità e il funzionamento del risultato delle attività.
\par Le responsabilità del ruolo sono:
\begin{itemize}
  \item Assicurare la qualità del contenuto prodotto dal gruppo, assicurando la coerenza con il \WoW\ stabilito e con gli obiettivi della modifica;
  \item Approvare un avanzamento di versione di un documento (escluso il rilascio).
\end{itemize}

\subsubsection{Gestione dei rischi}
\par La gestione dei rischi è un’attività svolta dal responsabile, con il supporto dell’amministratore per quanto riguarda i rischi legati all’infrastruttura. Questa attività è inclusa nel \PianoDiProgetto e viene eseguita alla fine di ogni sprint come parte del consuntivo di periodo. La pratica di gestione dei rischi alimenta e influenza l’analisi dei rischi, anch’essa riportata nel \PianoDiProgetto.

\subsubsubsection{Rischi: Notazione}
\par I rischi vengono identificati in modo univoco secondo questa notazione: 
\par \quad \textbf{R[Tipo][Numero]}
\par dove: 
\begin{itemize}
  \item \textbf{R}: indica la parola "rischio";
  \item \textbf{Tipo}: indica il tipo di rischio: 
  \begin{itemize}
    \item \textbf{O}: rischio organizzativo; 
    \item \textbf{T}: rischio tecnologico;
    \item \textbf{P}: rischio di natura personale.
  \end{itemize}
  \item \textbf{Numero}: è un numero progressivo che identifica in modo univoco il rischio per ogni tipologia.
\end{itemize}

\subsubsubsection{Rischi: Didascalia}
\par I rischi sono definiti come segue:
\begin{itemize}
  \item \textbf{Probabilità}: stima di occorrenza del rischio (Alta, Media, Bassa);
  \item \textbf{Grado di criticità}: impatto del rischio (Alto, Medio, Basso);
  \item \textbf{Descrizione}: descrizione del rischio;
  \item \textbf{Strategie di rilevamento}: metodi per anticipare l'insorgenza di situazioni problematiche;
  \item \textbf{Contromisure}: misure di mitigazione da applicare qualora il rischio si manifesti.
\end{itemize}

\subsubsection{Comunicazione}
\par Il gruppo adotta canali di comunicazione differenti per la comunicazione interna (tra i membri del gruppo) e quella esterna (con la Proponente e i Committenti). Di seguito sono illustrate le modalità di comunicazione e le applicazioni utilizzate.

\subsubsubsection{Comunicazione interna}
\par La comunicazione tra i membri del gruppo avviene attraverso le seguenti modalità di interazione:
\begin{itemize}
  \item \textbf{Telegram} (comunicazione asincrona): utilizzato per comunicazioni di interesse generale e di breve contenuto. È possibile creare gruppi di discussione per argomenti specifici. I messaggi più importanti vengono fissati all'interno della rispettiva chat;
  \item \textbf{Slack} (comunicazione asincrona): utilizzato per inviare promemoria ai verificatori riguardo alla revisione delle pull request;
  \item \textbf{Discord} (comunicazione sincrona): utilizzato per chiamate di gruppo, formali o informali, e per la comunicazione interna tra membri del team che ricoprono lo stesso ruolo. È possibile creare canali di testo e vocali per argomenti specifici. Inoltre, il gruppo utilizza Discord come piattaforma di supporto per la "pair programming", disponibile da remoto tramite condivisione schermo o per mezzo dell'estensione Live Share di \glossario{Visual Studio Code};
  \item \textbf{Google Meet} (comunicazione sincrona): fallback per le chiamate di gruppo, in caso di problemi con Discord;
  \item \textbf{Incontri in presenza} (comunicazione sincrona): il gruppo si riunisce in presenza con cadenza mensile (o bisettimanale a ridosso delle revisioni di avanzamento).
\end{itemize}

\vspace{0.5\baselineskip}
\par Questi strumenti devono essere integrati con un \glossario{Issue Tracking System}, poiché le decisioni prese durante gli incontri, a meno che non siano documentate in un verbale, sono più difficili da tracciare e recuperare in futuro.

\paragraph*{Strumenti}
\begin{itemize}
  \item \textbf{Telegram};
  \item \textbf{Slack};
  \item \textbf{Discord};
  \item \textbf{Google Meet}.
\end{itemize}

\subsubsubsection{Comunicazione esterna}
\par La comunicazione con la Proponente e i Committenti avviene attraverso le seguenti modalità di interazione:
\begin{itemize}
  \item \textbf{Gmail} (comunicazione asincrona): utilizzata per comunicazioni formali. La comunicazione esterna è gestita dal responsabile, attraverso il recapito di posta elettronica del gruppo \href{mailto:argo.unipd@gmail.com}{argo.unipd\-@\-gmail.com};
  \item \textbf{Zoom} (comunicazione sincrona): utilizzato per incontri formali.
\end{itemize}

\paragraph*{Strumenti}
\begin{itemize}
  \item \textbf{Gmail};
  \item \textbf{Zoom}.
\end{itemize}