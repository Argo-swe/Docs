\subsection{Gestione}

\subsubsection{Descrizione}
Il processo di gestione contiene le attività e i task che vengono adottati dal Responsabile per il coordinamento del processo.\\
Il processo consiste nelle seguenti attività:
\begin{itemize}
  \item Pianificazione;
  \item Esecuzione e controllo;
  \item Valutazione e approvazione;
\end{itemize}

\paragraph{Pianificazione}
Questa attività comprende tutta la programmazione di assegnazione ruoli e attività, scadenze e previsione del periodo corrente e dei successivi.

\paragraph{Esecuzione e controllo}
Il Responsabile provvede a far eseguire e mantenere il risultato della pianificazione, analizzando e risolvendo i problemi sorti durante l'avanzamento. Problemi e soluzioni saranno documentate.\\
Il Responsabile inoltre si occupa di comunicare con gli \glossario{stakeholder}.

\paragraph{Valutazione e approvazione}
Il Responsabile assicura la soddisfazione dei requisiti del software o la completezza e correttezza della documentazione durante e alla fine dell'esecuzione dei rispettivi processi.

\subsubsection{Ruoli}
Questo progetto didattico prevede l'assegnazione dei seguenti ruoli, con una rotazione costante e bilanciata che va considerata nella pianificazione.

\paragraph{Responsabile}
Il \Responsabile{} è il ruolo che coordina le attività dell'intero gruppo internamente e gestisce i contatti con la Proponente e i Committenti come riferimento unico per tutto il gruppo.\\
La responsabilità del ruolo sono:
\begin{itemize}
  \item Pianificazione dello sprint (task);
  \item Sviluppo del preventivo dello sprint;
  \item Redazione e cura del \PdP;
  \item Controllo delle attività del gruppo;
  \item Valutazione e gestione dei rischi;
  \item Preparazione dell'ordine del giorno e moderazione delle riunioni;
  \item Sviluppo del consuntivo dello sprint.
\end{itemize}

\paragraph{Amministratore}
L'\Amministratore{} è il ruolo legato alla gestione della configurazione e manutenzione dell'ambiente di sviluppo comune.\\
Le responsabilità del ruolo sono:
\begin{itemize}
  \item Attività legate al processo di gestione della configurazione;
  \item Cura delle \NdP{} (non come unico redattore, ma gestore delle modifiche da attuare);
  \item Misurazione delle metriche di qualità e aggiornamento del \PdQ;
\end{itemize}

\paragraph{Analista}
L'\Analista{} è il ruolo legato all'attività di analisi, del processo di sviluppo.
Le responsabilità del ruolo sono:
\begin{itemize}
  \item Attività descritte in \ref{analisi}
  \item Redazione e cura dell'\AdR.
\end{itemize}

\paragraph{Progettista}
Il \Progettista{} definisce soluzioni architetturali e implementative per lo sviluppo del prodotto.\\
Le responsabilità del ruolo sono:
\begin{itemize}
  \item Attività descritte in \ref{progettazione};
\end{itemize}

\paragraph{Programmatore}
Il \Programmatore{} implementa l'architettura definita producendo codice che ne soddisfa le necessità. Inoltre si occupa di implementare test di unità e integrazione per la verifica del codice scritto.\\
Le responsabilità del ruolo sono:
\begin{itemize}
  \item Attività descritte in \ref{codificatesting}.
\end{itemize}

\paragraph{Verificatore}
Il \Verificatore{} controlla il risultato del lavoro degli altri ruoli accertando la qualità e il funzionamento del risultato delle attività.\\
Le responsabilità del ruolo sono:
\begin{itemize}
  \item Assicurare la qualità del contenuto prodotto dal gruppo, assicurando la coerenza con il \WoW{} stabilito e con gli obiettivi della modifica;
  \item Approvare un avanzamento di versione di un documento (escluso il rilascio).
\end{itemize}

\subsubsection{Comunicazione}
\paragraph{Comunicazione interna}
La comunicazione tra i membri del gruppo è gestita attraverso Telegram e Discord.\\
Attraverso Telegram il gruppo comunica in modo asincrono e generale, pertanto è opportuno per comunicazioni di interesse di tutto il gruppo e di breve contenuto.\\
Attraverso Discord il gruppo partecipa a chiamate di gruppo, riunioni o meno, e tramite canali testuali divisi per ruolo è l'organizzazione interna di gruppi ristretti è favorita.\\
Questi strumenti non devono sovrapporsi tuttavia a mezzi di comunicazione e coordinamento, come ad esempio un \glossario{Issue Tracking System}, in quanto le infromazioni riportate tramite questi strumenti sono più difficilmente tracciabili e riferibili in momenti futuri.

\paragraph{Comunicazione esterna}
La comunicazione esterna è gestita dal Responsabile, attraverso il recapito di posta elettronica del gruppo \href{mailto:argo.unipd@gmail.com}{argo.unipd\-@\-gmail.com}.