\subsection{Miglioramento}
\subsubsection{Scopo}
Il processo di miglioramento si occupa di stabilire, riconoscere, misurare, controllare e migliorare i processi di ciclo di vita del software.

\subsubsection{Attività}
Il processo si compone di queste attività:
\begin{itemize}
  \item Stabilimento del processo;
  \item Valutazione del processo;
  \item Miglioramento del processo.
\end{itemize}

\subsubsection{Stabilimento del processo}
Vengono riconosciuti i processi da implementare e ne vengono documentate le norme stabilite nel \WoW. Quando opportuno, un meccanismo di controllo del processo va implementato per monitorare e migliorare il processo.

\subsubsection{Valutazione del proceso}
Il gruppo pianifica ed esegue revisioni dei processi ad intervalli regolari corrispondenti alle iterazioni di progetto, o più strettamente quando il processo richiede controllo costante, ad esempio durante le prime fasi di implementazione.\\
Le revisioni hanno l'obiettivo di assicurare l'efficacia del processo, individuando gli aspetti migliorabili per l'avvicinamento allo stato dell'arte.

\subsubsection{Miglioramento del processo}
Il gruppo effettua miglioramenti al processo evidenziati e riconosciuti durante la valutazione. Le norme inerenti al processo vengono aggiornate in modo da comprendere gli aspetti individuati.\\
Informazioni di storico, tecniche o di valutazione vengono collettivamente raccolte ed analizzate per riconoscere punti di forza e aspetti deboli dei processi impiegati. Il risultato dell'analisi viene impiegato come feedback per il miglioramento del processo.
