\subsection{Risoluzione dei problemi}\label{risoluzione_problemi}

\subsubsection{Scopo}
\par Il processo di risoluzione dei problemi serve ad analizzare, individuare e contrastare i problemi che possono insorgere durante lo sviluppo. Il gruppo si impegna ad analizzare continuamente lo stato del lavoro per individuare quanto prima la presenza di un problema che possa creare una criticità per il progetto. Individuato il problema, il gruppo si impegna ad arginarlo in modo da non sprecare risorse preziose allo sviluppo e mantenere l'efficienza. I metodi per segnalare i problemi sono molteplici: dalle discussioni durante i meeting, alla creazione di ticket su \glossario{Jira} che individuano attività critiche da risolvere, alle discussioni all'interno delle \glossario{pull request} su \glossario{GitHub} durante lo sviluppo di un branch.

\subsubsection{Implementazione del processo}
\par All'insorgere di una criticità all'interno del progetto, i membri del gruppo si impegnano a segnalare con quanto più preavviso possibile, l'insorgenza di un problema. Questo avviene con una comunicazione in uno degli strumenti utilizzati dal gruppo, come \glossario{Discord} o \glossario{Telegram}. Spesso le criticità possono anche essere rilevate durante le discussioni tra membri del gruppo, durante le riunioni.
\par Individuato il potenziale rischio, viene creato un opportuno ticket per la risoluzione del problema su \glossario{Jira}, in modo che uno o più membri del gruppo, lo possano prendere in carico per portarne avanti la risoluzione. La descrizione del problema è riportata sul ticket stesso o discussa tra i membri del gruppo nelgi appositi canali di comunicazione. 
\par Una volta individuato e segnalato il problema, questo viene assegnato a uno o più membri del gruppo, i quali si occuperanno di risolverlo e di validare la risoluzione.

\subsubsection{Risoluzione dei problemi}
\par Se il problema è stato rilevato all'interno del progetto in un branch già aperto in precedenza, i verificatori si occupano di segnalarlo con una correzione che viene poi integrata all'interno della \glossario{pull request}. 
\par Nel caso ciò non avvenga in un branch aperto, verrà aperto un branch apposito per la risoluzione. 

\subsubsection{Strumenti}
\IntroStrumenti
\begin{itemize}
    \item \textbf{Jira};
    \item \textbf{Zoom};
    \item \textbf{Discord};
    \item \textbf{Telegram};
    \item \textbf{Google Meet};
    \item \textbf{GitHub}.
\end{itemize}