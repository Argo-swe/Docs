\subsection{Revisione congiunta}\label{revisione_congiunta}

\subsubsection{Scopo}
\par Le revisioni congiunte consistono in attività per validare le attività di processo del gruppo da parte di un componente esterno, che ne valuta l'operato

\subsubsection{Implementazione del processo}
\par Le revisioni congiunte avvengono tra il gruppo e la \glossario{Proponente} per verificare che i requisiti individuati e la loro implementazione nell'applicativo, rispettino le aspettative della \glossario{Proponente}. Per fare ciò vengono organizzati degli incontri periodici tramite \glossario{Gmail} nei quali il gruppo propone un incontro. La \glossario{Proponente} risponde indicando il giorno e l'ora più opportuni. Da qui il gruppo conferma e si impegna ad inviare il link per la chiamata, effettuata tramite la piattaforma \glossario{Zoom}.
\par Durante l'incontro il gruppo espone i quesiti o le funzionalità sviluppate, interagendo ripetutamente con la \glossario{Proponente} per una validazione o un riscontro sull'operato o sui dubbi sorti.
\par Tutti gli incontri sono verbalizzati e sono poi sottoposti all'approvazione della \glossario{Proponente}.

\subsubsection{Revisioni tecniche congiunte}
\par Durante le revisioni vengono quasi sempre esposti dubbi sui requisiti individuati. Alternativamente le discusisoni vergono sull'approfondimento delle strategie di composizione del \glossario{prompt} e sulla correttezza delle query generate dagli \glossario{LLM}. Ciò viene fatto generando vari \glossario{prompt} mediante gli strumenti sviluppati dal gruppo: ogni \glossario{prompt} generato viene poi inserito in un \glossario{LLM} come \glossario{ChatGPT} o in altri modelli proposti in \glossario{chatLMsys} per avere un riscontro sulla correttezza delle query generate. Il tutto viene accompagnato da spiegazioni costruttive su come le query vengono costruite e perché potrebbero essere corrette o presentare potenziali rischi.

\subsection{Strumenti utilizzati}
\begin{itemize}
    \item \glossario{Zoom};
    \item \glossario{chatLMsys};
    \item \glossario{ChatGPT};
    \item \glossario{Gmail};
\end{itemize}