\paragraph*{Efficienza temporale}
\begin{itemize}
    \item \textbf{Notazione specifica}: M.PC.8;
    \item \textbf{Nome}: Efficienza temporale;
    \item \textbf{Descrizione}: Questa metrica misura il rapporto tra il tempo totale a disposizione (ore di orologio) e il tempo speso in attività produttive (ore produttive);
    \item \textbf{Motivo}: Valutare la capacità del team di utilizzare il tempo in modo efficiente per raggiungere obiettivi e completare attività pianificate. Un'efficienza temporale più alta indica una maggiore produttività e un uso ottimale del tempo a disposizione;
    \item \textbf{Misurazione}:
    \[
        \text{Efficienza temporale (\%)} = \frac{O_r}{O_p} \times 100
    \]
    dove:
    \begin{itemize}
        \item $O_{r}$: Ore di orologio;
        \item $O_{p}$: Ore produttive.
    \end{itemize}
\end{itemize}
