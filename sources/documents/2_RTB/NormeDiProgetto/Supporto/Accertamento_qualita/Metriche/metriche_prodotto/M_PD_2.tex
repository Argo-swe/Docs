\paragraph*{Requisiti desiderabili soddisfatti}
\begin{itemize}
    \item \textbf{Notazione specifica}: M.PD.2;
    \item \textbf{Nome}: Requisiti desiderabili soddisfatti;
    \item \textbf{Descrizione}: Questa metrica misura la percentuale di requisiti desiderabili soddisfatti rispetto al totale. I requisiti desiderabili sono definiti nel documento di \AnalisiDeiRequisiti;
    \item \textbf{Caratteristiche}: Funzionalità;
    \item \textbf{Motivo}: Valutare il grado di adempimento dei requisiti desiderabili, fornendo una misura del livello di soddisfazione del cliente;
    \item \textbf{Misurazione}:
    \[
    \text{Requisiti desiderabili soddisfatti (\%)} = \frac{R_{d}}{T_{d}} \times 100
    \]
    dove:
    \begin{itemize}
        \item $R_{d}$: Requisiti desiderabili soddisfatti;
        \item $T_{d}$: Numero totale dei requisiti desiderabili.
    \end{itemize}
\end{itemize}