\paragraph*{Requisiti opzionali soddisfatti}
\begin{itemize}
    \item \textbf{Notazione specifica}: M.PD.3;
    \item \textbf{Nome}: Requisiti opzionali soddisfatti;
    \item \textbf{Descrizione}: Questa metrica misura la percentuale di requisiti opzionali soddisfatti rispetto al totale. I requisiti opzionali sono definiti nel documento di \AnalisiDeiRequisiti;
    \item \textbf{Caratteristiche}: Funzionalità;
    \item \textbf{Motivo}: Valutare il grado di adempimento dei requisiti opzionali, fornendo una misura del livello di soddisfazione del cliente;
    \item \textbf{Misurazione}:
    \[
    \text{Requisiti opzionali soddisfatti (\%)} = \frac{R_{op}}{T_{op}} \times 100
    \]
    dove:
    \begin{itemize}
        \item $R_{op}$: Requisiti opzionali soddisfatti;
        \item $T_{op}$: Numero totale dei requisiti opzionali.
    \end{itemize}
\end{itemize}