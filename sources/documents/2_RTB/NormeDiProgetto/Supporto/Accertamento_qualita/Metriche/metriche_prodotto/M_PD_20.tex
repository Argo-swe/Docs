\paragraph*{Tolleranza agli errori}
\begin{itemize}
    \item \textbf{Notazione specifica}: M.PD.20;
    \item \textbf{Nome}: Tolleranza agli errori;
    \item \textbf{Descrizione}: Questa metrica misura la percentuale di errori che il prodotto è in grado di gestire;
    \item \textbf{Caratteristiche}: Affidabilità, operabilità;
    \item \textbf{Motivo}: Verificare che il software sia in grado di rilevare condizioni di errore e segnalarle con un opportuno messaggio.
    \item \textbf{Misurazione}:
    \[
        \text{Tolleranza agli errori} = \frac{Err_{s}}{Err_{p}} \times 100
    \]
    dove:
    \begin{itemize}
        \item $Err_{s}$: Numero di errori gestiti con successo;
        \item $Err_{p}$: Numero totale di errori previsti.
    \end{itemize}
\end{itemize}
