\subsection{Verifica}\label{verifica}

\subsubsection{Scopo}

\par Il processo di verifica ha lo scopo di determinare se i risultati di un'attività soddisfano i requisiti, le condizioni e i vincoli stabiliti nel \PianoDiQualifica. Questo processo può includere:
\begin{itemize}
  \item \textbf{Analisi};
  \item \textbf{Revisione};
  \item \textbf{Testing}.
\end{itemize}

\vspace{0.5\baselineskip}
\par I task coinvolti nel processo di verifica sono finalizzati a garantire l'adeguatezza, completezza e coerenza del prodotto. Questi task comprendono:
\begin{itemize}
  \item Verifica dei processi;
  \item Verifica dei requisiti software;
  \item Verifica del design;
  \item Verifica del codice sorgente;
  \item Verifica dell'integrazione;
  \item Verifica della documentazione.
\end{itemize}

\subsubsection{Descrizione}

\par Per assicurare la conformità dei risultati prodotti, a ogni azione di modifica è associato un passo di verifica. L’avanzamento di versione avviene soltanto a valle di verifica e conseguente approvazione. Il processo di revisione viene svolto dai membri impiegati nel ruolo di verificatore. Come indicato nella \sezione{sec:pull_request}, il gruppo ha definito delle Branch Protection Rules, al fine di garantire un'integrazione controllata delle modifiche all'interno del \glossario{repository}. In linea con le specifiche di GitHub, la verifica non può essere effettuata dallo stesso componente a cui è stato assegnato il task.

\subsubsection{Analisi statica}
\par TODO

\subsubsection{Analisi dinamica}
\par TODO