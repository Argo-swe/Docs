\subsection{Gestione della configurazione}

\subsubsection{Scopo}
\par La seguente sezione viene redatta con lo scopo di formalizzare e automatizzare le procedure applicate dal team, durante tutto il ciclo di vita del software, nell'ambito del processo di "configuration management".

\subsubsection{Descrizione}
\par Il processo di gestione della configurazione si occupa di definire e gestire le componenti software utilizzate durante l'intero corso del progetto per mantenere la tracciabilità e gestire il versionamento e i rilasci di prodotti software e documenti. Si tratta di un processo di applicazione di procedure amministrative e tecniche finalizzate a:
\begin{itemize}
  \item Identificare le componenti software del sistema e stabilire un punto di riferimento da cui misurare i progressi nel tempo;
  \item Controllare le modifiche e i rilasci degli item;
  \item Mantenere la tracciabilità delle modifiche;
  \item Garantire la completezza, coerenza e correttezza degli item.
\end{itemize}

\subsubsection{Issue tracking system (ITS)}
\par Al fine di registrare, gestire e monitorare le attività e sotto-attività lungo l'intero ciclo di vita del software, il team impiega l'\glossario{Issue Tracking System} sviluppato da Atlassian: \glossario{Jira}.

\subsubsubsection{Ticket}
\par I \glossario{ticket} possono essere di tre tipi:
\begin{itemize}
  \item \textbf{Task}: un'attività o un compito specifico da portare a termine entro la fine di uno \glossario{sprint} e assegnato a un unico membro del team;
  \item \textbf{Sottotask}: un'attività dispendiosa può essere suddivisa in più sotto-task, che possono essere associati a componenti diversi del gruppo;
  \item \textbf{\glossario{Bug}}: un \glossario{ticket} etichettato come \glossario{bug} segnala la presenza di un'anomalia da risolvere tempestivamente, relativamente al prodotto software, alla documentazione o all'infrastruttura di gestione del progetto;
  \item \textbf{\glossario{Story}}: detta anche "User Story", rappresenta una funzionalità più ampia del sistema.
\end{itemize}
\par Una volta selezionata la tiplogia di \glossario{ticket}, il responsabile può definire le seguenti proprietà:
\begin{itemize}
  \item \textbf{Riepilogo}:;
  \item \textbf{ID}: ;
  \item \textbf{Descrizione}: ;
  \item \textbf{Assegnatario}: ;
  \item \textbf{Epic}:
  \item \textbf{Sprint}:;
  \item \textbf{Ticket collegati}: ;
  \item \textbf{Sviluppo}: .
\end{itemize}

\subsubsection{Versionamento}
Il gruppo mantiene un versionamento per la documentazione nel formato:
\[ X.Y.Z \]
\begin{itemize}
  \item[X] Avanza alla approvazione del Responsabile, corrisponde percui ad ogni rilascio;
  \item[Y] Avanza ad ogni verifica completa del documento;
  \item[Z] Avanza ad ogni modifica verificata di un documento.
\end{itemize}

\subsubsection{Repository}
Il gruppo utilizza due repository, disponibili in \glossario{Github}:
\begin{itemize}
  \item Repository della documentazione: \href{https://github.com/argo-swe/docs}{https://github.com/argo-swe/docs}
  \item Repository del codice sorgente: \href{https://github.com/argo-swe/chatsql}{https://github.com/argo-swe/chatsql}
\end{itemize}
Il gruppo utilizza inoltre, per hosting del sito \href{https://argo-swe.github.io}{argo-swe.github.io}, un repository, da non considerare all'interno del workflow in quanto aggiornata e mantenuta solo come "vetrina" del gruppo.
\begin{itemize}
  \item Repository del sito github.io: \href{https://github.com/argo-swe/argo-swe.github.io}{https://github.com/argo-swe/argo-swe.github.io}
\end{itemize}

\paragraph{Repository \emph{Docs}}
Il repository contiene il codice sorgente in LaTeX di tutta la documentazione ufficiale generata durante il progetto, oltre all'ambiente utile alla generazione dei file PDF corrispondenti.\\
Il repository include un file \emph{README.md} che illustra brevemente lo scopo del repository e i componenti del gruppo, un file \emph{.gitignore} per escludere il tracciamento di file ausiliari o artefatti di compilazione.\\
La directory \emph{Logo} contiene le versioni ufficiali del logo del gruppo, in formato SVG o PNG.\\
La directory \emph{sources} contiene il codice sorgente per la documentazione, separato in due directory: \emph{model} contiene i file di utilizzo globale all'interno della documentazione, \emph{documents} contiene, in maniera ordinata per fasi di progetto, la documentazione ufficiale.\\
La directory \emph{tools} contiene gli strumenti \glossario{Docker} per utilizzare un ambiente unico nella compilazione e uno script per compilare autoamticamente uno o più documenti.\\
\bigskip
Il repository contiene un ramo base, in cui vengono inserite le versioni verificate dei documenti caricate nel repository attraverso \glossario{feature branch} su cui viene eseguita la verifica prima di eseguire \glossario{merge}.


\paragraph{Repository \emph{ChatSQL}}
TODO