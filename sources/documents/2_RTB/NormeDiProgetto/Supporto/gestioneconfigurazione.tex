\subsection{Gestione della configurazione}

\subsubsection{Scopo}
\par La seguente sezione viene redatta con lo scopo di formalizzare e automatizzare le procedure applicate dal team, durante tutto il ciclo di vita del software, nell'ambito del processo di "configuration management".

\subsubsection{Descrizione}
\par Il processo di gestione della configurazione si occupa di definire e gestire le componenti software utilizzate durante l'intero corso del progetto per mantenere la tracciabilità e gestire il versionamento e i rilasci di prodotti software e documenti. Si tratta di un processo di applicazione di procedure amministrative e tecniche finalizzate a:
\begin{itemize}
  \item Identificare le componenti software del sistema e stabilire un punto di riferimento da cui misurare i progressi nel tempo;
  \item Controllare le modifiche e i rilasci degli item;
  \item Mantenere la tracciabilità delle modifiche;
  \item Garantire la completezza, coerenza e correttezza degli item.
\end{itemize}

\subsubsection{Issue tracking system (ITS)}
\par Al fine di registrare, gestire e monitorare le attività e sottoattività lungo l'intero ciclo di vita del software, il team impiega l'\glossario{Issue Tracking System} sviluppato da Atlassian: \glossario{Jira}.

\subsubsubsection{Ticket}
\par I \glossario{ticket} possono essere di quattro tipi:
\begin{itemize}
  \item \textbf{Task}: un'attività o un compito specifico da portare a termine entro la fine di uno \glossario{sprint} e assegnato a un unico membro del team;
  \item \textbf{Sottotask}: un \glossario{ticket} subordinato che consente di orientarsi verso una scomposizione più granulare del lavoro. Un'attività, ritenuta troppo onerosa per una singola risorsa, può quindi essere suddivisa in più sottotask, associabili a diversi componenti del gruppo;
  \item \textbf{\glossario{Bug}}: un \glossario{ticket} marcato come \glossario{bug} segnala la presenza di un'anomalia da risolvere tempestivamente, relativamente al prodotto software, alla documentazione o all'infrastruttura di gestione del progetto;
  \item \textbf{\glossario{Story}}: detta anche "User Story", rappresenta una funzionalità del sistema, un requisito espresso dal punto di vista dell'utente.
\end{itemize}
\par Il layout di un \glossario{ticket} è formato dai seguenti campi:
\begin{itemize}
  \item \textbf{Riepilogo}: un titolo che riassume brevemente l'incarico associato al \glossario{ticket};
  \item \textbf{ID}: un codice univoco autoincrementante generato automaticamente dal sistema secondo la formula ARGO-ID;
  \item \textbf{Descrizione}: una descrizione esaustiva dei risultati attesi al completamento del \glossario{ticket};
  \item \textbf{Assegnatario}: il componente del gruppo a cui è stata assegnato il compito di risolvere il \glossario{ticket};
  \item \textbf{Epic}: esprime obiettivi generali o grandi porzioni di lavoro che devono essere frammentati. Ogni \glossario{ticket} può essere associato a un epic;
  \item \textbf{\glossario{Sprint}}: ciascun \glossario{ticket} può essere correlato a uno \glossario{sprint}, a sua volta scomposto in più epic;
  \item \textbf{Ticket collegati}: \glossario{Jira} offre una funzionalità, sia nei campi di contesto che nella timeline di pianificazione, per collegare i \glossario{ticket} tra loro. Di seguito sono riportate le associazioni predefinite:
    \begin{itemize}
      \item blocca/è bloccato da (queste sono le due dipendenze più comuni tra i task);
      \item clona/è clonato da;
      \item duplica/è duplicato da;
      \item item correlato a.
    \end{itemize} 
  \item \textbf{Sviluppo}: un campo di integrazione tra \glossario{Jira} e \glossario{GitHub} che permette di monitorare lo stato di avanzamento dello sviluppo, con annessi link ai \glossario{branch}, \glossario{commit}, \glossario{pull request}, \glossario{build} e \glossario{distribuzioni} associati al \glossario{ticket};
  \item \textbf{Stato}: durante il suo ciclo di vita, un \glossario{ticket} attraversa tre stati: "To Do", "Doing" e "Done".
  \item \textbf{Versioni di correzione}: le versioni rappresentano punti temporali e traguardi intermedi nello svolgimento del progetto. Ciascun \glossario{ticket} può essere associato a una determinata versione. L'associazione \glossario{ticket}/versione può essere realizzata direttamente dal \glossario{backlog} del progetto. Le versioni possono trovarsi in uno dei seguenti tre stati:
    \begin{itemize}
      \item Non rilasciate;
      \item Rilasciate;
      \item Archiviate.
    \end{itemize}
  \item \textbf{Commenti}: elenco di commenti da affiancare ai messaggi di \glossario{commit} per contestualizzare le modifiche e ottimizzare il lavoro di \glossario{verifica};
  \item \textbf{Aggiungi allegato}: oltre ai commenti, è possibile allegare file di vario formato che non necessitano del controllo di versione;
  \item \textbf{Aggiungi un ticket figlio}: ogni \glossario{ticket} può avere uno o più \glossario{ticket} subordinati;
  \item \textbf{Azioni}: \glossario{Jira} offre la possibilità di creare, gestire e monitorare automazioni, come ad esempio la chiusura di un \glossario{ticket} una volta effettuato il \glossario{merge} di una \glossario{pull request};
  \item \textbf{Rilasci}: elenco delle versioni rilasciate a cui il \glossario{ticket} è associato.
\end{itemize}

\subsubsection{Automazione chiusura ticket}
\par Su \glossario{Jira}, nelle impostazioni del progetto, è presente una sezione denominata “Automazione”, a sua volta suddivisa in quattro sottosezioni:
\begin{itemize}
  \item \textbf{Regole}: elenco delle regole definite dall'amministratore; ciascuna regola richiede un trigger di innesco, ossia un evento che avvia l'esecuzione della procedura definita nel corpo della regola. Una volta stabilito il trigger di attivazione, l’amministratore può scegliere una delle seguenti opzioni:
  \begin{itemize}  
    \item THEN: aggiungi un'azione (es: transizione di un \glossario{ticket} da uno stato all'altro);
    \item IF: aggiungi una condizione (es: verifica se lo stato del \glossario{ticket} è diverso da “Completato”);
    \item FOR EACH: applica le azioni e le condizioni ad ogni task (es: esamina tutti i \glossario{ticket} collegati al \glossario{ticket} che ha attivato la regola ed esegue per ciascuno di essi una determinata azione);
  \end{itemize}
  \item \textbf{Audit log}: cronologia delle automazioni avviate, con dettagli sullo stato di esecuzione della regola, la data di attivazione e gli elementi associati;
  \item \textbf{Modelli}: set di regole predefinite da importare nel progetto;
  \item \textbf{Utilizzo}: numero di automazioni attivate mensilmente.
\end{itemize}

\vspace{0.5\baselineskip}
\par Il team ha deciso di introdurre una regola personalizzata per effettuare automaticamente la transizione dello stato dei \glossario{ticket}. Quando viene aperta una \glossario{pull request} finalizzata alla chiusura di un \glossario{ticket}, il titolo della richiesta deve essere corredato dal codice identificativo del \glossario{ticket} (ARGO-ID). In alternativa, è possibile menzionare il \glossario{ticket} nel nome del \glossario{branch} o nei \glossario{commit} associati alla \glossario{pull request}. Inoltre, l’assegnatario può lasciare un commento nella forma [ARGO-ID], affinché un bot, denominato jira[bot], possa convertire il commento in un link al \glossario{ticket} \glossario{Jira}. Una volta effettuato il \glossario{merge} della \glossario{pull request} su \glossario{GitHub}, il \glossario{ticket} collegato passerà in automatico allo stato "Completato".
\par Utilizzando i modelli predefiniti, il gruppo ha aggiunto altre due regole, rispettivamente per:
\begin{itemize}
  \item Chiudere automaticamente un \glossario{ticket} quando tutti i \glossario{ticket} subordinati (task, story, bug, sottotask) sono completati;
  \item Chiudere automaticamente un \glossario{ticket} quando tutti i sottotask sono completati.
\end{itemize}


\subsubsection{Versionamento}
Il gruppo mantiene un versionamento per la documentazione nel formato:
\[ X.Y.Z \]
\begin{itemize}
  \item[X] Avanza alla approvazione del Responsabile, corrisponde percui ad ogni rilascio;
  \item[Y] Avanza ad ogni verifica completa del documento;
  \item[Z] Avanza ad ogni modifica verificata di un documento.
\end{itemize}

\subsubsection{Repository}
Il gruppo utilizza due repository, disponibili in \glossario{Github}:
\begin{itemize}
  \item Repository della documentazione: \href{https://github.com/argo-swe/docs}{https://github.com/argo-swe/docs}
  \item Repository del codice sorgente: \href{https://github.com/argo-swe/chatsql}{https://github.com/argo-swe/chatsql}
\end{itemize}
Il gruppo utilizza inoltre, per hosting del sito \href{https://argo-swe.github.io}{argo-swe.github.io}, un repository, da non considerare all'interno del workflow in quanto aggiornata e mantenuta solo come "vetrina" del gruppo.
\begin{itemize}
  \item Repository del sito github.io: \href{https://github.com/argo-swe/argo-swe.github.io}{https://github.com/argo-swe/argo-swe.github.io}
\end{itemize}

\paragraph{Repository \emph{Docs}}
Il repository contiene il codice sorgente in LaTeX di tutta la documentazione ufficiale generata durante il progetto, oltre all'ambiente utile alla generazione dei file PDF corrispondenti.\\
Il repository include un file \emph{README.md} che illustra brevemente lo scopo del repository e i componenti del gruppo, un file \emph{.gitignore} per escludere il tracciamento di file ausiliari o artefatti di compilazione.\\
La directory \emph{Logo} contiene le versioni ufficiali del logo del gruppo, in formato SVG o PNG.\\
La directory \emph{sources} contiene il codice sorgente per la documentazione, separato in due directory: \emph{model} contiene i file di utilizzo globale all'interno della documentazione, \emph{documents} contiene, in maniera ordinata per fasi di progetto, la documentazione ufficiale.\\
La directory \emph{tools} contiene gli strumenti \glossario{Docker} per utilizzare un ambiente unico nella compilazione e uno script per compilare autoamticamente uno o più documenti.\\
\bigskip
Il repository contiene un ramo base, in cui vengono inserite le versioni verificate dei documenti caricate nel repository attraverso \glossario{feature branch} su cui viene eseguita la verifica prima di eseguire \glossario{merge}.


\paragraph{Repository \emph{ChatSQL}}
TODO