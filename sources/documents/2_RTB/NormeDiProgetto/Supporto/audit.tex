\subsection{Audit}\label{audit}

\subsubsection{Scopo}
\par Il processo di audit serve a determinare l'adesione ai requisiti, alla pianificazione e ai vincoli del progetto. Attuare il processo richiede solamente due parti interlocutrici, di cui una revisiona le attività e/o il prodotto dell'altra.

\subsubsection{Implementazione del processo}
\par Le revisioni facente parte del processo sono pianificate e concordate, nel contenuto e nella data, da ciascuna delle parti. Criticità individuate durante la revisione vanno documentate e gestite, attuando il processo di risoluzione problemi (\ref{risoluzione_problemi}). I risultati di una revisione e le azioni conseguenti devono essere stabiliti in comune accordo tra le due parti e documentati.

\subsubsection{Revisioni di project management}
\par Le revisioni di project management sono principalmente attuate durante riunioni interne, rintracciate attraverso i corrispondenti verbali e tramite il consuntivo del periodo nel \PdP. Monitorano principalmente il corso delle attività svolte durante lo sprint oggetto della riunione e individuano azioni correttive da attuare a breve o medio termine, in base all'urgenza dei problemi individuati.

\subsubsection{Strumenti}
\begin{itemize}
  \item \glossario{LaTeX};
  \item \glossario{Discord};
  \item \glossario{Telegram};
  \item \glossario{Google Meet}.
\end{itemize}