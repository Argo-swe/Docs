\subsection{Fornitura}\label{fornitura}

\subsubsection{Descrizione}
Il \glossario{processo} di fornitura consiste nell'insieme di attività e compiti svolte dal \glossario{Fornitore} nel rapporto con la \glossario{Proponente} Zucchetti S.p.A.. Il processo parte dalla candidatura al \glossario{capitolato} d'appalto e prosegue con la determinazione di procedure e risorse richieste per la gestione e assicurazione del progetto, incluso lo sviluppo e l'esecuzione di un \glossario{Piano di Progetto}.
L'obiettivo principale del processo è confrontare le aspettative della Proponente con i risultati del Fornitore durante il periodo del progetto, mantenendo dunque una metrica oggettiva tra il preventivato e lo stato corrente.\\
Il processo consiste nelle seguenti attività:
\begin{itemize}
  \item Selezione e studio fattibilità;
  \item Candidatura;
  \item Pianificazione;
  \item Esecuzione e controllo;
  \item Revisione e valutazione;
  \item Consegna e completamento.
\end{itemize}

\paragraph{Selezione e studio fattibilità}
Il Fornitore esamina i capitolati d'appalto e arriva a una decisione sulla candidatura per uno di essi.

\paragraph{Candidatura}
Il Fornitore definisce e prepara una candidatura al capitolato d'appalto scelto producendo i seguenti documenti:
\begin{itemize}
  \item \textbf{Lettera di Candidatura:} presentazione del gruppo rivolta al \glossario{Committente};
  \item \textbf{Stima dei Costi e Assunzione Impegni:} documento che contiene un preventivo sulla distribuzione ore del progetto, il suo costo, una pianificazione generale e una iniziale analisi dei rischi;
  \item \textbf{Valutazione Capitolati:} documento che contiene l'analisi e la valutazione da parte del gruppo dei capitolati disponibili.
\end{itemize}

\paragraph{Pianificazione}
Il Fornitore stabilisce i requisiti per la gestione, lo svolgimento e la misurazione della qualità del progetto. In seguito, sviluppa e documenta attraverso il \glossario{Piano di Progetto} i risultati attesi.

\paragraph{Esecuzione e controllo}
Il Fornitore esegue il Piano di Progetto sviluppato, attenendosi alle norme definite nella sezione \ref{sviluppo} e monitora la qualità del \glossario{prodotto software} nei seguenti modi:
\begin{itemize}
  \item Controllo del progresso di \glossario{performance}, costi, rendicontazione dello stato del progetto e organizzazione;
  \item Identificazione, tracciamento, analisi e risoluzione dei problemi.
\end{itemize}

\paragraph{Revisione e valutazione}
Il Fornitore coordina la revisione interna ed esegue verifica e validazione secondo le norme definite in \ref{verifica} e \ref{validazione}. Questo avviene in modo continuo e iterativo.



\subsubsection{Rapporti con la Proponente}
%TO BE DEFINED
TODO

\subsubsection{\glossario{Documentazione} fornita}\label{documentazionefornita}
Di seguito viene descritta la documentazione che il gruppo si impegna a rendere disponibile alla Proponente e ai Committenti.

\paragraph{Piano di Progetto}
%Breve descrizione dello scopo del documento e delle singole sezioni.
TODO

\paragraph{Analisi dei Requisiti}
%Breve descrizione dello scopo del documento e delle singole sezioni.
TODO

\paragraph{Piano di Qualifica}
%Breve descrizione dello scopo del documento e delle singole sezioni.
TODO

\paragraph{Lettera di Presentazione}
TODO

\paragraph{Glossario}
Raccolta esaustiva di tutti i termini tecnici utilizzati nella documentazione. Permette di eliminare ambiguità e fraintendimenti fornendo una defizione univoca ed esaustiva per l'intero gruppo e per chi consulta la documentazione prodotta.




\subsubsection{Strumenti}
Gli strumenti impiegati nel processo di fornitura sono:
\begin{itemize}
  \item \textbf{LaTeX:} \glossario{markup language} utilizzato per la redazione della documentazione;
  \item \textbf{Git:} \glossario{Version Control System} utilizzato dal gruppo;
  \item \textbf{Zoom:} Strumento per videochiamate utilizzato nei rapporti con la Proponente;
  \item \textbf{Google Sheets:} Strumento per la creazione di \glossario{spreadsheet} condivisi, utilizzato per la pianificazione di \glossario{sprint} e \glossario{rendicontazione ore}.
\end{itemize}
