\subsection{Fornitura}\label{fornitura}

\subsubsection{Descrizione}
Il \glossario{processo} di fornitura consiste nell'insieme di attività e compiti svolte dal \glossario{Fornitore} nel rapporto con la \glossario{Proponente} Zucchetti S.p.A.. Il processo parte dalla candidatura al \glossario{capitolato} d'appalto e prosegue con la determinazione di procedure e risorse richieste per la gestione e assicurazione del progetto, incluso lo sviluppo e l'esecuzione di un \glossario{Piano di Progetto}.
L'obiettivo principale del processo è confrontare le aspettative della Proponente con i risultati del Fornitore durante il periodo del progetto, mantenendo dunque una metrica oggettiva tra il preventivato e lo stato corrente.\\
Il processo consiste nelle seguenti attività:
\begin{itemize}
  \item Selezione e studio fattibilità;
  \item Candidatura;
  \item Pianificazione;
  \item Esecuzione e controllo;
  \item Revisione e valutazione;
  \item Consegna e completamento.
\end{itemize}

\paragraph{Selezione e studio fattibilità}
Il Fornitore esamina i capitolati d'appalto e arriva a una decisione sulla candidatura per uno di essi.

\paragraph{Candidatura}
Il Fornitore definisce e prepara una candidatura al capitolato d'appalto scelto producendo i seguenti documenti:
\begin{itemize}
  \item \textbf{Lettera di Candidatura:} presentazione del gruppo rivolta al \glossario{Committente};
  \item \textbf{Stima dei Costi e Assunzione Impegni:} documento che contiene un preventivo sulla distribuzione ore del progetto, il suo costo, una pianificazione generale e una iniziale analisi dei rischi;
  \item \textbf{Valutazione Capitolati:} documento che contiene l'analisi e la valutazione da parte del gruppo dei capitolati disponibili.
\end{itemize}

\paragraph{Pianificazione}
Il Fornitore stabilisce i requisiti per la gestione, lo svolgimento e la misurazione della qualità del progetto. In seguito, sviluppa e documenta attraverso il \glossario{Piano di Progetto} i risultati attesi.

\paragraph{Esecuzione e controllo}
Il Fornitore esegue il Piano di Progetto sviluppato, attenendosi alle norme definite nella sezione \ref{sviluppo} e monitora la qualità del \glossario{prodotto software} nei seguenti modi:
\begin{itemize}
  \item Controllo del progresso di \glossario{performance}, costi, rendicontazione dello stato del progetto e organizzazione;
  \item Identificazione, tracciamento, analisi e risoluzione dei problemi.
\end{itemize}

\paragraph{Revisione e valutazione}
Il Fornitore coordina la revisione interna ed esegue verifica e validazione secondo le norme definite in \ref{verifica} e \ref{validazione}. Questo avviene in modo continuo e iterativo.



\subsubsection{Rapporti con la Proponente}
%TO BE DEFINED
TODO

\subsubsection{\glossario{Documentazione} fornita}\label{documentazionefornita}
Di seguito viene descritta la documentazione che il gruppo si impegna a rendere disponibile alla Proponente e ai Committenti.

\paragraph{Piano di Progetto:}
%Breve descrizione dello scopo del documento e delle singole sezioni.
Il \PdP\ è il documento che tratta della gestione e dell'organizzazione del gruppo. Tratta dell'analisi e la gestione dei rischi, valuta lo stato del progetto rappresentando la divisione delle ore per \glossario{sprint}, il preventivo e il consuntivo orari ed economici e definisce una retrospettiva dello \glossario{sprint} precedente e la pianificazione degli \glossario{sprint} successivi.
Il documento è diviso nelle seguenti sezioni:
\begin{enumerate}
  \item \textbf{Introduzione:} indica lo scopo del documento e del prodotto con eventuali riferimenti normativi e informativi;
  \item \textbf{Analisi dei rischi:} quali sono i rischi ai quali il gruppo può andare incontro, l'impatto dei rischi sulle risorse del progetto, le strategie di rilevamento e di mitigazione per contrastarli. L'analisi dei rischi viene suddivisa in sezioni in base alla categoria del rischio:
  \begin{itemize}
    \item Rischi tecnologici;
    \item Rischi organizzativi;
    \item Rischi di natura personale;
  \end{itemize} 
  \item \textbf{Modello di sviluppo:} descrive il modello di sviluppo adottato dal gruppo, definendone la struttura e i punti di forza;
  \item \textbf{Stima temporale del progetto:} indica la stima delle date delle revisioni alle quali il gruppo si deve sottoporre. Viene indicata l'ultima data di aggiornamento della sezione;
  \item \textbf{Stima dei costi:} serve a dare una stima preliminare sui costi del progetto. Viene aggiornata ad ogni \glossario{sprint};
  \item \textbf{Pianificazione:} serve a pianificare lo \glossario{sprint} definendo gli obbiettivi che il gruppo vuole raggiungere. Viene disposta una sezione di pianificazione per ogni \glossario{sprint};
  \item \textbf{Preventivo:} dispone il preventivo per lo \glossario{sprint}. Ogni \glossario{sprint} ha sezione di preventivo. Il preventivo si compone di:
  \begin{itemize}
    \item Preventivo orario: tabella di distribuzione delle ore preventivate per ciascun membro del gruppo;
    \item Distribuzione ore per la coppia risorsa-ruolo: istogramma che indica le ore preventivate per ciascun ruolo;
    \item Distribuzione ore per ruolo: aereogramma che indica la distribuzione in percentuale preventivata per i ruoli;
    \item Preventivo economico: tabella che riporta il costo del ruolo per il numero di ore preventivate. Riporta il totale, ossia il preventivo economico per lo \glossario{sprint}.
  \end{itemize}
  \item \textbf{Consuntivo:} dispone il consuntivo per lo \glossario{sprint}. Ogni \glossario{sprint} ha sezione di consuntivo. Il consuntivo si compone di:
  \begin{itemize}
    \item Consuntivo orario: tabella di distribuzione delle ore effettivamente impiegate per ciascun membro del gruppo;
    \item Distribuzione ore per la coppia risorsa-ruolo: istogramma che indica le ore impiegate per ciascun ruolo;
    \item Distribuzione ore per ruolo: aereogramma che indica la distribuzione in percentuale impiegata per i ruoli;
    \item Consuntivo economico: tabella che riporta il costo del ruolo per il numero di ore impiegate. Riporta il totale, ossia il consuntivo economico per lo \glossario{sprint}.
    \item Copertura oraria rispetto al totale: aereogramma che indica la percentuale di tempo speso in rapporto al tempo totale per il completamento del progetto;
    \item Budget speso rispetto al totale: aereogramma che indica la percentuale di budget speso rispetto al budget totale per il completamento del progetto;
    \item Ore rimanenti per la coppia risorsa-ruolo: riporta il numero di ore per ruolo rimanenti ad ogni membro del gruppo. Indica anche le ore rimanenti in totale per membro;
    \item Revisione delle attività: indica le attività svolte durante lo \glossario{sprint};
    \item Retrospettiva: indica i risultati del questionario che ogni membro deve compilare al termine di ogni \glossario{sprint} per valutarne l'andamento. Oltre a ciò, vengono raccolte delle considerazioni sullo \glossario{sprint} come la presenza di rischi;
    \item Aggiornamento pianificazione e preventivo: definisce le azioni migliorative per il prossimo \glossario{sprint} e una prima pianificazione dei macro obiettivi da esaminare nella pianificazione dello \glossario{sprint} successivo. Delinea inoltre la gestione dei rischi, ossia i rischi rilevati durante lo \glossario{sprint}, il loro impatto e la loro mitigazione.
  \end{itemize}
\end{enumerate} 

\paragraph{Analisi dei Requisiti:}
L'\AdR\ è il documento che tratta l'analisi dei requisiti richiesti dal progetto e individuati dal gruppo. Sviluppa inoltre i casi d'uso per questi, ossia le interazioni tra il sistema e l'utente.
Il documento è diviso nelle seguenti sezioni:
\begin{enumerate}
  \item \textbf{Introduzione:} indica lo scopo del documento e del prodotto con eventuali riferimenti normativi e informativi;
  \item \textbf{Descrizione:} indica le funzioni principali del prodotto e le caratteristiche dell'utente, ossia i motivi che spingono l'utente ad usare l'applicativo;
  \item \textbf{Casi d'uso:} indica tutti i casi d'uso individuati dal gruppo durante l'analisi. La struttura dei casi d'uso è la seguente:
  \begin{itemize}
    \item La struttura di un UC è divisa nella seguente struttura:
    \begin{itemize}
      \item UCN - Nome UC;
      \item Descrizione;
      \item Attori principali;
      \item Precondizioni;
      \item Postcondizioni;
      \item Trigger;
      \item Scenario principale;
      \item Eventuale Scenario alternativo;
      \item Eventuale Inclusioni;
      \item Eventuali Estensioni;
      \item Eventuali sottocasi d'uso.
    \end{itemize}
    \item Le precondizioni dello UC devono essere condizioni necessarie per arrivare alla situazione che si presenta nello UC. Ad esempio per l'utilizzo dello strumento di debug per il Tecnico, saranno:
    \begin{itemize}
      \item Il Tecnico ha effettuato il login;
      \item Il Tecnico ha ricevuto un prompt generato sulla base di una richiesta in liguaggio naturale;
      \item Il Tecnico decide di utilizzare lo strumento di debug.
    \end{itemize}
    \item Le precondizioni vengono poste al passato per identificare la conclusione di un UC, possono includere altri UC;
    \item Le postcondizioni rappresentano cosa succede dopo lo sviluppo dello UC e sono per tanto descrittive e poste al presente;
    \item Lo scenario principale riprende i passaggi che sono stati necessari per il verificarsi dello UC. Pertanto si parte a descriverle dalla prima estensione che l'attore riscontra dopo l'avvio dell'applicativo fino ad arrivare allo UC;
    \item Lo scenario alternativo riprende i passaggi dello scenario principale fino all'estensione che porta allo UC alternativo;
    \item Ogni riferimento ad un UC viene riportato in precondizioni, postcondizioni, scenario principale, scenario alternativo, inclusione ed estensione;
    \item I sottocasi d'uso di uno UC vengono inseriti sotto lo UC principale nello stesso file;
    \item I sottocasi sono sempre inclusioni del caso d'uso "padre";
    \item I sottocasi vengono riferiti con un punto dopo il padre. Ad esempio un sottocaso potrebbe essere UC.1 e il sottocaso del sottocaso UC.1.1;
    \item Per gli errori si visualizza quasi sempre un messaggio. Quindi la postcondizione finale e la fine dello scenario principale sarà quasi sempre "Viene visualizzato un messaggio con i dettagli dell'errore";
    \item Il Trigger è l'azione che l'utente vuole svolgere e che viene soddisfatta dallo UC.
  \end{itemize}
  \item \textbf{Requisiti:} elenca i requisiti in forma tabulare e li raggruppa per categoria. Le categorie sono:
  \begin{itemize}
    \item Funzionali: corrispondono alle funzionalità del sistema;
    \item Qualitativi: garantiscono la qualità del prodotto;
    \item Di vincolo: indicano le restrizioni e i vincoli normativi del progetto;
    \item Obbligatorio: requisiti inderogabili;
    \item Desiderabile: requisiti non obbligatori, ma comunque di interesse per la \glossario{Proponente};
    \item Opzionale: l'implementazione di questi requisiti è lasciata alla discrezione del \glossario{Fornitore}.
    \item Sono inoltre indicate le fonti pero ogni requisito, ossia lo use case corrispondente o il riferimento ad un documento, nel quale il requisito veniva menzionato.
  \end{itemize}
\end{enumerate}

\paragraph{Piano di Qualifica:}
%Breve descrizione dello scopo del documento e delle singole sezioni.
Il \PdQ\ è il documento che si tratta della specifica degli obbiettivi quantitativi di qualità di prodotto e processo. Delinea quindi un'insieme di indici di vautazione e validazione del progetto.
Il documento è diviso nelle seguenti sezioni:
\begin{enumerate}
  \item \textbf{Introduzione:} indica lo scopo del documento e del prodotto con eventuali riferimenti normativi e informativi;
  \item \textbf{Obiettivi di qualità:} la sezione illustra i valori accettabili e ambiti per le metriche individuate dal team. Viene divisa per metriche e ogni sotto sezione è dotata di una tabella che descrive le righe delle metriche con le colonne: ID della metrica, nome, valore tollerabile e valore ambito. Le metriche sono così divise:
  \begin{itemize}
    \item Qualità di processo: indicatori utilizzati per monitorare e valutare la qualità dei processi coinvolti nello sviluppo software. 
    \item Qualità di prodotto: valutano in modo obiettivo le caratteristiche quantitative e qualitative del software e assicurano la conformità agli standard di qualità del cliente.
    \item Qualità per obiettivo: misura la qualità dei processi primari, di supporto e organizzativi dello sviluppo software.
  \end{itemize}
  \item \textbf{Verifica:} sezione dedicata alla verifica e al collaudo del software. Misura a rilevare, correggere e prevenire i possibili errori del software. È divisa in sezioni in base alla tipologia di test. Le sezioni sono dotate di tabelle che descrivono le righe dei test con le colonne: ID, descrizione e stato del test. I test sono così divisi:
  \begin{itemize}
    \item Test di unità:  attività di collaudo di singole \glossario{unità} del software;
    \item Test di integrazione: verificano che i diversi moduli, componenti o servizi utilizzati dall’applicazione funzionino in modo integrato;
    \item Test di sistema:  controllano il comportamento del sistema nel suo complesso e verificano che l’applicazione funzioni secondo i requisiti specificati
    \item Test di accettazione sono test formali che precedono il rilascio del prodotto e valutano se l’applicazione è conforme alle aspettative del cliente:
    \item Checklist: sono strumenti che affiancano il team nell'attività di ispezione. Sono diverse dai test riportati in precedenza poiché la loro tabella descrive le righe di checklist con le colonne: titolo e descrizione.
  \end{itemize}
  \item \textbf{Cruscotto di valutazione delle qualità:} sezione dedicata alla dimostrazione del rispetto delle norme. Le sottosezioni sono norme alle quali è stata data una rappresentazione grafica che ne misura l'andamento durante gli \glossario{sprint}. Il grafico in questione rileva, oltre all'andamento generale, tre rette che rappresentano il valore ambito, il valore tollerabile e la tendenza. Ogni grafico è dotato di una descrizione che ne descrive l'andamento. Le metriche nel cruscotto di qualità vengono aggiornate periodicamente per aggiornarle agli \glossario{sprint} più recenti.
\end{enumerate}

\paragraph{Lettera di Presentazione}
TODO

\paragraph{Glossario}
Raccolta esaustiva di tutti i termini tecnici utilizzati nella documentazione. Permette di eliminare ambiguità e fraintendimenti fornendo una defizione univoca ed esaustiva per l'intero gruppo e per chi consulta la documentazione prodotta.

\subsubsection{Strumenti}
Gli strumenti impiegati nel processo di fornitura sono:
\begin{itemize}
  \item \textbf{LaTeX:} \glossario{markup language} utilizzato per la redazione della documentazione;
  \item \textbf{Git:} \glossario{Version Control System} adottato dal gruppo;
  \item \textbf{Zoom:} Piattaforma per le riunioni con la Proponente e i Committenti;
  \item \textbf{Google Sheets:} Strumento per la creazione di \glossario{spreadsheet} condivisi, utilizzato per la pianificazione di \glossario{sprint} e per la rendicontazione delle ore;
  \item \textbf{Google Moduli:} Strumento per la creazione di questionari di valutazione degli \glossario{sprint};
  \item \textbf{Table Convert Online:} Strumento online per convertire file \glossario{csv} in \glossario{LaTeX} (disponibile al seguente link \href{https://tableconvert.com/it/csv-to-latex}{https://tableconvert.com/it/csv-to-latex});
  \item \textbf{PDF24 Tools:} Strumento online per convertire file SVG in PDF (disponibile al seguente link \href{https://tools.pdf24.org/it/svg-in-pdf}{https://tools.pdf24.org/it/svg-in-pdf}).
\end{itemize}

\subsubsubsection{Dashboard Google Sheets}
Lo \glossario{spreadsheet} condiviso, realizzato tramite \glossario{Google Sheets} e visibile su \glossario{Google Drive}, ha lo scopo di automatizzare la generazione di preventivi e consuntivi (orari ed economici). Il foglio di calcolo principale è suddiviso nei seguenti fogli interni:
\begin{itemize}
  \item Un foglio contenente le variabili globali del \PdP, inclusi il costo orario di ciascun ruolo e il budget totale;
  \item Un foglio nascosto per ciascun periodo, che include le risposte al questionario di valutazione dello sprint;
  \item Un foglio per ogni sprint con le seguenti informazioni:
  \begin{itemize}
    \item Numero dello \glossario{sprint};
    \item Date di inizio e termine dello \glossario{sprint};
    \item Tabella di assegnazione delle ore produttive per ciascun membro del team, accumulate in totali per persona e per ruolo;
    \item Distribuzione delle ore per ruolo, sotto forma di donut chart;
    \item Distribuzione delle ore per la coppia risorsa-ruolo, sotto forma di grafico a barre sovrapposte (così da poter assumere più ruoli per sprint);
    \item Preventivo economico dello \glossario{sprint};
    \item Tabella riassuntiva con ore e budget spesi e restanti;
    \item Pie chart con la stima delle ore spese sul totale;
    \item Pie chart con la stima del budget sul totale;
    \item Ore rimanenti per la coppia risorsa-ruolo.
  \end{itemize}
\end{itemize}
\vspace{0.5\baselineskip}
\par La dashboard è stata progettata per affiancare il responsabile nella stesura delle seguenti sezioni del \PdP:
\begin{itemize}
  \item Preventivo;
  \item Consuntivo.
\end{itemize}

\paragraph*{Preventivo}
Il responsabile può duplicare il foglio di calcolo dello \glossario{sprint} precedente e modificare le seguenti informazioni:
\begin{itemize}
  \item \textbf{Sprint-ID}, dove ID corrisponde al numero dello \glossario{sprint};
  \item I riferimenti temporali, nel formato "dal aaaa-mm-gg al aaaa-mm-gg";
  \item \textbf{Preventivo orario:} per ciascuna coppia risorsa-ruolo, il responsabile deve inserire nella cella apposita le ore produttive previste;
  \item La tabella "preventivo economico" si aggiorna dinamicamente.
\end{itemize}
\par Per esportare le tabelle in formato \glossario{csv}, il team ha creato un foglio di calcolo aggiuntivo chiamato "Export". Il responsabile può copiare una tabella e incollarla su questo foglio, tramite le combinazioni di tasti "Ctrl+C" e "Ctrl+Shift+V". Una volta scaricato il file csv, il responsabile può convertire i dati in \glossario{LaTeX} e inserirli nel \PdP.
\par I grafici sono generati automaticamente e si dividono in due categorie:
\begin{itemize}
  \item \textbf{Grafici a torta 3D:} possono essere scaricati direttamente in formato PDF;
  \item \textbf{Grafici a barre sovrapposte:} devono essere scaricati in SVG e poi convertiti in PDF.
\end{itemize}
\par Il PDF è un formato vettoriale, il che significa che le immagini possono essere scalate senza perdita di qualità. Questo è particolarmente utile per diagrammi e grafici, poiché mantengono la nitidezza anche se ingranditi. Inoltre, il formato PDF è compatibile e facilmente integrabile con \glossario{LaTeX}.
\par Nella sezione rendicontazione ore, il responsabile deve inserire:
\begin{itemize}
  \item Le date di inizio e fine dello \glossario{sprint};
  \item I ruoli di ciascun componente del team;
  \item Un link al questionario di valutazione dello \glossario{sprint}.
\end{itemize}
\par Il questionario è un modulo \glossario{Google Moduli} che può essere creato all'interno della dashboard. Il questionario ha la seguente struttura:
\begin{itemize}
  \item \textbf{Titolo:} Valutazione Sprint-ID, dove ID corrisponde al numero dello \glossario{sprint};
  \item \textbf{Descrizione:} Questionario per la valutazione dello sprint-ID;
  \item \textbf{Domande} (scelta multipla, scala lineare da 1 a 10, risposta breve, scala lineare da 1 a 5, paragrafo):
  \begin{itemize}
    \item "Come ti è sembrata l'organizzazione dello sprint?";
    \item "Come si potrebbe migliorare la pianificazione?";
    \item "Sapevi sempre cosa fare nel tuo ruolo?";
    \item "Spiega i motivi della risposta precedente (organizzazione, inesperienza, ecc.)";
    \item "Il numero di riunioni è stato adeguato?";
    \item "Le riunioni sono state organizzate con il giusto preavviso?";
    \item "Come ti è sembrata la conduzione dei meeting interni?";
    \item "Come ti è sembrata la conduzione dei meeting esterni?";
    \item "Quanto ti sei impegnato/a in questo sprint?";
    \item "Qual è stato il rapporto ore spese/ore produttive?";
    \item "Quali azioni correttive avvieresti dal prossimo sprint?".
  \end{itemize}
\end{itemize}
\par Dopo aver creato un nuovo modulo, il responsabile può utilizzare la funzione "Importa domande". Questa feature consente di importare quesiti da un modulo esistente. Cliccando il pulsante "Invia", è possibile inoltre copiare il link da incollare nel foglio di calcolo. Nella dashboard \glossario{Google Sheets} viene aggiunto in automatico un foglio contenente le risposte al questionario; i fogli degli sprint precedenti possono essere nascosti.

\paragraph*{Consuntivo}
Tutte le tabelle del consuntivo vengono aggiornate automaticamente in base alla rendicontazione delle ore. Di seguito sono riporate le tre tabelle che compongono il consuntivo:
\begin{itemize}
  \item \textbf{Consuntivo orario:} il team ha definito una formula dinamica che somma le ore produttive per la coppia risorsa-ruolo. In questo modo è possibile automatizzare il calcolo del consuntivo anche quando i membri del team assumono più ruoli. La somma delle ore produttive per la coppia risorsa-ruolo è arrotondata per difetto;
  \item \textbf{Ore rimanenti} per la coppia risorsa-ruolo: viene calcolata la differenza tra le ore rimanenti al termine dello \glossario{sprint} precedente e le ore impiegate nello \glossario{sprint} attuale;
  \item \textbf{Consuntivo economico}, formato dai seguenti campi:
  \begin{itemize}
    \item Ruolo;
    \item Ore per ruolo;
    \item Delta ore preventivo - consuntivo: differenza tra le ore preventivate e quelle effettivamente spese;
    \item Costo (in €);
    \item Delta costo preventivo - consuntivo: differenza tra il costo preventivato e quello effettivo.
    \item Ore e budget spesi negli \glossario{sprint} precedenti;
    \item Ore e budget restanti.
  \end{itemize}
\end{itemize}
\par Il processo di esportazione di tabelle e grafici segue le stesse regole del preventivo. Tutte le tabelle e i grafici del consuntivo devono essere inseriti nel \PdP. Una volta completata la stesura del consuntivo nel \PdP, il responsabile deve aggiornare le variabili globali nel foglio "Pdp-global":
\begin{itemize}
  \item \textbf{Ultimo-Sprint:} ID, dove ID è il numero dell'ultimo \glossario{sprint};
  \item \textbf{Preventivo complessivo} (da modificare qualora sia necessaria una ridistribuzione delle ore per ruolo):
  \begin{itemize}
    \item Ruolo;
    \item Ore per ruolo;
    \item Ore individuali;
    \item Costo orario (in €);
    \item Costo totale (in €);
    \item Ore e budget restanti, ricavati dal consuntivo economico dell'ultimo sprint.
  \end{itemize}
\end{itemize}
\par Se il preventivo complessivo dovesse mutare, sia la tabella che il grafico corrispondente andrebbero aggiornati nel \PdP.

\paragraph*{Rendicontazione ore} Ciascun foglio di calcolo dello \glossario{sprint} include una sezione dedicata alla rendicontazione delle ore. La tabella è organizzata come segue:
\begin{itemize}
  \item Data;
  \item Membro del team:
  \begin{itemize}
    \item Ore produttive;
    \item Ruolo;
    \item Descrizione delle attività.
  \end{itemize}
\end{itemize}
