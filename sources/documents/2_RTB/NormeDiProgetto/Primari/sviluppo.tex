\subsection{Sviluppo}\label{sviluppo}

\subsubsection{Descrizione}
Il \glossario{processo} di sviluppo contiene le attività e compiti dello \glossario{sviluppatore} sotto elencate:
\begin{itemize}
  \item Analisi dei requisiti;
  \item \glossario{Progettazione};
  \item \glossario{Codifica} e \glossario{testing}.
\end{itemize}


\subsubsection{Analisi dei Requisiti}\label{analisi}
\paragraph{Descrizione}
L'Analisi dei Requisiti è eseguita dall'Analista, che redige l'omonimo documento \AdR. Il documento considera i seguenti aspetti:
\begin{itemize}
  \item TODO % Elenco capitoli e conenuti AdR
\end{itemize}

TODO % Es.: Spiegazione diagrammi e sintassi casi d'uso

\subsubsection{Progettazione}\label{progettazione}
\paragraph{Descrizione}
L'attività, svolta dal Progettista segue quella di analisi e ha il compito di impostare un'\glossario{architettura} del software capace di soddisfare i requisiti definiti. Il Progettista sviluppa l'architettura attraverso la creazione di unità e di relazioni tra loro, utilizzando opportunamente dei \glossario{design pattern} architetturali.\\

\subsubsection{Codifica e testing}\label{codificatesting}
\paragraph{Descrizione}
La codifica segue l'attività di progettazione e viene svolta dal Programmatore. Ha lo scopo di trasformare l'architettura prodotta dal Progettista in codice rispettando le norme definite per ottenere codice mantenibile e di qualità. Il testing è una parte stessa dell'attività di codifica, necessaria ad assicurare la correttezza di ciascuna unità software.

\paragraph{Stile di codifica}
Di seguito viene definito lo stile di codifica per i principali linguaggi di programmazione utilizzato dal gruppo. Queste liste non sono esaustive di tutte le \glossario{best practices} da integrare nella codifica, rappresentano invece scelte nette del gruppo a fronte di più possibili opzioni stilistiche.

\subparagraph{Elementi comuni}
\begin{itemize}
  \item La lingua da utilizzare nella nomenclatura di termini all'interno del codice è inglese, sono esenti commenti o contenuti di testo che necessitano la lingua italiana;
  \item Sezioni incomplete vanno indicate con un commento con indicato TODO, sezioni di codice non funzionanti o da rivedere vanno indicate con FIXME per una individuazione più semplice e una ricerca agevolatta all'interno del codice.
\end{itemize}

\subparagraph{Python}
\begin{itemize}
  \item \textbf{Tab size:} 4;
  \item \textbf{Nome delle variabili:} minuscolo;
  \item \textbf{Nome dei metodi:} camelCase;
\end{itemize}

\subparagraph{JavaScript/TypeScript}
TODO