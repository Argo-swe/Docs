\subsection{Sviluppo}\label{sviluppo}

\subsubsection{Descrizione}
Il \glossario{processo} di sviluppo contiene le attività e compiti dello \glossario{sviluppatore} sotto elencate:
\begin{itemize}
  \item Analisi dei requisiti;
  \item \glossario{Progettazione};
  \item \glossario{Codifica} e \glossario{testing}.
\end{itemize}


\subsubsection{Analisi dei Requisiti}\label{analisi}
\paragraph{Descrizione}
L'Analisi dei Requisiti è eseguita dall'Analista, che redige l'omonimo documento \AdR. Il documento considera i seguenti aspetti:
\begin{itemize}
  \item Descrizione del prodotto e caratteristiche ad alto livello;
  \item Elenco dei casi d'uso;
  \item Elenco dei requisiti.
\end{itemize}

\paragraph{Casi d'uso}
La struttura di un UC è divisa nella seguente struttura:
\begin{itemize}
  \item UCN - Nome UC;
  \item Descrizione;
  \item Attori principali;
  \item Precondizioni;
  \item Postcondizioni;
  \item Trigger;
  \item Scenario principale;
  \item Eventuale Scenario alternativo;
  \item Eventuale Inclusioni;
  \item Eventuali Estensioni;
  \item Eventuali sottocasi d'uso.
\end{itemize}
Nella scrittura e definizione di un UC va tenuto conto delle seguentio considerazioni:
\begin{itemize}
  \item Le precondizioni dello UC devono essere condizioni necessarie per arrivare alla situazione che si presenta nello UC.
  \item Le precondizioni vengono poste al passato per identificare la conclusione di un UC, possono includere altri UC;
  \item Le postcondizioni rappresentano cosa succede dopo lo sviluppo dello UC e sono pertanto descrittive e poste al presente;
  \item Lo scenario principale riprende i passaggi che sono stati necessari per il verificarsi dello UC. Pertanto si parte a descriverle dalla prima estensione che l'attore incontra dopo l'avvio dell'applicativo fino ad arrivare allo UC;
  \item Lo scenario alternativo riprende i passaggi dello scenario principale fino all'estensione che porta allo UC alternativo;
  \item Ogni riferimento ad un UC viene riportato in precondizioni, postcondizioni, scenario principale, scenario alternativo, inclusione ed estensione;
  \item I sottocasi d'uso di uno UC vengono inseriti sotto lo UC principale nello stesso file;
  \item I sottocasi sono sempre inclusioni del caso d'uso "padre";
  \item I sottocasi vengono riferiti con un punto dopo il padre. Ad esempio un caso d'uso potrebbe essere UC1 e il sottocaso UC1.1;
  \item Per gli errori si visualizza quasi sempre un messaggio. Quindi la postcondizione finale e la fine dello scenario principale sarà quasi sempre "Viene visualizzato un messaggio con i dettagli dell'errore";
  \item Il Trigger è l'azione che l'utente vuole svolgere e che viene soddisfatta dallo UC.
\end{itemize}

\paragraph{Requisiti}
I requisiti del prodotto, delineati durante il processo di analisi, si suddividono nelle seguenti categorie:
\begin{itemize}
  \item \textbf{Funzionali (F):} corrispondono alle funzionalità del sistema;
  \item \textbf{Qualitativi (Q):} garantiscono la qualità del prodotto;
  \item \textbf{Di vincolo (V):} indicano le restrizioni e i vincoli normativi del progetto;
\end{itemize}
Ciasun requisito ha anche un valore di importanza:
\begin{itemize}
  \item \textbf{Obbligatorio (O):} requisiti inderogabili;
  \item \textbf{Desiderabile (D):} requisiti non obbligatori, ma comunque di interesse per la \glossario{Proponente};
  \item \textbf{Opzionale (OP):} l'implementazione di questi requisiti è lasciata alla discrezione del \glossario{Fornitore}.
\end{itemize}
Data tale classificazione, i requisiti vengono identificati da un indice univoco con la seguente struttura:
\[R[Tipologia].[Importanza].[Codice]\]
Dove \emph{R} sta per Requisito, \emph{Tipologia} è la sigla associata alla categoria del requisito, \emph{Importanza} è la sigla per il valore di importanza e \emph{Codice} è un valore numerico univoco.\\
Descrivere le fonti è una pratica necessaria per orientare sia l'Analista che gli altri membri del gruppo verso una definizione più ampia e chiara del requisito, per cui ciascuna fonte di un requisito viene indicata assieme alla sua descrizione.


\subsubsection{Progettazione}\label{progettazione}
\paragraph{Descrizione}
L'attività, svolta dal Progettista segue quella di analisi e ha il compito di impostare un'\glossario{architettura} del software capace di soddisfare i requisiti definiti. Il Progettista sviluppa l'architettura attraverso la creazione di unità e di relazioni tra loro, utilizzando opportunamente dei \glossario{design pattern} architetturali.\\

\subsubsection{Codifica e testing}\label{codificatesting}
\paragraph{Descrizione}
La codifica segue l'attività di progettazione e viene svolta dal Programmatore. Ha lo scopo di trasformare l'architettura prodotta dal Progettista in codice rispettando le norme definite per ottenere codice mantenibile e di qualità. Il testing è una parte stessa dell'attività di codifica, necessaria ad assicurare la correttezza di ciascuna unità software.

\paragraph{Stile di codifica}
Di seguito viene definito lo stile di codifica per i principali linguaggi di programmazione utilizzato dal gruppo. Queste liste non sono esaustive di tutte le \glossario{best practices} da integrare nella codifica, rappresentano invece scelte nette del gruppo a fronte di più possibili opzioni stilistiche.

\subparagraph{Elementi comuni}
\begin{itemize}
  \item La lingua da utilizzare nella nomenclatura di termini all'interno del codice è inglese, sono esenti commenti o contenuti di testo che necessitano la lingua italiana;
  \item Sezioni incomplete vanno indicate con un commento con indicato TODO, sezioni di codice non funzionanti o da rivedere vanno indicate con FIXME per una individuazione più semplice e una ricerca agevolatta all'interno del codice.
\end{itemize}

\subparagraph{Python}
\begin{itemize}
  \item \textbf{Tab size:} 4;
  \item \textbf{Nome delle variabili:} minuscolo;
  \item \textbf{Nome dei metodi:} camelCase;
\end{itemize}

\subparagraph{JavaScript/TypeScript}
TODO