\subsection{Sviluppo}\label{sviluppo}

\subsubsection{Descrizione}
Il \glossario{processo} di sviluppo contiene le attività e compiti dello \glossario{sviluppatore} sotto elencate:
\begin{itemize}
  \item \glossario{Analisi dei requisiti};
  \item \glossario{Progettazione};
  \item \glossario{Codifica} e \glossario{testing}.
\end{itemize}


\subsubsection{Analisi dei Requisiti}
\paragraph{Descrizione}
L'Analisi dei Requisiti è eseguita dall'\glossario{\Analista}, che redige l'omonimo documento \AnalisiDeiRequisiti. Il documento considera i seguenti aspetti:
\begin{itemize}
  \item TODO % Elenco capitoli e conenuti AdR
\end{itemize}

TODO % Es.: Spiegazione diagrammi e sintassi casi d'uso

\subsubsection{Progettazione}
\paragraph{Descrizione}
L'attività, svolta dal \glossario{\Progettista} segue quella di analisi e ha il compito di impostare un'\glossario{architettura} del software capace di soddisfare i requisiti definiti. Il Progettista sviluppa l'architettura attraverso la creazione di unità e di relazioni tra loro, utilizzando opportunamente dei \glossario{design pattern} architetturali.\\

TODO

\subsubsection{Codifica e testing}
\paragraph{Descrizione}
La codifica segue l'attività di progettazione e viene svolta dal \glossario{\Programmatore}. Ha lo scopo di trasformare l'architettura prodotta dal Progettista in codice rispettando le norme definite per ottenere codice mantenibile e di qualità. Il testing è una parte stessa dell'attività di codifica, necessaria ad assicurare la correttezza di ciascuna unità software.


TODO
