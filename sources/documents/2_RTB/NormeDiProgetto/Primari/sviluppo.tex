\subsection{Sviluppo}\label{sviluppo}

\subsubsection{Descrizione}
Il \glossario{processo} di sviluppo contiene le attività e compiti dello \glossario{sviluppatore} sotto elencate:
\begin{itemize}
  \item Analisi dei requisiti;
  \item \glossario{Progettazione};
  \item \glossario{Codifica} e \glossario{testing}.
\end{itemize}


\subsubsection{Analisi dei Requisiti}\label{analisi-requisiti}
\paragraph{Descrizione}
L'Analisi dei Requisiti è eseguita dall'Analista, che redige l'omonimo documento \AdR. Il documento considera i seguenti aspetti:
\begin{itemize}
  \item Descrizione del prodotto e caratteristiche ad alto livello;
  \item Elenco dei casi d'uso;
  \item Elenco dei requisiti.
\end{itemize}

\paragraph{Casi d'uso}
La struttura di un UC è divisa nella seguente struttura:
\begin{itemize}
  \item UCN - Nome UC;
  \item Descrizione;
  \item Attori principali;
  \item Precondizioni;
  \item Postcondizioni;
  \item Trigger;
  \item Scenario principale;
  \item Eventuale Scenario alternativo;
  \item Eventuale Inclusioni;
  \item Eventuali Estensioni;
  \item Eventuali sottocasi d'uso.
\end{itemize}
Nella scrittura e definizione di un UC va tenuto conto delle seguentio considerazioni:
\begin{itemize}
  \item Le precondizioni dello UC devono essere condizioni necessarie per arrivare alla situazione che si presenta nello UC.
  \item Le precondizioni vengono poste al passato per identificare la conclusione di un UC, possono includere altri UC;
  \item Le postcondizioni rappresentano cosa succede dopo lo sviluppo dello UC e sono pertanto descrittive e poste al presente;
  \item Lo scenario principale riprende i passaggi che sono stati necessari per il verificarsi dello UC. Pertanto si parte a descriverle dalla prima estensione che l'attore incontra dopo l'avvio dell'applicativo fino ad arrivare allo UC;
  \item Lo scenario alternativo riprende i passaggi dello scenario principale fino all'estensione che porta allo UC alternativo;
  \item Ogni riferimento ad un UC viene riportato in precondizioni, postcondizioni, scenario principale, scenario alternativo, inclusione ed estensione;
  \item I sottocasi d'uso di uno UC vengono inseriti sotto lo UC principale nello stesso file;
  \item I sottocasi sono sempre inclusioni del caso d'uso "padre";
  \item I sottocasi vengono riferiti con un punto dopo il padre. Ad esempio un caso d'uso potrebbe essere UC1 e il sottocaso UC1.1;
  \item Per gli errori si visualizza quasi sempre un messaggio. Quindi la postcondizione finale e la fine dello scenario principale sarà quasi sempre "Viene visualizzato un messaggio con i dettagli dell'errore";
  \item Il Trigger è l'azione che l'utente vuole svolgere e che viene soddisfatta dallo UC.
\end{itemize}

\paragraph{Requisiti}
I requisiti del prodotto, delineati durante il processo di analisi, si suddividono nelle seguenti categorie:
\begin{itemize}
  \item \textbf{Funzionali (F)}: corrispondono alle funzionalità del sistema;
  \item \textbf{Qualitativi (Q)}: garantiscono la qualità del prodotto;
  \item \textbf{Di vincolo (V)}: indicano le restrizioni e i vincoli normativi del progetto;
\end{itemize}
Ciasun requisito ha anche un valore di importanza:
\begin{itemize}
  \item \textbf{Obbligatorio (O)}: requisiti inderogabili;
  \item \textbf{Desiderabile (D)}: requisiti non obbligatori, ma comunque di interesse per la \glossario{Proponente};
  \item \textbf{Opzionale (OP)}: l'implementazione di questi requisiti è lasciata alla discrezione del \glossario{Fornitore}.
\end{itemize}
Data tale classificazione, i requisiti vengono identificati da un indice univoco con la seguente struttura:
\[R[Tipologia].[Importanza].[Codice]\]
Dove \emph{R} sta per Requisito, \emph{Tipologia} è la sigla associata alla categoria del requisito, \emph{Importanza} è la sigla per il valore di importanza e \emph{Codice} è un valore numerico univoco.\\
Descrivere le fonti è una pratica necessaria per orientare sia l'Analista che gli altri membri del gruppo verso una definizione più ampia e chiara del requisito, per cui ciascuna fonte di un requisito viene indicata assieme alla sua descrizione.

\paragraph{Grafici}
I grafici dei casi d'uso sono così composti:
\begin{itemize}
  \item È sempre presente almeno un attore che interagisce con il le parti del sistema. L'attore è indicato con la figura di riferimento in UML ed è dotato di un nome per identificarne il tipo;
  \item È presente il frame che identifica in quale parte del sistema si sta svolgendo il caso d'uso. È così composto:
  \begin{itemize}
    \item All'esterno del frame sono presenti gli attori che interagiscono con le sue parti;
    \item All'interno vengono rappresentati i casi d'uso e l'interazione che questi hanno tra di loro e con gli attori;
    \item Ogni frame è dotato di un'etichetta. Il termine 'ChatSQL' si riferisce al sistema nel suo complesso, mentre la notazione "UC[ID]" indica che il diagramma rappresenta i sotto-casi di un caso d'uso specifico.
  \end{itemize}
  \item Le connessioni tra attore e casi d'uso sono rappresentate da segmenti continui e un attore può interagire con molteplici casi d'uso;
  \item Le connessioni tra casi d'uso sono rappresentate da segmenti tratteggiati e direzionati;
  \item Questi segmenti sono dotati di etichetta che ne riporta il nome e rispettano lo standard UML per i casi d'uso;
  \item Gli ellissi che rappresentano i casi d'uso sono dotati di un'etichetta dove viene riportato il caso d'uso e il nome di questo;
  \item Se avviene un'estensione da un caso d'uso verso un'altro, questa viene esplicitata anche attraverso un \glossario{extension point}, il quale spiega le condizioni che portano allo scenario alternativo.
\end{itemize}

\subsubsection{Progettazione}\label{progettazione}
\subsubsubsection{Scopo}
\par L'attività di progettazione segue quella di analisi e ha lo scopo di definire un'architettura per il prodotto capace di soddisfare i requisiti specificati nell’\AnalisiDeiRequisiti. Questa attività viene svolta dai progettisti e prevede l'utilizzo di stili e pattern architetturali. La progettazione ha dunque il compito di assicurare il soddisfacimento delle caratteristiche individuate durante l'analisi. In genere, l'architettura è un sistema formato da componenti, che raggruppano delle unità software composte da uno o più moduli.

\par Una buona architettura dovrebbe possedere i seguenti attributi di qualità misurabili:
\begin{itemize}
  \item \textbf{Sufficienza}: adeguato livello di soddisfacimento dei requisiti;
  \item \textbf{Modularità}: scomposizione in elementi distinti tra loro e con compiti non sovrapposti;
  \item \textbf{Robustezza}: tolleranza a diverse classi di input;
  \item \textbf{Flessibilità}: capacità di adattarsi a cambiamenti sia adattivi che evolutivi;
  \item \textbf{Riusabilità}: agevola il riuso dei componenti in altri contesti;
  \item \textbf{Efficienza}: utilizzo ottimale delle risorse;
  \item \textbf{Affidabilità}: capacità di mantenere il livello di prestazioni in condizioni specifiche;
  \item \textbf{Semplicità}: minimizza i componenti ed evita l'introduzione di soluzioni o dettagli troppo complessi;
  \item \textbf{Incapsulamento}: nasconde i dettagli implementativi (black-box), con l'obiettivo di ridurre le dipendenze e aumentare la riusabilità e la manutenibilità;
  \item \textbf{Coesione}: gli elementi che concorrono agli stessi obiettivi devono essere raggruppati nello stesso componente;
  \item \textbf{Basso accoppiamento}: limita le dipendenze tra elementi distinti.
\end{itemize}

\subsubsubsection{Proof of Concept}\label{poc}
Oltre all’analisi dei requisiti, una precondizione necessaria per la revisione \glossario{RTB} è la dimostrazione della validità delle tecnologie scelte dal team. Sulla base delle richieste e delle necessità espresse dalla \glossario{Proponente}, il team valuta i pro e contro delle tecnologie disponibili, considerando fattori come la curva di apprendimento, le prestazioni e la modernità. Una volta ultimata la scelta tecnologica, i progettisti e i programmatori devono realizzare un \glossario{Proof of Concept}. Qualora il team ritenesse le tecnologie selezionate inadeguate, i progettisti e i programmatori saranno responsabili della realizzazione di un ulteriore Proof of Concept (\glossario{PoC}), avvalendosi, ove possibile, delle soluzioni già sviluppate. Lo scopo del PoC è:
\begin{itemize}
  \item Verificare la fattibilità del progetto;
  \item Dimostrare che il team è in grado di soddisfare i requisiti mediante l'uso delle tecnologie scelte;
  \item Acquisire un controllo quanto più completo delle tecnologie;
  \item Ricevere un feedback dalla \glossario{Proponente} in merito al grado di conformità delle soluzioni sviluppate rispetto al capitolato.
\end{itemize}

\subsubsubsection{Fasi di progettazione}\label{fasi-progettazione}
\par L'attività di progettazione è suddivisa in due fasi: 
\begin{itemize}
  \item \textbf{Progettazione logica};
  \item \textbf{Progettazione di dettaglio}.
\end{itemize}

\subsubsubsection{Progettazione logica}\label{progettazione-logica}
\par La progettazione logica (o progettazione architetturale) è la prima fase di progettazione e ha lo scopo di delineare l'architettura software. I progettisti definiscono la struttura complessiva del sistema, identificando i componenti principali, le loro responsabilità e le relazioni tra di essi. Per assicurare la coerenza e la consistenza dell'architettura, il team prevede l'uso di stili architetturali, che forniscono una soluzione progettuale di "alto livello". Questa fase è essenziale per garantire che il sistema risponda ai requisiti funzionali e non funzionali. La progettazione logica include le seguenti attività:
\begin{itemize}
  \item Individuazione dei principali componenti del sistema;
  \item Definizione delle interfacce e delle modalità di comunicazione tra i componenti;
  \item Definizione delle strutture dati necessarie per soddisfare i requisiti;
  \item Stesura iniziale del manuale utente;
  \item Progettazione dei test di integrazione;
  \item Revisione delle soluzioni e degli stili architetturali.
\end{itemize}

\subsubsubsection{Progettazione di dettaglio}\label{progettazione-dettaglio}
\par La progettazione di dettaglio precede l’attività di codifica e la vincola in modo sostanziale. In questa fase vengono definiti gli attributi e i metodi delle classi, inclusi i relativi tipi di dato. Il team agisce sulle unità architetturali, che devono essere progettate con una complessità tale da consentire la codifica da parte di un singolo programmatore. Le unità dell’architettura vengono suddivise in moduli; più unità formano un componente (definito durante la progettazione logica). Lo scopo della progettazione di dettaglio è organizzare e disciplinare la fase di programmazione, documentando le scelte progettuali, assegnando i requisiti a ciascuna unità e definendo gli strumenti necessari per eseguire i test di unità. Il team deve trovare il giusto equilibrio tra scomposizione del sistema e complessità di coordinamento tra i moduli. La progettazione di dettaglio include le seguenti attività:
\begin{itemize}
  \item Descrizione dettagliata dei componenti, con enfasi sulla definizione specifica di ciascuna unità e modulo;
  \item Aggiornamento ed espansione del manuale utente;
  \item Definizione delle specifiche dei test di integrazione;
  \item Definizione delle specifiche dei test di unità.
\end{itemize}

\subsubsubsection{Specifica tecnica}\label{specifica-tecnica}
\par La fase di progettazione deve essere documentata in modo chiaro ed esaustivo, poiché costituisce la base fondamentale per le successive attività di codifica e testing del software. Le scelte progettuali sono dettagliate nel documento 'Specifica tecnica', la cui stesura inizia al termine della prima fase della revisione \glossario{RTB}.

\subsubsection{Codifica}\label{codifica}

\subsubsubsection{Scopo}
\par La codifica, effettuata dai programmatori, ha l'obiettivo di sviluppare il software in conformità con le specifiche architetturali definite dai progettisti. I programmatori traducono la struttura del sistema in codice sorgente utilizzando uno o più linguaggi di programmazione. Questa fase influisce sia sul testing che sulla manutenzione del software. Durante lo sviluppo, è essenziale che i programmatori aderiscano agli standard di programmazione stabiliti.

\subsubsubsection{Elementi comuni di stile}\label{sec:elementi-comuni-stile}
\par Di seguito sono riportati gli elementi comuni di stile:
\begin{itemize}
  \item \textbf{Nomi autoesplicativi} (self-explanatory): i nomi di classi, metodi e variabili devono essere sufficientemente descrittivi da rendere chiaro il loro significato, scopo o contenuto senza necessità di ulteriori commenti esplicativi, salvo nei casi in cui la complessità del codice lo richieda. È fondamentale trovare un equilibrio tra lunghezza e chiarezza descrittiva;
  \item \textbf{Lingua}: utilizzare la lingua inglese per la nomenclatura all'interno del codice sorgente, inclusi i commenti;
  \item \textbf{Abbreviazioni}: utilizzare abbreviazioni e acronimi consolidati o riconosciuti dalla comunità informatica;
  \item \textbf{Sezioni incomplete}: vanno indicate con un commento preceduto dalla dicitura "TODO";
  \item \textbf{Sezioni in attesa di revisione o miglioramento}: vanno indicate con un commento preceduto dalla dicitura "FIXME";
  \item \textbf{Indentazione}: utilizzare quattro spazi o un carattere di tabulazione;
  \item \textbf{Spaziatura}: aggiungere uno spazio attorno agli operatori binari e ternari per migliorare la leggibilità del codice.
\end{itemize}

\subsubsubsection{Stile di codifica: Python}
\paragraph*{Struttura dei file}
\begin{itemize}
  \item \textbf{Import}: insieme di istruzioni per includere moduli, librerie o pacchetti nello script:
  \begin{itemize}
    \item Librerie standard o di terze parti;
    \item Librerie interne o moduli personalizzati.
  \end{itemize}
  \item \textbf{Variabili globali}: variabili accessibili in tutto il programma;
  \item \textbf{Classi}: definiscono un tipo di oggetto o raggruppano funzioni di utilità.
\end{itemize}

\paragraph*{Struttura delle classi}
\begin{itemize}
  \item \textbf{Riferimenti}: lista di classi base (ereditarietà) o interfaccia implementata;
  \item \textbf{Costruttore}: metodo speciale chiamato \verb|__init__| (inizializzatore);
  \item \textbf{Metodi}: funzioni definite all'interno di una classe;
  \item \textbf{Attributi}: variabili definite all'interno di una classe.
\end{itemize}

\paragraph*{Documentazione del codice}
\vspace{0.5\baselineskip}
\par La documentazione del codice sorgente deve essere conforme alle linee guida di Google relative alla scrittura di \glossario{docstring} in Python. In particolare:
\begin{itemize}
  \item Ogni classe deve essere corredata da:
  \begin{itemize}
    \item Panoramica del ruolo o dello scopo della classe;
    \item Descrizione dettagliata (opzionale);
    \item Lista degli argomenti del costruttore;
    \item Lista degli attributi, accompagnati da una breve descrizione; 
    \item Lista dei metodi.
  \end{itemize}
  \item Ogni metodo deve essere corredato da:
  \begin{itemize}
    \item Panoramica del comportamento o della funzionalità del metodo;
    \item Descrizione dettagliata (opzionale);
    \item Lista degli argomenti, accompagnati da una breve descrizione; 
    \item Lista degli eventuali valori di ritorno, accompagnati da una breve descrizione;
    \item Lista delle eventuali eccezioni sollevate, accompagnate da una breve descrizione.
  \end{itemize}
\end{itemize}

\paragraph*{Best practices}
\begin{itemize}
  \item \textbf{Import}: importare moduli, librerie o pacchetti all'inizio del file, evitando l'uso dell'asterisco;
  \item \textbf{Self}: utilizzare la parola chiave "self" per riferirsi all'istanza dell'oggetto su cui viene chiamato un metodo;
  \item \textbf{Metodi setter e getter}: utilizzare metodi dedicati per interagire con gli attributi, evitando l'accesso diretto alle proprietà di una classe;
  \item \textbf{Concordanza nomi attributi-argomenti}: gli argomenti di un metodo devono avere lo stesso nome degli attributi corrispondenti nella classe;
  \item \textbf{Lunghezza massima riga (codice)}: 79 caratteri;
  \item \textbf{Lunghezza massima riga (commenti)}: 72 caratteri;
  \item \textbf{Lunghezza massima riga (docstring)}: 72 caratteri.
\end{itemize}

\paragraph*{Convenzioni sintattiche}
\begin{itemize}
  \item \textbf{Nomenclatura}: seguire le convenzioni stabilite da PEP 8;
  \item \textbf{Nomi dei file}: \glossario{snake\_case};
  \item \textbf{Nomi della classi}: \glossario{PascalCase};
  \item \textbf{Nomi dei metodi}: snake\_case;
  \item \textbf{Nomi delle variabili}: snake\_case;
  \item \textbf{Nomi delle costanti}: \glossario{UPPER\_SNAKE\_CASE};
  \item \textbf{Commenti}: utilizzare il simbolo \verb|#| per i commenti;
  \item \textbf{Docstring}: utilizzare il simbolo \verb|"""| per le docstring;
  \item \textbf{Stringhe}: utilizzare il simbolo \verb|'| oppure \verb|'''| per le stringhe.
\end{itemize}

\paragraph*{Formattazione del codice}
\par Il team aderisce alle convenzioni di formattazione stabilite da PEP 8 (Python Enhancement Proposal 8), la guida ufficiale per lo stile di codifica Python pubblicata dalla Python Software Foundation (PSF). Il rispetto delle linee guida è assicurato e automatizzato tramite uno strumento integrato nell'estensione Python di Visual Studio Code.

\subsubsubsection{Struttura front-end}
\par Il front-end contiene le seguenti cartelle principali:
\begin{itemize}
  \item \textbf{public}: risorse statiche come immagini, temi e favicon;
  \item \textbf{src/components}: componenti riutilizzabili;
  \item \textbf{src/components/layout}: componenti che definiscono la struttura generale dell'applicazione (es.: header, footer, sidebar);
  \item \textbf{src/views}: pagine dell'applicazione;
  \item \textbf{src/services}: moduli per la gestione delle interazioni con servizi interni o esterni (chiamate API, autenticazione, messaggi, servizi di utilità);
  \item \textbf{src/assets}: risorse come font e fogli di stile del layout;
  \item \textbf{src/locales}: file JSON per la gestione dell'interfaccia multilingua;
  \item \textbf{src/router}: definizione delle \glossario{route}.
\end{itemize}

\subsubsubsection{Componenti (Vue.js)}
\par Un SFC (Single File Component) è suddiviso in 3 sezioni:
\begin{itemize}
  \item \textbf{Script}: definisce la logica e il comportamento del componente. Il tag da utilizzare è \verb|<script setup>|, che gestisce automaticamente l'esportazione del componente e migliora la leggibilità e la concisione;
  \item \textbf{Template}: definisce la struttura del componente;
  \item \textbf{Style}: definisce lo stile e l'aspetto del componente.
\end{itemize}

\vspace{0.5\baselineskip}
\par Queste sezioni appaiono nell'ordine: \textbf{script}, \textbf{template}, \textbf{style}.

\paragraph*{Best practices}
\begin{itemize}
  \item \textbf{Riusabilità}: creare componenti riutilizzabili, evitando la duplicazione del codice;
  \item \textbf{Numero di componenti per file}: definire un solo componente per ogni file;
  \item \textbf{Funzioni di utilità}: raggruppare le funzioni di utilità in classi dedicate, evitando di includere le "utility" all'interno dei componenti che ne usufruiscono.
\end{itemize}

\paragraph*{Convenzioni sintattiche}
\begin{itemize}
  \item \textbf{Nome dei file}: \glossario{kebab-case}.
\end{itemize}.

\subsubsubsection{Stile di codifica: JavaScript/TypeScript}
\paragraph*{Struttura dei file}
\begin{itemize}
  \item \textbf{Import}: insieme di istruzioni per includere componenti, servizi o altre dipendenze nello script:
  \begin{itemize}
    \item Dipendenze esterne;
    \item Dipendenze interne.
  \end{itemize}
  \item \textbf{Variabili};
  \item \textbf{Funzioni}.
\end{itemize}

\paragraph*{Documentazione del codice}
\vspace{0.5\baselineskip}
\par La documentazione del codice sorgente deve essere conforme al formato \glossario{JSDoc}, compatibile sia con JavaScript che con TypeScript.

\paragraph*{Convenzioni sintattiche}
\begin{itemize}
  \item \textbf{Nomi dei file}: \glossario{kebab-case};
  \item \textbf{Nomi della classi}: \glossario{PascalCase};
  \item \textbf{Nomi delle funzioni}: \glossario{camelCase};
  \item \textbf{Nomi delle variabili}: camelCase;
  \item \textbf{Nomi delle costanti}: \glossario{UPPER\_SNAKE\_CASE};
  \item \textbf{Commenti}: utilizzare il simbolo \verb|//| per i commenti;
  \item \textbf{Blocchi JSDoc}: utilizzare il simbolo \verb|/** ... */| per i blocchi JSDoc.
\end{itemize}

\paragraph*{Formattazione del codice}
\par Il team aderisce alle convenzioni di formattazione stabilite da \glossario{Prettier}. Prettier può essere integrato in Visual Studio Code ed è compatibile con \glossario{ESLint}.

\subsubsubsection{Stile di codifica: CSS}
\par Il team aderisce alle regole di formattazione e stile per CSS definite al seguente indirizzo: \href{https://google.github.io/styleguide/htmlcssguide.html}{https://google.github.io/styleguide/htmlcssguide.html} (ultimo accesso: 2024-07-23), con alcune eccezioni (vedi \ref{sec:elementi-comuni-stile}). Il codice CSS è organizzato in tre sezioni:
\begin{itemize}
  \item \textbf{Layout}: disposizione generale degli elementi all'interno dell'applicazione; include classi per la gestione di griglie, allineamenti e margini;
  \item \textbf{Componente}: stile di un singolo componente; definisce le sue proprietà distintive e le sue caratteristiche visive;
  \item \textbf{Tema}: personalizzazione dell'aspetto estetico (tema chiaro/scuro).
\end{itemize}

\paragraph*{Convenzioni sintattiche}
\begin{itemize}
  \item \textbf{Nomi dei file}: \glossario{kebab-case};
  \item \textbf{Nome delle proprietà}: kebab-case;
  \item \textbf{Commenti}: utilizzare il simbolo \verb|/* ... */| per i commenti.
\end{itemize}

\subsubsubsection{Stile di codifica: HTML}
\par Il team aderisce alle regole di formattazione e stile per HTML definite al seguente indirizzo: \href{https://google.github.io/styleguide/htmlcssguide.html}{https://google.github.io/styleguide/htmlcssguide.html} (ultimo accesso: 2024-07-23), con alcune eccezioni (vedi \sezione{sec:elementi-comuni-stile}).

\paragraph*{Best practices}
\begin{itemize}
  \item \textbf{Attributi ARIA}: utilizzare attributi ARIA per migliorare l'accessibilità.
\end{itemize}

\paragraph*{Convenzioni sintattiche}
\begin{itemize}
  \item \textbf{Nomi dei file}: \glossario{kebab-case};
  \item \textbf{Nome delle classi}: kebab-case;
  \item \textbf{Nome degli id}: kebab-case;
  \item \textbf{Commenti}: utilizzare il simbolo \verb|<!-- ... -->| per i commenti.
\end{itemize}

\subsubsubsection{Strumenti}
\begin{itemize}
  \item \textbf{Git};
  \item \textbf{Visual Studio Code}.
\end{itemize}

\subsubsection{Testing del codice}\label{testing-codice}
\par Il testing ha lo scopo di verificare il comportamento di ciascuna unità software; viene condotto in parallelo all'attività di codifica. Di seguito sono riportate le tipologie di test che il team si impegna a eseguire come parte integrante del processo di verifica.

\subsubsubsection{Tipologie di test}
\par I test eseguiti nella fase antecedente la convalida sono suddivisi in tre categorie principali:
\begin{itemize}
  \item \textbf{Test di unità};
  \item \textbf{Test di integrazione};
  \item \textbf{Test di sistema}.
\end{itemize}

\subsubsubsection{Test di unità}
\par I test di unità sono una categoria di test del software che si focalizza sulla verifica del corretto funzionamento dei singoli moduli del codice. In genere, una “unità” è la più piccola porzione o segmento (una funzione, un metodo o una classe) testabile in modo autonomo e isolato all’interno del sistema. I test di unità vengono definiti durante la progettazione di dettaglio con l’obiettivo di individuare difetti o malfunzionamenti. La maggior parte dei difetti rilevati tramite analisi dinamica (circa 2/3 del totale) deriva dall’esecuzione dei test di unità. A ogni unità software può essere associata una suite o batteria di test. Nelle fasi iniziali dello sviluppo, il team potrebbe riscontrare delle difficoltà nel testare singole unità, poiché alcuni moduli potrebbero essere incompleti o non disponibili. Pertanto, sono stati introdotti degli strumenti a supporto dell'analisi dinamica:
\begin{itemize}
  \item \textbf{Stub} (o metodo stub): simula una funzionalità del sistema (non ancora codificata e non oggetto di test) come parte del processo di testing di un’unità software;
  \item \textbf{Driver}: componente di supporto che controlla l'esecuzione dei test e aiuta a configurare l'ambiente. Sostituisce il modulo chiamante di una funzionalità sotto test;
  \item \textbf{Mock object}: simula il comportamento di oggetti reali in condizioni controllate; può fornire una risposta preimpostata.
\end{itemize}

\vspace{0.5\baselineskip}
\par I test di unità si dividono in due categorie:
\begin{itemize}
  \item \textbf{Test funzionali} (black-box): utilizzano dati di ingresso capaci di provocare l’esito atteso. L'obiettivo è accertare che, fornito un determinato input e definita un'aspettativa di output, l'esecuzione di una funzionalità generi il comportamento previsto. I test black-box contribuiscono a misurare quante specifiche e requisiti funzionali sono soddisfatti dal prodotto software. Non possono valutare la correttezza e la completezza della logica interna dell’unità, pertanto devono essere integrati con test strutturali;
  \item \textbf{Test strutturali} (white-box): verificano la struttura interna dell’unità software, esaminando i cammini di esecuzione all’interno dell’unità stessa. L'obiettivo è valutare il funzionamento interno del software, analizzando la logica, i flussi di controllo e la copertura del codice.
\end{itemize}

\vspace{0.5\baselineskip}
\par Per l'esecuzione dei test funzionali e strutturali, il team utilizza strumenti di automazione appropriati e coerenti con le tecnologie e i linguaggi di programmazione selezionati. Ciascuna unità software deve rispettare le specifiche stabilite durante la progettazione di dettaglio.

\subsubsubsection{Test di integrazione}
\par I test di integrazione vengono definiti durante la progettazione architetturale e si basano sui componenti in essa specificati. L’integrazione può essere di tipo incrementale, aumentando il valore funzionale a ogni passo. Questo approccio prevede che i componenti vengano integrati in insiemi già verificati, agevolando la rimozione di eventuali difetti o malfunzionamenti in seguito a cambiamenti. Lo scopo dei test di integrazione è rilevare difetti di design o una scarsa qualità dei test di unità, garantendo che i dati scambiati attraverso ciascuna interfaccia siano conformi alle specifiche. Vi sono due modalità di integrazione incrementale:
\begin{itemize}
  \item \textbf{Bottom-up}: l’integrazione avviene partendo dai moduli più interni al sistema, che sono meno visibili a livello utente e possiedono un minor numero di dipendenze funzionali. Questa strategia richiede l’uso di pochi stub, ma può rallentare la fornitura di servizi di "alto livello". Nell’approccio bottom-up, i driver vengono utilizzati per testare i moduli nei layer inferiori, qualora i moduli nei layer superiori siano ancora incompleti;
  \item \textbf{Top-down}: l’integrazione avviene partendo dai moduli più esterni, che possiedono un maggior numero di dipendenze funzionali e sono quindi di maggior interesse per l’utente. Questa strategia richiede l'uso di molti stub per simulare i moduli mancanti, consentendo di integrare fin da subito le funzionalità di "alto livello".
\end{itemize}

\subsubsubsection{Test di sistema}
\par I test di sistema verificano il comportamento dinamico dell'intero sistema rispetto ai requisiti specificati nel documento di \AnalisiDeiRequisiti. Questa fase della verifica ha inizio al completamento dei test di unità e di integrazione, ed è il precursore del collaudo. I test di sistema sono inerentemente funzionali (black-box) e, pertanto, non dovrebbero richiedere conoscenza della logica interna del software. L’ambiente di esecuzione dovrebbe essere simile a quello di produzione, al fine di replicare le condizioni reali di utilizzo del sistema. I test di sistema coprono il software nel suo complesso, verificando tutte le funzionalità integrate e interconnesse.

\subsubsubsection{Test: Notazione}
\par I test vengono identificati in modo univoco secondo questa notazione: 
\par \quad \textbf{T.[Tipo].[Codice]}
\par dove: 
\begin{itemize}
    \item \textbf{T}: indica la parola "test";
    \item \textbf{Tipo}: indica la tipologia di test: 
        \begin{itemize}
            \item \textbf{U} (Unità); 
            \item \textbf{I} (Integrazione);
            \item \textbf{S} (Sistema);
            \item \textbf{A} (Accettazione).
        \end{itemize}
    \item \textbf{Codice}: è un numero progressivo che identifica in modo univoco i test per ogni tipologia.
\end{itemize}

\subsubsubsection{Test: Stato}
\par Ogni test è corredato da un flag che segnala il suo stato, aggiornato in base alla versione corrente del prodotto software. Il flag può assumere i seguenti valori:
\begin{itemize}
    \item \textbf{N-D}: test non definito (o non disponibile);
    \item \textbf{S}: test eseguito con successo;
    \item \textbf{F}: test fallito.
\end{itemize}

\subsubsubsection{Strumenti}
\begin{itemize}
  \item \textbf{GitHub};
  \item \textbf{Visual Studio Code}.
\end{itemize}

\subsubsection{Integrazione software}
\par Il team adotta la pratica di \glossario{continuous integration} (CI), che consiste nell’allineamento frequente delle modifiche con l’ambiente condiviso. La cronologia dello sviluppo viene mantenuta in un ramo chiamato develop, mentre nel branch principale è registrata la cronologia dei rilasci. A partire dal branch develop, ciascun membro del team può creare un ramo di feature, orientato allo sviluppo di nuove funzionalità, alla definizione di test o alla correzione di bug. Una volta completate le modifiche, viene aperta una pull request per richiedere la verifica del codice. Come definito nella \sezione{sec:pull_request}, ciascuna pull request deve essere opportunamente etichettata.

\par Il processo di integrazione continua implica l’automazione delle fasi di build e test. Pertanto, quando viene aperta una pull request, GitHub Actions avvia un workflow che esegue i test di unità, di integrazione e di sistema definiti nel \PianoDiQualifica. L’obiettivo del team è garantire che il ramo di sviluppo rimanga stabile e il più possibile privo di difetti.

\subsubsubsection{Strumenti}
\begin{itemize}
  \item \textbf{GitHub};
  \item \textbf{GitHub Actions}.
\end{itemize}