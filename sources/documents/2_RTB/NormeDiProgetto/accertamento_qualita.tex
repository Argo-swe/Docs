\section{Accertamento di qualità}

\subsection{Scopo}
\par Il processo di certificazione della qualità mira a garantire che i prodotti software e i processi coinvolti nel ciclo di vita del progetto siano conformi ai requisiti e alle aspettative. L'obiettivo primario dell'accertamento della qualità è garantire che il lavoro svolto rispetti gli standard e le linee guida. È essenziale stabilire internamente parametri misurabili per valutare il grado di aderenza alle \glossario{best practices} dell'ingegneria del software, al fine di condurre un controllo e un miglioramento continuo dei processi. L'assicurazione della qualità può avvalersi dei risultati di altri processi di supporto (es.: verifica e validazione).

\subsection{Garanzia della qualità}

\par Per garantire il raggiungimento e il mantenimento degli standard di qualità prefissati, il team applica il ciclo di \glossario{PDCA}, un metodo di gestione iterativo che contribuisce al controllo e al miglioramento continuo dei processi e dei prodotti all'interno di un'organizzazione; consente inoltre di adattarsi a cambiamenti nel lungo periodo. Il PDCA, noto anche come ciclo di Deming, si divide in 4 fasi interconnesse:
\begin{itemize}
    \item \textbf{Plan} (Pianificare): in questa fase vengono definiti gli obiettivi di qualità da conseguire, nonché le strategie e le azioni necessarie per raggiungerli e misurarli. È importante identificare chiaramente le risorse disponibili, i tempi e le modalità di implementazione del piano. La pianificazione aiuta a stabilire obiettivi e processi necessari per fornire i risultati desiderati;
    \item \textbf{Do} (Fare): una volta stabilito, il piano viene messo in pratica. Questa fase coinvolge l'attuazione delle azioni pianificate, l'allocazione delle risorse e l'esecuzione delle attività secondo le specifiche stabilite al passaggio precedente. Inoltre, vengono raccolti dati per la generazione di grafici e analisi;
    \item \textbf{Check} (Verificare): in questa fase si valutano i risultati ottenuti confrontandoli con gli obiettivi pianificati e gli standard di qualità prefissati. Attraverso un insieme di indicatori, il team può determinare la qualità dei processi e verificare se i risultati prodotti sono in linea con le attese. I grafici dei dati possono agevolare il processo di test, in quanto è possibile osservare le tendenze di più cicli di PDCA;
    \item \textbf{Act} (Agire): sulla base dei risultati della fase di verifica, vengono identificate eventuali discrepanze, non conformità, opportunità di miglioramento o inefficienze. Durante questa fase, si attuano azioni correttive per migliorare la qualità dei processi e del prodotto.
\end{itemize}

\subsection{Notazione delle metriche}
\par Le metriche vengono identificate in modo univoco secondo questa notazione: 
\par \textbf{M.[Tipo].[Codice]}
dove: 
\begin{itemize}
    \item \textbf{M}: indica la parola "metrica";
    \item \textbf{Tipo}: indica il tipo di qualità: 
        \begin{itemize}
            \item \textbf{PC}: qualità di processo; 
            \item \textbf{PD}: qualità di prodotto.
        \end{itemize}
    \item \textbf{Codice}: è un numero progressivo che identifica in modo univoco le metriche per ogni tipologia.
\end{itemize}

\subsection{Didascalia}
Le metriche sono descritte dai seguenti campi:
\begin{itemize}
    \item \textbf{Notazione}: segue le specifiche sopra elencate;
    \item \textbf{Nome}: nome della metrica;
    \item \textbf{Descrizione}: descrizione della metrica;
    \item \textbf{Caratteristiche}: una o più caratteristiche definite dallo standard di riferimento. Questo campo è disponibile per le metriche di prodotto;
    \item \textbf{Motivo}: ragione per cui la metrica viene misurata;
    \item \textbf{Misurazione}: formula e/o strumenti con cui calcolare la metrica.
\end{itemize}

\subsection{Standard di riferimento per la qualità di prodotto}
\par Per l'identificazione e la classificazione delle metriche, il team segue lo standard ISO/IEC 9126, che suddivide la qualità in: esterna (comportamento del software durante la sua esecuzione), interna (si applica al software non eseguibile) e in uso. Il modello di qualità è suddiviso in sei caratteristiche generali:
\begin{itemize}
    \item \textbf{Funzionalità}: capacità del software di fornire le funzioni necessarie per soddisfare esigenze specifiche operando in determinate condizioni;
    \item \textbf{Affidabilità}: capacità del software di mantenere un determinato livello di prestazioni quando viene usato in condizioni specifiche per un certo periodo di tempo;
    \item \textbf{Usabilità}: capacità del software di essere compreso dall'utente. L'usabilità comprende un insieme di attributi che incidono sullo sforzo necessario per l'uso del prodotto;
    \item \textbf{Efficienza}: capacità del software di fornire prestazioni adeguate in relazione alla quantità di risorse usate;
    \item \textbf{Manutenibilità}: capacità del software di essere modificato per includere correzioni, miglioramenti o adattamenti;
    \item \textbf{Portabilità}: capacità del software di essere trasferito tra ambienti di lavoro diversi, che possono variare sia per hardware che per sistema operativo.
\end{itemize}

\vspace{0.5\baselineskip}
\par La qualità in uso rappresenta il punto di vista dell'utente sul software. Nel contesto della qualità in uso, le metriche misurano le seguenti caratteristiche:
\begin{itemize}
    \item \textbf{Efficacia}: capacità del software di consentire agli utenti di raggiungere gli obiettivi desiderati con accuratezza e completezza;
    \item \textbf{Produttività}: capacità del software di permettere agli utenti di impiegare una quantità di risorse appropriate in relazione all'efficacia ottenuta in un determinato contesto d'uso;
    \item \textbf{Soddisfazione}: capacità del software di soddisfare gli utenti che ne usufruiscono;
    \item \textbf{Sicurezza}: capacità del software di raggiungere livelli accettabili di rischio, indipendentemente dalla natura del rischio.
\end{itemize}





\subsection{Elenco delle metriche}


\subsubsubsection{Metriche di processo}

\subsubsubsection{Percentuale di metriche soddisfatte}
\begin{itemize}
    \item \textbf{Notazione specifica}: M.PC.1;
    \item \textbf{Nome}: Percentuale di metriche soddisfatte;
    \item \textbf{Descrizione}: Questa metrica misura la percentuale di metriche che soddisfano i criteri di accettazione rispetto al totale delle metriche. I valori tollerati e ambiti sono specificati nel \VersionePQ;
    \item \textbf{Motivo}: Valutare il grado di conformità dei processi e del prodotto agli standard di qualità;
    \item \textbf{Misurazione}:
    \[
        \text{Percentuale di metriche soddisfatte (\%)} =\frac{M_{s}}{M_{t}} \times 100 
    \]
    dove:
    \begin{itemize}
        \item $M_{s}$: Numero di metriche soddisfatte;
        \item $M_{t}$: Numero totale di metriche.
    \end{itemize}
\end{itemize}

\subsubsubsection{AC (Actual cost)}
\begin{itemize}
    \item \textbf{Notazione specifica}: M.PC.2;
    \item \textbf{Nome}: AC (Actual cost);
    \item \textbf{Descrizione}: Questa metrica misura il costo effettivo sostenuto alla data corrente;
    \item \textbf{Motivo}: Controllare i costi e calcolare la spesa effettiva in funzione dell'EAC;
    \item \textbf{Misurazione}: Costo (in €) speso per il progetto.
\end{itemize}

\paragraph*{EV (Earned Value)}
\begin{itemize}
    \item \textbf{Notazione specifica}: M.PC.3;
    \item \textbf{Nome}: EV (Earned Value);
    \item \textbf{Descrizione}: Questa metrica misura il valore (in €) delle attività realizzate alla data corrente;
    \item \textbf{Motivo}: Misurare il progresso e calcolare il valore prodotto dal progetto in funzione dell'EAC;
    \item \textbf{Misurazione}:
    \[
        \text{EV} = \textit{BAC} \times (\% {\textit{Lavoro completato}})
    \]
    dove:
    \begin{itemize}
        \item BAC (Budget at Completion): budget previsto per la realizzazione del progetto (riportato nel \PianoDiProgetto).
    \end{itemize}
\end{itemize}

\paragraph*{PV (Planned Value)}
\begin{itemize}
    \item \textbf{Notazione specifica}: M.PC.4;
    \item \textbf{Nome}: PV (Planned Value);
    \item \textbf{Descrizione}: Questa metrica misura il costo pianificato (in €) alla data corrente;
    \item \textbf{Motivo}: Controllare i costi e monitorare il progresso;
    \item \textbf{Misurazione}:
    \[
        \text{PV} = \textit{BAC} \times (\% {\textit{Lavoro pianificato}})
    \]
    dove:
    \begin{itemize}
        \item BAC (Budget at Completion): budget previsto per la realizzazione del progetto (riportato nel \PianoDiProgetto).
    \end{itemize}
\end{itemize}

\paragraph*{EAC (Estimated at Completion)}
\begin{itemize}
    \item \textbf{Notazione specifica}: M.PC.5;
    \item \textbf{Nome}: EAC (Estimated at Completion);
    \item \textbf{Descrizione}: Questa metrica misura il budget stimato per la realizzazione del progetto (costo sostenuto + stima costo ancora da sostenere);
    \item \textbf{Motivo}: Calcolare il BAC rivisto allo stato corrente del progetto;
    \item \textbf{Misurazione}:
    \[
        \text{EAC} = \textit{AC} + (\textit{BAC} - \textit{EV})
    \]
    dove:
    \begin{itemize}
        \item BAC (Budget at Completion): budget previsto per la realizzazione del progetto (riportato nel \PianoDiProgetto).
    \end{itemize}
\end{itemize}

\paragraph*{Variazione del budget tra preventivo e consuntivo}
\begin{itemize}
    \item \textbf{Notazione specifica}: M.PC.6;
    \item \textbf{Nome}: Variazione del budget tra preventivo e consuntivo;
    \item \textbf{Descrizione}: Questa metrica misura la variazione tra il costo pianificato e il costo effettivo di un progetto alla data corrente;
    \item \textbf{Motivo}: Valutare con che velocità il team sta spendendo il proprio budget rispetto al preventivo. Identificare eventuali sforamenti che potrebbero richiedere interventi correttivi;
    \item \textbf{Misurazione}:
    \[
        \text{Variazione del budget (\%)} = \frac{C_p - C_a}{C_p} \times 100
    \]
    dove:
    \begin{itemize}
        \item $C_{p}$: Costo pianificato;
        \item $C_{a}$: Costo attuale (effettivo).
    \end{itemize}
\end{itemize}

\paragraph*{Variazione del piano tra preventivo e consuntivo}
\begin{itemize}
    \item \textbf{Notazione specifica}: M.PC.7;
    \item \textbf{Nome}: Variazione del piano tra preventivo e consuntivo;
    \item \textbf{Descrizione}: Questa metrica misura la variazione tra il numero di task pianificati e quelli completati entro un certo periodo di tempo. La pianificazione dei task è riportata nel \PianoDiProgetto;
    \item \textbf{Motivo}: Valutare la capacità del team di rispettare le scadenze ed evitare ritardi;
    \item \textbf{Misurazione}:
    \[
        \text{Variazione del piano} = \frac{T_p - T_c}{T_p} \times 100
    \]
    dove:
    \begin{itemize}
        \item $T_{p}$: Numero di task pianificati;
        \item $T_{c}$: Numero di task completati.
    \end{itemize}
\end{itemize}

\subsubsubsection{Efficienza temporale}
\begin{itemize}
    \item \textbf{Notazione specifica}: M.PC.8;
    \item \textbf{Nome}: Efficienza temporale;
    \item \textbf{Descrizione}: Questa metrica misura il rapporto tra il tempo totale a disposizione (ore di orologio) e il tempo speso in attività produttive (ore produttive);
    \item \textbf{Motivo}: Valutare la capacità del team di utilizzare il tempo in modo efficiente per raggiungere obiettivi e completare attività pianificate. Un'efficienza temporale più alta indica una maggiore produttività e un uso ottimale del tempo a disposizione;
    \item \textbf{Misurazione}:
    \[
        \text{Efficienza temporale (\%)} = \frac{O_r}{O_p} \times 100
    \]
    dove:
    \begin{itemize}
        \item $O_{r}$: Ore di orologio;
        \item $O_{p}$: Ore produttive.
    \end{itemize}
\end{itemize}

\paragraph*{Frequenza di merge delle pull request}
\begin{itemize}
    \item \textbf{Notazione specifica}: M.PC.9;
    \item \textbf{Nome}: Frequenza di merge delle pull request;
    \item \textbf{Descrizione}: Questa metrica misura la frequenza con cui le pull request vengono approvate e unite al ramo base;
    \item \textbf{Motivo}: Valutare la capacità del team di integrare in modo efficiente le nuove funzionalità, modifiche e correzioni nell’ambiente condiviso, favorendo un flusso di lavoro continuo e una collaborazione  stretta tra i membri. La chiusura delle pull request include il processo di revisione del codice, l’innesto delle modifiche richieste dal \Verificatore\ e l’approvazione finale;
    \item \textbf{Misurazione}:
    \[
        \text{Frequenza di merge delle pull request} = \frac{N_{pr}}{T}
    \]
    dove:
    \begin{itemize}
        \item $N_{pr}$: Numero totale di pull request approvate e unite al ramo base;
        \item $T$: Periodo di tempo considerato (in giorni).
    \end{itemize}
\end{itemize}

\subsubsubsection{Indice di stabilità dei requisiti}
\begin{itemize}
    \item \textbf{Notazione specifica}: M.PC.10;
    \item \textbf{Nome}: Indice di stabilità dei requisiti;
    \item \textbf{Descrizione}: Questa metrica misura la variazione dei requisiti durante il ciclo di vita del software;
    \item \textbf{Motivo}: Valutare l'impatto delle modifiche ai requisiti e la solidità dell'analisi condotta nel documento di \AnalisiDeiRequisiti;
    \item \textbf{Misurazione}:
    \[
        \text{Indice di stabilità dei requisiti} = (1 - \frac{R_a + R_m + R_c}{N_r}) \times 100
    \]
    dove:
    \begin{itemize}
      \item $R_{a}$: Requisiti aggiunti;
      \item $R_{m}$: Requisiti modificati;
      \item $R_{c}$: Requisiti cancellati;
      \item $N_{r}$: Numero totale di requisiti;
    \end{itemize}
\end{itemize}


\subsubsubsection{Rischi inattesi}
\begin{itemize}
    \item \textbf{Notazione specifica}: M.PC.11;
    \item \textbf{Nome}: Rischi inattesi;
    \item \textbf{Descrizione}: Questa metrica misura il numero di rischi non previsti in un determinato periodo;
    \item \textbf{Motivo}: Valutare l'accuratezza della fase di identificazione e analisi dei rischi;
    \item \textbf{Misurazione}:
    \[
        \text{Rischi inattesi} = \textit{N° Rischi inattesi}
    \]
\end{itemize}

\subsubsubsection{Efficienza delle contromisure nei rischi}
\begin{itemize}
    \item \textbf{Notazione specifica}: M.PC.12;
    \item \textbf{Nome}: Efficienza delle contromisure nei rischi;
    \item \textbf{Descrizione}: Indicatore utile per valutare l'efficacia delle azioni correttive intraprese per mitigare i rischi;
    \item \textbf{Caratteristica di riferimento}: Gestione dei rischi;
    \item \textbf{Motivo}: Questa metrica permette di misurare quanto le contromisure adottate siano state in grado di ridurre o eliminare i rischi identificati e quante si sono rivelate incomplete per valutare se un rischio ha bisogno di modifiche correttive;
    \item \textbf{Misurazione}:
    \[
        \text{Efficienza delle contromisure nei rischi} = \frac{R_s}{R_t} \times 100
    \]
    dove:
    \begin{itemize}
        \item $R_{s}$: Rischi gestiti con successo;
        \item $N_{r}$: Numero totale di rischi emersi.
    \end{itemize}
\end{itemize}

\subsubsection{Metriche di prodotto e di qualità del software:}


\subsubsubsection{Requisiti obbligatori soddisfatti}
\begin{itemize}
    \item \textbf{Notazione specifica}: M.PD.1;
    \item \textbf{Nome}: Requisiti obbligatori soddisfatti;
    \item \textbf{Descrizione}: Questa metrica misura la percentuale di requisiti obbligatori soddisfatti rispetto al totale. I requisiti obbligatori sono definiti nel documento di \VersioneAR;
    \item \textbf{Caratteristiche}: Funzionalità;
    \item \textbf{Motivo}: Valutare il grado di adempimento dei requisiti ritenuti essenziali per il funzionamento del sistema;
    \item \textbf{Misurazione}:
    \[
    \text{Requisiti obbligatori soddisfatti (\%)} = \frac{R_{o}}{T_{o}} \times 100
    \]
    dove:
    \begin{itemize}
        \item $R_{o}$: Requisiti obbligatori soddisfatti;
        \item $T_{o}$: Numero totale dei requisiti obbligatori.
    \end{itemize}
\end{itemize}

\paragraph*{Requisiti desiderabili soddisfatti}
\begin{itemize}
    \item \textbf{Notazione specifica}: M.PD.2;
    \item \textbf{Nome}: Requisiti desiderabili soddisfatti;
    \item \textbf{Descrizione}: Questa metrica misura la percentuale di requisiti desiderabili soddisfatti rispetto al totale. I requisiti desiderabili sono definiti nel documento di \AnalisiDeiRequisiti;
    \item \textbf{Caratteristiche}: Funzionalità;
    \item \textbf{Motivo}: Valutare il grado di adempimento dei requisiti desiderabili, fornendo una misura del livello di soddisfazione del cliente;
    \item \textbf{Misurazione}:
    \[
    \text{Requisiti desiderabili soddisfatti (\%)} = \frac{R_{d}}{T_{d}} \times 100
    \]
    dove:
    \begin{itemize}
        \item $R_{d}$: Requisiti desiderabili soddisfatti;
        \item $T_{d}$: Numero totale dei requisiti desiderabili.
    \end{itemize}
\end{itemize}
\paragraph*{Requisiti opzionali soddisfatti}
\begin{itemize}
    \item \textbf{Notazione specifica}: M.PD.3;
    \item \textbf{Nome}: Requisiti opzionali soddisfatti;
    \item \textbf{Descrizione}: Questa metrica misura la percentuale di requisiti opzionali soddisfatti rispetto al totale. I requisiti opzionali sono definiti nel documento di \AnalisiDeiRequisiti;
    \item \textbf{Caratteristiche}: Funzionalità;
    \item \textbf{Motivo}: Valutare il grado di adempimento dei requisiti opzionali, fornendo una misura del livello di soddisfazione del cliente;
    \item \textbf{Misurazione}:
    \[
    \text{Requisiti opzionali soddisfatti (\%)} = \frac{R_{op}}{T_{op}} \times 100
    \]
    dove:
    \begin{itemize}
        \item $R_{op}$: Requisiti opzionali soddisfatti;
        \item $T_{op}$: Numero totale dei requisiti opzionali.
    \end{itemize}
\end{itemize}
\subsubsubsection{Indice Gulpease}
\begin{itemize}
    \item \textbf{Notazione specifica}: M.PD.4;
    \item \textbf{Nome}: Indice Gulpease;
    \item \textbf{Descrizione}: Questa metrica misura l'indice di leggibilità di un testo in lingua italiana. I valori sono compresi tra 0 e 100, dove 100 indica una leggibilità più alta mentre 0 una leggibilità più bassa. I punteggi si dividono in:
    \begin{itemize}
        \item punteggi inferiori a 80: difficili da leggere per chi possiede la licenza elementare;
        \item punteggi inferiori a 60: difficili da leggere per chi possiede la licenza media;
        \item punteggi inferiori a 40: difficili da leggere per chi possiede un diploma superiore.
    \end{itemize}
    \item \textbf{Caratteristica di riferimento}: Usabilità, comprensibilità;
    \item \textbf{Motivo}: Garantire che i documenti di progetto siano comprensbili per la maggior parte dei lettori;
    \item \textbf{Misurazione}:
    \[
        \text{Indice Gulpease} = 89 + \frac{{300 \cdot {{N_f}} - 10 \cdot {{N_l}}}}{{{{N_p}}}}
    \]
    dove:
    \begin{itemize}
        \item $N_{f}$: Numero totale di frasi;
        \item $N_{l}$: Numero totale di lettere;
        \item $N_{p}$: Numero totale di parole.
    \end{itemize}
\end{itemize}

\subsubsubsection{Completezza descrittiva}
\begin{itemize}
    \item \textbf{Notazione specifica}: M.PD.5;
    \item \textbf{Nome}: Completezza descrittiva;
    \item \textbf{Descrizione}: Questa metrica misura il grado di completezza delle funzionalità descritte nella documentazione del prodotto;
    \item \textbf{Caratteristiche}: Usabilità, comprensibilità;
    \item \textbf{Motivo}: Garantire che gli utenti possano comprendere:
    \begin{itemize}
        \item Se il prodotto è adeguato alle loro esigenze;
        \item Come utilizzare il prodotto per determinati scopi.
    \end{itemize}
    \item \textbf{Misurazione}:
    \[
    \text{Completezza descrittiva} = \frac{F_{d}}{F_{t}} \times 100
    \]
    dove:
    \begin{itemize}
        \item $F_{d}$: Numero di funzioni descritte nella documentazione;
        \item $F_{t}$: Numero totale di funzioni previste.
    \end{itemize}
\end{itemize}

\paragraph*{Browser supportati}
\begin{itemize}
    \item \textbf{Notazione specifica}: M.PD.6;
    \item \textbf{Nome}: Browser supportati;
    \item \textbf{Descrizione}: Questa metrica misura la percentuale di browser sui quali il prodotto risulta fruibile;
    \item \textbf{Caratteristiche}: Funzionalità, interoperabilità, compatibilità;
    \item \textbf{Motivo}: Valutare la capacità del software di interagire con i browser specificati;
    \item \textbf{Misurazione}:
    \[
    \text{Browser supportati} = \frac{N_{bs}}{N_{bt}} \times 100
    \]
    dove:
    \begin{itemize}
        \item $N_{bs}$: Numero di browser supportati;
        \item $N_{bt}$: Numero totale di browser.
    \end{itemize}
\end{itemize}

\subsubsubsection{Profondità (click necessari per reperire un'informazione)}
\begin{itemize}
    \item \textbf{Notazione specifica}: M.PD.7;
    \item \textbf{Nome}: Profondità (click necessari per reperire un'informazione);
    \item \textbf{Descrizione}: Questa metrica misura il numero di click necessari per raggiungere un obiettivo;
    \item \textbf{Caratteristica di riferimento}: Usabilità, accessibilità;
    \item \textbf{Motivo}: Valutare la facilità di navigazione dell'applicazione web;
    \item \textbf{Misurazione}: Calcolare il numero di click necessari per reperire un'informazione seguendo il percorso più ampio.
\end{itemize}

\subsubsubsection{Ampiezza (opzioni nel menu di navigazione principale)}
\begin{itemize}
    \item \textbf{Notazione specifica}: M.PD.8;
    \item \textbf{Nome}: Ampiezza (opzioni nel menu di navigazione principale);
    \item \textbf{Descrizione}: Questa metrica misura il numero di opzioni presenti nel menu di navigazione principale;
    \item \textbf{Caratteristica di riferimento}: Usabilità, accessibilità;
    \item \textbf{Motivo}: Valutare la facilità di navigazione dell'applicazione web;
    \item \textbf{Misurazione}: Calcolare il numero di opzioni nel menu di navigazione principale.
\end{itemize}

\subsubsubsection{Tempo di apprendimento}
\begin{itemize}
    \item \textbf{Notazione specifica}: M.PD.9;
    \item \textbf{Nome}: Tempo di apprendimento;
    \item \textbf{Descrizione}: Questa metrica misura il tempo necessario a un utente per comprendere l'utilizzo corretto di una funzione;
    \item \textbf{Caratteristiche}: Usabilità, accessibilità, comprensibilità, apprendibilità;
    \item \textbf{Motivo}: Valutare il design dell'interfaccia e l'esperienza utente;
    \item \textbf{Misurazione}: Test dell'applicazione da parte di un campione eterogeneo di utenti.
\end{itemize}
\subsubsubsection{Tempo di risposta}
\begin{itemize}
    \item \textbf{Notazione specifica}: M.PD.10;
    \item \textbf{Nome}: Tempo di risposta;
    \item \textbf{Descrizione}: Questa metrica misura l'efficienza con cui l'applicazione completa una transazione (task);
    \item \textbf{Caratteristica di riferimento}: Efficienza;
    \item \textbf{Motivo}: Misurare e migliorare il tempo medio in cui il sistema risponde a una richiesta dell'utente;
    \item \textbf{Misurazione}: Tempo che intercorre tra l’immissione di un comando e la generazione della risposta da parte del sistema.
\end{itemize}

\subsubsubsection{Code coverage}
\begin{itemize}
    \item \textbf{Notazione specifica}: M.PD.11;
    \item \textbf{Nome}: Code coverage;
    \item \textbf{Descrizione}: Questa metrica misura la copertura dei test, ossia la percentuale di codice sorgente che è stata eseguita durante l'esecuzione dei test automatici.
    \item \textbf{Caratteristiche}: Manutenibilità, testabilità, affidabilità, maturità;
    \item \textbf{Motivo}: Valutare la copertura dei test automatici, al fine di garantire la testabilità del prodotto;
    \item \textbf{Misurazione}: 
    \[
    \text{Code coverage} = \frac{L_{t}}{N_{l}} \times 100
    \]
    dove:
    \begin{itemize}
        \item $L_{t}$: Linee di codice testate;
        \item $N_{l}$: Numero totale di linee di codice.
    \end{itemize}
\end{itemize}

\subsubsubsection{Adeguatezza delle funzioni sviluppate}
\begin{itemize}
    \item \textbf{Notazione specifica}: M.PD.12;
    \item \textbf{Nome}: Adeguatezza delle funzioni sviluppat;
    \item \textbf{Descrizione}: Questa metrica misura il livello di adeguatezza delle funzioni sviluppate rispetto ai requisiti;
    \item \textbf{Caratteristiche}: Funzionalità, adeguatezza;
    \item \textbf{Motivo}: Valutare la conformità ai requisiti funzionali.
    \item \textbf{Misurazione}: 
    \[
    \text{Code coverage} = \frac{F_{a}}{N_{f}} \times 100
    \]
    dove:
    \begin{itemize}
        \item $F_{a}$: Funzioni ritenute adeguate allo svolgimento del task associato;
        \item $N_{f}$: Numero totale di funzioni sviluppate.
    \end{itemize}
\end{itemize}

\subsubsubsection{Accuratezza della risposta}
\begin{itemize}
    \item \textbf{Notazione specifica}: M.PD.13;
    \item \textbf{Nome}: Accuratezza della risposta;
    \item \textbf{Descrizione}: Questa metrica misura la correttezza dei risultati forniti dal sistema rispetto alle esigenze specificate;
    \item \textbf{Caratteristica di riferimento}: Funzionalità, adeguatezza;
    \item \textbf{Motivo}: Garantire che il sistema fornisca risposte in linea con le aspettative dell'utente;
    \item \textbf{Misurazione}:
    \[
    \text{Accuratezza della risposta} = \frac{R_{c}}{N_{t}} \times 100
    \]
    Dove:
    \begin{itemize}
        \item $R_{c}$: Numero di risposte corrette;
        \item $N_{t}$: Numero di tentativi.
    \end{itemize}
    \par Il team eseguirà i test definiti con la \glossario{Proponente} durante i verbali esterni.
 \end{itemize}

\subsubsubsection{Linee medie di codice per metodo}
\begin{itemize}
    \item \textbf{Notazione specifica}: M.PD.14;
    \item \textbf{Nome}: Linee medie di codice per metodo;
    \item \textbf{Descrizione}: Questa metrica misura che indica la lunghezza media, in termini di linee di codice, dei metodi o funzioni all'interno del codice sorgente;
    \item \textbf{Caratteristiche}: Manutenibilità, testabilità, modificabilità, comprensibilità;
    \item \textbf{Motivo}: Verificare che non vi sia mancanza di modularità e chiarezza nel codice. Funzioni e metodi più corti sono generalmente preferibili perché risultano più semplici da comprendere, testare e mantenere; 
    \item \textbf{Misurazione}: Il calcolo viene effettuato con strumenti di analisi statica.
\end{itemize}

\paragraph*{Complessità ciclomatica}
\begin{itemize}
    \item \textbf{Notazione specifica}: M.PD.15;
    \item \textbf{Nome}: Complessità ciclomatica;
    \item \textbf{Descrizione}: Questa metrica misura il numero di cammini linearmente indipendenti attraverso il grafo di controllo di flusso;
    \item \textbf{Caratteristiche}: Manutenibilità, comprensibilità;
    \item \textbf{Motivo}: Valutare e limitare la complessità di un programma;
    \item \textbf{Misurazione}: 
    \[
    \text{Complessità ciclomatica} = \textit{E} - \textit{N} + \textit{2P}
    \]
    dove:
    \begin{itemize}
        \item $E$: Numero di archi del grafo;
        \item $N$: Numero di nodi del grafo;
        \item $P$: Numero di componenti connesse.
    \end{itemize}
\end{itemize}

\paragraph*{Indice di manutenibilità}
\begin{itemize}
    \item \textbf{Notazione specifica}: M.PD.16;
    \item \textbf{Nome}: Indice di manutenibilità;
    \item \textbf{Descrizione}: Questa metrica misura la capacità del software di essere modificato, corretto o adattato;
    \item \textbf{Caratteristiche}: Manutenibilità;
    \item \textbf{Motivo}: Fornire una misura quantitativa della manutenibilità del software;
    \item \textbf{Misurazione}: Il calcolo viene effettuato con SonarCloud.
\end{itemize}

\paragraph*{Indice di manutenibilità}
\begin{itemize}
    \item \textbf{Notazione specifica}: M.PD.17;
    \item \textbf{Nome}: Indice di manutenibilità;
    \item \textbf{Descrizione}: Questa metrica misura la capacità del software di essere modificato, corretto o adattato;
    \item \textbf{Caratteristiche}: Manutenibilità;
    \item \textbf{Motivo}: Fornire una misura quantitativa della manutenibilità del software;
    \item \textbf{Misurazione}: Il calcolo viene effettuato con strumenti di analisi statica.
\end{itemize}

\subsubsubsection{Percentuale di test superati}
\begin{itemize}
    \item \textbf{Notazione specifica}: M.PD.18;
    \item \textbf{Nome}: Percentuale di test superati;
    \item \textbf{Descrizione}: Questa metrica misura la percentuale di test eseguiti con successo;
    \item \textbf{Caratteristiche}: Funzionalità, affidabilità, maturità;
    \item \textbf{Motivo}: Garantire la piena copertura dei requisiti funzionali e di qualità;
    \item \textbf{Misurazione}:
    \[
    \text{Test superati (\%)} = \frac{N_{ts}}{N_{te}} \times 100
    \]
    dove:
    \begin{itemize}
        \item $N_{ts}$: Numero di test eseguiti con successo;
        \item $N_{tot}$: Numero di test eseguiti.
    \end{itemize}
\end{itemize}

\paragraph*{Tolleranza agli errori}
\begin{itemize}
    \item \textbf{Notazione specifica}: M.PD.19;
    \item \textbf{Nome}: Tolleranza agli errori;
    \item \textbf{Descrizione}: Questa metrica misura la percentuale di errori che il prodotto è in grado di gestire;
    \item \textbf{Caratteristiche}: Affidabilità, operabilità;
    \item \textbf{Motivo}: Verificare che il software sia in grado di rilevare condizioni di errore e segnalarle con un opportuno messaggio;
    \item \textbf{Misurazione}:
    \[
        \text{Tolleranza agli errori} = \frac{Err_{s}}{Err_{p}} \times 100
    \]
    dove:
    \begin{itemize}
        \item $Err_{s}$: Numero di errori gestiti con successo;
        \item $Err_{p}$: Numero totale di errori previsti.
    \end{itemize}
\end{itemize}

\subsubsubsection{Tolleranza agli errori}
\begin{itemize}
    \item \textbf{Notazione specifica}: M.PD.20;
    \item \textbf{Nome}: Tolleranza agli errori;
    \item \textbf{Descrizione}: Questa metrica misura la percentuale di errori che il prodotto è in grado di gestire;
    \item \textbf{Caratteristica di riferimento}: Affidabilità, operabilità;
    \item \textbf{Motivo}: Verificare che il software sia in grado di rilevare condizioni di errore e segnalarle con un opportuno messaggio.
    \item \textbf{Misurazione}:
    \[
        \text{Tolleranza agli errori} = \frac{Err_{s}}{Err_{p}} \times 100
    \]
    dove:
    \begin{itemize}
        \item $Err_{s}$: Numero di errori gestiti con successo;
        \item $Err_{p}$: Numero totale di errori previsti.
    \end{itemize}
\end{itemize}

\paragraph*{Impatto delle modifiche}
\begin{itemize}
    \item \textbf{Notazione specifica}: M.PD.21;
    \item \textbf{Nome}: Impatto delle modifiche;
    \item \textbf{Descrizione}: Questa metrica misura l'impatto sulla corretta esecuzione del software procurato dalle modifiche al codice;
    \item \textbf{Caratteristiche}: Manutenibilità, stabilità;
    \item \textbf{Motivo}: Valutare la stabilità del prodotto a seguito di cambiamenti. Un alto impatto negativo può indicare un sistema vulnerabile, in cui le modifiche causano effetti a catena significativi;
    \item \textbf{Misurazione}: 
    \[
    \text{Impatto delle modifiche} = \frac{I_{m}}{M_{t}} \times 100
    \]
    dove:
    \begin{itemize}
        \item $I_{m}$: Numero di modifiche che hanno influito negativamente sul corretto funzionamento del software o sulle sue prestazioni;
        \item $M_{t}$: Numero totale di modifiche eseguite.
    \end{itemize}
\end{itemize}



%\subsection{Azioni correttive}
%\par Le metriche elencate verranno utilizzate e valutate all'interno di un modello \glossario{SPICE} al fine di promuovere un ciclo continuo di valutazione e miglioramento. Alla fine di ogni sprint, dove possibile in modo automatico o manualmente per alcune metriche, verranno valutate le aree di maggior difficoltà e quelle su cui è necessario migliorare. Successivamente, saranno implementate azioni correttive mirate per affrontare eventuali problemi riscontrati.

%Ad esempio, nel caso delle metriche del questionario interno, verrà valutato il livello di conformità rispetto ai criteri prestabiliti. In caso di risultati inferiori alle aspettative, saranno adottate azioni correttive specifiche. Come ad esempio, se più di 5 attività vengono lasciate fuori dallo sprint (o una a testa per ogni ruolo), lo sprint non verrà concluso e saranno identificate le cause sottostanti per migliorare il processo di pianificazione e gestione delle attività.

