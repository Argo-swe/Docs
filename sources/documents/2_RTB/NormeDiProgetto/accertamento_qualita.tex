\section{Accertamento diqualità}

\subsection{Scopo}
\par Il processo di certificazione della qualità mira a garantire che i prodotti software e i processi coinvolti nel ciclo di vita del progetto siano conformi ai requisiti e alle aspettative. L'obiettivo primario dell'accertamento della qualità è garantire che il lavoro svolto rispetti gli standard e le linee guida. È essenziale stabilire internamente parametri misurabili per valutare il grado di aderenza alle \glossario{best practices} dell'ingegneria del software, al fine di condurre un controllo e un miglioramento continuo dei processi. L'assicurazione della qualità può avvalersi dei risultati di altri processi di supporto (es.: verifica e validazione).

\subsection{Garanzia della qualità}

\par Per garantire il raggiungimento e il mantenimento degli standard di qualità prefissati, il team applica il ciclo di \glossario{PDCA}, un metodo di gestione iterativo che contribuisce al controllo e al miglioramento continuo dei processi e dei prodotti all'interno di un'organizzazione; consente inoltre di adattarsi a cambiamenti nel lungo periodo. Il PDCA, noto anche come ciclo di Deming, si divide in 4 fasi interconnesse:
\begin{itemize}
    \item \textbf{Plan} (Pianificare): in questa fase vengono definiti gli obiettivi di qualità da conseguire, nonché le strategie e le azioni necessarie per raggiungerli e misurarli. È importante identificare chiaramente le risorse disponibili, i tempi e le modalità di implementazione del piano. La pianificazione aiuta a stabilire obiettivi e processi necessari per fornire i risultati desiderati;
    \item \textbf{Do} (Fare): una volta stabilito, il piano viene messo in pratica. Questa fase coinvolge l'attuazione delle azioni pianificate, l'allocazione delle risorse e l'esecuzione delle attività secondo le specifiche stabilite al passaggio precedente. Inoltre, vengono raccolti dati per la generazione di grafici e analisi;
    \item \textbf{Check} (Verificare): in questa fase si valutano i risultati ottenuti confrontandoli con gli obiettivi pianificati e gli standard di qualità prefissati. Attraverso un insieme di indicatori, il team può determinare la qualità dei processi e verificare se i risultati prodotti sono in linea con le attese. I grafici dei dati possono agevolare il processo di test, in quanto è possibile osservare le tendenze di più cicli di PDCA;
    \item \textbf{Act} (Agire): sulla base dei risultati della fase di verifica, vengono identificate eventuali discrepanze, non conformità, opportunità di miglioramento o inefficienze. Durante questa fase, si attuano azioni correttive per migliorare la qualità dei processi e del prodotto.
\end{itemize}

\subsection{Notazione delle metriche}
\par Le metriche vengono identificate in modo univoco secondo questa notazione: 
\par \textbf{M.[Tipo].[Codice]}
dove: 
\begin{itemize}
    \item \textbf{M}: indica la parola "metrica";
    \item \textbf{Tipo}: indica il tipo di qualità: 
        \begin{itemize}
            \item \textbf{PC}: qualità di processo; 
            \item \textbf{PD}: qualità di prodotto.
        \end{itemize}
    \item \textbf{Codice}: è un numero progressivo che identifica in modo univoco le metriche per ogni tipologia.
\end{itemize}

\subsection{Didascalia}
Le metriche sono descritte dai seguenti campi:
\begin{itemize}
    \item \textbf{Notazione}: segue le specifiche sopra elencate;
    \item \textbf{Nome}: nome della metrica;
    \item \textbf{Descrizione}: descrizione della metrica;
    \item \textbf{Caratteristiche}: una o più caratteristiche definite dallo standard di riferimento. Questo campo è disponibile per le metriche di prodotto;
    \item \textbf{Motivo}: ragione per cui la metrica viene misurata;
    \item \textbf{Misurazione}: formula e/o strumenti con cui calcolare la metrica.
\end{itemize}

\subsection{Standard di riferimento per la qualità di prodotto}
\par Per l'identificazione e la classificazione delle metriche, il team segue lo standard ISO/IEC 9126, che suddivide la qualità in: esterna (comportamento del software durante la sua esecuzione), interna (si applica al software non eseguibile) e in uso. Il modello di qualità è suddiviso in sei caratteristiche generali:
\begin{itemize}
    \item \textbf{Funzionalità}: capacità del software di fornire le funzioni necessarie per soddisfare esigenze specifiche operando in determinate condizioni;
    \item \textbf{Affidabilità}: capacità del software di mantenere un determinato livello di prestazioni quando viene usato in condizioni specifiche per un certo periodo di tempo;
    \item \textbf{Usabilità}: capacità del software di essere compreso dall'utente. L'usabilità comprende un insieme di attributi che incidono sullo sforzo necessario per l'uso del prodotto;
    \item \textbf{Efficienza}: capacità del software di fornire prestazioni adeguate in relazione alla quantità di risorse usate;
    \item \textbf{Manutenibilità}: capacità del software di essere modificato per includere correzioni, miglioramenti o adattamenti;
    \item \textbf{Portabilità}: capacità del software di essere trasferito tra ambienti di lavoro diversi, che possono variare sia per hardware che per sistema operativo.
\end{itemize}

\vspace{0.5\baselineskip}
\par La qualità in uso rappresenta il punto di vista dell'utente sul software. Nel contesto della qualità in uso, le metriche misurano le seguenti caratteristiche:
\begin{itemize}
    \item \textbf{Efficacia}: capacità del software di consentire agli utenti di raggiungere gli obiettivi desiderati con accuratezza e completezza;
    \item \textbf{Produttività}: capacità del software di permettere agli utenti di impiegare una quantità di risorse appropriate in relazione all'efficacia ottenuta in un determinato contesto d'uso;
    \item \textbf{Soddisfazione}: capacità del software di soddisfare gli utenti che ne usufruiscono;
    \item \textbf{Sicurezza}: capacità del software di raggiungere livelli accettabili di rischio, indipendentemente dalla natura del rischio.
\end{itemize}





\subsection{Metriche}
TODO.

\subsubsection{Tipologie}

\subsubsection{Metriche di prodotto e di qualità del software:}


\subsubsubsection{Code coverage}
\begin{itemize}
    \item \textbf{Notazione specifica}: M.1.1
    \item \textbf{Nome}: Code coverage
    \item \textbf{Descrizione}: Serve per valutare la percentuale di codice sorgente di un'applicazione software che è stata eseguita durante l'esecuzione dei test automatizzati. Indica quindi la quantità di codice che viene testata rispetto alla totalità del codice sorgente. Una copertura topologica del test del 100\% di tipo code coverage garantisce di aver eseguito almeno una volta tutte le istruzioni, ma non tutti i rami;
    \item \textbf{Caratteristica di riferimento}: Affidabilità e manutenibilità del software;
    \item \textbf{Motivo}: È stata introdotta per valutare l'efficacia dei test automatizzati nel garantire la correttezza e l'affidabilità del software. Una copertura del codice elevata suggerisce una maggiore confidenza nella stabilità e nella qualità del software;
    \item \textbf{Misurazione}: In Python è stato utilizzato lo strumento \texttt{coverage.py} che permette di eseguire test del codice sorgente e generare report sulla copertura del codice.

\end{itemize}

\subsubsubsection{Modified condition/decision coverage}
\begin{itemize}
    \item \textbf{Notazione specifica}: M.1.2;
    \item \textbf{Nome}: Modified condition/decision coverage;
    \item \textbf{Descrizione}: È una combinazione delle metriche di function coverage (copertura delle funzioni chiamate) e branch coverage (copertura dei \glossario{branch} delle strutture di controllo). Questa metrica richiede che ogni punto di entrata o uscita in un programma sia invocato almeno una volta e che per ogni decisione condizionale vengano considerati tutti i possibili esiti. La versione modified richiede inoltre che entrambe le coperture siano soddisfatte, ed in particolare che ogni condizione influenzi gli esiti condizionali indipendentemente;
    \item \textbf{Caratteristica di riferimento}: Affidabilità e correttezza del software;
    \item \textbf{Motivo}: La metrica è stata inserita per garantire una copertura completa e accurata delle condizioni e delle decisioni nel codice sorgente. Ciò aiuta a ridurre il rischio di errori di logica nel software e a garantire che tutte le possibili combinazioni di risultati di condizioni siano testate in modo esaustivo;
    \item \textbf{Misurazione}: I valori sono presi dallo strumento di analisi per Python ovvero Coverage.py.
\end{itemize}

\subsubsubsection{Test superati}
\begin{itemize}
    \item \textbf{Notazione specifica}: M.1.3;
    \item \textbf{Nome}: Test superati;
    \item \textbf{Descrizione}: Questa metrica indica la percentuale di test che sono stati superati con successo rispetto al numero totale di test eseguiti;
    \item \textbf{Caratteristica di riferimento}: Efficacia dei test nel rilevare difetti;
    \item \textbf{Motivo}: È stata introdotta per valutare l'efficacia dei test eseguiti nel rilevare eventuali difetti nel software. Misura quanto bene i test eseguiti stiano identificando e segnalando le eventuali anomalie nel comportamento del software;
    \item \textbf{Misurazione}: La metrica viene calcolata utilizzando la seguente formula:
    \[
    \text{Test superati (\%)} = \frac{N_{ts}}{N_{te}} \times 100
    \]
    dove:
    \begin{itemize}
        \item $N_{ts}$ rappresenta il numero totale di test che sono stati superati con successo;
        \item $N_{te}$ rappresenta il numero di test effettivamente eseguiti.
    \end{itemize}
    
\end{itemize}

\subsubsubsection{Test superati}
\begin{itemize}
    \item \textbf{Notazione specifica}: M.1.4
    \item \textbf{Nome}: Test superati
    \item \textbf{Descrizione}: Questa metrica indica la percentuale di test che sono stati superati con successo rispetto al numero totale di test eseguiti;
    \item \textbf{Caratteristica di riferimento}: Efficacia dei test nel rilevare difetti;
    \item \textbf{Motivo}: È stata introdotta per valutare l'efficacia dei test eseguiti nel rilevare eventuali difetti nel software. Misura quanto bene i test eseguiti stiano identificando e segnalando le eventuali anomalie nel comportamento del software;
    \item \textbf{Misurazione}: La metrica viene calcolata utilizzando la seguente formula:
    \[
    \text{Test superati (\%)} = \frac{N_{ts}}{N_{te}} \times 100
    \]
    Dove:
    \begin{itemize}
        \item $N_{ts}$ rappresenta il numero totale di test che sono stati superati con successo.
        \item $N_{te}$ rappresenta il numero di test effettivamente eseguiti.
    \end{itemize}
    %\item \textbf{Range}: 90-100\%
    %\item \textbf{Modifiche nel tempo}: 
    %\item \textbf{Azioni Correttive}:
    %\begin{itemize}
        %\item Analizzare i test falliti per identificare le cause (es. errori nel codice, problemi di configurazione) e implementare le correzioni necessarie.
        %\item Rivedere e migliorare i casi di test per coprire meglio le funzionalità critiche del software e ridurre i falsi positivi.
        %\item Automatizzare i test, ove possibile, per aumentare la copertura e ridurre il rischio di errori manuali.
        %\item Condurre revisioni periodiche dei test per aggiornare i test esistenti e svilupparne di nuovi in base alle modifiche del software.
    %\end{itemize}
    %\item \textbf{Obiettivi di Miglioramento}:
    %\begin{itemize}
        %\item Aumentare la percentuale di test superati fino a raggiungere e mantenere un livello superiore al 95\%.
        %\item Garantire che i test coprano tutte le funzionalità critiche e rilevino efficacemente i difetti nel software.
        %\item Ridurre il numero di difetti rilevati durante il testing migliorando la qualità del software attraverso pratiche di sviluppo rigorose.
        %\item Promuovere una cultura di testing continuo e accurato all'interno del team di sviluppo per mantenere elevata l'efficacia dei test nel tempo.
    %\end{itemize}
\end{itemize}

\subsubsubsection{Gestione delle operazioni non permesse}
\begin{itemize}
    \item \textbf{Notazione specifica}: M.1.5;
    \item \textbf{Nome}:  Gestione delle operazioni non permesse;
    \item \textbf{Descrizione}: Appresenta la percentuale di operazioni non consentite o non gestite correttamente dal software durante l'esecuzione;
    \item \textbf{Caratteristica di riferimento}: Affidabilità e robustezza;
    \item \textbf{Motivo}: La gestione delle operazioni non permesse è stata introdotta per valutare la robustezza e l'affidabilità del software nell'affrontare input imprevisti o non validi. Misura quanto bene il sistema sia in grado di gestire eccezioni e situazioni anomale senza interrompere o compromettere il normale flusso di esecuzione;
    \item \textbf{Misurazione}: Viene calcolata come:
    \[
    \text{Gestione delle operazioni non permesse (\%)} = \frac{N_{np}}{N_{t}} \times 100
    \]
    
    dove:
    \begin{itemize}
        \item $N_{np}$: rappresenta il numero di operazioni non permesse o non gestite correttamente;
        \item $N_{t}$: rappresenta il numero totale di operazioni eseguite durante il test.
    \end{itemize}
\end{itemize}
\subsubsection{Accuratezza rispetto alle attese}
\begin{itemize}
    \item \textbf{Notazione specifica}: M.1.6
    \item \textbf{Nome}: Accuratezza rispetto alle attese
    \item \textbf{Descrizione}: Rappresenta la percentuale di risultati dei test che rispettano quanto previsto. Misura l'aderenza dei risultati dei test alle aspettative stabilite durante la fase di pianificazione dei test;
    \item \textbf{Caratteristica di riferimento}: Affidabilità e conformità del software ai requisiti specificati;
    \item \textbf{Motivo}: L'accuratezza rispetto alle attese è stata introdotta per valutare quanto bene i risultati dei test corrispondano alle aspettative previste durante la fase di pianificazione dei test. Una percentuale alta indica una maggiore affidabilità del software nel produrre risultati conformi alle aspettative;
    \item \textbf{Misurazione}: Viene calcolata utilizzando la seguente formula:
    \[
    \text{Accuratezza rispetto alle attese (\%)} = \left(1 - \frac{N_{rd}}{N_{te}}\right) \times 100
    \]
    Dove:
    \begin{itemize}
        \item $N_{rd}$: Numero di test che producono risultati discordanti.
        \item $N_{te}$: Numero di test eseguiti.
    \end{itemize} 
    %\item \textbf{Range}: 80-100\%
    %\item \textbf{Modifiche nel tempo}:  
\end{itemize}

\subsubsubsection{Core size}
\begin{itemize}
    \item \textbf{Notazione specifica}: M.1.7;
    \item \textbf{Nome}: Core size;
    \item \textbf{Descrizione}: I core file sono file altamente interconnessi da una catena di dipendenze cicliche, i quali sono maggiormente propensi ad avere difetti. Core size è la percentuale di file con una o più dipendenze che hanno un alto fan-in (numero di moduli che dipendono da essi) ed un alto fan-out (numero di moduli da cui dipendono). Questi file sono critici per la stabilità e la manutenibilità del sistema poiché la loro complessità e interconnessione li rendono più vulnerabili a difetti e più difficili da modificare senza introdurre errori;
    \item \textbf{Caratteristica di riferimento}: Affidabilità, Manutenibilità;
    \item \textbf{Motivo}: Identificare i file critici nel sistema che sono suscettibili a difetti a causa della loro alta interconnessione e complessità. Monitorare il core size aiuta a prevenire la formazione di "nodi critici" che possono influire negativamente sulla qualità del software;
    \item \textbf{Misurazione}: Il calcolo di questa metrica viene effettuato attraverso la libreria networkx per analizzare le dipendenze.
\end{itemize}

\subsubsubsection{Gestione delle operazioni non permesse}
\begin{itemize}
    \item \textbf{Notazione specifica}: M.1.8
    \item \textbf{Nome}:  Gestione delle operazioni non permesse;
    \item \textbf{Descrizione}: appresenta la percentuale di operazioni non consentite o non gestite correttamente dal software durante l'esecuzione;
    \item \textbf{Caratteristica di riferimento}: affidabilità e robustezza;
    \item \textbf{Motivo}: la gestione delle operazioni non permesse è stata introdotta per valutare la robustezza e l'affidabilità del software nell'affrontare input imprevisti o non validi. Misura quanto bene il sistema sia in grado di gestire eccezioni e situazioni anomale senza interrompere o compromettere il normale flusso di esecuzione;
    \item \textbf{Misurazione}: viene calcolata come:
    \[
    \text{Gestione delle operazioni non permesse (\%)} = \frac{N_{np}}{N_{t}} \times 100
    \]
    
    dove:
    \begin{itemize}
        \item $N_{np}$: rappresenta il numero di operazioni non permesse o non gestite correttamente.
        \item $N_{t}$: rappresenta il numero totale di operazioni eseguite durante il test.
    \end{itemize}
    %\item \textbf{Range}:dal 0\% al 100\%, dove il 0\% indica che tutte le operazioni non permesse sono state gestite correttamente e il 100\% indica che nessuna operazione non permessa è stata gestita correttamente.
    %\item \textbf{Modifiche nel tempo}:  
\end{itemize}
\subsubsection{Numero di parametri per funzione}
\begin{itemize}
    \item \textbf{Notazione specifica}: M.1.9
    \item \textbf{Nome}: Numero di parametri per funzione
    \item \textbf{Descrizione}: Misura il numero medio di parametri passati alle funzioni nel codice sorgente del software;
    \item \textbf{Caratteristica di riferimento}: Manutenibilità e complessità del software;
    \item \textbf{Motivo}: Il numero di parametri per funzione è stato introdotto per valutare la complessità delle funzioni all'interno del software. Un numero elevato di parametri può indicare una scarsa progettazione modulare e può rendere il codice difficile da comprendere, testare e mantenere. Questa metrica aiuta a identificare funzioni che potrebbero beneficiare di una rifattorizzazione per migliorare la qualità del codice;
    \item \textbf{Misurazione}: La metrica "Numero di Parametri per Funzione" viene calcolata utilizzando la seguente formula:
    \[
    \text{Numero di Parametri per Funzione} = \frac{\text{TP}}{\text{TF}}
    \]
    dove:
    \begin{itemize}
        \item \textbf{TP}: somma del numero di parametri di tutte le funzioni nel codice.
        \item \textbf{TF}: numero totale di funzioni nel codice.
    \end{itemize}
    \item \textbf{Interpretazione del Risultato}:
    \begin{itemize}
        \item \textbf{0-3 parametri per funzione}: Bassa complessità; funzioni ben progettate e facili da mantenere.
        \item \textbf{4-6 parametri per funzione}: Moderata complessità; le funzioni potrebbero essere accettabili, ma si potrebbe considerare una rifattorizzazione.
        \item \textbf{7-10 parametri per funzione}: Alta complessità; potrebbe essere necessario semplificare le funzioni per migliorare la manutenibilità.
        \item \textbf{>10 parametri per funzione}: Molto alta complessità; fortemente raccomandata la rifattorizzazione delle funzioni per ridurre il numero di parametri.
    \end{itemize}
\end{itemize}

\subsubsubsection{Accuratezza della risposta}
\begin{itemize}
    \item \textbf{Notazione specifica}: M.1.10;
    \item \textbf{Nome}: Accuratezza della risposta;
    \item \textbf{Descrizione}: L'accuratezza della risposta è una misura della correttezza e precisione con cui il sistema risponde a una interrogazione in linguaggio naturale. La metrica viene valutata in base a a dei test descritti nel \glossario{Dizionario dati};
    \item \textbf{Caratteristica di riferimento}: Funzionalità, Affidabilità;
    \item \textbf{Motivo}: Garantire che il sistema fornisca risposte corrette e affidabili è cruciale per la soddisfazione dell'utente e il funzionamento efficiente del software. Misurare l'accuratezza delle risposte aiuta a identificare e correggere errori nel sistema, migliorando la qualità complessiva del software;
    \item \textbf{Misurazione}: L'accuratezza della risposta è misurata in base alla distanza della risposta ottenuta dopo la generazione di un prompt su frasi preselezionate e la risposta attesa scritta dall'attore tecnico nel dizionario dati.Questa percentuale l'abbiamo calcolata come :
    \[
    \text{Accuratezza della risposta} = \frac{R_{c}}{R_{a}} \times 100
    \]
    Dove:
    \begin{itemize}
        \item $R_{c}$: Risposta corretta;
        \item $N_{a}$: Risposta attesa.
    \end{itemize}
 \end{itemize}

\subsubsubsection{Indice di manutenibilità}
\begin{itemize}
    \item \textbf{Notazione specifica}: M.1.11;
    \item \textbf{Nome}: Indice di manutenibilità;
    \item \textbf{Descrizione}: L'indice di manutenibilità è una misura che riflette quanto sia facile mantenere, comprendere, modificare e correggere il codice sorgente;
    \item \textbf{Caratteristica di riferimento}: Manutenibilità;
    \item \textbf{Motivo}: Fornire una misura quantitativa della manutenibilità del software, aiutando a identificare aree del codice che potrebbero richiedere rifattorizzazione o miglioramenti per ridurre i costi di manutenzione e migliorare la qualità del software;
    \item \textbf{Misurazione}: Il calcolo è attuato con lo strumento SonarQube.
    
\end{itemize}

\subsubsubsection{Linee medie di codice per metodo}
\begin{itemize}
    \item \textbf{Notazione specifica}: M.1.12;
    \item \textbf{Nome}: Linee medie di codice per metodo;
    \item \textbf{Descrizione}: Le linee medie di codice per metodo è una misura che indica la lunghezza media, in termini di linee di codice, dei metodi o funzioni all'interno del codice sorgente;
    \item \textbf{Caratteristica di riferimento}: Manutenibilità;
    \item \textbf{Motivo}: Valutare la leggibilità e la manutenibilità del codice. Metodi più corti sono generalmente preferibili perché tendono a essere più semplici da comprendere, testare e mantenere. Identificare metodi eccessivamente lunghi può aiutare a focalizzare gli sforzi di rifattorizzazione;
    \item \textbf{Misurazione}: Il calcolo di questa metrica è effettuato con lo strumento SonarQube.
\end{itemize}

\subsubsubsection{Accuratezza della risposta}
\begin{itemize}
    \item \textbf{Notazione specifica}: M.1.13;
    \item \textbf{Nome}: Accuratezza della risposta;
    \item \textbf{Descrizione}: L'accuratezza della risposta è una misura della correttezza e precisione con cui il sistema risponde a una interrogazione in linguaggio naturale. La metrica viene valutata in base a a dei test descritti nel \glossario{Dizionario dati};
    \item \textbf{Caratteristica di riferimento}: Funzionalità, Affidabilità;
    \item \textbf{Motivo}: Garantire che il sistema fornisca risposte corrette e affidabili è cruciale per la soddisfazione dell'utente e il funzionamento efficiente del software. Misurare l'accuratezza delle risposte aiuta a identificare e correggere errori nel sistema, migliorando la qualità complessiva del software;
    \item \textbf{Misurazione}: L'accuratezza della risposta è misurata in base alla distanza della risposta ottenuta dopo la generazione di un prompt su frasi preselezionate e la risposta attesa scritta dall'attore tecnico nel dizionario dati.Questa percentuale l'abbiamo calcolata come :
    \[
    \text{Accuratezza della risposta} = \frac{R_{c}}{R_{a}} \times 100
    \]
    Dove:
    \begin{itemize}
        \item $R_{c}$: Risposta corretta;
        \item $N_{a}$: Risposta attesa.
    \end{itemize}
 \end{itemize}

\subsubsubsection{Efficienza dell'installazione}
\begin{itemize}
    \item \textbf{Notazione specifica}: M.1.14;
    \item \textbf{Nome}: Efficienza dell'installazione;
    \item \textbf{Descrizione}: L'efficienza dell'installazione è una metrica che valuta la rapidità con cui il software può essere installato e configurato il sistema;
    \item \textbf{Caratteristica di riferimento}: Efficienza;
    \item \textbf{Motivo}: Un'installazione efficiente è cruciale per garantire che gli utenti possano iniziare a utilizzare il software rapidamente e senza difficoltà. Ridurre il tempo e lo sforzo necessari per installare il software migliora l'esperienza utente complessiva e riduce la possibilità di errori durante l'installazione;
    \item \textbf{Misurazione}: TODO.
\end{itemize}

\subsubsubsection{Tempo di risposta}
\begin{itemize}
    \item \textbf{Notazione specifica}: M.1.15;
    \item \textbf{Nome}: Tempo di risposta;
    \item \textbf{Descrizione}: Questa metruca serve per misurare l’efficienza del prodotto. Il tempo di risposta indica il tempo medio in cui il sistema genera il \glossario{prompt} in base al comando immesso dall’utente;
    \item \textbf{Caratteristica di riferimento}: Efficienza-Comportamento;
    \item \textbf{Motivo}: Misurare e migliorare il tempo medio in cui il sistema risponde;
\end{itemize}

\subsubsubsection{Accessibilità}
\begin{itemize}
    \item \textbf{Notazione specifica}: M.1.16;
    \item \textbf{Nome}: Accessibilità;
    \item \textbf{Descrizione}: L'accessibilità è una misura che valuta la facilità con cui gli utenti, inclusi coloro con disabilità, possono interagire con il sistema software;
    \item \textbf{Caratteristica di riferimento}: Usabilità, Accessibilità;
    \item \textbf{Motivo}: Garantire che il sistema sia accessibile a tutti gli utenti, indipendentemente dalle loro capacità fisiche o cognitive, è fondamentale per fornire un'esperienza utente inclusiva e soddisfacente;
    \item \textbf{Misurazione}: L'accessibilità può essere valutata utilizzando strumenti automatizzati di valutazione dell'accessibilità come Axe, che analizza l'accessibilità dell'interfaccia utente rispetto alle linee guida WCAG (Web Content Accessibility Guidelines).
    %\item \textbf{Range}: Conforme / Non conforme agli standard di accessibilità
    %\item \textbf{Spiegazione}: Un sistema è considerato conforme agli standard di accessibilità se rispetta le linee guida e i criteri stabiliti dagli standard di accessibilità pertinenti (ad esempio, WCAG). Mantenere l'accessibilità conforme agli standard aiuta a garantire che il sistema sia accessibile a una vasta gamma di utenti, migliorando l'esperienza utente complessiva e dimostrando un impegno per l'inclusione e l'equità.
\end{itemize}

\subsubsubsection{Tempo di risposta}
\begin{itemize}
    \item \textbf{Notazione specifica}: M.1.17;
    \item \textbf{Nome}: Tempo di risposta;
    \item \textbf{Descrizione}: Fa parte delle metriche esterne di comportamento rispetto al tempo necessario; serve per misurare l’efficienza del prodotto. Il tempo di risposta indica il tempo medio in cui il sistema risponde ad un comando immesso dall’utente;
    \item \textbf{Caratteristica di riferimento}: Efficienza-Comportamento;
    \item \textbf{Motivo}: Misurare e migliorare il tempo medio in cui il sistema risponde ad un comando immesso dall’utente;
    \item \textbf{Misurazione}: Il tempo di risposta (M) è calcolato moltiplicando il tempo (T) che intercorre tra l’immissione del comando da parte dell’operatore e la presentazione della risposta del sistema: \( M = T \).
    %\item \textbf{Range}: Dipende dal contesto dell'applicazione e dalle aspettative degli utenti. In generale, un tempo di risposta inferiore a 2 secondi è considerato accettabile per molte applicazioni interattive.
\end{itemize}


\subsubsection{Metriche di processo:}

\subsubsubsection{Requisiti opzionali soddisfatti}
\begin{itemize}
    \item \textbf{Notazione specifica}: M.2.1
    \item \textbf{Nome}: Requisiti opzionali soddisfatti
    \item \textbf{Descrizione}: Questa metrica indica la percentuale di requisiti opzionali soddisfatti dal software rispetto al totale dei requisiti opzionali definiti nel documento di specifica dei requisiti.
    \item \textbf{Caratteristica di riferimento}: Funzionalità, Qualità (ISO/IEC 25010)
    \item \textbf{Motivo}: Misura il grado di adempimento dei requisiti opzionali, fornendo una valutazione della completezza del software rispetto alle richieste supplementari specificate.
    \item \textbf{Misurazione}: La percentuale di requisiti opzionali soddisfatti (\textbf{ROS}) può essere calcolata utilizzando la seguente formula:
    \[ ROS = \frac{R_o}{R_t} \times 100\% \]
    dove:
    \begin{itemize}
        \item \textbf{R_o}: Requisiti opzionali soddisfatti
        \item \textbf{R_t}: Totale dei requisiti opzionali
    \end{itemize}
    \item \textbf{Range}: Il range ideale per questa metrica è dal 100\% (tutti i requisiti opzionali soddisfatti) al 0\% (nessun requisito opzionale soddisfatto).
    \item \textbf{Spiegazione}: Una percentuale più alta di requisiti opzionali soddisfatti indica una maggiore capacità del software di soddisfare richieste aggiuntive che non sono considerate essenziali ma che possono migliorare l'esperienza dell'utente o fornire funzionalità extra.
\end{itemize}

\subsubsubsection{Variazione pianificazione task completati}
\begin{itemize}
    \item \textbf{Notazione specifica}: M.2.2;
    \item \textbf{Nome}: Variazione pianificazione task completati;
    \item \textbf{Descrizione}: Questa metrica misura la variazione tra il numero di task pianificati e quelli effettivamente completati entro un periodo di tempo specifico. È un indicatore della capacità del team di rispettare le scadenze e le pianificazioni stabilite;
    \item \textbf{Caratteristica di riferimento}: Gestione del progetto, Adempimento delle scadenze;
    \item \textbf{Motivo}: Valutare l'efficacia della pianificazione del progetto e l'aderenza del team alle tempistiche programmate, identificando eventuali deviazioni che potrebbero richiedere interventi correttivi;
    \item \textbf{Misurazione}: La variazione pianificazione task completati è calcolato utilizzando la seguente formula:
    \[
        \text{Variazione pianificazione task completati} =\frac{T_p - T_c}{T_p} \times 100
    \]
    dove:
    \begin{itemize}
        \item $T_{p}$: Numero di task pianificati;
        \item $T_{c}$: Numero di task completati.
    \end{itemize}
    
    %\item \textbf{Range}: Il range ideale per questa metrica è dal 0\% (nessuna variazione, tutti i task completati come pianificato) al 100\% (tutti i task pianificati non completati).
    %\item \textbf{Spiegazione}: Una percentuale più bassa indica una maggiore aderenza del team alla pianificazione originale, suggerendo un'efficace gestione del progetto e capacità di rispettare le scadenze.
\end{itemize}

\subsubsubsection{Variazione di costo}
\begin{itemize}
    \item \textbf{Notazione specifica}: M.2.3
    \item \textbf{Nome}: Variazione di costo
    \item \textbf{Descrizione}: Questa metrica misura la variazione tra il costo pianificato e il costo effettivo di un progetto o di una fase di un progetto. È un indicatore dell'efficacia della gestione dei costi e della capacità del team di mantenere il progetto entro il budget stabilito.
    \item \textbf{Caratteristica di riferimento}: Gestione del progetto, Controllo dei costi
    \item \textbf{Motivo}: Valutare la precisione delle stime dei costi iniziali e la capacità del progetto di rimanere entro il budget, identificando eventuali sforamenti che potrebbero richiedere interventi correttivi.
    \item \textbf{Misurazione}: La variazione di costo (\textbf{VC}) può essere calcolata utilizzando la seguente formula:
    \[ VC = \frac{C_p - C_a}{C_p} \times 100\% \]
    dove:
    \begin{itemize}
        \item \textbf{C_p}: Costo pianificato
        \item \textbf{C_a}: Costo attuale (effettivo)
    \end{itemize}
    \item \textbf{Range}: Il range ideale per questa metrica è dal 0\% (nessuna variazione, costo effettivo uguale al costo pianificato) al 100\% (costo effettivo doppio rispetto al costo pianificato).
    \item \textbf{Spiegazione}: Una percentuale più bassa indica una migliore gestione dei costi del progetto, suggerendo che il progetto è stato completato entro il budget previsto. Deviazioni significative possono indicare problemi di stima dei costi o di controllo del budget.
\end{itemize}

\subsubsubsection{Efficienza temporale}
\begin{itemize}
    \item \textbf{Notazione specifica}: M.2.4;
    \item \textbf{Nome}: Efficienza temporale;
    \item \textbf{Descrizione}: Questa metrica misura l'efficienza con cui il tempo viene utilizzato per completare un progetto o una fase di un progetto rispetto al tempo pianificato;
    \item \textbf{Caratteristica di riferimento}: Gestione del progetto, Adempimento delle scadenze;
    \item \textbf{Motivo}: Valutare la capacità del progetto di rispettare le tempistiche programmate, identificando eventuali ritardi e migliorando la gestione del tempo;
    \item \textbf{Misurazione}: L'efficienza temporale è calcolata utilizzando la seguente formula:
    \[
        \text{Efficienza temporale (\%)} =\frac{T_p}{T_e} \times 100
    \]
    dove:
    \begin{itemize}
        \item $T_{p}$: Tempo pianificato;
        \item $T_{e}$: Tempo effettivo.
    \end{itemize}
    %\item \textbf{Range}: Il range ideale per questa metrica è dal 100\% (tempo effettivo uguale al tempo pianificato) a valori inferiori (tempo effettivo superiore al tempo pianificato).
    %\item \textbf{Spiegazione}: Una percentuale più alta indica una migliore efficienza nell'uso del tempo, suggerendo che il progetto è stato completato entro il tempo previsto. Valori inferiori a 100\% possono indicare ritardi o inefficienze nella gestione del tempo.
\end{itemize}

\subsubsection{M.2.5 Velocità di verifica dopo una PR}
\begin{itemize}
    \item \textbf{Notazione specifica}: M.2.5
    \item \textbf{Nome}: Velocità di verifica dopo una PR
    \item \textbf{Descrizione}: Questa metrica misura il tempo medio impiegato per verificare e approvare una pull request (PR) dopo che è stata sottomessa. È un indicatore della rapidità e dell'efficienza del processo di revisione del codice.
    \item \textbf{Caratteristica di riferimento}: Efficienza, Manutenibilità
    \item \textbf{Motivo}: Valutare la rapidità con cui il team di sviluppo riesce a verificare e integrare le modifiche proposte, migliorando il flusso di lavoro e riducendo i tempi di attesa.
    \item \textbf{Misurazione}: La velocità di verifica dopo una PR (\textbf{VPR}) può essere calcolata utilizzando la seguente formula:
    \[ VPR = \frac{1}{N} \sum_{i=1}^{N} T_i \]
    dove:
    \begin{itemize}
        \item \textbf{N}: Numero di PR verificate
        \item \textbf{T_i}: Tempo impiegato per verificare la \(i\)-esima PR
    \end{itemize}
    \item \textbf{Range}: Il range ideale per questa metrica può variare a seconda del progetto, ma generalmente un tempo medio di verifica inferiore a 24 ore è considerato efficiente.
    \item \textbf{Spiegazione}: Un tempo medio di verifica più basso indica una maggiore efficienza nel processo di revisione del codice, suggerendo un flusso di lavoro più snello e una più rapida integrazione delle modifiche.
\end{itemize}

\subsubsection{M.2.6 Dimensione del commit}
\begin{itemize}
    \item \textbf{Notazione specifica}: M.2.6
    \item \textbf{Nome}: Dimensione del commit
    \item \textbf{Descrizione}: Questa metrica misura la quantità di codice aggiunto, modificato o rimosso in un singolo commit. È un indicatore della granularità delle modifiche e della frequenza dei cambiamenti nel repository.
    \item \textbf{Caratteristica di riferimento}: Manutenibilità, Qualità del codice
    \item \textbf{Motivo}: Valutare la dimensione dei commit aiuta a comprendere la strategia di gestione del codice e a identificare pratiche di sviluppo che favoriscono la leggibilità e la facilità di revisione del codice.
    \item \textbf{Misurazione}: La dimensione del commit (\textbf{DC}) può essere calcolata utilizzando la seguente formula:
    \[ DC = L_{added} + L_{modified} + L_{deleted} \]
    dove:
    \begin{itemize}
        \item \textbf{L_{added}}: Numero di linee di codice aggiunte
        \item \textbf{L_{modified}}: Numero di linee di codice modificate
        \item \textbf{L_{deleted}}: Numero di linee di codice rimosse
    \end{itemize}
    \item \textbf{Range}: Il range ideale per questa metrica può variare a seconda delle pratiche di sviluppo del progetto, ma generalmente i commit contenenti tra 10 e 200 linee di cambiamento sono considerati di dimensione ottimale.
    \item \textbf{Spiegazione}: I commit di dimensione ottimale facilitano la revisione del codice e la gestione delle modifiche, migliorando la qualità complessiva del software e la collaborazione tra i membri del team.
\end{itemize}

\subsubsubsection{Tempo medio tra commit}
\begin{itemize}
    \item \textbf{Notazione specifica}: M.2.7;
    \item \textbf{Nome}: Tempo medio tra commit;
    \item \textbf{Descrizione}: Questa metrica misura l'intervallo medio di tempo tra i commit successivi nel repository del progetto. È un indicatore della frequenza con cui gli sviluppatori effettuano commit nel sistema di controllo versione;
    \item \textbf{Caratteristica di riferimento}: Frequenza del commit, Efficienza dello sviluppo;
    \item \textbf{Motivo}: Valutare il tempo medio tra commit aiuta a comprendere il ritmo di sviluppo e l'approccio del team alla gestione delle versioni del codice. Una frequenza di commit regolare può indicare un flusso di lavoro sano e continuo;
    \item \textbf{Misurazione}: Il tempo medio tra commit è calcolato utilizzando la seguente formula:
    \[
        \text{Tempo medio tra commit} =\frac{\sum_{i=1}^{N-1} (T_{i+1} - T_i)}{N-1}
    \]
    dove:
    \begin{itemize}
        \item $N$: Numero totale di commit;
        \item $T_{i}$: Tempo del \(i\)-esimo commit.
    \end{itemize}
    %\item \textbf{Range}: Il range ideale per questa metrica varia a seconda delle pratiche di sviluppo del progetto, ma generalmente un tempo medio tra commit inferiore a un giorno (24 ore) è considerato indicativo di un flusso di lavoro continuo e regolare.
    %\item \textbf{Spiegazione}: Un tempo medio tra commit più breve suggerisce una frequenza di commit regolare, che può facilitare la gestione delle versioni, la revisione del codice e l'integrazione continua. Intervalli più lunghi possono indicare potenziali problemi nel flusso di lavoro o nella gestione delle versioni.
\end{itemize}

\subsubsection{M.2.8 Frequenza di merge}
\begin{itemize}
    \item \textbf{Notazione specifica}: M.2.8
    \item \textbf{Nome}: Frequenza di merge
    \item \textbf{Descrizione}: Questa metrica misura la frequenza con cui vengono effettuati i merge di branch nel repository del progetto. Indica la frequenza con cui il codice viene integrato nel branch principale dopo aver completato una determinata funzionalità o correzione.
    \item \textbf{Caratteristica di riferimento}: Integrazione continua, Efficienza dello sviluppo
    \item \textbf{Motivo}: Valutare la frequenza di merge aiuta a comprendere la rapidità con cui le nuove funzionalità o correzioni vengono integrate nel codice principale del progetto, favorendo un flusso di lavoro continuo e una collaborazione efficace tra i membri del team.
    \item \textbf{Misurazione}: La frequenza di merge (\textbf{FM}) può essere calcolata come il numero totale di merge effettuati nel periodo di interesse.
    \item \textbf{Range}: Il range ideale per questa metrica può variare a seconda delle pratiche di sviluppo del progetto, ma generalmente una frequenza di merge regolare indica una buona pratica di integrazione continua.
    \item \textbf{Spiegazione}: Una frequenza di merge regolare suggerisce che il team sta seguendo un flusso di lavoro collaborativo e integrando il codice in modo tempestivo, migliorando l'efficienza dello sviluppo e riducendo i rischi di conflitti di merge e divergenze nel codice.
\end{itemize}


\subsubsection{Metriche di gestione dei rischi:}

\subsubsubsection{Rischi inattesi}
\begin{itemize}
    \item \textbf{Notazione specifica}: M.3.1;
    \item \textbf{Nome}: Rischi inattesi;
    \item \textbf{Descrizione}: Questa metrica misura il rapporto tra il numero effettivo di rischi non previsti che emergono durante lo sviluppo del software in ciascuno sprint e il numero massimo previsto di rischi per lo stesso periodo;
    \item \textbf{Caratteristica di riferimento}: Gestione dei rischi;
    \item \textbf{Motivo}: Monitorare i rischi inattesi consente di valutare l'efficacia del processo di identificazione e gestione dei rischi, nonché la capacità del team di rispondere in modo tempestivo e efficace a situazioni impreviste;
    \item \textbf{Misurazione}: Il rapporto tra il numero effettivo di rischi non previsti e il numero massimo previsto di rischi per lo sprint secondo le formula:
    \[
        \text{Rischi inattesi} =\frac{NR_{sprint}}{NR_{max}} 
    \]
    dove:
    \begin{itemize}
        \item $NR_{sprint}$: Numero effettivo di rischi non previsti durante lo sprint;
        \item $NR_{max}$: Numero massimo previsto di rischi per lo sprint.
    \end{itemize}
    %\item \textbf{Range}: Il range ideale per questa metrica è < 1, il che significa che il numero effettivo di rischi non previsti è inferiore al numero massimo previsto di rischi per lo sprint.
    %\item \textbf{Spiegazione}: Un valore inferiore a 1 indica che il numero effettivo di rischi non previsti è inferiore al numero massimo previsto, evidenziando una pianificazione efficace e una gestione proattiva dei rischi durante lo sprint.
\end{itemize}

\input{Accertamento_qualita/Metriche/metriche_M3/M_3_2}
\subsubsubsection{M.3.3 Rischi non previsti in assoluto per sprint}
\begin{itemize}
    \item \textbf{Notazione specifica}: M.3.3
    \item \textbf{Nome}: Rischi non previsti in assoluto per sprint
    \item \textbf{Descrizione}: Questa metrica misura il rapporto tra il numero effettivo di rischi non previsti che emergono durante lo sviluppo del software in ciascuno sprint e il numero massimo previsto di rischi per lo stesso periodo.
    \item \textbf{Caratteristica di riferimento}: Gestione dei rischi
    \item \textbf{Motivo}: Monitorare i rischi non previsti consente di valutare l'efficacia del processo di identificazione e gestione dei rischi, nonché la capacità del team di rispondere in modo tempestivo e efficace a situazioni impreviste.
    \item \textbf{Misurazione}: Il rapporto tra il numero effettivo di rischi non previsti (\(NR_{sprint}\)) e il numero massimo previsto di rischi (\(NR_{max}\)) per lo sprint.
    \item \textbf{Formula di Misurazione}:
    \[
    RNP_{sprint} = \frac{NR_{sprint}}{NR_{max}}
    \]
    Dove:
    \begin{itemize}
        \item \(RNP_{sprint}\) rappresenta il rapporto tra il numero effettivo di rischi non previsti durante lo sprint e il numero massimo previsto di rischi per lo stesso periodo.
        \item \(NR_{sprint}\) è il numero effettivo di rischi non previsti durante lo sprint.
        \item \(NR_{max}\) è il numero massimo previsto di rischi per lo sprint.
    \end{itemize}
    \item \textbf{Range}: Il range ideale per questa metrica è < 1, il che significa che il numero effettivo di rischi non previsti è inferiore al numero massimo previsto di rischi per lo sprint.
    \item \textbf{Spiegazione}: Un valore inferiore a 1 indica che il numero effettivo di rischi non previsti è inferiore al numero massimo previsto, evidenziando una pianificazione efficace e una gestione proattiva dei rischi durante lo sprint.
\end{itemize}

\subsubsubsection{Rischi non previsti su successi}
\begin{itemize}
    \item \textbf{Notazione specifica}: M.3.4;
    \item \textbf{Nome}: Rischi non previsti su successi;
    \item \textbf{Descrizione}: Questa metrica rappresenta il rapporto tra il numero di rischi non previsti che emergono durante lo sviluppo del software e il numero di errori di sistema successi. Gli errori di sistema successi sono quelli che sono stati risolti e corretti con successo durante il processo di sviluppo;
    \item \textbf{Caratteristica di riferimento}: Gestione dei rischi, Affidabilità;
    \item \textbf{Motivo}: Questa metrica fornisce una misura della proporzione di rischi non previsti rispetto agli errori di sistema che sono stati risolti con successo. Aiuta a valutare l'efficacia della gestione dei rischi nel ridurre l'incidenza degli errori nel sistema;
    \item \textbf{Misurazione}: Il rapporto dei rischi è calcolato con questa formula:
    \[
        \text{Percentuale di metriche soddisfatte (\%)} =\frac{R_{n}}{E_{s}}
    \]
    dove:
    \begin{itemize}
        \item $R_{n}$: Numero i rischi non previsti;
        \item $E_{s}$: Numero di sistema successi.
    \end{itemize}
    %\item \textbf{Range}: Il range ideale per questa metrica è tendente a zero, il che indica una bassa incidenza di rischi non previsti rispetto agli errori di sistema successi.
    %\item \textbf{Spiegazione}: Un valore prossimo a zero per questa metrica indica che il numero di rischi non previsti è relativamente basso rispetto al numero di errori di sistema che sono stati correttamente identificati e risolti, suggerendo una gestione efficace dei rischi nel processo di sviluppo del software.
\end{itemize}

\subsubsection{Metriche per la documentazione:}

\subsubsection{M.4.1 Indice Gulpease}
\begin{itemize}
    \item \textbf{Notazione specifica}: M.4.1
    \item \textbf{Nome}: Indice Gulpease
    \item \textbf{Descrizione}: L'Indice Gulpease è una metrica di leggibilità del testo di un documento. È basato sulla lunghezza delle parole e sulla lunghezza delle frasi. La formula è la seguente: \[ 89+\frac{300 \cdot \text{{numero delle frasi}} - 10 \cdot \text{{numero delle lettere}}}{\text{{numero delle parole}}} \]
    \item \textbf{Caratteristica di riferimento}: Qualità del testo
    \item \textbf{Motivo}: L'Indice Gulpease fornisce una valutazione della leggibilità di un testo, importante per garantire la comprensibilità e l'accessibilità delle informazioni contenute nel documento.
    \item \textbf{Misurazione}: Per garantire la leggibilità della documentazione in lingua italiana si applica il calcolo dell’indice di leggibilità Gulpease sui testi, considerando la lunghezza in lettere di parole e frasi.
    \[ 89 + \left( \frac{{300 \cdot \text{{numero delle frasi}} - 10 \cdot \text{{numero delle lettere}}}}{{\text{{numero delle parole}}}} \right) \]
    \item \textbf{Range}: Il range ideale per questa metrica è tra 0 e 100, dove valori più alti indicano una maggiore leggibilità del testo.
    \item \textbf{Spiegazione}: La formula per il calcolo dell’indice di Gulpease restituisce un punteggio in centesimi che determina il grado di comprensibilità rapportato al livello di istruzione del lettore:
    \begin{itemize}
        \item punteggi intorno a 0: testi a leggibilità più bassa;
        \item punteggi minori di 40: testi di difficile comprensione per lettori in possesso di un diploma di scuola superiore;
        \item punteggi minori di 60: testi di difficile comprensione per lettori in possesso di licenza di scuola media;
        \item punteggi minori di 80: testi di difficile comprensione per lettori in possesso di licenza elementare;
        \item punteggi intorno a 100: i testi con maggiore livello di comprensione.
    \end{itemize}
\end{itemize}

\subsubsubsection{M.4.2 Indice Gunning Fog}
\begin{itemize}
    \item \textbf{Notazione specifica}: M.4.2
    \item \textbf{Nome}: Indice Gunning Fog
    \item \textbf{Descrizione}: Per valutare la complessità e la leggibilità di un testo, si utilizza l'Indice Gunning Fog. Questa metrica tiene conto della lunghezza delle parole e delle frasi all'interno del testo.
    \item \textbf{Formula}:
    \[ 0.4 \left( \frac{{\text{{numero totale di parole}}}}{{\text{{numero totale di frasi}}}} + 100 \left( \frac{{\text{{numero di parole con più di tre sillabe}}}}{{\text{{numero totale di parole}}}} \right) \right) \]
    \item \textbf{Caratteristica di riferimento}: Complessità e leggibilità del testo
    \item \textbf{Motivo}: L'Indice Gunning Fog restituisce un valore che indica il grado di difficoltà di lettura del testo, fondamentale per valutare la comprensibilità del contenuto da parte del pubblico.
    \item \textbf{Misurazione}: Utilizza la formula sopra indicata, dove il numero totale di parole, frasi e parole con più di tre sillabe è calcolato all'interno del testo.
    \item \textbf{Range di valori e interpretazione}: 
    \begin{itemize}
        \item Valori intorno a 6: testi con una leggibilità molto semplice, adatti a un vasto pubblico, incluso chi ha completato solo la scuola elementare.
        \item Valori compresi tra 7 e 9: testi con una leggibilità media, adatti a persone con un livello di istruzione medio, come chi ha completato la scuola media.
        \item Valori superiori a 10: testi con una leggibilità elevata, adatti a lettori con un alto livello di istruzione, come chi ha completato la scuola superiore o universitaria.
    \end{itemize}
\end{itemize}

\subsubsubsection{Vocaboli inseriti nel vocabolario ad ogni sprint}
\begin{itemize}
    \item \textbf{Notazione specifica}: M.4.3;
    \item \textbf{Nome}: Vocaboli inseriti nel vocabolario ad ogni sprint;
    \item \textbf{Descrizione}: Questa metrica misura il numero di nuove parole che vengono introdotte nel vocabolario del progetto ad ogni sprint. Il vocabolario del progetto include tutti i termini tecnici, acronimi e altri concetti specifici utilizzati nella documentazione e nelle comunicazioni del team;
    \item \textbf{Caratteristica di riferimento}: Comunicazione, comprensione del linguaggio tecnico e accessibilità;
    \item \textbf{Motivo}: Monitorare il tasso di crescita del vocabolario del progetto aiuta a valutare la complessità e l'evoluzione del linguaggio tecnico utilizzato all'interno del team e può evidenziare la necessità di formazione aggiuntiva o di chiarimenti sui concetti;
    \item \textbf{Misurazione}: La metrica misura il numero di nuove parole introdotte nel vocabolario secondo la formula:
    \[
        \text{Parole inserite nel vocabolario ad ogni sprint (\%)} =\frac{P_{a}}{P_{t}} \times 100 
    \]
    dove:
    \begin{itemize}
        \item $P_{a}$: Numero di vocaboli aggiunti nello sprint;
        \item $P_{t}$: Numero di vocaboli totali.
    \end{itemize}
    
    
    %\item \textbf{Range}: N/A (il valore ottimale dipende dalle esigenze specifiche del progetto e dalla sua complessità)
    %\item \textbf{Spiegazione}: Un aumento significativo nel numero di parole introdotte potrebbe indicare una crescente complessità del progetto o una comunicazione inefficiente, mentre una diminuzione potrebbe indicare una maggiore stabilità o un'efficace condivisione della conoscenza.
\end{itemize}





\subsection{Azioni correttive}
Le metriche elencate verranno utilizzate e valutate all'interno di un modello SPICE(g) al fine di promuovere un ciclo continuo di valutazione e miglioramento. Alla fine di ogni sprint, dove possibile in modo automatico o manualmente per alcune metriche, verranno valutate le aree di maggior difficoltà e quelle su cui è necessario migliorare. Successivamente, saranno implementate azioni correttive mirate per affrontare eventuali problemi riscontrati.

Ad esempio, nel caso delle metriche del questionario interno, verrà valutato il livello di conformità rispetto ai criteri prestabiliti. In caso di risultati inferiori alle aspettative, saranno adottate azioni correttive specifiche. Come ad esempio, se più di 5 attività vengono lasciate fuori dallo sprint (o una a testa per ogni ruolo), lo sprint non verrà concluso e saranno identificate le cause sottostanti per migliorare il processo di pianificazione e gestione delle attività.

