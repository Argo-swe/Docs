\section{Strumenti}
\par Nella seguente sezione sono elencati gli strumenti utilizzati dal team per lo svolgimento delle attività di progetto.

\subsection{Discord}
\par \href{https://discord.com}{https://discord.com} (Ultimo accesso: 2024-05-18)
\par Piattaforma utilizzata per lo svolgimento di riunioni formali ed informali del team, attraverso la creazione di un server apposito con un canale dedicato ai messaggi di testo e diverse sale virtuali dedicate alle conversazioni vocali.
    
\subsection{GitHub}
\par \href{https://github.com/}{https://github.com} (Ultimo accesso: 2024-05-18)
\par Piattaforma di hosting e collaborazione per lo sviluppo di software che offre strumenti per il controllo di versione, consentendo agli sviluppatori di lavorare insieme, monitorare le modifiche al codice, gestire problemi e pubblicare il proprio lavoro in modo collaborativo.
    
\subsection{Google Calendar}
\par \href{https://calendar.google.com}{https://calendar.google.com} (Ultimo accesso: 2024-05-18)
\par Applicazione di calendario basata su cloud sviluppata da Google che consente agli utenti di organizzare gli eventi, le riunioni e le attività quotidiane, sincronizzando automaticamente le informazioni su tutti i dispositivi connessi e facilitando la condivisione di calendari con altri utenti.
    
\subsection{Google Gmail}
\par \href{https://mail.google.com}{https://mail.google.com} (Ultimo accesso: 2024-05-18)
\par Servizio di posta elettronica sviluppato da Google che offre una piattaforma di email, con funzionalità avanzate di organizzazione.
    
\subsection{Jira}
\par \href{https://www.atlassian.com/it/software/jira}{https://www.atlassian.com/it/software/jira} (Ultimo accesso: 2024-05-18)
\par Software utilizzato per l'ITS e gli strumenti di organizzazione del lavoro tra i membri del team.
    
\subsection{Telegram}
\par \href{https://www.telegram.com}{https://www.telegram.com} (Ultimo accesso: 2024-05-18)
\par Applicazione di messaggistica, che rende disponibile le Community, una funzionalità dedicata ai gruppi che permette di creare sottogruppi tematici all'interno di un gruppo principale, facilitando la comunicazione e l'organizzazione all'interno di un team.
    
\subsection{Docker}
\par Piattaforma software utilizzata per sviluppare, distribuire e gestire applicazioni in contenitori, permettendo di creare un ambiente isolato per l'esecuzione di processi.
    
\subsection{LaTeX}
\par Linguaggio di markup utilizzato per la preparazione di documenti tecnici e scientifici, noto per la sua capacità di gestire la struttura e la formattazione dei documenti in modo flessibile e professionale.
    
\subsection{Latexmk (XeLaTeX)}
\par \href{https://pypi.org/project/latexmk.py/}{https://pypi.org/project/latexmk.py} (Ultimo accesso: 2024-05-18)
\par Strumento di automazione utilizzato per semplificare il processo di compilazione dei documenti scritti in LaTeX utilizzando il motore di composizione XeLaTeX. Automatizza la sequenza di passaggi necessari per produrre un documento finale, gestendo automaticamente le dipendenze e garantendo che la compilazione avvenga in modo efficiente.
    
\subsection{Visual Studio Code}
\par Editor di codice sorgente sviluppato da Microsoft che offre funzionalità avanzate per lo sviluppo e la gestione di progetti software.
