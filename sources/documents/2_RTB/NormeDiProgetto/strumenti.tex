\section{Strumenti}\label{strumenti}
\par Nella seguente sezione sono elencati gli strumenti utilizzati dal team per lo svolgimento delle attività di progetto.

\subsection{Canva}
\par \href{https://www.canva.com}{https://www.canva.com} (Ultimo accesso: 2024-07-14)
\par Strumento di progettazione grafica utilizzato per la creazione di loghi e immagini.

\subsection{chatLMsys}
\par \href{https://chat.lmsys.org/}{https://chat.lmsys.org/} (Ultimo accesso: 2024-07-24)
\par Sito web per benchmarking di \glossario{LLM}.

\subsection{Colour Contrast Analyzer}
\par \href{https://www.tpgi.com/color-contrast-checker}{https://www.tpgi.com/color-contrast-checker} (Ultimo accesso: 2024-07-14)
\par Strumento utilizzato per la scelta della palette di colori, in conformità con le linee guida WCAG 2.1 sull'accessibilità del rapporto di contrasto.

\subsection{Diagrams.net}
\par \href{https://app.diagrams.net}{https://app.diagrams.net} (Ultimo accesso: 2024-07-14)
\par Strumento utilizzato per la creazione dei diagrammi dei casi d'uso, di attività e di sequenza in UML.

\subsection{Discord}
\par \href{https://discord.com}{https://discord.com} (Ultimo accesso: 2024-07-18)
\par Piattaforma utilizzata per lo svolgimento di riunioni formali e informali. Il team ha configurato un server con un canale dedicato ai messaggi di testo e diverse sale virtuali riservate alle conversazioni vocali.
    
\subsection{Docker}
\par \href{https://www.docker.com}{https://www.docker.com} (Ultimo accesso: 2024-05-18)
\par Piattaforma software utilizzata per sviluppare, testare e distribuire applicazioni in contenitori, consentendo di creare un ambiente isolato per l'esecuzione di processi.

\subsection{Git}
\par \href{https://git-scm.org}{https//git-scm.org} (Ultimo accesso: 2024-07-18)
\par \glossario{Version Control System} distribuito e open source.

\subsection{GitHub}
\par \href{https://github.com}{https://github.com} (Ultimo accesso: 2024-07-18)
\par Piattaforma di hosting per lo sviluppo software collaborativo. GitHub offre strumenti per il controllo di versione, tracciamento delle attività, verifica del codice, integrazione delle modifiche e distribuzione del codice.
    
\subsection{Google Calendar}
\par \href{https://calendar.google.com}{https://calendar.google.com} (Ultimo accesso: 2024-07-18)
\par Applicazione utilizzata per organizzare gli eventi e le riunioni formali, sincronizzando in automatico le informazioni su tutti i dispositivi connessi.
    
\subsection{Google Docs}
\par \href{https://docs.google.com/document}{https://docs.google.com/document} (Ultimo accesso: 2024-07-18)
\par Applicazione utilizzata per elaborare documenti online collaborando con altri utenti in tempo reale.

\subsection{Google Drawings}
\par \href{https://docs.google.com/drawings}{https://docs.google.com/drawings} (Ultimo accesso: 2024-07-18)
\par Software utilizzato per la creazione di diagrammi e disegni di natura tecnica.

\subsection{Google Forms}
\par \href{https://docs.google.com/forms}{https://docs.google.com/forms} (Ultimo accesso: 2024-07-18)
\par Applicazione utilizzata per la creazione di questionari di valutazione degli sprint in combinazione con Google Sheets.

\subsection{Google Gmail}
\par \href{https://mail.google.com}{https://mail.google.com} (Ultimo accesso: 2024-07-18)
\par Servizio di posta elettronica utilizzato per la comunicazione via email con i Committenti e il Proponente.

\subsection{Google Sheets}
\par \href{https://docs.google.com/spreadsheets}{https://docs.google.com/spreadsheets} (Ultimo accesso: 2024-07-18)
\par Applicazione utilizzata per creare e gestire fogli di calcolo online, agevolando la collaborazione in tempo reale e la generazione di grafici.

\subsection{Google Slides}
\par \href{https://docs.google.com/presentation}{https://docs.google.com/presentation} (Ultimo accesso: 2024-07-18)
\par Applicazione utilizzata per la creazione delle slide per i diari di bordo e per le presentazioni relative alle revisioni di avanzamento.
    
\subsection{Jira}
\par \href{https://www.atlassian.com/it/software/jira}{https://www.atlassian.com/it/software/jira} (Ultimo accesso: 2024-07-18)
\par Software utilizzato per il monitoraggio delle attività e la gestione dei progetti sviluppati con metodologie agili.
    
\subsection{LaTeX}
\par \href{https://www.latex-project.org}{https://www.latex-project.org} (Ultimo accesso: 2024-05-18)
\par Linguaggio di markup utilizzato per la stesura di documenti tecnici e scientifici; LaTeX garantisce una struttura e una formattazione flessibili e professionali.
    
\subsection{Latexmk}
\par \href{https://pypi.org/project/latexmk.py/}{https://pypi.org/project/latexmk.py} (Ultimo accesso: 2024-05-18)
\par Strumento utilizzato per automatizzare il processo di compilazione di un documento scritto in LaTeX. Latexmk deriva dall'utilità generica “make” ed è in grado di determinare in automatico le dipendenze. Inoltre, risolve i riferimenti incrociati ed esegue nuovamente LaTeX ogni volta che un file sorgente viene aggiornato.

\subsection{PDF24 Tools}
\par \href{https://tools.pdf24.org/it/svg-in-pdf}{https://tools.pdf24.org/it/svg-in-pdf} (Ultimo accesso: 2024-07-24)
\par Strumento utilizzato per convertire file SVG in PDF.

\subsection{Slack}
\par \href{https://slack.com}{https://slack.com} (Ultimo accesso: 2024-07-18)
\par Software utilizzato in combinazione con GitHub per inviare promemoria a intervalli di tempo regolari. I promemoria sono indirizzati ai verificatori a cui sono state assegnate pull request pronte per la revisione.

\subsection{StarUML}
\par \href{https://staruml.io}{https://staruml.io} (Ultimo accesso: 2024-07-18)
\par Software utilizzato per la creazione dei diagrammi delle classi in UML.

\subsection{Table Convert Online}
\par \href{https://tableconvert.com/it/csv-to-latex}{https://tableconvert.com/it/csv-to-latex} (Ultimo accesso: 2024-07-24)
\par Strumento utilizzato per convertire file \glossario{csv} in \glossario{LaTeX}.

\subsection{Telegram}
\par \href{https://web.telegram.org}{https://web.telegram.org} (Ultimo accesso: 2024-07-18)
\par Applicazione di messaggistica utilizzata per creare gruppi tematici, semplificando la comunicazione e l'organizzazione all'interno del team.
    
\subsection{Visual Studio Code}
\par \href{https://code.visualstudio.com}{https://code.visualstudio.com} (Ultimo accesso: 2024-05-18)
\par Editor di codice sorgente utilizzato per lo sviluppo software. Visual Studio Code offre un’ampia gamma di funzionalità, tra cui l'integrazione con Git e Copilot, strumenti di debug ed estensioni che permettono di personalizzare l’ambiente di sviluppo.
