\section{Introduzione}

\subsection{Scopo del documento}
Il presente documento si propone di offrire una trattazione esaustiva dei casi d'uso e dei \glossario{requisiti} del progetto "ChatSQL", seguendo gli \glossario{standard di qualità} dell'ingegneria del software. Tale analisi è stata condotta attraverso un'approfondita valutazione del \glossario{capitolato} C9 presentato dalla Proponente(G) Zucchetti S.p.A., integrata da una stretta collaborazione durante gli incontri dedicati.
L'obiettivo principale è definire con precisione le caratteristiche e le necessità della web application proposta, la quale verrà definita propriamente all'interno della sezione 2)[qua ci mettiamo un link].

\subsection{Scopo del prodotto}
L’utilizzo di strumenti di facilitazione basati sull’utilizzo di \glossario{Intelligenza Artificiale} si sta diffondendo in maniera esponenziale, e anche gli utenti non facenti parte dell’area IT (\glossario{Information Technology}) li utilizzano con sempre maggior frequenza per velocizzare le attività quotidiane. L’azienda Zucchetti ha proposto lo sviluppo di un’applicazione che tramite l’utilizzo di LLM (\glossario{Large Language Model}) consente la generazione di \glossario{prompt} ottimizzati all’interrogazione di modelli dati basati su \glossario{database SQL}. I \glossario{prompt} potranno essere poi proposti a servizi di LLM come ChatGPT per la generazione della \glossario{query} di interrogazione da eseguire sul database.
L'applicazione sarà implementata come una web application accessibile attraverso i principali browser come Chrome, Firefox, Edge e Safari. Gli utenti di tipo operatore potranno caricare, a seguito di autenticazione, nell’applicativo i modelli dati da poter utilizzare per la generazione dei \glossario{prompt}. Gli utenti finali potranno chiedere tramite linguaggio naturale la generazione del \glossario{prompt} basandosi su uno specifico modello di dati. Il \glossario{prompt} generato e passato ad un servizio di LLM genererà la \glossario{query} da poter eseguire.

\subsection{Glossario}
\GlossarioIntroduzione

\subsection{Riferimenti}
\subsubsection{Riferimenti normativi}
\begin{itemize}
  \item Capitolato C9 - ChatSQL:\\ \href{https://www.math.unipd.it/~tullio/IS-1/2023/Progetto/C9.pdf}{https://www.math.unipd.it/\~tullio/IS-1/2023/Progetto/C9.pdf} \\ \href{https://www.math.unipd.it/~tullio/IS-1/2023/Progetto/C9.pdf}{https://www.math.unipd.it/\~tullio/IS-1/2023/Progetto/C9p.pdf}
  \item Norme di Progetto:\\ \href{https://github.com/Argo-swe/argo-swe.github.io/blob/main/2_RTB/NormeDiProgetto.pdf}{https://github.com/Argo-swe/argo-swe.github.io/blob/main/2\_RTB/NormeDiProgetto.pdf}
  \item Verbale esterno 11-03-2024:\\ \href{https://github.com/Argo-swe/argo-swe.github.io/blob/main/1_Candidatura/Verbali/Interni/VerbaleInterno_2024-03-09.pdf}{https://github.com/Argo-swe/argo-swe.github.io/blob/main/1\_Candidatura/Verbali/Interni/VerbaleInterno\_2024-03-09.pdf}
  \item Regolamento del Progetto Didattico:\\ \href{https://www.math.unipd.it/~tullio/IS-1/2023/Dispense/PD2.pdf}{https://www.math.unipd.it/~tullio/IS-1/2023/Dispense/PD2.pdf}
\end{itemize}

\subsubsection{Riferimenti informativi}
\begin{itemize}
  \item Dispense dal corso Ingegneria del Software: T5 - Analisi dei Requisiti:\\ \href{https://www.math.unipd.it/~tullio/IS-1/2023/Dispense/T5.pdf}{https://www.math.unipd.it/\~tullio/IS-1/2023/Dispense/T5.pdf}
  \item Dispense dal corso Ingegneria del Software: P2 - Diagrammi Use Case:\\ \href{https://www.math.unipd.it/~rcardin/swea/2022/Diagrammi%20Use%20Case.pdf}{https://www.math.unipd.it/\~rcardin/swea/2022/Diagrammi\_Use\_Case.pdf}
\end{itemize}
