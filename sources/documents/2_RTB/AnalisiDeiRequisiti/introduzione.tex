\section{Introduzione}

\subsection{Scopo del documento}
\par Il presente documento si propone di offrire una trattazione esaustiva dei \glossario{casi d'uso} e dei requisiti del progetto ChatSQL, seguendo gli standard e le convenzioni dell'ingegneria del software. Tale analisi è stata condotta attraverso una valutazione del capitolato C9 presentato dalla \glossario{Proponente} Zucchetti S.p.A., integrata da una collaborazione durante gli incontri dedicati. L'obiettivo principale è definire le funzionalità e le caratteristiche dell’applicativo, fornendo una base per la progettazione del sistema.

\subsection{Scopo del prodotto}
\par L’utilizzo di strumenti di supporto basati sull’\glossario{Intelligenza Artificiale} si sta diffondendo su larga scala, e anche gli utenti non facenti parte dell’area IT (\glossario{Information Technology}) ne dispongono con maggior frequenza per semplificare le attività quotidiane. L’azienda Zucchetti S.p.A. ha proposto lo sviluppo di un’applicazione che consenta la generazione di \glossario{prompt} mirati all’interrogazione di LLM (\glossario{Large Language Model}). L'applicazione sarà implementata come una web application accessibile attraverso i principali browser come Chrome, Firefox, Edge e Safari. Gli operatori potranno caricare nell’applicativo, a seguito di autenticazione, i \glossario{dizionari dati} da utilizzare per la ricerca semantica. Una volta selezionato un dizionario dati, gli utenti potranno formulare richieste in linguaggio naturale e ricevere prompt da copiare su ChatGPT (o affini), al fine di ottenere in output una \glossario{query} SQL.

\subsection{Glossario}
\GlossarioIntroduzione

\subsection{Riferimenti}
\subsubsection{Riferimenti normativi}
\begin{itemize}
  \item Capitolato C9 - ChatSQL:\\ \href{https://www.math.unipd.it/~tullio/IS-1/2023/Progetto/C9.pdf}{https://www.math.unipd.it/\~tullio/IS-1/2023/Progetto/C9.pdf} \\ \href{https://www.math.unipd.it/~tullio/IS-1/2023/Progetto/C9.pdf}{https://www.math.unipd.it/\~tullio/IS-1/2023/Progetto/C9p.pdf};
  \item Norme di Progetto:\\ \href{https://github.com/Argo-swe/argo-swe.github.io/blob/main/2_RTB/NormeDiProgetto.pdf}{https://github.com/Argo-swe/argo-swe.github.io/blob/main/2\_RTB/\-NormeDiProgetto.pdf};
  \item Verbale esterno 11-03-2024:\\ \href{https://github.com/Argo-swe/argo-swe.github.io/blob/main/1_Candidatura/Verbali/Interni/VerbaleInterno_2024-03-09.pdf}{https://github.com/Argo-swe/argo-swe.github.io/blob/main/1\_Candidatura/ \\ Verbali/Interni/VerbaleInterno\_2024-03-09.pdf};
  \item Regolamento del Progetto Didattico:\\ \href{https://www.math.unipd.it/~tullio/IS-1/2023/Dispense/PD2.pdf}{https://www.math.unipd.it/~tullio/IS-1/2023/Dispense/PD2.pdf};
  \item Linee guida per l'accessibilità (WCAG 2.1):\\ \href{https://www.w3.org/Translations/WCAG21-it}{https://www.w3.org/Translations/WCAG21-it}.
\end{itemize}

\subsubsection{Riferimenti informativi}
\begin{itemize}
  \item Dispense dal corso Ingegneria del Software: T5 - Analisi dei Requisiti:\\ \href{https://www.math.unipd.it/~tullio/IS-1/2023/Dispense/T5.pdf}{https://www.math.unipd.it/\~tullio/IS-1/2023/Dispense/T5.pdf}  \\ (Ultimo accesso: 2024-05-20);
  \item Dispense dal corso Ingegneria del Software: P2 - Diagrammi Use Case:\\ \href{https://www.math.unipd.it/~rcardin/swea/2022/Diagrammi%20Use%20Case.pdf}{https://www.math.unipd.it/\~rcardin/swea/2022/Diagrammi\_Use\_Case.pdf}  \\ (Ultimo accesso: 2024-06-21);
  \item Analisi dei requisiti e casi d'uso:\\ \href{https://www.cs.unibo.it/~cianca/wwwpages/ids/6.pdf}{https://www.cs.unibo.it/~cianca/wwwpages/ids/6.pdf}  \\ (Ultimo accesso: 2024-06-19);
  \item Strutturare i casi d'uso:\\ \href{https://guides.visual-paradigm.com/structuring-use-cases-with-base-include-and-extend-a-guide-for-effective-software-development}{https://guides.visual-paradigm.com/structuring-use-cases-with-base-include-and-extend-a-guide-for-effective-software-development}  \\ (Ultimo accesso: 2024-06-19);
  \item Casi d'uso - FAQ:\\ \href{https://analisi-disegno.com/usecases/usecases-faq}{https://analisi-disegno.com/usecases/usecases-faq/}  \\ (Ultimo accesso: 2024-06-18);
  \item Extension points:\\ \href{https://www.cs.unibo.it/gabbri/MaterialeCorsi/2.usecasediagramUML.favini.pdf}{https://www.cs.unibo.it/gabbri/MaterialeCorsi/2.usecasediagramUML.favini.pdf}  \\ (Ultimo accesso: 2024-06-19);
  \item Relazione di generalizzazione:\\ \href{https://www.geeksforgeeks.org/use-case-diagram}{https://www.geeksforgeeks.org/use-case-diagram}  \\ (Ultimo accesso: 2024-06-21);
  \item Esempi di diagrammi dei casi d'uso:\\ \href{https://www.uml-diagrams.org/use-case-diagrams-examples.html}{https://www.uml-diagrams.org/use-case-diagrams-examples.html}  \\ (Ultimo accesso: 2024-06-21).
\end{itemize}
