\section{Casi d'uso}

\subsection{Scopo}
Questa sezione descrive nel dettaglio i Casi d’Uso individuati dal gruppo durante l'analisi.

\subsection{Attori}
Il prodotto prevede due attori:
\begin{itemize}
  \item Utente: un utente generico che ha accesso alle funzionalità richieste per la generazione di un \glossario{prompt} utilizzando i dizionari dati precaricati;
  \item Tecnico: un utente che ha completato con successo l'autenticazione ed è registrato come Tecnico, ha accesso a tutte le funzionalità dell'utente generico e in aggiunta può gestire i dizionari dati ed accedere a funzionalità di testing e debug.
\end{itemize}

\subsection{Gestione degli errori}
Alcuni dei casi d'uso gestiscono situazioni di errore, questo permette ai programmatori di gestire eventuali situazioni eccezionali, in modo che vengano intercettate e non portino ad un malfunzionamento del sistema, ma anzi gestite e notificate quando questo è utile.

\subsection{Elenco dei \glossario{casi d'uso}}

\subsubsection{UC1 - Autenticazione}\label{UC1}

\begin{figure}[H]
  \centering
  \includegraphics[width=0.90\textwidth]{assets/uc1.png}
  \caption{UC1}
\end{figure}

\paragraph*{Descrizione}
L'autenticazione corrisponde al processo di login, tramite il quale l'Utente può passare alla schermata del Tecnico e ampliare le funzionalità disponibili.

\paragraph*{Attori principali}
Utente

\paragraph*{Precondizioni}
\begin{itemize}
  \item Il sistema è attivo e funzionante;
  \item L'Utente non è autenticato nella sessione corrente.
\end{itemize}

\paragraph*{Postcondizioni}
\begin{itemize}
  \item La procedura di autenticazione si è conclusa con successo;
  \item L'Utente acquisisce il ruolo di Tecnico;
  \item Il Tecnico visualizza le funzionalità aggiuntive nell'interfaccia.  
\end{itemize}

\paragraph*{Trigger}
L'Utente vuole effettuare il login all'area riservata al Tecnico.

\paragraph*{Scenario principale}
\begin{itemize}
  \item L'Utente accede al sistema;
  \item L'Utente seleziona la funzionalità "Login";
  \item L'Utente inserisce le proprie credenziali di accesso;
  \item L'utente accede alla schermata propria del Tecnico.
\end{itemize}

\paragraph*{Scenario alternativo}
\begin{enumerate}
  \item Il sistema riconosce un errore durante l'inserimento delle credenziali da parte dell'Utente (\hyperref[UC2]{UC2});
  \item Viene visualizzato un messaggio con i dettagli dell'errore.
\end{enumerate}

\paragraph*{Estensioni}
\begin{itemize}
  \item Errore per fallita autenticazione (\hyperref[UC2]{UC2}).
\end{itemize}

\paragraph*{Inclusioni}
\begin{itemize}
  \item Inserimento e-mail (\hyperref[UC1point1]{UC1.1});
  \item Inserimento password (\hyperref[UC1point2]{UC1.2}).
\end{itemize}

%%%%%%%%%%%%%%%%%%%%%%%%%%%%%%%%%%%%%%%%%%%%%%%%%%%%%%%%%%%%%%%%%%%%%%%%%%%%%%

\subsubsection{UC1.1 - Inserimento username}\label{UC1point1}

\paragraph*{Descrizione}
La procedura di "inserimento username" corrisponde all'immissione del nome utente nella sezione apposita di login.

\paragraph*{Attori principali}
Utente

\paragraph*{Precondizioni}
\begin{itemize}
  \item Il sistema è attivo e funzionante;
  \item L'Utente ha avviato la procedura di autenticazione (\hyperref[UC1]{UC1}).  
\end{itemize}

\paragraph*{Postcondizioni}
\begin{itemize}
  \item Lo username è stato correttamente inserito nel campo apposito.
\end{itemize}

\paragraph*{Trigger}
L'Utente vuole inserire il proprio username per accedere alla sezione del Tecnico.

\paragraph*{Scenario principale}
\begin{enumerate}
  \item L'utente inserisce il proprio username come parte del processo di autenticazione.
\end{enumerate}

%%%%%%%%%%%%%%%%%%%%%%%%%%%%%%%%%%%%%%%%%%%%%%%%%%%%%%%%%%%%%%%%%%%%%%%%%%%%%%

\subsubsection{UC1.2 - Inserimento password}\label{UC1point2}
\paragraph*{Descrizione}
La procedura di "inserimento password" corrisponde all'immissione della password nella sezione apposita di login.

\paragraph*{Attori principali}
Utente

\paragraph*{Precondizioni}
\begin{itemize}
  \item Il sistema è attivo e funzionante;
  \item L'Utente ha avviato la procedura di autenticazione (\hyperref[UC1]{UC1}). 
\end{itemize}

\paragraph*{Postcondizioni}
\begin{itemize}
  \item La password è stata correttamente inserita nel campo apposito.
\end{itemize}

\paragraph*{Trigger}
L'Utente vuole inserire la propria password per accedere alla sezione del Tecnico.

\paragraph*{Scenario principale}
\begin{enumerate}
  \item L'utente inserisce la propria password come parte del processo di autenticazione.
\end{enumerate}

\subsubsection{UC2 - Visualizzazione errore autenticazione}\label{UC2}

\paragraph*{Descrizione}
Nel caso in cui il sistema rilevi delle irregolarità durante la validazione delle credenziali, il Tecnico viene informato della natura del problema tramite un apposito messaggio di errore.

\paragraph*{Attori principali}
Tecnico

\paragraph*{Precondizioni}
\begin{itemize}
  \item Il Tecnico ha inserito le proprie credenziali nell'area di login (\hyperref[UC1]{UC1});
  \item Le credenziali non sono valide.
\end{itemize}

\paragraph*{Postcondizioni}
\begin{itemize}
  \item Viene visualizzato un messaggio d'errore esplicativo;
  \item Il Tecnico non ottiene l'accesso.
\end{itemize}

\paragraph*{Scenario principale}
\begin{enumerate}
  \item Il sistema riscontra un problema nella validazione delle credenziali di accesso;
  \item Viene restituito un messaggio di errore;
  \item Il sistema invita il Tecnico a riprovare l'autenticazione.
\end{enumerate}
\subsubsection{UC3 - Inserimento richiesta in linguaggio naturale}\label{UC3}
\paragraph*{Descrizione}
L’Utente desidera inserire una richiesta in qualsiasi linguaggio naturale al fine di ottenere il \glossario{prompt} che selezioni la parte del \glossario{dizionario dati} più inerente alla richiesta. 

\paragraph*{Attori principali}
Utente

\paragraph*{Precondizioni}
\begin{itemize}
  \item L'applicazione è stata avviata con successo;
  \item L’Utente ha in precedenza selezionato un \glossario{dizionario dati} (\hyperref[UC4]{UC4}); 
  \item L'Utente seleziona la lingua di inserimento, se non selezionata viene utilizzato l'italiano di default (\hyperref[UC7]{UC7}).
\end{itemize}

\paragraph*{Postcondizioni}
\begin{itemize}
  \item L’Utente ha scritto nell’apposito campo di testo un'interrogazione in linguaggio naturale.
\end{itemize}

\paragraph*{Scenario principale}
\begin{enumerate}
  \item L’Utente scrive un'interrogazione nell’apposito box su cui poi il sistema potrà produrre un \glossario{prompt} in seguito.
\end{enumerate}

\subsubsection{UC4 - Selezione \glossario{dizionario dati}}\label{UC4}

\begin{figure}[H]
  \centering
  \includegraphics[width=0.90\textwidth]{assets/uc4.png}
  \caption{UC4}
\end{figure}

\paragraph*{Descrizione}
L’Utente desidera selezionare il \glossario{dizionario dati} sul quale basare in seguito l’interrogazione in linguaggio naturale.

\paragraph*{Attori principali}
Utente

\paragraph*{Precondizioni}
\begin{itemize}
  \item L'applicazione è stata avviata con successo;
  \item Nell'applicazione web è stato caricato precedentemente almeno un \glossario{dizionario dati} (\hyperref[UC13]{UC13}).
\end{itemize}

\paragraph*{Postcondizioni}
\begin{itemize}
  \item Un \glossario{dizionario dati} è stato selezionato in modo corretto ed univoco.
\end{itemize}

\paragraph*{Scenario principale}
\begin{enumerate}
  \item L’Utente visualizza la lista dei dizionari disponibili (\hyperref[UC10]{UC10});
  \item L'Utente sceglie un \glossario{dizionario dati} tra quelli presenti.
\end{enumerate}

\subsubsection{UC5 - Generazione \glossario{prompt}}\label{UC5}
\paragraph*{Descrizione}
L’Utente desidera ottenere un \glossario{prompt} al fine di utilizzarlo poi con LLM esterni per generare \glossario{\glossario{query}} sql fornendo parti ristrette di \glossario{dizionario dati}.

\paragraph*{Attori principali}
Utente

\paragraph*{Precondizioni}
\begin{itemize}
  \item L'applicazione è stata avviata con successo;
  \item È presente almeno un \glossario{dizionario dati} nel sistema.
\end{itemize}

\paragraph*{Postcondizioni}
\begin{itemize}
  \item Il sistema genera un \glossario{prompt} in base alla richiesta ricevuta e al \glossario{dizionario dati} scelto. Attraverso dei metadati restituisce un \glossario{prompt} con il quale stabilisce quali parti di database sono maggiormente necessarie ed efficienti per scrivere la richiesta ricevuta in SQL;
  \item L’Utente riceve il \glossario{prompt} generato.
\end{itemize}

\paragraph*{Scenario principale}
\begin{enumerate}
  \item L’Utente desidera ottenere il \glossario{prompt} che permette di generare la \glossario{query} SQL;
  \item L’Utente seleziona un \glossario{dizionario dati} sul quale baserà l’interrogazione;
  \item L’Utente sceglie la lingua su cui farà l’interrogazione.La lingua è in italiano di default se non viene cambiata;
  \item L’Utente inserisce una interrogazione in linguaggio naturale nel box testuale apposito;
  \item Inizia il processo di generazione del \glossario{prompt} e di raccolta di metadati.
\end{enumerate}

\paragraph*{Scenario alternativo}
\begin{enumerate}
  \item Errore nella generazione del \glossario{prompt} (\hyperref[UC6]{UC6}).
\end{enumerate}

\paragraph*{Inclusione}
\begin{itemize}
  \item Inserimento richiesta in linguaggio naturale (\hyperref[UC3]{UC3});
  \item Selezione \glossario{dizionario dati} (\hyperref[UC4]{UC4}).
\end{itemize}

\paragraph*{Estensione}
\begin{itemize}
  \item Cambio lingua (\hyperref[UC7]{UC7});
  \item Selezione e copia del \glossario{prompt} risultato (\hyperref[UC8]{UC8}).
\end{itemize}

\subsubsection{UC6 - Messaggio d'errore nella generazione del \glossario{prompt}}\label{UC6}
\paragraph*{Descrizione}
L’Utente ha cercato di generare un \glossario{prompt} con una frase non totalmente inerente al \glossario{dizionario dati} scelto.

\paragraph*{Attori principali}
Utente

\paragraph*{Precondizioni}
\begin{itemize}
  \item L'applicazione è stata avviata con successo;
  \item L’Utente ha inserito una frase in linguaggio naturale (\hyperref[UC3]{UC3});
  \item L’Utente ha richiesto la generazione del \glossario{prompt} a partire dalla frase inserita.  
\end{itemize}

\paragraph*{Postcondizioni}
\begin{itemize}
  \item L’applicazione ha riscontrato problemi nella generazione del \glossario{prompt} a causa di una non adeguata aderenza tra la frase inserita e il \glossario{dizionario dati} scelto;
  \item Viene visualizzato un messaggio di errore per orientare l’Utente alla comprensione del funzionamento del sistema.
\end{itemize}

\paragraph*{Scenario principale}
\begin{enumerate}
  \item L’Utente ha inserito una frase in linguaggio naturale (\hyperref[UC3]{UC3});
  \item L'Utente ha richiesto la generazione del \glossario{prompt} (\hyperref[UC5]{UC5});
  \item Il sistema attua il meccanismo per identificare la parte di \glossario{dizionario dati} adeguata da restituire;
  \item Il sistema non è in grado di trovare le correlazioni adeguate e quindi non riesce a restituire un \glossario{prompt} coerente;
  \item Viene visualizzato un messaggio di errore.
\end{enumerate}

\subsubsection{UC7 - Cambio lingua}\label{UC7}

\begin{figure}[H]
  \centering
  \includegraphics[width=0.90\textwidth]{assets/uc7.png}
  \caption{UC7}
\end{figure}

\paragraph*{Descrizione}
L’Utente desidera inserire una richiesta in linguaggio naturale in una lingua diversa dall’italiano.

\paragraph*{Attori principali}
Utente

\paragraph*{Precondizioni}
\begin{itemize}
  \item L'applicazione è stata avviata con successo;
  \item L'Utente ha selezionato un \glossario{dizionario dati} (\hyperref[UC4]{UC4});
  \item L’Utente vuole inserire una frase in linguaggio naturale in un'altra lingua.
\end{itemize}

\paragraph*{Postcondizioni}
\begin{itemize}
  \item L'Utente ha selezionato una lingua tra quelle disponibili;
  \item L'Utente può ora inserire una richiesta in linguaggio naturale nella lingua scelta (\hyperref[UC3]{UC3}).
\end{itemize}

\paragraph*{Scenario principale}
\begin{enumerate}
  \item L'Utente seleziona un dizionario dati (\hyperref[UC4]{UC4});
  \item L’Utente seleziona una lingua da una lista predefinita di lingue. La scelta è tra le seguenti lingue:
    \begin{itemize}
      \item Italiano;
      \item Inglese;
      \item Francese;
      \item Spagnolo;
      \item Tedesco.
    \end{itemize}
\end{enumerate}

\paragraph*{Inclusioni}
\begin{itemize}
  \item Selezione \glossario{dizionario dati} (\hyperref[UC4]{UC4}).
\end{itemize}

% TODO [in caso di errore ed incongruenza tra la lingua scritta e quella selezionata teniamo solo errore uc6 del \glossario{prompt}? O viene controllata prima?]

\subsubsection{UC8 - Selezione e copia del prompt generato}\label{UC8}

\paragraph*{Descrizione}
L’Utente vuole selezionare il prompt che è stato generato dal sistema per poterlo usare all’interno della richiesta ad un LLM esterno al fine di generare una query.

\paragraph*{Attori principali}
Utente

\paragraph*{Precondizioni}
\begin{itemize}
  \item L'applicazione è stata avviata con successo;
  \item L’Utente ha generato un prompt con successo a partire da una richiesta in linguaggio naturale.  
\end{itemize}

\paragraph*{Postcondizioni}
\begin{itemize}
  \item Il prompt che è stato generato dal sistema come risultato dell’interrogazione è copiato negli appunti del sistema dell’Utente.
\end{itemize}

\paragraph*{Scenario principale}
\begin{enumerate}
  \item Dopo aver visualizzato il prompt di risposta del sistema l’Utente seleziona la funzione di copia;
  \item L’Utente salva nei suoi appunti di sistema una copia del prompt che potrà poi incollare su LLM esterni per la generazione della query.
\end{enumerate}
\subsubsection{UC9 - Visualizzazione lista dizionari dati caricati nel sistema}\label{UC9}

\begin{figure}[H]
  \centering
  \includegraphics[width=0.90\textwidth]{assets/uc9.png}
  \caption{UC9}
\end{figure}

\paragraph*{Descrizione}
L'Utente visualizza la lista dei \glossario{dizionari dati} che sono stati caricati nel sistema.

\paragraph*{Attori principali}
Utente

\paragraph*{Precondizioni}
\begin{itemize}
  \item Il sistema è attivo e funzionante;
  \item È stato caricato almeno un \glossario{dizionario dati} nel sistema.  
\end{itemize}

\paragraph*{Postcondizioni}
\begin{itemize}
\item Viene visualizzata correttamente la lista dei \glossario{dizionari dati} presenti nel sistema.
\end{itemize}

\paragraph*{Trigger}
L'Utente desidera visualizzare i \glossario{dizionari dati} disponibili.

\paragraph*{Scenario principale}
\begin{enumerate}
  \item L'Utente visualizza la lista dei \glossario{dizionari dati}.
\end{enumerate}

\begin{figure}[H]
  \centering
  \includegraphics[width=0.90\textwidth]{assets/uc9_1.png}
  \caption{UC9 - Sottocaso d'uso}
\end{figure}

% Da aggiungere forse un UC per il fallimento di visualizzazione della lista

\subsubsection{UC9.1 - Visualizzazione singolo dizionario dati}\label{UC9point1}
\paragraph*{Descrizione}
L'Utente visualizza un singolo dizionario all'interno della lista dei \glossario{dizionari dati} caricati nel sistema.

\paragraph*{Attori principali}
Utente

\paragraph*{Precondizioni}
\begin{itemize}
  \item Il sistema è attivo e funzionante;
  \item È stato caricato almeno un \glossario{dizionario dati} nel sistema;
  \item È visibile la lista dei \glossario{dizionari dati} (\hyperref[UC9]{UC9}).
\end{itemize}

\paragraph*{Postcondizioni}
\begin{itemize}
  \item L'Utente visualizza correttamente un singolo \glossario{dizionario dati}.
\end{itemize}

\paragraph*{Trigger}
L'Utente vuole visualizzare un singolo \glossario{dizionari dati} presente nella lista.

\paragraph*{Scenario principale}
\begin{enumerate}
  \item L'Utente visualizza un singolo \glossario{dizionario dati}.
\end{enumerate}

\paragraph*{Inclusioni}
\begin{itemize}
  \item Visualizzazione caratteristiche \glossario{dizionario dati} (\hyperref[UC10]{UC10});
\end{itemize}

\subsubsection{UC10 - Visualizzazione lista \glossario{dizionario dati}}\label{UC10}

\begin{figure}[H]
  \centering
  \includegraphics[width=0.90\textwidth]{assets/uc10.png}
  \caption{UC10}
\end{figure}

\paragraph*{Descrizione}
L’Utente visualizza la lista dei \glossario{dizionari dati} che sono stati caricati nel sistema.

\paragraph*{Attori principali}
Utente

\paragraph*{Precondizioni}
\begin{itemize}
  \item È presente almeno un \glossario{dizionario dati} nel sistema.  
\end{itemize}

\paragraph*{Postcondizioni}
\begin{itemize}
  \item L’Utente visualizza la lista completa dei \glossario{dizionari dati} nel sistema.
\end{itemize}

\paragraph*{Scenario principale}
\begin{enumerate}
  \item L’Utente decide di visualizzare i \glossario{dizionari dati} caricati;
  \item Viene esposta la lista dei \glossario{dizionari dati}.
\end{enumerate}

\paragraph*{Inclusioni}
\begin{itemize}
  \item Visualizzazione singolo \glossario{dizionario dati} (\hyperref[UC9]{UC9}).
\end{itemize}

% Da aggiungere forse un UC per il fallimento di visualizzazione della lista

\subsubsection{UC11 - Visualizzazione errore generazione del \glossario{prompt}}\label{UC11}
\paragraph*{Descrizione}
Viene visualizzato un messaggio a seguito di errori interni al sistema durante il processo di generazione del \glossario{prompt}.

\paragraph*{Attori principali}
Utente

\paragraph*{Precondizioni}
\begin{itemize}
  \item L'applicazione è stata avviata con successo;
  \item Nel sistema è stato caricato almeno un \glossario{dizionario dati};
  \item L'Utente ha richiesto la generazione del \glossario{prompt} (\hyperref[UC5]{UC5}).  
\end{itemize}

\paragraph*{Postcondizioni}
\begin{itemize}
  \item Viene visualizzato correttamente il messaggio di errore.
\end{itemize}

\paragraph*{Scenario principale}
\begin{enumerate}
  \item Il sistema è attivo e funzionante;
  \item L'Utente richiede la generazione del \glossario{prompt} (\hyperref[UC5]{UC5});
  \item Il sistema rileva degli errori interni;
  \item Viene visualizzato un messaggio di errore appropriato che suggerisce all'Utente di riprovare l'operazione, attendere o contattare il supporto tecnico.
\end{enumerate}

\subsubsection{UC12 - Logout}\label{UC12}
\paragraph*{Descrizione}
Un Utente può disconnettersi dalla piattaforma tramite la procedura di logout.

\paragraph*{Attori principali}
Utente

\paragraph*{Precondizioni}
\begin{itemize}
  \item L’Utente è correttamente autenticato. 
\end{itemize}

\paragraph*{Postcondizioni}
\begin{itemize}
  \item L’Utente ha eseguito il logout e non ha più accesso alle funzionalità base dell’applicativo.
\end{itemize}

\paragraph*{Scenario principale}
\begin{enumerate}
  \item L’Utente desidera terminare la sessione corrente e clicca sul tasto di logout;
  \item La sessione dell’Utente viene terminata.  
\end{enumerate}

\subsubsection{UC13 - Caricamento \glossario{dizionario dati}}\label{UC13}
\paragraph*{Descrizione}
Il caricamento del \glossario{dizionario dati} corrisponde alla procedura di importazione di un \glossario{dizionario dati} aggiuntivo all’interno del sistema.

\paragraph*{Attori principali}
Tecnico

\paragraph*{Precondizioni}
\begin{itemize}
  \item Il Tecnico è correttamente autenticato tramite login (\hyperref[UC1]{UC1});
  \item Il Tecnico ha selezionato l’opzione di caricamento di un nuovo \glossario{dizionario dati}.  
\end{itemize}

\paragraph*{Postcondizioni}
\begin{itemize}
  \item Il Tecnico ha inserito un nuovo \glossario{dizionario dati} nel sistema.
\end{itemize}

\paragraph*{Scenario principale}
\begin{enumerate}
  \item Il Tecnico ha selezionato il pulsante per avviare la procedura di inserimento di un nuovo \glossario{dizionario dati};
  \item Il sistema permette al Tecnico di inserire un nuovo file dal formato appropriato;
  \item Il Tecnico inserisce il file relativo al nuovo \glossario{dizionario dati};
  \item Il \glossario{dizionario dati} viene aggiunto alla lista di quelli presenti nell’applicativo.  
\end{enumerate}

\paragraph*{Sottocasi d'uso}
\begin{itemize}
  \item Visualizzazione errore per mancato caricamento del dizionario (\hyperref[UC13point1]{UC13.1}).
\end{itemize}

%%%%%%%%%%%%%%%%%%%%%%%%%%%%%%%%%%%%%%%%%%%%%%%%%%%%%%%%%%%%%%%%%%%%%%%%%%%%%%

\subsubsection{UC13.1 - Visualizzazione errore per mancato caricamento del dizionario}\label{UC13point1}
\paragraph*{Descrizione}
Il sistema mostra un messaggio d’errore relativo al mancato caricamento del nuovo \glossario{dizionario dati} che si desidera inserire nel momento in cui vengono rilevate anomalie relative al suo inserimento.

\paragraph*{Attori principali}
Tecnico

\paragraph*{Precondizioni}
\begin{itemize}
  \item Il Tecnico è correttamente autenticato tramite login (\hyperref[UC1]{UC1});
  \item Il Tecnico ha selezionato l’opzione di caricamento di un nuovo \glossario{dizionario dati};
  \item Il Tecnico ha inserito il file relativo al nuovo \glossario{dizionario dati} che desidera inserire.  
\end{itemize}

\paragraph*{Postcondizioni}
\begin{itemize}
  \item Il sistema mostra un messaggio d’errore per il mancato inserimento del nuovo \glossario{dizionario dati}.
\end{itemize}

\paragraph*{Scenario principale}
\begin{enumerate}
  \item Il Tecnico ha avviato la procedura di caricamento di un nuovo \glossario{dizionario dati};
  \item Il sistema permette al Tecnico di inserire un nuovo file;
  \item Il sistema registra un errore nel caricamento del file;
  \item Il sistema segnala il mancato caricamento al Tecnico tramite un messaggio di errore.  
\end{enumerate}

\subsubsection{UC14 - Visualizzazione dati dizionario}\label{UC14}
\paragraph*{Descrizione}
Viene mostrato il \glossario{dizionario dati} caricato dal Tecnico, comprendendo tutti i suoi dati in maniera esaustiva, con nome, descrizione, impostazione del \glossario{database}, sinonimi, parte di test.

\paragraph*{Attori principali}
Utente

\paragraph*{Precondizioni}
\begin{itemize}
  \item Il Utente è correttamente autenticato tramite login (\hyperref[UC1]{UC1});
  \item È presente almeno un \glossario{dizionario dati}.
\end{itemize}

\paragraph*{Postcondizioni}
\begin{itemize}
  \item Vengono visualizzati i dati del \glossario{dizionario dati} scelto.
\end{itemize}

\paragraph*{Scenario principale}
\begin{enumerate}
  \item Il Utente naviga la lista dei \glossario{dizionari dati} (\hyperref[UC10]{UC10});
  \item Il Utente seleziona un \glossario{dizionario dati};
  \item Il Utente visualizza i dati del dizionario selezionato.
\end{enumerate}

\paragraph*{Inclusione}
\begin{itemize}
  \item Visualizzazione singolo \glossario{dizionario dati} (\hyperref[UC9]{UC9});
  \item Visualizzazione tabelle del dizionario (\hyperref[UC14point1]{UC14.1});
  \item Visualizzazione descrizione e sinonimi delle tabelle (\hyperref[UC14point2]{UC14.2});
  \item Visualizzazione sinonimi delle colonne (\hyperref[UC14point3]{UC14.3});
\end{itemize}

%%%%%%%%%%%%%%%%%%%%%%%%%%%%%%%%%%%%%%%%%%%%%%%%%%%%%%%%%%%%%%%%%%%%%%%%%%%%%%

\subsubsection{UC15 - Verifica correttezza dizionario}\label{UC15}
\paragraph*{Descrizione}
Il Tecnico vuole poter testare il \glossario{dizionario dati} e interroga il sistema con frasi predeterminate collegandosi direttamente al \glossario{database} per testarne i dati con quelli forniti da una \glossario{query} di confronto prefissata.

\paragraph*{Attori principali}
Tecnico

\paragraph*{Precondizioni}
\begin{itemize}
  \item Il Tecnico è correttamente autenticato tramite login (\hyperref[UC1]{UC1});
  \item Il Tecnico naviga alla lista dei dizionari dati caricati;
  \item Il Tecnico seleziona un \glossario{dizionario dati} di cui vuole modificare qualcosa;
  \item Il Tecnico seleziona l’interfaccia di verifica;
  \item Il Tecnico visualizza la lista dei test di verifica preimpostati;
  \item Il Tecnico avvia la procedura di verifica.
\end{itemize}

\paragraph*{Postcondizioni}
\begin{itemize}
  \item Il sistema genera un report che valida per ogni test la corrispondenza dei dati prodotti dalla \glossario{query} generata con quelli prodotti dalla \glossario{query} prefissata.
\end{itemize}

\paragraph*{Scenario principale}
\begin{enumerate}
  \item Il Tecnico desidera verificare la correttezza delle \glossario{query} generate dal dizionario;
  \item Il Tecnico seleziona un \glossario{dizionario dati} sul quale baserà il processo di verifica;
  \item Il Tecnico ottiene il risultato della verifica delle \glossario{query} generate.
\end{enumerate}

\paragraph*{Inclusioni}
\begin{itemize}
  \item Esecuzione test verifica correttezza (\hyperref[UC15point1]{UC15.1});
\end{itemize}

%%%%%%%%%%%%%%%%%%%%%%%%%%%%%%%%%%%%%%%%%%%%%%%%%%%%%%%%%%%%%%%%%%%%%%%%%%%%%%

\subsubsection{UC15.1 - Esecuzione test verifica correttezza}\label{UC15point1}
\paragraph*{Descrizione}
L’utente vuole verificare la correttezza, confrontando i dati in uscita tra una \glossario{query} generata e una prefissata.

\paragraph*{Attori principali}
Tecnico

\paragraph*{Precondizioni}
\begin{itemize}
  \item Il Tecnico è correttamente autenticato tramite login (\hyperref[UC1]{UC1});
  \item E’ presente almeno un\glossario{dizionario dati} nel sistema;
  \item Il Tecnico seleziona lo strumento di verifica;
  \item Il Tecnico visualizza la lista dei test di verifica preimpostati; %UC? Aggiungere
  \item Il Tecnico seleziona il test da avviare. %UC? Aggiungere e come precondizione ha il precedente
\end{itemize}

\paragraph*{Postcondizioni}
\begin{itemize}
  \item Il Tecnico visualizza un report che mostra se i dati in uscita dalla \glossario{query} generata corrispondono a quella fornita per il test.
\end{itemize}

\paragraph*{Scenario principale}
\begin{enumerate}
  \item Il Tecnico ha selezionato il test da eseguire;
  \item Il sistema genera il \glossario{prompt} a partire dalla frase di test inserita (\hyperref[UC5]{UC5});
  \item Il Tecnico visualizza il risultato del test in una tabella in cui si misurano i risultati attesi e quelli ottenuti. %UC 
\end{enumerate}

%\paragraph*{Inclusione}
%\begin{itemize}
%  \item Inserimento \glossario{query} SQL (\hyperref[UC19]{UC19}).
%  \item Visualizzazione tabella test
%\end{itemize}

\subsubsection{UC16 - Modifica \glossario{dizionario dati}}\label{UC16}
\paragraph*{Descrizione}
Il Tecnico vuole modificare un \glossario{dizionario dati} già presente nel sistema.

\paragraph*{Attori principali}
Tecnico

\paragraph*{Precondizioni}
\begin{itemize}
  \item Il Tecnico è correttamente autenticato tramite login (\hyperref[UC1]{UC1});
  \item Il Tecnico naviga alla lista dei dizionari dati caricati;
  \item Il Tecnico seleziona un \glossario{dizionario dati} di cui vuole modificare qualcosa.
\end{itemize}

\paragraph*{Postcondizioni}
\begin{itemize}
  \item Il Tecnico ha modificato una o più cose tra: titolo , descrizione o contenuto del \glossario{dizionario dati}.
\end{itemize}

\paragraph*{Scenario principale}
\begin{enumerate}
  \item Il Tecnico sceglie un \glossario{dizionario dati} che vuole modificare e lo seleziona;
  \item Il Tecnico sceglie quali campi modificare tra : titolo, descrizione o file che contiene il \glossario{dizionario dati};
  \item Il Tecnico seleziona quello che vuole modificare e inserisce una nuova stringa per titolo e descrizione oppure carica un nuovo file per il \glossario{dizionario dati};
  \item Il Tecnico salva le modifiche.
\end{enumerate}

\paragraph*{Sottocasi d'uso}
\begin{itemize}
  \item Modifica titolo \glossario{dizionario dati} (\hyperref[UC16point1]{UC16.1});
  \item Modifica descrizione \glossario{dizionario dati} (\hyperref[UC16point2]{UC16.2});
  \item Modifica file configurazione \glossario{dizionario dati} (\hyperref[UC16point3]{UC16.3});
\end{itemize}

%%%%%%%%%%%%%%%%%%%%%%%%%%%%%%%%%%%%%%%%%%%%%%%%%%%%%%%%%%%%%%%%%%%%%%%%%%%%%%

\subsubsection{UC16.1 - Modifica titolo \glossario{dizionario dati}}\label{UC16point1}
\paragraph*{Descrizione}
Il Tecnico vuole modificare il titolo di un \glossario{dizionario dati}.

\paragraph*{Attori principali}
Tecnico

\paragraph*{Precondizioni}
\begin{itemize}
  \item Il Tecnico è correttamente autenticato tramite login (\hyperref[UC1]{UC1});
  \item Il Tecnico ha selezionato un \glossario{dizionario dati} esistente di cui vuole modificare il titolo.
\end{itemize}

\paragraph*{Postcondizioni}
\begin{itemize}
  \item Il \glossario{dizionario dati} ha come titolo il nuovo valore inserito dal Tecnico.
\end{itemize}

\paragraph*{Scenario principale}
\begin{enumerate}
  \item Il Tecnico ha selezionato il \glossario{dizionario dati} di cui vuole cambiare il titolo e inserisce il nuovo valore nell’apposito box;
  \item Il Tecnico salva la modifica.
\end{enumerate}

%%%%%%%%%%%%%%%%%%%%%%%%%%%%%%%%%%%%%%%%%%%%%%%%%%%%%%%%%%%%%%%%%%%%%%%%%%%%%%

\subsubsection{UC16.2 - Modifica descrizione \glossario{dizionario dati}}\label{UC16point2}
\paragraph*{Descrizione}
Il Tecnico vuole modificare la descrizione di un \glossario{dizionario dati}.

\paragraph*{Attori principali}
Tecnico

\paragraph*{Precondizioni}
\begin{itemize}
  \item Il Tecnico è correttamente autenticato tramite login (\hyperref[UC1]{UC1});
  \item Il Tecnico ha selezionato un \glossario{dizionario dati} esistente di cui vuole modificare la descrizione.
\end{itemize}

\paragraph*{Postcondizioni}
\begin{itemize}
  \item Il \glossario{dizionario dati} ha come descrizione il nuovo valore inserito dal Tecnico.
\end{itemize}

\paragraph*{Scenario principale}
\begin{enumerate}
  \item Il Tecnico ha selezionato il \glossario{dizionario dati} di cui vuole cambiare la descrizione e inserisce il nuovo valore nell’apposito box;
  \item Il Tecnico salva la modifica.
\end{enumerate}

%%%%%%%%%%%%%%%%%%%%%%%%%%%%%%%%%%%%%%%%%%%%%%%%%%%%%%%%%%%%%%%%%%%%%%%%%%%%%%

\subsubsection{UC16.3 - Modifica file \glossario{dizionario dati}}\label{UC16point3}
\paragraph*{Descrizione}
Il Tecnico vuole modificare il file di configurazione di un \glossario{dizionario dati}.

\paragraph*{Attori principali}
Tecnico

\paragraph*{Precondizioni}
\begin{itemize}
  \item Il Tecnico è correttamente autenticato tramite login (\hyperref[UC1]{UC1});
  \item Il Tecnico ha selezionato un \glossario{dizionario dati} esistente di cui vuole modificare il file di configurazione.
\end{itemize}

\paragraph*{Postcondizioni}
\begin{itemize}
  \item Il \glossario{dizionario dati} ha come file di configurazione il nuovo file inserito dal Tecnico.
\end{itemize}

\paragraph*{Scenario principale}
\begin{enumerate}
  \item Il Tecnico ha selezionato il \glossario{dizionario dati} di cui vuole cambiare la il file di configurazione e inserisce il nuovo file nell’apposito input;
  \item Il Tecnico salva la modifica.
\end{enumerate}

\subsubsection{UC17 - Elimina \glossario{dizionario dati}}\label{UC17}
\paragraph*{Descrizione}
Il Tecnico desidera eliminare uno dei dizionari dati inseriti in precedenza dalla lista.

\paragraph*{Attori principali}
Tecnico

\paragraph*{Precondizioni}
\begin{itemize}
  \item Il Tecnico è correttamente autenticato tramite login (\hyperref[UC1]{UC1});
  \item Il Tecnico naviga alla lista dei dizionari dati caricati.  
\end{itemize}

\paragraph*{Postcondizioni}
\begin{itemize}
  \item Il dizionario selezionato è stato eliminato.
\end{itemize}

\paragraph*{Scenario principale}
\begin{enumerate}
  \item Il Tecnico ha selezionato il dizionario da eliminare;
  \item Il Tecnico conferma la scelta di eliminazione;
  \item Il \glossario{dizionario dati} selezionato viene rimosso dalla lista dei dizionari dati;  
\end{enumerate}

\paragraph*{Estensione}
\begin{itemize}
  \item Visualizzazione errore per mancata eliminazione del \glossario{dizionario dati} (\hyperref[UC17point1]{UC17.1}).
\end{itemize}

%%%%%%%%%%%%%%%%%%%%%%%%%%%%%%%%%%%%%%%%%%%%%%%%%%%%%%%%%%%%%%%%%%%%%%%%%%%%%%

\subsubsection{UC17.1 - Visualizzazione errore per mancata eliminazione del \glossario{dizionario dati}}\label{UC17point1}
\paragraph*{Descrizione}
Il sistema mostra un messaggio d’errore relativo alla mancata eliminazione del \glossario{dizionario dati} selezionato dal Tecnico.

\paragraph*{Attori principali}
Tecnico

\paragraph*{Precondizioni}
\begin{itemize}
  \item Il Tecnico è correttamente autenticato tramite login (\hyperref[UC1]{UC1});
  \item Il sistema ha riscontrato un problema nell'eliminazione di un \glossario{dizionario dati} selezionato dal Tecnico.
\end{itemize}

\paragraph*{Postcondizioni}
\begin{itemize}
  \item Il sistema mostra un messaggio d’errore per la mancata eliminazione del nuovo \glossario{dizionario dati}.
\end{itemize}

\paragraph*{Scenario principale}
\begin{enumerate}
  \item Il Tecnico ha selezionato il dizionario da eliminare;
  \item Il Tecnico conferma la scelta di eliminazione;
  \item Il sistema segnala la mancata eliminazione al Tecnico tramite un messaggio di errore.
\end{enumerate}

\subsubsection{UC18 - Debug della generazione del \glossario{prompt}}\label{UC18}

\paragraph*{Descrizione}
Il Tecnico vuole poter testare il \glossario{prompt} generato e interroga il sistema per comprendere la corretta formattazione del \glossario{dizionario dati}. Visualizza una tabella contenente i campi del \glossario{dizionario dati} e lo score ottenuto per questi.

\paragraph*{Attori principali}
Tecnico

\paragraph*{Precondizioni}
\begin{itemize}
  \item Il Tecnico è correttamente autenticato tramite login (\hyperref[UC1]{UC1});
  \item È presente almeno un \glossario{dizionario dati} nel sistema;
  \item Il Tecnico inserisce una richiesta in linguaggio naturale (\hyperref[UC3]{UC3});
  \item Il sistema genera il \glossario{prompt} (\hyperref[UC5]{UC5});
  \item Il Tecnico seleziona lo strumento di debug.
\end{itemize}

\paragraph*{Postcondizioni}
\begin{itemize}
  \item Il sistema genera un \glossario{log};
  \item Il sistema genera una tabella dei migliori risultati individuati;
\end{itemize}

\paragraph*{Scenario principale}
\begin{enumerate}
  \item Il Tecnico riceve il \glossario{prompt} dal sistema (\hyperref[UC5]{UC5});
  \item Il Tecnico seleziona lo strumento di debug;
  \item Lo strumento di debug genera una tabella dei risultati individuati dal sistema; 
  \item Lo strumento di debug genera un \glossario{log} descrittivo dei passaggi per la composizione del \glossario{prompt}.  
\end{enumerate}

\paragraph*{Estensione}
\begin{itemize}
  \item Errore nella generazione del \glossario{prompt} (\hyperref[UC6]{UC6});
  \item Errore nello strumento di debug (\hyperref[UC24]{UC24});
  \item Download del log di debug (\hyperref[UC22]{UC22});
  \item Visualizzazione della tabella di debug (\hyperref[UC25]{UC25}).
\end{itemize}
\subsubsection{UC19 - Visualizzazione errore validazione file}\label{UC19}

\begin{figure}[H]
  \centering
  \includegraphics[width=0.90\textwidth]{assets/uc19.png}
  \caption{UC19}
\end{figure}

\paragraph*{Descrizione}
Nel caso in cui il sistema riscontri delle anomalie nella validazione del file scelto come glossario{dizionario dati}, l'Utente visualizza un messaggio d'errore.

\paragraph*{Attori principali}
Tecnico

\paragraph*{Precondizioni}
\begin{itemize}
  \item L'Utente ha richiesto il salvataggio o la modifica di un \glossario{dizionario dati} nel sistema;
  \item Il file caricato non è valido.
\end{itemize}

\paragraph*{Postcondizioni}
\begin{itemize}
  \item Viene mostrato correttamente il messaggio d'errore.
\end{itemize}

\paragraph*{Scenario principale}
\begin{enumerate}
  \item Il sistema riscontra un errore nella validazione del file scelto come \glossario{dizionario dati};
  \item Il sistema restituisce un messaggio con i dettagli dell'errore. 
\end{enumerate}
\subsubsection{UC20 - Errore per modifica incorretta del \glossario{dizionario dati}}\label{UC20}
\paragraph*{Descrizione}
Il sistema mostra un messaggio d’errore relativo ad una modifica incorretta del \glossario{dizionario dati}.

\paragraph*{Attori principali}
Tecnico

\paragraph*{Precondizioni}
\begin{itemize}
  \item Il Tecnico è correttamente autenticato tramite login (\hyperref[UC1]{UC1});
  \item Il Tecnico ha visualizza i dettagli del \glossario{dizionario dati} da modificare (\hyperref[UC14]{UC14});
  \item Il Tecnico sceglie di modificare il \glossario{dizionario dati} individuato (\hyperref[UC16]{UC16}).
\end{itemize}

\paragraph*{Postcondizioni}
\begin{itemize}
  \item Il sistema mostra un messaggio d’errore dovuto ad una modifica invalida.
\end{itemize}

\paragraph*{Scenario principale}
\begin{enumerate}
  \item Il Tecnico ha avviato la procedura di modifica del \glossario{dizionario dati} (\hyperref[UC16]{UC16});
  \item Il Tecnico modifica il \glossario{dizionario dati};
  \item Il sistema riscontra un errore nella modifica;
  \item Viene mostrato un messaggio d'errore con il dettaglio dell'invalidità nella modifica.  
\end{enumerate}

\paragraph*{Inclusioni}
\begin{itemize}
    \item Errore per modifica incorretta del titolo del \glossario{dizionario dati} (\hyperref[UC20point1]{UC20.1});
    \item Errore per modifica incorretta della descrizione del \glossario{dizionario dati} (\hyperref[UC20point2]{UC20.2});
    \item Errore per mancato caricamento del \glossario{dizionario dati} (\hyperref[UC19]{UC19}).
\end{itemize}

%%%%%%%%%%%%%%%%%%%%%%%%%%%%%%%%%%%%%%%%%%%%%%%%%%%

\subsubsection{UC20.1 - Errore per modifica incorretta del titolo del \glossario{dizionario dati}}\label{UC20point1}
\paragraph*{Descrizione}
Il sistema mostra un messaggio d’errore relativo ad una modifica del titolo con un formato invalido.

\paragraph*{Attori principali}
Tecnico

\paragraph*{Precondizioni}
\begin{itemize}
  \item Il Tecnico è correttamente autenticato tramite login (\hyperref[UC1]{UC1});
  \item Il Tecnico sceglie di modificare il \glossario{dizionario dati} individuato (\hyperref[UC16]{UC16});
  \item Il Tecnico sceglie di modificare il titolo.
\end{itemize}

\paragraph*{Postcondizioni}
\begin{itemize}
  \item Il sistema mostra un messaggio d’errore dovuto ad una modifica con formato invalido del titolo.
\end{itemize}

\paragraph*{Scenario principale}
\begin{enumerate}
  \item Il Tecnico ha avviato la procedura di modifica del \glossario{dizionario dati} (\hyperref[UC16]{UC16});
  \item Il Tecnico modifica il titolo del \glossario{dizionario dati};
  \item Il sistema riscontra un errore nella modifica del titolo;
  \item Viene mostrato un messaggio d'errore con il dettaglio dell'invalidità nella modifica.  
\end{enumerate}

%%%%%%%%%%%%%%%%%%%%%%%%%%%%%%%%%%%%%%%%%%%%%%%%

\subsubsection{UC20.2 - Errore per modifica incorretta della descrizione del \glossario{dizionario dati}}\label{UC20point2}
\paragraph*{Descrizione}
Il sistema mostra un messaggio d’errore relativo ad una modifica della descrizione con un formato invalido.

\paragraph*{Attori principali}
Tecnico

\paragraph*{Precondizioni}
\begin{itemize}
  \item Il Tecnico è correttamente autenticato tramite login (\hyperref[UC1]{UC1});
  \item Il Tecnico sceglie di modificare il \glossario{dizionario dati} individuato (\hyperref[UC16]{UC16});
  \item Il Tecnico sceglie di modificare la descrizione.
\end{itemize}

\paragraph*{Postcondizioni}
\begin{itemize}
  \item Il sistema mostra un messaggio d’errore dovuto ad una modifica con formato invalido della descrizione.
\end{itemize}

\paragraph*{Scenario principale}
\begin{enumerate}
  \item Il Tecnico ha avviato la procedura di modifica del \glossario{dizionario dati} (\hyperref[UC16]{UC16});
  \item Il Tecnico modifica la descrizione del \glossario{dizionario dati};
  \item Il sistema riscontra un errore nella modifica della descrizione;
  \item Viene mostrato un messaggio d'errore con il dettaglio dell'invalidità nella modifica.  
\end{enumerate}
\subsubsection{UC21 - Visualizzazione errore eliminazione dizionario dati}\label{UC21}
\paragraph*{Descrizione}
Il sistema mostra un messaggio di errore a causa della mancata eliminazione di un \glossario{dizionario dati}.

\paragraph*{Attori principali}
Tecnico

\paragraph*{Precondizioni}
\begin{itemize}
  \item Il Tecnico ha effettuato l'autenticazione (\hyperref[UC1]{UC1});
  \item Il Tecnico ha richiesto l'eliminazione di un \glossario{dizionario dati} (\hyperref[UC18]{UC18});
  \item Il sistema ha riscontrato un errore nel tentativo di eliminare il dizionario dati.
\end{itemize}

\paragraph*{Postcondizioni}
\begin{itemize}
  \item Viene visualizzato un messaggio d'errore esplicativo;
  \item Il \glossario{dizionario dati} non viene rimosso dal sistema.
\end{itemize}

\paragraph*{Scenario principale}
\begin{enumerate}
  \item Il sistema restituisce un messaggio con i dettagli dell'errore.
\end{enumerate}
\subsubsection{UC25 - Visualizzazione file di debug}\label{UC25}
\paragraph*{Descrizione}
Il Tecnico visualizza un file di \glossario{log} che illustra il processo di generazione del \glossario{prompt}. Il debug può aiutare il Tecnico a capire come migliorare il \glossario{dizionario dati}, in particolare le descrizioni in linguaggio naturale delle tabelle e delle colonne del \glossario{database}. Nel file di log, infatti, viene documentato il modo in cui le descrizioni interagiscono con il modello di \glossario{AI}.

\paragraph*{Attori principali}
Tecnico

\paragraph*{Precondizioni}
\begin{itemize}
  \item Il sistema è attivo e funzionante;
  \item Il Tecnico ha effettuato correttamente l'autenticazione (\hyperref[UC1]{UC1});
  \item Il sistema ha generato almeno un \glossario{prompt} e il rispettivo file di \glossario{log}.
\end{itemize}

\paragraph*{Postcondizioni}
\begin{itemize}
  \item Il contenuto del file viene visualizzato correttamente.
\end{itemize}

\paragraph*{Scenario principale}
\begin{enumerate}
  \item Il Tecnico accede alla sezione dedicata ai \glossario{log};
  \item Il Tecnico visualizza l'ultimo file di \glossario{log} generato;
  \item Il sistema mostra tutte le informazioni contenute nel file:
    \begin{itemize}
      \item Data e ora di generazione del file;
      \item Richiesta in linguaggio naturale;
      \item Prima fase della generazione del \glossario{prompt} (lista delle tabelle considerate rilevanti dal modello):
      \begin{itemize}
        \item Nome della tabella;
        \item Punteggio assegnato alla tabella;
        \item Descrizione della tabella;
        \item Classifica di importanza dei termini presenti nella descrizione della tabella;
        \item Descrizione della colonna più rilevante;
        \item Classifica di importanza dei termini presenti nella descrizione della colonna;
      \end{itemize}
      \item Seconda fase della generazione del \glossario{prompt} (lista delle tabelle pertinenti):
      \begin{itemize}
        \item Spiegazione del motivo per cui una tabella viene inserita o meno nel \glossario{prompt}.
      \end{itemize}
    \end{itemize}
\end{enumerate}
\subsubsection{UC23 - Download file di log}\label{UC23}
\paragraph*{Descrizione}
Il sistema consente al Tecnico di scaricare, in formato testuale, il \glossario{log} creato durante la generazione del \glossario{prompt}.

\paragraph*{Attori principali}
Tecnico

\paragraph*{Precondizioni}
\begin{itemize}
  \item Il sistema è attivo e funzionante;
  \item Il Tecnico ha effettuato l'autenticazione (\hyperref[UC1]{UC1});
  \item Il sistema ha generato almeno un \glossario{prompt} e il rispettivo \glossario{log}.
\end{itemize}

\paragraph*{Postcondizioni}
\begin{itemize}
  \item Il file è stato scaricato correttamente.
\end{itemize}

\paragraph*{Trigger}
Il Tecnico vuole scaricare un file di \glossario{log}.

\paragraph*{Scenario principale}
\begin{enumerate}
  \item Il Tecnico accede alla sezione dedicata al \glossario{debug} del prompt;
  \item Il Tecnico seleziona la funzionalità "Scarica file";
  \item Il sistema esegue il download del file.
\end{enumerate}

\paragraph*{Scenario alternativo}
\begin{enumerate}
  \item Il sistema rileva un errore durante il download del file (\hyperref[UC24]{UC24});
  \item Viene visualizzato un messaggio con i dettagli dell'errore.
\end{enumerate}

\paragraph*{Estensioni}
\begin{itemize}
  \item Visualizzazione errore download file (\hyperref[UC24]{UC24}):
  \begin{itemize}
    \item Extension point: Errore durante il download del file.
  \end{itemize}
\end{itemize}
\subsubsection{UC24 - Visualizzazione errore download file}\label{UC24}
\paragraph*{Descrizione}
Il Tecnico visualizza un messaggio d'errore per il mancato download del file.

\paragraph*{Attori principali}
Tecnico

\paragraph*{Precondizioni}
\begin{itemize}
  \item Il Tecnico ha effettuato l'autenticazione (\hyperref[UC1]{UC1});
  \item Il Tecnico ha richiesto il download di un file (\hyperref[UC23]{UC23});
  \item Il sistema ha riscontrato un errore durante il download.
\end{itemize}

\paragraph*{Postcondizioni}
\begin{itemize}
  \item Viene visualizzato un messaggio d'errore esplicativo;
  \item Il file non è stato scaricato.
\end{itemize}

\paragraph*{Scenario principale}
\begin{enumerate}
  \item Viene visualizzato un messaggio con i dettagli dell'errore.
\end{enumerate}
\subsubsection{UC25 - Visualizzazione singolo messaggio}\label{UC25}
\paragraph*{Descrizione}
L'Utente visualizza un messaggio nella chat. Questo può essere un messaggio di richiesta in linguaggio naturale o un messaggio di risposta contenente il \glossario{prompt}. 

\paragraph*{Attori principali}
Utente

\paragraph*{Precondizioni}
\begin{itemize}
  \item Dev'essere presente almeno un messaggio nella chat.
\end{itemize}

\paragraph*{Postcondizioni}
\begin{itemize}
  \item L'Utente visualizza correttamente il messaggio nella chat.
\end{itemize}

\paragraph*{Scenario principale}
\begin{enumerate}
  \item Nel sistema è presente almeno un messaggio della chat.
  \item L'Utente visualizza il messaggio.
\end{enumerate}

\paragraph*{Inclusioni}
\begin{itemize}
    \item Visualizzazione messaggio di richiesta (\hyperref[UC25point1]{UC25.1});
    \item Visualizzazione messaggio di risposta (\hyperref[UC25point2]{UC25.2});
\end{itemize}

\begin{figure}[H]
  \centering
  \includegraphics[width=0.90\textwidth]{assets/uc25.png}
  \caption{UC25 - Sottocasi d'uso}
\end{figure}

%%%%%%%%%%%%%%%%%%%%%%%%%%%%%%%%%%%%%%%%%%%

\subsubsection{UC25.1 - Visualizzazione messaggio di richiesta}\label{UC25point1}
\paragraph*{Descrizione}
L'Utente visualizza un messaggio con la sua richiesta in linguaggio naturale nella chat.

\paragraph*{Attori principali}
Utente

\paragraph*{Precondizioni}
\begin{itemize}
  \item È presente almeno un messaggio di richiesta in linguaggio naturale nella chat (\hyperref[UC3]{UC3}).
  \item L'Utente visualizza un messaggio nella chat.
\end{itemize}

\paragraph*{Postcondizioni}
\begin{itemize}
  \item Il messaggio visualizzato è un messaggio con una richiesta in linguaggio naturale.
\end{itemize}

\paragraph*{Scenario principale}
\begin{enumerate}
  \item Nel sistema è presente almeno un messaggio di richiesta in linguaggio naturale (\hyperref[UC3]{UC3}).
  \item L'Utente visualizza il messaggio di richiesta.
\end{enumerate}

%%%%%%%%%%%%%%%%%%%%%%%%%%%%%%%%%%%%%%%%%%%

\subsubsection{UC25.2 - Visualizzazione messaggio di risposta}\label{UC25point2}
\paragraph*{Descrizione}
L'Utente viualizza un messaggio di risposta, contenente il \glossario{prompt} generato da una richiesta in linguaggio naturale.

\paragraph*{Attori principali}
Utente

\paragraph*{Precondizioni}
\begin{itemize}
  \item È presente almeno un messaggio di risposta contenente un \glossario{prompt} (\hyperref[UC5]{UC5}).
  \item L'Utente visualizza un messaggio nella chat.
\end{itemize}

\paragraph*{Postcondizioni}
\begin{itemize}
  \item Il messaggio visualizzato è un messaggio con una riposta contenente un \glossario{prompt}.
\end{itemize}

\paragraph*{Scenario principale}
\begin{enumerate}
  \item Nel sistema è presente almeno un messaggio di risposta contenente un \glossario{prompt} (\hyperref[UC5]{UC5}).
  \item L'Utente visualizza il messaggio di riposta.
\end{enumerate}


\subsubsection{UC26 - Selezione DBMS}\label{UC26}

\paragraph*{Descrizione}
Il Tecnico seleziona un \glossario{DBMS} tra quelli disponibili, da utilizzare per la generazione del \glossario{prompt} e la conseguente costruzione della query SQL tramite \glossario{LLM}.

\paragraph*{Attori principali}
Utente

\paragraph*{Precondizioni}
\begin{itemize}
  \item L'applicazione è stata avviata con successo;
  \item L'interfaccia per la selezione del \glossario{DBMS} è accessibile.
\end{itemize}

\paragraph*{Postcondizioni}
\begin{itemize}
  \item Il \glossario{DBMS} è stato selezionato correttamente.
\end{itemize}

\paragraph*{Scenario principale}
\begin{enumerate}
  \item L'Utente visualizza la lista dei \glossario{DBMS} disponibili;
  \item L'Utente sceglie un \glossario{DBMS} da una lista predefinita. La scelta è tra i seguenti DBMS:
    \begin{itemize}
      \item MySQL (valore di default, se nessun DBMS è selezionato in modo esplicito);
      \item MariaDB;
      \item PostgreSQL;
      \item Microsoft SQL Server;
      \item Oracle Database;
      \item SQLite.
    \end{itemize} 
\end{enumerate}

\subsubsection{UC27 - Eliminazione cronologia della chat}\label{UC27}

\begin{figure}[H]
  \centering
  \includegraphics[width=0.90\textwidth]{assets/uc27.png}
  \caption{UC27}
\end{figure}

\paragraph*{Descrizione}
L'Utente elimina la cronologia della chat. La chat è composta da un elenco di messaggi, che possono essere richieste dell'utente o risposte del ChatBOT.

\paragraph*{Attori principali}
Utente

\paragraph*{Precondizioni}
\begin{itemize}
  \item L'applicazione è stata avviata con successo;
  \item È presente almeno un messaggio all'interno della chat.
\end{itemize}

\paragraph*{Postcondizioni}
\begin{itemize}
  \item La cronologia della chat è stata eliminata.
\end{itemize}

\paragraph*{Trigger}
L'Utente vuole eliminare la cronologia della chat.

\paragraph*{Scenario principale}
\begin{enumerate}
  \item L'utente accede alla chat;
  \item L'utente seleziona l'opzione per eliminare la cronologia della chat;
  \item La cronologia della chat viene eliminata.
\end{enumerate}

