\subsubsection{UC7 - Cambio lingua}\label{UC7}
\paragraph*{Descrizione}
L’Utente desidera inserire una richiesta in linguaggio naturale in una lingua diversa dall’italiano.

\paragraph*{Attori principali}
Utente

\paragraph*{Precondizioni}
\begin{itemize}
  \item L'applicazione è stata avviata con successo;
  \item L’autenticazione ad Utente è andata a buon fine;
  \item L’Utente vuole inserire una frase in linguaggio naturale.
\end{itemize}

\paragraph*{Postcondizioni}
\begin{itemize}
  \item L'applicazione opera la generazione del prompt considerando la lingua selezionata.
\end{itemize}

\paragraph*{Scenario principale}
\begin{enumerate}
  \item L’Utente seleziona una lingua da una lista predefinita di lingue. La scelta è tra:
    \begin{itemize}
      \item Italiano;
      \item Inglese;
      \item Francese;
      \item Spagnolo;
      \item Tedesco.
    \end{itemize}
  \item L’Utente inserisce una interrogazione in linguaggio naturale nella lingua selezionata nell’ apposito box su cui poi il sistema potrà produrre un prompt in seguito. 
\end{enumerate}

% TODO [in caso di errore ed incongruenza tra la lingua scritta e quella selezionata teniamo solo errore uc6 del prompt? O viene controllata prima?]
