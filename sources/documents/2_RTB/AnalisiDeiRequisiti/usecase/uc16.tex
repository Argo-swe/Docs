\subsubsection{UC16 - Modifica \glossario{dizionario dati}}\label{UC16}
\paragraph*{Descrizione}
Il Tecnico vuole modificare un \glossario{dizionario dati} già presente nel sistema.

\paragraph*{Attori principali}
Tecnico

\paragraph*{Precondizioni}
\begin{itemize}
  \item Il Tecnico è correttamente autenticato tramite login (\hyperref[UC1]{UC1});
  \item Il Tecnico naviga alla lista dei dizionari dati caricati;
  \item Il Tecnico seleziona un \glossario{dizionario dati} di cui vuole modificare qualcosa.  
\end{itemize}

\paragraph*{Postcondizioni}
\begin{itemize}
  \item Il Tecnico ha modificato una o più cose tra: titolo , descrizione o contenuto del \glossario{dizionario dati}.
\end{itemize}

\paragraph*{Scenario principale}
\begin{enumerate}
  \item Il Tecnico sceglie un \glossario{dizionario dati} che vuole modificare e lo seleziona;
  \item Il Tecnico sceglie quali campi modificare tra : titolo, descrizione o file che contiene il \glossario{dizionario dati};
  \item Il Tecnico seleziona quello che vuole modificare e inserisce una nuova stringa per titolo e descrizione oppure carica un nuovo file per il \glossario{dizionario dati};
  \item Il Tecnico salva le modifiche.  
\end{enumerate}

\paragraph*{Sottocasi d'uso}
\begin{itemize}
  \item Modifica titolo \glossario{dizionario dati} (\hyperref[UC16point1]{UC16.1});
  \item Modifica descrizione \glossario{dizionario dati} (\hyperref[UC16point2]{UC16.2});
  \item TODO
\end{itemize}

%%%%%%%%%%%%%%%%%%%%%%%%%%%%%%%%%%%%%%%%%%%%%%%%%%%%%%%%%%%%%%%%%%%%%%%%%%%%%%

\subsubsection{UC16.1 - Modifica titolo \glossario{dizionario dati}}\label{UC16point1}
\paragraph*{Descrizione}
Il Tecnico vuole modificare il titolo di un \glossario{dizionario dati}.

\paragraph*{Attori principali}
Tecnico

\paragraph*{Precondizioni}
\begin{itemize}
  \item Il Tecnico è correttamente autenticato tramite login (\hyperref[UC1]{UC1});
  \item Il Tecnico ha selezionato un \glossario{dizionario dati} esistente di cui vuole modificare il titolo.  
\end{itemize}

\paragraph*{Postcondizioni}
\begin{itemize}
  \item Il \glossario{dizionario dati} ha come titolo il nuovo valore inserito dal Tecnico.
\end{itemize}

\paragraph*{Scenario principale}
\begin{enumerate}
  \item Il Tecnico ha selezionato il \glossario{dizionario dati} di cui vuole cambiare il titolo e inserisce il nuovo valore nell’apposito box;
  \item Il Tecnico salva la modifica.  
\end{enumerate}

%%%%%%%%%%%%%%%%%%%%%%%%%%%%%%%%%%%%%%%%%%%%%%%%%%%%%%%%%%%%%%%%%%%%%%%%%%%%%%

\subsubsection{UC16.2 - Modifica descrizione \glossario{dizionario dati}}\label{UC16point2}
\paragraph*{Descrizione}
Il Tecnico vuole modificare la descrizione di un \glossario{dizionario dati}.

\paragraph*{Attori principali}
Tecnico

\paragraph*{Precondizioni}
\begin{itemize}
  \item Il Tecnico è correttamente autenticato tramite login (\hyperref[UC1]{UC1});
  \item Il Tecnico ha selezionato un \glossario{dizionario dati} esistente di cui vuole modificare il la descrizione.  
\end{itemize}

\paragraph*{Postcondizioni}
\begin{itemize}
  \item Il \glossario{dizionario dati} ha come descrizione il nuovo valore inserito dal Tecnico.
\end{itemize}

\paragraph*{Scenario principale}
\begin{enumerate}
  \item Il Tecnico ha selezionato il \glossario{dizionario dati} di cui vuole cambiare la descrizione e inserisce il nuovo valore nell’apposito box;
  \item Il Tecnico salva la modifica.
\end{enumerate}

TODO Descrizioni vuote ammissibili o no?
