\subsubsection{UC15 - Verifica correttezza dizionario}\label{UC15}
\paragraph*{Descrizione}
Il Tecnico vuole poter testare il \glossario{dizionario dati} e interroga il sistema con frasi predeterminate collegandosi direttamente al \glossario{database} per testarne i dati con quelli forniti da una \glossario{query} di confronto prefissata.

\paragraph*{Attori principali}
Tecnico

\paragraph*{Precondizioni}
\begin{itemize}
  \item Il Tecnico è correttamente autenticato tramite login (\hyperref[UC1]{UC1});
  \item Il Tecnico naviga alla lista dei dizionari dati caricati;
  \item Il Tecnico seleziona un \glossario{dizionario dati} di cui vuole modificare qualcosa;
  \item Il Tecnico seleziona l’interfaccia di verifica;
  \item Il Tecnico visualizza la lista dei test di verifica preimpostati;
  \item Il Tecnico avvia la procedura di verifica.
\end{itemize}

\paragraph*{Postcondizioni}
\begin{itemize}
  \item Il sistema genera un report che valida per ogni test la corrispondenza dei dati prodotti dalla \glossario{query} generata con quelli prodotti dalla \glossario{query} prefissata.
\end{itemize}

\paragraph*{Scenario principale}
\begin{enumerate}
  \item Il Tecnico desidera verificare la correttezza delle \glossario{query} generate dal dizionario;
  \item Il Tecnico seleziona un \glossario{dizionario dati} sul quale baserà il processo di verifica;
  \item Il Tecnico ottiene il risultato della verifica delle \glossario{query} generate.
\end{enumerate}

\paragraph*{Inclusioni}
\begin{itemize}
  \item Esecuzione test verifica correttezza (\hyperref[UC15point1]{UC15.1});
\end{itemize}

%%%%%%%%%%%%%%%%%%%%%%%%%%%%%%%%%%%%%%%%%%%%%%%%%%%%%%%%%%%%%%%%%%%%%%%%%%%%%%

\subsubsection{UC15.1 - Esecuzione test verifica correttezza}\label{UC15point1}
\paragraph*{Descrizione}
L’utente vuole verificare la correttezza, confrontando i dati in uscita tra una \glossario{query} generata e una prefissata.

\paragraph*{Attori principali}
Tecnico

\paragraph*{Precondizioni}
\begin{itemize}
  \item Il Tecnico è correttamente autenticato tramite login (\hyperref[UC1]{UC1});
  \item E’ presente almeno un\glossario{dizionario dati} nel sistema;
  \item Il Tecnico seleziona lo strumento di verifica;
  \item Il Tecnico visualizza la lista dei test di verifica preimpostati; %UC? Aggiungere
  \item Il Tecnico seleziona il test da avviare. %UC? Aggiungere e come precondizione ha il precedente
\end{itemize}

\paragraph*{Postcondizioni}
\begin{itemize}
  \item Il Tecnico visualizza un report che mostra se i dati in uscita dalla \glossario{query} generata corrispondono a quella fornita per il test.
\end{itemize}

\paragraph*{Scenario principale}
\begin{enumerate}
  \item Il Tecnico ha selezionato il test da eseguire;
  \item Il sistema genera il \glossario{prompt} a partire dalla frase di test inserita (\hyperref[UC5]{UC5});
  \item Il Tecnico visualizza il risultato del test in una tabella in cui si misurano i risultati attesi e quelli ottenuti. %UC 
\end{enumerate}

%\paragraph*{Inclusione}
%\begin{itemize}
%  \item Inserimento \glossario{query} SQL (\hyperref[UC19]{UC19}).
%  \item Visualizzazione tabella test
%\end{itemize}
