\subsubsection{UC18 - Debug della generazione del prompt}\label{UC18}
TODO Approfondire, espandere, dettagliare
\paragraph*{Descrizione}
Il Tecnico vuole poter testare il \glossario{dizionario dati} e interroga il sistema per comprendere la corretta formattazione del \glossario{dizionario dati}, riceverà in risposta uno schema motivato della selezione dei campi del dizionario per permettere eventuali modifiche volte ad ottimizzare le interrogazioni.

\paragraph*{Attori principali}
Tecnico

\paragraph*{Precondizioni}
\begin{itemize}
  \item Il Tecnico è correttamente autenticato tramite login (\hyperref[UC1]{UC1});
  \item È presente almeno un \glossario{dizionario dati} nel sistema;
  \item Il Tecnico seleziona l’interfaccia di debug.
\end{itemize}

\paragraph*{Postcondizioni}
\begin{itemize}
  \item Il sistema genera uno schema facilmente comprensibile e motivato della selezione dei singoli campi delle tabelle per dare al Tecnico una visione dettagliata dei criteri di scelta di generazione del prompt.
\end{itemize}

\paragraph*{Scenario principale}
\begin{enumerate}
  \item Il Tecnico desidera testare la correttezza di un dizionario; 
  \item Il Tecnico seleziona un \glossario{dizionario dati} sul quale baserà l’interrogazione;
  \item Il Tecnico sceglie la lingua su cui farà l’interrogazione. La lingua è in italiano di default se non viene cambiata;
  \item Il Tecnico inserisce un’interrogazione in linguaggio naturale nel box testuale apposito;
  \item Il sistema fornisce lo schema di debug relativo al sotto-modello selezionato per l’utilizzo nel prompt.  
\end{enumerate}

\paragraph*{Inclusione}
\begin{itemize}
  \item Inserimento richiesta in linguaggio naturale (\hyperref[UC3]{UC3});
  \item Selezione \glossario{dizionario dati} (\hyperref[UC4]{UC4}).
\end{itemize}

\paragraph*{Estensione}
\begin{itemize}
  \item Cambio lingua (\hyperref[UC7]{UC7});
  \item Messaggio d'errore nella generazione del prompt (\hyperref[UC6]{UC6}).
\end{itemize}
