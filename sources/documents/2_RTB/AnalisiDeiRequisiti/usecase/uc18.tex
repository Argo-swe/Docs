\subsubsection{UC18 - Debug della generazione del prompt}\label{UC18}

\paragraph*{Descrizione}
Il Tecnico vuole poter testare il prompt generato e interroga il sistema per comprendere la corretta formattazione del \glossario{dizionario dati}. Riceverà in risposta uno schema motivato della selezione dei campi del dizionario e un log, che hanno portato al risultato ottenuto.

\paragraph*{Attori principali}
Tecnico

\paragraph*{Precondizioni}
\begin{itemize}
  \item Il Tecnico è correttamente autenticato tramite login (\hyperref[UC1]{UC1});
  \item È presente almeno un \glossario{dizionario dati} nel sistema;
  \item Il Tecnico seleziona l’interfaccia di debug.
\end{itemize}

\paragraph*{Postcondizioni}
\begin{itemize}
  \item Il sistema genera una lista dei migliori risultati individuati dal sistema, affiancati allo \glossario{score} ottenuto;
  \item Il sistema genera un log che dia una narrazione della strutturazione del prompt. %da rivedere
\end{itemize}

\paragraph*{Scenario principale}
\begin{enumerate}
  \item Il Tecnico seleziona lo strumento di debug al ricevimento del prompt dal sistema;
  \item Lo strumento di debug genera una lista dei risultati individuati dal sistema;
  \item Lo strumento di debug genera un log descrittivo dei passaggi per la composizione del prompt.  
\end{enumerate}

\paragraph*{Estensione}
\begin{itemize}
  \item Messaggio d'errore nella generazione del prompt (\hyperref[UC6]{UC6}).
\end{itemize}

%Sottocasi d'uso (lista/log)?
