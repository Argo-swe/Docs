\subsubsection{UC14 - Visualizzazione dati dizionario}\label{UC14}
\paragraph*{Descrizione}
Viene mostrato il \glossario{dizionario dati} caricato dal Tecnico, comprendendo tutti i suoi dati in maniera esaustiva, con nome, descrizione, struttura del database, sinonimi, parte di test.

\paragraph*{Attori principali}
Tecnico

\paragraph*{Precondizioni}
\begin{itemize}
  \item Il Tecnico è correttamente autenticato tramite login (\hyperref[UC1]{UC1}).
\end{itemize}

\paragraph*{Postcondizioni}
\begin{itemize}
  \item Vengono visualizzati i dati del \glossario{dizionario dati} scelto.
\end{itemize}

\paragraph*{Scenario principale}
\begin{enumerate}
  \item Il Tecnico naviga alla lista dei dizionari dati;
  \item Il Tecnico seleziona il pulsante per visualizzare i dati di un dizionario;
  \item Il Tecnico visualizza i dati del dizionario.
\end{enumerate}

\paragraph*{Inclusione}
\begin{itemize}
  \item Visualizzazione singolo \glossario{dizionario dati} (\hyperref[UC9]{UC9}).
\end{itemize}

\paragraph*{Sottocasi d'uso}
\begin{itemize}
  \item Visualizzazione tabelle del dizionario (\hyperref[UC14point1]{UC14.1});
  \item Visualizzazione descrizione e sinonimi delle tabelle (\hyperref[UC14point2]{UC14.2});
  \item Visualizzazione sinonimi delle colonne (\hyperref[UC14point3]{UC14.3});
  \item Visualizzazione test del dizionario TODO .  
\end{itemize}