\section{Descrizione}

\subsection{Obiettivi del prodotto}
Lo scopo del prodotto è quello di favorire l’interrogazione di un database di tipo SQL attraverso un’applicazione web, nel quale inserire richieste in linguaggio naturale che verranno poi convertite in \glossario{query} di interrogazione tramite l’utilizzo di un \glossario{LLM}

\subsection{Funzioni del prodotto}
Le funzioni principali necessarie al corretto funzionamento del prodotto sono:
\begin{itemize}
  \item Login: necessaria per identificare lo specifico ruolo dell’utente e quindi abilitarne le funzionalità ad esso dedicate;
  \item Gestione dei dizionari dati: tramite le operazioni \glossario{CRUDL} è possibile gestire i vari dizionari dati utilizzati poi per l’interrogazione;
  \item Selezione del \glossario{dizionario dati} da interrogare;
  \item Selezione della lingua di interrogazione;
  \item Copia del \glossario{prompt} SQL;
  \item Esecuzione di una richiesta con risposta di un report per il debug;
  \item Esecuzione di una batteria di test predefinita nella configurazione del \glossario{dizionario dati}.
\end{itemize}

\subsection{Caratteristiche utente}
Dopo una confronto e analisi con la Proponente, è stato evidenziato un interesse nello sviluppo di questo progetto per varie tipologie di utenti.\\
Potrebbe essere richiesto per:
\begin{itemize}
  \item L'ottimizzazione di \glossario{query} durante lo sviluppo di nuovi programmi;
  \item Generare \glossario{query} per interrogare direttamente il database da parte di utenti con conoscenze di SQL scarse o nulle;
  \item Aiutare i reclutatori a generare \glossario{query} da poter utilizzare come confronto in sede di colloquio per possibile collaboratori.
\end{itemize}

\subsubsection{Utenti di Zucchetti}
TODO
\subsubsection{Utenti del nostro prodotto}
TODO

\subsection{Tecnologie e analisi della struttura di progetto}
TODO

\subsection{Piattaforma di esecuzione}
TODO

