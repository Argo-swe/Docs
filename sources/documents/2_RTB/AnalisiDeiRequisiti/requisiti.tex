\section{Requisiti}
All'interno di questa sezione sarano elencati, in formato tabulare e raggruppati per categoria, i requisiti alla base dello sfiluppo dell'applicativo ChatSQL. Questi sono divisi nelle seguenti tipologie:
\begin{description}
  \item[Funzionali (F):] I requisiti funzionali corrispondono alle funzionalità del sistema. Ciò equivale all’insieme di azioni che l’utente e lo stesso sistema sono in grado di compiere;
  \item[Qualitativi (Q):] I requisiti di qualità si fondano sull'assolvimento degli standard qualitativi al fine di garantire e preservare la qualità del prodotto;
  \item[Di vincolo (V):] I requisiti di vincolo delineano le restrizioni e vincoli normativi da rispettare nel corso dello sviluppo dell’applicativo.
\end{description}
A ciascun requisito è inoltre assegnato un grado di importanza:
\begin{description}
  \item[Obbligatorio (O):] L'implementazione del requisito risulta inderogabile;
  \item[Desiderabile (D):] L'implementazione del requisito non viene specificata come obbligatoria ma risulta appetibile alla \glossario{Proponente}.
  \item[Opzionale (OP):] L'implementazione del requisito è lasciata alla discrezione del \glossario{Fornitore}.
\end{description}
A seguito di tale classificazione, ogni requisito verrà identificato con la forma:
\textbf{\[R.[Tipologia][Importanza][Codice]\]}
Dove \textbf{\(R\)} indica il termine \emph{Requisito}, \emph{Tipologia} e \emph{Importanza} fanno riferimento alle definizioni precedenti e sono indicate con le sigle corrispondenti, infine \emph{Codice} è un identificativo numerico univoco.

\subsection{Requisiti funzionali}
\begin{longtable}{|P{3cm}|P{6,5cm}|>{\arraybackslash}P{4cm}|}
    \hline
    \textbf{Codice} & \textbf{Descrizione} & \textbf{Fonti} \\
    \hline
    \textbf{RF.O.1} & L’Utente deve poter effettuare la procedura di login per passare al profilo di Tecnico. &  \hyperref[UC1]{UC1}, \emph{Verbale Esterno 2024-04-09}, \emph{Verbale Interno 2024-04-20}\\
    \hline
    \textbf{RF.O.1.1} & L’Utente deve poter inserire la propria e-mail per autenticarsi. & \hyperref[UC1point1]{UC1.1}, \emph{Verbale Interno 2024-04-20}\\
    \hline
    \textbf{RF.O.1.2} & L’Utente deve poter inserire la propria password per autenticarsi. & \hyperref[UC1point2]{UC1.2}, \emph{Verbale Interno 2024-04-20}\\
    \hline
    \textbf{RF.O.2} & L'Utente riceve un messaggio d'errore in caso di login errato. &  \hyperref[UC2]{UC2}\\
    \hline
    \textbf{RF.O.2.1} & L'Utente  riceve un messaggio d'errore in caso di incompleto inserimento delle credenziali. & \hyperref[UC2point1]{UC2.1}\\
    \hline
    \textbf{RF.O.2.2} & L’Utente  riceve un messaggio d'errore in caso di credenziali errate. & \hyperref[UC2point2]{UC2.2}\\
    \hline
    \textbf{RF.O.2.3} & L’Utente  riceve un messaggio d'errore nel caso dell'inserimento di una password errata per quell'username & \hyperref[UC2point3]{UC2.3}\\
    \hline
    \textbf{RF.O.3} & L’Utente  deve poter inserire una richiesta in linguaggio naturale. &  \hyperref[UC3]{UC3}, \emph{Capitolato}\\
    \hline
    \textbf{RF.O.4} & L’Utente  deve poter selezionare il \glossario{dizionario dati}. &  \hyperref[UC4]{UC4}, \emph{Capitolato}\\
    \hline
    \textbf{RF.O.5} & L'applicazione deve generare il \glossario{prompt} come output. &  \hyperref[UC5]{UC5}, \emph{Capitolato}\\
    \hline
    \textbf{RF.O.6} & L’Utente  riceve un messaggio d'errore in caso di errore durante la generazione del \glossario{prompt}. &  \hyperref[UC6]{UC6}\\
    \hline
    \textbf{RF.D.7} & L’Utente  deve poter scegliere la lingua in cui inserire la richiesta in linguaggio naturale. &  \hyperref[UC7]{UC7}\\
    \hline
    \textbf{RF.O.8} & L’Utente  deve poter copiare il \glossario{prompt} generato. &  \hyperref[UC8]{UC8}, \emph{Capitolato}\\
    \hline
    \textbf{RF.O.9} & L’Utente  deve poter visualizzare il dettaglio dei \glossario{dizionario dati} caricati, vedendone nome e descrizione. &  \hyperref[UC9]{UC9}\\
    \hline
    \textbf{RF.O.10} & L’Utente  deve poter visualizzare la lista dei \glossario{dizionario dati}. &  \hyperref[UC10]{UC10}\\
    \hline
    \textbf{RF.O.11} & L’Utente  riceve un messaggio d'errore in caso di errore durante il caricamento del dettaglio del\glossario{dizionario dati}. &  \hyperref[UC11]{UC11}\\
    \hline
    \textbf{RF.O.12} & Il Tecnico deve poter effettuare il logout. &  \hyperref[UC12]{UC12}\\
    \hline
    \textbf{RF.O.13} & Il Tecnico deve poter caricare un \glossario{dizionario dati}. &  \hyperref[UC13]{UC13}, \emph{Capitolato}\\
    \hline
    \textbf{RF.O.14} & Il Tecnico deve poter vedere il dettaglio completo del \glossario{dizionario dati}. &  \hyperref[UC14]{UC14}, \emph{Capitolato}\\
    \hline
    \textbf{RF.OP.15} & Il Tecnico deve poter vedere il dettaglio dei test di verifica correttezza del dizionario dati. &  \hyperref[UC15]{UC15}\\
    \hline
    \textbf{RF.OP.15.1} & Il Tecnico deve poter eseguire i test di verifica correttezza basandosi su un reale database &  \hyperref[UC15point1]{UC15.1}\\
    \hline
    \textbf{RF.O.16} & Il Tecnico deve poter modificare il \glossario{dizionario dati}. &  \hyperref[UC16]{UC16}, \emph{Capitolato}\\
    \hline
    \textbf{RF.O.16.1} & Il Tecnico deve poter modificare il nome del \glossario{dizionario dati}. &  \hyperref[UC16point1]{UC16.1}\\
    \hline
    \textbf{RF.O.16.2} & Il Tecnico deve poter modificare la descrizione del \glossario{dizionario dati}. &  \hyperref[UC16point2]{UC16.2}\\
    \hline
    \textbf{RF.O.16.3} & Il Tecnico deve poter modificare il file di configurazinoe del \glossario{dizionario dati}. &  \hyperref[UC16point3]{UC16.3}\\
    \hline
    \textbf{RF.O.17} & Il Tecnico deve poter eliminare un \glossario{dizionario dati}. &  \hyperref[UC17]{UC17}\\
    \hline
    \textbf{RF.O.18} & Il Tecnico deve poter eseguire il debug del \glossario{prompt} &  \hyperref[UC18]{UC18}, \emph{Verbale Esterno 2024-04-09}\\
    \hline
    \textbf{RF.O.19} & Il Tecnico riceve un messaggio d'errore nel caso in cui non riesca a caricare un \glossario{dizionario dati}. & \hyperref[UC19]{UC19} \\
    \hline 
    \textbf{RF.O.20} & Il Tecnico riceve un messaggio d'errore nel caso in cui non riesca a modificare correttamente il \glossario{dizionario dati}. & \hyperref[UC20]{UC20} \\
    \hline
    \textbf{RF.O.20.1} & Il Tecnico riceve un messaggio d'errore nel caso in cui non riesca a modificare correttamente il titolo del \glossario{dizionario dati}. & \hyperref[UC20point1]{UC20.1} \\
    \hline
    \textbf{RF.O.20.2} & Il Tecnico riceve un messaggio d'errore nel caso in cui non riesca a modificare correttamente la descrizione del \glossario{dizionario dati}. & \hyperref[UC20point2]{UC20.2} \\
    \hline
    \textbf{RF.O.21} & Il Tecnico riceve un messaggio d'errore in caso di errore durante l'eliminazione del \glossario{dizionario dati}. &  \hyperref[21]{UC21}\\
    \hline
    \textbf{RF.O.22} & Il Tecnico deve poter scaricare un log dallo strumento di \glossario{debug}&  \hyperref[UC22]{UC22}\\
    \hline
    \textbf{RF.O.23} & Il Tecnico riceve un messaggio d'errore nel caso in cui il download del log non vada a buon fine. &  \hyperref[UC23]{UC23}\\
    \hline
    \textbf{RF.O.24} & Il Tecnico riceve un messaggio d'errore nel caso in cui ci siano anomalie nella generazione della tabella o del log nello strumento di debug. &  \hyperref[UC24]{UC24}\\
    \hline
    \textbf{RF.O.25} & Il Tecnico deve poter visualizzare una tabella con i nomi, le descrizioni e gli score delle tabelle considerate dal \glossario{dizionario dati}. &  \hyperref[UC25]{UC25}\\
    \hline
    \textbf{RF.O.25.1} & Il Tecnico deve poter visualizzare una riga nella tabella di debug &  \hyperref[UC25point1]{UC25.1}\\
    \hline
    \textbf{RF.O.25.1.1} & Il Tecnico deve poter visualizzare il nome nella riga della tabella di debug. &  \hyperref[UC25poin1point1]{UC25.1.1}\\
    \hline
    \textbf{RF.O.25.1.2} & Il Tecnico deve poter visualizzare la descrizione nella riga della tabella di debug. &  \hyperref[UC25poin1point2]{UC25.1.2}\\
    \hline
    \textbf{RF.O.25.1.3} & Il Tecnico deve poter visualizzare lo score nella riga della tabella di debug. &  \hyperref[UC25poin1point3]{UC25.1.3}\\
    \hline
    \textbf{RF.O.26} & Il sistema deve memorizzare i dizionari caricati dal Tecnico. & Capitolato \\
    \hline
    \textbf{RF.D.27} & Il sistema deve memorizzare gli \glossario{embeddings index} creati durante il caricamento o la modifica del dizionario dati. & Verbale Esterno 2024-05-06 \\
    \hline
    \textbf{RF.O.28} & Nel caso in cui un utente formuli una richiesta non pertinente al dizionario dati, il ChatBOT deve fornire una risposta cordiale, invitando l'utente ad attenersi all'argomento del database. & Verbale Interno \\
    \hline
    \textbf{RF.OP.30} & Il prompt generato dall'applicazione deve essere compatibile con sistemi di AI alternativi a ChatGPT. & Capitolato \\
    \hline
    \textbf{RF.OP.31} & Oltre a visualizzare il prompt generato, il ChatBOT mostra all'utente anche la frase SQL prodotta dal sistema di AI. & Capitolato \\
    \hline
    \textbf{RF.OP.32} & La richiesta dell'utente deve poter essere inserita tramite input vocale. & Capitolato \\
    \hline
    \hline
    \textbf{RF.O.33} & Il dizionario dati deve rispecchiare la struttura di un database, e contenere descrizioni in linguaggio naturale delle tabelle e dei rispettivi campi. & Verbale Esterno 2024-04-09 \\
    \hline
    \textbf{RF.O.34} & Il prompt deve contenere almeno le descrizioni delle tabelle e delle colonne. & Verbale esterno 2024-06-07 \\
    \hline
    %\textbf{EX R.FOP19} & Il Tecnico deve poter incollare la \glossario{query} SQL. &  \hyperref[UC19]{UC19}\\ versione vecchia di un caso che non si sa se sia UC o test di unità
\caption{Requisiti funzionali}
\label{requisitifunzionali}
\end{longtable}

\subsection{Requisiti di qualità}
In questa sezione vengono descritti i requisiti volti a garantire la quelità del sistema.

\begin{longtable}{|P{3cm}|P{6,5cm}|>{\arraybackslash}P{4cm}|}
  \hline
  \textbf{Codice} & \textbf{Descrizione} & \textbf{Fonti} \\
  \hline
  \textbf{RQ.O.1} & Rispetto delle norme indicate nel documento \NdP. & \NdP \\
  \hline
  \textbf{RQ.O.2} & Rispetto delle metriche indicate nel documento \PdQ. & \NdP \\
  \hline
  \textbf{RQ.D.3} & Il codice sorgente dell'applicazione deve essere pubblicato sul sito "github.com" o essere accessibile tramite altri \glossario{repository} pubblici. & Capitolato \\
  \hline
  \textbf{RQ.D.4} & Il progetto deve essere di natura open source, per agevolare la continuità del prodotto risultante. & Capitolato \\
  \hline
  \textbf{RQ.O.5} & Deve essere fornito un manuale utente per l'utilizzo dell'applicazione. & Capitolato \\
  \hline
  \textbf{RQ.O.6} & Deve essere fornito un manuale per chiunque voglia estendere l'applicazione. & Capitolato \\
  \hline
  \textbf{RQ.O.7} & Il sistema di generazione del prompt deve avere un'alta recall, per evitare di escludere tabelle rilevanti. & Verbale Esterno 2024-05-22 \\
  \hline
  \textbf{RQ.O.8} & Il prompt deve riportare, per ogni tabella, le informazioni necessarie affinché il modello di AI generi una query SQL corretta. & Verbale Esterno 2024-06-07 \\
  \hline
  \textbf{RQ.O.9} & L'applicazione deve avere un design responsive. & Verbale Interno \\
  \hline
\caption{Requisiti di qualità}
\label{requisitiqualita}
\end{longtable}

\subsection{Requisiti di dominio / vincolo}
In questa sezione vengono descritti i requisiti che pongolo le condizioni da rispettare durante lo sviluppo del sistema.

TODO una volta identificate le tecnologie e versioni (es. python, steamlit, struttura dizionario dati, ...) vanno inserite qui
\begin{longtable}{|P{3cm}|P{6,5cm}|>{\arraybackslash}P{4cm}|}
  \hline
  \textbf{Codice} & \textbf{Descrizione} & \textbf{Fonti} \\
  \hline
  \textbf{RV.O.1} & L'applicazione deve essere organica. Le funzionalità del prodotto devono essere incorporate in un unico sistema che permetta di utilizzarle in modo integrato. & Capitolato \\
  \hline
  \textbf{RV.O.2} & Sviluppo di un'applicazione web. & Verbale esterno 2024-04-09 \\
  \hline
  \textbf{RV.O.3} & Il \glossario{dizionario dati} deve essere fornito tramite un file JSON. & Verbale interno \\
  \hline
  \textbf{RV.O.4} & Il dizionario dati in formato JSON deve rispettare uno schema predefinito. & Verbale Interno \\
  \hline
  \textbf{RV.O.5} & La dimensione del file JSON utilizzato come dizionario dati deve essere inferiore a 1MB. & Verbale Interno \\
  \hline
  \textbf{RV.OP.6} & Il file impiegato come dizionario dati può essere anche un \glossario{dump SQL} (esportabile direttamente da una base di dati). & Verbale Interno \\
  \hline
  \textbf{RV.O.7} & Il \glossario{back-end} dell'applicazione deve essere implementato utilizzando il linguaggio \glossario{Python} versione 3.10.12. & Verbale Interno \\
  \hline
  \textbf{RV.O.8} & Il back-end dell'applicazione deve importare la libreria \glossario{FastAPI} versione 0.110.0 o successive. & Verbale Interno \\
  \hline
  \textbf{RV.O.9} & Il back-end dell'applicazione deve importare la libreria \glossario{txtai} versione 7.0.0 o successive. & Verbale Interno \\
  \hline
  \textbf{RV.O.10} & Il back-end dell'applicazione deve utilizzare modelli di \glossario{sentence similarity} per la generazione degli \glossario{embeddings}. & Verbale Esterno 2024-05-06 \\
  \hline
  \textbf{RV.O.11} & Il \glossario{front-end} dell'applicazione deve essere implementato utilizzando \glossario{Node.js} versione 20.14.0. & Verbale Interno \\
  \hline
  \textbf{RV.O.12} & Il front-end dell'applicazione deve essere implementato utilizzando \glossario{Vue.js} versione 3.4.21. & Verbale Interno \\
  \hline
  \textbf{RV.O.13} & L'applicazione deve essere accessibile e fruibile da dispositivi mobili con sistema operativo Android versione 13 e successive. & Verbale Interno \\
  \hline
  \textbf{RV.O.14} & L'applicazione deve essere accessibile e fruibile da dispositivi mobili con sistema operativo iOS versione 17 e successive. & Verbale Interno \\
  \hline
  \textbf{RV.O.15} & L'applicazione deve essere compatibile con il browser Google Chrome versione 110 e successive. & Verbale Interno \\
  \hline
  \textbf{RV.O.16} & L'applicazione deve essere compatibile con il browser Mozilla Firefox versione 109 e successive. & Verbale Interno \\
  \hline
  \textbf{RV.O.17} & L'applicazione deve essere compatibile con il browser Safari versione 15 e successive. & Verbale Interno \\
  \hline
  \textbf{RV.D.18} & L'applicazione deve essere compatibile con il browser Opera versione 94 e successive. & Verbale Interno \\
  \hline
  \textbf{RV.D.19} & L'applicazione deve essere compatibile con il browser Microsoft Edge versione 110 e successive. & Verbale Interno \\
  \hline
  \textbf{RV.OP.20} & L'applicazione deve essere sviluppata, testata e distribuita mediante \glossario{Docker}. & Verbale Interno \\
  \hline
  \textbf{RV.OP.21} & I colori di primo piano e di sfondo devono avere un rapporto di contrasto di almeno 4.5:1 per testo di dimensioni "normali", e almeno 3:1 per testo grande (linee guida WCAG 2.1, livello di conformità AA). & Verbale Interno \\
  \hline
\caption{Requisiti di vincolo e dominio}
\label{requisitivincolo}
\end{longtable}

\subsection{Fonti - Requisiti}
Di seguito vengono mostrate le corrispondenze fonte - requisito, raggruppate per fonte.

\subsubsection{Fonti - Requisiti funzionali}

\begin{longtable}{|P{4,5cm}|>{\arraybackslash}P{9cm}|}
  \hline
  \textbf{Fonte} & \textbf{Requisiti} \\
  \hline
  % capitolato
  \emph{Capitolato} & RF.O.3, RF.O.4, RF.O.5, RF.O.8, RF.O.13, RF.O.14, RF.O.16, RV.O.1, RV.O.2 \\
  \hline
  % documenti
  \NdP{} & RQ.O.1, RQ.O.2 \\
  \hline
  % verbali
  \emph{Verbale Esterno 2024-04-09} & RF.O.1, RF.O.18 \\
  \hline
  \emph{Verbale Interno 2024-04-20} & RF.O.1, RF.O.1.1, RF.O.1.2,  \\
  \hline
  % casi d'uso
  \hyperref[UC1]{UC1} & RF.O.1 \\
  \hline
  \hyperref[UC1point1]{UC1.1} & RF.O.1.1 \\
  \hline
  \hyperref[UC1]{UC1.2} & R.F.O.1.2 \\
  \hline
  \hyperref[UC2]{UC2} & RF.O.2 \\
  \hline
  \hyperref[UC2point1]{UC2.1} & RF.O.2.1 \\
  \hline
  \hyperref[UC2point2]{UC2.2} & RF.O.2.2 \\
  \hline
  \hyperref[UC2point3]{UC2.3} & RF.O.2.3 \\
  \hline
  \hyperref[UC3]{UC3} & RF.O.3 \\
  \hline
  \hyperref[UC4]{UC4} & RF.O.4 \\
  \hline
  \hyperref[UC5]{UC5} & RF.O.5 \\
  \hline
  \hyperref[UC6]{UC6} & RF.O.6 \\
  \hline
  \hyperref[UC7]{UC7} & RF.D.7 \\
  \hline
  \hyperref[UC8]{UC8} & RF.O.8 \\
  \hline
  \hyperref[UC9]{UC9} & RF.O.9 \\
  \hline
  \hyperref[UC10]{UC10} & RF.O.10 \\
  \hline
  \hyperref[UC11]{UC11} & RF.O.11 \\
  \hline
  \hyperref[UC12]{UC12} & RF.O.12 \\
  \hline
  \hyperref[UC13]{UC13} & RF.O.13 \\
  \hline
  \hyperref[UC14]{UC14} & RF.O.14 \\
  \hline
  \hyperref[UC15]{UC15} & RF.OP.15 \\
  \hline
  \hyperref[UC15point1]{UC15.1} & RF.OP.15.1 \\
  \hline
  \hyperref[UC16]{UC16} & RF.O.16 \\
  \hline
  \hyperref[UC16point1]{UC16.1} & RF.O.16.1 \\
  \hline
  \hyperref[UC16point2]{UC16.2} & RF.O.16.2 \\
  \hline
  \hyperref[UC16point3]{UC16.3} & RF.O.16.3 \\
  \hline
  \hyperref[UC17]{UC17} & RF.O.17 \\
  \hline
  \hyperref[UC18]{UC18} & RF.O.18 \\
  \hline
  \hyperref[UC19]{UC19} & RF.O.19 \\
  \hline
  \hyperref[UC20]{UC20} & RF.O.20 \\
  \hline
  \hyperref[UC20point1]{UC20.1} & RF.O.20.1 \\
  \hline
  \hyperref[UC20point2]{UC20.2} & RF.O.20.2 \\
  \hline
  \hyperref[UC21]{UC21} & RF.O.21 \\
  \hline
  \hyperref[UC22]{UC22} & RF.O.22 \\
  \hline
  \hyperref[UC23]{UC23} & RF.O.23 \\
  \hline
  \hyperref[UC24]{UC24} & RF.O.24 \\
  \hline
  \hyperref[UC25]{UC25} & RF.O.25 \\
  \hline
  \hyperref[UC25point1]{UC25.1} & RF.O.25.1 \\
  \hline
  \hyperref[UC25point1point1]{UC25.1.1} & RF.O.25.1.1 \\
  \hline
  \hyperref[UC25poin1point2]{UC25.1.2} & RF.O.25.1.2 \\
  \hline
  \hyperref[UC25poin1point3]{UC25.1.3} & RF.O.25.1.3 \\
  \hline
\caption{Fonti- Requisiti funzionali}
\end{longtable}
