\section{Requisiti}
All'interno di questa sezione sarano elencati, in formato tabulare e raggruppati per categoria, i requisiti alla base dello sfiluppo dell'applicativo ChatSQL. Questi sono divisi nelle seguenti tipologie:
\begin{description}
  \item[Funzionali (F):] I requisiti funzionali corrispondono alle funzionalità del sistema. Ciò equivale all’insieme di azioni che l’utente e lo stesso sistema sono in grado di compiere;
  \item[Qualitativi (Q):] I requisiti di qualità si fondano sull'assolvimento degli standard qualitativi al fine di garantire e preservare la qualità del prodotto;
  \item[Di vincolo (V):] I requisiti di vincolo delineano le restrizioni e vincoli normativi da rispettare nel corso dello sviluppo dell’applicativo.
\end{description}
A ciascun requisito è inoltre assegnato un grado di importanza:
\begin{description}
  \item[Obbligatorio (O):] L'implementazione del requisito risulta inderogabile;
  \item[Desiderabile (D):] L'implementazione del requisito non viene specificata come obbligatoria ma risulta appetibile alla \glossario{Proponente}.
  \item[Opzionale (OP):] L'implementazione del requisito è lasciata alla discrezione del \glossario{Fornitore}.
\end{description}
A seguito di tale classificazione, ogni requisito verrà identificato con la forma:
\textbf{\[R.[Tipologia][Importanza][Codice]\]}
Dove \textbf{\(R\)} indica il termine \emph{Requisito}, \emph{Tipologia} e \emph{Importanza} fanno riferimento alle definizioni precedenti e sono indicate con le sigle corrispondenti, infine \emph{Codice} è un identificativo numerico univoco.

\subsection{Requisiti funzionali}
\begin{longtable}{|P{3cm}|P{6,5cm}|>{\arraybackslash}P{4cm}|}
    \hline
    \textbf{Codice} & \textbf{Descrizione} & \textbf{Fonti} \\
    \hline
    \textbf{RF.O.1} & L'Utente deve poter effettuare il login per passare al profilo di Tecnico. & \hyperref[UC1]{UC1}, Verbale Esterno, Verbale Interno\\
    \hline
    \textbf{RF.O.1.1} & L'Utente deve inserire uno username in fase di autenticazione. & \hyperref[UC1point1]{UC1.1}, Verbale Interno\\
    \hline
    \textbf{RF.O.1.2} & L'Utente deve inserire una password in fase di autenticazione. & \hyperref[UC1point2]{UC1.2}, Verbale Interno\\
    \hline
    \textbf{RF.O.2} & Il sistema deve restituire un messaggio d'errore in caso di fallimento del login. & \hyperref[UC2]{UC2}\\
    \hline
    \textbf{RF.O.2.1} & L'Utente deve visualizzare un messaggio di errore se inserisce delle credenziali incomplete. & \hyperref[UC2point1]{UC2.1}\\
    \hline
    \textbf{RF.O.2.2} & L'Utente deve visualizzare un messaggio di errore se inserisce delle credenziali errate. & \hyperref[UC2point2]{UC2.2}\\
    \hline
    \textbf{RF.O.3} & L'Utente deve poter inserire una richiesta in linguaggio naturale. & \hyperref[UC3]{UC3}, Capitolato\\
    \hline
    \textbf{RF.O.4} & L'Utente deve poter selezionare un \glossario{dizionario dati} da utilizzare nel sistema. & \hyperref[UC4]{UC4}, Capitolato\\
    \hline
    \textbf{RF.O.5} & Il sistema deve generare un \glossario{prompt} in risposta alla richiesta dell'Utente. & \hyperref[UC5]{UC5}, Capitolato\\
    \hline
    \textbf{RF.O.6} & L'Utente deve visualizzare un avviso in caso venga formulata una richiesta non idonea. & \hyperref[UC6]{UC6}\\
    \hline
    \textbf{RF.D.7} & L'Utente deve poter scegliere la lingua con cui effettuare una richiesta al ChatBOT. & \hyperref[UC7]{UC7}\\
    \hline
    \textbf{RF.O.8} & L'Utente deve poter copiare il contenuto del \glossario{prompt} generato. Il prompt può essere fornito in input a un \glossario{LLM} esterno per convertire la frase dell'Utente in una query \glossario{SQL}. & \hyperref[UC8]{UC8}, Capitolato\\
    \hline
    \textbf{RF.O.9} & L'Utente deve poter visualizzare la lista dei \glossario{dizionario dati} caricati nel sistema. & \hyperref[UC10]{UC10}\\
    \hline
    \textbf{RF.O.9.1} & L'Utente deve poter visualizzare un singolo \glossario{dizionario dati} all'interno della lista. & \hyperref[UC9point1]{UC9.1}\\
    \hline
    \textbf{RF.O.10} & L'Utente deve poter visualizzare le caratteristiche di un \glossario{dizionario dati}. & \hyperref[UC10]{UC10}\\
    \hline
    \textbf{RF.O.10.1} & L'Utente deve poter visualizzare il nome di un \glossario{dizionario dati}. & \hyperref[UC10point1]{UC10.1}\\
    \hline
    \textbf{RF.O.10.2} & L'Utente deve poter visualizzare l'estensione del file scelto come \glossario{dizionario dati}. & \hyperref[UC10point2]{UC10.2}\\
    \hline
    \textbf{RF.O.10.3} & L'Utente deve poter visualizzare la descrizione di un \glossario{dizionario dati}. & \hyperref[UC10point3]{UC10.3}\\
    \hline
    \textbf{RF.O.10.4} & L'Utente deve poter visualizzare la dimensione del file scelto come \glossario{dizionario dati}. & \hyperref[UC10point4]{UC10.4}\\
    \hline
    \textbf{RF.O.10.5} & L'Utente deve poter visualizzare la data di ultimo aggiornamento di un \glossario{dizionario dati}. & \hyperref[UC10point5]{UC10.5}\\
    \hline
    \textbf{RF.O.11} & L'Utente deve visualizzare un messaggio esplicativo nel caso in cui si verifichi un errore durante la generazione del \glossario{prompt}. & \hyperref[UC11]{UC11}\\
    \hline
    \textbf{RF.O.12} & Il Tecnico deve poter effettuare il logout per terminare la sessione corrente. & \hyperref[UC12]{UC12}\\
    \hline
    \textbf{RF.O.13} & Il Tecnico deve poter inserire un nuovo \glossario{dizionario dati} all'interno del sistema. & \hyperref[UC13]{UC13}\\
    \hline
    \textbf{RF.O.14} & L'Utente deve poter visualizzare il contenuto testuale di un \glossario{dizionario dati}. & \hyperref[UC14]{UC14}, Verbale Esterno\\
    \hline
    \textbf{RF.O.14.1} & L'Utente deve poter visualizzare il nome del database riportato nel dizionario dati. & \hyperref[UC14point1]{UC14.1}\\
    \hline
    \textbf{RF.O.14.2} & L'Utente deve poter visualizzare la descrizione del database riportato nel dizionario dati. & \hyperref[UC14point2]{UC14.2}\\
    \hline
    \textbf{RF.O.14.3} & L'Utente deve poter visualizzare la lista delle tabelle descritte nel dizionario dati. & \hyperref[UC14point3]{UC14.3}\\
    \hline
    \textbf{RF.O.14.3.1} & L'Utente deve poter visualizzare il nome di una singola tabella tra quelle descritte nel dizionario dati. & \hyperref[UC14point3point1]{UC14.3.1}\\
    \hline
    \textbf{RF.O.14.3.2} & L'Utente deve poter visualizzare la descrizione di una singola tabella tra quelle descritte nel dizionario dati. & \hyperref[UC14point3point2]{UC14.3.2}\\
    \hline
    \textbf{RF.OP.15} & Il Tecnico deve poter caricare il file relativo a un \glossario{dizionario dati}. & \hyperref[UC15]{UC15}\\
    \hline
    \textbf{RF.O.16} & Il Tecnico deve poter inserire il nome di un \glossario{dizionario dati}. & \hyperref[UC16]{UC16}, \emph{Capitolato}\\
    \hline
    \textbf{RF.O.17} & Il Tecnico deve poter inserire la descrizione di un \glossario{dizionario dati}. & \hyperref[UC17]{UC17}\\
    \hline
    \textbf{RF.O.18} & Il Tecnico deve poter eliminare un \glossario{dizionario dati} dal sistema. & \hyperref[UC18]{UC18}\\
    \hline
    \textbf{RF.O.19} & Il sistema restituisce un messaggio d'errore qualora non riesca a validare il file scelto come \glossario{dizionario dati}. & \hyperref[UC19]{UC19} \\
    \hline 
    \textbf{RF.O.20} & Il Tecnico deve poter sovrascrivere il file di configurazione di un \glossario{dizionario dati}. & \hyperref[UC20]{UC20} \\
    \hline
    \textbf{RF.O.21} & Il Tecnico deve visualizzare un errore se l'eliminazione di un \glossario{dizionario dati} fallisce. & \hyperref[21]{UC21}\\
    \hline
    \textbf{RF.O.22} & Il Tecnico deve poter visualizzare l'ultimo messaggio di \glossario{debug}. Il debug illustra il processo di generazione del prompt. & \hyperref[22]{UC22}\\
    \hline
    \textbf{RF.O.23} & Il Tecnico deve poter scaricare il file di \glossario{log} creato durante la generazione del prompt. & \hyperref[23]{UC23}\\
    \hline
    \textbf{RF.O.24} & Il Tecnico deve visualizzare un errore nel caso in cui il download del file non vada a buon fine. &  \hyperref[UC24]{UC24}\\
    \hline
    \textbf{RF.O.25} & L'Utente deve poter visualizzare il contenuto della chat. &  \hyperref[UC25]{UC25}\\
    \hline
    \textbf{RF.O.25.1} & L'Utente deve poter visualizzare un singolo messaggio nella chat. &  \hyperref[UC25point1]{UC25.1}\\
    \hline
    \textbf{RF.O.25.1.1} & L'Utente deve poter visualizzare il contenuto di un singolo messaggio nella chat. &  \hyperref[UC25point1point1]{UC25.1.1}\\
    \hline
    \textbf{RF.O.25.1.2} & L'Utente deve poter visualizzare il mittente di un messaggio. &  \hyperref[UC25poin1point2]{UC25.1.2}\\
    \hline
    \textbf{RF.O.26} & L'Utente deve poter selezionare un \glossario{DBMS} tra quelli disponibili. & \hyperref[UC26]{UC26} \\
    \hline
    \textbf{RF.O.27} & L'Utente deve poter eliminare la cronologia della chat. & \hyperref[UC27]{UC27} \\
    \hline
    \textbf{RF.O.28} & Il sistema restituisce un messaggio d'errore se il file inserito dal Tecnico è in un formato non supportato. & \hyperref[UC28]{UC28} \\
    \hline
    \textbf{RF.O.29} & Il Tecnico deve poter modificare il nome di un \glossario{dizionario dati}. & \hyperref[UC29]{UC29} \\
    \hline
    \textbf{RF.O.30} & Il Tecnico deve poter modificare la descrizione di un \glossario{dizionario dati}. & \hyperref[UC30]{UC30} \\
    \hline
    \textbf{RF.O.31} & Il sistema restituisce un messaggio di errore qualora non riesca a validare il nome di un \glossario{dizionario dati}. & \hyperref[UC31]{UC31} \\
    \hline
    \textbf{RF.O.31.1} & Il sistema restituisce un messaggio di errore nel caso in cui il formato del nome del \glossario{dizionario dati} non sia valido. & \hyperref[UC31point1]{UC31.1} \\
    \hline
    \textbf{RF.O.31.2} & Il sistema restituisce un messaggio di errore nel caso in cui il nome del \glossario{dizionario dati} non sia univoco. & \hyperref[UC31point2]{UC31.2} \\
    \hline
    \textbf{RF.O.32} & Il sistema restituisce un messaggio di errore qualora non riesca a validare la descrizione di un \glossario{dizionario dati}. & \hyperref[UC32]{UC32} \\
    \hline
    \textbf{RF.O.33} & Il sistema restituisce un messaggio di errore se il file inserito dal Tecnico ha una dimensione superiore a 1 MB. & \hyperref[UC33]{UC33} \\
    \hline
    \textbf{RF.O.34} & Il sistema restituisce un messaggio di errore se il file inserito dal Tecnico non è conforme allo schema predefinito. & \hyperref[UC34]{UC34} \\
    \hline
    \textbf{RF.O.35} & L'Utente deve poter visualizzare il contenuto del \glossario{prompt} generato. & \hyperref[UC35]{UC35} \\
    \hline
    \textbf{RF.O.36} & L'Utente deve poter effettuare una ricerca tra i \glossario{dizionari dati} caricati nel sistema. & \hyperref[UC36]{UC36} \\
    \hline
    \textbf{RF.O.37} & Il sistema deve memorizzare i dizionari caricati dal Tecnico. & Capitolato \\
    \hline
    \textbf{RF.D.38} & Il sistema deve memorizzare gli \glossario{indici vettoriali} creati durante il caricamento o la modifica del dizionario dati. & Verbale Esterno\\
    \hline
    \textbf{RF.D.39} & Nel caso in cui un utente formuli una richiesta non idonea, il ChatBOT deve fornire una risposta cordiale, invitando l'utente ad attenersi all'argomento del \glossario{dizionario dati}. & Verbale Interno \\
    \hline
    \textbf{RF.OP.40} & Il prompt generato dall'applicazione deve essere compatibile con sistemi di AI alternativi a ChatGPT. & Capitolato \\
    \hline
    \textbf{RF.OP.41} & Oltre a visualizzare il prompt generato, il ChatBOT mostra all'utente anche la frase SQL prodotta dal sistema di \glossario{AI}. & Capitolato \\
    \hline
    \textbf{RF.OP.42} & La richiesta dell'utente deve poter essere inserita tramite input vocale. & Capitolato \\
    \hline
    \textbf{RF.O.43} & Il dizionario dati deve rispecchiare la struttura di un database, e contenere descrizioni in linguaggio naturale delle tabelle e dei rispettivi campi. & Verbale Esterno \\
    \hline
    \textbf{RF.O.44} & Il prompt deve contenere almeno le descrizioni delle tabelle e delle colonne. & Verbale esterno\\
    \hline
    \textbf{RF.D.45} & La cronologia della chat deve essere eliminata quando viene attivato un nuovo \glossario{dizionario dati} nel sistema. & Verbale interno \\
    \hline
    \textbf{RF.O.46} & Quando il Tecnico richiede la generazione del prompt, il sistema elabora in automatico un file di \glossario{log}. & Verbale esterno \\
    \hline
    \textbf{RF.OP.47} & Il sistema di generazione del prompt deve supportare richieste in lingua inglese. & Verbale interno \\
    \hline
    \textbf{RF.O.48} & La scelta del dizionario dati deve essere disponibile nella pagina di generazione del prompt. & Verbale interno \\
    \hline
    \textbf{RF.O.49} & Per procedere con l'inserimento di una richiesta, è necessario aver selezionato un dizionario dati. & Verbale interno \\
    \hline
    \textbf{RF.O.50} & La scelta della lingua deve essere disponibile nella pagina di generazione del prompt. & Verbale interno \\
    \hline
    \textbf{RF.O.51} & La scelta del DBMS deve essere disponibile nella pagina di generazione del prompt. & Verbale interno \\
    \hline
    \textbf{RF.D.52} & Il sistema di generazione del prompt deve supportare richieste in lingue diverse dall'inglese. & Capitolato \\
    \hline
    \textbf{RF.OP.53} & Quando un utente ripete più volte lo stesso messaggio, il ChatBOT restituisce la risposta formulata alla prima occorrenza del messaggio. & Verbale interno \\
    \hline
\caption{Requisiti funzionali}
\label{requisitifunzionali}
\end{longtable}

\subsection{Requisiti di qualità}
In questa sezione vengono descritti i requisiti volti a garantire la quelità del sistema.

\begin{longtable}{|P{3cm}|P{6,5cm}|>{\arraybackslash}P{4cm}|}
  \hline
  \textbf{Codice} & \textbf{Descrizione} & \textbf{Fonti} \\
  \hline
  \textbf{RQ.O.1} & Rispetto delle norme indicate nel documento \NdP. & \NdP \\
  \hline
  \textbf{RQ.O.2} & Rispetto delle metriche indicate nel documento \PdQ. & \NdP \\
  \hline
  \textbf{RQ.D.3} & Il codice sorgente dell'applicazione deve essere pubblicato sul sito "github.com" o essere accessibile tramite altri \glossario{repository} pubblici. & Capitolato \\
  \hline
  \textbf{RQ.D.4} & Il progetto deve essere di natura open source, per agevolare la continuità del prodotto risultante. & Capitolato \\
  \hline
  \textbf{RQ.O.5} & Deve essere fornito un manuale utente per l'utilizzo dell'applicazione. & Capitolato \\
  \hline
  \textbf{RQ.O.6} & Deve essere fornito un manuale per chiunque voglia estendere l'applicazione. & Capitolato \\
  \hline
  \textbf{RQ.D.7} & Il sistema di generazione del prompt deve avere un'alta recall, per evitare di escludere tabelle rilevanti. & Verbale Esterno 2024-05-22 \\
  \hline
  \textbf{RQ.O.8} & Il prompt deve riportare, per ogni tabella, le informazioni necessarie affinché il modello di AI generi una query SQL corretta. & Verbale Esterno 2024-06-07 \\
  \hline
  \textbf{RQ.O.9} & L'applicazione deve avere un design responsive. & Verbale Interno \\
  \hline
  \textbf{RQ.OP.10} & Il sistema deve accertarsi che la query \glossario{SQL} ricevuta da un servizio esterno sia valida e funzionante. & Capitolato \\
  \hline
\caption{Requisiti di qualità}
\label{requisitiqualita}
\end{longtable}

\subsection{Requisiti di dominio / vincolo}
In questa sezione vengono descritti i requisiti che pongolo le condizioni da rispettare durante lo sviluppo del sistema.

TODO una volta identificate le tecnologie e versioni (es. python, steamlit, struttura dizionario dati, ...) vanno inserite qui
\begin{longtable}{|P{3cm}|P{6,5cm}|>{\arraybackslash}P{4cm}|}
  \hline
  \textbf{Codice} & \textbf{Descrizione} & \textbf{Fonti} \\
  \hline
  \textbf{RV.O.1} & L'applicazione deve essere organica. Le funzionalità del prodotto devono essere incorporate in un unico sistema che permetta di utilizzarle in modo integrato. & Capitolato \\
  \hline
  \textbf{RV.O.2} & Sviluppo di un'applicazione web. & Verbale esterno 2024-04-09 \\
  \hline
  \textbf{RV.O.3} & Il \glossario{dizionario dati} deve essere fornito tramite un file JSON. & Verbale interno \\
  \hline
  \textbf{RV.O.4} & Il dizionario dati in formato JSON deve rispettare uno schema predefinito. & Verbale Interno \\
  \hline
  \textbf{RV.O.5} & La dimensione del file JSON utilizzato come dizionario dati deve essere inferiore a 1MB. & Verbale Interno \\
  \hline
  \textbf{RV.OP.6} & Il file impiegato come dizionario dati può essere anche un \glossario{dump SQL} (esportabile direttamente da una base di dati). & Verbale Interno \\
  \hline
  \textbf{RV.O.7} & Il \glossario{back-end} dell'applicazione deve essere implementato utilizzando il linguaggio \glossario{Python} versione 3.10.12. & Verbale Interno \\
  \hline
  \textbf{RV.O.8} & Il back-end dell'applicazione deve importare la libreria \glossario{FastAPI} versione 0.110.0 o successive. & Verbale Interno \\
  \hline
  \textbf{RV.O.9} & Il back-end dell'applicazione deve importare la libreria \glossario{txtai} versione 7.0.0 o successive. & Verbale Interno \\
  \hline
  \textbf{RV.O.10} & Il back-end dell'applicazione deve utilizzare modelli di \glossario{sentence similarity} per la generazione degli \glossario{embeddings}. & Verbale Esterno 2024-05-06 \\
  \hline
  \textbf{RV.O.11} & Il \glossario{front-end} dell'applicazione deve essere implementato utilizzando \glossario{Node.js} versione 20.14.0. & Verbale Interno \\
  \hline
  \textbf{RV.O.12} & Il front-end dell'applicazione deve essere implementato utilizzando \glossario{Vue.js} versione 3.4.21. & Verbale Interno \\
  \hline
  \textbf{RV.O.13} & L'applicazione deve essere accessibile e fruibile da dispositivi mobili con sistema operativo Android versione 13 e successive. & Verbale Interno \\
  \hline
  \textbf{RV.O.14} & L'applicazione deve essere accessibile e fruibile da dispositivi mobili con sistema operativo iOS versione 17 e successive. & Verbale Interno \\
  \hline
  \textbf{RV.O.15} & L'applicazione deve essere compatibile con il browser Google Chrome versione 110 e successive. & Verbale Interno \\
  \hline
  \textbf{RV.O.16} & L'applicazione deve essere compatibile con il browser Mozilla Firefox versione 109 e successive. & Verbale Interno \\
  \hline
  \textbf{RV.O.17} & L'applicazione deve essere compatibile con il browser Safari versione 15 e successive. & Verbale Interno \\
  \hline
  \textbf{RV.D.18} & L'applicazione deve essere compatibile con il browser Opera versione 94 e successive. & Verbale Interno \\
  \hline
  \textbf{RV.D.19} & L'applicazione deve essere compatibile con il browser Microsoft Edge versione 110 e successive. & Verbale Interno \\
  \hline
  \textbf{RV.OP.20} & L'applicazione deve essere sviluppata, testata e distribuita mediante \glossario{Docker}. & Verbale Interno \\
  \hline
  \textbf{RV.OP.21} & I colori di primo piano e di sfondo devono avere un rapporto di contrasto di almeno 4.5:1 per testo di dimensioni "normali", e almeno 3:1 per testo grande (linee guida WCAG 2.1, livello di conformità AA). & Verbale Interno \\
  \hline
\caption{Requisiti di vincolo e dominio}
\label{requisitivincolo}
\end{longtable}

\subsection{Fonti - Requisiti}
Di seguito vengono mostrate le corrispondenze fonte - requisito, raggruppate per fonte.

\subsubsection{Fonti - Requisiti funzionali}

\begin{longtable}{|P{4,5cm}|>{\arraybackslash}P{9cm}|}
  \hline
  \textbf{Fonte} & \textbf{Requisiti} \\
  \hline
  % capitolato
  \emph{Capitolato} & RF.O.3, RF.O.4, RF.O.5, RF.O.8, RF.O.16, RF.O.37, RF.OP.40, RF.O.41, RF.OP.42, RF.D.52, RQ.D.3, RQ.D.4, RQ.O.5, RQ.O.6, RQ.OP.10, RV.O.1,  \\
  \hline
  % documenti
  \NdP{} & RQ.O.1, RQ.O.2 \\
  \hline
  % verbali
  \emph{Verbali Esterni} & RF.O.1, RF.O.14, RF.D.38, RF.O.43, RF.O.44, RF.O.46, RF.O.48, RF.O.49, RF.O.50, RF.O.51, RF.O.53, RQ.D.7, RQ.O.8, RV.O.2, RV.O.10, \\
  \hline
  \emph{Verbali Interni} & RF.O.1, RF.O.1.1, RF.O.1.2, RF.D.39, RF.D.45, RF.OP.47, RQ.O.9, RV.O.3, RV.O.4, RV.O.5, RV.OP.6, RV.O.7, RV.O.8, RV.O.9, RV.O.11, RV.O.12, RV.O.13, RV.O.14, RV.O.15, RV.O.16, RV.O.17, RV.D.18, RV.D.19, RV.OP.20, RV.OP.21 \\
  \hline
  % casi d'uso
  \hyperref[UC1]{UC1} & RF.O.1 \\
  \hline
  \hyperref[UC1point1]{UC1.1} & RF.O.1.1 \\
  \hline
  \hyperref[UC1]{UC1.2} & R.F.O.1.2 \\
  \hline
  \hyperref[UC2]{UC2} & RF.O.2 \\
  \hline
  \hyperref[UC2point1]{UC2.1} & RF.O.2.1 \\
  \hline
  \hyperref[UC2point2]{UC2.2} & RF.O.2.2 \\
  \hline
  \hyperref[UC3]{UC3} & RF.O.3 \\
  \hline
  \hyperref[UC4]{UC4} & RF.O.4 \\
  \hline
  \hyperref[UC5]{UC5} & RF.O.5 \\
  \hline
  \hyperref[UC6]{UC6} & RF.O.6 \\
  \hline
  \hyperref[UC7]{UC7} & RF.D.7 \\
  \hline
  \hyperref[UC8]{UC8} & RF.O.8 \\
  \hline
  \hyperref[UC9]{UC9} & RF.O.9 \\
  \hline
  \hyperref[UC9point1]{UC9.1} & RF.O.9.1 \\
  \hline
  \hyperref[UC10]{UC10} & RF.O.10 \\
  \hline
  \hyperref[UC10point1]{UC10.1} & RF.O.10.1 \\ 
  \hline
  \hyperref[UC10point2]{UC10.2} & RF.O.10.2\\
  \hline
  \hyperref[UC10point3]{UC10.3} & RF.O.10.3\\
  \hline
  \hyperref[UC10point4]{UC10.4} & RF.O.10.4\\
  \hline
  \hyperref[UC10point5]{UC10.5} & RF.O.10.5\\
  \hline
  \hyperref[UC11]{UC11} & RF.O.11 \\
  \hline
  \hyperref[UC12]{UC12} & RF.O.12 \\
  \hline
  \hyperref[UC13]{UC13} & RF.O.13 \\
  \hline
  \hyperref[UC14]{UC14} & RF.O.14 \\
  \hline
  \hyperref[UC14point1]{UC14.1} & RF.O.14.1 \\
  \hline
  \hyperref[UC14point2]{UC14.2} & RF.O.14.2 \\
  \hline
  \hyperref[UC14point3]{UC14.3} & RF.O.14.3 \\
  \hline
  \hyperref[UC14point3point1]{UC14.3.1} & RF.O.14.3.1 \\
  \hline
  \hyperref[UC14point3point2]{UC14.3.2} & RF.O.14.3.2 \\
  \hline
  \hyperref[UC15]{UC15} & RF.OP.15 \\
  \hline
  \hyperref[UC16]{UC16} & RF.O.16 \\
  \hline
  \hyperref[UC17]{UC17} & RF.O.17 \\
  \hline
  \hyperref[UC18]{UC18} & RF.O.18 \\
  \hline
  \hyperref[UC19]{UC19} & RF.O.19 \\
  \hline
  \hyperref[UC20]{UC20} & RF.O.20 \\
  \hline
  \hyperref[UC21]{UC21} & RF.O.21 \\
  \hline
  \hyperref[UC22]{UC22} & RF.O.22 \\
  \hline
  \hyperref[UC23]{UC23} & RF.O.23 \\
  \hline
  \hyperref[UC24]{UC24} & RF.O.24 \\
  \hline
  \hyperref[UC25]{UC25} & RF.O.25 \\
  \hline
  \hyperref[UC25point1]{UC25.1} & RF.O.25.1 \\
  \hline
  \hyperref[UC25point1point1]{UC25.1.1} & RF.O.25.1.1 \\
  \hline
  \hyperref[UC25poin1point2]{UC25.1.2} & RF.O.25.1.2 \\
  \hline
  \hyperref[UC26]{UC26} & RF.O.26 \\
  \hline
  \hyperref[UC27]{UC27} & RF.O.27 \\
  \hline
  \hyperref[UC28]{UC28} & RF.O.28 \\
  \hline
  \hyperref[UC29]{UC29} & RF.O.29 \\
  \hline
  \hyperref[UC30]{UC30} & RF.O.30 \\
  \hline
  \hyperref[UC31]{UC31} & RF.O.31 \\
  \hline
  \hyperref[UC31point1]{UC31.1} & RF.O.31.1 \\
  \hline
  \hyperref[UC31point2]{UC31.2} & RF.O.31.2 \\
  \hline
  \hyperref[UC32]{UC32} & RF.O.32 \\
  \hline
  \hyperref[UC33]{UC33} & RF.O.33 \\
  \hline
  \hyperref[UC34]{UC34} & RF.O.34 \\
  \hline
  \hyperref[UC35]{UC35} & RF.O.35\\
  \hline
  \hyperref[UC36]{UC36} & RF.O.36 \\
  \hline
\caption{Fonti- Requisiti funzionali}
\end{longtable}
