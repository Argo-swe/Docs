\section{Riunione}
\subsection{Ordine del giorno}
\begin{itemize}
	\item Discussione per i punti del diario di bordo;
	\item Presentazione della pianificazione del prossimo sprint;
	\item Preparazione incontro con la Proponente;
	\item Pianificazione di un incontro per allineamento su \glossario{txtai};
\end{itemize}

\subsection{Discussione e decisioni}
\subsubsection{Pianificazione sprint 4}
Viene presentata una bozza di pianificazione dello sprint 4, che il gruppo approva senza apportare ulteriori modifiche.

\subsubsection{Preparazione incontro con la Proponente}
In avvicinamento alla fine dello sprint il gruppo si accorda sugli argomenti principali da discutere durante l'incontro con la Proponente. Gli argomenti ritenuti di maggior interesse sono:
\begin{itemize}
	\item Presentazione sulla libreria Streamlit;
	\item Discussione delle metriche di prodotto individuate, con ricerca di metriche d'interesse per la Proponente non individuate.
\end{itemize}

\subsubsection{Allineamento sulle tecnologie}
Visto il rapido avanzamento e l'introduzione di librerie complesse avvenuto nel corso degli ultimi due sprint, viene valutato un incontro per allineare tutto il gruppo nell'utilizzo e funzione di txtai e Streamlit. Il gruppo stabilisce Sabato 18 come giornata per riunirsi.

\subsubsection{Punti da trattare nel prossimo diario di bordo}
\par Il gruppo ha discusso i punti salienti da trattare nel diario di bordo del 17 maggio 2024.
\paragraph{Done}
\begin{itemize}
	\item Modifica della struttura del \PdP;
	\item Completata la bozza dell'interfaccia grafica;
	\item Sviluppo di test con txtai per accuratezza query-risultato;
	\item Ottimizzazione delle modalità di generazione del prompt;
	\item Esplorazione tecnologie alternative per sviluppo frontend e backend;
	\item Aggiornamento della documentazione (\PdQ, \NdP, \AdR);
	\item Pianificazione e preventivo dello sprint 4;
	\item Inizio dello sviluppo della web application;
	\item Organizzazione incontro per presentare le tecnologie e delle metriche di valutazione individuate.
\end{itemize}
\paragraph{Difficoltà}
\begin{itemize}
	\item Benchmarking dei \glossario{Large Language Models};
	\item Definizione di categorie per le metriche di qualità;
	\item Paragone tra framework alternativi per la produzione di web application.
\end{itemize}	
\paragraph{Todo}
\begin{itemize}
	\item Sviluppo delle funzionalità di supporto del backend;
	\item Sviluppo dei casi d'uso all'interno dell'applicativo;
	\item Aggiornamento della documentazione;
	\item Incontro per allineare tutto il gruppo sulle tecnologie utilizzate;
	\item Consuntivo sprint 3;
	\item Applicazione delle metriche di qualità nel \WoW.
\end{itemize}
\paragraph{Dubbi}
\begin{itemize}
	\item Persistenza delle tecnologie adottate nel PoC ma valutate inadeguate in seguito;
	\item Individuazione di standard ben riconosciuti per la valutazione della qualità.
\end{itemize}

\par Gli argomenti sopracitati sono stati riportati nella presentazione per il diario di bordo del 17 maggio 2024.
