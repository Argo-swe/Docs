\section{Riunione}
\subsection{Ordine del giorno}
\begin{itemize}
	\item Discussione riguardante i punti del diario di bordo;
	\item Pianificazione del prossimo \glossario{sprint};
	\item Preparazione incontro con la \glossario{Proponente};
	\item Pianificazione incontro per allineamento su \glossario{txtai}.
\end{itemize}

\subsection{Discussione e decisioni}
\subsubsection{Pianificazione sprint 4}
\par È stata presentata una bozza di pianificazione per il quarto \glossario{sprint}, che il gruppo ha approvato senza apportare ulteriori modifiche.

\subsubsection{Preparazione incontro con la Proponente}
\par In avvicinamento alla fine dello \glossario{sprint}, il gruppo si è accordato sugli argomenti principali da discutere durante l'incontro con la \glossario{Proponente}. Gli argomenti ritenuti di maggior interesse sono stati:
\begin{itemize}
	\item Presentazione dei pro e contro del framework \glossario{Streamlit};
	\item Discussione sulle metriche di prodotto e ricerca di quelle di maggior interesse per la Proponente.
\end{itemize}

\subsubsection{Allineamento sulle tecnologie}
\par Considerata l'introduzione di librerie complesse durante gli ultimi due sprint, è stato valutato un incontro per allineare il gruppo sull'utilizzo e le funzionalità di \glossario{txtai} e \glossario{Streamlit}. Il gruppo ha fissato il workshop per sabato 18.

\subsubsection{Punti da trattare nel prossimo diario di bordo}
\par Il gruppo ha discusso i punti salienti da trattare nel diario di bordo del 17 maggio.
\paragraph{Done}
\begin{itemize}
	\item Modifica della struttura del \PdP;
	\item Ultimazione della bozza dell'interfaccia grafica;
	\item Sviluppo di test con \glossario{txtai} per verificare l'accuratezza query-risultato;
	\item Ottimizzazione delle modalità di generazione del \glossario{prompt};
	\item Esplorazione di tecnologie alternative per sviluppo front-end e back-end;
	\item Aggiornamento della documentazione (\PdQ, \NdP, \AdR);
	\item Pianificazione e preventivo dello sprint 4;
	\item Inizio dello sviluppo della web app;
	\item Organizzazione incontro con la \glossario{Proponente} per presentare le tecnologie.
\end{itemize}

\paragraph{Difficoltà}
\begin{itemize}
	\item Benchmark dei \glossario{Large Language Model};
	\item Definizione di categorie per le metriche di qualità;
	\item Confronto tra framework alternativi per lo sviluppo di web app.
\end{itemize}

\paragraph{Todo}
\begin{itemize}
	\item Sviluppo delle funzionalità di supporto del back-end;
	\item Sviluppo dei casi d'uso all'interno dell'applicativo;
	\item Aggiornamento della documentazione;
	\item Incontro per allineare il gruppo sulle tecnologie utilizzate;
	\item Applicazione delle metriche di qualità nel \WoW.
\end{itemize}

\paragraph{Dubbi}
\begin{itemize}
	\item Persistenza delle tecnologie adottate nel PoC ma valutate inadeguate in seguito;
	\item Individuazione di standard per la valutazione della qualità.
\end{itemize}

\vspace{0.5\baselineskip}
\par Gli argomenti sopracitati sono stati riportati nella presentazione per il diario di bordo del 17 maggio.
