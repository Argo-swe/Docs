\section{Riunione}
\subsection{Ordine del giorno}
\begin{itemize}
	\item Analisi e revisione delle \glossario{metriche di qualità};
	\item Selezione e discussione sui valori desiderabili e tollerabili di range per le \glossario{metriche di prodotto};
	\item Revisione dell’analisi dei requisiti e decisione caso d’uso sulla funzionalità di \glossario{debugging};
	\item Definizione nuovi ticket per l'aggiornamento dei documenti.
\end{itemize}

\subsection{Discussione e decisioni}
\par All'inizio del meeting il gruppo ha discusso le \glossario{metriche} e le funzionalità da aggiungere del progetto.Successivamente è stata affrontata la funzionalità di \glossario{debugging} nel codice.
\par Di seguito sono riportati i punti della discussione:
\begin{itemize}
	\item Selezione delle \glossario{metriche di prodotto}
	\item Si è deciso di modificare di M.2.4 in : Variazione temporale;
	\item Aggiunta di una metrica per valutare l'Efficienza Temporale rispetto le aspettative del preventivo;
	\item M.2.6 si è deciso di eliminarla per bassa fattibilità;
	\item Il gruppo si è accordato sul fatto che i valori desiderabili sono quelli che secondo noi si raggiungerebbero con un \glossario{way of working} ideale a cui dovremmo aspirare mentre i valori tollerabili sono quelli entro i quali noi consideriamo le \glossario{metriche} positive;
	\item La metrica Rischi non previsti in assoluto per sprint(M.3.2) deve essere rinominata in: Rischi inattesi;
	\item Per valutare le meriche di efficienza delle contrmisure il gruppo ha avuto difficoltà a trovare una modalità oggettiva per il calcolo e anche nella decisione dei range dato che gli sprint potrebbero essere di durate differenti. La soluzione emersa è stata modificare la misurazione con una sommatoria pesata con valori booleani in base al fatto che un rischio è stato gestito o meno;
	\item Sono emersi dei dubbi sugli strumento da usare per valutare le metriche del codice. Durante lo \glossario{sprint} si testeranno : SonarQube, Coverage.py, Networkx e Pylint;
	\item E' stato discusso lo strumento di debugging. In particolare si è deciso di visualizzare il \glossario{log} come un file separato scaricabile dal \glossario{Tecnico} mentre la visualizzazione sarà strutturata con dati tabulari;
	\item I requisiti dovranno essere rivisti in base alle nuove funzionalità che si stanno sviluppando, in particolare si deve aggiungere i casi d'uso per la funzioinalità di \glossario{debugging} e si dovrà modificare in base alle prossime riunioni la visualizzazione del \glossario{dizionario dati}(UC14).
\end{itemize}

