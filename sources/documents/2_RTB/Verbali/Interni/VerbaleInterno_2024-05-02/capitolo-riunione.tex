\section{Riunione}
\subsection{Ordine del giorno}
\begin{itemize}
	\item Analisi delle attività svolte durante la settimana;
	\item Discussione dei punti che verranno trattati al diario di bordo del 3 maggio;
	\item Organizzazione di un incontro con la \glossario{Proponente} a titolo di aggiornamento sullo stato del progetto.
\end{itemize}

\subsection{Discussione e decisioni}
\par All'inizio dell'incontro, ciascun membro del gruppo ha presentato le attività svolte durante la settimana.

\subsubsection{Analista}
\par L'analista ha segnalato delle difficoltà nella conversione in LaTeX del documento di \AdR. È stata poi espressa la necessità di approfondire il caso d'uso legato alla funzionalità di \glossario{debug} per il profilo Tecnico. In aggiunta, l’analista ha evidenziato la possibilità di includere un caso d’uso per la registrazione in caso di interazione con \glossario{API} a pagamento.

\par L'analista ha proposto un incontro con il Professor Riccardo Cardin per chiarire il caso d’uso di login, attualmente disponibile anche per il Tecnico autenticato. Il problema è stato mitigato definendo delle precondizioni.

\subsubsection{Amministratore}
\par L'amministratore ha completato il foglio di calcolo necessario per la generazione automatica delle tabelle e dei grafici da includere nel \PdP. Inoltre, è stato definito un file di configurazione per automatizzare la compilazione dei file LaTeX all'interno dell'ambiente condiviso.

\subsubsection{Responsabile} 
\par Il responsabile si è occupato della stesura della pianificazione, del preventivo e del consuntivo nel \PdP. In aggiunta, sono state definite le attività nel diagramma di \glossario{Gantt} ed è stato redatto il verbale del meeting precedente.

\subsubsection{Programmatore}
\par Il programmatore ha illustrato al gruppo le modalità con cui \glossario{txtai} effettua l’elaborazione del linguaggio naturale. Tra le idee proposte, vi era quella di formattare il \glossario{dizionario dati} e la richiesta dell'utente. In altre parole, la richiesta dell'utente verrebbe pre-processata per renderla più comprensibile ai meccanismi interni di \glossario{txtai}. Il programmatore ha aggiunto due nuovi moduli per il preprocessing.

\par Inoltre, è stata discussa la possibilità di creare un indice intermedio per la ricerca semantica. Questa soluzione sarebbe utile in quanto migliorerebbe la qualità dei risultati parziali.

\subsubsection{Progettista}
\par Il progettista ha collaborato con il programmatore per definire un dizionario dati in formato \glossario{JSON}. Inoltre, il progettista ha suggerito al gruppo di adottare il linguaggio \glossario{Python} per il back-end.

\subsubsection{Ripartizione dei ruoli}
\par In seguito, il team ha distribuito le ore tra i membri, assegnando a ciascuno di loro più ruoli. Questa decisione è stata presa per ottimizzare l'allocazione delle risorse, poiché alcuni ruoli avevano un numero esiguo di ore rimaste o richiedevano uno sforzo maggiore. In particolare, il gruppo ha assegnato più ore ai ruoli di progettista, programmatore e amministratore. 

\subsubsection{Incontro con la Proponente}
\par Il team ha fissato i punti da trattare con la \glossario{Proponente} durante la riunione del 6 maggio:
\begin{itemize}
	\item Proposta di adozione del framework \glossario{Streamlit};
	\item Utilizzo di \glossario{Python} come linguaggio per il \glossario{back-end};
	\item Aggiornamento sullo stato del progetto e sulle attività imminenti.
\end{itemize}

\subsubsection{Punti da trattare nel prossimo diario di bordo}
\par Il gruppo ha discusso i punti salienti da trattare nel diario di bordo del 3 maggio.

\paragraph{Done}
\begin{itemize}
	\item Pianificazione e preventivo \glossario{sprint} 3;
	\item Inizio stesura del \PdQ;
	\item Conversione in \glossario{LateX} del documento di \AdR;
	\item Inserimento dei grafici nel \PdP;
	\item Progettazione del prompt per \glossario{ChatGPT};
	\item Definizione di un dizionario dati in formato \glossario{JSON};
	\item Aggiornamento delle \NdP\ e del \Gls;
	\item Pianificazione di un incontro con la Proponente.
\end{itemize}

\paragraph{Difficoltà}
\begin{itemize}
	\item Poche risorse assegnate al ruolo di amministratore;
	\item Rotazione dei ruoli;
	\item Caricamento automatico dei documenti da un \glossario{repository} a un altro.
\end{itemize}	

\paragraph{Todo}
\begin{itemize}
	\item Creazione di un indice intermedio per una maggior accuratezza nell'interazione tra il dizionario dati e la richiesta dell'utente;
	\item Bozza di interfaccia grafica per la web app;
	\item Avanzamento nella scrittura dei documenti;
	\item Incontro con la Proponente.
\end{itemize}

\paragraph{Dubbi}
\begin{itemize}
	\item Livello di dettaglio delle \NdP.
\end{itemize}

\vspace{0.5\baselineskip}
\par Gli argomenti sopracitati sono stati riportati nella presentazione per il diario di bordo del 3 maggio.