\section{Riunione}
\subsection{Ordine del giorno}
\begin{itemize}
	\item Analisi dell'andamento dello \glossario{sprint} in vista del diario di bordo;
	\item Esposizione delle funzionalità di \glossario{Jira} e organizzazione della migrazione da \glossario{GitHub} a Jira;
	\item Pianificazione delle attività da svolgere entro la fine dello \glossario{sprint}.
\end{itemize}

\subsection{Discussione e decisioni}
\par All'inizio del meeting, il gruppo ha discusso le attività svolte durante la prima settimana di \glossario{sprint}.

\subsubsection{Programmatore} \label{sec:programmatore}
\par Il programmatore ha descritto brevemente le tecnologie studiate e ha mostrato al team i test svolti su una versione preliminare del \glossario{dizionario dati} (definita con l'intento di comprendere il funzionamento dei \glossario{modelli}). Sono stati esposti inoltre i dubbi e le difficoltà riscontrate, con un confronto collettivo per mitigare i problemi.

\par Di seguito sono riportati i punti della discussione:
\begin{itemize}
	\item L'impiego di \glossario{txtai} richiedeva del materiale per svolgere i test; per questo il programmatore aveva utilizzato un \glossario{dizionario dati} prelevato da un \glossario{database} di esempio (\glossario{chinook}) di \glossario{SQLite};
	\item L'idea iniziale era quella di prendere uno schema di \glossario{database}, in formato \glossario{JSON}, che potesse essere analizzato assieme alla richiesta (in linguaggio naturale) dell'utente e restituire la porzione di dizionario dati pertinente alla richiesta originale e un punteggio che ne indicasse l'accuratezza;
	\item Era emerso un problema che coinvolgeva il formato \glossario{JSON}; quest'ultimo, infatti, non si armonizzava con il \glossario{modello}, che prediligeva invece documenti in linguaggio naturale;
	\item Il programmatore ha spiegato che \glossario{txtai} si occupa della generazione e archiviazione di \glossario{embeddings}, ovvero \glossario{vettori} rappresentanti specifici documenti (ad esempio frasi, nel nostro caso), all'interno di un \glossario{indice}, per facilitare la comparazione con uno schema già indicizzato;
	\item Il programmatore ha consigliato, per il prossimo \glossario{sprint}, di assegnare più risorse al ruolo di programmatore, in modo da disporre di maggior personale impiegato nella risoluzione dei problemi;
	\item Il team ha deciso di confermare il programmatore attuale anche per lo sprint successivo, cosicché l'apprendimento delle tecnologie e la condivisione delle conoscenze possano risultare più immediati;
	\item Si è discussa la possibilità di modificare il formato del dizionario dati (in modo che il linguaggio risulti meno complesso per il modello) o, in alternativa, trovare un modello che supporti la lettura del dizionario dati nel formato richiesto;
	\item Il gruppo ha valutato lo sviluppo di un \glossario{parser} per convertire il dizionario dati dal formato JSON a un formato in linguaggio naturale.
\end{itemize}

\subsubsection{Amministratore}
\par L'amministratore ha illustrato le modifiche effettuate al template \glossario{LaTeX} e ha aggiornato il gruppo sulla stesura del \Gls. È stata evidenziata la necessità di chiarire con il Professor Tullio Vardanega il criterio col quale inserire i termini nel \Gls. Relativamente al \Gls, si è discusso anche se il documento fosse a uso interno o esterno. Questi dubbi verranno esposti nel diario di bordo del 26 aprile.

\subsubsection{Progettista}
\par Il team ha esaminato la struttura del \glossario{dizionario dati}, che il progettista ha usato come prototipo in fase di apprendimento delle tecnologie. Durante la prima settimana di \glossario{sprint}, il progettista ha anche studiato la possibile composizione del \glossario{prompt} da fornire in output all'utente.

\subsubsection{Analista}
\par La discussione si è poi spostata sul documento di \AdR\ e sulla sua conversione in \glossario{LateX}. Entro la fine dello sprint, verrà finalizzata la conversione in LaTeX e, contestualmente, verranno sviluppati i \glossario{casi d'uso}.

\subsubsection{Jira}
\par Sono state esposte al gruppo le funzionalità essenziali di \glossario{Jira}, specialmente la creazione del \glossario{diagramma di Gantt} in fase di pianificazione. In aggiunta, l'amministratore ha illustrato il sistema di apertura dei \glossario{ticket}, l'interazione con le \glossario{pull request} di \glossario{GitHub} e la prassi da seguire per automatizzare il cambio di stato dei task.

\subsubsection{Punti da trattare nel prossimo diario di bordo}
\par Il gruppo ha discusso i punti salienti da trattare nel diario di bordo del 26 aprile.

\paragraph{Done}
\begin{itemize}
	\item Passaggio a \glossario{Jira} come \glossario{Issue Tracking System};
	\item Studio preliminare delle tecnologie (\glossario{txtai} e interazione tra i \glossario{modelli} e il \glossario{dizionario dati} in formato \glossario{JSON});
	\item Definizione del dizionario dati;
	\item Avanzamento scrittura dei documenti;
	\item Consuntivo del primo \glossario{sprint};
	\item Modifica del template \glossario{LaTeX} (modifiche estetiche e funzionali).
\end{itemize}

\paragraph{Difficoltà}
\begin{itemize}
	\item Gestione delle attività di natura organizzativa su \glossario{GitHub};
	\item Configurazione di \glossario{Jira}.
\end{itemize}

\paragraph{Todo}
\begin{itemize}
	\item Aggiornamento migliorativo della pianificazione nel consuntivo;
	\item Stesura del \PdQ;
	\item Approfondimento della funzionalità di debug nel documento di \AdR\ (come suggerito dalla \glossario{Proponente});
	\item Individuazione delle tecnologie di sviluppo per la \glossario{web app};
	\item Inserimento nel \Gls\ dei termini identificati durante la stesura dei documenti.	
\end{itemize}

\paragraph{Dubbi}
\begin{itemize}
	\item \Gls\ come documento interno o esterno;
	\item Criterio di scelta dei termini da inserire nel \Gls;
	\item Numero di occorrenze di un termine del \Gls\ da formattare;
	\item Modalità di ridistribuzione di eventuali risorse, in eccesso o in difetto, durante la stesura del consuntivo.
\end{itemize}

\vspace{0.5\baselineskip}
\par Gli argomenti sopracitati sono stati riportati nella presentazione per il diario di bordo del 26 aprile.