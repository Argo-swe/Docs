\section{Riunione}
\subsection{Ordine del giorno}
\begin{itemize}
	\item Esaminare lo stato di avanzamento dei documenti in lavorazione.
\end{itemize}

\subsection{Discussione e decisioni}

\subsubsection{Responsabile}
\par Il responsabile ha presentato lo \glossario{spreadsheet} condiviso che racchiude le informazioni inerenti al preventivo e al consuntivo (orario ed economico). Ogni \glossario{sprint} ha un proprio foglio che contiene le seguenti informazioni:
\begin{itemize}
	\item Numero dello \glossario{sprint};
	\item Date di inizio e termine dello sprint;
	\item Tabella di assegnazione delle ore produttive per ciascun membro del team, accumulate in totali per persona e per ruolo;
	\item Distribuzione delle ore per ruolo, sotto forma di donut chart;
	\item Distribuzione delle ore per la coppia risorsa-ruolo, sotto forma sia di tabella che di grafico a barre;
	\item Preventivo economico dello sprint;
	\item Tabella riassuntiva con ore e budget spesi e restanti;
	\item Pie chart con la stima delle ore spese sul totale;
	\item Pie chart con la stima del budget sul totale;
	\item Ore rimanenti per la coppia risorsa-ruolo.
\end{itemize}

\vspace{0.5\baselineskip}
\par È stata suggerita la possibilità, negli \glossario{sprint} successivi al primo, di aggiungere alla tabella di preventivo una riga considerante la spesa ore/budget degli sprint pregressi, da includere poi nel pie chart, così da avere anche un rapporto sul lavoro passato, oltre che sul totale. Si è inoltre discussa l'aggiunta di una sezione di rendicontazione ore, per avere un confronto tra preventivo e consuntivo.

\subsubsection{Amministratore}
\par Riguardo al \Gls, il gruppo ha valutato positivamente il caricamento di una struttura iniziale su \glossario{GitHub}, anche vuota, così da poter accumulare i termini utilizzati. In aggiunta, il team ha considerato la creazione di un documento condiviso che permetta di raccogliere le parole chiave.

\par Durante la discussione delle \NdP, è stata analogamente evidenziata la necessità di un file condiviso in cui appuntare le linee guida e le procedure da formalizzare.

\subsubsection{Analisti e progettista}
\par Dopo aver esaminato attentamente i \glossario{casi d'uso}, la priorità è stata redarre il documento di \AdR, specialmente i capitoli legati alla descrizione del prodotto e dei \glossario{casi d'uso}. Dal resoconto degli analisti è emersa inoltre la necessità di ridefinire l'ordinamento generale del documento, che può risultare ridondante.

\subsubsection{Discussioni finali}
\par Terminati i controlli complessivi, il team ha pianificato per il secondo \glossario{sprint} lo studio delle tecnologie di sviluppo, suggerite dalla \glossario{Proponente} o individuate come risposta alle necessità nate nell'\AdR.

\par Si è inoltre discusso della possibilità di definire una struttura di \glossario{database} da usare come prototipo per i prossimi \glossario{sprint}; sono state proposte come soluzioni sia la ricerca online di qualche struttura, sia l'uso di alcuni progetti del corso di Basi di Dati.

\par Infine è stato raccomandato di avanzare le \glossario{pull request} incrementalmente in modo da favorire il lavoro di \glossario{verifica} e ridurre la gravità di errori di inesperienza con \glossario{Git} o \glossario{LaTeX}.