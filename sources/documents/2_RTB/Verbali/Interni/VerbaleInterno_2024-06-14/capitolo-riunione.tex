\section{Riunione}
\subsection{Ordine del giorno}
\begin{itemize}
	\item Divisione delle metriche per l'espansione del \PdQ;
	\item Divisione dei compiti dei programmatori;
	\item Discussione dei punti per il diario di bordo del 14 giugno;
	\item Discussione per l'impostazione del template del \glossario{front-end}.
\end{itemize}

\subsection{Discussione e decisioni}
\subsubsection{Divisione delle metriche per l'espansione del \PdQ}
\par Il gruppo si è ritrovato per discutere della divisione dei compiti in vista del prossimo \glossario{sprint}. 
La prima divisione viene fatta tra gli amministratori, per l'espansione del \PdQ con i grafici di metriche calcolate nel tempo. 
Vengono divise le metriche da inserire graficamente tra gli amministratori in maniera proporzionale al numero di ore per lo \glossario{sprint}: lo scopo è creare una vista che possa mostrare l'andamento delle metriche durante gli sprint, inserendo un valore effettivo che si spera rimanga all'interno di un intervallo tra il loro valore ambito e tollerabile. 
Ovviamente lo scopo è mantenere le metriche sul valore ambito nei prossimi \glossario{sprint}.
Le metriche vengono anche dotate di un paragrafo che ha lo scopo di descrivere l'andamento nel tempo.

\subsubsection{Divisione dei compiti dei programmatori}
TODO

\subsubsection{Discussione dei punti per il diario di bordo}
\par Il gruppo ha discusso i punti salienti da trattare nel diario di bordo del 14 giugno 2024.
\paragraph{Done}
\begin{itemize}
	\item Implementazione \glossario{front-end} con il \glossario{framework} \glossario{Vue.js};
	\item Implementazione del \glossario{back-end} con la libreria \glossario{Flask};
	\item Confronto tra le librerie \glossario{Flask} e \glossario{FastAPI};
	\item Aggiunta delle descrizioni tabelle e colonne nel \glossario{prompt};
	\item Modifica gestione chiavi esterne del \glossario{dizionario dati};
	\item Test di generazione \glossario{query SQL} con modelli in locale tramite \glossario{LM Studio};
	\item Revisione dei casi d’uso \AdR;
	\item Aggiunta requisiti funzionali, di qualità e di vincolo nell'\AdR.
	
\end{itemize}
\paragraph{Difficoltà}
\begin{itemize}
	\item Configurazione \glossario{Docker} per interazione con il \glossario{front-end};
	\item Modellazione di \glossario{DTO} con \glossario{Flask};
	\item Comprendere come legare tra loro i casi d’uso;
	\item Configurazione del server locale di \glossario{LM Studio};
	\item Utilizzo di una libreria \glossario{front-end}.
\end{itemize}	
\paragraph{Todo}
\begin{itemize}
	\item Aggiunta dei grafici nell’\AdR;
	\item Aggiornamento della documentazione (\PdP, \PdQ, \AdR);
	\item Scegliere quali metodi esporre al \glossario{front-end};
	\item Completare lo sviluppo delle componenti grafiche;
	\item Continuare l’integrazione tra \glossario{front-end} e \glossario{back-end}.		
\end{itemize}
\paragraph{Dubbi}
\begin{itemize}
	\item I test di correttezza del \glossario{prompt} richiesti dalla Proponente possono rientrare tra i requisiti di qualità?
\end{itemize}

\par Gli argomenti sopracitati sono stati riportati nella presentazione per il diario di bordo del 14 giugno 2024.
