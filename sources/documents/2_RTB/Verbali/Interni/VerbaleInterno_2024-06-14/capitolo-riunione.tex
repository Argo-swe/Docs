\section{Riunione}
\subsection{Ordine del giorno}
\begin{itemize}
	\item Divisione delle metriche per l'espansione del \PdQ;
	\item Suddivisione dei compiti dei programmatori;
	\item Discussione dei punti per il diario di bordo del 14 giugno;
	\item Impostazione del template \glossario{front-end}.
\end{itemize}

\subsection{Discussione e decisioni}
\subsubsection{Divisione delle metriche per l'espansione del \PdQ}
\par Il gruppo si è ritrovato per discutere della divisione dei compiti in vista del prossimo \glossario{sprint}. 
La prima ripartizione è stata effettuata tra gli amministratori, al fine di espandere il \PdQ\ con i grafici di misurazione delle metriche. 
Le metriche sono state suddivise tra gli amministratori in maniera proporzionale al numero di ore produttive assegnate nel preventivo. Lo scopo del cruscotto di valutazione della qualità è creare una vista che possa mostrare l'andamento delle metriche durante gli sprint; il valore misurato dovrebbe rimanere all'interno di un intervallo tra il loro valore ambito e tollerabile. 
L'obiettivo del progetto è raggiungere o avvicinarsi il più possibile al valore ambito.
Il team ha pianificato anche la stesura di un paragrafo per descrivere in forma testuale i grafici.

\subsubsection{Divisione dei compiti dei programmatori}
\par I programmatori hanno analizzato il template front-end e suddiviso i compiti relativi allo sviluppo della chat. Le seguenti attività sono state definite come prioritarie:
\begin{itemize}
	\item Inserimento dei messaggi nella chat;
	\item Scelta delle componenti grafiche per la selezione del dizionario dati, della lingua e del DBMS.
\end{itemize}
\par Per quanto riguarda il back-end, invece, il team di programmatori ha risolto alcuni problemi legati alla struttura del \glossario{PoC}. In seguito, sono state pianificate le seguenti attività:
\begin{itemize}
	\item Conclusione della funzionalità di login;
	\item Sviluppo dei metodi per la gestione dei dizionari dati.
\end{itemize}

\subsubsection{Discussione dei punti per il diario di bordo}
\par Il gruppo ha discusso i punti salienti da trattare nel diario di bordo del 14 giugno.
\paragraph{Done}
\begin{itemize}
	\item Setup del \glossario{front-end} con il \glossario{framework} \glossario{Vue.js};
	\item Implementazione del \glossario{back-end} con la libreria \glossario{Flask};
	\item Confronto tra le librerie Flask e \glossario{FastAPI};
	\item Aggiunta delle descrizioni di tabelle e colonne nel \glossario{prompt};
	\item Modifica delle chiavi esterne nel \glossario{dizionario dati};
	\item Test di generazione di \glossario{query SQL} con modelli locali tramite \glossario{LM Studio};
	\item Revisione dei casi d'uso nell'\AdR;
	\item Aggiunta di requisiti funzionali, di qualità e di vincolo nell'\AdR.
	
\end{itemize}
\paragraph{Difficoltà}
\begin{itemize}
	\item Configurazione di \glossario{Docker} per l'interazione con il \glossario{front-end};
	\item Modellazione di \glossario{DTO} con \glossario{Flask};
	\item Configurazione del server locale di \glossario{LM Studio}.
\end{itemize}	

\paragraph{Todo}
\begin{itemize}
	\item Aggiunta dei grafici nell'\AdR;
	\item Aggiornamento della documentazione (\PdP, \PdQ, \AdR);
	\item Esposizione dei metodi al \glossario{front-end};
	\item Conclusione dello sviluppo delle componenti grafiche;
	\item Prosecuzione dell'integrazione tra \glossario{front-end} e \glossario{back-end}.		
\end{itemize}

\paragraph{Dubbi}
\begin{itemize}
	\item Legame tra test di correttezza e requisiti di qualità.
\end{itemize}

\vspace{0.5\baselineskip}
\par Gli argomenti sopracitati sono stati riportati nella presentazione per il diario di bordo del 14 giugno.
