\section{Riunione}
\subsection{Ordine del giorno}
\begin{itemize}
	\item Discussione delle attività da portare a termine in vista della \glossario{RTB};
	\item Pianificazione di un incontro con la \glossario{Proponente};
	\item Discussione sulla lettera di presentazione per la RTB;
	\item Pianificazione di un incontro in presenza prima della revisione RTB.
\end{itemize}

\subsection{Discussione e decisioni}
\subsubsection{Suddivisione delle attività imminenti}
\par All'inizio del meeting, il gruppo ha espresso la necessità di fissare delle macro-attività in vista della revisione RTB. Tali attività riguardano principalmente la documentazione di progetto. 
Tra le attività principali sono state evidenziate:
\begin{itemize}
	\item Aggiornamento del \Gls\ con i termini mancanti;
	\item Inserimento dell'ottavo sprint nel \PdP;
	\item Revisione generale dell'\AdR\ per verificare che i casi d'uso siano corretti e che i grafici siano coerenti;
	\item Aggiornamento del \PdQ\ con gli ultimi grafici delle metriche;
	\item Espansione e aggiornamento delle \NdP.
\end{itemize}
\par Queste macro-attività verranno ripartite tra i membri del gruppo in base al ruolo che ricoprono; per ciascun ruolo verranno assegnati dei sotto-task tali da poter suddividere equamente il carico di lavoro.

\subsubsection{Pianificazione incontro con la Proponente}
\par Gli ultimi \glossario{sprint} hanno comportato un avanzamento significativo del \glossario{PoC}. 
Pertanto, il gruppo ha deciso di organizzare un incontro con la \glossario{Proponente} al fine di presentare il lavoro svolto prima della revisione RTB. 
Il responsabile ha quindi elaborato una bozza della mail con la quale chiedere un incontro per martedì 9 luglio.

\subsubsection{Pianificazione incontro in presenza}
\par Dopo l'incontro con la \glossario{Proponente}, il gruppo ha concordato sull'importanza di organinzzare un incontro in presenza. L'obiettivo dell'incontro è di valutare lo stato di avanzamento e completare le attività rimanenti prima dell'incontro con il Professor Riccardo Cardin. L'organizzazione di riunioni in presenza era già stata valutata positivamente, poiché il team ha riscontrato una collaborazione più stretta e una maggiore chiarezza comunicativa.  
L'incontro in presenza è stato fissato per mercoledì 10 luglio in sede universitaria. Una volta analizzata la situazione attuale, il gruppo organizzerà un incontro con il Professor Cardin per discutere i requisiti e le tecnologie. Verrà pertanto inviata una mail per stabilire la data dell'incontro.

\subsubsection{Discussione sulla lettera di presentazione per la RTB}
\par Il gruppo ha selezionato le informazioni da inserire nella lettera di presentazione per la revisione RTB. La struttura della lettera di presentazione verrà definita durante l'incontro in presenza.
