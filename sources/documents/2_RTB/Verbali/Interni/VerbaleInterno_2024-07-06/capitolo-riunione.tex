\section{Riunione}
\subsection{Ordine del giorno}
\begin{itemize}
	\item Discussione delle attività da concludere per il completamento della documentazione
	\item Pianificazione incontro con la Proponente
	\item Pianificazione incontro con il Professor Riccardo Cardin
	\item Discussione lettera di presentazione e candidatura \RTB
	\item Pianificazione di un incontro in presenza prima della discussione con il Professor Riccardo Cardin
\end{itemize}

\subsection{Discussione e decisioni}
\subsubsection{Discussione delle attività da concludere per il completamento della documentazione}
\par Il gruppo all'inizio del meeting, ha espresso la necessità di stabilire delle attività che possano portare ad un punto della documentazione tale che questa possa essere presentata alla prima revisione. 
Tra le attività principali sono state evidenziate:
\begin{itemize}
	\item Popolazione del \Gls con i termini mancanti;
	\item Aggiornamento del \PdP in modo che sia al passo con lo \glossario{sprint} con il quale il gruppo si presenterà all'\RTB;
	\item Revisione generale dell'\AdR per controllare che i casi d'uso siano corretti e che i loro grafici siano coerenti;
	\item Aggiornamento del \PdQ con gli ultimi grafici delle metriche e la descrizione di questi;
	\item Espansione e aggiornamento delle \NdP;
\end{itemize}
\par Queste macro attività verranno distribuite ai vari membri in base al ruolo: per ogni membro di un ruolo verranno create delle sotto-task tali da poter dividere il lavoro e concluderlo in maniera efficiente.

\subsubsection{Pianificazione incontro con la Proponente}
\par Gli ultimi \glossario{sprint} hanno comportato un avanzamento importante nel Proof of Concept che ora è pronto per una valutazione da parte della Proponente. 
Pertanto il gruppo è deciso a mobilitarsi per organizzare un incontro con la Proponente in modo da poter presentare il lavoro svolto nelle ultime settimane, prima dell'incontro per l'\RTB. 
Viene creata una bozza per una mail da inviare, con la quale chiedere un incontro per mercoledì 10 luglio 2024.

\subsubsection{Pianificazione incontro con il Professor. Riccardo Cardin}
\par Dopo l'incontro con la Proponente, il gruppo vuole organizzare un incontro con il Professor. Riccardo Cardin per presentare la documentazione e il Proof of Concept per una valutazione per l'\RTB. Verrà pertanto inviata una mail per stabilire la data dell'incontro.

\subsubsection{Discussione lettera di presentazione e candidatura \RTB}
\par Assieme all'incontro con il Professor. Riccardo Cardin, il gruppo presenterà la lettera di candidatura al Professor. Tullio Vardanega per ricevere la valutazione e l'approvazione per la prima revisione \RTB.

\subsubsection{Pianificazione di un incontro in presenza prima della discussione con il Professor. Riccardo Cardin}
\par Infine il gruppo ha discusso per organizzare un incontro in presenza per lavorare alle ultime attività rimaste, prima dell'incontro con il Professor. Riccardo Cardin. Quest'incontro viene organizzato poiché il gruppo più volte si è trovato a lavorare in maniera più efficiente quando si è ritrovato a lavorare in presenza, grazie alla possibilità di poter collaborare a contatto più stretto.
L'incontro in presenza viene fissato per mercoledì 10 luglio 2024 in sede universitaria.