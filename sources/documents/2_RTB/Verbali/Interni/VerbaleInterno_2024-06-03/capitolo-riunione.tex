\section{Riunione}
\subsection{Ordine del giorno}
\begin{itemize}
	\item Discussione definitiva per la decisione di accorciamento dei prossimi \glossario{sprint}, idealmente i prossimi 2;
	\item Definizione degli elementi finali del Proof of Concept;
	\item Resoconto delle attività attive nel backlog corrente che vanno riassegnate al prossimo \glossario{sprint};
	\item Programmazione di un incontro con il Professor Riccardo Cardin per la discussione dei dubbi del gruppo sui documenti di \AdR e \PdQ;
	\item Pianificazione di un incontro con la Proponente per la fine del prossimo glossario{sprint};
	\item Proposta di delegazione ai membri uscenti da un ruolo per la definizione delle attività per i membri entranti.
\end{itemize}

\subsection{Discussione e decisioni}
\subsubsection{Retrospettiva e Pianificazione}
\par Il gruppo si è ritrovato come concordato per una retrospettiva dello \glossario{sprint} 3, vista esserne prossima la fine. Vengono individuate alcune le attività principali per lo \glossario{sprint} successivo. 
Viene inoltre riproposto al gruppo di organizzare i prossimi due \glossario{sprint} in modo che la loro durata sia di dieci giorni o una settimana. 
Questo permetterebbe al gruppo di focalizzarsi e concludere attività con più facilità in vista della revisione \RTB.
Le attività da concludere nei prossimi due periodi consisterebbero nel completamento del Proof of Concept e nella conclusione dei documenti come l'\AdR e le \NdP.
È stato evidenziato durante l'incontro come soprattutto quest'ultimo documento abbia bisogno di un aggiornamento sostanziale, in quanto necessita di espansioni nelle sezioni riguardanti la struttura dei documenti.

\subsubsection{Modifica delle tecnologie}
\par Per quanto riguarda il Proof of Concept nell'ultima settimana erano stati ultimati i lavori con il framework \glossario{Streamlit}, di conseguenza il gruppo si è proposto di fissare un incontro con la Proponente per presentare la versione demo e mostrare il lavoro svolto per la parte funzionale dell'applicativo.
Tuttavia nell'ultimo periodo il gruppo si è reso conto dei limiti del \glossario{framework} utilizzato: infatti questo mostrava significativi ostacoli sia nella personalizzazione dell'interfaccia utente, sia difetti che non permettevano di poter realizzare l'applicativo seguendo un'architettura come richiesto dal corso di studi.
Ciò ha portato il gruppo a cambiare \glossario{framework} e orientarsi su strumenti che potessero scavalcare gli ostacoli posti da \glossario{Streamlit}. 
I \glossario{framework} individuati sono:
\begin{itemize}
	\item \glossario{Vue} per il \glossario{front-end}, in particolare per la componentistica;
	\item \glossario{Bootstrap} come alternativa per il \glossario{front-end};
	\item \glossario{Flask} per il \glossario{back-end}.
\end{itemize}
Questi sono strumenti che erano già stati individuati in precedenza, ma abbandonati per favorire l'uso di \glossario{Streamlit}; di conseguenza occorreranno delle sessioni di studio per conformare il gruppo sull'utilizzo di queste tecnologie. 
Inoltre saranno limitati i lavori su \glossario{txtai}: ciò è dovuto alla necessità di una conversione rapida dell'applicativo ai nuovi strumenti e dal fatto che i lavori sulla parte funzionale dell'applicazione sono già a buon punto.

\subsubsection{Divisione dei ruoli}
\par Per il prossimo \glossario{sprint} viene proposto di avere un maggior numero di programmatori, in modo che i lavori di converisone possano procedere rapidamente e senza troppi intoppi.
Si propone una rionione di aggiornamento sulle tecnologie, per allineare i vari membri sui nuovi \glossario{framework} che si andranno ad utilizzare.
Questo implica che lo \glossario{sprint} 5 sarà incentrato sulla migrazione delle tecnologie avrà durata di una settimana e mezza, in modo da avere un paio di giorni per lo studio e l'aggiornamento e il resto per i lavori.
Vengono inoltre stabiliti i documenti che dovranno essere aggiornati durante il prossimo \glossario{sprint}:
\begin{itemize}
	\item \PdP per l'aggiornamento dello \glossario{sprint};
	\item \NdP per i motivi sopra spiegati;
	\item \AdR per un'aggiornamento sui casi d'uso e l'inserimento dei grafici;
	\item \Gls per un aggiornamento dei termini (meno focale rispetto ai documenti sopra).
\end{itemize}
Vengono infine ripartite le ore per ruolo ai vari membri, in modo da coprire tutti i ruoli pur mantenendo un numero di membri concentrato sul ruolo di Programmatore e viene anche riassegnata la data della revisione \RTB, idealmente fissata per fine giugno con un ritardo di circa due settimane rispetto a quanto preventivato inizialmente.

\subsubsection{Incontro con il Professor Riccardo Cardin}
\par Il gruppo esprime nuovamente la volontà di organizzare un incontro con il Professor Cardin, visto lo stato dei documenti di \AdR e \PdQ che permetterebbero un confronto per individuare eventuali errori o imprecisioni da correggere.
I punti principali da trattare con il Professore saranno:
\begin{itemize}
	\item Metriche: per capire se la quantità e la divisione sono corretti;
	\item Requisiti: per discutere su alcuni casi d'uso che hanno messo in difficoltà il gruppo, come sono stati collegati tra loro e altre domande più specifiche;
	\item Funzionamento di \glossario{txtai}: per fornire un aggiornamento sullo stato del progetto e capire se includere alcuni lavori nell'\AdR.
\end{itemize}

Per finire il gruppo ha stabilito che, per i prossimi \glossario{sprint}, i membri dei ruoli uscenti definiranno le attività per i membri entranti con tanto di descrizione dell'attività su \glossario{Jira}. Questo dovrebbe permettere di avanzare più velocemente nell'andamento delle attività e avere una coordinazione più veloce. 
