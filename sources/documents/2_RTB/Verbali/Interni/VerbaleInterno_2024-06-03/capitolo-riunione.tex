\section{Riunione}
\subsection{Ordine del giorno}
\begin{itemize}
	\item Decisione definitiva sulla durata dei prossimi \glossario{sprint};
	\item Selezione delle funzionalità da implementare nel \glossario{Proof of Concept};
	\item Resoconto delle attività svolte durante lo \glossario{sprint};
	\item Programmazione di un incontro con il Professor Riccardo Cardin per chiarire i dubbi sull'\AdR;
	\item Pianificazione di un incontro con la Proponente per la fine del prossimo \glossario{sprint}.
\end{itemize}

\subsection{Discussione e decisioni}
\subsubsection{Retrospettiva e pianificazione}
\par Il gruppo si è ritrovato per la retrospettiva dello \glossario{sprint} 4.
È stato concordato di organizzare \glossario{sprint} più brevi, della durata di una settimana o massimo dieci giorni.
Questo permetterebbe al gruppo di ridurre al minimo i periodi di inattività in vista della sessione di esami e della revisione \glossario{RTB}.
Per i prossimi due sprint, il team ha fissato le seguenti attività:
\begin{itemize}
	\item Avanzamento del Proof of Concept;
	\item Ultimazione del documento di \AdR.
\end{itemize}

\vspace{0.5\baselineskip}
\par Inoltre, il team ha convenuto che le \NdP\ debbano essere aggiornate per approfondire le sezioni riguardanti la struttura dei documenti.

\subsubsection{Sostituzione delle tecnologie}
\par Per quanto riguarda il \glossario{PoC}, il gruppo ha ultimato lo sviluppo con il framework \glossario{Streamlit}. Pertanto, il team ha deciso di fissare un incontro con la \glossario{Proponente} per presentare una demo funzionante.
Tuttavia, nel corso dell'ultimo sprint il gruppo ha constatato le limitazioni di Streamlit; in particolare, limiti nella personalizzazione e nelle prestazioni. Inoltre, il team ha riscontrato delle difficoltà nella riusabilità del codice.
Ciò ha portato il gruppo a cambiare \glossario{framework} e a orientarsi su tecnologie che potessero ovviare ai problemi posti da \glossario{Streamlit}.
I framework individuati sono:
\begin{itemize}
	\item \glossario{Vue.js} per il front-end;
	\item BootstrapVue (o PrimeVue) come libreria di componenti;
	\item \glossario{Flask} per il back-end.
\end{itemize}

\vspace{0.5\baselineskip}
\par Questi \glossario{framework} erano già stati valutati in precedenza; tuttavia, il team ha preventivato un periodo di formazione e di adattamento per garantire una transizione efficace.

\subsubsection{Divisione dei ruoli}
\par Per il prossimo \glossario{sprint}, il team ha assegnato un maggior numero di risorse al ruolo di programmatore, in modo da agevolare il cambiamento tecnologico.
Inoltre, è stata proposta una riunione intermedia di aggiornamento sulle tecnologie, con l'obiettivo di allineare le conoscenze del gruppo.
Di conseguenza, il quinto \glossario{sprint} sarà incentrato sul cambio di tecnologie e avrà durata di una settimana e mezza.
Infine, il team ha stabilito i documenti che dovranno essere aggiornati:
\begin{itemize}
	\item \PdP;
	\item \NdP;
	\item \AdR\ (aggiornamento dei casi d'uso e inserimento dei grafici);
	\item \Gls.
\end{itemize}

\vspace{0.5\baselineskip}
\par A causa del cambio di tecnologie, il team ha posticipato la data della revisione \glossario{RTB}, prevedendo un ritardo di circa due o tre settimane rispetto al piano iniziale.

\subsubsection{Incontro con il Professor Cardin}
\par Il gruppo ha programmato un incontro con il Professor Riccardo Cardin, al fine di individuare eventuali errori o imprecisioni nell'\AdR\ e nel \PdQ.
I punti da trattare saranno i seguenti:
\begin{itemize}
	\item Quantità e suddivisione delle metriche;
	\item Caso d'uso di autenticazione;
	\item Inclusioni ed estensioni.
\end{itemize}

\vspace{0.5\baselineskip}
\par Il team ha stabilito che, per i prossimi \glossario{sprint}, i membri che cambieranno ruolo dovranno supportare il responsabile nella definizione delle attività per il ruolo appena ricoperto. Con questo approccio, il gruppo ritiene di poter coordinare le attività con maggior precisione.
