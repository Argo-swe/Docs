\section{Riunione}
\subsection{Ordine del giorno}
\begin{itemize}
	\item Discussione sull'aggiunta di comandi in \glossario{LaTeX} per riferirsi ad altri documenti senza includere il numero di versione;
  \item Valutazione dell'andamento dello \glossario{sprint} corrente e definizione di eventuali azioni correttive;
  \item Discussione sull'impiego di un modello di sviluppo differente.
\end{itemize}

\subsection{Discussione e decisioni}

\subsubsection{Nuovi pacchetti e comandi LaTeX}
\par Il responsabile ha proposto all'amministratore di aggiungere nuovi comandi in \glossario{LaTeX} per menzionare altri documenti senza includere il numero di versione. Questa scelta era motivata dal fatto che durante la pianificazione del primo \glossario{sprint}, era stata fissata la stesura iniziale dei documenti di progetto. Tuttavia, compilando il \PdP\ a ridosso della \glossario{RTB}, il numero di versione verrebbe modificato ovunque in 1.0.0, invalidando la struttura della pianificazione. In altre sezioni, invece, come nel segmento relativo ai riferimenti normativi, è corretto affiancare il numero di versione al nome dei documenti. Dunque, è preferibile avere a disposizione due comandi, in modo da poter citare un documento con e senza numero di versione.
\par Inoltre, si è deciso di aggiungere il pacchetto fontawesome5, che fornisce il supporto \glossario{LaTeX} per il set di icone "Font Awesome 5 Free". Un miglioramento grafico pensato dal team riguardava la visualizzazione di un'etichetta prima del numero di versione.

\subsubsection{Tracciamento delle attività}
\par Dopo aver concordato gli aggiornamenti al template \glossario{LaTeX}, il gruppo ha analizzato l'andamento del primo \glossario{sprint}, valutando positivamente la partecipazione delle singole risorse e la suddivisione in team di lavoro più piccoli. In seguito, ciascun componente ha esposto le difficoltà incontrate, che andranno registrate nel diario di bordo e gestite in vista del prossimo \glossario{sprint}. Per migliorare la gestione del progetto, il team di analisti ha proposto l’utilizzo di una To-Do List condivisa in formato testuale.
\par Come stabilito dal \glossario{\WoW}, la creazione e assegnazione dei task deriva dalla pianificazione dello \glossario{sprint}, dalle riunioni interne e dai meeting tra risorse che ricoprono lo stesso ruolo. Di conseguenza, si rende indispensabile avere a disposizione un file di appunti condiviso, in aggiunta ai verbali (interni ed esterni). A partire da questi documenti, il responsabile può agilmente creare gli \glossario{issue} su \glossario{GitHub}.

\subsubsection{Strumenti da affiancare a GitHub}
\par Nel corso del primo \glossario{sprint}, il team si è accorto che \glossario{GitHub} non è il servizio ottimale per monitorare le attività di natura organizzativa e logistica. Inoltre, a meno di integrazioni con strumenti di terze parti, GitHub non offre la possibilità di visualizzare i task su un calendario. Perciò si è deciso di valutare possibili alternative all'\glossario{Issue Tracking System} di GitHub, che verranno discusse durante la prossima riunione.

\subsubsection{Modello di sviluppo}
\par In seguito alla riunione tra cliente e fornitore tenutasi il 9 aprile, il responsabile ha analizzato il modello di sviluppo scelto dal team. Come riportato nel \href{https://argo-swe.github.io/2_RTB/Verbali/Esterni/VerbaleEsterno_2024-04-09.pdf}{\emph{Verbale esterno del 2024-04-09 \ \faIcon{tag} v1.0.0}}, la \glossario{Proponente} ha consigliato al gruppo di non impiegare una metodologia troppo stringente, specialmente per quanto riguarda l'organizzazione delle riunioni. Il responsabile ha quindi esposto al team i risultati (raccolti nell'arco di una settimana) delle sue analisi, elencando i pro e contro dei modelli individuati. Dopo una valutazione collettiva, la scelta finale è stata quella di confermare il modello di sviluppo \glossario{Agile}, proseguendo così con la linea intrapresa nel primo \glossario{sprint}, a cui andranno applicate le dovute rettifiche in fase di \glossario{retrospettiva}.


