\section{Riunione}
\subsection{Ordine del giorno}
\begin{itemize}
	\item Retrospettiva \glossario{sprint}\ 3;
	\item Allineamento dei ruoli;
	\item Definizione dello \glossario{sprint backlog};
	\item Discussione dei punti per il diario di bordo del 24 maggio;
\end{itemize}

\subsection{Discussione e decisioni}
\par Il meeting è iniziato con una discussione sulle mansioni del progettista per il prossimo \glossario{sprint}. 
Dopodiché si è proposto un incontro interno tra programmatori e progettisti per valutare come muoversi nelle settimane a seguire. 
Questo incontro ha lo scopo di selezionare le attività essenziali per ultimare il collaudo di \glossario{Streamlit}, al fine di comprendere se il suddetto \glossario{framework} possa supportare concretamente il team nel soddisfacimento dei requisiti del progetto.
Dopo la riunione con la \glossario{Proponente}, infatti, il gruppo ha confermato la volontà di utilizzare \glossario{Streamlit}\ per realizzare l'interfaccia grafica, ponendosi come obiettivo l'implementazione delle feature chiave in modo da capirne la fattibilità.
\par Il team ha valutato inoltre l'adozione, dopo la prima revisione di avanzamento, del pattern architetturale \glossario{MVC} per suddividere le aree dell'applicativo. 
In sintesi, lo scopo finale per questo \glossario{sprint} sarà quello di stabilire se utilizzare definitivamente il \glossario{framework} Streamlit o se adottare altre tecnologie; tale decisione verrà ponderata sviluppando i \glossario{casi d'uso} più interessanti per il Proof of Concept.
\subsubsection{Metriche di qualità}
\par Il team ha deciso di organizzare un incontro (già menzionato nelle settimane precedenti, ma posticipato) con il Professor. Riccardo Cardin, per chiarire alcuni dubbi sulle metriche di qualità e sull'\AdR. L'incontro sarà fissato tra la fine dello \glossario{sprint} corrente e l'inizio di quello successivo.
La proposta è stata avanzata dopo aver ricevuto un riscontro da parte di un gruppo del secondo lotto, il quale ha espresso un parere positivo e trovato l'incontro costruttivo.
\par In seguito alla discussione con la \glossario{Proponente} riguardo le metriche di prodotto, il team ha considerato una rivisitazione delle metriche precedentemente individuate. Inoltre, su consiglio della Proponente, il gruppo ha analizzato la quantità delle metriche, cercando di trovare il giusto equilibrio.
L'obiettivo per lo \glossario{sprint}\ sarà quello di individuare e applicare degli strumenti che permettano un calcolo preciso delle metriche o, in alternativa, di definire delle procedure di calcolo personalizzate e rigorose. 
Andranno anche rivalutate le finalità delle metriche, specialmente quelle di prodotto, con una conseguente scrematura e riformulazione della struttura del \PdQ.
Pertanto, il team ha fissato un incontro a metà \glossario{sprint} per identificare le metriche più significative.
Per lo \glossario{sprint} corrente, il gruppo ha assegnato una priorità maggiore alle metriche di processo, in quanto richiedono un'applicazione immediata, mentre quelle di prodotto saranno approfondite verso la fine del periodo.
\subsubsection{Rendicontazione delle ore}
\par Un altro argomento discusso riguarda l'organizzazione e la rendicontazione delle ore produttive; per mantenere una distribuzione oraria uniforme, il team ha stabilito di avvisare, con sufficiente anticipo, il responsabile prima di un eventuale scostamento (in positivo o negativo) delle ore produttive.
\subsubsection{Pianificazione delle attività}
\par Il discorso è tornato a vergere attorno ai programmatori. Trattandosi del ruolo più critico, data la continua evoluzione delle tecnologie, il team ha ritenuto opportuno suddividere i macro-obiettivi in task di dimensione sufficientemente piccola. Le attività individuate sono le seguenti:
\begin{itemize}
	\item Scelta del \glossario{DBMS} (MySQL, PostgreSQL);
	\item Connessione al \glossario{database};
	\item Caricamento del \glossario{dizionario dati};
	\item Visualizzazione del dizionario dati in linguaggio naturale;
	\item Gestione del login per l'utente Tecnico; 
	\item Funzionalità di debug per il Tecnico;
	\item Test di correttezza del \glossario{prompt} generato;
	\item Ricerca del \glossario{dizionario dati}.
\end{itemize}
\par L'ultimo argomento affrontato prima della definizione dei punti per il diario di bordo riguarda l'organizzazione dei prossimi \glossario{sprint}. Il team ha considerato la possibilità di programmare degli \glossario{sprint} più brevi in vista della revisione \glossario{RTB}, in modo da concentrare gli sforzi sulle attività di massima priorità.

\subsubsection{Punti da trattare nel prossimo diario di bordo}
\par Il gruppo ha discusso i punti salienti da trattare nel diario di bordo del 24 maggio 2024.
\paragraph{Done}
\begin{itemize}
	\item Aggiornamento della documentazione (\PdP, \PdQ, \NdP, \AdR, \Gls);
	\item Allineamento delle conoscenze del gruppo sulle tecnologie (\glossario{txtai}\ e \glossario{Streamlit});
	\item Riorganizzazione del codice;
	\item Valutazione di modalità alternative per la ricerca semantica;
	\item Prova di traduzione della richiesta utente;
	\item Inserimento nelle \NdP delle descrizioni e dei metodi di calcolo associati alle metriche;
	\item Discussione con la \glossario{Proponente} riguardo le metriche di qualità;
	\item Individuazione di metriche specifiche per misurare la qualità del \glossario{prompt} generato.
\end{itemize}
\paragraph{Difficoltà}
\begin{itemize}
	\item Valutazione di \glossario{framework} alternativi, in particolare Django e Next.js;
	\item Test delle funzionalità avanzate di \glossario{txtai};
	\item Efficienza dei modelli di traduzione;
	\item Identificazione di strumenti e/o procedure per calcolare automaticamente le metriche.
\end{itemize}	
\paragraph{Todo}
\begin{itemize}
	\item Scelta del \glossario{DBMS} e connessione al \glossario{database};
	\item Sviluppo dei casi d'uso all'interno dell'applicativo;
	\item Implementazione della funzionalità di \glossario{debug};
	\item Configurazione iniziale del \glossario{framework} di test;
	\item Modalità di valutazione della correttezza del \glossario{prompt};
	\item Aggiornamento della documentazione (\PdP, \PdQ, \NdP, \AdR, \Gls);
	\item \glossario{Dockerizzazione} dell'ambiente di codifica.	
\end{itemize}
\paragraph{Dubbi}
\begin{itemize}
	\item Quantità di metriche da applicare;
	\item Dove documentare il processo di scelta dei range di tollerabilità, ossia in quale documento inserire le motivazioni dietro alla scelta dei range.
\end{itemize}

\par Gli argomenti sopracitati sono stati riportati nella presentazione per il diario di bordo del 24 maggio 2024.
