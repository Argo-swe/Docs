\section{Riunione}
\subsection{Ordine del giorno}
\begin{itemize}
	\item Retrospettiva \glossario{sprint}\ 3;
	\item Allineamento dei ruoli;
	\item Formazione del \glossario{backlog};
	\item Discussione dei punti per il diario di bordo del 24/05;
\end{itemize}

\subsection{Discussione e decisioni}
\par Il meeting inizia con una richiesta sulle mansioni del progettista per il prossimo \glossario{sprint}. 
Si propone pertanto un incontro interno tra programmatori e progettisti per discutere sul come muoversi nelle settimane a seguire. 
Questo incontro viene fatto in seguito all'incontro con la Proponente dove la presentazione del \glossario{framework}\ \glossario{Streamlit} non ha toccato i punti di interesse della Proponente che richiedeva una presentazione che spiegasse come la tecnologia potesse risolvere i requisiti del progetto (questa parte gente devo rivederla. O così o più blanda).
\par La proposta è quella quindi di usare come stumento proprio \glossario{Streamlit}\ per sviluppare le feature richieste per il PoC in modo da capirne la fattibilità dell'implementazione. 
Si propone inoltre uno sviluppo di tipo \glossario{MVC} per dividere le aree dell'applicativo. 
Lo scopo finale per questo \glossario{sprint} sarà per tanto quello di stabilire se utilizzare definitivamente il \glossario{framework} in questione o se adottare altre tecnologie.
\par Viene menzionata poi la possibilità di organizzare un incontro con il Professor. Riccardo Cardin, definendo delle richieste sulle metriche e su consigli in via generale per l'avanzamento del progetto. 
La richiesta viene avanzata dopo aver avuto contatti da un altro gruppo il quale, effettuato l'incontro, ha espresso un parere molto positivo e trovato l'incontro molto utile e soddisfacente.
\par Avendo menzionato le metriche, la discussione si sposta su queste, visti i dubbi espressi alla Proponente per la definizione di metriche del prodotto (e un consiglio sulla quantità), i quali hanno avuto una risposta che ha lasciato al gruppo da riflettere su un'eventuale rivisitazione di quelle create fino ad ora. 
L'obbiettivo per lo \glossario{sprint}\ è quello di utilizzare degli sturmenti che permettano un calcolo preciso delle metriche o lo sviluppo interno di queste con operazioni rigorose. 
Sono inoltre da rivalutare gli obbiettivi delle metriche, con una conseguente rivisitazione di queste per fare una cernita e eventuali modifiche di quelle già stabilite.
Si propone per tanto un incontro da fare verso la fine dello \glossario{sprint}, per l'affermazione di metriche per il pordotto.
Al momento le metriche definite come fondamentali sono quelle di processo, con la conseguenza che quelle di prodotto, non indispensabili, passano in secondo piano a fine \glossario{sprint}.
Rimane quindi un dubbio sul come procedere con le metriche per il ruolo di Amministratore: la definizione delle metriche riprenderà duncque dopo l'incontro con il Professor. Riccardo Cardin e per il momento il focus sarà quello dell'avanzamento delle \NdP.
\par Un punto successivo trattato riguarda l'organizzazione e la rendicontazione delle ore produttive: per tenere in asse tutto il gruppo, si propone di avvisare prima di un eventuale sforo (in positivo o negativo) delle ore produttive, così da procedere a misure per sistemare il deficit o il surplus.
\par Il discorso torna poi a vergere attorno ai programmatori: essendo questo il ruolo più ostico per la novità delle tecnologie utilizzate, vengono spiegati gli obbiettivi che si vorrebbero implementare durante lo \glossario{sprint}\ al fine della creazione del PoC. I punti sono i seguenti:
\begin{itemize}
	\item Connessione al \glossario{database};
	\item Quale tecnologia usare per ospitare il \glossario{database} (\glossario{PostgreSQL}, \glossario{MongoDB}, \glossario{SQLite}, ...);
	\item Caricamento del dizionario dati;
	\item Visualizzazione del dizionario dati con vista per l'utente;
	\item Pagina e gestione del \glossario{login}; 
	\item Debug dell'applicazione;
	\item Sezione di selezione della lingua;
	\item Ricerca del dizionario dati.
\end{itemize}
\par L'ultimo punto trattato prima della definizione dei punti per il diario di bordo riguarda l'organizzazione dei prossimi \glossario{sprint}: si propone di fare degli sprint più brevi in vista della prima revisione, così da poter definire e concludere delle attività in maniera più rapida.

\subsubsection{Punti da trattare nel prossimo diario di bordo}
\par Il gruppo ha discusso i punti salienti da trattare nel diario di bordo del 24 maggio 2024.
\paragraph{Done}
\begin{itemize}
	\item Aggiornamento della documentazione (\PdP, \PdQ, \NdP, \AdR, \Gls);
	\item Incontro per allineare le conoscenze del gruppo sulle tecnologie (\glossario{txtai}\ e \glossario{Streamlit});
	\item Primo refactoring del codice;
	\item Valutazione di modalità alternative per la ricerca semantica;
	\item Prova di traduzione della richiesta utente;
	\item Inserimento nelle \NdP delle descrizioni e metodi di calcolo associati alle metriche;
	\item Discussione con la Proponente in merito alle metriche di qualità;
	\item Individuazione di metriche specifiche per la qualità del \glossario{prompt} generato.
\end{itemize}
\paragraph{Difficoltà}
\begin{itemize}
	\item Valutazione di framework alternativi;
	\item Funzionalità avanzate di \glossario{txtai};
	\item Efficienza dei modelli di traduzione;
	\item Calcolo automatico delle metriche.
\end{itemize}	
\paragraph{Todo}
\begin{itemize}
	\item Scelta di \glossario{DBMS} e connessione a \glossario{database};
	\item Sviluppo dei casi d’uso all’interno dell’applicativo;
	\item Sviluppo della funzionalità di debug;
	\item Approfondimento e configurazione \glossario{framework} di test;
	\item Modalità di valutazione della correttezza del \glossario{prompt};
	\item Aggiornamento della documentazione (\PdP, \PdQ, \NdP, \AdR, \Gls);
	\item Dockerizzazione ambiente di codifica.	
\end{itemize}
\paragraph{Dubbi}
\begin{itemize}
	\item Quantità di metriche da applicare
	\item Dove documentare il processo di scelta dei range di tollerabilità: ossia in quale documento andare a inserire le descrizioni per i range di tollerabilità.
\end{itemize}

\par Gli argomenti sopracitati sono stati riportati nella presentazione per il diario di bordo del 22 maggio 2024.