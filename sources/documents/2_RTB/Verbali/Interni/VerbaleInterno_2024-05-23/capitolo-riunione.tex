\section{Riunione}
\subsection{Ordine del giorno}
\begin{itemize}
	\item Retrospettiva \glossario{sprint}\ 3;
	\item Allineamento dei ruoli;
	\item Definizione dello \glossario{sprint backlog};
	\item Discussione dei punti per il diario di bordo del 24 maggio;
\end{itemize}

\subsection{Discussione e decisioni}
\par Il meeting è iniziato con una discussione sulle mansioni del progettista per il prossimo \glossario{sprint}. 
Dopodiché si è proposto un incontro interno tra programmatori e progettisti per valutare come muoversi nelle settimane a seguire. 
Questo incontro ha lo scopo di selezionare le attività essenziali per ultimare il collaudo di \glossario{Streamlit}, al fine di comprendere se il suddetto \glossario{framework} possa supportare concretamente il team nel soddisfacimento dei requisiti del progetto.
Dopo la riunione con la \glossario{Proponente}, infatti, il gruppo ha confermato la volontà di utilizzare \glossario{Streamlit}\ per realizzare l'interfaccia grafica, ponendosi come obiettivo l'implementazione delle feature chiave in modo da capirne la fattibilità.
\par Il team ha valutato inoltre l'adozione, dopo la prima revisione di avanzamento, del pattern architetturale \glossario{MVC} per suddividere le aree dell'applicativo. 
Lo scopo finale per questo \glossario{sprint} sarà per tanto quello di stabilire se utilizzare definitivamente il \glossario{framework} in questione o se adottare altre tecnologie: ciò verrà fatto sviluppando i casi d'uso interessanti al fine del Proof of Concept e studiarne la fattibilità.
\par È stata poi menzionata la possibilità di organizzare un incontro con il Professor. Riccardo Cardin, definendo delle richieste sulle metriche e su consigli in via generale per l'avanzamento del progetto. 
La richiesta viene avanzata dopo aver avuto contatti da un altro gruppo il quale, effettuato l'incontro, ha espresso un parere molto positivo e trovato l'incontro estremamente utile e soddisfacente.
\par Avendo menzionato le metriche, la discussione si è spostata su queste, visti i dubbi espressi alla Proponente per la definizione di metriche del prodotto (e un consiglio sulla quantità), i quali hanno avuto una risposta che ha lasciato al gruppo da riflettere su un'eventuale rivisitazione di quelle create fino ad ora. 
L'obbiettivo per lo \glossario{sprint}\ sarà quello di utilizzare degli sturmenti che permettano un calcolo preciso delle metriche o lo sviluppo interno di queste con operazioni rigorose. 
Sono inoltre da rivalutare gli obbietivi delle metriche, con una conseguente rivisitazione di queste per fare una cernita e eventuali modifiche di quelle già stabilite.
Si è proposto per tanto un incontro da fare verso la fine dello \glossario{sprint}, per l'affermazione di metriche per il pordotto.
Al momento le metriche definite come fondamentali sono quelle di processo, con la conseguenza che quelle di prodotto, non indispensabili, passano in secondo piano a fine \glossario{sprint}.
Rimane quindi un dubbio sul come procedere con le metriche per il ruolo di Amministratore: la definizione delle metriche riprenderà dunque dopo l'incontro con il Professor. Riccardo Cardin e per il momento il focus sarà quello dell'avanzamento delle \NdP.
\par Un punto successivo trattato riguarda l'organizzazione e la rendicontazione delle ore produttive: per tenere in asse tutto il gruppo, si è propostp di avvisare prima di un eventuale sforo (in positivo o negativo) delle ore produttive, così da procedere a misure per sistemare il deficit o il surplus orario.
\par Il discorso è tornato poi a vergere attorno ai programmatori: essendo questo il ruolo più ostico per la novità delle tecnologie utilizzate, sono stati spiegati gli obbietivi che si vorrebbero implementare durante lo \glossario{sprint}\ al fine della creazione del Proof of Concept. I punti sono i seguenti:
\begin{itemize}
	\item Scelta del \glossario{DBMS} (MySQL, PostgreSQL);
	\item Quale tecnologia usare per ospitare il \glossario{database} (\glossario{PostgreSQL}, \glossario{MongoDB}, \glossario{SQLite}, ...);
	\item Caricamento del \glossario{dizionario dati};
	\item Visualizzazione del dizionario dati in linguaggio naturale;
	\item Gestione del login per l'utente Tecnico; 
	\item Funzionalità di debug per il Tecnico;
	\item Sezione di selezione della lingua;
	\item Ricerca del dizionario dati.
\end{itemize}
\par L'ultimo punto trattato prima della definizione dei punti per il diario di bordo riguarda l'organizzazione dei prossimi \glossario{sprint}: si è proposto di fare degli \glossario{sprint} più brevi in vista della prima revisione, così da poter definire e concludere delle attività in maniera più rapida.

\subsubsection{Punti da trattare nel prossimo diario di bordo}
\par Il gruppo ha discusso i punti salienti da trattare nel diario di bordo del 24 maggio 2024.
\paragraph{Done}
\begin{itemize}
	\item Aggiornamento della documentazione (\PdP, \PdQ, \NdP, \AdR, \Gls);
	\item Allineamento delle conoscenze del gruppo sulle tecnologie (\glossario{txtai}\ e \glossario{Streamlit});
	\item Riorganizzazione del codice;
	\item Valutazione di modalità alternative per la ricerca semantica;
	\item Prova di traduzione della richiesta utente;
	\item Inserimento nelle \NdP delle descrizioni e dei metodi di calcolo associati alle metriche;
	\item Discussione con la \glossario{Proponente} riguardo le metriche di qualità;
	\item Individuazione di metriche specifiche per misurare la qualità del \glossario{prompt} generato.
\end{itemize}
\paragraph{Difficoltà}
\begin{itemize}
	\item Valutazione di \glossario{framework} alternativi, in particolare Django e Next.js;
	\item Test delle funzionalità avanzate di \glossario{txtai};
	\item Efficienza dei modelli di traduzione;
	\item Identificazione di strumenti e/o procedure per calcolare automaticamente le metriche.
\end{itemize}	
\paragraph{Todo}
\begin{itemize}
	\item Scelta del \glossario{DBMS} e connessione al \glossario{database};
	\item Sviluppo dei casi d'uso all'interno dell'applicativo;
	\item Implementazione della funzionalità di \glossario{debug};
	\item Configurazione iniziale del \glossario{framework} di test;
	\item Modalità di valutazione della correttezza del \glossario{prompt};
	\item Aggiornamento della documentazione (\PdP, \PdQ, \NdP, \AdR, \Gls);
	\item \glossario{Dockerizzazione} dell'ambiente di codifica.	
\end{itemize}
\paragraph{Dubbi}
\begin{itemize}
	\item Quantità di metriche da applicare;
	\item Dove documentare il processo di scelta dei range di tollerabilità, ossia in quale documento inserire le motivazioni dietro alla scelta dei range.
\end{itemize}

\par Gli argomenti sopracitati sono stati riportati nella presentazione per il diario di bordo del 24 maggio 2024.
