\section{Riunione}
\subsection{Ordine del giorno}
\begin{itemize}
	\item Discussione modifiche \AdR;
	\item Discussione sui dubbi dell'impostazione del layout (\glossario{front-end});
	\item Organizzazione di un incontro in presenza;
	\item Pianificazione dello \glossario{sprint} 7.
\end{itemize}

\subsection{Discussione e decisioni}
\subsubsection{Discussione modifiche \AdR}
\par Il gruppo si è incontrato inizialmente per esaminare le modifiche fatte dall'analista al documento di \AdR. Le modifiche consistono prevalentemente in una rivisitazione e sistemazione completa del documento, dove sono state applicate le correzioni evidenziate durante l'incontro con il Professor Riccardo Cardin. Sono stati inoltre aggiunti casi d'uso che riguardano le interazioni tra l'Utente e l'interfaccia dell'applicativo.

\subsubsection{Discussione sui dubbi dell'impostazione del layout (\glossario{front-end})}
\par Il gruppo ha poi discusso del lavoro fatto dai programmatori nell'impostazione del layout del \glossario{front-end}. Sono stati riscontrati alcuni dubbi sul posizionamento del footer e dell'header. Quest'ultimo si è deciso di spostarlo su una barra di navigazione laterale che permetta l'accesso anche alle impostazioni per la futura generazione del \glossario{prompt}. Inoltre si è discusso del posizionamento dei selettori di lingua e BDMS che verranno spostati nella parte in alto a destra, invece che in basso vicino alla textarea dove viene fatta la richiesta in linguaggio naturale. Un ultimo dubbio riguardava la strutturazione del dialog di login e la sua presentazione.

\subsubsection{Organizzazione di un incontro in presenza}
\par Viene poi proposta l'organizzazione di un incontro in presenza in data lunedì 24 giugno. Quest'incontro avrà lo scopo di definire le attività per il prossimo \glossario{sprint} e rendere più veloce la transizione da un ruolo ad un altro per i membri del gruppo. La proposta viene approvata e il gruppo decide di ritrovarsi nella data proposta.

\subsubsection{Pianificazione dello \glossario{sprint} 7}
\par Il gruppo ha poi discusso della pianificazione dello \glossario{sprint} 7 con la divisione delle ore per ruolo e l'assegnazione dei ruoli ai membri.