\section{Riunione}
\subsection{Ordine del giorno}
\begin{itemize}
	\item Discussione sulle modifiche al documento di \AdR;
	\item Analisi del layout \glossario{front-end};
	\item Organizzazione di un incontro in presenza;
	\item Pianificazione dello \glossario{sprint} 7.
\end{itemize}

\subsection{Discussione e decisioni}
\subsubsection{Discussione modifiche \AdR}
\par Il gruppo si è incontrato per esaminare le modifiche effettuate dall'analista al documento di \AdR. Le modifiche consistono in una revisione completa del documento, con l'obiettivo di applicare le correzioni evidenziate durante l'incontro con il Professor Riccardo Cardin. Inoltre, sono stati aggiunti i requisiti relativi alle interazioni tra l'Utente e l'interfaccia dell'applicativo.

\subsubsection{Discussione sull'impostazione del layout front-end}
\par In seguito, il gruppo ha analizzato il lavoro svolto dai programmatori nell'impostazione del layout \glossario{front-end}. Sono stati riscontrati alcuni dubbi sul posizionamento dell'header e del footer. Il team ha deciso di spostare il footer su una barra di navigazione laterale. È stato valutato inoltre il posizionamento dei selettori di lingua e \glossario{DBMS}; i widget grafici corrispondenti saranno posizionati nella sezione superiore della chat. Un ulteriore dubbio riguardava la struttura del dialog di login e la sua presentazione.

\subsubsection{Organizzazione di un incontro in presenza}
\par Successivamente, il gruppo ha organizzato un incontro in presenza per lunedì 24 giugno. Questo incontro ha lo scopo di definire le attività per il prossimo \glossario{sprint} e agevolare la rotazione dei ruoli.

\subsubsection{Pianificazione del settimo sprint}
\par Infine, il team ha redatto il preventivo per lo \glossario{sprint} 7, indicando la distribuzione delle ore per ciascun ruolo e assegnando i ruoli ai membri del gruppo.