\section{Riunione}
\subsection{Ordine del giorno}
\begin{itemize}
	\item Definizione della checklist delle attività da ultimare nella documentazione.
	\item Discussione dei punti da trattareal diario di bordo del 19 luglio.
\end{itemize}

\subsection{Discussione e decisioni}
\subsubsection{Definizionde della checklist per la documentazione}
\par Il gruppo si è ritrovato inizialmente per dare una definizione più precisa delle attività da completare in vista della prossima revisione per la \RTB. Tali attività erano già state discusse, ma sono ora state definite concretamente all'interno di un foglio in \glossario{Google Sheets} che conta anche di un checkbox per la completazione dell'attività, una descrizione e un assegnatario.
Ciò è stato fatto per dividere già le attività tra i ruoli durante la riunione. Verranno poi creati gli opportuni ticket su \glossario{Jira} per indicare anche una data di scadenza per l'attività e definire l'assegnatario in maniera formale. Alcune attività sono anche già state concluse nel periodo trascorso dall'ultimo meeting.
Alcune di queste attività da concludere, comprendono:
\begin{itemize}
	\item Aggiornamento dei consunitivi fino allo \glossario{sprint} 8 e aggiunta dei diagrammi di gnatt;
	\item Revisionare il documento dell'\AdR dopo la correzione del Professor Riccardo Cardin;
	\item Aggiornamento del \Gls\ con le definizioni mancanti;
	\item Aggiuntgere sezioni di validazione, verifica, revisione congiunta, revisione e risoluzione dei problemi nelle \NdP;
	\item Sezione di sviluppo dell'\AdR\ nelle \NdP;
	\item Espansione della sezione strumenti nelle \NdP;
	\item Pianificazione dello \glossario{sprint} 9;
	\item Definizione delle attività del verificatore nelle \NdP;
	\item Revisione delle \NdP;
	\item Revisione del \PdP;
\end{itemize} 

\subsubsection{Discussione dei punti per il diario di bordo}
\par Il gruppo ha discusso i punti salienti da trattare nel diario di bordo del 19 luglio.
\paragraph{Done}
\begin{itemize}
	\item Incontro RTB con il professor Cardin;
	\item Espansione delle NdP nelle seguenti sezioni:
	\begin{itemize}
		\item Sviluppo;
		\item Verifica; 
		\item Validazione;
		\item Strumenti.
	\end{itemize}
	\item Stesura sezioni consuntivo e preventivo sprint nel PdP;
	\item Inizio stesura Specifica tecnica.
\end{itemize}
\paragraph{Difficoltà}
\begin{itemize}
	\item Non sono state riscontrate difficoltà durante questo periodo.
\end{itemize}	

\paragraph{Todo}
\begin{itemize}
	\item Presentazione RTB con il Professor Vardanega;
	\item Ricerca di pattern architetturali per il back-end e front-end;
	\item Inizio progettazione logica;
	\item Completare la stesura della sezione dedicata alle scelte tecnologiche;
	\item Eventuali correzioni alla documentazione post colloquio RTB;
	\item Aggiornamento della documentazione (Piano di Progetto, Piano di Qualifica, Specifica tecnica).
\end{itemize}

\paragraph{Dubbi}
\begin{itemize}
	\item Non sono stati riscontrati dubbi durante questo periodo.
\end{itemize}

\vspace{0.5\baselineskip}
\par Gli argomenti sopracitati sono stati riportati nella presentazione per il diario di bordo del 19 luglio.