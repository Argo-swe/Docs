\section{Riunione}
\subsection{Ordine del giorno}
\begin{itemize}
	\item Stesura della lettera di presentazione per la revisione \glossario{RTB};
	\item Definizione della struttura della presentazione dei requisiti e delle tecnologie;
	\item Suddivisione delle attività in vista della RTB;
	\item Rilascio della versione 1.0.0 del \glossario{PoC}.
\end{itemize}

\subsection{Discussione e decisioni}
\subsubsection{Stesura della lettera di presentazione per la RTB}
\par Il gruppo si è ritrovato per discutere ed elaborare la lettera di presentazione per la prima revisione di avanzamento. 
Nella lettera sono stati inseriti i seguenti riferimenti:
\begin{itemize}
	\item Documentazione di progetto (\AdR\ e verbali);
	\item Repository ChatSQL, contentente il PoC.
\end{itemize}

\vspace{0.5\baselineskip}
\par In seguito, il team ha definito la struttura e i punti da trattare nella presentazione dei requisiti e delle tecnologie. La presentazione è stata divisa in 7 sezioni principali. Gli argomenti individuati sono i seguenti:
\begin{itemize}
	\item Panoramica generale del prodotto;
	\item Requisiti funzionali e non funzionali;
	\item Architettura della web app; 
	\item Tecnologie back-end: \glossario{txtai} e Python;
	\item Tecnologie back-end: \glossario{FastAPI}, Pydantic e SQLAlchemy;
	\item Tecnologie front-end: \glossario{Vue.js}, Axios e PrimeVue;
	\item Funzionalità del Proof of Concept e dimostrazione dell'applicativo.
\end{itemize}

\vspace{0.5\baselineskip}
\par Gli argomenti sopracitati sono stati inseriti all'interno di una presentazione Google Slides tramite elenco puntato.

\subsubsection{Suddivisione delle attività in vista della RTB}
\par Il gruppo ha identificato le attività da completare in vista della RTB. In particolare, sono state fissate le seguenti attività:
\begin{itemize}
	\item Aggiornamento del \PdP\ (preventivo "a finire" e gestione dei rischi);
	\item Revisione del documento di \AdR;
	\item Espansione della sezione del \PdQ\ relativa all'Indice Gulpease;
	\item Ultimazione dei verbali mancanti o incompleti.
\end{itemize}

\vspace{0.5\baselineskip}
\par Dopo aver individuato le attività, il gruppo ha lavorato in autonomia per portarle a termine.

\subsubsection{Rilascio della versione 1.0.0 del PoC}
\par In seguito all'approvazione della \glossario{Proponente} e alla risoluzione dei conflitti con il main branch, il team ha rilasciato la prima versione del prodotto.
Una volta effettuato il merge, il gruppo ha eseguito dei test per verificare il corretto funzionamento del sistema.