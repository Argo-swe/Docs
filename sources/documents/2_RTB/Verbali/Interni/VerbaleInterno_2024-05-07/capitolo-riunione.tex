\section{Riunione}
\subsection{Ordine del giorno}
\begin{itemize}
	\item Allineare i membri del gruppo su ruoli nuovi;
	\item Discutere le nuove attività dello \glossario{sprint} 3;
	\item Stesura del backlog per lo \glossario{sprint} 3;
	\item Chiudere i branch ancora aperti;
	\item Continuazione dei lavori
\end{itemize}

\subsection{Discussione e decisioni}
\par Il gruppo si è ritrovato in sede Paolotti per aggiornare i membri entranti in ruoli nuovi sulle attività da svolgere per lo \glossario{sprint} 3. I membri sono stati informati sulle tecnologie usate nei relativi ruoli con brevi spiegazioni o dimostrazioni.
\par Il focus successivo si è incentrato sulla stesura delle attività da svolgere nella settimana corrente per avere delle attività che potessero essere in parte o completamente concluse per il prossimo diario di bordo. È stato quindi richiesto ad ogni ruolo di impostare delle attività che potessero essere svolte per il diario di bordo.
\par Impostate le attività, il meeting si è incentrato sulla cooperazione per l'avanzamento dei documenti e sulla chiusura dei \glossario{branch} ancora aperti. Di seguito sono anche iniziati i lavori sulla progettazione di un'interfaccia grafica che potesse interagire con una porzione funzionale che selezionasse il risultato più vicino alla \glossario{query} inserita dall'utente. Questi risultati serviranno per una prima bozza del Proof of Concept.
\par Il meeting si è concluso con le attività definite che verranno convertite in ticket su \glossario{Jira}: queste si concentreranno sull'avanzamento dei lavori sui documenti quali \AdR, \Gls, \PdP, \NdP\ e \PdQ e sul proseguimento dello sviluppo del Proof of Concept.