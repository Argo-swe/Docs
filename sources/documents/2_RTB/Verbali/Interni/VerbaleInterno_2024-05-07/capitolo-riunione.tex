\section{Riunione}
\subsection{Ordine del giorno}
\begin{itemize}
	\item Allineare i membri del gruppo sui ruoli inediti;
	\item Discutere le attività dello \glossario{sprint} 3;
	\item Chiudere i branch ancora aperti;
	\item Sessione di sviluppo collaborativa.
\end{itemize}

\subsection{Discussione e decisioni}
\par Il gruppo si è riunito presso la sede Paolotti per aggiornare i membri sulle mansioni da svolgere nello \glossario{sprint} 3. I componenti sono stati informati sulle tecnologie utilizzate nei rispettivi ruoli attraverso brevi spiegazioni e dimostrazioni.

\par In seguito, si è posta attenzione sulle attività da completare prima della presentazione del prossimo diario di bordo. È stato quindi richiesto a ciascun membro del gruppo di definire i task con priorità più alta.

\par Una volta impostate le attività, il meeting si è incentrato sulla stesura collaborativa dei documenti e sulla chiusura dei \glossario{branch} ancora aperti. Successivamente, il team ha avviato la progettazione dell'interfaccia grafica. L’obiettivo era integrare il back-end e il front-end al fine di ottenere una prima bozza del \glossario{Proof of Concept}.

\par Il meeting si è concluso con la discussione generale delle attività, che saranno poi formalizzate come ticket su \glossario{Jira}. Queste si concentreranno sull'aggiornamento dei documenti e sul proseguimento dello sviluppo del \glossario{PoC}.