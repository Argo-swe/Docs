\section{Riunione}
\subsection{Ordine del giorno}
\begin{itemize}
\item Discutere delle tecnologie per lo sviluppo dell'applicativo;
\item Presentare i dubbi emersi durante lo studio di \glossario{txtai};
\item Valutare l'ampliamento dei \glossario{casi d'uso}.
\end{itemize}

\subsection{Argomenti e temi dell'incontro}

\subsubsection{Tecnologie per lo sviluppo della web app}

\textbf{Domanda:} Tra le diverse tecnologie solitamente indicate per lo sviluppo di un'applicazione web (quali Django, Flask, React, Vue.js), abbiamo individuato il \glossario{framework} \glossario{Streamlit}. 
Il \glossario{framework} in questione permette di mantenere il linguaggio \glossario{Python} per sviluppare l'interfaccia grafica, seppur conceda meno spazio alla personalizzazione. 
Secondo lei, potrebbe essere una scelta opportuna?

\textbf{Risposta:} La \glossario{Proponente} non è familiare col framework Streamlit, ma ne approverebbe l'utilizzo a seguito di una presentazione informativa da parte del gruppo in cui si spiegano le motivazioni di tale scelta.

\subsubsection{Resoconto sullo studio preliminare di txtai}

\textbf{Descrizione:} Il programmatore espone i risultati conseguiti durante lo studio della libreria \glossario{txtai}, con particolare enfasi sulla componente di \glossario{indicizzazione} e ricerca semantica. 
Pone poi l'attenzione sulla necessità o meno delle fasi di pre-processing della richiesta dell'utente, nel tentativo di semplificare la ricerca semantica. 
L'idea del team sarebbe di processare la richiesta in linguaggio naturale per \glossario{tokenizzare} solo le componenti semantiche ritenute fondamentali. Allo stesso modo, sarebbe prevista un'elaborazione del \glossario{dizionario dati} in formato \glossario{JSON} per renderlo simile al linguaggio naturale, cosicché i \glossario{modelli}, allenati su quest'ultimo, riescano ad estrarne con più facilità il significato.

\textbf{Risposta:} La \glossario{Proponente} ritiene che le fasi di pre-processing non siano necessarie, poiché il formato del \glossario{dizionario dati} risulta sufficiente per creare degli \glossario{embeddings} efficaci.
Illustra poi l'utilizzo di txtai per la ricerca semantica tramite sentence-BERTino di efederici, un modello operante su lingua italiana. Inoltre, la Proponente invita a trarre vantaggio dalle funzioni di load e save degli indici, per ridurre l'\glossario{overhead} e velocizzare la fase di testing.

\textbf{Descrizione:} Il programmatore espone il problema dello scarso punteggio dei risultati restituiti dalla funzione search di txtai, e suggerisce la mitigazione del problema tramite l'aggiunta di un indice intermedio. 

\textbf{Risposta:} La \glossario{Proponente} accoglie la proposta e ne invita la sperimentazione. 
Riguardo allo score basso degli output forniti dalla search, la Proponente suggerisce l'esplorazione di più modelli. 
Spiega inoltre che txtai utilizza, in maniera del tutto trasparente, la libreria \glossario{FAISS}, che si occupa della \glossario{vettorizzazione} e indicizzazione oltre che della ricerca semantica. In aggiunta, la Proponente suggerisce di confrontare FAISS con altre librerie.

\subsubsection{Analisi dei casi d'uso}

\textbf{Domanda:} L'analista illustra i \glossario{casi d'uso} relativi alla funzionalità di \glossario{debug} disponibile per l'utente Tecnico. Il report ideato dal team comprende una lista di tabelle con annesso il punteggio di rilevanza. 
La \glossario{Proponente} immagina delle funzionalità aggiuntive per il debugging?

\textbf{Risposta:} La \glossario{Proponente} conferma come lo scopo del debugging sia mostrare l'accuratezza e la completezza del dizionario dati caricato dal Tecnico. 
Suggerisce poi che venga fornita una spiegazione sull'eventuale esclusione di un risultato pertinente a favore di uno meno attinente.

\textbf{Descrizione:} L'analista chiede se sia ragionevole generare un elenco di punteggi, mostrando l'intero \glossario{indice} o gran parte di esso.

\textbf{Risposta:} La \glossario{Proponente} spiega che grazie alla logica di \glossario{back-end}, il processo di ricerca non avviene per calcolo sequenziale, ma attraversa una struttura gerarchica multistrato. Pertanto, la query utente non deve essere confrontata con tutti i vettori dell'indice. 
Eseguire la funzione search non è costoso, mentre lo sarebbe visitare in maniera sequenziale l'intero indice. 
L'unica operazione onerosa, per cui potrebbero essere necessari anche dei minuti, riguarda la creazione dell'indice; per tale motivo sono state introdotte le già menzionate funzioni load e save.

\subsubsection{Organizzazione della presentazione su Streamlit}

\textbf{Descrizione:} Il responsabile domanda alla \glossario{Proponente} se preferisca una presentazione del \glossario{framework} \glossario{Streamlit}, evidenziandone vantaggi e svantaggi, o se preferisca una relazione da spedire via email.

\textbf{Risposta:} La \glossario{Proponente} suggerisce al team di elaborare una breve presentazione da esporre durante l'incontro successivo.