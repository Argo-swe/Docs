\section{Riunione}
\subsection{Ordine del giorno}
\begin{itemize}
	\item Presentazione del \glossario{PoC} realizzato con le nuove tecnologie.
\end{itemize}

\subsection{Argomenti e temi dell'incontro}

\subsubsection{Presentazione del Proof of Concept}

\par Il gruppo espone il motivo dell'incontro, ovvero presentare la versione finale del Proof of Concept, realizzata con \glossario{Vue.js} come strumento per il \glossario{front-end} e \glossario{FastAPI} per il \glossario{back-end}.
In primo luogo, il team illustra i cambiamenti effettuati dall'ultima riunione, specialmente quelli riguardanti la gestione delle chiavi esterne composte.
Vengono poi menzionati i modelli in lingua italiana utilizzati per eseguire i test di generazione del \glossario{prompt}. In particolare, sono stati testati i modelli "sentence-BERTino" e "mmarco-sentence-BERTino" di efederici. I test in locale sono stati effettuati con \glossario{LMstudio} utilizzando "Llama 3". 
Questi test sono stati eseguiti mantenendo lo stesso \glossario{prompt}, e chiedendo a un \glossario{LLM} di riscrivere la frase dell'utente per un certo numero di volte. Le \glossario{query} risultanti sono state salvate in un documento di testo al fine di realizzare un confronto.

\par In seguito, il gruppo presenta il Proof of Concept tramite il quale la \glossario{Proponente} può osservare le funzionalità implementate, tra cui:
\begin{itemize}
	\item Processo di autenticazione per accedere alla sezione riservata all'admin;
	\item Generazione di un \glossario{prompt};
	\item Copia del prompt.
\end{itemize}

\vspace{0.5\baselineskip}
\par Successivamente, il team mostra il prompt generato a partire dalla seguente richiesta: "the surname of users who paid for all their orders with PayPal".
Il risultato viene copiato e incollato su ChatGPT e chat.lmsys.org per confrontare le query \glossario{SQL} generate da modelli differenti.

\par La \glossario{Proponente} si sofferma sulla query restituita da ChatGPT, che utilizza soltanto un operatore "NOT EXISTS". Il gruppo evidenzia che, sulla base dei test svolti, ChatGPT ritorna spesso \glossario{query} più contenute rispetto ad altri modelli.
La Proponente chiede di testare il \glossario{prompt} con altri modelli, utilizzando chat.lmsys.org. Ad ogni tentativo, la Proponente esamina con il gruppo il significato e la correttezza delle \glossario{query}, mostrando come alcuni modelli sembrino omettere dei controlli o costruire query eccessivamente complesse.
La Proponente domanda al gruppo se sono stati svolti dei test anche su un \glossario{database} locale; il gruppo afferma di aver svolto dei test con la stessa richiesta, senza verificare, però, il caso in cui un utente non abbia effettuato alcun ordine.
La Proponente sottolinea come non vi siano errori strutturali evidenti nella \glossario{query}. Tuttavia, da questo esempio si evince che potrebbero verificarsi condizioni di errore; pertanto, viene chiesto al gruppo di svolgere ulteriori test con \glossario{database} reali.

\vspace{1.5\baselineskip}
\par Al termine dei test, il gruppo riporta l'attenzione al Proof of Concept per illustrare le funzionalità rimanenti:
\begin{itemize}
	\item Debug della generazione del \glossario{prompt}, con la possibilità di scaricare un file in formato testuale;
	\item Gestione \glossario{CRUD} dei dizionari dati;
	\item Persistenza degli \glossario{indici};
	\item Processo di logout;
	\item Pagina della chat, con una sezione per le impostazioni (selezione del dizionario dati, del \glossario{DBMS} e della lingua);
	\item Visualizzazione di un'anteprima del dizionario dati;
	\item Eliminazione della cronologia della chat.
\end{itemize}

\vspace{0.5\baselineskip}
\par Infine, il team illustra le impostazioni di accessibilità con le quali è possibile cambiare la lingua, il tema e le dimensioni dell'interfaccia.

\par La Proponente rinnova l’invito a effettuare test su un database locale, al fine di verificare la correttezza sintattica e semantica delle query generate. Gli aspetti da valutare includono:
\begin{itemize}
	\item Correttezza delle query;
	\item Posizionamento delle parentesi;
	\item Complessità delle query;
	\item Operatori utilizzati (LIKE oppure =, != oppure <>).
\end{itemize}

\subsubsection{Conclusioni}
\par Il gruppo conclude con una domanda in vista della revisione \glossario{RTB}.

\par \textbf{Domanda:} Ritiene che sia necessario aggiungere un flag nel dizionario dati per segnalare i campi indicizzati del database?
\par \textbf{Risposta:} No, non è necessario ai fini della correttezza della query SQL. Bisogna considerare inoltre il rischio derivante dall'aggiunta di questa informazione nel prompt, che potrebbe indurre un \glossario{Large Language Model} a generare una query non appropriata.