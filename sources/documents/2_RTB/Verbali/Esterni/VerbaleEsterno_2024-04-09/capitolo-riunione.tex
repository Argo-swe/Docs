\section{Riunione}
\subsection{Ritrovo iniziale}
Il gruppo si è riunito circa mezz’ora prima dell’appuntamento con l’azienda per discutere la chiarezza delle domande e organizzare una scaletta. Sono stati formulati inoltre dei quesiti aggiuntivi per garantire un’interazione più fluida con la \glossario{Proponente}.

Il gruppo ha affidato il compito di moderare l’incontro al responsabile in carica, delegando la discussione sui \glossario{casi d’uso} agli analisti. Come concordato per via telematica, il referente di Zucchetti si è unito al meeting alle ore 10:30.

\subsection{Argomenti e temi dell'incontro}


\subsubsection{Diagramma dei casi d'uso}

\textbf{Domanda:} È stata ideata una bozza riguardante i \glossario{casi d'uso} per avere un riscontro dalla \glossario{Proponente} e capire se rispetta la visione dell’applicazione proposta nel \glossario{capitolato}.
Nel documento sono stati pensati tre \glossario{attori} principali: un Utente Generico, un Utente Non Autenticato e un Tecnico.
L'Utente Generico può visualizzare il \glossario{prompt} finale a seguito dell'inserimento della richiesta in linguaggio naturale e della selezione del \glossario{dizionario dati}. La visualizzazione della lista dei \glossario{dizionari dati} è sempre possibile per l'Utente Generico. Quest'ultimo, autenticandosi, acquisisce permessi aggiuntivi e diventa un Tecnico.
La motivazione di questa scelta deriva dalla seguente considerazione: un Utente Generico, una volta autenticato, diviene un Tecnico, il quale non ha più opportunità di fare il login, ma solo il logout. Se un Tecnico ereditasse le funzionalità direttamente da un Utente Generico, dotato dell'opportunità di autenticarsi, questa sarebbe disponibile anche al Tecnico. Da qui la necessità di distinzione tra Utente Generico e Utente Non Autenticato.

\textbf{Risposta:} La \glossario{Proponente} suggerisce di rimanere attinenti ai metodi appresi nel corso di Ingegneria del Software, mantenendo come priorità il rispetto dei \glossario{requisiti} minimi del \glossario{capitolato}.

\subsubsection{Gestione login per il Tecnico}

\textbf{Domanda:} La \glossario{Proponente} suggerisce un metodo diverso da quello proposto per separare la funzione di login dalle funzioni proprie del Tecnico?

\textbf{Risposta:} Nello schema proposto dai fornitori, è ragionevole che l'Utente Generico non sia interessato al login, tuttavia la soluzione ottimale sarebbe poterlo fare per ogni \glossario{attore}. Quindi, accedendo alla \glossario{web app}, ci si dovrebbe poter autenticare a priori.
Ciò che è stato ideato dai fornitori nei \glossario{casi d'uso} ammette l'interazione da parte dell'Utente Generico con l'applicazione senza autenticazione. La \glossario{Proponente} sarebbe favorevole ad estendere la funzione di login anche per questo \glossario{attore}, tenendo come riferimento prioritario le nozioni fornite dal corso. 

\subsubsection{Login per l'Utente Generico}

\textbf{Domanda:} C'è la preferenza di autenticare anche l'Utente Generico?

\textbf{Risposta:} La \glossario{Proponente} non ha preferenze specifiche, sottolinea però che se l'utente dovesse consumare risorse costose, come delle chiamate \glossario{API} a pagamento, sarebbe preferibile limitare l'accesso tramite il login. Perciò, se i fornitori decidessero di implementare il \glossario{requisito} opzionale delle chiamate \glossario{API} (sottoscrivendo un abbonamento a pagamento), allora la \glossario{Proponente} consiglierebbe di autenticare anche l'Utente Generico. Per il Tecnico invece, siccome i dati verranno resi in qualche modo \glossario{persistenti} e richiedono spazio di archiviazione, la funzione di login è necessaria.

\subsubsection{Numero di attori}

\textbf{Domanda:} La necessità della \glossario{Proponente} è quindi di avere solamente due \glossario{attori}?

\textbf{Risposta:} Sì, due \glossario{attori} sono adeguati.


\subsubsection{Ulteriori funzionalità per l'utente Tecnico}

\textbf{Domanda:} Il Tecnico necessita di ulteriori funzionalità?

\textbf{Risposta:} Alcuni gruppi del primo lotto hanno stabilito che il Tecnico può svolgere anche le attività proprie dell’Utente Generico, ma, quando invia delle richieste, potrebbe aver bisogno di una funzionalità di \glossario{debug}.
La struttura del \glossario{dizionario dati} sarà a discrezione dei fornitori e conterrà la lista delle tabelle e le relazioni tra di esse. Inoltre sarà presente sia una descrizione tecnica che una descrizione in linguaggio naturale: esse entreranno in relazione tra loro quando elaborate dal \glossario{modello}. In seguito alla definizione del \glossario{dizionario dati}, il Tecnico potrebbe aver bisogno di eseguire dei test per avere un riscontro sul legame tra queste descrizioni e la richiesta. Di conseguenza, una funzionalità di \glossario{debug} sarebbe utile al Tecnico per capire come il \glossario{dizionario dati} si armonizza con il \glossario{modello} per la generazione del \glossario{prompt}.


\subsubsection{Pre-prompt}

\textbf{Domanda:} Quanto è importante il \glossario{prompt engineering}?

\textbf{Risposta:} Quando il \glossario{dizionario dati} è di dimensioni ridotte (ad esempio un \glossario{dizionario dati} con tre tabelle), quest'ultimo può essere contenuto interamente all’interno del \glossario{prompt}. Tuttavia, se il \glossario{dizionario} è di grandi dimensioni, si presentano dei problemi, tra cui il costo elevato dell’elaborazione di un \glossario{prompt} con numerosi \glossario{token}. A livello teorico è possibile avere \glossario{prompt} di grandi dimensioni ma è meglio mantenerli asciutti, stabilendo quali siano le sezioni pertinenti alla richiesta dell’utente. Diventa così necessario capire il metodo con cui estrarre le porzioni rilevanti dal \glossario{dizionario dati} durante la creazione del \glossario{prompt}.
 A dimostrazione di ciò, chiedendo a un \glossario{LLM} di elaborare un \glossario{prompt} e aggiungendo progressivamente dettagli superflui, è evidente come il \glossario{modello} generi dati incoerenti.

\subsubsection{Funzionalità di debug}

\textbf{Domanda:} Il \glossario{debug} è uno stato intermedio rispetto a quanto visto?

\textbf{Risposta:} Sì, il \glossario{debug} è utile a capire come è stato generato il \glossario{prompt} e perché sono presenti determinate porzioni del \glossario{dizionario dati}.


\subsubsection{Stima della correttezza del prompt}

\textbf{Domanda:} Come si può stimare la correttezza del \glossario{prompt}?

\textbf{Risposta:} La richiesta dell'utente viene inserita nella sezione finale del \glossario{prompt}. Quello che ci interessa è il segmento precedente, ovvero la porzione del \glossario{dizionario dati} attinente alla richiesta in linguaggio naturale. Bisogna capire come si è arrivati all'estrazione di quella porzione che viene prefissata alla richiesta.

\subsubsection{Visualizzazione debug}

\textbf{Domanda:} Per il \glossario{debug}, si può sovrascrivere la funzionalità di visualizzazione del \glossario{prompt} quando l'\glossario{attore} che lo richiede è un Tecnico. È l’approccio corretto?

\textbf{Risposta:} Un’opzione sarebbe estendere il \glossario{caso d'uso} di partenza, per gestire la situazione in cui il Tecnico è interessato a ottenere, oltre al \glossario{prompt}, un \glossario{report} sull'interazione tra il \glossario{modello} e il \glossario{dizionario dati}.


\subsubsection{Modalità di visualizzazione per l'utente Tecnico}

\textbf{Domanda:} Il Tecnico possiede solo la visualizzazione con \glossario{debugging} o anche quella normale?

\textbf{Risposta:} Entrambe, perché sono due circostanze differenti. Se il \glossario{dizionario} è molto piccolo, la soluzione è semplice. Quando invece il \glossario{dizionario dati} è grande, il \glossario{prompt} può diventare fuorviante per il \glossario{modello} se contiene (o non contiene) determinate tabelle. Facendo una controprova ed eseguendo delle richieste in cui i dati forniti sono differenti dagli stretti necessari, si osserva che quando i dati sono in eccesso, il \glossario{modello} si confonde, quando invece mancano, genera dati casuali. La funzionalità di \glossario{debug} può aiutare il Tecnico a capire come riformulare il \glossario{dizionario dati} affinché l'applicazione restituisca il \glossario{prompt} desiderato.

\subsubsection{Utilizzo di txtai in locale}

\textbf{Domanda:} Durante il primo incontro, la \glossario{Proponente} aveva consigliato \glossario{txtai}. È collegato alla fase di \glossario{ricerca semantica}?

\textbf{Risposta:} Sì. Quello che è interessante è che la \glossario{ricerca semantica} in \glossario{txtai} opera tramite la \glossario{sentence similarity}; inoltre, si possono usare \glossario{modelli} molto piccoli ad uso di macchine meno potenti.
Utilizzando un \glossario{modello} specifico, che viene fatto girare su un computer della \glossario{Proponente} attrezzato con una GPU RTX 3060, si può osservare che, nonostante venga eseguito in locale, non ci sono ostacoli al funzionamento o problemi di surriscaldamento.


\subsubsection{LLM di Hugging Face}

\textbf{Domanda:} Da dove è stato selezionato il \glossario{modello} utilizzato nell'esempio precedente?

\textbf{Risposta:} Da \glossario{Hugging Face}. Si tratta della versione quantizzata di \glossario{OpenChat} in formato \glossario{GGUF}.

\subsubsection{Implementazione lingue alternative}

\textbf{Domanda:} Per adottare lingue diverse da quella del \glossario{dizionario dati}, basterebbero dei sinonimi nella lingua alternativa, inseriti all'interno del \glossario{dizionario}?

\textbf{Risposta:} A tal proposito si osserva che, eseguendo una richiesta in lingua inglese e fornendo una descrizione delle tabelle in italiano, il \glossario{modello} opera con successo.
Il \glossario{modello} è in grado di dedurre da sé cosa è richiesto, però bisogna ragionare su come trattare il \glossario{dizionario dati} avendo una richiesta in una lingua differente.

\subsubsection{Gestione delle richieste in lingue alternative}

\textbf{Domanda:} Avendo un \glossario{dizionario dati} da filtrare in base alla richiesta, quando arriva una richiesta in russo, è necessario gestire il filtraggio con la lingua corrispondente?

\textbf{Risposta:} Si possono caricare \glossario{dizionari} in lingue differenti, altrimenti, utilizzando \glossario{txtai}, non ce n'è bisogno. In alternativa, potreste impiegare dei \glossario{modelli} di traduzione, che sono solitamente molto piccoli. Su \glossario{Hugging Face}, per esempio, esistono i \glossario{modelli} di traduzione Helsinki.
Impiegando il \glossario{modello} Mistral, è possibile tradurre una richiesta dalla lingua italiana a quella inglese, fornendo la \glossario{query} \glossario{SQL} corretta.

\subsubsection{Valutazione criticità lingue alternative}

\textbf{Domanda:} A questo punto l'applicazione non diventa poco estendibile, perché è necessario ragionare prima per valutare le lingue disponibili?

\textbf{Risposta:} Sì. Altrimenti è possibile tradurre direttamente la richiesta, ma non è un \glossario{requisito} obbligatorio.

\subsubsection{Metodologia Agile}

\textbf{Domanda:} I fornitori valutano l’adozione di una metodologia \glossario{Agile} per la gestione di progetto. Sarebbe quindi prevista una riunione alla fine di ogni \glossario{sprint} (della durata di circa 2 settimane), sia interna che con il cliente. Secondo la \glossario{Proponente} è fattibile organizzare un incontro ogni due settimane?

\textbf{Risposta:} Sì. La \glossario{Proponente} attende una richiesta tramite email per programmare gli incontri. 
Tuttavia, la \glossario{Proponente} consiglia al gruppo di non impiegare una metodologia troppo stringente, specialmente per quanto riguarda l'organizzazione delle riunioni.
\subsubsection{Feedback}

\textbf{Domanda:} Ci sono delle attività su cui porre l'attenzione nel prossimo \glossario{sprint}?

\textbf{Risposta:} La sezione riguardante i \glossario{casi d'uso} è stata adeguatamente approfondita; sarà utile concentrarsi sull'apprendimento del processo di generazione del \glossario{prompt}.
Inoltre, è essenziale sviscerare l'interazione tra il \glossario{dizionario dati}, la richiesta dell'utente e la generazione del \glossario{prompt}.