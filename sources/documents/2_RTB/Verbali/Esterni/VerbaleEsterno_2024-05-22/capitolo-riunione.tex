\section{Riunione}
\subsection{Ordine del giorno}
\begin{itemize}
	\item
\end{itemize}

\subsection{Argomenti e temi dell'incontro}

\subsubsection{Presentazione \glossario{Streamlit}\ e confronto con altre tecnologie}

\par Vengono esposte delle diapositive dai Programmatori, per la discussione del \glossario{framework}\ \glossario{Streamlit}, poiché sconosciuto alla Proponente. Vengono indicati i vantaggi e gli svantaggi della tecnologia, qali:
\begin{itemize}
	\item Sviluppo del \glossario{backend}\ in \glossario{Python};
	\item Sviluppo del \glossario{frontend}\ in \glossario{TypeScript}\ e con implementazioni con la libreria \glossario{React};
	\item Utilizza l'estensione \glossario{JsXml};
	\item Tecnologia open source;
	\item Principalmente per data science e \glossario{machine learning}, visualizzazione dati, \glossario{LLM}, ricerca semantica e lavorazione con linguaggio naturale;
	\item Disponibilità di componenti interattivi integrati facilmente con widget;
	\item Creazione semplice di widget e un'istanziazione posta nella posizione più comoda di default;
	\item Comunità ampia e molto attiva, soprattutto nella risoluzione di problemi di personalizzaizone.
\end{itemize}

\par \textbf{Risposta:} La Proponente non si dichiara soddisfatta con la presentazione; questo poiché essa pone come obbietivo una discussione tecnica della tecnologia presa in esame, senza soffermarsi su come questa possa e debba risolvere i problemi posti al capitolato. 
Viene inoltre spiegato che adottando una tecnologia si accettano i ricschi che ne conseguono: la Proponente si fida del produttore e richiede che siano rispettati i requisiti presentati nel capitolato. 
Gli aspetti tecnici del framework preso in analisi non sono di interesse.

\par \textbf{Domanda:} Il gruppo espone quindi l'opzione di raggiungere un prodotto soddisfacente con la tecnologia presentata e mostrare i risultati ottenuti tramite l'interazione tra \glossario{txtai}\ e \glossario{Streamlit}\ al prossimo incontro. 
Inoltre dichiara di aver già esplorato la tecnologia e iniziato le prime fasi di interazione, ma che queste non toccano ancora tutti i requisiti obbligatori proposti. 
Si riconferma inoltre che la tecnologia è valida poiché utilizza lo stesso linguaggio usato per lo sviluppo della parte funzionale del progetto.

\par \textbf{Risposta:} La proponente conferma l'opzione proposta, chiedendo di essere contattata qual'ora ci siano degli avanzamenti significativi sullo sviluppo dei casi d'uso principali.

\subsubsection{Discussione delle metriche di qualità}

\par \textbf{Domanda:} Vista la novità nell'approccio alle metriche per il gruppo, si voleva chiedere se ci sono delle metriche a priori che si vuole vengano rispettate: il dubbio è per tanto volto alla richiesta di un consiglio alla Proponente a riguardo.
