\section{Riunione}
\subsection{Ordine del Giorno}
\begin{itemize}
	\item Discussione sulla decisione finale per la scelta del capitolato
\end{itemize}

\subsection{Discussione e decisioni}
\subsubsection{Discussione sulla decisione finale per la scelta del capitolato}
All’inizio del meeting ogni membro del gruppo ha partecipato a una sessione di discussione, elencando pregi, difetti e riflessioni personali sui due capitolati considerati come possibili candidati alla scelta finale. Si inizia con una discussione riguardo pregi e difetti del progetto SyncCity.\\
I vantaggi rilevati del progetto SyncCity sono i seguenti:
\begin{itemize}
	\item Disponibilità dell’azienda e possibilità di incontri anche in sede;
	\item Concretezza e possibilità didattiche; 
	\item Interesse del gruppo nelle tematiche di IoT e Big Data;
	\item Conoscenza dell’ambito di applicazione e di alcune tecnologie proposte.
\end{itemize}
Gli svantaggi osservati del capitolato sono i seguenti:
\begin{itemize}
	\item Preoccupazione per le richieste aggiuntive sui vincoli facoltativi dopo l’osservazione dei risultati ottenuti dai gruppi del primo lotto;
	\item Preoccupazione per il possibile conflitto di interessi sull'aggiudicazione dell'appalto, quando il numero di richieste eccede gli slot disponibili.
\end{itemize}

Si passa poi a una discussione affine per il progetto ChatSQL.\\
I vantaggi riscontrati per il progetto in questione sono i seguenti:
\begin{itemize}
	\item Idea attuale con risvolti interessanti dal punto di vista didattico e professionale;
	\item Utilizzo di strumenti e tecnologie all’avanguardia;
	\item Sfida di apprendimento stimolante;
	\item Maggiore disponibilità di slot per l’aggiudicazione dell’appalto.	
\end{itemize}
Come criticità invece sono stati evidenziati:
\begin{itemize}
	\item L'astrazione del progetto e la sua ampiezza, se confrontate con le stesse caratteristiche dell'altro capitolato in esame;
	\item La scarsa conoscenza dell’ambito trattato e delle tecnologie.	
\end{itemize}
Viene poi avviato un sondaggio che vede il 57\% del gruppo a favore di SyncCity e il 43\% a favore di ChatSQL. Viene quindi aggiornato il tabellone di coordinamento dei gruppi (con SyncCity a priorità 1 e ChatSQL a priorità 2) e viene inviato un messaggio per organizzare un incontro con i rappresentanti dei vari gruppi volto alla risoluzione del conflitto.

\subsubsection{Obiettivi fissati}
\begin{itemize}
	\item Divisione in sottogruppi per la stesura di una bozza del documento di valutazione dei capitolati, la cui scadenza viene impostata a domenica 17 marzo. La divisione viene presentata nella tabella a fine pagina;
	\item Prossimo meeting per la discussione della bozza del documento di valutazione;
	\item Riduzione dei meeting telematici dopo un’osservazione sull’impiego di tempo che questi comportano. Questo per ridurre meeting superflui e concentrarsi su incontri per eventi significativi.
\end{itemize}
Gruppi per la stesura del documento di valutazione:
\begin{itemize}
	\item SyncCity[C6]
	\begin{itemize}
		\item \mattia
		\item \marco
		\item \tommaso
	\end{itemize}
	\item ChatSQL[C9]\begin{itemize}
		\item \sebastiano
		\item \martina
	\end{itemize}
	\item EasyMeal[C3]
	\begin{itemize}
		\item \raul
		\item \riccardo
	\end{itemize}
\end{itemize}