\subsubsection{UC5 - Generazione prompt}\label{UC5}
\paragraph*{Descrizione}
L’Utente desidera ottenere un prompt al fine di utilizzarlo poi con LLM esterni per generare \glossario{query} sql fornendo parti ristrette di \glossario{dizionario dati}.

\paragraph*{Attori principali}
Utente

\paragraph*{Precondizioni}
\begin{itemize}
  \item L'applicazione è stata avviata con successo;
  \item L’autenticazione ad Utente è andata a buon fine;
  \item È presente almeno un \glossario{dizionario dati} nel sistema.
\end{itemize}

\paragraph*{Postcondizioni}
\begin{itemize}
  \item Il sistema genera un prompt in base alla richiesta ricevuta e al \glossario{dizionario dati} scelto. Attraverso dei metadati restituisce un prompt con il quale stabilisce quali parti di database sono maggiormente necessarie ed efficienti per scrivere la richiesta ricevuta in SQL;
  \item L’Utente riceve il prompt generato.
\end{itemize}

\paragraph*{Scenario principale}
\begin{enumerate}
  \item L’Utente desidera ottenere il prompt che permette di generare la query SQL;
  \item L’Utente seleziona un \glossario{dizionario dati} sul quale baserà l’interrogazione;
  \item L’Utente sceglie la lingua su cui farà l’interrogazione.La lingua è in italiano di default se non viene cambiata;
  \item L’Utente inserisce una interrogazione in linguaggio naturale nel box testuale apposito;
  \item Inizia il processo di generazione del prompt e di raccolta di metadati.
\end{enumerate}

\paragraph*{Scenario alternativo}
\begin{enumerate}
  \item Errore nella generazione del prompt (\hyperref[UC6]{UC6}).
\end{enumerate}

\paragraph*{Inclusione}
\begin{itemize}
  \item Inserimento richiesta in linguaggio naturale (\hyperref[UC3]{UC3});
  \item Selezione \glossario{dizionario dati} (\hyperref[UC4]{UC4}).
\end{itemize}

\paragraph*{Estensione}
\begin{itemize}
  \item Cambio lingua (\hyperref[UC7]{UC7});
  \item Selezione e copia del prompt risultato (\hyperref[UC8]{UC8}).
\end{itemize}
