\section{Requisiti}
All'interno di questa sezione sarano elencati, in formato tabulare e raggruppati per categoria, i requisiti alla base dello sfiluppo dell'applicativo ChatSQL. Questi sono divisi nelle seguenti tipologie:
\begin{description}
  \item[Funzionali (F):] I requisiti funzionali corrispondono alle funzionalità del sistema. Ciò equivale all’insieme di azioni che l’utente e lo stesso sistema sono in grado di compiere;
  \item[Qualitativi (Q):] I requisiti di qualità si fondano sull'assolvimento degli standard qualitativi al fine di garantire e preservare la qualità del prodotto;
  \item[Di vincolo (V):] I requisiti di vincolo delineano le restrizioni e vincoli normativi da rispettare nel corso dello sviluppo dell’applicativo.
\end{description}
A ciascun requisito è inoltre assegnato un grado di importanza:
\begin{description}
  \item[Obbligatorio (O):] L'implementazione del requisito risulta inderogabile;
  \item[Desiderabile (D):] L'implementazione del requisito non viene specificata come obbligatoria ma risulta appetibile alla \glossario{Proponente}.
  \item[Opzionale (OP):] L'implementazione del requisito è lasciata alla discrezione del \glossario{Fornitore}.
\end{description}
A seguito di tale classificazione, ogni requisito verrà identificato con la forma:
\textbf{\[R.[Tipologia][Importanza][Codice]\]}
Dove \textbf{\(R\)} indica il termine \emph{Requisito}, \emph{Tipologia} e \emph{Importanza} fanno riferimento alle definizioni precedenti e sono indicate con le sigle corrispondenti, infine \emph{Codice} è un identificativo numerico univoco.

\subsection{Requisiti funzionali}
\begin{table}[H]
\begin{center}
  \begin{longtable}{|P{3cm}|P{6,5cm}|>{\arraybackslash}P{4cm}|}
    \hline
    \textbf{Codice} & \textbf{Descrizione} & \textbf{Fonti} \\
    \hline 
    
    \textbf{R.FD1} & L’Utente Non Autenticato deve poter effettuare la procedura di login. &  \hyperref[UC1]{UC1}, \emph{Verbale Esterno 2024-04-09}\\
    \hline \textbf{R.FD1.1} & L’Utente Non Autenticato deve poter inserire la propria e-mail per autenticarsi. & \hyperref[UC1point1]{UC1.1}\\
    \hline \textbf{R.FD1.2} & L’Utente Non Autenticato deve poter inserire la propria password per autenticarsi. & \hyperref[UC1point2]{UC1.2}\\
    \hline
  \end{longtable}
\end{center}
\caption{Requisiti funzionali}
\label{requisitifunzionali}
\end{table}

TODO proseguire/aggiungere tabelle requisiti

\subsection{Requisiti di qualità}
TODO

\subsection{Requisiti di vincolo}
TODO

\subsection{Tracciamento}
TODO

\subsection{Fonti - Requisiti}
Di seguito vengono mostrate le corrispondenze fonte - requisito, raggruppate per fonte.

\subsubsection{Fonti - Requisiti funzionali}
\begin{table}[H]
  \begin{center}
    \begin{longtable}{|P{4,5cm}|>{\arraybackslash}P{9cm}|}
      \hline
      \textbf{Fonte} & \textbf{Requisiti} \\
      \hline 
      
      \hyperref[UC1]{UC1} & R.FD1 \\
      \hline
    \end{longtable}
  \end{center}
  \caption{Fonti- Requisiti funzionali}
\end{table}

TODO proseguire con le tabelle fonti