\section{Casi d'uso}

\subsection{Scopo}
TODO
\subsection{Introduzione ai casi d'uso}
TODO

\subsection{Attori}
Il prodotto prevede tre attori:
\begin{itemize}
  \item Utente non autenticato: è un utente generico che si presenta nella pagina di login dell'applicativo;
  \item Utente autenticato: un utente che ha completato con successo l'autenticazione ed ha accesso alle funzionalità richieste per la generazione di un prompt utilizzando i dizionari dati precaricati;
  \item Tecnico: un utente che ha completato con successo l'autenticazione ed è registrato come tecnico, ha accesso a tutte le funzionalità dell'utente autenticato e in aggiunta può gestire i dizionari dati ed accedere a funzioanlità di testing e debug.
\end{itemize}

TODO immagine
\subsubsection{Dettagli aggiuntivi}
TODO
%Qua non lo fa quasi nessuno ma si può espandere se qualche punto dell'elenco merita di essere espanso (probabilmente no)

\subsection{Gestione degli errori}
TODO
%Spiegazione della presenza dei casi d'uso per le situazioni che portano ad errore (dimostrando che il sistema è consistente e robusto)

\subsection{Elenco dei \glossario{casi d'uso}}

\subsubsection{UC1 - Autenticazione}\label{UC1}
%TODO immagine\paragraph*{Descrizione}
L’Utente desidera inserire una richiesta in qualsiasi linguaggio naturale al fine di ottenere, in seguito, il prompt che selezioni la parte del dizionario dati più inerente alla richiesta.

\paragraph*{Attori principali} Utente

\paragraph*{Precondizioni}
\begin{itemize}
  \item L'applicazione è stata avviata con successo;
  \item L’autenticazione ad Utente è andata a buon fine;
  \item L’Utente ha in precedenza selezionato un \glossario{dizionario dati};
  \item Se non è stata selezionata una diversa lingua per l’inserimento quella di default è l’italiano, se è stata selezionata è quella su cui si baserà la traduzione.
\end{itemize}

\paragraph*{Postcondizioni}
\begin{itemize}
  \item L’Utente ha scritto nell’apposito campo di testo una interrogazione in linguaggio naturale.
\end{itemize}

\paragraph*{Scenario principale}
\begin{enumerate}
  \item L’Utente scrive una interrogazione nell’ apposito box su cui poi il sistema potrà produrre un prompt in seguito.
\end{enumerate}


\subsubsection{UC4 - Selezione dizionario dati}\label{UC4}
TODO immagine\paragraph*{Descrizione}
L’Utente desidera selezionare il \glossario{dizionario dati} sul quale basare in seguito l’interrogazione in linguaggio naturale.
\paragraph*{Attori principali} Utente

\paragraph*{Precondizioni}
\begin{itemize}
  \item L'applicazione è stata avviata con successo;
  \item L’autenticazione ad Utente è andata a buon fine;
  \item Nella applicazione web è stato caricato precedentemente almeno un dizionario dati.
\end{itemize}

\paragraph*{Postcondizioni}
\begin{itemize}
  \item Un dizionario dati è stato selezionato in modo corretto ed univoco;
  \item L’Utente ora può inserire una frase in linguaggio naturale.
\end{itemize}

\paragraph*{Scenario principale}
\begin{enumerate}
  \item L’Utente visualizza la lista dei dizionari disponibili;
  \item L’Utente seleziona un dizionario da quelli nella lista sul quale vuole operare.
\end{enumerate}


\subsubsection{UC5 - Generazione prompt}\label{UC5}
TODO immagine\paragraph*{Descrizione}
L’Utente desidera ottenere un prompt al fine di utilizzarlo poi con LLM esterni per generare \glossario{query} sql fornendo parti ristrette di dizionario dati.

\paragraph*{Attori principali} Utente

\paragraph*{Precondizioni}
\begin{itemize}
  \item L'applicazione è stata avviata con successo;
  \item L’autenticazione ad Utente è andata a buon fine;
  \item È presente almeno un dizionario dati nel sistema.
\end{itemize}

\paragraph*{Postcondizioni}
\begin{itemize}
  \item Il sistema genera un prompt in base alla richiesta ricevuta e al dizionario dati scelto. Attraverso dei metadati restituisce un prompt con il quale stabilisce quali parti di database sono maggiormente necessarie ed efficienti per scrivere la richiesta ricevuta in SQL;
  \item L’Utente riceve il prompt generato.
\end{itemize}

\paragraph*{Scenario principale}
\begin{enumerate}
  \item L’Utente desidera ottenere il prompt che permette di generare la query SQL;
  \item L’Utente seleziona un dizionario dati sul quale baserà l’interrogazione;
  \item L’Utente sceglie la lingua su cui farà l’interrogazione.La lingua è in italiano di default se non viene cambiata;
  \item L’Utente inserisce una interrogazione in linguaggio naturale nel box testuale apposito;
  \item Inizia il processo di generazione del prompt e di raccolta di metadati.
\end{enumerate}

\paragraph*{Scenario alternativo}
\begin{enumerate}
  \item Errore nella generazione del prompt (\hyperref[UC6]{UC6}).
\end{enumerate}

\paragraph*{Inclusione}
\begin{itemize}
  \item Inserimento richiesta in linguaggio naturale (\hyperref[UC3]{UC3});
  \item Selezione dizionario dati (\hyperref[UC4]{UC4}).
\end{itemize}

\paragraph*{Estensione}
\begin{itemize}
  \item Cambio lingua (\hyperref[UC7]{UC7});
  \item Selezione del prompt risultato (\hyperref[UC8]{UC8}).
\end{itemize}


\subsubsection{UC6 - Messaggio d'errore nella generazione del prompt}\label{UC6}

\paragraph*{Descrizione}
L’Utente ha cercato di generare un prompt con una frase non totalmente inerente al dizionario dati scelto.

\paragraph*{Attori principali} Utente

\paragraph*{Precondizioni}
\begin{itemize}
  \item L'applicazione è stata avviata con successo;
  \item L’autenticazione ad Utente è andata a buon fine;
  \item L’Utente ha inserito una frase in linguaggio naturale;
  \item L’Utente ha richiesto la generazione del prompt a partire dalla frase inserita.  
\end{itemize}

\paragraph*{Postcondizioni}
\begin{itemize}
  \item L’applicazione ha riscontrato problemi nella generazione del prompt a causa di una non adeguata aderenza tra la frase inserita e il dizionario dati scelto;
  \item Viene visualizzato un messaggio di errore per orientare l’Utente alla comprensione del funzionamento del sistema.
\end{itemize}

\paragraph*{Scenario principale}
\begin{enumerate}
  \item Dopo che l’Utente ha inserito una frase in linguaggio naturale e richiesto la generazione del prompt il sistema attua il meccanismo per identificare la parte di dizionario dati adeguata da restituire;
  \item Il sistema non è in grado di trovare le correlazioni adeguate e quindi non riesce a restituire un prompt coerente;
  \item Viene visualizzato un messaggio di errore.  
\end{enumerate}


\subsubsection{UC7 - Cambio lingua}\label{UC7}

\paragraph*{Descrizione}
L’Utente desidera inserire una richiesta in linguaggio naturale in una lingua diversa dall’italiano.

\paragraph*{Attori principali} Utente

\paragraph*{Precondizioni}
\begin{itemize}
  \item L'applicazione è stata avviata con successo;
  \item L’autenticazione ad Utente è andata a buon fine;
  \item L’Utente vuole inserire una frase in linguaggio naturale.
\end{itemize}

\paragraph*{Postcondizioni}
\begin{itemize}
  \item L'applicazione opera la generazione del prompt considerando la lingua selezionata.
\end{itemize}

\paragraph*{Scenario principale}
\begin{enumerate}
  \item L’Utente seleziona una lingua da una lista predefinita di lingue.La scelta è tra:
    \begin{itemize}
      \item Italiano;
      \item Inglese;
      \item Francese;
      \item Spagnolo;
      \item Tedesco.
    \end{itemize}
  \item L’Utente inserisce una interrogazione in linguaggio naturale nella lingua selezionata nell’ apposito box su cui poi il sistema potrà produrre un prompt in seguito. 
\end{enumerate}

TODO


\subsubsection{UC8 - Selezione e copia del prompt generato}\label{UC8}

\paragraph*{Descrizione}
L’Utente vuole selezionare il prompt che è stato generato dal sistema per poter usarlo all’interno della richiesta ad un LLM esterno al fine di generare una query.

\paragraph*{Attori principali} Utente

\paragraph*{Precondizioni}
\begin{itemize}
  \item L'applicazione è stata avviata con successo;
  \item L’autenticazione ad Utente è andata a buon fine;
  \item L’Utente ha generato un prompt con successo a partire da una richiesta in linguaggio naturale.  
\end{itemize}

\paragraph*{Postcondizioni}
\begin{itemize}
  \item Il prompt che è stato generato dal sistema come risultato dell’interrogazione è copiato negli appunti del sistema dell’Utente.
\end{itemize}

\paragraph*{Scenario principale}
\begin{enumerate}
  \item Dopo aver visualizzato il prompt di risposta del sistema l’Utente seleziona la funzione di copia;
  \item L’Utente salva nei suoi appunti di sistema una copia del prompt che potrà poi incollare su LLM esterni per la generazione della query.
\end{enumerate}


\subsubsection{UC9 - Visualizzazione lista dizionario dati}\label{UC9}
TODO immagine\paragraph*{Descrizione}
L’Utente visualizza la lista dei dizionari dati che sono stati caricati nel sistema.

\paragraph*{Attori principali} Utente

\paragraph*{Precondizioni}
\begin{itemize}
  \item L’Utente è stato autenticato con successo;
  \item È presente almeno un dizionario dati nel sistema.  
\end{itemize}

\paragraph*{Postcondizioni}
\begin{itemize}
  \item L’Utente visualizza la lista completa dei dizionari dati presenti nel sistema.
\end{itemize}

\paragraph*{Scenario principale}
\begin{enumerate}
  \item L’Utente seleziona l’opzione di visualizzazione dei dizionari dati caricati;
  \item Viene esposta la lista dei dizionari dati.  
\end{enumerate}

\paragraph*{Estensione}
\begin{itemize}
  \item Visualizzazione del messaggio errore riguardante il fallimento dell’operazione di caricamento dei dizionari dati presenti (\hyperref[UC10]{UC10}).
\end{itemize}

\paragraph*{Sottocasi d'uso}
\begin{itemize}
  \item Visualizzazione singolo dizionario dati (\hyperref[UC9point1]{UC9.1}).
\end{itemize}

\subsubsection{UC9.1 - Visualizzazione singolo dizionario dati}\label{UC9point1}
TODO immagine\paragraph*{Descrizione}
L’Utente visualizza il dizionario dati identificato dalle sue informazioni principali, nome e descrizione.
\paragraph*{Attori principali} Utente
\paragraph*{Precondizioni}
\begin{itemize}
  \item L’Utente è stato autenticato con successo;
  \item Esiste almeno un dizionario dati caricato.  
\end{itemize}
\paragraph*{Postcondizioni}
\begin{itemize}
  \item Viene mostrato a schermo nome e descrizione del dizionario dati.
\end{itemize}
\paragraph*{Scenario principale}
\begin{enumerate}
  \item L’Utente naviga alla lista dei dizionari dati;
  \item L’Utente visualizza il singolo dizionario dati.
\end{enumerate}
\paragraph*{Sottocasi d'uso}
\begin{itemize}
  \item Visualizzazione nome dizionario dati (\hyperref[UC9point1point1]{UC9.1.1});
  \item Visualizzazione descrizione dizionario dati (\hyperref[UC9point1point2]{UC9.1.2}).
\end{itemize}

\subsubsection{UC9.1.1 - Visualizzazione nome dizionario dati}\label{UC9point1point1}
TODO immagine 9.1.1 da creare
\paragraph*{Descrizione}
L’Utente visualizza il nome del dizionario dati.

\paragraph*{Attori principali} Utente
\paragraph*{Precondizioni}
\begin{itemize}
  \item L’Utente è stato autenticato con successo;
  \item Esiste almeno un dizionario dati caricato.
\end{itemize}
\paragraph*{Postcondizioni}
\begin{itemize}
  \item Viene mostrato a schermo il nome del dizionario dati.
\end{itemize}
\paragraph*{Scenario principale}
\begin{enumerate}
  \item L’Utente naviga alla visualizzazione di un dizionario dati;
  \item Viene visualizzato il nome del dizionario.  
\end{enumerate}


\subsubsection{UC9.1.2 - Visualizzazione descrizione dizionario dati}\label{UC9point1point2}
TODO immagine\paragraph*{Descrizione}
L’Utente visualizza la descrizione del dizionario dati.
\paragraph*{Attori principali} Utente
\paragraph*{Precondizioni}
\begin{itemize}
  \item L’Utente è stato autenticato con successo;
  \item Esiste almeno un dizionario dati caricato.
\end{itemize}
\paragraph*{Postcondizioni}
\begin{itemize}
  \item Viene mostrata a schermo la descrizione del dizionario dati.
\end{itemize}
\paragraph*{Scenario principale}
\begin{enumerate}
  \item L’Utente naviga alla visualizzazione di un dizionario dati;
  \item Viene visualizzato il nome del dizionario.
\end{enumerate}

\subsubsection{UC10 - Messaggio errore fallimento caricamento dizionario dati}\label{UC10}

\paragraph*{Descrizione} L’Utente visualizza il messaggio di errore che informa del fallimento del caricamento dei dizionari dati.

\paragraph*{Attori principali} Utente

\paragraph*{Precondizioni}
\begin{itemize}
  \item L’Utente è stato autenticato con successo;
  \item Esiste almeno un dizionario dati caricato.  
\end{itemize}

\paragraph*{Postcondizioni}
\begin{itemize}
  \item Viene mostrato il messaggio di errore.
\end{itemize}

\paragraph*{Scenario principale}
\begin{enumerate}
  \item L’Utente accede alla modalità di visualizzazione della lista dei dizionari dati;
  \item Il sistema non è in grado di elaborare la richiesta;
  \item Viene presentato il messaggio d’errore corrispondente al tipo del problema.  
\end{enumerate}


\subsubsection{UC11 - Visualizzazione frase SQL}\label{UC11}
%TODO immagine\paragraph*{Descrizione} Un Utente può disconnettersi dalla piattaforma tramite la procedura di logout.

\paragraph*{Attori principali} Utente

\paragraph*{Precondizioni}
\begin{itemize}
  \item L’Utente è correttamente autenticato.
\end{itemize}

\paragraph*{Postcondizioni}
\begin{itemize}
  \item L’Utente ha eseguito il logout e non ha più accesso alle funzionalità base dell’applicativo.
\end{itemize}

\paragraph*{Scenario principale}
\begin{enumerate}
  \item L’Utente desidera terminare la sessione corrente e clicca sul tasto di logout;
  \item La sessione dell’Utente viene terminata.  
\end{enumerate}


\subsubsection{UC13 - Caricamento dizionario dati}\label{UC13}
TODO immagine\paragraph*{Descrizione} Il caricamento del dizionario dati corrisponde alla procedura di importazione di un dizionario dati aggiuntivo all’interno del sistema.

\paragraph*{Attori principali} Tecnico

\paragraph*{Precondizioni}
\begin{itemize}
  \item Il Tecnico è correttamente autenticato tramite login (\hyperref[UC1]{UC1});
  \item Il Tecnico ha selezionato l’opzione di caricamento di un nuovo dizionario dati.  
\end{itemize}

\paragraph*{Postcondizioni}
\begin{itemize}
  \item Il Tecnico ha inserito un nuovo dizionario dati nel sistema.
\end{itemize}

\paragraph*{Scenario principale}
\begin{enumerate}
  \item Il Tecnico ha selezionato il pulsante per avviare la procedura di inserimento di un nuovo dizionario dati;
  \item Il sistema permette al Tecnico di inserire un nuovo file dal formato appropriato;
  \item Il Tecnico inserisce il file relativo al nuovo dizionario dati;
  \item Il dizionario dati viene aggiunto alla lista di quelli presenti nell’applicativo.  
\end{enumerate}

\paragraph*{Sottocasi d'uso}
\begin{itemize}
  \item Visualizzazione errore per mancato caricamento del dizionario (\hyperref[UC13point1]{UC13.1}).
\end{itemize}


\subsubsection{UC13.1 - Visualizzazione errore per mancato caricamento del dizionario}\label{UC13point1}
\paragraph*{Descrizione} Il sistema mostra un messaggio d’errore relativo al mancato caricamento del nuovo dizionario dati che si desidera inserire nel momento in cui vengono rilevate anomalie relative al suo inserimento.
\paragraph*{Attori principali} Tecnico
\paragraph*{Precondizioni}
\begin{itemize}
  \item Il Tecnico è correttamente autenticato tramite login (\hyperref[UC1]{UC1});
  \item Il Tecnico ha selezionato l’opzione di caricamento di un nuovo dizionario dati;
  \item Il Tecnico ha inserito il file relativo al nuovo dizionario dati che desidera inserire.
  
\end{itemize}
\paragraph*{Postcondizioni}
\begin{itemize}
  \item Il sistema mostra un messaggio d’errore per il mancato inserimento del nuovo dizionario dati.
\end{itemize}
\paragraph*{Scenario principale}
\begin{enumerate}
  \item Il Tecnico ha avviato la procedura di caricamento di un nuovo dizionario dati;
  \item Il sistema permette al Tecnico di inserire un nuovo file;
  \item Il sistema registra un errore nel caricamento del file;
  \item Il sistema segnala il mancato caricamento al Tecnico tramite un messaggio di errore.  
\end{enumerate}


\subsubsection{UC14 - Visualizzazione dati dizionario}\label{UC14}
TODO immagine
\paragraph*{Descrizione} Viene mostrato il dizionario dati caricato dal Tecnico, comprendendo tutti i suoi dati in maniera esaustiva, con nome, descrizione, struttura del database, sinonimi, parte di test.

\paragraph*{Attori principali} Tecnico

\paragraph*{Precondizioni}
\begin{itemize}
  \item Il Tecnico è correttamente autenticato tramite login (\hyperref[UC1]{UC1}).
\end{itemize}

\paragraph*{Postcondizioni}
\begin{itemize}
  \item Vengono visualizzati i dati del dizionario dati scelto.
\end{itemize}

\paragraph*{Scenario principale}
\begin{enumerate}
  \item Il Tecnico naviga alla lista dei dizionari dati;
  \item Il Tecnico seleziona il pulsante per visualizzare i dati di un dizionario;
  \item Il Tecnico visualizza i dati del dizionario.
\end{enumerate}

\paragraph*{Estensione}

\paragraph*{Sottocasi d'uso}
\begin{itemize}
  \item Visualizzazione nome dizionario dati (\hyperref[UC9point1point1]{UC9.1.1});
  \item Visualizzazione descrizione dizionario dati (\hyperref[UC9point1point2]{UC9.1.2});
  \item Visualizzazione tabelle del dizionario (\hyperref[UC14point1]{UC14.1});
  \item Visualizzazione descrizione e sinonimi delle tabelle (\hyperref[UC14point2]{UC14.2});
  \item Visualizzazione sinonimi delle colonne (\hyperref[UC14point3]{UC14.3});
  \item Visualizzazione test del dizionario TODO .  
\end{itemize}

\subsubsection{UC15 - Verifica correttezza dizionario}\label{UC15}
TODO


\subsubsection{UC16 - Modifica dizionario dati}\label{UC16}
TODO immagine
\paragraph*{Descrizione} Il Tecnico vuole modificare un dizionario dato già presente nel sistema.

\paragraph*{Attori principali} Tecnico

\paragraph*{Precondizioni}
\begin{itemize}
  \item Il Tecnico è correttamente autenticato tramite login (\hyperref[UC1]{UC1});
  \item Il Tecnico naviga alla lista dei dizionari dati caricati;
  \item Il Tecnico seleziona un dizionario dati di cui vuole modificare qualcosa.  
\end{itemize}

\paragraph*{Postcondizioni}
\begin{itemize}
  \item Il Tecnico ha modificato una o più cose tra: titolo , descrizione o contenuto del dizionario dati.
\end{itemize}

\paragraph*{Scenario principale}
\begin{enumerate}
  \item Il Tecnico sceglie un dizionario dati che vuole modificare e lo seleziona;
  \item Il Tecnico sceglie quali campi modificare tra : titolo, descrizione o file che contiene il dizionario dati;
  \item Il Tecnico seleziona quello che vuole modificare e inserisce una nuova stringa per titolo e descrizione oppure carica un nuovo file per il dizionario dati;
  \item Il Tecnico salva le modifiche.  
\end{enumerate}

\paragraph*{Sottocasi d'uso}
\begin{itemize}
  \item Modifica titolo dizionario dati (\hyperref[UC16point1]{UC16.1});
  \item Modifica descrizione dizionario dati (\hyperref[UC16point2]{UC16.2});
  \item TODO
\end{itemize}


\subsubsection{UC16.1 - Modifica titolo dizionario dati}\label{UC16point1}
\paragraph*{Descrizione} Il Tecnico vuole modificare il titolo di un dizionario dati.
\paragraph*{Attori principali} Tecnico
\paragraph*{Precondizioni}
\begin{itemize}
  \item Il Tecnico è correttamente autenticato tramite login (\hyperref[UC1]{UC1});
  \item Il Tecnico ha selezionato un dizionario dati esistente di cui vuole modificare il titolo.  
\end{itemize}
\paragraph*{Postcondizioni}
\begin{itemize}
  \item Il dizionario dati ha come titolo il nuovo valore inserito dal Tecnico.
\end{itemize}
\paragraph*{Scenario principale}
\begin{enumerate}
  \item Il Tecnico ha selezionato il dizionario dati di cui vuole cambiare il titolo e inserisce il nuovo valore nell’apposito box;
  \item Il Tecnico salva la modifica.  
\end{enumerate}


\subsubsection{UC16.2 - Modifica descrizione dizionario dati}\label{UC16point2}
\paragraph*{Descrizione} Il Tecnico vuole modificare la descrizione di un dizionario dati.
\paragraph*{Attori principali} Tecnico
\paragraph*{Precondizioni}
\begin{itemize}
  \item Il Tecnico è correttamente autenticato tramite login (\hyperref[UC1]{UC1});
  \item Il Tecnico ha selezionato un dizionario dati esistente di cui vuole modificare il la descrizione.  
\end{itemize}
\paragraph*{Postcondizioni}
\begin{itemize}
  \item Il dizionario dati ha come descrizione il nuovo valore inserito dal Tecnico.
\end{itemize}
\paragraph*{Scenario principale}
\begin{enumerate}
  \item Il Tecnico ha selezionato il dizionario dati di cui vuole cambiare la descrizione e inserisce il nuovo valore nell’apposito box;
  \item Il Tecnico salva la modifica.
\end{enumerate}

TODO Descrizioni vuote ammissibili o no?


\subsubsection{UC17 - Elimina dizionario dati}\label{UC17}
TODO immagine
\paragraph*{Descrizione} Il Tecnico desidera eliminare uno dei dizionari dati inseriti in precedenza dalla lista.

\paragraph*{Attori principali} Tecnico

\paragraph*{Precondizioni}
\begin{itemize}
  \item Il Tecnico è correttamente autenticato tramite login (\hyperref[UC1]{UC1});
  \item Il Tecnico naviga alla lista dei dizionari dati caricati.  
\end{itemize}

\paragraph*{Postcondizioni}
\begin{itemize}
  \item Il dizionario selezionato è stato eliminato.
\end{itemize}

\paragraph*{Scenario principale}
\begin{enumerate}
  \item Il Tecnico ha selezionato il dizionario da eliminare;
  \item Il Tecnico conferma la scelta di eliminazione;
  \item Il dizionario dati selezionato viene rimosso dalla lista dei dizionari dati;  
\end{enumerate}

\paragraph*{Sottocasi d'uso}
\begin{itemize}
  \item Visualizzazione errore per mancata eliminazione del dizionario dati (\hyperref[UC17point1]{UC17.1}).
\end{itemize}


\subsubsection{UC17.1 - Visualizzazione errore per mancata eliminazione del dizionario dati}\label{UC17point1}
\paragraph*{Descrizione} Il sistema mostra un messaggio d’errore relativo alla mancata eliminazione del dizionario dati selezionato dal Tecnico.

\paragraph*{Attori principali} Tecnico
\paragraph*{Precondizioni}
\begin{itemize}
  \item Il Tecnico è correttamente autenticato tramite login (\hyperref[UC1]{UC1});
  \item Il sistema ha riscontrato un problema nell'eliminazione di un dizionario dati selezionato dal Tecnico.
\end{itemize}
\paragraph*{Postcondizioni}
\begin{itemize}
  \item Il sistema mostra un messaggio d’errore per la mancata eliminazione del nuovo dizionario dati.
\end{itemize}
\paragraph*{Scenario principale}
\begin{enumerate}
  \item Il Tecnico ha selezionato il dizionario da eliminare;
  \item Il Tecnico conferma la scelta di eliminazione;
  \item Il sistema segnala la mancata eliminazione al Tecnico tramite un messaggio di errore.
\end{enumerate}


\subsubsection{UC18 - Debug della generazione del prompt}\label{UC18}
TODO immagine
TODO Approfondire, espandere, dettagliare
\paragraph*{Descrizione} Il Tecnico vuole poter testare il dizionario dati e interroga il sistema per comprendere la corretta formattazione del dizionario dati, riceverà in risposta uno schema motivato della selezione dei campi del dizionario per permettere eventuali modifiche volte ad ottimizzare le interrogazioni.

\paragraph*{Attori principali} Tecnico

\paragraph*{Precondizioni}
\begin{itemize}
  \item Il Tecnico è correttamente autenticato tramite login (\hyperref[UC1]{UC1});
  \item È presente almeno un dizionario dati nel sistema;
  \item Il Tecnico seleziona l’interfaccia di debug.
\end{itemize}

\paragraph*{Postcondizioni}
\begin{itemize}
  \item Il sistema genera uno schema facilmente comprensibile e motivato della selezione dei singoli campi delle tabelle per dare al Tecnico una visione dettagliata dei criteri di scelta di generazione del prompt.
\end{itemize}

\paragraph*{Scenario principale}
\begin{enumerate}
  \item Il Tecnico desidera testare la correttezza di un dizionario; 
  \item Il Tecnico seleziona un dizionario dati sul quale baserà l’interrogazione;
  \item Il Tecnico sceglie la lingua su cui farà l’interrogazione. La lingua è in italiano di default se non viene cambiata;
  \item Il Tecnico inserisce un’interrogazione in linguaggio naturale nel box testuale apposito;
  \item Il sistema fornisce lo schema di debug relativo al sotto-modello selezionato per l’utilizzo nel prompt.  
\end{enumerate}

\paragraph*{Inclusione}
\begin{itemize}
  \item Inserimento richiesta in linguaggio naturale (\hyperref[UC3]{UC3});
  \item Selezione dizionario dati (\hyperref[UC4]{UC4}).
\end{itemize}

\paragraph*{Estensione}
\begin{itemize}
  \item Cambio lingua (\hyperref[UC7]{UC7});
  \item Errore generazione prompt (\hyperref[UC6]{UC6}).
\end{itemize}
