\noindent\begin{minipage}{\textwidth}
\subsubsection{RT1: Scarso know-how tecnologico}

\bgroup
\begin{adjustwidth}{-0.5cm}{-0.5cm}
	% MAX 12.5cm
 	\begin{longtable}{P{4.5cm}|>{\justifying \arraybackslash}P{9cm}}

		\textbf{Probabilità} & Media \\
        \hline
        \textbf{Grado di criticità} & Alto \\
        \hline
        \textbf{Descrizione} & Utilizzo di tecnologie sconosciute. Nessun membro del gruppo ha
        esperienze pregresse con gli strumenti e le librerie suggerite dalla Proponente; di conseguenza, l’avanzamento del progetto rischia di subire rallentamenti dovuti a fasi di apprendimento delle nuove tecnologie.
        Con questo si intendono anche fasi di esplorazione di tecnologie che possono risultare più vantaggiose rispetto a quelle utilizzate al momento. \\
        \hline
        \textbf{Strategie di rilevamento} &  Analisi, individuale o collaborativa, per valutare la
        curva di apprendimento. Rilevamento di tecnologie diverse che possano ottimizzare lo sviluppo \\
        \hline
        \textbf{Contromisure} & Studio preventivo o, qualora fosse necessario, sospensione del
        lavoro per dedicarsi all’approfondimento di una determinata tecnologia, magari attraverso sessioni collaborative volte ad allineare più rapidamente le conoscenze del gruppo.  
	\end{longtable}
\end{adjustwidth}
\egroup
\end{minipage}