\section{Riunione}
\subsection{Ordine del Giorno}
\begin{itemize}
	\item Visualizzare lo stato di avanzamento attuale per i documenti in lavorazione
\end{itemize}

\subsection{Discussione e decisioni}

\subsubsection{Responsabile}
Viene presentato lo spreadsheet condiviso che comprende le informazioni inerenti a spesa di budget e ore. Ogni sprint ha un proprio foglio che contiene le seguenti informazioni:
\begin{itemize}
	\item Numero dello sprint;
	\item Date di inizio e termine dello sprint;
	\item Tabella di assegnazione ore produttive dello sprint per ciascuna persona, accumulate in totali per ciascuna persona e per ciascun ruolo;
	\item Distribuzione ore per ruolo, sotto forma di donut chart;
	\item Distribuzione ore per la coppia Risorsa-Ruolo, sotto forma sia di tabella che di grafico a barre;
	\item Preventivo economico, tabella riassuntiva con ore e budget spesi nello sprint e restanti;
	\item Pie chart con la stima delle ore spese sul totale;
	\item Pie chart con la stima del budget sul totale;
	\item Ore rimanenti per la coppia Risorsa-Ruolo.
\end{itemize}
\vspace{0.5\baselineskip}

Viene suggerita la possibilità negli sprint successivi al primo di aggiungere alla tabella di preventivo una riga considerante la spesa ore/budget degli sprint pregressi, da includere poi alle pie, così da avere anche un rapporto sul lavoro passato, oltre che sul totale. Viene inoltre discussa dal responsabile l'aggiunta di una sezione di rendicontazione ore, per avere un confronto tra preventivo e consuntivo dello sprint.\\

\subsubsection{Amministratore}
Riguardo al Glossario, viene discusso il caricamento di una struttura iniziale su Git, anche vuota, in modo da poter accumulare i termini utilizzati. Viene anche considerato in alternativa un documento condiviso che permetta di lavorare parallelamente.\\
\\
Discutendo il documento delle Norme di Progetto, viene analogamente evidenziata la necessità di un documento condiviso per modifiche e aggiunte parallele.

\subsubsection{Analisti e Progettista}
Dopo aver esaminato attentamente i casi d'uso, le priorità sono redarre il documento di Analisi dei Requisiti, specialmente nei capitoli legati alla descrizione del prodotto e dei casi d'uso, inoltre è indicato come da rivedere l'ordinamento generale, che può risultare ridondante.

\subsubsection{Discussioni finali}
Conclusi i controlli complessivi, viene discussa la necessità di studiare le tecnologie da utilizzare, suggerita dalla Proponente o individuati come risposta delle necessità nate nell'Analisi dei Requisiti.\\
Si è discussa la possibilità di ricercare già un esempio di struttura di base di dati su cui ci si possa basare per i primi periodi, sono state proposte come soluzioni sia la ricerca online di qualche struttura, sia l'uso di alcuni progetti del corso di Basi di Dati.\\
Infine viene raccomandato di avanzare le pull request incrementalmente in modo da favorire il lavoro di verifica e ridurre la gravità di errori di inesperienza con Git o LaTeX.