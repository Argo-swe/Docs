\section{Riunione}
\subsection{Ordine del Giorno}
\begin{itemize}
	\item Esaminare lo stato di avanzamento dei documenti in lavorazione.
\end{itemize}

\subsection{Discussione e decisioni}

\subsubsection{Responsabile}
Viene presentato lo spreadsheet condiviso che racchiude le informazioni inerenti al preventivo e al consuntivo (orario ed economico). Ogni sprint ha un proprio foglio che contiene le seguenti informazioni:
\begin{itemize}
	\item Numero dello sprint;
	\item Date di inizio e termine dello sprint;
	\item Tabella di assegnazione delle ore produttive per ciascun membro del team, accumulate in totali per persona e per ruolo;
	\item Distribuzione delle ore per ruolo, sotto forma di donut chart;
	\item Distribuzione delle ore per la coppia Risorsa-Ruolo, sotto forma sia di tabella che di grafico a barre;
	\item Preventivo economico dello sprint;
	\item Tabella riassuntiva con ore e budget spesi e restanti;
	\item Pie chart con la stima delle ore spese sul totale;
	\item Pie chart con la stima del budget sul totale;
	\item Ore rimanenti per la coppia Risorsa-Ruolo.
\end{itemize}

\vspace{0.5\baselineskip}
Viene suggerita la possibilità, negli sprint successivi al primo, di aggiungere alla tabella di preventivo una riga considerante la spesa ore/budget degli sprint pregressi, da includere poi nel pie chart, così da avere anche un rapporto sul lavoro passato, oltre che sul totale. Viene inoltre discussa dal responsabile l'aggiunta di una sezione di rendicontazione ore, per avere un confronto tra preventivo e consuntivo dello sprint.

\subsubsection{Amministratore}
Riguardo al Glossario, viene valutato positivamente il caricamento di una struttura iniziale su GitHub, anche vuota, così da poter accumulare i termini utilizzati. In aggiunta, viene considerata la creazione di un documento condiviso che permetta di raccogliere le parole chiave.\\
\\
Discutendo il documento delle Norme di Progetto, viene analogamente evidenziata la necessità di un file condiviso in cui appuntare le linee guida e le procedure da formalizzare.

\subsubsection{Analisti e Progettista}
Dopo aver esaminato attentamente i casi d'uso, la priorità è redarre il documento di Analisi dei Requisiti, specialmente i capitoli legati alla descrizione del prodotto e dei casi d'uso. Dal resoconto degli analisti è emersa inoltre la necessità di ridefinire l'ordinamento generale del documento, che può risultare ridondante.

\subsubsection{Discussioni finali}
Terminati i controlli complessivi, viene pianificato lo studio delle tecnologie di sviluppo, suggerite dalla Proponente o individuate come risposta alle necessità nate nell'Analisi dei Requisiti.\\
Si è discussa inoltre la possibilità di definire una struttura di base di dati da usare come prototipo per i primi periodi; sono state proposte come soluzioni sia la ricerca online di qualche struttura, sia l'uso di alcuni progetti del corso di Basi di Dati.\\
Infine viene raccomandato di avanzare le pull request incrementalmente in modo da favorire il lavoro di verifica e ridurre la gravità di errori di inesperienza con Git o LaTeX.