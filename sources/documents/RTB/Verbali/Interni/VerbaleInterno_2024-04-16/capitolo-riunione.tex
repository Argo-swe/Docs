\section{Riunione}
\subsection{Ordine del Giorno}
\begin{itemize}
	\item Discussione sull'aggiunta di comandi in LaTeX per riferirsi ad altri documenti senza includere il numero di versione;
  \item Valutazione dell'andamento dello sprint corrente e definizione di eventuali azioni correttive;
  \item Discussione sull'impiego di un modello di sviluppo differente.
\end{itemize}

\subsection{Discussione e decisioni}

\subsubsection{Nuovi pacchetti e comandi LaTeX}
\par Il responsabile ha proposto all'amministratore di aggiungere nuovi comandi in LaTeX per menzionare altri documenti senza includere il numero di versione. Questa scelta è motivata dal fatto che durante la pianificazione del primo periodo, è stata fissata la stesura iniziale dei documenti di progetto. Tuttavia, compilando il Piano di Progetto a ridosso della RTB, il numero di versione verrebbe modificato ovunque in 1.0.0, invalidando la struttura della pianificazione. In altre sezioni, invece, come nel segmento relativo ai riferimenti normativi, è corretto affiancare il numero di versione ai documenti. Dunque, è preferibile avere a disposizione due comandi, in modo da poter citare un documento con e senza numero di versione.
\par Inoltre, si è deciso di aggiungere il pacchetto fontawesome5, che fornisce il supporto LaTeX per il set di icone "Font Awesome 5 Free". Un miglioramento grafico pensato dal team riguarda la visualizzazione di un'etichetta prima del numero di versione.

\subsubsection{Tracciamento delle attività}
\par Dopo aver concordato gli aggiornamenti al template LaTeX, il gruppo ha analizzato l'andamento del primo sprint, valutando positivamente la partecipazione delle singole risorse e la suddivisione in team di lavoro più piccoli. In seguito, ciascun componente ha esposto le difficoltà incontrate durante il periodo, che andranno registrate nel diario di bordo e gestite in vista del prossimo sprint. Per migliorare la gestione di progetto, il team di analisti ha proposto l’utilizzo di una To-Do List condivisa in forma testuale.
\par Come stabilito nel Way of Working, la creazione e assegnazione dei task deriva dalla pianificazione dello sprint, dalle riunioni interne e dai meeting tra risorse che ricoprono lo stesso ruolo. Di conseguenza, si rende indispensabile avere a disposizione un file di appunti condiviso, in aggiunta ai verbali (interni ed esterni). A partire da questi documenti, il responsabile può agilmente creare gli issue su GitHub.

\subsubsection{Strumenti da affiancare a GitHub}
\par Nel corso del primo sprint, il team si è accorto che GitHub non è il servizio ottimale per monitorare le attività di natura organizzativa e logistica. Inoltre, a meno di integrazioni con strumenti di terzi parti, GitHub non offre la possibilità di visualizzare i task su un calendario. Perciò si è deciso di valutare possibili alternative all'Issue Tracking System di GitHub, che verranno discusse durante la prossima riunione.

\subsubsection{Modello di sviluppo}
\par In seguito alla riunione tra cliente e fornitore tenutasi il 9 aprile, il responsabile ha analizzato il modello di sviluppo scelto dal team. Come riportato nel \emph{Verbale Esterno del 9 aprile 2024 v1.0.0}, la Proponente ha consigliato al gruppo di non impiegare una metodologia troppo stringente, specialmente per quanto riguarda l’organizzazione delle riunioni. Il responsabile ha quindi esposto al team i risultati (raccolti nell’arco di una settimana) delle sue analisi, elencando i pro e contro dei modelli individuati. Dopo una valutazione collettiva, la scelta finale è stata quella di confermare il modello di sviluppo agile, proseguendo così con la linea intrapresa nel primo sprint, a cui andranno applicate le dovute rettifiche in fase di retrospettiva.


