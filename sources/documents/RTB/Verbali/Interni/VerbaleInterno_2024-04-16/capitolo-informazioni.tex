\section{Informazioni}
\begin{itemize}
	\item \textbf{Inizio incontro:} 18:00
	\item \textbf{Fine incontro:} 18:45
	\item \textbf{Pianificazione incontro:} Telegram
	\item \textbf{Tipo incontro:} remoto (Discord)
\end{itemize}

\subsection{Descrizione}
\DocDescription

\subsection{Partecipanti}

\begin{itemize}
	\item \textbf{\GroupName:}
	\begin{itemize}
		\item \tommaso \ \rightarrow\ 45 minuti
		\item \marco \ \rightarrow\ 45 minuti
		\item \sebastiano \ \rightarrow\ 45 minuti
		\item \martina \ \rightarrow\ 45 minuti
		\item \riccardo \ \rightarrow\ 45 minuti
		\item \mattia \ \rightarrow\ 30 minuti
	\end{itemize}
\end{itemize}

\subsection{Glossario}
Allo scopo di evitare incomprensioni relative al linguaggio utilizzato nella documentazione di progetto, viene fornito un \Gls, nel quale ciascun termine è corredato da una spiegazione che mira a disambiguare il suo significato. I vocaboli ritenuti ambigui vengono formattati in corsivo all'interno dei rispettivi documenti e identificati con una lettera \ped{G} in pedice. Tutte le ricorrenze di un termine definito nel \Gls\ subiscono la formattazione sopracitata.

\clearpage