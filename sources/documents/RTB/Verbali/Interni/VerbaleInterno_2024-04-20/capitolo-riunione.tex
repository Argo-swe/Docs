\section{Riunione}
\subsection{Ordine del giorno}
\begin{itemize}
	\item Raccolta dei risultati del questionario e valutazione dello \glossario{sprint} appena concluso;
  \item Proposte di miglioramento per il prossimo \glossario{sprint} e definizione di un piano d'azione;
  \item Pianificazione dell'iterazione successiva;
  \item Resoconto dell'\AdR;
  \item Discussione sul \Gls;
  \item Analisi degli \glossario{ITS} individuati come alternative a \glossario{GitHub}.
\end{itemize}

\subsection{Discussione e decisioni}

\subsubsection{Valutazione del primo sprint}
\par Di seguito sono riportati i risultati del questionario di valutazione dello \glossario{sprint}, realizzato dal responsabile per ottimizzare la fase di \glossario{retrospettiva}:
\begin{itemize}
  \item Organizzazione dello \glossario{sprint} - Valutazione: 7,5;
  \item Conduzione dei meeting interni - Valutazione: 7,5;
  \item Conduzione dei meeting esterni - Valutazione: 8;
  \item Impegno e partecipazione dei singoli membri - Valutazione: 7;
  \item Non tutti i membri del team erano a conoscenza delle proprie mansioni;
  \item La numerosità delle riunioni è adeguata;
  \item Le riunioni sono state organizzate quasi sempre con il giusto preavviso;
  \item Da migliorare il rapporto ore spese/ore produttive.
\end{itemize}

\vspace{0.5\baselineskip}
\par Le valutazioni raccolte hanno evidenziato una pianificazione idonea e un carico di lavoro non eccessivo. Tuttavia, il passaggio da un'attività precedente a quella successiva non è sempre stato immediato, e questo ha portato a delle fasi, seppur brevi, di stallo. Il team ha riscontrato un feedback positivo per quanto riguarda l'organizzazione e la conduzione delle riunioni. Al contrario, sono stati rilevati come aspetti da migliorare la coesione interna e il rendimento delle singole risorse. Si è ritenuto inoltre necessario proseguire sulla strada dei micro-gruppi, suddividendo il lavoro in team più piccoli capaci di auto-coordinarsi e lavorare in sincronia.

\subsubsection{Piano d'azione per il prossimo sprint}
\par Dato che la configurazione di \glossario{Docker}, la stesura dei documenti in \glossario{LaTeX} e la fase di \glossario{verifica} su \glossario{GitHub} hanno rappresentato degli ostacoli durante il primo \glossario{sprint}, il team ha deciso che nella prossima riunione utile (ovvero quando saranno presenti tutti i membri del gruppo) verrà dedicato uno spazio per condividere le conoscenze tra le risorse.
\par In seguito, il team ha definito un piano d'azione per migliorare l'organizzazione e la produttività del prossimo \glossario{sprint}:
\begin{itemize}
  \item Definire una To-Do List più precisa;
  \item Organizzare riunioni brevi e mirate;
  \item Interazione più frequente tra il responsabile e il team di sviluppo;
  \item Pianificare lo studio delle tecnologie in maniera graduata;
  \item Verificare costantemente il progresso delle attività e la documentazione di progetto;
  \item Programmare un incontro in presenza con cadenza mensile;
  \item Sinergia tra ruoli con funzioni complementari;
  \item Sfruttare l'esperienza acquisita da chi ricopriva un ruolo in precedenza;
  \item Possibilità di assumere più ruoli durante uno \glossario{sprint}.
\end{itemize}

\vspace{0.5\baselineskip}
\par Queste analisi a posteriori saranno riportate nel consuntivo del primo \glossario{sprint} all'interno del \PdP, più precisamente nella sezione dedicata alla fase di \glossario{retrospettiva}.

\subsubsection{Pianificazione secondo sprint}
\par La seconda iterazione si svolgerà dal 22 aprile al 6 maggio 2024. Fino al 6 maggio, il
responsabile del gruppo è \raul. Per il secondo \glossario{sprint} sono stati individuati i seguenti ruoli:
\begin{itemize}
	\item 1 amministratore: \riccardo;
	\item 1 responsabile: \raul;
	\item 1 analista: \tommaso;
	\item 1 progettista: \sebastiano;
  \item 1 programmatore: \marco;
	\item 2 verificatori: \mattia\ e \martina.
\end{itemize}

\vspace{0.5\baselineskip}
\par Le ore sono state distribuite equamente tra i membri del team, tenendo in considerazione i costi, gli \glossario{sprint} pregressi e le attività da svolgere. L'amministratore si occuperà principalmente di aggiornare le \NdP\ e predisporre il \PdQ, mentre l'analista convertirà in \glossario{LaTeX} l'\AdR. È stata introdotta la figura del programmatore, che studierà le tecnologie di sviluppo e ne valuterà l'efficacia, interfacciandosi con il progettista per definire la struttura del \glossario{dizionario dati} e del \glossario{prompt}. La pianificazione dello \glossario{sprint} sarà opportunamente documentata nel \PdP. 
\par Per quanto riguarda il \PdP, il responsabile ha esposto le difficoltà incontrate durante la stesura iniziale del consuntivo. L'incertezza riguardava essenzialmente la ridistribuzione di eventuali risorse, in eccesso o in difetto. La visione condivisa dal team era la seguente: se, alla fine di uno \glossario{sprint}, il quantitativo di ore effettivamente svolte fosse inferiore rispetto a quanto preventivato, tale situazione dovrebbe essere riportata nell'analisi a posteriori e, contestualmente, tradursi in un aggiornamento migliorativo della pianificazione. Meno chiaro, invece, era il modo in cui aggiornare a livello concreto la pianificazione futura e il preventivo “a finire”. Il responsabile ha proposto di inviare una mail al Prof. Tullio Vardanega per chiarire i dubbi emersi durante la discussione.

\subsubsection{Resoconto dell'Analisi dei Requisiti}
\par Il team di analisti ha condiviso un resoconto del lavoro svolto nel primo \glossario{sprint}, con un focus sugli \glossario{attori} che interagiscono con il sistema. Sono state mostrate due possibili alternative:
\begin{itemize}
	\item Divisione tra utente non autenticato e tecnico. Un utente non autenticato può utilizzare l'applicativo anche senza registrarsi; una volta effettuato l'accesso con le credenziali dell'admin, l'utente non autenticato diventa un tecnico. Il tecnico può svolgere tutte le attività di un utente non autenticato e, in aggiunta, caricare ed eliminare uno o più \glossario{dizionario dati};
  \item Divisione tra utente generico e tecnico. Sia il tecnico che l'utente generico devono registrarsi per poter utilizzare l'applicativo. La differenza rispetto all'opzione precedente è la necessità di implementare un sistema di registrazione e di gestione delle credenziali. Nonostante questa soluzione sembri allontanarsi dai \glossario{requisiti} del \glossario{capitolato}, è in realtà il frutto di una considerazione della \glossario{Proponente}. Durante la riunione del 9 aprile 2024, infatti, si era discussa l'eventualità che l'applicativo potesse subire dei cali di prestazione dovuti a richieste multiple generate da bot di servizi esterni.
\end{itemize}

\vspace{0.5\baselineskip}
\par Si è deciso di programmare un incontro con il Prof. Riccardo Cardin per chiarire i dubbi sull'\AdR\ e verificare la correttezza dei diagrammi dei \glossario{casi d'uso}. Inoltre, il team ha approfondito la funzionalità di \glossario{debug} proposta dal cliente, stabilendo di restituire all'admin un \glossario{report} testuale sulle modalità di generazione del \glossario{prompt}.

\subsubsection{Discussione sul Glossario}
\par I termini da inserire nel \Gls\ sono stati individuati durante il primo \glossario{sprint}, ma non sono stati inseriti all'interno dei verbali con una formattazione ad hoc. Di conseguenza, il team ha fissato come obiettivo la revisione dei verbali (interni ed esterni) per aggiungere i riferimenti a termini presenti nel \Gls. Tuttavia, è emerso un dubbio sul numero di occorrenze da formattare. In attesa del prossimo diario di bordo, il team ha deciso di formattare solamente la prima ricorrenza.

\newpage
\subsubsection{Alternative all'ITS di GitHub}
\par Il team ha evidenziato le seguenti alternative all'\glossario{ITS} di \glossario{GitHub}:
\begin{itemize}
  \item \glossario{Trello}, un software per la gestione di progetto in stile \glossario{Kanban};
  \item \glossario{Jira Work Management}, uno strumento della suite di \glossario{Jira} che semplifica e centralizza la pianificazione e la collaborazione per team che non sono necessariamente coinvolti nello sviluppo software;
  \item \glossario{Jira Software}, uno strumento progettato principalmente per team di sviluppo che seguono metodologie agili come \glossario{Scrum} o \glossario{Kanban}.
\end{itemize}

\vspace{0.5\baselineskip}
\par Entro il 22 aprile, \riccardo\ e \mattia\ esploreranno e confronteranno gli strumenti sopra elencati, valutando se interrrompere l'utilizzo dell'\glossario{ITS} di \glossario{GitHub}.