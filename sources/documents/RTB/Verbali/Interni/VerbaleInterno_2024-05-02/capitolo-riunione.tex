\section{Riunione}
\subsection{Ordine del giorno}
\begin{itemize}
	\item Analisi delle attività svolte durante la settimana dai membri del gruppo;
	\item Discussione dei punti che verranno trattati al diario di bordo del 3 maggio 2024;
	\item Invio di una mail per l'organizzazione di un incontro con La Proponente per aggiornarLa sullo stato del progetto.
\end{itemize}

\subsection{Discussione e decisioni}
\par All'inizio dell'incontro, viene chiesto di esporre al gruppo le attività svolte durante la settimana in vista della fine dello \glossario{sprint}.
\subsubsection{Analista}
\par L'Analista ha evidenziato difficoltà nella conversione del documento di \AdR\ e ha stimato a giorni la conclusione della conversione. Tra le altre attività, sono state svolte:
\begin{itemize}
	\item Approfonfimento dei casi d'uso del login per comprenderne la fattibilità, evidenziando la possibilità di avere un utente premium in caso di interazioni con l'\glossario{API};
	\item Trovare mezzi e idee per l'approfondimento dei casi d'uso dello strumento di debug fornito al tecnico;
\end{itemize}
\par Sono state inoltre evidenziate delle difficoltà sull'ordinamento del documento di \AdR\ date dallo strumento \glossario{LateX}. Infine ha esposto la necessità di un incontro con il Professor. Riccardo Cardin per una funzione di login utente che sarebbe disponibile anche per il tecnico già loggato. Il problema sarebbe arginabile definendo delle precondizioni per il login. 
\subsubsection{Amministratore}
\par Vengono discussi i nuovi comandi per la citazione dei documenti in \glossario{LateX}, inseriti nel documento di \NdP. Vengono inoltre elencate le attività e i dubbi quali:
\begin{itemize}
	\item Avanzamento dei lavori sul \PdP\ e inizio dei lavori sul \PdQ;
	\item Elaborazione di tabelle e grafici da inserire per preventivi e consuntivi del \PdP;
	\item Inizio dei lavori sui file di configurazione per la compilazione dei file pdf in au1tomatico.
\end{itemize}
\subsubsection{Responsabile} 
\par Il Responsabile indica le attività svolte in collaborazione con l'Amministratore quali:
\begin{itemize}
	\item Conversione di pianificazione e preventivo del primo \glossario{sprint} e l'elaborazione di tabelle e grafici per preventivo e consuntivo degli \glossario{sprint}; 
	\item Stesura del verbale del meeting precedente 
	\item Definizione attività e grafico \glossario{Gannt}\ su \glossario{Jira}.
\end{itemize} 
\subsubsection{Progettista}
\par Il Progettista espone i lavori fatti in collaborazione con il Programmatore.
\begin{itemize}
	\item Strutturazione di un dizionario dati in formato \glossario{JSON}\ che potesse essere più efficiente nell'interazione con \glossario{ChatGPT};
	\item Espone la presenza di due tipi di formato in linguaggio naturale che più sembrano essere ottimali per gli \glossario{LLM};
	\item Propone di utilizzare il linguaggio \glossario{Python}\ per il \glossario{backend}\ vista la quantità già scritta di codice per la parte funzionale del progetto;
\end{itemize}

\subsubsection{Ridistribuzione dei ruoli}
\par Per il terzo sprint vengono distribuite nuovamente le ore ai vari membri del gruppo. Da questo sprint, le ore non si concentreranno solo su un ruolo, ma spazieranno su più ruoli per fornire sostegno a membri che necessitano di maggiori risorse e per evitare di assegnare solo un ruolo con poche ore rimanenti. In particolare:
\begin{itemize}
	\item Vengono richieste più ore sul ruolo di progettista per dare una migliore definizione del lavoro al Programmatore. Vengono tuttavia assegnate più ore al Programmatore in vista di una ridistribuzione più mirata;
	\item Vengono inoltre richieste più risorse al ruolo di Amministratore vista la mole di lavoro che altrimenti dovrebbe affrontare un solo membro. Vengono quindi assegnate altre ore a 2 membri del gruppo, in modo da offrire assistenza. 
\end{itemize}

\subsubsection{Incontro con La Proponente}
\par Vengono in seguito discussi i punti da trattare con La Proponente in un meeting pianificato per il 6 maggio 2024.
\begin{itemize}
	\item Proposta dell'utilizzo del framework \glossario{Streamlit} per usare \glossario{Python}\ anche per a parte relativa al \glossario{frontend}\ o in alternativa l'utilizzo di altri framework per velocizzare i lavori;
	\item Proposta dell'utilizzo di \glossario{Python}\ anche per il \glossario{backend}\ come da suggerimento del Progettista;
	\item Aggiornamento sullo stato del progetto e sulle attività future.
\end{itemize}

\subsubsection{Punti da trattare nel prossimo diario di bordo}
\par Il gruppo ha discusso i punti salienti da trattare nel diario di bordo del 3 maggio 2024.
\paragraph{Done}
\begin{itemize}
	\item Pianificazione e preventivo \glossario{sprint} 3;
	\item Inizio stesura del \PdQ;
	\item Conversione in \glossario{LateX}\ del documento di \AdR;
	\item Inserimento di grafici nel \PdP;
	\item Progettazione della conversione da linguaggio naturale al prompt finale;
	\item Definizione di un dizionario dati efficiente;
	\item Aggiornamento \NdP\ e popolamento del \Gls;
	\item Pianificazione di un incontro con La Proponente.
\end{itemize}
\paragraph{Difficoltà}
\begin{itemize}
	\item Poche risorse assegnate al ruolo di amministratore
	\item Difficoltà date dal nuovo ruolo nella rotazione
	\item Caricamento automatico di documenti da un \glossario{repository}\ ad un altro.
\end{itemize}	
\paragraph{Todo}
\begin{itemize}
	\item Creazione di un'indice intermedio per una maggior accuratezza nell'interazione tra dizionario dati e \glossario{query};
	\item Bozza di interfaccia grafica della \glossario{web app};
	\item Progettazione iniziale della \glossario{web app};
	\item Avanzamento nella scrittura dei documenti (\Gls, \PdP, \AdR, \NdP, \PdQ);
	\item Aggiornameto sullo stato del progetto alla proponente: qui verranno anche proposti gli strumenti di sviluppo della webapp.
\end{itemize}
\paragraph{Dubbi}
\begin{itemize}
	\item Dubbio sul come far valere le ore che ci si ritrova a fare in un ruolo diverso;
	\item A che livello di dettaglio si dovrebbe arrivare nel documento di \NdP.
\end{itemize}

\vspace{0.5\baselineskip}
\par Gli argomenti sopracitati sono stati riportati nella presentazione per il diario di bordo del 3 maggio 2024.