\section{Riunione}
\subsection{Ordine del Giorno}
\begin{itemize}
	\item Discussione dei punti in vista del Diario di Bordo.
	\item Esposizione delle funzioni di \glossario{Jira} e pianificazione della migrazione.
	\item Discussione delle attività svolte dai membri del gruppo durante la prima settimana di \glossario{sprint}.
\end{itemize}

\subsection{Attività svolte}
\par All'inizio del meeting viene chiesto ai del gruppo di esporre le attività svolte durante la prima settimana di \glossario{sprint}.
\subsubsection{\Programmatore}
\par Vengono spiegate le tecnologie studiate e i test svolti su un \glossario{dizionario dati} di prova per comprendere il funzionamento dei modelli. Vengono inoltre esposti i dubbi e le difficoltà riscontrati, con un confronto tra gli altri membri per organizzare incontri e sistemare i problemi sorti.
\begin{itemize}
	\item Sono state riscontrate delle difficoltà iniziali nel lavorare ad un motore di \glossario{database} con esempi di database \glossario{SQLite}, altrimenti non c'era modo di testare con [txtai].
	\item L'idea iniziale è stata quella di prendere uno schema di \glossario{database} in formato \glossario{JSON} che potesse essere analizzato assieme alla \glossario{query} dell'utente e restituire il \glossario{prompt} con le parole prese dal \glossario{dizionario dati} e un punteggio in base all'accuratezza.
	\item È stato tuttavia riscontrato un problema in \glossario{txtai} che coinvolgeva il formato \glossario{JSON}: infatti questo si presentava inadatto al modello che prediligeva quesiti posti in linguaggio naturale.
	\item Così facendo \glossario{txtai} può trasformare la richiesta in un vettore che viene inserita in un indice di vettori.
	\item Viene dato un suggerimento sulla distribuzione dei ruoli per il prossimo \glossario{sprint}, raccomandando più programmatori e mantenere fisso quello attuale, in modo da facilitare l'apprendimento delle tecnologie al membro nuovo nel ruolo e avere maggiore personale al lavoro sul risolvimento del problema.
	\item Viene anche esposta la necessità di trovare un formato migliore per la \glossario{LLM} che verrà poi scelta, in modo che il linguaggio risulti meno tecnico per la \glossario{LLM} o trovare un modello che supporti la lettura del \glossario{dizionario dati} nel formato richiesto.
	\item Infine viene teorizzato un sistema per sviluppare un \glossario{parser} che consenta di convertire il \glossario{dizionario dati} dal formato \glossario{JSON} ad un formato in linguaggio naturale. Questa viene inserita nelle attività da fare
\end{itemize}

\subsubsection{\Amministratore}
\par Vengono spiegate le attività svolte come la stesura della struttura del \Gls, [mancano parti che ha fatto Tommaso]. Viene inoltre puntualizzata la necessità di un incontro con il Professor. Tullio Vardanega per un chiarimento sullo stabilimento di un criterio col quale inserire i termini nel \Gls. Relativamente al \Gls\, si è discusso anche se il documento fosse a uso interno o esterno, e il numero di occorrenze di termini che nei documenti vanno indicati come riferimento al \Gls. Questi dubbi verranno esposti al diario di bordo del 26 aprile 2024.

\subsubsection{\Progettista}
\par Viene esposta la strutturazione del \glossario{dizionario dati} in formato \glossario{JSON} che è stato poi utilizzato come test dal progettista in fase di apprendimento delle tecnologie. È stata inoltre studiata la possibile composizione del \glossario{prompt} finale che sarà disponibile all'utente.

\subsubsection{\Analista}
\par La discussione si è centrata sulla continuazione del documento \AdR\ e sulla prossima trascrizione del documento in formato \glossario{LateX}.

\subsection{Esposizione \glossario{Jira}}
\par Vengono esposte al gruppo le funzionalità di \glossario{Jira}: in particolare la strutturazione del \glossario{diagramma di Gantt} per avere una vista grafica sullo stato di pianificazione dello sprint e sulle attività a questo legate. Vengono inoltre spiegati i ticket e l'interazione tra questi e le pull request di Github, le quali si potranno aprire collegando un ticket e chiudere con la conseguente chiusura di questi.

\subsection{Discussione dei punti da trattare nel prossimo diario di bordo}
\par Vengono discussi i punti salienti da trattare nel prossimo diario di bordo del 26 aprile 2024:
\subsubsection{Done}
\begin{itemize}
	\item Passaggio Passaggio a \glossario{Jira Software} come \glossario{Issue Tracking System}
	\item Studio preliminare delle tecnologie \ \rightarrow\ Studio di [Txtai], modelli e interazione tra questi e il \glossario{dizionario dati} in formato \glossario{JSON}.
	\item Stesura del \glossario{dizionario dati}
	\item Avanzamento scrittura dei documenti (\PdP, \NdP, \AdR, \Gls)
	\item Consuntivo primo \glossario{sprint}
	\item Modifica del template \glossario{LaTeX} \ \rightarrow\ Modifiche estetiche e funzionali, come il cambiamento della tabella delle attività da svolgere con l'inserimento delle colonne per \glossario{ticket} e date di scadenza associate.
\end{itemize}
\subsubsection{Difficoltà}
\begin{itemize}
	\item Gestione delle attività di natura organizzativa su \glossario{GitHub} \ \rightarrow\ Questa difficoltà è stata risolta con il passaggio agli strumenti di \glossario{Jira Software}
	\item Configurazione di \glossario{Jira Software}
\end{itemize}	
\subsubsection{ToDo}
\begin{itemize}
	\item Aggiornamento migliorativo della pianificazione futura nel \PdP;
	\item Stesura del \PdQ;
	\item Approfondire la funzionalità di debug nel documento di \AdR \ \rightarrow\ Sulla base delle richieste della Proponente per lo strumento di supporto al Tecnico
	\item Individuazione delle tecnologie per sviluppare la web app
	\item Inserimento nel \Gls\ dei termini individuati durante la stesura dei documenti di progetto \ \rightarrow\ Il \Gls\ è dotato di una struttura completa, ma deve essere popolato con i termini individuati finora nei documenti.	
\end{itemize}
\subsubsection{Dubbi}
\begin{itemize}
	\item \Gls\ come documento interno o esterno \ \rightarrow\ Ossia se i termini da inserire sono a discrezione del gruppo o se devono essere aggiunti anche termini sconosciuti all'esterno [rivedere]
	\item Criterio di scelta dei termini da inserire nel \Glossario \ \rightarrow\ Ossia se inserire anche termini scontati per i membri del gruppo, ma che possono non esserlo per utenti inesperti
	\item Numero di occorrenze di un termine del \Gls\ da formattare \ \rightarrow\ Ossia se nei documenti ogni termine indicato nel \Gls\ deve essere segnalato o se bastano le prime istanze di questo.
	\item Modalità di ridistribuzione di eventuali risorse (in eccesso o in difetto) durante la stesura del consuntivo \ \rightarrow\ Ossia vista la staticità del preventivo, come far fronte ad eventuali cambiamenti in vista dei consuntivi di periodo.
\end{itemize}
\par Le seguenti sezioni sono state poi inserite all'interno del documento per la presentazione al Diario di Bordo del 2024-04-26.

\newpage
\subsection{Note}
\par Di seguito delle note per me e il verificatore per la stesura finale del verbale:
\begin{itemize}
	\item Ci sono delle note che vanno sistemate o approfondite riguardo i temi trattati durante l'esposizione delle attività
	\item Devo rivedere la sezione di \marco  per capire se farla narrativa o tenerla a punti siccome è la più lunga
	\item Sono da rivedere tutti i termini per inserire quelli che vengono menzionati nel \Glossario
  \item Ho caricato anche la firma che però rimane commentata fino all'approvazione
\end{itemize}