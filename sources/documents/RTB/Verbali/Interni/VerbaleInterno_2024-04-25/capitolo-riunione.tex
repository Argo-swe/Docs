\section{Riunione}
\subsection{Ordine del giorno}
\begin{itemize}
	\item Analisi dell'andamento dello \glossario{sprint} in vista del diario di bordo;
	\item Esposizione delle funzionalità di \glossario{Jira} e organizzazione della migrazione da \glossario{GitHub} a \glossario{Jira Software};
	\item Pianificazione delle attività da svolgere entro la fine dello \glossario{sprint}.
\end{itemize}

\subsection{Discussione e decisioni}
\par All'inizio del meeting il gruppo ha discusso le attività svolte durante la prima settimana di \glossario{sprint}.
\subsubsection{Programmatore} \label{sec:programmatore}
\par Il Programmatore ha descritto brevemente le tecnologie studiate e ha mostrato al team i test svolti su una versione preliminare del \glossario{dizionario dati} (definita con l'intento di comprendere il funzionamento dei \glossario{modelli}). Sono stati esposti inoltre i dubbi e le difficoltà riscontrate, con un confronto collettivo per mitigare i problemi.
\par Di seguito sono riportati i punti della discussione:
\begin{itemize}
	\item È stato necessario l'impiego di \glossario{txtai} che richiedeva del materiale per svolgere i test: per questo è stato utilizzato un \glossario{dizionario dati} prelevato da un \glossario{database} di esempio (\glossario{chinook}) di \glossario{SQLite};
	\item L'idea iniziale è stata quella di prendere uno schema di \glossario{database}, in formato \glossario{JSON}, che potesse essere analizzato assieme alla richiesta (in linguaggio naturale) dell'utente e restituire la porzione di \glossario{dizionario dati} relativa alla richiesta originale e un punteggio che ne indicasse l'accuratezza;
	\item È emerso un problema che coinvolgeva il formato \glossario{JSON}; quest'ultimo, infatti, non si armonizzava con il \glossario{modello}, che prediligeva invece documenti in linguaggio naturale;
	\item Il Programmatore ha spiegato che \glossario{txtai} si occupa della generazione e archiviazione di \glossario{embeddings}, ovvero \glossario{vettori} rappresentanti specifici documenti (ad esempio frasi, nel nostro caso), all'interno di un \glossario{indice}, per facilitare la comparazione con uno schema già precedentemente indicizzato;
	\item Il Programmatore ha consigliato, per il prossimo \glossario{sprint}, di assegnare più risorse al ruolo di programmatore, in modo da disporre di maggior personale impiegato nella risoluzione dei problemi incontrati;
	\item Il team ha deciso di confermare il Programmatore attuale anche per l’iterazione successiva, cosicché l'apprendimento delle tecnologie e la condivisione delle conoscenze possano risultare più immediati;
	\item Si è discussa la possibilità di modificare il formato del \glossario{dizionario dati} (in modo che il linguaggio risulti meno complesso per il \glossario{modello}) o, in alternativa, trovare un \glossario{modello} che supporti la lettura del \glossario{dizionario dati} nel formato richiesto;
	\item Il gruppo ha valutato lo sviluppo di un \glossario{parser} per convertire il \glossario{dizionario dati} dal formato \glossario{JSON} a un formato in linguaggio naturale.
\end{itemize}

\subsubsection{Amministratore}
\par L'Amministratore ha illustrato le modifiche effettuate al template \glossario{LaTeX} e ha aggiornato il gruppo sulla stesura del \Gls. È stata evidenziata la necessità di chiarire con il Prof. Tullio Vardanega il criterio col quale inserire i termini nel \Gls. Relativamente al \Gls, si è discusso anche se il documento fosse a uso interno o esterno. Questi dubbi verranno esposti nel diario di bordo del 26 aprile 2024.

\subsubsection{Progettista}
\par Il team ha esaminato la struttura del \glossario{dizionario dati}, in formato \glossario{JSON}, che il Progettista ha usato come prototipo in fase di apprendimento delle tecnologie. Durante la prima settimana di \glossario{sprint}, il Progettista ha anche studiato la possibile composizione del \glossario{prompt} da fornire in output all'utente.

\subsubsection{Analista}
\par La discussione si è poi spostata sul documento di \AdR\ e sulla sua conversione in \glossario{LateX}. Entro la fine dello \glossario{sprint}, verrà finalizzata la conversione in \glossario{LaTeX} e, contestualmente, verranno sviluppati i \glossario{casi d'uso}.

\subsubsection{Jira Software}
\par Sono state esposte al gruppo le funzionalità essenziali di \glossario{Jira}, specialmente la creazione del \glossario{diagramma di Gantt} in fase di pianificazione dello \glossario{sprint}. In aggiunta, l'Amministratore ha illustrato il sistema di apertura dei \glossario{ticket}, l'interazione con le \glossario{pull request} di \glossario{GitHub} e la prassi da seguire per automatizzare il cambio di stato dei task.

\subsubsection{Punti da trattare nel prossimo diario di bordo}
\par Il gruppo ha discusso i punti salienti da trattare nel diario di bordo del 26 aprile 2024.
\paragraph{Done}
\begin{itemize}
	\item Passaggio a \glossario{Jira Software} come \glossario{Issue Tracking System};
	\item Studio preliminare delle tecnologie (\glossario{txtai} e interazione tra i \glossario{modelli} e il \glossario{dizionario dati} in formato \glossario{JSON});
	\item Definizione del \glossario{dizionario dati};
	\item Avanzamento scrittura dei documenti (\PdP, \NdP, \AdR, \Gls);
	\item Consuntivo primo \glossario{sprint};
	\item Modifica del template \glossario{LaTeX} (modifiche estetiche e funzionali).
\end{itemize}
\paragraph{Difficoltà}
\begin{itemize}
	\item Gestione delle attività di natura organizzativa su \glossario{GitHub};
	\item Configurazione di \glossario{Jira Software}.
\end{itemize}	
\paragraph{Todo}
\begin{itemize}
	\item Aggiornamento migliorativo della pianificazione futura nel consuntivo;
	\item Stesura del \PdQ;
	\item Approfondire la funzionalità di debug nel documento di \AdR\ (come suggerito dalla \glossario{Proponente});
	\item Individuazione delle tecnologie di sviluppo per la \glossario{web app};
	\item Inserimento nel \Gls\ dei termini individuati durante la stesura dei documenti di progetto.	
\end{itemize}
\paragraph{Dubbi}
\begin{itemize}
	\item \Gls\ come documento interno o esterno;
	\item Criterio di scelta dei termini da inserire nel \Gls;
	\item Numero di occorrenze di un termine del \Gls\ da formattare;
	\item Modalità di ridistribuzione di eventuali risorse, in eccesso o in difetto, durante la stesura del consuntivo.
\end{itemize}

\vspace{0.5\baselineskip}
\par Gli argomenti sopracitati sono stati riportati nella presentazione per il diario di bordo del 26 aprile 2024.