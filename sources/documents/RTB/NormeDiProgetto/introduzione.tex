\section{Introduzione}

\subsection{Scopo del documento}
Il presente documento ha lo scopo di delineare le \glossario{best practices} e il \glossario{way of working} che il gruppo \textit{Argo} ha individuato e adotta durante tutto lo svolgimento del progetto didattico. Poiché il way of working è definito incrementalmente durante il corso del progetto, questo documento non è da considerarsi un testo definitivo o completo.

\subsection{Scopo del prodotto}
\ScopoDelProdotto


\subsection{Glossario}
\GlossarioIntroduzione


\subsection{Riferimenti}

\subsubsection{Riferimenti normativi}
\begin{itemize}
  \item C9 ChatSQL: creare frasi SQL da linguaggio naturale (Zucchetti S.p.A.):\\ \url{https://www.math.unipd.it/~tullio/IS-1/2023/Progetto/C9.pdf}\\ \url{https://www.math.unipd.it/~tullio/IS-1/2023/Progetto/C9p.pdf};
  \item Standard ISO/IEC 12207:1995:\\ \url{https://www.math.unipd.it/~tullio/IS-1/2009/Approfondimenti/ISO_12207-1995.pdf};
\end{itemize}

\subsubsection{Riferimenti informativi}

