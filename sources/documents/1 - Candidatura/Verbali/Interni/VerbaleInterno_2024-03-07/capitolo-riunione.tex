\section{Riunione}
\subsection{Ordine del Giorno}
\begin{itemize}
	\item Assegnazione e conferma dei ruoli per le riunioni interne ed esterne;
	\item Analisi dei pro e contro di ciascun capitolato;
	\item Votazione capitolati;
	\item Predisposizione repository del team.
\end{itemize}

\subsection{Discussione e decisioni}

\subsubsection{Assegnazione ruoli}
I ruoli assegnati per la riunione corrente sono stati:
\begin{itemize}
	\item Responsabile: \sebastiano, confermato come responsabile di progetto dopo la proposta dell’incontro precedente;
	\item Segretari: \tommaso{} e \raul.
\end{itemize}

\subsubsection{Analisi iniziale dei pro e contro di ciascun capitolato}
Dopo aver esaminato individualmente i documenti di presentazione forniti dalle Proponenti, ciascun membro del gruppo ha esposto i risultati delle proprie valutazioni. Si sono evidenziati i punti a favore e le criticità dei tre capitolati disponibili, chiarendo al contempo eventuali dubbi.

\paragraph{Capitolato 3 - EasyMeal}
\begin{itemize}
	\item \textbf{Pro:}
	\begin{itemize}
		\item Esposizione dettagliata del problema con casi d'uso chiari e ben definiti;
		\item Ampio supporto da parte del proponente;
		\item Libertà sul dominio tecnologico.
	\end{itemize}
	\item \textbf{Contro:}
	\begin{itemize}
		\item Cardinalità dei requisiti, alcuni dei quali piuttosto onerosi (sistema di chat, scalabilità);
		\item Sfida di apprendimento poco stimolante;
		\item Difficoltà nell'integrazione di Websocket (parte della comunicazione client-server);
		\item Cifratura delle comunicazioni presentata come requisito opzionale ma ritenuta dal gruppo quasi indispensabile;
		\item Difficoltà nella gestione dei database (eventualmente replicati) e delle transazioni.
	\end{itemize}
\end{itemize}

\paragraph{Capitolato 6: SyncCity}
\begin{itemize}
	\item \textbf{Pro:}
	\begin{itemize}
		\item Interesse significativo da parte del gruppo, soprattutto nelle tematiche di IoT e data stream processing;
		\item Ampia disponibilità dell'azienda durante l’avanzamento del progetto.
	\end{itemize}
	\item \textbf{Contro:}
	\begin{itemize}
		\item Requisiti aggiuntivi nebulosi e potenzialmente ampi;
		\item Curva di apprendimento notevole;
		\item Dubbi sulla configurazione dei database noSQL proposti;
		\item La gestione di dati con diversi formati e qualità potrebbe risultare complessa.
	\end{itemize}
\end{itemize}

\paragraph{Capitolato 9: ChatSQL}
\begin{itemize}
	\item \textbf{Pro:}
	\begin{itemize}
		\item Architettura meno complessa rispetto agli altri capitolati;
		\item Espandibilità del progetto;
		\item Libertà sul dominio tecnologico;
		\item Possibilità di lavorare con LLM (Large Language Models).
	\end{itemize}
	\item \textbf{Contro:}
	\begin{itemize}
		\item Processo iterativo basato sul trial-and-error ripetitivo e potenzialmente scoraggiante;
		\item La pianificazione temporale potrebbe risultare complessa;
		\item Le operazioni di pulizia prima della costruzione del prompt potrebbero richiedere diverse sessioni di apprendimento.	
	\end{itemize}
\end{itemize}

\subsubsection{Votazione capitolati}
Alla luce della discussione precedente, il gruppo ha avviato un sondaggio per individuare il capitolato di maggior interesse. SyncCity ha ottenuto 4 voti a favore, risultando essere il progetto più interessante e stimolante, seguito da ChatSQL con 3 voti e EasyMeal con 0 voti. Di conseguenza, il team ha deciso di pianificare un incontro con le aziende Zucchetti S.p.A. e SyncLab S.r.l..

\subsubsection{Predisposizione repository del team}
Il gruppo ha accettato la proposta di Sebastiano Lewental di convertire l’account Github in un’organizzazione, da definire durante il prossimo incontro. Le repository saranno inoltre suddivise tra documentazione e codice.

\clearpage
