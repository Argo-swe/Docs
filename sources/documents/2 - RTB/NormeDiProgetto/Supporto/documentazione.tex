\subsection{Documentazione}

\subsubsection{Descrizione}
Il processo di documentazione registra l'informazione generata da altri processi o attività. Il processo contiene l'insieme di attività che pianificano, producono, modificano, rilasciano e mantengono i documenti legati al progetto.\\
Il processo consiste nelle seguenti attività:
\begin{itemize}
  \item Implementazione del processo;
  \item Progettazione e sviluppo;
  \item Rilascio.
\end{itemize}

\paragraph{Implementazione del processo}\label{implementazioneprocessodocumentazione}
Questa attività definisce quali documenti saranno generati durante il progetto, definendo per ciascuno:
\begin{itemize}
  \item Titolo;
  \item Scopo;
  \item Descrizione;
  \item Responsabilità per contribuzione, redazione, verifica e approvazione;
  \item Pianificazione per versioni provvisorie e finali.
\end{itemize}

\paragraph{Progettazione e sviluppo}
Questa attività consiste nel progettare e redarre ciascun documento nel rispetto degli standard definiti per formato e contenuto, successivamente controllati dal \glossario{Verificatore}.

\paragraph{Rilascio}
Questa attività comincia con l'approvazione finale del documento da parte del \glossario{Responsabile} in carica, e della Proponente nel caso di verbali ad uso esterno. Prosegue con la pubblicazione del documento nel \glossario{repository} apposito della documentazione.

\subsubsection{Lista documenti}
I documenti da produrre e mantenere durante il corso del progetto sono:
\begin{itemize}
  \item \emph{Piano di Progetto};
  \item \emph{Norme di Progetto};
  \item \emph{Piano di Qualifica};
  \item \emph{Analisi dei Requisiti};
  \item \emph{Manuale Utente};
  \item \emph{Glossario};
  \item \emph{Verbali Interni};
  \item \emph{Verbali Esterni}.
\end{itemize}

\subsubsection{Ciclo di vita}
Il ciclo di vita di un documento è composto dai seguenti eventi:
\begin{enumerate}
  \item Vengono definite le caratteristiche di base del documento o di una sua parte come da sezione \ref{implementazioneprocessodocumentazione};
  \item Il Redattore stila una bozza iniziale. Se è necessario l'input di più persone in maniera sincrona, tale bozza viene prodotto in un ambiente condiviso;
  \item Prodotta una bozza di tutto il contenuto necessario, il Redattore produce una versione del documento con la forma e i metodi stabiliti in queste norme;
  \item Viene sottoposto a verifica il risultato della redazione. Se il Verificatore propone delle modifiche, vengono attuate ritornando alla fase precedente;
  \item In seguito a un esito positivo della verifica, se il risultato è un documento completo e che richiede rilascio, viene sottoposto ad un'approvazione finale del responsabile, bloccante in modo analogo alla verifica.
\end{enumerate}

\subsubsection{Ambiente di lavoro}
\paragraph{\glossario{LaTeX}}
Per lo sviluppo della documentazione del gruppo viene utilizzato un \glossario{template} \glossario{LaTeX} personalizzato. All'interno del template è definito lo stile della pagina iniziale, delle intestazioni e della formattazione generale.
Parte del template permette l'uso di comandi personalizzati per favorire la consistenza di termini specifici spesso utilizzati (es.: nomi di documenti, nomi dei membri), inoltre è gestita sempre attraverso il template l'interazione con i termini per il Glossario.\\
L'utilizzo del template garantisce:
\begin{itemize}
  \item Il disaccoppiamento di forma e contenuto della documentazione;
  \item L'uniformità dello stile della documentazione;
  \item La responsabilità del Redattore è il solo contenuto;
  \item La possibilità di creare documenti in maniera modulare, conciliata in modo uniforme.
\end{itemize}

\paragraph{\glossario{Docker}}
La compilazione di file LaTeX può differire in base al compilatore utlizzato, il sistema operativo o altre caratteristiche del sistema locale. Per garantirne l'uniformità, la compilazione dei documenti viene effettuata all'interno di un container Docker costruito a partire da un'immagine comune.

\paragraph{Google Docs}
Per scrivere un documento è spesso necessario lavorare in maniera sincrona, Google Docs permette la condivisione e il lavoro contemporaneo di più persone. I limiti del software tuttavia non permettono di generare un documento finale adeguato, per cui le produzioni tramite questo mezzo sono da considerarsi bozza da cui eseguire la coversione.

\subsubsection{Struttura documenti}
Ciascun documento è fornito di questi elementi:
\begin{itemize}
  \item Prima pagina:
  \begin{itemize}
    \item Logo del gruppo;
    \item Titolo;
    \item Nome del gruppo;
    \item Nome del progetto;
    \item Versione attuale;
    \item Approvatore;
    \item Uso del documento (Interno/Esterno);
    \item Destinatari del documento;
    \item Logo dell'Università di Padova.
  \end{itemize}
  \item Registro delle modifiche:
  \begin{itemize}
    \item Versione del documento in seguito alla modifica;
    \item Data della modifica;
    \item Redattore della modifica (coincide con il Verificatore nel caso di riga associata alla verifica generale, col responsabile del caso di riga associata al rilascio);
    \item Verificatore della modifica (coincide con il Responsabile nel caso di riga associata al rilascio);
    \item Descrizione della modifica.
  \end{itemize}
  \item Indice dei contenuti;
\end{itemize}

\paragraph{Verbali}
I verbali oltre agli elementi forniti sopra possiedono una ulteriore definizione della struttura:
\begin{enumerate}
  \item Informazioni:
  \begin{itemize}
    \item Orario di inizio incontro;
    \item Orario di fine incontro;
    \item Mezzo di pianificazione dell'incontro;
    \item Tipo di incontro (di persona/da remoto).
    \item Descrizione dell'incontro;
    \item Partecipanti all'incontro e durata partecipazione.
  \end{itemize}
  \item Riunione:
  \begin{itemize}
    \item Ordine del giorno dell'incontro;
    \item Discussione e decisioni prese durante l'incontro, contiene il corpo principale del verbale.
  \end{itemize}
  \item Tabella di task ToDo/In progress:
  \begin{itemize}
    \item Codice della \glossario{issue} \glossario{GitHub} relativa all'incarico;
    \item Incarico;
    \item Incaricato/a;
    \item Scandenza.
  \end{itemize}
\end{enumerate}

\subsubsection{Stile}
Di seguito sono elencate la convenzioni stilistiche adottate dalla documentazione del gruppo.

\paragraph{Utilizzo del femminile}
Quando è necessario fare riferimento tramite ruolo di progetto ad un membro del gruppo con il genere femminile, si utilizzano i seguenti termini:
\begin{itemize}
  \item \textbf{Responsabile} è invariato;
  \item \textbf{Amministratrice} al posto di Amministratore;
  \item \textbf{Analista} è invariato;
  \item \textbf{Progettista} è invariato;
  \item \textbf{Programmatrice} al posto di Programmatore;
  \item \textbf{Redattrice} al posto di Redattore;
  \item \textbf{Verificatrice} al posto di Verificatore.
\end{itemize}

\paragraph{Formattazione testo}
\begin{itemize}
  \item \textbf{Termini nel Glossario:} Indicati in \textit{corsivo} e con una \textit{\small{G}} a fine parola. In base a ciascun documento tale formattazione può comparire alla sola prima occorrenza (quando il documento ha lo scopo di essere letto dall'inizio alla fine), o in maniera più frequente (quando il documento può essere letto in maniera più frammentata);
  \item \textbf{Nomi di documento:} Indicati in \textit{corsivo} con le iniziali di parola maiuscole eccetto preposizioni (es.: \textit{Piano di Progetto}, non \textit{Piano Di Progetto});
  \item \textbf{Nomi di ruolo:} Indicati con la iniziale maiuscola e in \textit{corsivo};
  \item \textbf{Data:} Indicata in formato YYYY-MM-DD nelle tabelle riassuntive e nei nomi dei file, in formato esteso (esempio: 20 aprile 2024) quando si trova all'interno di testo discorsivo.
\end{itemize}

\subsubsection{Strumenti}
Gli strumenti impiegati nel processo di documentazione sono:
\begin{itemize}
  \item \textbf{Git:} \glossario{Version Control System} utilizzato dal gruppo;
  \item \textbf{GitHub:} Piattaforma ospite del repository del gruppo;
  \item \textbf{LaTeX:} \glossario{markup language} per la scrittura di documenti;
  \item \textbf{Docker:} Software per \glossario{containerizzazione} utilizzato dal gruppo per uniformare la generazione di documenti;
  \item \textbf{Google Docs:} Strumento per la creazione di documenti condivisi, utilizzato per la collaborazione nella redazione di un documento.
\end{itemize}
