\section{Riunione}
\subsection{Ordine del Giorno}
\begin{itemize}
	\item Decisione ruoli riunione
	\item Analisi dettagli, pro e contro di ciascun capitolato
	\item Organizzazione repository del team e iniziale documentazione
\end{itemize}

\subsection{Discussione e decisioni}

\subsubsection{Decisione ruoli riunione}
I ruoli assegnati per la riunione corrente sono stati:
\begin{itemize}
	\item Responsabile: \sebastiano
	\item Moderatore: \tommaso
	\item Verbalizzante: \raul
	\item Verificatore: \martina
\end{itemize}

\subsubsection{Analisi dettagli, pro e contro di ciascun capitolato}
Il gruppo discute dei punti a favore o contrari per ogni capitolato proposto:

\paragraph{Capitolato 3 - EasyMeal}
\begin{itemize}
	\item A favore del capitolato sono state riconosciute un’esposizione molto dettagliata del problema, con presentazione casi degli use case, un ampio supporto del proponente e una libertà sulle tecnologie da utilizzare per la maggior parte del progetto
	\item A sfavore della scelta del capitolato sono stati evidenziati una numerosa quantità di requisiti, sia obbligatori che minimi, con alcuni di complessità piuttosto ampia (sistema di chat, scalabilità). Inoltre la maggior parte del gruppo ha dimostrato più interesse per altri capitolati. Websocket difficile da integrare (comunque parte di comunicazione client-server dovendo pensare ad un gran numero di utenti), opzionale crittazione comunicazioni ma molta enfasi (domanda per azienda su questo tema), database replicati (ridondanza) cioè studiare bene la consistenza nelle transazioni.
	\item Architettura da portare come progetto va effettivamente implementata per larga scala? Domanda all’azienda
\end{itemize}

\paragraph{Capitolato 6: SyncCity}
\begin{itemize}
	\item A favore del capitolato una buona parte del gruppo lo ha evidenziato come l’argomento più interessante tra i proposti (in particolare l’affrontare IoT e stream processing). Il capitolato evidenzia anche un’ampia disponibilità da parte dell’azienda, anche per deep dives sulle tecnologie da utilizzare per il capitolato.
	\item A sfavore del capitolato, nella descrizione di funzionalità aggiuntive rispetto alla consegna minima i requisiti appaiono molto generici e potenzialmente molto ampi (come la previsione di dati). Inoltre le tecnologie proposte, e più in generale l’intero topic, è per la maggior parte una novità che può necessitare ulteriore tempo per l’apprendimento
	\item dubbi sul database (ClickHouse/colonnare) perché non classico SQL?
	\item coordinare dati con diversi formati e qualità che devono essere sottoposti ad analisi e coordinati (come standardizzare)
\end{itemize}

\paragraph{Capitolato 9: ChatSQL}
\begin{itemize}
	\item A favore del capitolato, la richiesta è più semplice delle altre, anche se espandibile. Le tecnologie utilizzate non sono vincolate
	\item A sfavore del capitolato, il prompt engineering è un processo iterativo (non necessariamente complesso, ma basato sul trial-and-error), anche a causa dell’interazione con black box.
	\item processo iterativo (non tanto complesso, ma si deve andare un po' a tentativi ed errori), costi temporali maggiori (nel fare tentativi), lavorazione anche di informazioni sensibili alle quali prestare attenzione,pulizia del pre-prompt complicata e lunga.
\end{itemize}

In seguito all’analisi dei capitolati singolarmente si è raccolta una preferenza sui capitolati, suddivisa completamente tra il capitolato 6 (3 preferenze) e 9 (4 preferenze).

\subsubsection{Organizzazione repository del team e iniziale documentazione}
VI\_2017-03-2\_D1: Per rendere la demo il più utile e affidabile possibile per una situazione di utilizzo reale, si è deciso di fare in modo che il fattorino, per le consegne del ristorante che possiede la Bubble \& Eat, possa prenotare le consegne, in modo da poter gestire il proprio tempo ed il proprio percorso più efficientemente. In questo modo è inoltre possibile prevenire l'impostazione di un fittizio stato di consegna, nel caso in cui il fattorino stia portando a termine una data consegna e voglia prenderne in carico altre. Viene quindi così impedito a un fattorino di prenotare consegne che non può portare a termine, bloccando di fatto la lista per eventuali altri colleghi.

\clearpage
