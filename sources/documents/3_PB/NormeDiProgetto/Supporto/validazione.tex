\subsection{Validazione}\label{validazione}

\subsubsection{Scopo}
\par Il processo di validazione ha lo scopo di determinare se le caratteristiche del software soddisfano i bisogni dell'utente e l’uso previsto. La validazione è una conferma finale, e consiste in una serie di attività volte a garantire che il prodotto soddisfi i requisiti specificati nell’\AnalisiDeiRequisiti.

\subsubsection{Implementazione del processo}
\par L’esito positivo dei test di unità, di integrazione e di sistema rappresenta una precondizione necessaria per avviare il processo di convalida del software. La validazione viene eseguita alla presenza dell'azienda proponente. Il processo di validazione include le seguenti attività:
\begin{itemize}
  \item \textbf{Revisione dei requisiti}: il team accerta che tutte le funzionalità richieste siano state implementate;
  \item \textbf{Collaudo}: il team esegue i test di accettazione per assicurarsi che il prodotto funzioni come previsto.
\end{itemize}

\vspace{0.5\baselineskip}
\par Una volta accertato che i requisiti sono stati implementati in maniera consistente, il prodotto viene convalidato dall’azienda proponente.

\subsubsection{Test di accettazione}
\par I test di accettazione accertano il soddisfacimento dei requisiti utente. Sono test black-box il cui obiettivo principale non è tanto quello di individuare difetti, quanto di confermare che il prodotto risponda alle reali necessità dell’utente finale. I test di accettazione prevedono l’esecuzione di una serie di casi di test, ognuno dei quali simula uno scenario di utilizzo del prodotto, il più possibile conforme alle condizioni d’uso reali. Si tratta di test formali che si concentrano sulla replica dei comportamenti degli utenti e, pertanto, richiedono che tutta l'applicazione sia attiva. Possono essere definiti già a partire dal capitolato.

\subsubsection{Strumenti}
\IntroStrumenti
\begin{itemize}
  \item Zoom.
\end{itemize}