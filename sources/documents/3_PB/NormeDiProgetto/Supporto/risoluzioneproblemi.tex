\subsection{Risoluzione dei problemi}\label{risoluzione_problemi}

\subsubsection{Scopo}
\par Il processo di risoluzione dei problemi mira ad analizzare e contrastare i problemi che possono insorgere durante lo sviluppo. Il gruppo esamina costantemente lo stato di esecuzione dei processi per individuare criticità e difetti. Una volta rilevato un problema, il gruppo si impegna ad arginarlo tempestivamente, documentando le cause e pianificando soluzioni per ridurre la probabilità che si ripresenti. 

\subsubsection{Implementazione del processo}
\par All'insorgere di una criticità, i membri del gruppo devono segnalare prontamente il problema, tracciandolo nell’\glossario{ITS} come ticket di tipo "bug". Inoltre, il problema deve essere comunicato al resto del team tramite i canali di comunicazione dedicati (\sezione{sec:comunicazione}). Chi rileva il bug ha il compito di descrivere la natura del problema e aprire uno spazio di discussione su \glossario{GitHub}. Ogni problema viene classificato per categoria e priorità.
\par Spesso le criticità emergono durante le riunioni; in tal caso, verranno documentate in un verbale interno. Una volta individuato e segnalato un problema, questo viene preso in carico da o uno più membri del gruppo, i quali si occuperanno di risolverlo e di formalizzare la soluzione.

\subsubsection{Risoluzione dei problemi}
\par Se il problema è stato rilevato in un \glossario{branch} già aperto, un membro del team può apportare direttamente una correzione (bug fix) che verrà poi integrata all'interno dell'ambiente condiviso.
\par In caso contrario, verrà aperto un branch dedicato per la risoluzione. 

\subsubsection{Strumenti}
\IntroStrumenti
\begin{itemize}
    \item Jira;
    \item Zoom;
    \item Discord;
    \item Telegram;
    \item Google Meet;
    \item GitHub.
\end{itemize}