\subsection{Audit}\label{audit}

\subsubsection{Scopo}
\par Il processo di audit serve a determinare l'adesione ai requisiti, alla pianificazione e ai vincoli del progetto. Attuare il processo richiede solamente due parti interlocutrici, di cui una revisiona le attività e/o il prodotto dell'altra.

\subsubsection{Implementazione del processo}
\par Le revisioni sono pianificate e concordate, nel contenuto e nella data, da ciascuna delle parti coinvolte. Eventuali criticità individuate durante la revisione devono essere documentate e gestite, attuando il processo di risoluzione problemi (\ref{risoluzione_problemi}). I risultati di una revisione e le azioni conseguenti devono essere stabiliti in comune accordo tra le due parti e opportunamente documentati.

\subsubsection{Revisioni di project management}
\par Le revisioni di project management sono attuate durante le riunioni interne, e registrate attraverso i corrispondenti verbali e tramite il consuntivo di periodo nel \PdP. Monitorano principalmente il corso delle attività svolte durante lo sprint in esame e individuano azioni correttive da attuare a breve o medio termine, in base all'urgenza dei problemi individuati.

\subsubsection{Strumenti}
\IntroStrumenti
\begin{itemize}
  \item LaTeX;
  \item Discord;
  \item Telegram;
  \item Google Meet.
\end{itemize}