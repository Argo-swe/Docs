\subsection{Revisione congiunta}\label{revisione_congiunta}

\subsubsection{Scopo}
\par Le revisioni congiunte consistono in attività di valutazione dei risultati conseguiti dal gruppo da parte di un componente esterno.

\subsubsection{Implementazione del processo}
\par Le revisioni congiunte avvengono tra il gruppo e la \glossario{Proponente} con l'obiettivo di verificare che i requisiti e la loro implementazione rispettino le aspettative. A tal fine, vengono organizzati degli incontri periodici durante i quali il team espone il lavoro svolto. Il gruppo propone un incontro tramite \glossario{Gmail}; la Proponente risponde indicando il giorno e l'ora più opportuni. Da qui il gruppo conferma e si impegna a inviare un link per la chiamata su \glossario{Zoom}.
\par Durante l'incontro, il team espone i risultati delle attività, interagendo con la \glossario{Proponente} per ottenere un riscontro sull'operato o sui dubbi emersi.
\par Le riunioni vengono verbalizzate e i documenti elaborati vengono sottoposti all'approvazione della \glossario{Proponente}.

\subsubsection{Revisioni tecniche congiunte}
\par Durante le revisioni, il team e la \glossario{Proponente} discutono i requisiti individuati. Inoltre, le riunioni vertono sull'approfondimento delle strategie di composizione del \glossario{prompt} e sulla valutazione della correttezza delle query generate dagli \glossario{LLM}. Questo processo viene effettuato generando uno o più prompt; ciascun prompt viene fornito in input a \glossario{ChatGPT} o altri modelli su \glossario{chatLMsys} per ottenere query SQL. L'attività di revisione è accompagnata da analisi costruttive sulla correttezza, sintattica e semantica, delle query.

\subsubsection{Strumenti}
\IntroStrumenti
\begin{itemize}
    \item Zoom;
    \item chatLMsys;
    \item ChatGPT;
    \item Gmail.
\end{itemize}