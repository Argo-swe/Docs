\subsection{Fornitura}\label{fornitura}

\subsubsection{Descrizione}
Il \glossario{processo} di fornitura consiste nell'insieme di attività e compiti svolti dal \glossario{fornitore} nel rapporto con la \glossario{Proponente} Zucchetti S.p.A.. Il processo parte dalla candidatura al \glossario{capitolato} d'appalto e prosegue con la determinazione di procedure e risorse richieste per la gestione e assicurazione del progetto, incluso lo sviluppo e l'esecuzione di un Piano di Progetto.
L'obiettivo principale del processo è confrontare le aspettative della Proponente con i risultati del fornitore durante lo svolgimento del progetto, mantenendo dunque una metrica oggettiva tra il preventivo e lo stato corrente.\\
Il processo consiste nelle seguenti attività:
\begin{itemize}
  \item Selezione e studio di fattibilità;
  \item Candidatura;
  \item Pianificazione;
  \item Esecuzione e controllo;
  \item Revisione e valutazione;
  \item Consegna e completamento.
\end{itemize}

\paragraph{Selezione e studio fattibilità}
Il fornitore esamina i capitolati d'appalto e arriva a una decisione sulla candidatura per uno di essi.

\paragraph{Candidatura}
Il fornitore definisce e prepara una candidatura al capitolato d'appalto scelto producendo i seguenti documenti:
\begin{itemize}
  \item \textbf{Lettera di Candidatura}: presentazione del gruppo rivolta al \glossario{Committente};
  \item \textbf{Stima dei Costi e Assunzione Impegni}: documento che contiene un preventivo sulla distribuzione delle ore, la stima dei costi, l'assunzione degli impegni e l'analisi dei rischi;
  \item \textbf{Valutazione Capitolati}: documento che contiene l'analisi e la valutazione da parte del gruppo dei capitolati disponibili.
\end{itemize}

\paragraph{Pianificazione}
\par Il fornitore stabilisce i requisiti per la gestione, lo svolgimento e la misurazione della qualità del progetto. In seguito, sviluppa e documenta attraverso il Piano di Progetto i risultati attesi.

\paragraph{Esecuzione e controllo}
\par Il fornitore attua il Piano di Progetto sviluppato, attenendosi alle norme definite nella sezione \ref{sviluppo} e monitora la qualità del \glossario{prodotto software} nei seguenti modi:
\begin{itemize}
  \item Controllo del progresso di \glossario{performance}, costi, rendicontazione dello stato del progetto e organizzazione;
  \item Identificazione, tracciamento, analisi e risoluzione dei problemi.
\end{itemize}

\paragraph{Revisione e valutazione}
Il fornitore coordina la revisione interna ed esegue verifica e validazione secondo le norme definite in \ref{verifica} e \ref{validazione}. Questo avviene in modo continuo e iterativo.

\subsubsection{Rapporti con la Proponente} La Proponente \Proponente{} fornisce l'indirizzo di posta elettronica del proprio rappresentante, Gregorio Piccoli, attraverso la quale il fornitore può comunicare in via asincrona o pianificare incontri tramite videochiamata.\\
I tipi di comunicazione tra Proponente e fornitore sono funzionali a diversi aspetti del progetto:
\begin{itemize}
  \item Raccolta dei requisiti;
  \item Raccolta di feedback sui risultati conseguiti;
  \item Presentazioni dell'avanzamento del prodotto software.
\end{itemize}
L'approccio adottato dalla Proponente durante il corso del progetto, in quanto \textit{didattico}, è quello di mantenere un ruolo da cliente, affidando al fornitore un grado di libertà ampio da cui consegue una maggiore responsabilità nella realizzazione del prodotto.\\
Ciascun incontro con la Proponente è accompagnato da un Verbale Esterno che ne riassume i punti chiave.


\subsubsection{Documentazione fornita}\label{documentazionefornita}
Di seguito viene descritta la documentazione che il gruppo rende disponibile alla Proponente e ai Committenti.

\paragraph{Piano di Progetto:}
Il \PdP\ è il documento che riguarda la gestione e l'organizzazione del gruppo. Descrive l'analisi e la gestione dei rischi e valuta lo stato di avanzamento del progetto, illustrando la pianificazione, il preventivo e il consuntivo di ciascuno \glossario{sprint}.
Il documento è diviso nelle seguenti sezioni:
\begin{enumerate}
  \item \textbf{Introduzione}: indica lo scopo del documento e del prodotto con eventuali riferimenti normativi e informativi;
  \item \textbf{Analisi dei rischi}: quali sono i rischi ai quali il gruppo può andare incontro, l'impatto di tali rischi sulle risorse del progetto, le strategie di rilevamento e di mitigazione. L'analisi dei rischi viene suddivisa in base alla categoria del rischio:
  \begin{itemize}
    \item Rischi tecnologici;
    \item Rischi organizzativi;
    \item Rischi di natura personale.
  \end{itemize}
  \item \textbf{Modello di sviluppo}: descrive il modello di sviluppo adottato dal gruppo, elencandone le modalità di attuazione e i punti di forza;
  \item \textbf{Stima temporale del progetto}: indica la stima delle date delle revisioni di avanzamento. Viene indicata l'ultima data di aggiornamento della sezione;
  \item \textbf{Stima dei costi}: fornisce una stima preliminare sui costi del progetto. Viene aggiornata a ogni sprint;
  \item \textbf{Pianificazione}: definisce gli obiettivi che il gruppo si impegna a raggiungere. Viene disposta una sezione di pianificazione per ogni sprint;
  \item \textbf{Preventivo}: definisce il preventivo per ciascuno sprint. Il preventivo si compone di:
  \begin{itemize}
    \item Preventivo orario: tabella di distribuzione delle ore preventivate per ciascun membro del gruppo;
    \item Distribuzione ore per la coppia risorsa-ruolo: istogramma che indica le ore preventivate per ciascun ruolo;
    \item Distribuzione ore per ruolo: aereogramma che indica la distribuzione in percentuale preventivata per i ruoli;
    \item Preventivo economico: tabella che riporta il costo del ruolo per il numero di ore preventivate. Riporta il costo totale, ossia il preventivo economico dello sprint.
  \end{itemize}
  \item \textbf{Consuntivo}: dispone il consuntivo per lo \glossario{sprint}. Ogni sprint prevede una sezione dedicata al consuntivo. Il consuntivo si compone di:
  \begin{itemize}
    \item Consuntivo orario: tabella di distribuzione delle ore effettivamente impiegate per ciascun membro del gruppo;
    \item Distribuzione ore per la coppia risorsa-ruolo: istogramma che indica le ore impiegate per ciascun ruolo;
    \item Distribuzione ore per ruolo: aereogramma che indica la distribuzione in percentuale impiegata per i ruoli;
    \item Consuntivo economico: tabella che riporta il costo del ruolo per il numero di ore impiegate. Riporta il totale, ossia il consuntivo economico per lo sprint.
    \item Copertura oraria rispetto al totale: aereogramma che indica la percentuale di tempo speso in rapporto al tempo totale per il completamento del progetto;
    \item Budget speso rispetto al totale: aereogramma che indica la percentuale di budget speso rispetto al budget totale per il completamento del progetto;
    \item Ore rimanenti per la coppia risorsa-ruolo: riporta il numero di ore per ruolo rimanenti ad ogni membro del gruppo. Indica anche le ore rimanenti in totale per membro;
    \item Revisione delle attività: indica le attività svolte durante lo sprint;
    \item Retrospettiva: indica i risultati del questionario che ogni membro compila al termine di uno sprint per valutarne l'andamento. Oltre a ciò, vengono raccolte delle considerazioni aggiuntive sullo sprint, come ad esempio l'affioramento di rischi;
    \item Aggiornamento pianificazione e preventivo: definisce le azioni migliorative per il prossimo sprint e una prima pianificazione dei macro obiettivi da esaminare nella pianificazione futura. Delinea inoltre la gestione dei rischi, ossia i rischi rilevati durante lo sprint, il loro impatto e la loro mitigazione.
  \end{itemize}
\end{enumerate}

\paragraph{Analisi dei Requisiti:}
L'\AdR\ è il documento che tratta l'analisi dei requisiti richiesti dal progetto e individuati dal gruppo. Sviluppa inoltre i casi d'uso per questi, ossia le interazioni tra il sistema e l'utente.
Il documento è diviso nelle seguenti sezioni:
\begin{enumerate}
  \item \textbf{Introduzione}: indica lo scopo del documento e del prodotto con eventuali riferimenti normativi e informativi;
  \item \textbf{Descrizione}: indica le funzioni principali del prodotto e le caratteristiche dell'utente, ossia i motivi che spingono l'utente ad usare l'applicativo;
  \item \textbf{Casi d'uso}: indica tutti i casi d'uso individuati dal gruppo durante l'analisi.
  \item \textbf{Requisiti}: elenco esaustivo dei requisiti del prodotto, indicizzati in base a categorie e fonti da cui proviene il tracciamento.
\end{enumerate}

\paragraph{Piano di Qualifica:}
Il \PdQ\ è il documento che tratta della specifica degli obiettivi di qualità di prodotto e processo. Delinea quindi un insieme di indici di valutazione e validazione del progetto.
Il documento è diviso nelle seguenti sezioni:
\begin{enumerate}
  \item \textbf{Introduzione}: indica lo scopo del documento e del prodotto con eventuali riferimenti normativi e informativi;
  \item \textbf{Obiettivi di qualità}: la sezione illustra i valori accettabili e ambiti per le metriche individuate dal team. Viene divisa per metriche e ogni sotto sezione è dotata di una tabella che descrive le metriche con: ID della metrica, nome, valore tollerabile e valore ambito. Le metriche sono così divise:
  \begin{itemize}
    \item Qualità di processo: indicatori utilizzati per monitorare e valutare la qualità dei processi coinvolti nello sviluppo software;
    \item Qualità di prodotto: valutano in modo obiettivo le caratteristiche quantitative e qualitative del software e assicurano la conformità alle aspettative del cliente;
    \item Qualità per obiettivo: suddivisione delle metriche in base all'obiettivo (fornitura, sviluppo, gestione dei rischi, ecc.).
  \end{itemize}
  \item \textbf{Verifica}: sezione dedicata alla verifica e al collaudo del software. Ha lo scopo di rilevare, correggere e prevenire potenziali errori e difetti. È divisa in sezioni in base alla tipologia di analisi. Le sezioni di test sono dotate di tabelle che descrivono i test con: ID, descrizione e stato del test. Gli strumenti di verifica sono suddivisi in:
  \begin{itemize}
    \item Test di unità: attività di collaudo di singole \glossario{unità} del software;
    \item Test di integrazione: verificano che i diversi moduli, componenti o servizi utilizzati dall’applicazione funzionino in modo integrato;
    \item Test di sistema: controllano il comportamento del sistema nel suo complesso e verificano che l’applicazione funzioni secondo i requisiti specificati;
    \item Test di accettazione: sono test formali che precedono il rilascio del prodotto e valutano se l’applicazione è conforme alle aspettative del cliente;
    \item Checklist: sono strumenti che affiancano il team nell'attività di ispezione. Sono diverse dai test riportati in precedenza poiché la loro tabella è composta solamente da un titolo e una descrizione.
  \end{itemize}
  \item \textbf{Cruscotto di valutazione della qualità}: sezione dedicata alla dimostrazione del rispetto degli standard di qualità. Ciascuna metrica è accompagnata da una rappresentazione grafica che ne misura l'andamento durante gli \glossario{sprint}. Il grafico in questione rileva, oltre all'andamento generale, tre rette che rappresentano il valore ambito, il valore tollerabile e la tendenza. Ogni grafico è dotato di una descrizione che ne esamina l'andamento. Il cruscotto di qualità viene aggiornato periodicamente per allineare le valutazioni agli \glossario{sprint} più recenti.
\end{enumerate}

\paragraph{Lettera di Presentazione}
\par La Lettera di Presentazione è il documento con il quale il gruppo intende candidarsi alla revisione di avanzamento \RTB. Vengono indicati i \glossario{repository} di documentazione e di codice sorgente, assieme all'indirizzo dove è collocato il Proof of Concept. Infine, viene dato l'aggiornamento degli impegni con la stima del preventivo "a finire".

\paragraph{Glossario}
\par Raccolta esaustiva di tutti i termini tecnici utilizzati nella documentazione. Permette di eliminare ambiguità e fraintendimenti, fornendo una definizione univoca ed esaustiva per l'intero gruppo e per chi consulta la documentazione.

\subsubsection{Strumenti}
\IntroStrumenti
\begin{itemize}
  \item LaTeX;
  \item Git;
  \item Zoom;
  \item Google Sheets;
  \item Google Moduli;
  \item Table Convert Online;
  \item PDF24 Tools.
\end{itemize}

\subsubsubsection{Dashboard Google Sheets}
Lo \glossario{spreadsheet} condiviso, realizzato tramite \glossario{Google Sheets} e visibile su \glossario{Google Drive}, ha lo scopo di automatizzare la generazione di preventivi e consuntivi (orari ed economici). Il foglio di calcolo principale è suddiviso nei seguenti fogli interni:
\begin{itemize}
  \item Un foglio contenente le variabili globali del \PdP, inclusi il costo orario di ciascun ruolo e il budget totale;
  \item Un foglio nascosto per ciascun periodo, che include le risposte al questionario di valutazione dello sprint;
  \item Un foglio per ogni sprint con le seguenti informazioni:
  \begin{itemize}
    \item Numero dello \glossario{sprint};
    \item Date di inizio e termine dello \glossario{sprint};
    \item Tabella di assegnazione delle ore produttive per ciascun membro del team, accumulate in totali per persona e per ruolo;
    \item Distribuzione delle ore per ruolo, sotto forma di donut chart;
    \item Distribuzione delle ore per la coppia risorsa-ruolo, sotto forma di grafico a barre sovrapposte (così da poter assumere più ruoli per sprint);
    \item Preventivo economico dello \glossario{sprint};
    \item Tabella riassuntiva con ore e budget spesi e restanti;
    \item Pie chart con la stima delle ore spese sul totale;
    \item Pie chart con la stima del budget sul totale;
    \item Ore rimanenti per la coppia risorsa-ruolo.
  \end{itemize}
\end{itemize}
\vspace{0.5\baselineskip}
\par La dashboard è stata progettata per affiancare il \Responsabile{} nella stesura delle seguenti sezioni del \PdP:
\begin{itemize}
  \item Preventivo;
  \item Consuntivo.
\end{itemize}

\paragraph*{Preventivo}
Il \Responsabile{} può duplicare il foglio di calcolo dello \glossario{sprint} precedente e modificare le seguenti informazioni:
\begin{itemize}
  \item \textbf{Sprint-ID}, dove ID corrisponde al numero dello \glossario{sprint};
  \item I riferimenti temporali, nel formato "dal aaaa-mm-gg al aaaa-mm-gg";
  \item \textbf{Preventivo orario}: per ciascuna coppia risorsa-ruolo, il \Responsabile{} deve inserire nella cella apposita le ore produttive previste;
  \item La tabella "preventivo economico" si aggiorna dinamicamente.
\end{itemize}
\par Per esportare le tabelle in formato \glossario{csv}, il team ha creato un foglio di calcolo aggiuntivo chiamato "Export". Il \Responsabile{} può copiare una tabella e incollarla su questo foglio, tramite le combinazioni di tasti "Ctrl+C" e "Ctrl+Shift+V". Una volta scaricato il file csv, il \Responsabile{} può convertire i dati in \glossario{LaTeX} e inserirli nel \PdP.
\par I grafici sono generati automaticamente e si dividono in due categorie:
\begin{itemize}
  \item \textbf{Grafici a torta 3D}: possono essere scaricati direttamente in formato PDF;
  \item \textbf{Grafici a barre sovrapposte}: devono essere scaricati in SVG e poi convertiti in PDF.
\end{itemize}
\par Il PDF è un formato vettoriale, il che significa che le immagini possono essere scalate senza perdita di qualità. Questo è particolarmente utile per diagrammi e grafici, poiché mantengono la nitidezza anche se ingranditi. Inoltre, il formato PDF è compatibile e facilmente integrabile con \glossario{LaTeX}.
\par Nella sezione rendicontazione ore, il \Responsabile{} deve inserire:
\begin{itemize}
  \item Le date di inizio e fine dello \glossario{sprint};
  \item I ruoli di ciascun componente del team;
  \item Un link al questionario di valutazione dello \glossario{sprint}.
\end{itemize}
\par Il questionario è un modulo \glossario{Google Moduli} che può essere creato all'interno della dashboard. Il questionario ha la seguente struttura:
\begin{itemize}
  \item \textbf{Titolo}: Valutazione Sprint-ID, dove ID corrisponde al numero dello \glossario{sprint};
  \item \textbf{Descrizione}: Questionario per la valutazione dello sprint-ID;
  \item \textbf{Domande} (scelta multipla, scala lineare da 1 a 10, risposta breve, scala lineare da 1 a 5, paragrafo):
  \begin{itemize}
    \item "Come ti è sembrata l'organizzazione dello sprint?";
    \item "Come si potrebbe migliorare la pianificazione?";
    \item "Sapevi sempre cosa fare nel tuo ruolo?";
    \item "Spiega i motivi della risposta precedente (organizzazione, inesperienza, ecc.)";
    \item "Il numero di riunioni è stato adeguato?";
    \item "Le riunioni sono state organizzate con il giusto preavviso?";
    \item "Come ti è sembrata la conduzione dei meeting interni?";
    \item "Come ti è sembrata la conduzione dei meeting esterni?";
    \item "Quanto ti sei impegnato/a in questo sprint?";
    \item "Qual è stato il rapporto ore spese/ore produttive?";
    \item "Quali azioni correttive avvieresti dal prossimo sprint?".
  \end{itemize}
\end{itemize}
\par Dopo aver creato un nuovo modulo, il \Responsabile{} può utilizzare la funzione "Importa domande". Questa feature consente di importare quesiti da un modulo esistente. Cliccando il pulsante "Invia", è possibile inoltre copiare il link da incollare nel foglio di calcolo. Nella dashboard \glossario{Google Sheets} viene aggiunto in automatico un foglio contenente le risposte al questionario; i fogli degli sprint precedenti possono essere nascosti.

\paragraph*{Consuntivo}
Tutte le tabelle del consuntivo vengono aggiornate automaticamente in base alla rendicontazione delle ore. Di seguito sono riporate le tre tabelle che compongono il consuntivo:
\begin{itemize}
  \item \textbf{Consuntivo orario}: il team ha definito una formula dinamica che somma le ore produttive per la coppia risorsa-ruolo. In questo modo è possibile automatizzare il calcolo del consuntivo anche quando i membri del team assumono più ruoli. La somma delle ore produttive per la coppia risorsa-ruolo è arrotondata per difetto;
  \item \textbf{Ore rimanenti} per la coppia risorsa-ruolo: viene calcolata la differenza tra le ore rimanenti al termine dello \glossario{sprint} precedente e le ore impiegate nello \glossario{sprint} attuale;
  \item \textbf{Consuntivo economico}, formato dai seguenti campi:
  \begin{itemize}
    \item Ruolo;
    \item Ore per ruolo;
    \item Delta ore preventivo - consuntivo: differenza tra le ore preventivate e quelle effettivamente spese;
    \item Costo (in €);
    \item Delta costo preventivo - consuntivo: differenza tra il costo preventivato e quello effettivo.
    \item Ore e budget spesi negli \glossario{sprint} precedenti;
    \item Ore e budget restanti.
  \end{itemize}
\end{itemize}
\par Il processo di esportazione di tabelle e grafici segue le stesse regole del preventivo. Tutte le tabelle e i grafici del consuntivo devono essere inseriti nel \PdP. Una volta completata la stesura del consuntivo nel \PdP, il \Responsabile{} deve aggiornare le variabili globali nel foglio "Pdp-global":
\begin{itemize}
  \item \textbf{Ultimo-Sprint}: ID, dove ID è il numero dell'ultimo \glossario{sprint};
  \item \textbf{Preventivo complessivo} (da modificare qualora sia necessaria una ridistribuzione delle ore per ruolo):
  \begin{itemize}
    \item Ruolo;
    \item Ore per ruolo;
    \item Ore individuali;
    \item Costo orario (in €);
    \item Costo totale (in €);
    \item Ore e budget restanti, ricavati dal consuntivo economico dell'ultimo sprint.
  \end{itemize}
\end{itemize}
\par Se il preventivo complessivo dovesse mutare, sia la tabella che il grafico corrispondente andrebbero aggiornati nel \PdP.

\paragraph*{Rendicontazione ore} Ciascun foglio di calcolo dello \glossario{sprint} include una sezione dedicata alla rendicontazione delle ore. La tabella è organizzata come segue:
\begin{itemize}
  \item Data;
  \item Membro del team:
  \begin{itemize}
    \item Ore produttive;
    \item Ruolo;
    \item Descrizione delle attività.
  \end{itemize}
\end{itemize}
