\subsection{Miglioramento}
\subsubsection{Scopo}
\par Il processo di miglioramento si occupa di stabilire, misurare, controllare e migliorare i processi di ciclo di vita del software.

\subsubsection{Attività}
\par Il processo si compone delle seguenti attività:
\begin{itemize}
  \item Definizione del processo;
  \item Valutazione del processo;
  \item Miglioramento del processo.
\end{itemize}

\subsubsection{Definizione del processo}
\par Vengono riconosciuti i processi da implementare e ne vengono documentate le norme stabilite dal \WoW. Quando opportuno, il team implementa un meccanismo di controllo per monitorare e migliorare il processo.

\subsubsection{Valutazione del proceso}
\par Il gruppo pianifica ed esegue revisioni dei processi a intervalli regolari corrispondenti alle iterazioni di progetto, o più strettamente quando il processo richiede controllo costante (ad esempio durante le prime fasi di implementazione).

\par Le revisioni hanno l'obiettivo di assicurare l'efficacia del processo, individuando gli aspetti migliorabili per l'avvicinamento allo stato dell'arte.

\subsubsection{Miglioramento del processo}
\par Il gruppo effettua miglioramenti al processo evidenziati e riconosciuti durante la valutazione. Le norme inerenti al processo vengono aggiornate in modo da comprendere gli aspetti individuati.

\par Informazioni di storico, tecniche o di valutazione vengono collettivamente raccolte e analizzate per riconoscere punti di forza e aspetti deboli dei processi impiegati. Il risultato dell'analisi viene impiegato come feedback per il miglioramento del processo.

\subsubsection{Strumenti}
\begin{itemize}
  \item Jira;
  \item GitHub;
  \item LaTeX;
  \item Discord;
  \item Telegram.
\end{itemize}