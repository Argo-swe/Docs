\section{Riunione}
\subsection{Ordine del giorno}
\begin{itemize}
	\item Retrospettiva \glossario{sprint} 11;
	\item Pianificazione \glossario{sprint} 12.
\end{itemize}

\subsection{Discussione e decisioni}
\subsubsection{Retrospettiva sprint 11}
\par All'inizio della riunione, il team ha esaminato i risultati dello \glossario{sprint} appena concluso, raccogliendo i feedback di ciascun membro. Il gruppo ha giudicato soddisfacente sia la pianificazione delle attività che il livello di collaborazione interna. Nonostante l'elevato numero di task, il carico di lavoro è stato ritenuto adeguato rispetto alle risorse disponibili. Successivamente, il team ha chiarito i dubbi relativi alla progettazione e al diagramma delle classi, affrontando nello specifico i seguenti punti:
\begin{itemize}
	\item Relazioni tra le classi;
	\item Rappresentazione delle istanze delle classi;
	\item Convenzioni interne per la scrittura dei nomi di attributi e metodi;
	\item Suddivisione del diagramma in tre sezioni:
	\begin{itemize}
		\item Incoming;
		\item Core;
		\item Outcoming.
	\end{itemize}
\end{itemize}

\vspace{0.5\baselineskip}
\par In seguito, il gruppo ha corretto alcuni segmenti ambigui del codice \glossario{back-end} per mantenere l'uniformità con il diagramma delle classi. Allo stesso tempo, sono stati individuati e analizzati potenziali difetti nella struttura del back-end. Questo ha portato all'apertura di ticket mirati alla revisione e all'ottimizzazione del codice sorgente.

\subsubsection{Pianificazione sprint 12}
\par Al termine della retrospettiva, e dopo aver verificato lo stato di avanzamento del progetto, il team ha identificato le seguenti attività come prioritarie (vedi \sezione{sec:todo}):
\begin{itemize}
	\item Avanzamento codifica \glossario{back-end};
	\item Avanzamento test di unità e di integrazione back-end;
	\item Ultimazione test dei componenti \glossario{front-end};
	\item Ampliamento della sezione relativa alla progettazione di dettaglio nel documento di \ST;
	\item Revisione del codice sorgente.
\end{itemize}