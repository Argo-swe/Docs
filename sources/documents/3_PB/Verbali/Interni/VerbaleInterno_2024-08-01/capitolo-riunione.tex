\section{Riunione}
\subsection{Ordine del giorno}
\begin{itemize}
	\item Suddivisione delle attività tra i progettisti;
	\item Organizzazione di un incontro con la \glossario{Proponente};
	\item Discussione dei punti per il diario di bordo.
\end{itemize}

\subsection{Discussione e decisioni}
\subsubsection{Suddivisione delle attività tra i progettisti}
\par All'inizio della riunione, il team ha discusso le convenzioni da utilizzare per la nomenclatura dei file e dei componenti \glossario{Vue}. I nomi dei componenti devono essere composti da più parole. Per mantenere la coerenza tra componenti e file, la nomenclatura a più parole verrà adottata anche per i nomi dei file. I prefissi da utilizzare sono:
\begin{itemize}
	\item App, Base o Site per i componenti globali;
	\item Cmp per i componenti che non sono parte integrante della struttura centrale dell'applicazione.
\end{itemize}

\vspace{0.5\baselineskip}
\par I suffissi da utilizzare sono:
\begin{itemize}
	\item View per le viste dell'applicazione.
\end{itemize}

\vspace{0.5\baselineskip}
\par In seguito, il team ha valutato i \glossario{framework} di testing per il back-end, basandosi su una relazione in formato PDF redatta dal progettista. Il framework \glossario{pytest} è stato preferito a \glossario{unittest} per la sua maggiore potenza, flessibilità e predisposizione ai test parametrizzati. Pertanto, sono state fisste le seguenti attività (vedi \sezione{sec:todo}):
\begin{itemize}
	\item Aggiornamento del file \textit{requirements.txt}: aggiungere pytest e rimuovere parameterized;
	\item Sperimentazione dei test unitari con pytest tramite esempi concreti.
\end{itemize}

\vspace{0.5\baselineskip}
\par Inoltre, sono state assegnate le seguenti attività ai progettisti (vedi \sezione{sec:todo}):
\begin{itemize}
	\item Progettazione del \glossario{database}: diagramma ER (Entità-Relazione);
	\item Valutazione delle architetture per il \glossario{back-end}: i risultati delle analisi devono essere documentati in un file PDF;
	\item Definizione ad alto livello dei componenti, o route, del back-end.
\end{itemize}

\vspace{0.5\baselineskip}
\par Infine, il team ha programmato i seguenti task relativi alla stesura del \MU{} (vedi \sezione{sec:todo}):
\begin{itemize}
	\item Suddivisione delle istruzioni tra Utente e Tecnico;
	\item Istruzioni per l'interazione con il ChatBOT;
	\item Istruzioni per la gestione dei \glossario{dizionari dati};
	\item Istruzioni per la configurazione del sistema.
\end{itemize}

\subsubsection{Organizzazione di un incontro con la Proponente}
\par Il gruppo ha organizzato una riunione con la \glossario{Proponente} per comunicare l'esito della revisione \glossario{RTB} ed esaminare i requisiti del progetto. L'incontro è stato pianificato per il 6-7 agosto. In preparazione all'incontro, il team ha deciso di sviluppare tre nuovi \glossario{dizionari dati} per facilitare l'esecuzione dei test.

\subsubsection{Discussione dei punti per il diario di bordo}
\par Il gruppo ha discusso i punti salienti da trattare nel diario di bordo del 2 agosto.
\paragraph{Done}
\begin{itemize}
	\item Presentazione \glossario{RTB} con il Professor Vardanega;
	\item Completata progettazione logica \glossario{front-end};
	\item Inizio progettazione di dettaglio, codifica e test di unità front-end;
	\item Definizione struttura del \MU;
	\item Valutazione delle possibili architetture per il \glossario{back-end};
	\item Progettazione iniziale del \glossario{database};
	\item Correzioni alla documentazione post colloquio RTB;
	\item Scelta degli strumenti di test;
	\item Sperimentazione dei test di unità per il back-end.
\end{itemize}

\paragraph{Todo}
\begin{itemize}
	\item Scelta definitiva dell'architettura per il \glossario{back-end};
	\item Definizione dei componenti back-end;
	\item Inizio progettazione di dettaglio back-end;
	\item Avanzamento progettazione di dettaglio \glossario{front-end};
	\item Avanzamento codifica front-end;
	\item Consuntivo \glossario{sprint} 10;
	\item Aggiornamento delle \NdP\ (sezione codifica e test di unità);
	\item Aggiornamento del \PdQ.
\end{itemize}

\vspace{0.5\baselineskip}
\par Gli argomenti sopracitati sono stati riportati nella presentazione per il diario di bordo del 2 agosto.