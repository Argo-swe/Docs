\subsection{Altre metriche del codice}

\subsubsection*{Metriche del codice - Ultimo aggiornamento: 2024-09-19}

\begin{table}[H]
  \centering
  \begin{tabularx}{\textwidth}{|P{3.5cm}|*{5}{>{\centering\arraybackslash}X|}}
      \hline
      Metriche del codice & \textbf{Sprint 10} & \textbf{Sprint 11} & \textbf{Sprint 12} & \textbf{Sprint 13} & \textbf{Sprint 14} \\
      \hline
      \textbf{M.PD.15 - Duplicazione del codice} & 3\% - 5\% & < 3\% & < 3\% & < 3\% & < 3\% \\
      \hline
      \textbf{M.PD.16 - Indice di manutenibilità} & B & A & A & A & A \\
      \hline
      \textbf{M.PD.7 - Click Depth} & 1 & 1 & 1 & 1 & 1 \\
      \hline
      \textbf{M.PD.8 - Ampiezza} & 3 & 2 & 2 & 2 & 2 \\
      \hline
      \textbf{M.PD.10 - Tempo di risposta } & 9 & 7 & 7 & 7 & 7 \\
      \hline
  \end{tabularx}
  \caption{Metriche del codice}
\end{table}

\par Durante il primo sprint della \glossario{PB}, la quantità di codice ripetuto all'interno del progetto è rimasta inferiore al 5\%, agevolando il lavoro di refactoring. Con l'implementazione del modello architetturale scelto, il gruppo ha raggiunto l'obiettivo desiderato, portando la duplicazione del codice a meno del 3\%. La struttura del progetto e la definizione di molteplici interfacce hanno permesso al team di distribuire le responsabilità, minimizzando le ridondanze. A partire dallo sprint 11, SonarCloud ha assegnato al gruppo una A come indice di manutenibilità. Questo risultato dimostra che le scelte progettuali e architetturali del team sono state conformi agli standard qualitativi previsti.

\par Per migliorare l'esperienza utente, il team ha mantenuto una click depth pari a 1, rendendo più semplice la navigazione tra le pagine. L'ampiezza del menu di navigazione è stata fissata a 2, con l'obiettivo di ridurre il sovraccarico cognitivo. Questi valori rispettano il modello di una struttura "ampia e poco profonda".

\par Il tempo medio di risposta oscilla tra 5 e 10 secondi. Questo valore tiene in considerazione il tempo necessario per la creazione degli \glossario{indici} e i ritardi dovuti alla \glossario{ricerca semantica}. Il miglioramento tra lo sprint 10 e 11 è attribuibile all'implementazione del modello architetturale e all'introduzione di accorgimenti per migliorare le prestazioni.