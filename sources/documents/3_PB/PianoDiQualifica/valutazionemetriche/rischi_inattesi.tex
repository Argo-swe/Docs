\subsection{M.PC.11 - Rischi inattesi}
\begin{figure}[H]
    \centering
    \includegraphics[width=\textwidth]{assets/rischi_inattesi.pdf}
    \caption{M.PC.11 - Rischi inattesi}
\end{figure}

\par Il grafico evidenzia l’inesperienza iniziale del team nell’individuare i rischi che possono emergere durante lo svolgimento di un progetto software. Nei primi tre \glossario{sprint}, infatti, il gruppo ha dovuto affrontare almeno un rischio inatteso. Ciononostante, il numero di rischi imprevisti è rimasto entro i limiti della soglia tollerabile. A partire dal quarto sprint, i rischi che si sono verificati erano già stati analizzati e documentati nel \PdP. Attraverso un’analisi più consapevole, una collaborazione stretta tra i membri e una comunicazione trasparente, il team ha mantenuto il numero di eventi imprevisti stabile e prossimo al valore ambito. L'unica eccezione è stata il quinto sprint, durante il quale è emerso un rischio inatteso legato al cambio di tecnologie. Nonostante il gruppo avesse previsto una possibile transizione e avesse testato diversi \glossario{framework} alternativi, l’entità del lavoro risultante ha superato le risorse disponibili, prolungando le scadenze prefissate. Per migliorare la gestione del progetto, il gruppo ha convenuto di discutere e monitorare i rischi durante le riunioni interne, fornendo al responsabile una base solida per la stesura del \PdP. 

\par Nel corso della \glossario{PB}, il team ha mantenuto una gestione stabile dei rischi. Nonostante l'elevato numero di attività da svolgere, inclusa la configurazione dei test automatici, il gruppo non ha rilevato la comparsa di rischi inattesi.
