\subsection{Test di integrazione}

\par La sezione relativa ai test di integrazione verrà aggiornata in seguito alla revisione \glossario{RTB}.

\bgroup
\begin{adjustwidth}{-0.5cm}{-0.5cm}
	% MAX 12.5cm
 	\begin{longtable}{|P{1.5cm}|>{\raggedright}P{9.5cm}|>{\arraybackslash}P{1.5cm}|}
		\caption{Test di integrazione}
  	\label{tab:test-integrazione} \\
	  \hline
		\textbf{ID} & \textbf{Descrizione} & \textbf{Stato} \\ 
		\hline
		\endfirsthead

		\caption[]{Test di integrazione (continua)} \\
		\hline
		\textbf{ID} & \textbf{Descrizione} & \textbf{Stato} \\ 
		\hline
		\endhead

		\hline
		\multicolumn{3}{|r|}{{Continua nella prossima pagina}} \\ 
		\hline
		\endfoot

		\hline
		\endlastfoot

		TU & Verificare che il componente "AppFooter" soddisfi le seguenti condizioni:
    \begin{itemize}
      \item Il logo deve essere renderizzato correttamente.
    \end{itemize} & S \\
    \hline TU & Verificare che il componente "AppMenu" soddisfi le seguenti condizioni:
    \begin{itemize}
      \item L'Utente deve poter accedere alle seguenti pagine:
      \begin{itemize}
        \item Chat.
      \end{itemize}
      \item Il Tecnico deve poter accedere alle seguenti pagine:
      \begin{itemize}
        \item Chat;
        \item Gestione dizionari dati.
      \end{itemize}
    \end{itemize} & S \\
    \hline TU & Verificare che il componente "AppTopbar" soddisfi le seguenti condizioni:
    \begin{itemize}
      \item Il logo deve essere renderizzato correttamente.
    \end{itemize} & S \\
    \hline TU & Verificare che il componente "MenuSidebar" soddisfi le seguenti condizioni:
    \begin{itemize}
      \item Il menu di navigazione principale deve essere visibile;
      \item Il footer deve essere visibile.
    \end{itemize} & S \\
    \hline TU & Verificare che il componente "StringDataModal" soddisfi le seguenti condizioni:
    \begin{itemize}
      \item Il messaggio di debug deve essere visualizzato correttamente quando la variabile StringData è impostata;
      \item Il messaggio di debug non deve essere renderizzato quando la variabile StringData è vuota.
    \end{itemize} & S \\
	\end{longtable}
\end{adjustwidth}
\egroup