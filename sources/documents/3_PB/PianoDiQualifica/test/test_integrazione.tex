\subsection{Test di integrazione}

\par La sezione relativa ai test di integrazione verrà aggiornata in seguito alla revisione \glossario{RTB}.

\bgroup
\begin{adjustwidth}{-0.5cm}{-0.5cm}
	% MAX 12.5cm
 	\begin{longtable}{|P{1.5cm}|>{\raggedright}P{9.5cm}|>{\arraybackslash}P{1.5cm}|}
		\caption{Test di integrazione}
  	\label{tab:test-integrazione} \\
	  \hline
		\textbf{ID} & \textbf{Descrizione} & \textbf{Stato} \\ 
		\hline
		\endfirsthead

		\caption[]{Test di integrazione (continua)} \\
		\hline
		\textbf{ID} & \textbf{Descrizione} & \textbf{Stato} \\ 
		\hline
		\endhead

		\hline
		\multicolumn{3}{|r|}{{Continua nella prossima pagina}} \\ 
		\hline
		\endfoot

		\hline
		\endlastfoot

		%authentication_service
		\hline TU & Verificare che il metodo "login" della classe "AuthenticationService" soddisfi le seguenti condizioni:
		\begin{itemize}
			\item Deve restituire lo status "NOT_FOUND" se l'utente con il nome utente fornito non esiste;
			\item Deve restituire lo status "BAD_CREDENTIAL" se la password fornita è errata, indicando il messaggio di errore appropriato;
			\item Deve restituire lo status "OK" e un token JWT quando il login è corretto, utilizzando il nome utente e la password corretti.
		\end{itemize} & S \\

		%dictionary_service
		% get_dictionary_list
		\hline TU & Verificare che il metodo "get_dictionary_list" della classe "DictionaryService" soddisfi le seguenti condizioni:
		\begin{itemize}
			\item Deve restituire una lista di dizionari esistenti con lo status "OK".
		\end{itemize} & S \\

		% get_dictionary_by_id
		\hline TU & Verificare che il metodo "get_dictionary_by_id" della classe "DictionaryService" soddisfi le seguenti condizioni:
		\begin{itemize}
			\item Deve restituire un dizionario esistente in base all'ID con lo status "OK";
			\item Deve restituire lo status "NOT_FOUND" se il dizionario con l'ID specificato non esiste.
		\end{itemize} & S \\

		% get_dictionary_file
		\hline TU & Verificare che il metodo "get_dictionary_file" della classe "DictionaryService" soddisfi le seguenti condizioni:
		\begin{itemize}
			\item Deve restituire il contenuto del file di un dizionario esistente;
			\item Deve restituire "None" se il dizionario con l'ID specificato non esiste.
		\end{itemize} & S \\

		% get_dictionary_preview
		\hline TU & Verificare che il metodo "get_dictionary_preview" della classe "DictionaryService" soddisfi le seguenti condizioni:
		\begin{itemize}
			\item Deve restituire un'anteprima del dizionario esistente con lo status "OK";
			\item Deve restituire lo status "NOT_FOUND" se il dizionario con l'ID specificato non esiste.
		\end{itemize} & S \\

		% create_dictionary
		\hline TU & Verificare che il metodo "create_dictionary" della classe "DictionaryService" soddisfi le seguenti condizioni:
		\begin{itemize}
			\item Deve creare un nuovo dizionario quando non esiste già un dizionario con lo stesso nome;
			\item Deve restituire lo status "CONFLICT" se un dizionario con lo stesso nome esiste già;
			\item Deve restituire lo status "BAD_REQUEST" se il contenuto fornito è mancante o non valido.
		\end{itemize} & S \\

		% update_dictionary_metadata
		\hline TU & Verificare che il metodo "update_dictionary_metadata" della classe "DictionaryService" soddisfi le seguenti condizioni:
		\begin{itemize}
			\item Deve aggiornare correttamente i metadati di un dizionario esistente con lo status "OK";
			\item Deve restituire lo status "CONFLICT" se esiste già un dizionario con lo stesso nome;
			\item Deve restituire lo status "NOT_FOUND" se il dizionario con l'ID specificato non esiste.
		\end{itemize} & S \\

		% update_dictionary_file
		\hline TU & Verificare che il metodo "update_dictionary_file" della classe "DictionaryService" soddisfi le seguenti condizioni:
		\begin{itemize}
			\item Deve aggiornare correttamente il file di un dizionario esistente con lo status "OK";
			\item Deve restituire lo status "BAD_REQUEST" se il contenuto del file è non valido;
			\item Deve restituire lo status "NOT_FOUND" se il dizionario con l'ID specificato non esiste.
		\end{itemize} & S \\

		% delete_dictionary
		\hline TU & Verificare che il metodo "delete_dictionary" della classe "DictionaryService" soddisfi le seguenti condizioni:
		\begin{itemize}
			\item Deve eliminare correttamente un dizionario esistente con lo status "OK";
			\item Deve restituire lo status "NOT_FOUND" se il dizionario con l'ID specificato non esiste.
		\end{itemize} & S \\


    	%prompt_manager_service
		\hline TU & Verificare che il metodo "generate_prompt" della classe "PromptManagerService" soddisfi le seguenti condizioni:
		\begin{itemize}
			\item Deve restituire lo status "NOT_FOUND" e il messaggio di errore appropriato se il dizionario non viene trovato;
			\item Deve restituire lo status "BAD_REQUEST" se la query fornita è mancante;
			\item Deve generare correttamente il prompt quando viene fornita una query valida, restituendo lo status "OK" e il prompt generato.
		\end{itemize} & S \\

		\hline TU & Verificare che il metodo "generate_prompt_with_debug" della classe "PromptManagerService" soddisfi le seguenti condizioni:
		\begin{itemize}
			\item Deve restituire lo status "NOT_FOUND" e il messaggio di errore appropriato se il dizionario non viene trovato;
			\item Deve restituire lo status "BAD_REQUEST" se la query fornita è mancante;
			\item Deve generare correttamente il prompt e includere informazioni di debug quando viene fornita una query valida, restituendo lo status "OK" e il prompt generato con le informazioni di debug.
		\end{itemize} & S \\

    TU & Verificare che il componente "AppMenu" soddisfi le seguenti condizioni:
    \begin{itemize}
      \item L'Utente deve poter accedere alle seguenti pagine:
      \begin{itemize}
        \item Chat.
      \end{itemize}
      \item Il Tecnico deve poter accedere alle seguenti pagine:
      \begin{itemize}
        \item Chat;
        \item Gestione dizionari dati.
      \end{itemize}
    \end{itemize} & S \\
    \hline TU & Verificare che il componente "AppTopbar" soddisfi le seguenti condizioni:
    \begin{itemize}
      \item Il logo deve essere renderizzato correttamente.
    \end{itemize} & S \\
    \hline TU & Verificare che il componente "AppFooter" soddisfi le seguenti condizioni:
    \begin{itemize}
      \item Il logo deve essere renderizzato correttamente.
    \end{itemize} & S \\
    \hline TU & Verificare che il componente "MenuSidebar" soddisfi le seguenti condizioni:
    \begin{itemize}
      \item Il menu di navigazione principale deve essere visibile;
      \item Il footer deve essere visibile.
    \end{itemize} & S \\
    \hline TU & Verificare che il componente "StringDataModal" soddisfi le seguenti condizioni:
    \begin{itemize}
      \item Il messaggio di debug deve essere visualizzato correttamente quando la variabile StringData è impostata;
      \item Il messaggio di debug non deve essere renderizzato quando la variabile StringData è vuota.
    \end{itemize} & S \\
    \hline TU & Verificare che il componente "AppLayout" soddisfi le seguenti condizioni:
    \begin{itemize}
      \item Il layout dell'applicazione deve essere renderizzato correttamente;
      \item Il wrapper globale deve contenere le classi corrette sulla base delle impostazioni e dello stato del layout;
      \item Il dialog di login deve essere aperto/chiuso al clic dell'apposito pulsante;
      \item La barra laterale delle impostazioni deve essere aperta/chiusa al clic dell'apposito pulsante;
      \item La barra laterale delle impostazioni deve essere chiusa se l'Utente clicca al di fuori di essa;
      \item La barra laterale del menu di navigazione deve essere aperta/chiusa al clic dell'apposito pulsante;
      \item La barra laterale del menu di navigazione deve essere chiusa se l'Utente clicca al di fuori di essa.
    \end{itemize} & S \\
    \hline TU & Verificare che il componente "ChatView" soddisfi le seguenti condizioni:
    \begin{itemize}
      \item L'anteprima del dizionario dati selezionato deve essere visualizzata al clic dell'apposito pulsante;
      \item L'anteprima del dizionario deve essere nascosta quando viene inviato un messaggio;
      \item Quando viene aperta l'anteprima del dizionario, i messaggi della chat devono essere nascosti;
      \item L'anteprima del dizionario deve essere nascosta al clic dell'apposito pulsante;
      \item Quando viene chiusa l'anteprima del dizionario, i messaggi della chat devono essere resi nuovamente visibili;
      \item Il pulsante di debug deve essere visualizzato se la richiesta è stata inviata dal Tecnico;
      \item I messaggi della chat devono essere cancellati al clic dell'apposito pulsante.
    \end{itemize} & S \\
	\end{longtable}
\end{adjustwidth}
\egroup