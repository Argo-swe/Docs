\subsection{Test di accettazione}

\par L'obiettivo dei test di accettazione è verificare se il sistema soddisfa le aspettative del Committente e del Proponente. I test di accettazione determinano se il software è pronto per essere rilasciato, e pertanto richiedono un focus sul comportamento degli utenti. Di seguito è riportato l'elenco dei test di accettazione:

\bgroup
\begin{adjustwidth}{-0.5cm}{-0.5cm}
	% MAX 12.5cm
 	\begin{longtable}{|P{1.5cm}|>{\raggedright}P{9.5cm}|>{\arraybackslash}P{1.5cm}|}
    \caption{Test di accettazione}
  	\label{tab:test-accettazione} \\
	  \hline
		\textbf{ID} & \textbf{Descrizione} & \textbf{Stato} \\
		\hline
		\endfirsthead

    \caption[]{Test di accettazione (continua)} \\
		\hline
		\textbf{ID} & \textbf{Descrizione} & \textbf{Stato} \\
		\hline
		\endhead

		\hline
		\multicolumn{3}{|r|}{{Continua nella prossima pagina}} \\
		\hline
		\endfoot

		\hline
		\endlastfoot

		TA.1 & Verificare che l'Utente possa effettuare il login:
		\begin{enumerate}
			\item Avviare la procedura di autenticazione (da qualsiasi pagina);
			\item Inserire uno username;
			\item Inserire una password;
			\item Richiedere l'accesso;
			\item Visualizzare un messaggio di conferma una volta effettuato il login.
		\end{enumerate}
		& S \\
		\hline TA.2 & Verificare che il Tecnico possa salvare un nuovo \glossario{dizionario dati} nel sistema:
		\begin{enumerate}
			\item Accedere alla pagina di gestione dei dizionari dati;
			\item Avviare la procedura di inserimento di un nuovo dizionario;
			\item Inserire il nome del dizionario;
			\item Inserire la descrizione del dizionario;
			\item Caricare il file \glossario{JSON} associato al dizionario;
			\item Confermare l'inserimento;
			\item Visualizzare un messaggio di conferma una volta che il dizionario è stato salvato.
		\end{enumerate}
		& S \\
		\hline TA.3 & Verificare che il Tecnico possa visualizzare la lista dei dizionari dati caricati nel sistema:
		\begin{enumerate}
			\item Visualizzare la lista dei dizionari disponibili;
			\item Visualizzare le caratteristiche di ciascun dizionario presente nella lista.
		\end{enumerate}
		& S \\
		\hline TA.4 & Verificare che il Tecnico possa eliminare un dizionario dati dal sistema:
		\begin{enumerate}
			\item Accedere alla pagina di gestione dei dizionari;
			\item Visualizzare la lista dei dizionari caricati nel sistema;
			\item Scegliere un dizionario tra quelli disponibili;
			\item Richiedere l'eliminazione del dizionario;
			\item Confermare la decisione;
			\item Visualizzare un messaggio di conferma una volta che il dizionario è stato eliminato.
		\end{enumerate}
		& S \\
		\hline TA.5 & Verificare che il Tecnico possa modificare il nome di un dizionario dati:
		\begin{enumerate}
			\item Accedere alla pagina di gestione dei dizionari;
			\item Visualizzare la lista dei dizionari caricati nel sistema;
			\item Scegliere un dizionario tra quelli disponibili;
			\item Modificare il nome del dizionario;
			\item Confermare la modifica;
			\item Visualizzare un messaggio di conferma una volta effettuata la modifica.
		\end{enumerate}
		& S \\
		\hline TA.6 & Verificare che il Tecnico possa modificare la descrizione di un dizionario dati:
		\begin{enumerate}
			\item Accedere alla pagina di gestione dei dizionari;
			\item Visualizzare la lista dei dizionari caricati nel sistema;
			\item Scegliere un dizionario tra quelli disponibili;
			\item Modificare la descrizione del dizionario;
			\item Confermare la modifica;
			\item Visualizzare un messaggio di conferma una volta effettuata la modifica.
		\end{enumerate}
		& S \\
		\hline TA.7 & Verificare che il Tecnico possa modificare il file di configurazione di un dizionario dati:
		\begin{enumerate}
			\item Accedere alla pagina di gestione dei dizionari;
			\item Visualizzare la lista dei dizionari caricati nel sistema;
			\item Scegliere un dizionario tra quelli disponibili;
			\item Caricare un nuovo file \glossario{JSON};
			\item Confermare la sostituzione;
			\item Visualizzare un messaggio di conferma una volta effettuata la modifica.
		\end{enumerate}
		& S \\
		\hline TA.8 & Verificare che il Tecnico possa scaricare un \glossario{dizionario dati}:
		\begin{enumerate}
			\item Accedere alla pagina di gestione dei dizionari;
			\item Visualizzare la lista dei dizionari caricati nel sistema;
			\item Scegliere un dizionario tra quelli disponibili;
			\item Richiedere il download del file;
			\item Visualizzare il file scaricato.
		\end{enumerate}
		& S \\
		\hline TA.9 & Verificare che il Tecnico possa effettuare correttamente il logout. & S \\
		\hline TA.10 & Verificare che l'Utente possa effettuare una ricerca per nome tra i dizionari dati:
		\begin{enumerate}
			\item Accedere alla chat o alla pagina di gestione dei dizionari;
			\item Visualizzare la lista dei dizionari disponibili;
			\item Richiedere al sistema di effettuare una ricerca per nome;
			\item Visualizzare i risultati della ricerca.
		\end{enumerate}
		& S \\
		\hline TA.11 & Verificare che il Tecnico possa effettuare una ricerca per descrizione tra i dizionari dati:
		\begin{enumerate}
			\item Accedere alla pagina di gestione dei dizionari;
			\item Visualizzare la lista dei dizionari disponibili;
			\item Richiedere al sistema di effettuare una ricerca per descrizione;
			\item Visualizzare i risultati della ricerca.
		\end{enumerate}
		& S \\
		\hline TA.12 & Verificare che l'Utente possa selezionare un dizionario dati da utilizzare nell'applicazione. & S \\
		\hline TA.13 & Verificare che l'Utente possa visualizzare un'anteprima del dizionario dati selezionato. & S \\
		\hline TA.14 & Verificare che l'Utente possa ottenere un \glossario{prompt} in risposta a un'interrogazione in linguaggio naturale:
		\begin{enumerate}
			\item Accedere alla chat;
			\item Inserire una richiesta in linguaggio naturale;
			\item Inviare il messaggio;
			\item Visualizzare il prompt generato dal sistema.
		\end{enumerate}
		& S \\
		\hline TA.15 & Verificare che l'Utente possa copiare il prompt generato:
		\begin{enumerate}
			\item Accedere alla chat;
			\item Visualizzare il prompt generato;
			\item Richiedere al sistema di copiare il contenuto del prompt;
			\item Visualizzare un messaggio di conferma una volta che il prompt è stato copiato negli appunti.
		\end{enumerate}
		& S \\
		\hline TA.16 & Verificare che l'Utente possa visualizzare il contenuto della chat:
		\begin{enumerate}
			\item Accedere alla chat;
			\item Visualizzare la lista dei messaggi;
			\item Visualizzare un singolo messaggio nella lista;
			\item Visualizzare il mittente del messaggio;
			\item Visualizzare il contenuto del messaggio.
		\end{enumerate}
		& S \\
		\hline TA.17 & Verificare che l'Utente possa eliminare la cronologia della chat. & S \\
		\hline TA.18 & Verificare che il Tecnico possa visualizzare il messaggio di \glossario{debug} elaborato durante la generazione del prompt. & S \\
		\hline TA.19 & Verificare che il Tecnico possa scaricare un file di \glossario{log}. & S \\
		\hline TA.20 & Verificare che il ChatBOT restituisca un messaggio esplicativo se la richiesta inserita dall'Utente non ha prodotto risultati rilevanti. & S \\
	\end{longtable}
\end{adjustwidth}
\egroup

\clearpage
\subsubsection{Tracciamento dei test di accettazione}

\bgroup
\begin{adjustwidth}{-0.5cm}{-0.5cm}
	\centering
	% MAX 12.5cm
  \begin{longtable}{|c|c|}
		\caption{Tracciamento test di accettazione}
  	\label{tab:tracciamento-test-accettazione} \\
    \hline
		\textbf{ID} & \textbf{Caso d'uso} \\
		\hline
		\endfirsthead

		\caption[]{Tracciamento test di accettazione (continua)} \\
		\hline
		\textbf{ID} & \textbf{Caso d'uso} \\
		\hline
		\endhead

		\hline
		\multicolumn{2}{|r|}{{Continua nella prossima pagina}} \\
		\hline
		\endfoot

		\hline
		\endlastfoot

    TA.1 & UC1, UC1.1, UC1.2\\
		\hline TA.2 & UC13, UC15, UC16, UC17\\
		\hline TA.3 & UC9, UC9.1, UC10, UC10.1, UC10.2, UC10.3\\
		\hline TA.4 & UC18\\
		\hline TA.5 & UC29\\
		\hline TA.6 & UC30\\
		\hline TA.7 & UC20\\
		\hline TA.8 & UC38\\
		\hline TA.9 & UC12\\
		\hline TA.10 & UC36\\
		\hline TA.11 & UC37\\
		\hline TA.12 & UC4\\
		\hline TA.13 & UC14, UC14.1, UC14.2, UC14.3, UC14.3.1, UC14.3.1.1, UC14.3.1.2\\
		\hline TA.14 & UC3, UC5, UC7, UC26\\
		\hline TA.15 & UC8\\
		\hline TA.16 & UC25, UC25.1, UC25.1.1, UC25.1.2\\
		\hline TA.17 & UC27\\
		\hline TA.18 & UC22\\
		\hline TA.19 & UC23\\
		\hline TA.20 & UC6\\
  \end{longtable}
\end{adjustwidth}
\egroup
