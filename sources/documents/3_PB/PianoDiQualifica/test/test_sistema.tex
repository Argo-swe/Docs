\subsection{Test di sistema}

\par I test di sistema devono assicurare una completa copertura dei requisiti concordati con la \glossario{Proponente} e/o specificati nel documento di \AdR. Di seguito è riportato l'elenco dei test di sistema:

\bgroup
\begin{adjustwidth}{-0.5cm}{-0.5cm}
	% MAX 12.5cm
 	\begin{longtable}{|P{1.5cm}|>{\raggedright}P{9.5cm}|>{\arraybackslash}P{1.5cm}|}
		\caption{Test di sistema}
  	\label{tab:test-sistema} \\
	  \hline
		\textbf{ID} & \textbf{Descrizione} & \textbf{Stato} \\ 
		\hline
		\endfirsthead

		\caption[]{Test di sistema (continua)} \\
		\hline
		\textbf{ID} & \textbf{Descrizione} & \textbf{Stato} \\ 
		\hline
		\endhead

		\hline
		\multicolumn{3}{|r|}{{Continua nella prossima pagina}} \\ 
		\hline
		\endfoot

		\hline
		\endlastfoot

		TS.1 & Verificare che l'Utente possa effettuare il login. & N-D \\
		\hline TS.2 & Verificare che l'Utente visualizzi un errore qualora inserisca delle credenziali errate in fase di autenticazione. & N-D \\
		\hline TS.3 & Verificare che il Tecnico possa inserire un nuovo \glossario{dizionario dati} nel sistema. & N-D \\ 
		\hline TS.4 & Verificare che il Tecnico possa modificare il nome di un dizionario dati. & N-D \\ 
		\hline TS.5 & Verificare che il Tecnico possa modificare la descrizione di un dizionario dati. & N-D \\
		\hline TS.6 & Verificare che il Tecnico possa modificare il file di configurazione di un dizionario dati. & N-D \\
		\hline TS.7 & Verificare che il Tecnico visualizzi un errore nel caso in cui il nome del dizionario dati contenga caratteri non supportati. & N-D \\ 
		\hline TS.8 & Verificare che il sistema restituisca un messaggio di errore qualora il Tecnico inserisca un nome già esistente per il dizionario dati. & N-D \\ 
		\hline TS.9 & Verificare che il Tecnico visualizzi un errore nel caso in cui la descrizione del dizionario dati contenga caratteri non supportati. & N-D \\ 
		\hline TS.10 & Verificare che il sistema restituisca un messaggio di errore qualora il Tecnico inserisca un dizionario dati con una dimensione superiore a 1 MB. & N-D \\
		\hline TS.11 & Verificare che il Tecnico visualizzi un errore nel caso in cui l'estensione del file caricato sia diversa da .json. & N-D \\ 
		\hline TS.12 & Verificare che il sistema restituisca un messaggio di errore qualora il Tecnico inserisca un dizionario dati non conforme allo schema predefinito. & N-D \\  
		\hline TS.13 & Verificare che il Tecnico possa eliminare un dizionario dati dal sistema. & N-D \\ 
		\hline TS.14 & Verificare che il sistema restituisca un messaggio di errore nel caso in cui l'eliminazione del dizionario dati abbia esito negativo. & N-D \\  
		\hline TS.15 & Verificare che il Tecnico possa scaricare un dizionario dati. & N-D \\
		\hline TS.16 & Verificare che l'Utente possa accedere alla chat e inserire un messaggio nella maschera di richiesta. & N-D \\   
		\hline TS.17 & Verificare che l'Utente possa selezionare un dizionario dati e renderlo operativo nel sistema. & N-D \\ 
		\hline TS.18 & Verificare che l'Utente possa visualizzare il contenuto del dizionario dati selezionato. Il sistema deve mostrare le seguenti informazioni:
		\begin{itemize}
			\item Nome del database;
			\item Descrizione del database;
			\item Lista delle tabelle del database. Per ciascuna tabella devono essere riportate le seguenti informazioni:
				\begin{itemize}
					\item Nome della tabella;
					\item Descrizione della tabella.
				\end{itemize}
		\end{itemize} & N-D \\  
		\hline TS.19 & Verificare che l'Utente possa inviare una richiesta al ChatBOT e ottenere il \glossario{prompt} risultante. & N-D \\ 
		\hline TS.20 & Verificare che il sistema restituisca un avviso qualora l'Utente inserisca una richiesta ritenuta non idonea dal modello di \glossario{AI}. & N-D \\
		\hline TS.21 & Verificare che il sistema restituisca un messaggio di errore nel caso in cui la generazione del prompt venga interrotta senza preavviso. & N-D \\ 
		\hline TS.22 & Verificare che l'Utente possa visualizzare correttamente il prompt generato. Il prompt deve contenere le seguenti informazioni:
		\begin{itemize}
			\item Lista delle tabelle pertinenti. Ogni tabella deve essere corredata da:
			\begin{itemize}
				\item Schema della tabella: composto dal nome della tabella e da una lista di colonne, a loro volte organizzate per nome e tipo (es.: integer, string);
				\item Chiave primaria;
				\item Descrizione della tabella;
				\item Descrizione delle colonne della tabella;
				\item Chiavi esterne;
			\end{itemize}
			\item DBMS di riferimento;
			\item Lingua di riferimento;
			\item Richiesta in linguaggio naturale.
		\end{itemize} & N-D \\
		\hline TS.23 & Verificare che il Tecnico possa effettuare il logout per terminare la sessione corrente. & N-D \\ 
		\hline TS.24 & Verificare che l'Utente possa copiare il contenuto del prompt generato. & N-D \\ 
		\hline TS.25 & Verificare che il sistema generi un \glossario{log} se la richiesta viene inviata dal Tecnico. & N-D \\ 
		\hline TS.26 & Verificare che il Tecnico possa copiare il messaggio di \glossario{debug} relativo alla richiesta inviata. & N-D \\ 
		\hline TS.27 & Verificare che il Tecnico possa scaricare un file di log contenente il debug del prompt. & N-D \\
		\hline TS.28 & Verificare che il sistema restituisca un messaggio di errore nel caso in cui il download di un file fallisca. & N-D \\  
		\hline TS.29 & Verificare che l'Utente possa visualizzare correttamente il contenuto della chat. & N-D \\  
		\hline TS.30 & Verificare che il Tecnico possa visualizzare i dizionari dati con le relative informazioni:
		\begin{itemize}
			\item Nome del dizionario dati;
			\item Estensione del file;
			\item Dimensione del file;
			\item Descrizione del dizionario dati;
			\item Data di ultimo aggiornamento.
		\end{itemize} & N-D \\  
		\hline TS.31 & Verificare che il sistema di generazione del \glossario{prompt} supporti richieste in lingue diverse dall'inglese. Di seguito sono riportate le lingue che devono essere verificate: 
		\begin{itemize}
			\item Italiano;
			\item Francese;
			\item Tedesco;
			\item Spagnolo. 
		\end{itemize}
		& N-D \\  
	\end{longtable}
\end{adjustwidth}
\egroup

\clearpage
\subsubsection{Tracciamento dei test di sistema}

\bgroup
\begin{adjustwidth}{-0.5cm}{-0.5cm}
	\centering
	% MAX 12.5cm
  \begin{longtable}{|c|c|}
		\caption{Tracciamento test di sistema}
  	\label{tab:tracciamento-test-sistema} \\
    \hline
		\textbf{ID} & \textbf{Requisito} \\ 
		\hline
		\endfirsthead

		\caption[]{Tracciamento test di sistema (continua)} \\
		\hline
		\textbf{ID} & \textbf{Requisito} \\ 
		\hline
		\endhead

		\hline
		\multicolumn{2}{|r|}{{Continua nella prossima pagina}} \\ 
		\hline
		\endfoot

		\hline
		\endlastfoot

    TS.1 & RF.O.1, RF.O.1.1, RF.O.1.2\\
		\hline TS.2 & R.F.O.2\\
		\hline TS.3 & RF.O.13, RF.O.15, RF.O.16, RF.O.17\\
		\hline TS.4 & RF.O.29\\
		\hline TS.5 & RF.O.30\\
		\hline TS.6 & RF.O.15, RF.O.20\\
		\hline TS.7 & RF.O.31, RF.O.31.1\\
		\hline TS.8 & RF.O.31, RF.O.31.2\\
		\hline TS.9 & RF.O.32\\
		\hline TS.10 & RF.O.33\\
		\hline TS.11 & RF.O.28\\
		\hline TS.12 & RF.O.34\\
		\hline TS.13 & RF.O.18\\
		\hline TS.14 & RF.O.21\\
		\hline TS.15 & RF.O.37\\
		\hline TS.16 & RF.O.5\\
		\hline TS.17 & RF.O.4\\
		\hline TS.18 & RF.O.14, RF.O.14.1, RF.O.14.2, RF.O.14.3, RF.O.14.3.1, RF.O.14.3.1.1, RF.O.14.3.1.2\\
		\hline TS.19 & RF.O.3\\
		\hline TS.20 & RF.O.6\\
		\hline TS.21 & RF.O.11\\
		\hline TS.22 & RF.O.35\\
		\hline TS.23 & RF.O.12\\
		\hline TS.24 & RF.O.8\\
		\hline TS.25 & RF.O.46\\
		\hline TS.26 & RF.O.38\\
		\hline TS.27 & RF.O.23\\
		\hline TS.28 & RF.O.24\\
		\hline TS.29 & RF.O.25, RF.O.25.1, RF.O.25.1.1, RF.O.25.1.2\\
		\hline TS.30 & RF.O.9, RF.O.9.1, RF.O.10, RF.O.10.1, RF.O.10.2, RF.O.10.3, RF.O.10.4, RF.O.10.5\\
		\hline TS.31 & RF.D.52, RF.D.7\\
  \end{longtable}
\end{adjustwidth}
\egroup