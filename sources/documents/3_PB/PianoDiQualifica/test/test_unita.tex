\subsection{Test di unità}

\par Lo scopo dei test di unità è verificare il corretto funzionamento delle "unità software", ossia delle porzioni o segmenti (come una funzione, una classe o un componente) testabili in modo autonomo e isolato all'interno del sistema. Di seguito è riportato l'elenco dei test di unità:

\bgroup
\begin{adjustwidth}{-0.5cm}{-0.5cm}
	% MAX 12.5cm
 	\begin{longtable}{|P{1.5cm}|>{\raggedright}P{9.5cm}|>{\arraybackslash}P{1.5cm}|}
		\caption{Test di unità}
  	\label{tab:test-unita} \\
	  \hline
		\textbf{ID} & \textbf{Descrizione} & \textbf{Stato} \\
		\hline
		\endfirsthead

		\caption[]{Test di unità (continua)} \\
		\hline
		\textbf{ID} & \textbf{Descrizione} & \textbf{Stato} \\
		\hline
		\endhead

		\hline
		\multicolumn{3}{|r|}{{Continua nella prossima pagina}} \\
		\hline
		\endfoot

		\hline
		\endlastfoot

		%json_file_adapter
		\hline TU & Verificare che il metodo "create" della classe "FileFactory" soddisfi le seguenti condizioni:
		\begin{itemize}
			\item Deve creare correttamente un'istanza di "JsonFileAdapter" quando il tipo di file specificato nella configurazione è "json", verificando che l'istanza sia anche un'implementazione di "FileRepository";
			\item Deve sollevare un'eccezione "ValueError" se il tipo di file specificato nella configurazione è sconosciuto, con un messaggio di errore che indichi il tipo di file non riconosciuto.
		\end{itemize} & S \\

		%json_schema_validator_adapter
		\hline TU & Verificare che il metodo "validate" della classe "JsonSchemaValidatorAdapter" soddisfi le seguenti condizioni:
		\begin{itemize}
			\item Deve restituire True se lo schema è validato secondo il modello di confronto per la validazione;
			\item Deve restituire False se lo schema non è validato secondo il modello di confronto per la validazione;
			\item Deve sollevare un'eccezione se non viene trovato il file con il modello di confronto per la validazione.
		\end{itemize} & S \\

		%sql_alchemy_authentication_repository_adapter
		\hline TU & Verificare che il metodo "get\_user\_by\_username" della classe "SqlAlchemyAuthenticationRepositoryAdapter" soddisfi le seguenti condizioni:
		\begin{itemize}
			\item Deve recuperare correttamente un utente esistente nel database in base al nome utente, assicurando che la query venga eseguita sul modello "Admins" con il filtro appropriato;
			\item Deve restituire "None" se l'utente con il nome utente specificato non esiste nel database, verificando che la query venga eseguita correttamente sul modello "Admins" con il filtro appropriato.
		\end{itemize} & S \\

		%sql_alchemy_db_manager_factory
		\hline TU & Verificare che il metodo "create\_authentication\_repository" della classe "SqlAlchemyDbManagerFactory" soddisfi le seguenti condizioni:
		\begin{itemize}
			\item Deve creare e restituire correttamente un'istanza di "SqlAlchemyAuthenticationRepositoryAdapter";
			\item Deve sollevare un'eccezione se si verifica un errore durante la creazione del repository di autenticazione;
			\item Deve garantire che il tipo di repository restituito sia corretto, ossia un'istanza di "SqlAlchemyAuthenticationRepositoryAdapter".
		\end{itemize} & S \\

		\hline TU & Verificare che il metodo "create\_dictionary\_repository" della classe "SqlAlchemyDbManagerFactory" soddisfi le seguenti condizioni:
		\begin{itemize}
			\item Deve creare e restituire correttamente un'istanza di "SqlAlchemyDictionaryRepositoryAdapter";
			\item Deve sollevare un'eccezione se si verifica un errore durante la creazione del repository di dizionari;
			\item Deve creare una nuova istanza del repository di dizionari ad ogni chiamata, verificando che più chiamate al metodo restituiscano istanze distinte;
			\item Deve garantire che il tipo di repository restituito sia corretto, ossia un'istanza di "SqlAlchemyDictionaryRepositoryAdapter".
		\end{itemize} & S \\

		\hline TU & Verificare che l'inizializzazione della classe "SqlAlchemyDbManagerFactory" soddisfi le seguenti condizioni:
		\begin{itemize}
			\item Deve chiamare il metodo "create\_all" durante l'inizializzazione per assicurare che tutte le tabelle del database siano create, utilizzando l'engine corretto.
		\end{itemize} & S \\

		%sql_alchemy_dictionary_respository_adapter
		\hline TU & Verificare che il metodo "create\_dictionary" della classe "SqlAlchemyDictionaryRepositoryAdapter" soddisfi le seguenti condizioni:
		\begin{itemize}
			\item Deve creare correttamente un nuovo dizionario e salvarlo nel database, verificando che le proprietà del dizionario siano impostate correttamente e che le operazioni di add, commit e refresh siano chiamate;
			\item Deve sollevare un'eccezione se si verifica un errore durante il commit del dizionario nel database.
		\end{itemize} & S \\

		\hline TU & Verificare che il metodo "update\_dictionary" della classe "SqlAlchemyDictionaryRepositoryAdapter" soddisfi le seguenti condizioni:
		\begin{itemize}
			\item Deve aggiornare correttamente le proprietà di un dizionario esistente e salvare le modifiche nel database, verificando che le operazioni di commit e refresh siano chiamate;
			\item Deve sollevare un'eccezione se si verifica un errore durante il commit delle modifiche al dizionario nel database;
			\item Deve sollevare un'eccezione "AttributeError" se si tenta di aggiornare un dizionario inesistente.
		\end{itemize} & S \\

		\hline TU & Verificare che il metodo "delete\_dictionary" della classe "SqlAlchemyDictionaryRepositoryAdapter" soddisfi le seguenti condizioni:
		\begin{itemize}
			\item Deve eliminare correttamente un dizionario esistente dal database, verificando che le operazioni di delete e commit siano chiamate;
			\item Non deve tentare di eliminare un dizionario inesistente, verificando che le operazioni di delete e commit non siano chiamate;
			\item Deve sollevare un'eccezione se si verifica un errore durante l'eliminazione di un dizionario nel database e assicurarsi che commit non sia chiamato;
			\item Deve garantire che l'operazione di commit venga sempre eseguita dopo l'eliminazione di un dizionario.
		\end{itemize} & S \\

		\hline TU & Verificare che il metodo "get\_all\_dictionaries" della classe "SqlAlchemyDictionaryRepositoryAdapter" soddisfi le seguenti condizioni:
		\begin{itemize}
			\item Deve restituire correttamente tutti i dizionari presenti nel database, verificando che la query sia eseguita sul modello "Dictionaries" e che il metodo all() venga chiamato.
		\end{itemize} & S \\

		\hline TU & Verificare che il metodo "get\_dictionary\_by\_id" della classe "SqlAlchemyDictionaryRepositoryAdapter" soddisfi le seguenti condizioni:
		\begin{itemize}
			\item Deve restituire correttamente un dizionario in base all'ID fornito, verificando che la query sia eseguita sul modello "Dictionaries" e che il filtro venga applicato correttamente.
		\end{itemize} & S \\

		\hline TU & Verificare che il metodo "get\_dictionary\_by\_name" della classe "SqlAlchemyDictionaryRepositoryAdapter" soddisfi le seguenti condizioni:
		\begin{itemize}
			\item Deve restituire correttamente un dizionario in base al nome fornito, verificando che la query sia eseguita sul modello "Dictionaries" e che il filtro venga applicato correttamente.
		\end{itemize} & S \\

		%db_manager_factory
		\hline TU & Verificare che il metodo "create" della classe "SqlAlchemyDbManagerFactory" soddisfi le seguenti condizioni:
		\begin{itemize}
			\item Deve creare e restituire un'istanza di "SqlAlchemyDbManagerFactory" quando la configurazione specifica l'uso di SQLAlchemy come DB manager;
			\item Deve sollevare un'eccezione "ValueError" se la configurazione contiene un tipo di DB manager sconosciuto, con un messaggio di errore che indichi il tipo sconosciuto;
			\item Deve sollevare un'eccezione "ValueError" se la configurazione manca della chiave "db\_manager", con un messaggio di errore che indichi l'assenza di tale chiave.
		\end{itemize} & S \\

		%txtai_prompt_manager_adapter
		\hline TU & Verificare che il metodo "prompt\_generator" della classe "TxtaiPromptManagerAdapter" soddisfi le seguenti condizioni:
		\begin{itemize}
			\item Deve generare correttamente il prompt includendo la richiesta dell'utente e i metadati dello schema, quando vengono trovati risultati rilevanti;
			\item Deve restituire un messaggio di errore specifico quando non vengono trovati risultati rilevanti;
			\item Deve generare e restituire correttamente il log delle fasi 1 e 2 se la registrazione dei log è attivata;
			\item Deve gestire correttamente il caso in cui la richiesta dell'utente è vuota, restituendo un messaggio appropriato;
			\item Deve generare il prompt nella lingua specificata, riflettendo la configurazione della lingua;
			\item Deve generare il prompt adattato al DBMS specificato, riflettendo la configurazione del DBMS;
			\item Deve gestire correttamente i fallimenti durante l'estrazione dei metadati dallo schema, restituendo un messaggio di errore appropriato o sollevando un'eccezione.
		\end{itemize} & S \\

		\hline TU & Verificare che il metodo "get\_tuples" della classe "TxtaiPromptManagerAdapter" soddisfi le seguenti condizioni:
		\begin{itemize}
			\item Deve formare correttamente la query SQL e eseguire la ricerca restituendo le tuple corrispondenti;
			\item Deve attivare la registrazione del log quando richiesto e restituire il contenuto del log generato;
			\item Deve gestire correttamente i risultati di ricerca vuoti, restituendo una lista vuota senza contenuto di log.
		\end{itemize} & S \\

		\hline TU & Verificare che il metodo "get\_relevant\_tuples" della classe "TxtaiPromptManagerAdapter" soddisfi le seguenti condizioni:
		\begin{itemize}
			\item Deve mantenere le tuple con punteggi elevati, indipendentemente dalla distanza di punteggio;
			\item Deve mantenere o scartare le tuple in base alla distanza di punteggio rispetto alla tupla precedente;
			\item Deve generare correttamente il contenuto del log quando la registrazione è attivata;
			\item Deve gestire correttamente una lista vuota di tuple, restituendo nessuna tupla rilevante e nessun contenuto di log;
			\item Deve restituire correttamente una singola tupla come rilevante.
		\end{itemize} & S \\

		%txtai_index_manager_adapter
		\hline TU & Verificare che il metodo "get\_embeddings" della classe "TxtaiIndexManagerAdapter" soddisfi le seguenti condizioni:
		\begin{itemize}
			\item Deve restituire un'istanza della classe "Embeddings".
		\end{itemize} & S \\

		\hline TU & Verificare che il metodo "create\_or\_load\_index" della classe "TxtaiIndexManagerAdapter" soddisfi le seguenti condizioni:
		\begin{itemize}
			\item Deve caricare un indice esistente e restituire "False" se l'indice esiste già;
			\item Deve creare un nuovo indice e restituire "True" se l'indice non esiste.
		\end{itemize} & S \\

		\hline TU & Verificare che il metodo "create\_index" della classe "TxtaiIndexManagerAdapter" soddisfi le seguenti condizioni:
		\begin{itemize}
			\item Deve creare un indice utilizzando i documenti estratti e salvarlo correttamente se "save\_index" è impostato a "True";
			\item Deve creare un indice senza salvarlo se "save\_index" è impostato a "False";
			\item Deve creare un indice anche se la lista dei documenti è vuota e comunque salvare l'indice se "save\_index" è impostato a "True";
			\item Deve sollevare un'eccezione se si verifica un errore durante l'indicizzazione e non deve salvare l'indice.
		\end{itemize} & S \\

		\hline TU & Verificare che il metodo "save\_index" della classe "TxtaiIndexManagerAdapter" soddisfi le seguenti condizioni:
		\begin{itemize}
			\item Deve salvare correttamente l'indice utilizzando il percorso del file corretto;
			\item Deve sollevare un'eccezione se si verifica un errore durante il salvataggio dell'indice.
		\end{itemize} & S \\

		\hline TU & Verificare che il metodo "load\_index" della classe "TxtaiIndexManagerAdapter" soddisfi le seguenti condizioni:
		\begin{itemize}
			\item Deve caricare correttamente l'indice utilizzando il percorso del file corretto.
		\end{itemize} & S \\

		\hline TU & Verificare che il metodo "delete\_index" della classe "TxtaiIndexManagerAdapter" soddisfi le seguenti condizioni:
		\begin{itemize}
			\item Deve eliminare correttamente la directory dell'indice se esiste;
			\item Non deve tentare di eliminare la directory dell'indice se questa non esiste.
		\end{itemize} & S \\

		%txtai_embeddings_manager_factory
		\hline TU & Verificare che il metodo "create\_index\_manager" della classe "TxtaiEmbeddingsManagerFactory" soddisfi le seguenti condizioni:
		\begin{itemize}
			\item Deve creare e restituire un'istanza di "TxtaiIndexManagerAdapter" quando viene fornita una configurazione valida;
			\item Deve generare un'eccezione "KeyError" se la configurazione fornita manca di chiavi obbligatorie, e il messaggio dell'errore deve indicare le chiavi mancanti;
			\item Deve generare un'eccezione "KeyError" durante l'inizializzazione della factory se manca la chiave "txtai" nella configurazione, e il messaggio dell'errore deve menzionare la chiave mancante;
			\item Deve restituire correttamente un'istanza di "TxtaiIndexManagerAdapter" anche se la configurazione manca della chiave "txtai".
		\end{itemize} & S \\

		\hline TU & Verificare che il metodo "create\_prompt\_manager\_with\_dependencies" della classe "TxtaiEmbeddingsManagerFactory" soddisfi le seguenti condizioni:
		\begin{itemize}
			\item Deve creare e restituire un'istanza di "TxtaiPromptManagerAdapter" quando viene fornita una configurazione valida e un'istanza di "TxtaiIndexManagerAdapter" come dipendenza.
		\end{itemize} & S \\

		\hline TU & Verificare che il metodo "create\_prompt\_manager" della classe "TxtaiEmbeddingsManagerFactory" soddisfi le seguenti condizioni:
		\begin{itemize}
			\item Deve creare e restituire un'istanza di "TxtaiPromptManagerAdapter" quando viene fornita una configurazione valida e gestisce internamente la creazione del "TxtaiIndexManagerAdapter".
		\end{itemize} & S \\

		%txtai_debug_manager
		\hline TU & Verificare che il metodo "semantic\_search\_log\_custom\_algorithm" della classe "TxtaiDebugManagerAdapter" soddisfi le seguenti condizioni:
		\begin{itemize}
			\item Deve restituire un log con il messaggio "No tables found." quando vengono passate liste vuote di tuple;
			\item Deve restituire un log che confermi che tutte le tabelle sono mantenute se il loro punteggio è superiore o uguale a 0.45;
			\item Deve restituire un log che indichi quali tabelle sono mantenute e quali sono scartate quando alcune tabelle hanno un punteggio inferiore a 0.45;
			\item Deve mantenere tabelle con un punteggio inferiore a 0.45 se la differenza di punteggio con la tabella precedente è inferiore a 0.25.
		\end{itemize} & S \\

		\hline TU & Verificare che il metodo "semantic\_search\_log" della classe "TxtaiDebugManagerAdapter" soddisfi le seguenti condizioni:
		\begin{itemize}
			\item Deve restituire un log appropriato quando viene passata una lista vuota di tuple, includendo l'indicazione che non sono state trovate tabelle rilevanti;
			\item Deve restituire un log dettagliato della ricerca semantica quando viene passata una singola tupla, includendo i termini più rilevanti;
			\item Deve restituire un log corretto quando viene passata una lista con più tuple, riportando i dettagli e i termini rilevanti per ogni descrizione.
		\end{itemize} & S \\

		%json_file_adapter
		\hline TU & Verificare che il metodo "save" della classe "JsonFileAdapter" soddisfi le seguenti condizioni:
		\begin{itemize}
			\item Deve apritre un file con permessi di scrittura;
			\item Deve eseguire almeno un'operazione di scrittura sul file aperto.
		\end{itemize} & S \\

		\hline TU & Verificare che il metodo "load" della classe "JsonFileAdapter" soddisfi le seguenti condizioni:
		\begin{itemize}
			\item Deve restituire il path del file di schema.
		\end{itemize} & S \\

		\hline TU & Verificare che il metodo "delete" della classe "JsonFileAdapter" soddisfi le seguenti condizioni:
		\begin{itemize}
			\item Deve eseguire l'operazione di verifica di presenza del file;
			\item Deve eseguire l'operazione di eliminazione del file.
		\end{itemize} & S \\

		\hline TU & Verificare che il metodo "get\_preview" della classe "JsonFileAdapter" soddisfi le seguenti condizioni:
		\begin{itemize}
			\item Deve restituire None se non è presente lo schema;
			\item Deve restituire un oggetto che rappresenti lo schema se presente.
		\end{itemize} & S \\

		\hline TU & Verificare che il metodo "extract\_index\_metadata" della classe "JsonFileAdapter" soddisfi le seguenti condizioni:
		\begin{itemize}
			\item Deve restituire l'indicizzzazione di tutte le colonne per ogni tabella.
		\end{itemize} & S \\

		\hline TU & Verificare che il metodo "extract\_schema\_metadata" della classe "JsonFileAdapter" soddisfi le seguenti condizioni:
		\begin{itemize}
			\item Deve restituire la rappresentazione delle tabelle facenti parte dello schema e delle chiavi esterne.
		\end{itemize} & S \\

		%componenti front-end
		\hline TU & Verificare che il componente "AppLogo" soddisfi le seguenti condizioni:
    \begin{itemize}
      \item Il logo deve essere renderizzato con il percorso dell'immagine corretto;
			\item L'altezza e la larghezza dell'immagine devono essere impostate con valori predefiniti se non vengono passate come props;
			\item Il colore dell'immagine deve cambiare in base al tema selezionato.
    \end{itemize} & S \\
		\hline TU & Verificare che il componente "AppMenu" soddisfi le seguenti condizioni:
    \begin{itemize}
      \item Il menu dell'Utente deve avere avere una o più opzioni;
			\item Il menu del Tecnico deve avere due o più opzioni.
    \end{itemize} & S \\
		\hline TU & Verificare che il componente "AppMenuItem" soddisfi le seguenti condizioni:
    \begin{itemize}
      \item La voce di menu deve contenere l'etichetta corretta;
			\item La voce di menu non deve essere attiva se non è stata selezionata;
			\item Una volta selezionata, la voce di menu deve essere attiva;
			\item La voce di menu può essere la radice di un menu;
			\item La voce di menu può contenere un sotto-menu;
			\item La voce di menu può aprire o chiudere un sotto-menu.
    \end{itemize} & S \\
		\hline TU & Verificare che il componente "AppTopbar" soddisfi le seguenti condizioni:
    \begin{itemize}
      \item Il pulsante di login deve essere visibile se l'Utente non ha effettuato l'accesso;
			\item Il pulsante di logout deve essere visibile se l'Utente ha effettuato l'accesso;
			\item Quando viene cliccato il pulsante di logout, la funzione di logout deve essere chiamata;
			\item Il menu delle opzioni deve essere aperto al clic dell'apposito pulsante;
			\item Il menu delle opzioni deve essere chiuso quando l'Utente clicca al di fuori di esso.
    \end{itemize} & S \\
		\hline TU & Verificare che il componente "AppFooter" soddisfi le seguenti condizioni:
    \begin{itemize}
      \item Il footer deve essere renderizzato correttamente;
			\item All'interno del footer deve essere visualizzata la versione corretta dell'applicazione.
    \end{itemize} & S \\
		\hline TU & Verificare che il componente "ConfigSidebar" soddisfi le seguenti condizioni:
    \begin{itemize}
      \item La sezione per modificare la scala deve essere visibile;
			\item La sezione per modificare il tema deve essere visibile;
			\item La sezione per modificare la lingua dell'interfaccia deve essere visibile;
			\item La scala deve essere diminuita al clic del'apposito pulsante;
			\item La scala deve essere aumentata al clic dell'apposito pulsante;
			\item Il pulsante per diminuire la scala deve essere disabilitato se la scala è al minimo;
			\item Il pulsante per aumentare la scala deve essere disabilitato se la scala è al massimo;
			\item Il contenuto del localStorage deve essere aggiornato al cambio del tema;
			\item Il contenuto del localStorage deve essere aggiornato al cambio della lingua.
    \end{itemize} & S \\
		\hline TU & Verificare che il componente "MenuSidebar" soddisfi le seguenti condizioni:
    \begin{itemize}
			\item Il pulsante per chiudere la sidebar deve essere nascosto su schermi di grandi dimensioni;
      \item Il pulsante per chiudere la sidebar deve essere visibile su schermi di piccole dimensioni.
    \end{itemize} & S \\
		\hline TU & Verificare che il componente "LoginDialog" soddisfi le seguenti condizioni:
    \begin{itemize}
      \item Il dialog deve essere chiuso al clic dell'apposito pulsante;
			\item La funzione "messageSuccess" deve essere chiamata se il login viene effettuato con successo;
			\item Una volta completato il login, il dialog deve essere chiuso;
			\item Il contenuto del localStorage deve essere aggiornato se il login viene effettuato con successo;
			\item Una volta completato il login, il form deve essere resettato;
			\item La funzione "messageError" deve essere chiamata se il login fallisce.
    \end{itemize} & S \\
		\hline TU & Verificare che il componente "DictPreview" soddisfi le seguenti condizioni:
    \begin{itemize}
      \item L'anteprima del dizionario dati deve essere nascosta se la variabile DetailsVisible è uguale a false;
			\item Se è visibile, l'anteprima deve mostrare i dati corretti;
			\item Lo stato di espansione dell'anteprima deve essere attivato/disattivato al clic dell'apposito pulsante;
			\item Quando viene cliccato il pulsante di chiusura, il componente deve inviare un segnale.
    \end{itemize} & S \\
		\hline TU & Verificare che il componente "ChatMessage" soddisfi le seguenti condizioni:
    \begin{itemize}
      \item Il messaggio inviato dall'Utente deve essere visualizzato correttamente;
			\item Il messaggio inviato dal ChatBOT deve essere visualizzato correttamente;
			\item I pulsanti di azione devono essere nascosti se il messaggio è stato inviato dall'Utente;
			\item La funzione "Copia negli appunti" deve essere chiamata con i parametri corretti al clic dell'apposito pulsante;
			\item Il pulsante di debug deve essere nascosto se l'Utente non ha effettuato il login;
			\item La funzione "Apri debug modal" deve essere chiamata con i parametri corretti al clic dell'apposito pulsante.
    \end{itemize} & S \\
		\hline TU & Verificare che il componente "DebugMessage" soddisfi le seguenti condizioni:
    \begin{itemize}
      \item Il componente deve visualizzare il messaggio di debug correttamente;
			\item Quando viene cliccato il pulsante "Scarica file", la funzione di download deve essere chiamata con i parametri corretti.
    \end{itemize} & S \\
		\hline TU & Verificare che il componente "ChatDeleteBtn" soddisfi le seguenti condizioni:
    \begin{itemize}
      \item Il pulsante deve essere disabilitato se non è presente alcun messaggio nella chat;
      \item Il pulsante deve essere disabilitato se il ChatBOT sta elaborando un nuovo messaggio;
      \item Quando viene cliccato, il pulsante deve inviare un segnale.
    \end{itemize} & S \\
		\hline TU & Verificare che il componente "CreateUpdateDictionaryModal" soddisfi le seguenti condizioni:
    \begin{itemize}
      \item Il form di creazione e modifica deve essere visualizzato correttamente;
			\item Il pulsante di invio deve essere disabilitato se il form è incompleto;
			\item La funzione "messageSuccess" deve essere chiamata se l'inserimento viene effettuato con successo;
			\item La funzione "messageError" deve essere chiamata se l'inserimento fallisce;
			\item Il pulsante di invio deve essere disabilitato se il nome del dizionario dati contiene caratteri non supportati;
			\item Il pulsante di invio deve essere disabilitato se il formato del file non è valido;
			\item La funzione "messageSuccess" deve essere chiamata se l'aggiornamento dei metadati viene effettuato con successo;
			\item La funzione "messageError" deve essere chiamata se l'aggiornamento dei metadati fallisce;
			\item La funzione "messageSuccess" deve essere chiamata se l'aggiornamento del file viene effettuato con successo;
			\item La funzione "messageError" deve essere chiamata se l'aggiornamento del file fallisce;
			\item L'upload del file deve essere disabilitato se un file è già stato caricato;
			\item Il file selezionato deve essere rimosso al clic dell'apposito pulsante.
    \end{itemize} & S \\
		\hline TU & Verificare che il componente "DictionariesListView" soddisfi le seguenti condizioni:
    \begin{itemize}
      \item La lista dei dizionari dati deve essere visualizzata correttamente;
			\item La funzione "Apri dictionary modal" deve essere chiamata con i parametri corretti al clic del pulsante di inserimento;
			\item La funzione "Apri dictionary modal" deve essere chiamata con i parametri corretti al clic del pulsante di modifica dei metadati;
			\item La funzione "Apri dictionary modal" deve essere chiamata con i parametri corretti al clic del pulsante di modifica del file;
			\item Quando viene cliccato il pulsante "Scarica file", la funzione di download deve essere chiamata con i parametri corretti;
			\item La funzione "messageSuccess" deve essere chiamata se l'eliminazione viene effettuata con successo;
			\item La funzione "messageError" deve essere chiamata se l'eliminazione fallisce.
    \end{itemize} & S \\
		\hline TU & Verificare che il componente "ChatView" soddisfi le seguenti condizioni:
    \begin{itemize}
      \item Il contenuto della chat deve essere renderizzato correttamente;
			\item Il form di selezione pre-richiesta deve essere attivato/disattivato al click dell'apposito pulsante;
			\item Nella chat deve essere visualizzato il dizionario dati attivo con la relativa estensione;
			\item Quando un dizionario dati viene selezionato, il contenuto del localStorage deve essere aggiornato;
			\item Il pulsante di invio deve essere disattivato se non è stato selezionato alcun dizionario dati;
			\item Il pulsante di invio deve essere disattivato se non è stata inserita alcuna richiesta;
			\item La richiesta deve poter essere inviata premendo il tasto "Invio";
			\item La funzione di generazione del prompt deve essere chiamata con i parametri corretti al clic del pulsante di invio;
			\item Quando il ChatBOT ritorna una risposta, il contenuto del sessionStorage deve essere aggiornato;
			\item Quando viene inviata una richiesta, il campo di testo deve essere resettato;
			\item La funzione "messageError" deve essere chiamata se la generazione del prompt fallisce.
    \end{itemize} & S \\
		\hline TU & Verificare che il metodo "logout" della classe "auth.service" soddisfi le seguenti condizioni:
		\begin{itemize}
			\item Il metodo deve rimuovere il token di autenticazione dal localStorage;
			\item Il metodo deve inviare un segnale di notifica.
    \end{itemize} & S \\
		\hline TU & Verificare che il metodo "isLogged" della classe "auth.service" soddisfi le seguenti condizioni:
		\begin{itemize}
			\item Il metodo deve restituire false se il token di autenticazione non è presente;
			\item Il metodo deve restituire false se il token è scaduto;
			\item Il metodo deve restituire true se il token è valido;
			\item Il metodo deve restituire false se la data di scadenza non è definita.
    \end{itemize} & S \\
		\hline TU & Verificare che il metodo "downloadFile" della classe "utils.service" soddisfi le seguenti condizioni:
		\begin{itemize}
			\item Il metodo deve creare un collegamento per il download del file;
			\item Il metodo deve attivare il download cliccando sul collegamento.
    \end{itemize} & S \\
		\hline TU & Verificare che il metodo "stringToSnakeCase" della classe "utils.service" soddisfi le seguenti condizioni:
		\begin{itemize}
			\item Il metodo deve convertire una stringa in snake\_case.
    \end{itemize} & S \\
		\hline TU & Verificare che il metodo "addCapitalizeValues" della classe "utils.service" soddisfi le seguenti condizioni:
		\begin{itemize}
			\item Il metodo deve aggiungere chiavi in maiuscolo a un determinato oggetto.
    \end{itemize} & S \\
		\hline TU & Verificare che il metodo "capitalizeString" della classe "utils.service" soddisfi le seguenti condizioni:
		\begin{itemize}
			\item Il metodo deve convertire in maiuscolo la prima lettera di una stringa.
    \end{itemize} & S \\
	\end{longtable}
\end{adjustwidth}
\egroup
