\subsection{Test di unità}

\par Lo scopo dei test di unità è verificare il corretto funzionamento delle "unità software", ossia delle porzioni o segmenti (come una funzione, una classe o un componente) testabili in modo autonomo e isolato all'interno del sistema. Di seguito è riportato l'elenco dei test di unità:

\bgroup
\begin{adjustwidth}{-0.5cm}{-0.5cm}
	% MAX 12.5cm
 	\begin{longtable}{|P{1.5cm}|>{\raggedright}P{9.5cm}|>{\arraybackslash}P{1.5cm}|}
		\caption{Test di unità}
  	\label{tab:test-unita} \\
	  \hline
		\textbf{ID} & \textbf{Descrizione} & \textbf{Stato} \\ 
		\hline
		\endfirsthead

		\caption[]{Test di unità (continua)} \\
		\hline
		\textbf{ID} & \textbf{Descrizione} & \textbf{Stato} \\ 
		\hline
		\endhead

		\hline
		\multicolumn{3}{|r|}{{Continua nella prossima pagina}} \\ 
		\hline
		\endfoot

		\hline
		\endlastfoot

		TU.1 & Verificare che il componente "ChatDeleteBtn" soddisfi le seguenti condizioni:
    \begin{itemize}
      \item Il pulsante deve essere disabilitato se non è presente alcun messaggio nella chat;
      \item Il pulsante deve essere disabilitato se il ChatBOT sta elaborando un nuovo messaggio;
      \item Quando viene cliccato, il pulsante deve inviare un segnale.
    \end{itemize} & S \\
	\end{longtable}
\end{adjustwidth}
\egroup