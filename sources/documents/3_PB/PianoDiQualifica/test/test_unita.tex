\subsection{Test di unità}

\par Lo scopo dei test di unità è verificare il corretto funzionamento delle "unità software", ossia delle porzioni o segmenti (come una funzione, una classe o un componente) testabili in modo autonomo e isolato all'interno del sistema. Di seguito è riportato l'elenco dei test di unità:

\bgroup
\begin{adjustwidth}{-0.5cm}{-0.5cm}
	% MAX 12.5cm
 	\begin{longtable}{|P{1.5cm}|>{\raggedright}P{9.5cm}|>{\arraybackslash}P{1.5cm}|}
		\caption{Test di unità}
  	\label{tab:test-unita} \\
	  \hline
		\textbf{ID} & \textbf{Descrizione} & \textbf{Stato} \\
		\hline
		\endfirsthead

		\caption[]{Test di unità (continua)} \\
		\hline
		\textbf{ID} & \textbf{Descrizione} & \textbf{Stato} \\
		\hline
		\endhead

		\hline
		\multicolumn{3}{|r|}{{Continua nella prossima pagina}} \\
		\hline
		\endfoot

		\hline
		\endlastfoot

		%json_file_adapter
		\hline TU.1 & Verificare che il metodo "create" della classe "FileFactory" soddisfi le seguenti condizioni:
		\begin{itemize}
			\item Deve creare correttamente un'istanza di "JsonFileAdapter" quando il tipo di file specificato nella configurazione è "json";
			\item Deve sollevare un'eccezione "ValueError" se il tipo di file specificato nella configurazione è sconosciuto.
		\end{itemize} & S \\

		%json_schema_validator_adapter
		\hline TU.2 & Verificare che il metodo "validate" della classe "JsonSchemaValidatorAdapter" soddisfi le seguenti condizioni:
		\begin{itemize}
			\item Deve restituire "True" se lo schema è valido secondo il modello di confronto per la validazione;
			\item Deve restituire "False" se lo schema non è valido secondo il modello di confronto per la validazione;
			\item Deve sollevare un'eccezione se non viene trovato il file da validare.
		\end{itemize} & S \\

		%sql_alchemy_authentication_repository_adapter
		\hline TU.3 & Verificare che il metodo "get\_user\_by\_username" della classe "SqlAlchemyAuthenticationRepositoryAdapter" soddisfi le seguenti condizioni:
		\begin{itemize}
			\item Deve recuperare correttamente un utente dal database in base allo username specificato, verificando che la query venga eseguita sul modello "Admins" applicando il filtro appropriato;
			\item Deve restituire "None" se non esiste alcun utente associato allo username fornito.
		\end{itemize} & S \\

		%sql_alchemy_db_manager_factory
		\hline TU.4 & Verificare che il metodo "create\_authentication\_repository" della classe "SqlAlchemyDbManagerFactory" soddisfi le seguenti condizioni:
		\begin{itemize}
			\item Deve creare correttamente un'istanza di "SqlAlchemyAuthenticationRepositoryAdapter";
			\item Deve sollevare un'eccezione se si verifica un errore durante la creazione dell'istanza.
		\end{itemize} & S \\

		\hline TU.5 & Verificare che il metodo "create\_dictionary\_repository" della classe "SqlAlchemyDbManagerFactory" soddisfi le seguenti condizioni:
		\begin{itemize}
			\item Deve creare correttamente un'istanza di "SqlAlchemyDictionaryRepositoryAdapter";
			\item Deve sollevare un'eccezione se si verifica un errore durante la creazione dell'istanza;
			\item Deve creare una nuova istanza ad ogni chiamata, verificando che più chiamate al metodo restituiscano istanze distinte.
		\end{itemize} & S \\

		\hline TU.6 & Verificare che l'inizializzazione della classe "SqlAlchemyDbManagerFactory" soddisfi le seguenti condizioni:
		\begin{itemize}
			\item Deve chiamare il metodo "create\_all" per garantire che tutte le tabelle del database siano create.
		\end{itemize} & S \\

		%sql_alchemy_dictionary_respository_adapter
		\hline TU.7 & Verificare che il metodo "create\_dictionary" della classe "SqlAlchemyDictionaryRepositoryAdapter" soddisfi le seguenti condizioni:
		\begin{itemize}
			\item Deve creare un nuovo dizionario e salvarlo nel database, verificando che le proprietà del dizionario siano impostate correttamente;
			\item Deve chiamare le operazioni di add, commit e refresh durante l'inserimento;
			\item Deve sollevare un'eccezione se si verifica un errore durante il commit del dizionario.
		\end{itemize} & S \\

		\hline TU.8 & Verificare che il metodo "update\_dictionary" della classe "SqlAlchemyDictionaryRepositoryAdapter" soddisfi le seguenti condizioni:
		\begin{itemize}
			\item Deve aggiornare correttamente le proprietà di un dizionario esistente e salvare le modifiche nel database;
			\item Deve chiamare le operazioni di commit e refresh durante l'aggiornamento;
			\item Deve sollevare un'eccezione se si verifica un errore durante il commit delle modifiche;
			\item Deve sollevare un'eccezione "AttributeError" se si tenta di aggiornare un dizionario inesistente.
		\end{itemize} & S \\

		\hline TU.9 & Verificare che il metodo "delete\_dictionary" della classe "SqlAlchemyDictionaryRepositoryAdapter" soddisfi le seguenti condizioni:
		\begin{itemize}
			\item Deve eliminare correttamente un dizionario dal database;
			\item Deve chiamare le operazioni di delete e commit durante l'eliminazione;
			\item Non deve chiamare le operazioni di delete e commit se si tenta di eliminare un dizionario inesistente;
			\item Deve sollevare un'eccezione se si verifica un errore durante l'eliminazione del dizionario.
		\end{itemize} & S \\

		\hline TU.10 & Verificare che il metodo "get\_all\_dictionaries" della classe "SqlAlchemyDictionaryRepositoryAdapter" soddisfi le seguenti condizioni:
		\begin{itemize}
			\item Deve restituire correttamente tutti i dizionari presenti nel database, verificando che la query venga eseguita sul modello "Dictionaries";
			\item Deve chiamare il metodo "all" durante il recupero dei dizionari.
		\end{itemize} & S \\

		\hline TU.11 & Verificare che il metodo "get\_dictionary\_by\_id" della classe "SqlAlchemyDictionaryRepositoryAdapter" soddisfi le seguenti condizioni:
		\begin{itemize}
			\item Deve restituire correttamente un dizionario in base all'ID fornito, verificando che la query venga eseguita sul modello "Dictionaries" applicando il filtro appropriato.
		\end{itemize} & S \\

		\hline TU.12 & Verificare che il metodo "get\_dictionary\_by\_name" della classe "SqlAlchemyDictionaryRepositoryAdapter" soddisfi le seguenti condizioni:
		\begin{itemize}
			\item Deve restituire correttamente un dizionario in base al nome fornito, verificando che la query venga eseguita sul modello "Dictionaries" applicando il filtro appropriato.
		\end{itemize} & S \\

		%db_manager_factory
		\hline TU.13 & Verificare che il metodo "create" della classe "DbManagerFactory" soddisfi le seguenti condizioni:
		\begin{itemize}
			\item Deve creare correttamente un'istanza di "SqlAlchemyDbManagerFactory" quando il tipo di database manager specificato nella configurazione è "sqlalchemy";
			\item Deve sollevare un'eccezione "ValueError" se il tipo di database manager specificato nella configurazione è sconosciuto.
			\item Deve sollevare un'eccezione "ValueError" se manca la chiave "db\_manager" nella configurazione.
		\end{itemize} & S \\

		%txtai_prompt_manager_adapter
		\hline TU.14 & Verificare che il metodo "prompt\_generator" della classe "TxtaiPromptManagerAdapter" soddisfi le seguenti condizioni:
		\begin{itemize}
			\item Deve generare correttamente il \glossario{prompt} includendo la richiesta dell'Utente, i metadati dello schema, la lingua e il \glossario{DBMS};
			\item Deve restituire un prompt vuoto se non vengono trovati risultati rilevanti;
			\item Deve generare correttamente il \glossario{log} se la modalità di \glossario{debug} è attiva;
			\item Deve gestire correttamente il caso in cui la richiesta dell'Utente è vuota;
			\item Deve gestire correttamente gli errori durante l'estrazione dei metadati dallo schema.
		\end{itemize} & S \\

		\hline TU.15 & Verificare che il metodo "get\_tuples" della classe "TxtaiPromptManagerAdapter" soddisfi le seguenti condizioni:
		\begin{itemize}
			\item Deve formulare correttamente la query \glossario{SQL} ed eseguire la ricerca restituendo le tuple corrispondenti;
			\item Deve attivare la registrazione del log quando richiesto e restituire il contenuto del log generato;
			\item Deve restituire una lista vuota se la ricerca semantica non produce risultati.
		\end{itemize} & S \\

		\hline TU.16 & Verificare che il metodo "get\_relevant\_tuples" della classe "TxtaiPromptManagerAdapter" soddisfi le seguenti condizioni:
		\begin{itemize}
			\item Deve mantenere le tuple con punteggi elevati, indipendentemente dalla distanza di punteggio;
			\item Deve mantenere o scartare le tuple in base alla distanza di punteggio rispetto alle tuple precedenti;
			\item Deve generare correttamente il log se la modalità di debug è attiva;
			\item Deve gestire correttamente una lista vuota di tuple.
		\end{itemize} & S \\

		%txtai_index_manager_adapter
		\hline TU.17 & Verificare che il metodo "get\_embeddings" della classe "TxtaiIndexManagerAdapter" soddisfi le seguenti condizioni:
		\begin{itemize}
			\item Deve restituire un'istanza della classe "Embeddings".
		\end{itemize} & S \\

		\hline TU.18 & Verificare che il metodo "create\_or\_load\_index" della classe "TxtaiIndexManagerAdapter" soddisfi le seguenti condizioni:
		\begin{itemize}
			\item Deve ripristinare un \glossario{indice} e restituire "False" se il dizionario ha già un indice associato;
			\item Deve creare un nuovo indice e restituire "True" se il dizionario non ha un indice associato.
		\end{itemize} & S \\

		\hline TU.19 & Verificare che il metodo "create\_index" della classe "TxtaiIndexManagerAdapter" soddisfi le seguenti condizioni:
		\begin{itemize}
			\item Deve creare un indice utilizzando i documenti estratti e salvarlo se "save\_index" è impostato a "True";
			\item Deve creare un indice senza salvarlo se "save\_index" è impostato a "False";
			\item Deve creare un indice anche se la lista dei documenti è vuota e salvarlo se "save\_index" è impostato a "True";
			\item Deve sollevare un'eccezione se si verifica un errore durante l'\glossario{indicizzazione}.
		\end{itemize} & S \\

		\hline TU.20 & Verificare che il metodo "save\_index" della classe "TxtaiIndexManagerAdapter" soddisfi le seguenti condizioni:
		\begin{itemize}
			\item Deve salvare l'indice utilizzando il percorso corretto;
			\item Deve sollevare un'eccezione se si verifica un errore durante il salvataggio dell'indice.
		\end{itemize} & S \\

		\hline TU.21 & Verificare che il metodo "load\_index" della classe "TxtaiIndexManagerAdapter" soddisfi le seguenti condizioni:
		\begin{itemize}
			\item Deve ripristinare l'indice utilizzando il percorso corretto.
		\end{itemize} & S \\

		\hline TU.22 & Verificare che il metodo "delete\_index" della classe "TxtaiIndexManagerAdapter" soddisfi le seguenti condizioni:
		\begin{itemize}
			\item Deve eliminare correttamente la directory dell'indice;
			\item Non deve tentare di eliminare la directory dell'indice se quest'ultima non esiste.
		\end{itemize} & S \\

		%txtai_embeddings_manager_factory
		\hline TU.23 & Verificare che il metodo "create\_index\_manager" della classe "TxtaiEmbeddingsManagerFactory" soddisfi le seguenti condizioni:
		\begin{itemize}
			\item Deve creare correttamente un'istanza di "TxtaiIndexManagerAdapter";
			\item Deve creare correttamente un'istanza di "TxtaiIndexManagerAdapter" anche se manca la chiave "txtai" nella configurazione.
		\end{itemize} & S \\

		\hline TU.24 & Verificare che il metodo "create\_prompt\_manager\_with\_dependencies" della classe "TxtaiEmbeddingsManagerFactory" soddisfi le seguenti condizioni:
		\begin{itemize}
			\item Deve creare un'istanza di "TxtaiPromptManagerAdapter" quando viene fornita un'istanza di "TxtaiIndexManagerAdapter" come dipendenza.
		\end{itemize} & S \\

		\hline TU.25 & Verificare che il metodo "create\_prompt\_manager" della classe "TxtaiEmbeddingsManagerFactory" soddisfi le seguenti condizioni:
		\begin{itemize}
			\item Deve creare un'istanza di "TxtaiPromptManagerAdapter", gestendo internamente la creazione di un'istanza di "TxtaiIndexManagerAdapter".
		\end{itemize} & S \\

		%txtai_debug_manager
		\hline TU.26 & Verificare che il metodo "semantic\_search\_log" della classe "TxtaiDebugManagerAdapter" soddisfi le seguenti condizioni:
		\begin{itemize}
			\item Deve restituire un log corretto quando viene fornita una lista con più tuple, riportando i dettagli e i termini rilevanti per ogni descrizione;
			\item Deve restituire un log appropriato quando viene fornita una lista vuota di tuple.
		\end{itemize} & S \\

		\hline TU.27 & Verificare che il metodo "semantic\_search\_log\_custom\_algorithm" della classe "TxtaiDebugManagerAdapter" soddisfi le seguenti condizioni:
		\begin{itemize}
			\item Deve registrare le tabelle con un punteggio superiore o uguale a 0.45;
			\item Deve registrare le tabelle con un punteggio inferiore a 0.45 se la differenza di punteggio rispetto alle tabelle precedenti è inferiore a 0.2 o il punteggio globale è compreso tra 0.35 e 0.45;
			\item Deve scartare le tabelle che non rispettano le condizioni precedenti;
			\item Deve gestire correttamente una lista vuota di tuple.
		\end{itemize} & S \\

		%json_file_adapter
		\hline TU.28 & Verificare che il metodo "save" della classe "JsonFileAdapter" soddisfi le seguenti condizioni:
		\begin{itemize}
			\item Deve aprire un file con permessi di scrittura;
			\item Deve eseguire almeno un'operazione di scrittura sul file aperto.
		\end{itemize} & S \\

		\hline TU.29 & Verificare che il metodo "load" della classe "JsonFileAdapter" soddisfi le seguenti condizioni:
		\begin{itemize}
			\item Deve restituire il percorso del file.
		\end{itemize} & S \\

		\hline TU.30 & Verificare che il metodo "delete" della classe "JsonFileAdapter" soddisfi le seguenti condizioni:
		\begin{itemize}
			\item Deve controllare l'esistenza del file;
			\item Deve eseguire l'operazione di eliminazione del file.
		\end{itemize} & S \\

		\hline TU.31 & Verificare che il metodo "get\_preview" della classe "JsonFileAdapter" soddisfi le seguenti condizioni:
		\begin{itemize}
			\item Deve restituire un oggetto che rappresenti lo schema del file;
			\item Deve restituire "None" se lo schema del file non è presente.
		\end{itemize} & S \\

		\hline TU.32 & Verificare che il metodo "extract\_index\_metadata" della classe "JsonFileAdapter" soddisfi le seguenti condizioni:
		\begin{itemize}
			\item Deve restituire i metadati del dizionario per l'indicizzazione.
		\end{itemize} & S \\

		\hline TU.33 & Verificare che il metodo "extract\_schema\_metadata" della classe "JsonFileAdapter" soddisfi le seguenti condizioni:
		\begin{itemize}
			\item Deve restituire i metadati del dizionario in forma di prompt.
		\end{itemize} & S \\

		%componenti front-end
		\hline TU.34 & Verificare che il componente "AppLogo" soddisfi le seguenti condizioni:
    \begin{itemize}
      \item Il logo deve essere renderizzato con il percorso dell'immagine corretto;
			\item L'altezza e la larghezza dell'immagine devono essere impostate con valori predefiniti se non vengono passate come props;
			\item Il colore dell'immagine deve cambiare in base al tema selezionato.
    \end{itemize} & S \\

		\hline TU.35 & Verificare che il componente "AppMenu" soddisfi le seguenti condizioni:
    \begin{itemize}
      \item Il menu dell'Utente deve avere avere una o più opzioni;
			\item Il menu del Tecnico deve avere due o più opzioni.
    \end{itemize} & S \\

		\hline TU.36 & Verificare che il componente "AppMenuItem" soddisfi le seguenti condizioni:
    \begin{itemize}
      \item La voce di menu deve contenere l'etichetta corretta;
			\item La voce di menu non deve essere attiva se non è stata selezionata;
			\item Una volta selezionata, la voce di menu deve essere attiva;
			\item La voce di menu può essere la radice di un menu;
			\item La voce di menu può contenere un sotto-menu;
			\item La voce di menu può aprire o chiudere un sotto-menu;
			\item L'icona "submenu-toggler" deve essere visibile se la voce di menu contiene un sotto-menu.
    \end{itemize} & S \\

		\hline TU.37 & Verificare che il componente "AppTopbar" soddisfi le seguenti condizioni:
    \begin{itemize}
      \item Il pulsante di login deve essere visibile se l'Utente non ha effettuato l'accesso;
			\item Il pulsante di logout deve essere visibile se l'Utente ha effettuato l'accesso;
			\item Quando viene cliccato il pulsante di logout, la funzione di logout deve essere chiamata;
			\item L'option menu deve essere aperto al clic dell'apposito pulsante;
			\item L'option menu deve essere chiuso quando l'Utente clicca al di fuori di esso.
    \end{itemize} & S \\

		\hline TU.38 & Verificare che il componente "AppFooter" soddisfi le seguenti condizioni:
    \begin{itemize}
      \item Il footer deve essere renderizzato correttamente;
			\item All'interno del footer deve essere visualizzata la versione corretta dell'applicazione.
    \end{itemize} & S \\

		\hline TU.39 & Verificare che il componente "ConfigSidebar" soddisfi le seguenti condizioni:
    \begin{itemize}
      \item La sezione per modificare la scala deve essere visibile;
			\item La sezione per modificare il tema deve essere visibile;
			\item La sezione per modificare la lingua dell'interfaccia deve essere visibile;
			\item La scala deve essere diminuita al clic del'apposito pulsante;
			\item La scala deve essere aumentata al clic dell'apposito pulsante;
			\item Il pulsante per diminuire la scala deve essere disabilitato se la scala è al minimo;
			\item Il pulsante per aumentare la scala deve essere disabilitato se la scala è al massimo;
			\item Il contenuto del localStorage deve essere aggiornato al cambio del tema;
			\item Il contenuto del localStorage deve essere aggiornato al cambio della lingua.
    \end{itemize} & S \\

		\hline TU.40 & Verificare che il componente "MenuSidebar" soddisfi le seguenti condizioni:
    \begin{itemize}
			\item Il pulsante per chiudere la sidebar deve essere nascosto su schermi di grandi dimensioni;
      \item Il pulsante per chiudere la sidebar deve essere visibile su schermi di piccole dimensioni.
    \end{itemize} & S \\

		\hline TU.41 & Verificare che il componente "LoginDialog" soddisfi le seguenti condizioni:
    \begin{itemize}
      \item Il dialog deve essere chiuso al clic dell'apposito pulsante;
			\item La funzione "messageSuccess" deve essere chiamata se il login viene effettuato con successo;
			\item Una volta completato il login, il dialog deve essere chiuso;
			\item Il contenuto del localStorage deve essere aggiornato se il login viene effettuato con successo;
			\item Una volta completato il login, il form deve essere resettato;
			\item La funzione "messageError" deve essere chiamata se il login fallisce.
    \end{itemize} & S \\

		\hline TU.42 & Verificare che il componente "DictPreview" soddisfi le seguenti condizioni:
    \begin{itemize}
      \item L'anteprima del dizionario dati deve essere nascosta se la variabile detailsVisible è uguale a "false";
			\item Se è visibile, l'anteprima deve mostrare i dati corretti;
			\item Lo stato di espansione dell'anteprima deve essere attivato/disattivato al clic dell'apposito pulsante;
			\item Quando viene cliccato il pulsante di chiusura, il componente deve inviare un segnale.
    \end{itemize} & S \\

		\hline TU.43 & Verificare che il componente "ChatMessage" soddisfi le seguenti condizioni:
    \begin{itemize}
      \item Il messaggio inviato dall'Utente deve essere visualizzato correttamente;
			\item Il messaggio inviato dal ChatBOT deve essere visualizzato correttamente;
			\item Se il messaggio inviato dal ChatBOT è "null", il sistema deve mostrare un avviso all'Utente;
			\item I pulsanti di azione devono essere nascosti se il messaggio è stato inviato dall'Utente;
			\item La funzione "Copia negli appunti" deve essere chiamata con i parametri corretti al clic dell'apposito pulsante;
			\item Il pulsante di debug deve essere nascosto se l'Utente non ha effettuato il login;
			\item La funzione "Apri debug modal" deve essere chiamata con i parametri corretti al clic dell'apposito pulsante.
    \end{itemize} & S \\

		\hline TU.44 & Verificare che il componente "DebugMessage" soddisfi le seguenti condizioni:
    \begin{itemize}
      \item Il componente deve visualizzare il messaggio di debug correttamente;
			\item Quando viene cliccato il pulsante "Scarica file", la funzione di download deve essere chiamata con i parametri corretti.
    \end{itemize} & S \\

		\hline TU.45 & Verificare che il componente "ChatDeleteBtn" soddisfi le seguenti condizioni:
    \begin{itemize}
      \item Il pulsante deve essere disabilitato se non è presente alcun messaggio nella chat;
      \item Il pulsante deve essere disabilitato se il ChatBOT sta elaborando un nuovo messaggio;
      \item Quando viene cliccato, il pulsante deve inviare un segnale.
    \end{itemize} & S \\

		\hline TU.46 & Verificare che il componente "CreateUpdateDictionaryModal" soddisfi le seguenti condizioni:
    \begin{itemize}
      \item Il form di creazione e modifica deve essere visualizzato correttamente;
			\item Il pulsante di invio deve essere disabilitato se il form è incompleto;
			\item La funzione "messageSuccess" deve essere chiamata se l'inserimento viene effettuato con successo;
			\item La funzione "messageError" deve essere chiamata se l'inserimento fallisce;
			\item Il pulsante di invio deve essere disabilitato se il nome del dizionario dati contiene caratteri non supportati;
			\item Il pulsante di invio deve essere disabilitato se il formato del file non è valido;
			\item La funzione "messageSuccess" deve essere chiamata se l'aggiornamento dei metadati viene effettuato con successo;
			\item La funzione "messageError" deve essere chiamata se l'aggiornamento dei metadati fallisce;
			\item La funzione "messageSuccess" deve essere chiamata se l'aggiornamento del file viene effettuato con successo;
			\item La funzione "messageError" deve essere chiamata se l'aggiornamento del file fallisce;
			\item L'upload del file deve essere disabilitato se un file è già stato caricato;
			\item Il file selezionato deve essere rimosso al clic dell'apposito pulsante.
    \end{itemize} & S \\

		\hline TU.47 & Verificare che il componente "AppLayout" soddisfi le seguenti condizioni:
    \begin{itemize}
      \item Il layout dell'applicazione deve essere renderizzato correttamente;
      \item Il wrapper globale deve contenere le classi corrette sulla base delle impostazioni e dello stato del layout.
    \end{itemize} & S \\

		\hline TU.48 & Verificare che il componente "DictionariesListView" soddisfi le seguenti condizioni:
    \begin{itemize}
      \item La lista dei dizionari dati deve essere visualizzata correttamente;
			\item Se la lista dei dizionari è vuota, il sistema deve mostrare un messaggio esplicativo;
			\item I risultati della ricerca dei dizionari per nome devono essere visualizzati correttamente;
			\item Se la ricerca dei dizionari per nome non produce risultati, il sistema deve mostrare un messaggio esplicativo;
			\item I risultati della ricerca dei dizionari per descrizione devono essere visualizzati correttamente;
			\item Se la ricerca dei dizionari per descrizione non produce risultati, il sistema deve mostrare un messaggio esplicativo;
			\item La funzione "Apri dictionary modal" deve essere chiamata con i parametri corretti al clic del pulsante di inserimento;
			\item La funzione "Apri dictionary modal" deve essere chiamata con i parametri corretti al clic del pulsante di modifica dei metadati;
			\item La funzione "Apri dictionary modal" deve essere chiamata con i parametri corretti al clic del pulsante di modifica del file;
			\item Quando viene cliccato il pulsante "Scarica file", la funzione di download deve essere chiamata con i parametri corretti;
			\item La funzione "messageSuccess" deve essere chiamata se l'eliminazione viene effettuata con successo;
			\item La funzione "messageError" deve essere chiamata se l'eliminazione fallisce.
    \end{itemize} & S \\

		\hline TU.49 & Verificare che il componente "ChatView" soddisfi le seguenti condizioni:
    \begin{itemize}
      \item Il contenuto della chat deve essere renderizzato correttamente;
			\item Il form di selezione pre-richiesta deve essere attivato/disattivato al click dell'apposito pulsante;
			\item Nella chat deve essere visualizzato il dizionario dati attivo con la relativa estensione;
			\item Quando un dizionario viene selezionato, il contenuto del localStorage deve essere aggiornato;
			\item Il pulsante di invio deve essere disattivato se non è stato selezionato alcun dizionario;
			\item Il pulsante di invio deve essere disattivato se non è stata inserita alcuna richiesta;
			\item La richiesta deve poter essere inviata premendo il tasto "Invio";
			\item La funzione di generazione del prompt deve essere chiamata con i parametri corretti al clic del pulsante di invio;
			\item Quando il ChatBOT ritorna una risposta, il contenuto del sessionStorage deve essere aggiornato;
			\item Quando viene inviata una richiesta, il campo di testo deve essere resettato;
			\item La funzione "messageError" deve essere chiamata se la generazione del prompt fallisce;
			\item Il pulsante "Scroll to bottom" deve essere visualizzato quando l'Utente scorre i messaggi;
			\item L'ultimo messaggio della chat deve essere visualizzato se viene cliccato il pulsante "Scroll to bottom".
    \end{itemize} & S \\

		\hline TU.50 & Verificare che il metodo "logout" della classe "auth.service" soddisfi le seguenti condizioni:
		\begin{itemize}
			\item Il metodo deve rimuovere il token di autenticazione dal localStorage;
			\item Il metodo deve inviare un segnale di notifica.
    \end{itemize} & S \\

		\hline TU.51 & Verificare che il metodo "isLogged" della classe "auth.service" soddisfi le seguenti condizioni:
		\begin{itemize}
			\item Il metodo deve restituire "false" se il token di autenticazione non è presente;
			\item Il metodo deve restituire "false" se il token è scaduto;
			\item Il metodo deve restituire "true" se il token è valido;
			\item Il metodo deve restituire "false" se la data di scadenza non è definita.
    \end{itemize} & S \\

		\hline TU.52 & Verificare che il metodo "downloadFile" della classe "utils.service" soddisfi le seguenti condizioni:
		\begin{itemize}
			\item Il metodo deve creare un collegamento per il download del file;
			\item Il metodo deve attivare il download cliccando sul collegamento.
    \end{itemize} & S \\

		\hline TU.53 & Verificare che il metodo "stringToSnakeCase" della classe "utils.service" soddisfi le seguenti condizioni:
		\begin{itemize}
			\item Il metodo deve convertire una stringa in snake\_case.
    \end{itemize} & S \\

		\hline TU.54 & Verificare che il metodo "addCapitalizeValues" della classe "utils.service" soddisfi le seguenti condizioni:
		\begin{itemize}
			\item Il metodo deve aggiungere chiavi in maiuscolo a un determinato oggetto.
    \end{itemize} & S \\

		\hline TU.55 & Verificare che il metodo "capitalizeString" della classe "utils.service" soddisfi le seguenti condizioni:
		\begin{itemize}
			\item Il metodo deve convertire in maiuscolo la prima lettera di una stringa.
    \end{itemize} & S \\
	\end{longtable}
\end{adjustwidth}
\egroup
