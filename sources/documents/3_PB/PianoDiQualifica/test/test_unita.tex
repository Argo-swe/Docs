\subsection{Test di unità}

\par Lo scopo dei test di unità è verificare il corretto funzionamento delle "unità software", ossia delle porzioni o segmenti (come una funzione, una classe o un componente) testabili in modo autonomo e isolato all'interno del sistema. Di seguito è riportato l'elenco dei test di unità:

\bgroup
\begin{adjustwidth}{-0.5cm}{-0.5cm}
	% MAX 12.5cm
 	\begin{longtable}{|P{1.5cm}|>{\raggedright}P{9.5cm}|>{\arraybackslash}P{1.5cm}|}
		\caption{Test di unità}
  	\label{tab:test-unita} \\
	  \hline
		\textbf{ID} & \textbf{Descrizione} & \textbf{Stato} \\ 
		\hline
		\endfirsthead

		\caption[]{Test di unità (continua)} \\
		\hline
		\textbf{ID} & \textbf{Descrizione} & \textbf{Stato} \\ 
		\hline
		\endhead

		\hline
		\multicolumn{3}{|r|}{{Continua nella prossima pagina}} \\ 
		\hline
		\endfoot

		\hline
		\endlastfoot

		% Esempio struttura test unità back-end
		%\hline TU & Verificare che il metodo "get_user" della classe "authentication_repository_adapter" soddisfi le seguenti condizioni:
		%\begin{itemize}
			%\item Il metodo deve restituire false se ...;
			%\item Il metodo deve restituire true se ...;
			%\item Il metodo deve sollevare un'eccezione se ...;
    %\end{itemize} & S \\
		
		TU & Verificare che il componente "AppLogo" soddisfi le seguenti condizioni:
    \begin{itemize}
      \item Il logo deve essere renderizzato con il percorso dell'immagine corretto;
			\item L'altezza e la larghezza dell'immagine devono essere impostate con valori predefiniti se non vengono passate come props;
			\item Il colore dell'immagine deve cambiare in base al tema selezionato.
    \end{itemize} & S \\
		\hline TU & Verificare che il componente "AppFooter" soddisfi le seguenti condizioni:
    \begin{itemize}
      \item Il footer deve essere renderizzato correttamente;
			\item All'interno del footer deve essere visualizzata la versione corretta dell'applicazione.
    \end{itemize} & S \\
		\hline TU & Verificare che il componente "AppMenu" soddisfi le seguenti condizioni:
    \begin{itemize}
      \item Il menu dell'Utente deve avere avere una o più opzioni;
			\item Il menu del Tecnico deve avere due o più opzioni.
    \end{itemize} & S \\
		\hline TU & Verificare che il componente "AppMenuItem" soddisfi le seguenti condizioni:
    \begin{itemize}
      \item La voce di menu deve contenere l'etichetta corretta;
			\item La voce di menu non deve essere attiva se non è stata selezionata;
			\item Una volta selezionata, la voce di menu deve essere attiva;
			\item La voce di menu può essere la radice di un menu;
			\item La voce di menu può contenere un sotto-menu;
			\item La voce di menu può aprire e chiudere un sotto-menu. 
    \end{itemize} & S \\
		\hline TU & Verificare che il componente "AppTopbar" soddisfi le seguenti condizioni:
    \begin{itemize}
      \item Il pulsante di login deve essere visibile se l'Utente non ha effettuato l'accesso;
			\item Il pulsante di logout deve essere visibile se l'Utente ha effettuato l'accesso;
			\item Quando viene cliccato il pulsante di logout, la funzione di logout deve essere chiamata;
			\item Il menu delle opzioni deve essere aperto al clic dell'apposito pulsante;
			\item Il menu delle opzioni deve essere chiuso quando l'Utente clicca al di fuori di esso.
    \end{itemize} & S \\
		\hline TU & Verificare che il componente "ConfigSidebar" soddisfi le seguenti condizioni:
    \begin{itemize}
      \item La sezione per modificare la scala deve essere visibile;
			\item La sezione per modificare il tema deve essere visibile;
			\item La sezione per modificare la lingua dell'interfaccia deve essere visibile;
			\item La scala deve essere diminuita al clic del'apposito pulsante;
			\item La scala deve essere aumentata al clic dell'apposito pulsante;
			\item Il pulsante per diminuire la scala deve essere disabilitato se la scala è al minimo;
			\item Il pulsante per aumentare la scala deve essere disabilitato se la scala è al massimo;
			\item Il contenuto del localStorage deve essere aggiornato al cambio del tema;
			\item Il contenuto del localStorage deve essere aggiornato al cambio della lingua.
    \end{itemize} & S \\
		\hline TU & Verificare che il componente "MenuSidebar" soddisfi le seguenti condizioni:
    \begin{itemize}
			\item Il pulsante per chiudere la sidebar non deve essere visibile su schermi di grandi dimensioni.
      \item Il pulsante per chiudere la sidebar deve essere visibile su schermi di piccole dimensioni.
    \end{itemize} & S \\
		\hline TU & Verificare che il componente "LoginDialog" soddisfi le seguenti condizioni:
    \begin{itemize}
      \item Il dialog deve essere chiuso al clic dell'apposito pulsante;
			\item La funzione "messageSuccess" deve essere chiamata se il login viene effettuato con successo;
			\item Una volta completato il login, il dialog deve essere chiuso;
			\item Il contenuto del localStorage deve essere aggiornato se il login viene effettuato con successo;
			\item Una volta completato il login, il form deve essere resettato;
			\item La funzione "messageError" deve essere chiamata se il login fallisce.
    \end{itemize} & S \\
		\hline TU & Verificare che il componente "DictPreview" soddisfi le seguenti condizioni:
    \begin{itemize}
      \item L'anteprima del dizionario dati non deve essere visibile se la variabile DetailsVisible è uguale a false;
			\item Se è visibile, l'anteprima deve mostrare i dati corretti;
			\item Lo stato di espansione dell'anteprima deve essere attivato/disattivato al clic dell'apposito pulsante;
			\item Quando viene cliccato il pulsante di chiusura, il componente deve inviare un segnale.
    \end{itemize} & S \\
		\hline TU & Verificare che il componente "ChatMessage" soddisfi le seguenti condizioni:
    \begin{itemize}
      \item Il messaggio inviato dall'Utente deve essere visualizzato correttamente;
			\item Il messaggio inviato dal ChatBOT deve essere visualizzato correttamente;
			\item I pulsanti di azione devono essere nascosti se il messaggio è stato inviato dall'Utente;
			\item La funzione "Copia negli appunti" deve essere chiamata con i parametri corretti al clic dell'apposito pulsante;
			\item Il pulsante di debug deve essere nascosto se l'Utente non ha effettuato il login;
			\item La funzione "Apri debug modal" deve essere chiamata con i parametri corretti al clic dell'apposito pulsante.
    \end{itemize} & S \\
		\hline TU & Verificare che il componente "DebugMessage" soddisfi le seguenti condizioni:
    \begin{itemize}
      \item Il componente deve visualizzare il messaggio di debug correttamente;
			\item Quando viene cliccato il pulsante "Scarica file", la funzione di download deve essere chiamata con i parametri corretti.
    \end{itemize} & S \\
		\hline TU & Verificare che il componente "ChatDeleteBtn" soddisfi le seguenti condizioni:
    \begin{itemize}
      \item Il pulsante deve essere disabilitato se non è presente alcun messaggio nella chat;
      \item Il pulsante deve essere disabilitato se il ChatBOT sta elaborando un nuovo messaggio;
      \item Quando viene cliccato, il pulsante deve inviare un segnale.
    \end{itemize} & S \\
		\hline TU & Verificare che il componente "CreateUpdateDictionaryModal" soddisfi le seguenti condizioni:
    \begin{itemize}
      \item Il form di creazione e modifica deve essere visualizzato correttamente;
			\item Il pulsante di submit deve essere disabilitato se il form è incompleto;
			\item La funzione "messageSuccess" deve essere chiamata se l'inserimento viene effettuato con successo;
			\item La funzione "messageError" deve essere chiamata se l'inserimento fallisce;
			\item Il pulsante di submit deve essere disabilitato se il formato del file non è valido;
			\item La funzione "messageSuccess" deve essere chiamata se l'aggiornamento dei metadati viene effettuato con successo;
			\item La funzione "messageError" deve essere chiamata se l'aggiornamento dei metadati fallisce;
			\item La funzione "messageSuccess" deve essere chiamata se l'aggiornamento del file viene effettuato con successo;
			\item La funzione "messageError" deve essere chiamata se l'aggiornamento del file fallisce;
			\item L'upload del file deve essere disabilitato se un file è già stato caricato;
			\item Il file selezionato deve essere rimosso al clic dell'apposito pulsante.
    \end{itemize} & S \\
		\hline TU & Verificare che il metodo "logout" della classe "auth.service" soddisfi le seguenti condizioni:
		\begin{itemize}
			\item Il metodo deve rimuovere il token di autenticazione dal localStorage;
			\item Il metodo deve inviare un segnale di notifica.
    \end{itemize} & S \\
		\hline TU & Verificare che il metodo "isLogged" della classe "auth.service" soddisfi le seguenti condizioni:
		\begin{itemize}
			\item Il metodo deve restituire false se il token di autenticazione non è presente;
			\item Il metodo deve restituire false se il token è scaduto;
			\item Il metodo deve restituire true se il token è valido;
			\item Il metodo deve restituire false se la data di scadenza non è definita.
    \end{itemize} & S \\
		\hline TU & Verificare che il metodo "downloadFile" della classe "utils.service" soddisfi le seguenti condizioni:
		\begin{itemize}
			\item Il metodo deve creare un collegamento per il download del file;
			\item Il metodo deve attivare il download cliccando sul collegamento.
    \end{itemize} & S \\
		\hline TU & Verificare che il metodo "stringToSnakeCase" della classe "utils.service" soddisfi le seguenti condizioni:
		\begin{itemize}
			\item Il metodo deve convertire una stringa in snake_case.
    \end{itemize} & S \\
		\hline TU & Verificare che il metodo "addCapitalizeValues" della classe "utils.service" soddisfi le seguenti condizioni:
		\begin{itemize}
			\item Il metodo deve aggiungere chiavi in maiuscolo a un determinato oggetto.
    \end{itemize} & S \\
		\hline TU & Verificare che il metodo "capitalizeString" della classe "utils.service" soddisfi le seguenti condizioni:
		\begin{itemize}
			\item Il metodo deve convertire in maiuscolo la prima lettera di una stringa.
    \end{itemize} & S \\
	\end{longtable}
\end{adjustwidth}
\egroup