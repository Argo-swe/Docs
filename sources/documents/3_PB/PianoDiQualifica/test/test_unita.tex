\subsection{Test di unità}

\par Lo scopo dei test di unità è verificare il corretto funzionamento delle "unità software", ossia delle porzioni o segmenti (come una funzione, una classe o un componente) testabili in modo autonomo e isolato all'interno del sistema. Di seguito è riportato l'elenco dei test di unità:

\bgroup
\begin{adjustwidth}{-0.5cm}{-0.5cm}
	% MAX 12.5cm
 	\begin{longtable}{|P{1.5cm}|>{\raggedright}P{9.5cm}|>{\arraybackslash}P{1.5cm}|}
		\caption{Test di unità}
  	\label{tab:test-unita} \\
	  \hline
		\textbf{ID} & \textbf{Descrizione} & \textbf{Stato} \\ 
		\hline
		\endfirsthead

		\caption[]{Test di unità (continua)} \\
		\hline
		\textbf{ID} & \textbf{Descrizione} & \textbf{Stato} \\ 
		\hline
		\endhead

		\hline
		\multicolumn{3}{|r|}{{Continua nella prossima pagina}} \\ 
		\hline
		\endfoot

		\hline
		\endlastfoot

		TU.1 & Verificare che il componente "DictPreview" soddisfi le seguenti condizioni:
    \begin{itemize}
      \item L'anteprima del dizionario dati non deve essere visibile se la variabile DetailsVisible è uguale a false;
			\item Se è visibile, l'anteprima deve mostrare i dati correttamente;
			\item Lo stato di espansione dell'anteprima deve essere attivato/disattivato al clic dell'apposito pulsante;
			\item Quando viene cliccato il pulsante di chiusura, il componente deve inviare un segnale.
    \end{itemize} & S \\
		\hline TU.2 & Verificare che il componente "ChatMessage" soddisfi le seguenti condizioni:
    \begin{itemize}
      \item Il messaggio inviato dall'Utente deve essere visualizzato correttamente;
			\item Il messaggio inviato dal ChatBOT deve essere visualizzato correttamente;
			\item I pulsanti di azione devono essere nascosti se il messaggio è stato inviato dall'Utente;
			\item La funzione "Copia negli appunti" deve essere chiamata con i parametri corretti al clic dell'apposito pulsante;
			\item Il pulsante di debug deve essere nascosto se l'Utente non ha effettuato il login;
			\item La funzione "Apri debug modal" deve essere chiamata con i parametri corretti al clic dell'apposito pulsante.
    \end{itemize} & S \\
		\hline TU.3 & Verificare che il componente "StringDataModal" soddisfi le seguenti condizioni:
    \begin{itemize}
      \item Il messaggio di debug deve essere visualizzato correttamente quando la variabile StringData è impostata;
      \item Il messaggio di debug non deve essere renderizzato quando la variabile StringData è vuota.
    \end{itemize} & S \\
		\hline TU.4 & Verificare che il componente "DebugMessage" soddisfi le seguenti condizioni:
    \begin{itemize}
      \item Il componente deve visualizzare il messaggio di debug correttamente;
			\item Quando viene cliccato il pulsante "Scarica file", la funzione di download deve essere chiamata con i parametri corretti.
    \end{itemize} & S \\
		\hline TU.5 & Verificare che il componente "ChatDeleteBtn" soddisfi le seguenti condizioni:
    \begin{itemize}
      \item Il pulsante deve essere disabilitato se non è presente alcun messaggio nella chat;
      \item Il pulsante deve essere disabilitato se il ChatBOT sta elaborando un nuovo messaggio;
      \item Quando viene cliccato, il pulsante deve inviare un segnale.
    \end{itemize} & S \\
		\hline TU.6 & Verificare che il componente "AppLogo" soddisfi le seguenti condizioni:
    \begin{itemize}
      \item Il logo deve essere renderizzato con il percorso dell'immagine corretto.
    \end{itemize} & S \\
	\end{longtable}
\end{adjustwidth}
\egroup