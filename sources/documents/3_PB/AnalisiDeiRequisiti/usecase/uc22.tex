\subsubsection{UC22 - Visualizzazione messaggio di debug}\label{UC22}
\paragraph*{Descrizione}
Il Tecnico visualizza un messaggio di \glossario{debug} che illustra il processo di generazione del \glossario{prompt}. Il debug può aiutare il Tecnico a capire come migliorare il \glossario{dizionario dati}, in particolare le descrizioni in linguaggio naturale delle tabelle e delle colonne del database. Nel messaggio, infatti, è documentato il modo in cui le descrizioni interagiscono con il modello di \glossario{AI}.

\paragraph*{Attori principali}
Tecnico

\paragraph*{Precondizioni}
\begin{itemize}
  \item Il sistema è attivo e funzionante;
  \item Il Tecnico ha effettuato l'autenticazione (\hyperref[UC1]{UC1});
  \item Il sistema ha generato almeno un \glossario{prompt} e il rispettivo \glossario{log}.
\end{itemize}

\paragraph*{Postcondizioni}
\begin{itemize}
  \item Il messaggio di \glossario{debug} viene visualizzato correttamente.
\end{itemize}

\paragraph*{Trigger}
Il Tecnico vuole visualizzare un messaggio di \glossario{debug}.

\paragraph*{Scenario principale}
\begin{enumerate}
  \item Il Tecnico accede alla sezione dedicata al debug del \glossario{prompt};
  \item Il Tecnico visualizza un messaggio di \glossario{debug};
  \item Il sistema mostra tutte le informazioni contenute nel messaggio:
    \begin{itemize}
      \item Data e ora di generazione del \glossario{log};
      \item Richiesta in linguaggio naturale;
      \item Prima fase della generazione del \glossario{prompt} (lista delle tabelle considerate rilevanti dal modello):
      \begin{itemize}
        \item Nome della tabella;
        \item Punteggio assegnato alla tabella;
        \item Descrizione della tabella;
        \item Classifica di importanza dei termini presenti nella descrizione della tabella;
        \item Descrizione della colonna più rilevante;
        \item Classifica di importanza dei termini presenti nella descrizione della colonna.
      \end{itemize}
      \item Seconda fase della generazione del \glossario{prompt} (lista delle tabelle pertinenti):
      \begin{itemize}
        \item Spiegazione del motivo per cui una tabella viene inserita o meno nel prompt.
      \end{itemize}
    \end{itemize}
\end{enumerate}