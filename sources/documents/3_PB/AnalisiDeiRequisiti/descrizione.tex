\section{Descrizione}

\subsection{Obiettivi del prodotto}
\par L’obiettivo finale del prodotto è consentire agli utenti di ottenere in risposta da ChatGPT (o altri servizi) una \glossario{query} SQL valida e funzionante, usando come input il \glossario{prompt} generato dal sistema ChatSQL. L’applicazione web deve semplificare il processo di conversione di interrogazioni in linguaggio naturale in query da eseguire su database di tipo SQL.

\subsection{Funzioni del prodotto}
\par Le funzioni necessarie al corretto funzionamento del sistema sono:
\begin{itemize}
  \item Autenticazione: necessaria per abilitare le funzionalità del profilo tecnico;
  \item Gestione dei dizionari dati: tramite le operazioni \glossario{CRUDL} è possibile gestire i dizionari dati nel sistema;
  \item Selezione di un \glossario{dizionario dati};
  \item Selezione della lingua di interrogazione;
  \item Selezione di un \glossario{DBMS};
  \item Generazione di un \glossario{prompt} a partire da una richiesta in linguaggio naturale;
  \item Copia del prompt generato;
  \item Visualizzazione del debug del prompt.
\end{itemize}

\subsection{Caratteristiche utente}
Dopo un confronto con la \glossario{Proponente}, è stato evidenziato un interesse nello sviluppo di questo progetto per diverse tipologie di utenti e ambiti, tra cui:
\begin{itemize}
  \item Ottimizzazione delle \glossario{query} durante lo sviluppo di nuovi programmi;
  \item Generazione di query per interrogare un database da parte di utenti con conoscenze di \glossario{SQL} limitate o nulle;
  \item Supporto ai reclutatori (o assuntori) nella costruzione di query da poter utilizzare come confronto in sede di colloquio.
\end{itemize}