\section{Introduzione}

\vspace{2em}
\subsection*{Scopo del documento}
\par Il \Gls\ ha lo scopo di raccogliere i termini che potrebbero risultare ambigui, al fine di prevenire incomprensioni relative al linguaggio utilizzato nella documentazione di progetto. Al suo interno, è possibile trovare termini tecnici, acronimi o vocaboli di altra natura considerati ambigui. L'obiettivo è allineare le conoscenze del gruppo e mantenere una coerenza interna.

\vspace{2em}
\subsection*{Impostazione del documento}
\par Il documento è suddiviso in diverse sezioni, ciascuna delle quali raccoglie i termini che iniziano con una specifica lettera, seguendo l'ordine alfabetico. Nella documentazione di progetto, i termini contenuti nel \Gls\ sono formattati in corsivo e contrassegnati con una \ped{G} in pedice.