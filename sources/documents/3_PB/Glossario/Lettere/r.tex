\section{R}

\vspace{2em}
\subsection*{Ramo base}
\par Il ramo base è il branch di destinazione delle pull request, ossia il ramo in cui vengono integrate le modifiche proposte. Nella maggior parte dei casi, il ramo base corrisponde al branch di default ("main branch", "master" o "develop"). È considerato il punto di riferimento per lo sviluppo e viene utilizzato come base per la creazione di nuovi branch di feature. Il ramo base rappresenta uno stato stabile e funzionante del codice sorgente, pertanto viene protetto da modifiche dirette non autorizzate. 

\vspace{2em}
\subsection*{React}
\par React è una libreria front-end e open-source utilizzata per lo sviluppo di interfacce utente web e native. Le interfacce vengono costruite a partire da singole porzioni di codice (modulari, riutilizzabili e scritte in JavaScript o TypeScript), chiamate componenti.

\vspace{2em}
\subsection*{Release}
\par Una release è una versione specifica di un software o di un prodotto che viene distribuita al pubblico o agli utenti finali. Rappresenta un punto di riferimento nello sviluppo del software in cui le nuove funzionalità sono state implementate, i bug sono stati corretti e il codice è stato testato e valutato come pronto per l'uso. Le release vengono numerate o denominate in modo sequenziale (ad esempio versione 1.0, versione 2.0, ecc.) e spesso sono accompagnate da note di rilascio che descrivono le modifiche apportate, i requisiti di sistema e altre informazioni pertinenti. Le release sono essenziali per il ciclo di vita del software e consentono agli utenti di accedere alle nuove funzionalità e correzioni in modo organizzato e controllato.

\vspace{2em}
\subsection*{Report}
\par Un report è uno strumento informativo che può essere impiegato per fornire una panoramica generale sull’andamento di un progetto o per monitorare i risultati di analisi, ricerche e processi di elaborazione dei dati.

\vspace{2em}
\subsection*{Repository}
\par Un repository è un archivio o una collezione di file digitali e di dati, organizzati in modo strutturato e gestiti tramite un sistema di versionamento. È utilizzato nello sviluppo software per archiviare e gestire il codice sorgente, ma può contenere anche documentazione, risorse multimediali e file di configurazione. I repository forniscono un ambiente centralizzato per la collaborazione, il controllo delle versioni e la gestione dei cambiamenti, consentendo a più utenti di lavorare sugli stessi file senza rischio di sovrascrittura o perdita di dati.

\vspace{2em}
\subsection*{Requisito}
\par Un requisito è una specifica o una condizione che deve essere soddisfatta da un sistema, un prodotto o un servizio per raggiungere determinati obiettivi. I requisiti possono essere di diversi tipi: requisiti funzionali (descrivono le funzionalità che il sistema deve fornire), requisiti non funzionali (delineano attributi qualitativi come prestazioni, sicurezza e usabilità), e requisiti di vincolo (rappresentano limitazioni e restrizioni). La raccolta, l'analisi e la gestione dei requisiti garantiscono che il prodotto finale soddisfi le aspettative e le esigenze del cliente.

\vspace{2em}
\subsection*{Retrospettiva}
\par La retrospettiva è una pratica utilizzata nei processi agili di sviluppo software per riflettere a posteriori sui risultati conseguiti in un determinato periodo di tempo, identificando i punti di forza, di debolezza e le opportunità di miglioramento. Durante una retrospettiva, il team si riunisce per analizzare il lavoro svolto, discutere delle difficoltà incontrate e collaborare per avviare azioni correttive. Le retrospettive vengono solitamente condotte alla fine di ciascuna iterazione o sprint per garantire un miglioramento continuo.

\vspace{2em}
\subsection*{Ricerca semantica}
\par La ricerca semantica è un approccio che mira a comprendere il significato del testo oltre alla corrispondenza delle parole chiave. Utilizza tecniche di analisi del linguaggio naturale (NLP) e di rappresentazione semantica per carpire il contesto e le relazioni tra i concetti. Questo approccio consente di ottenere risultati più pertinenti e accurati, in quanto considera il significato implicito e la rilevanza semantica, piuttosto che limitarsi esclusivamente alla presenza di parole chiave.

\vspace{2em}
\subsection*{RTB}
\par La requirements and technology baseline è la prima revisione dello stato di avanzamento del progetto. La presentazione della RTB deve illustrare i requisiti e le tecnologie individuate dal team. Questa fase definisce i requisiti funzionali e non funzionali del sistema, insieme alle tecnologie scelte per soddisfare tali requisiti. La RTB fornisce una base solida per la progettazione e lo sviluppo del prodotto, assicurando coerenza e allineamento con gli obiettivi e le esigenze del cliente.
