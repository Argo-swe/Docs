\section{U}

\vspace{2em}
\subsection*{UML}
\par In ingegneria del software, UML (Unified Modeling Language) è un linguaggio di modellazione che consente di rappresentare un sistema secondo tre aspetti principali, per ciascuno dei quali sono disponibili diagrammi specifici:
\begin{itemize}
  \item Modello funzionale: descrive il sistema dal punto di vista dell'utente; corrisponde all'analisi dei requisiti e utilizza i diagrammi dei casi d'uso;
  \item Modello a oggetti: rappresenta la struttura del sistema sulla base del paradigma orientato agli oggetti;
  \item Modello dinamico: descrive il comportamento degli oggetti del sistema.
\end{itemize}

\vspace{2em}
\subsection*{Unittest}
\par Unittest è un framework integrato in Python, basato sul modello JUnit di Java, che fornisce strumenti per scrivere ed eseguire suite di test automatici.

\vspace{2em}
\subsection*{Unità}
\par Una "unità software" è una porzione di codice (come una funzione, una classe o un componente) che può essere testata in modo autonomo e isolato all'interno di un sistema.

\vspace{2em}
\subsection*{UPPER\_SNAKE\_CASE}
\par UPPER\_SNAKE\_CASE è una convenzione di denominazione in cui tutte le lettere delle parole sono scritte in maiuscolo e le parole sono separate da underscore (\_).

\vspace{2em}
\subsection*{Use case}
\par Vedi \glossario{Casi d'uso}.
