\section{C}


\vspace{2em}
\subsection*{Capitolato}
\par Il capitolato è un documento contrattuale che definisce le specifiche e le condizioni di un progetto o un appalto. In genere viene rilasciato da un'organizzazione o da un cliente per descrivere i requisiti, gli obiettivi e le condizioni contrattuali che devono essere soddisfatte dal fornitore o dall'appaltatore. Il capitolato fornisce dettagli sullo scopo del progetto, sulle funzionalità richieste, sui vincoli tecnici ecc. È utilizzato come base per la negoziazione e l'esecuzione del contratto tra le parti coinvolte nel progetto.

\vspace{2em}
\subsection*{Casi d'uso}
\par I casi d'uso sono una tecnica utilizzata nell'ingegneria del software per descrivere le interazioni tra un sistema e gli \glossario{attori} esterni o interni ad esso. Ogni caso d'uso rappresenta uno scenario completo di interazione che descrive come un attore utilizza il sistema per raggiungere un obiettivo specifico. Ogni caso d'uso è composto da una serie di passi o azioni che il sistema compie in risposta alle azioni dell'attore. I casi d'uso sono spesso rappresentati graficamente mediante diagrammi dei casi d'uso e sono utilizzati come strumento per definire e analizzare i requisiti del sistema, nonché per guidare il processo di progettazione e sviluppo del software.

\vspace{2em}
\subsection*{ChatBOT}
\par Un ChatBOT è un software che simula una conversazione umana, consentendo agli utenti di interagire con i dispositivi digitali come se stessero comunicando con una persona reale. Alcuni ChatBOT sono in grado di apprendere ed evolversi per fornire livelli crescenti di personalizzazione quando raccolgono ed elaborano le informazioni.

\vspace{2em}
\subsection*{ChatGPT}
\par ChatGPT (Chat Generative Pre-trained Transformer) è un ChatBOT basato su intelligenza artificiale e apprendimento automatico, sviluppato da OpenAI e addestrato a partire da modelli linguistici di grandi dimensioni come InstructGPT o GPT-3.5.

\vspace{2em}
\subsection*{Chinook}
\par Chinook è un esempio di database SQL comunemente utilizzato a scopo didattico e di esempio. Include una serie di tabelle che rappresentano uno scenario realistico di gestione di un negozio di musica, con dati relativi ad album, artisti, clienti, ordini e dipendenti. È spesso impiegato per dimostrare query SQL, operazioni \glossario{CRUD} e altre funzionalità del database, fornendo un contesto pratico per l'apprendimento e la pratica delle tecniche di gestione dei database relazionali.

\vspace{2em}
\subsection*{Codifica}
\par La codifica è il processo di scrittura di istruzioni in un linguaggio di programmazione che un calcolatore può elaborare. Questo processo include la traduzione della logica di business e dei requisiti del software in codice sorgente, seguendo la sintassi e le regole specifiche del linguaggio scelto, come Python, Java o C++. La codifica richiede competenze in algoritmi, strutture dati e principi di ingegneria del software per garantire che il programma sia efficiente, manutenibile e privo di errori.

\vspace{2em}
\subsection*{Commit}
\par Un commit è un'operazione in un sistema di controllo di versione che salva le modifiche apportate al codice sorgente o ai documenti all'interno di un repository. Rappresenta una "istantanea" dello stato del progetto in un determinato momento, accompagnata da un messaggio che descrive le modifiche effettuate. Il commit permette di tracciare la storia delle modifiche, agevolando il lavoro collaborativo e il mantenimento del codice, consentendo di tornare a versioni precedenti se necessario.

\vspace{2em}
\subsection*{Committente}
\par Il committente è il soggetto che commissiona un lavoro o richiede una prestazione. Nell'ambito del progetto di ingegneria del software, il ruolo di committente è ricoperto dal Professor Tullio Vardanega e dal Professor Riccardo Cardin.

\vspace{2em}
\subsection*{Containerizzazione}
\par La containerizzazione è una tecnologia di virtualizzazione a livello di sistema operativo che consente di eseguire applicazioni e le loro dipendenze in ambienti isolati chiamati container. A differenza delle macchine virtuali, i container condividono il kernel del sistema operativo dell'host, ma hanno un proprio file system, librerie e configurazioni. Questo approccio migliora la portabilità, la scalabilità e l'efficienza delle applicazioni, facilitando il deployment e la gestione in ambienti cloud e DevOps. Docker è uno degli strumenti più popolari per la containerizzazione.

\vspace{2em}
\subsection*{Continuous Integration}
\par La continuous integration (CI) è una pratica di integrazione frequente (ovvero "molte volte al giorno") delle modifiche locali all'interno di un ambiente condiviso in cui vengono eseguiti processi di build e test. È una delle principali best practice DevOps e, come tale, implica l'adozione di un version control system.

\vspace{2em}
\subsection*{CRUD}
\par Acronimo che fa riferimento alle 4 operazioni base relative alla gestione della persistenza dei dati. Le operazioni sono: Create (creazione), Read (lettura), Update (aggiornamento), Delete (cancellazione).

\vspace{2em}
\subsection*{CRUDL}
\par Con riferimento a \glossario{Large Language Model} aggiunge l'operazione List, per la rappresentazione in forma di lista di risorse multiple.

\vspace{2em}
\subsection*{CSV}
\par Formato di file basato su righe di testo in cui i valori sono separati da virgole. Viene utilizzato principalmente per l'importazione ed esportazione dei dati. La struttura semplice non permette la gestione di modelli di dati complessi.
