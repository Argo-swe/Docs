\section{Y}

\vspace{2em}
\subsection*{YAML}
\par YAML è un linguaggio di serializzazione dei dati versatile e di facile lettura comunemente utilizzato per scrivere file di configurazione.
Fornisce un formato standardizzato per rappresentare i dati strutturati in un modo comprensibile per l'uomo e interpretabile dalle macchine. "YAML" è un acronimo che sta per "YAML Ain't Markup Language" o "Yet Another Markup Language"; il primo ha lo scopo di sottolineare che il linguaggio è destinato ai dati piuttosto che ai documenti.
In sostanza, YAML è progettato pensando alla semplicità e alla leggibilità. Utilizza una sintassi pulita e minimalista, basata su indentazione, coppie chiave-valore e convenzioni intuitive. Questo approccio consente agli sviluppatori e agli utenti di esprimere strutture di dati complesse in un formato simile al linguaggio naturale.
