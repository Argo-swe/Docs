\section{P}

\vspace{2em}
\subsection*{Parser}
\par Un parser è un componente software che analizza una sequenza di dati in input secondo una determinata grammatica o sintassi e produce una rappresentazione strutturata o un'altra forma di output. È comunemente utilizzato nel campo dell'informatica per analizzare e interpretare il codice sorgente di linguaggi di programmazione, il markup di documenti o altri formati di dati strutturati. Un parser suddivide il testo in token o simboli atomici e applica regole grammaticali per costruire una rappresentazione intermedia della struttura del testo.

\vspace{2em}
\subsection*{PB}
\par Vedi \glossario{Product Baseline}.

\vspace{2em}
\subsection*{PDCA}
\par PDCA (Plan-Do-Check-Act) è un ciclo di gestione iterativo utilizzato per il controllo continuo della qualità e il miglioramento dei processi. Consiste in quattro fasi: Pianificazione (Plan), in cui vengono identificati problemi e obiettivi; Esecuzione (Do), in cui vengono implementate le soluzioni; Verifica (Check), in cui vengono valutati i risultati rispetto agli obiettivi prefissati; e Azione (Act), in cui vengono standardizzate le soluzioni ritenute efficaci e vengono intraprese ulteriori iniziative per migliorare il processo. Il ciclo di PDCA promuove un approccio sistematico per il miglioramento continuo.

\vspace{2em}
\subsection*{Performance}
\par La performance, nel contesto dello sviluppo software, si riferisce all'efficienza con cui un sistema o un'applicazione esegue le sue funzioni. Gli aspetti chiave includono la velocità di esecuzione, il tempo di risposta, l'uso delle risorse (CPU, memoria, disco), la scalabilità e la capacità di gestire carichi di lavoro elevati. La performance è cruciale per garantire una buona esperienza utente e può essere ottimizzata attraverso pratiche di programmazione come il testing e il tuning delle risorse. Gli strumenti di monitoraggio e profiling sono utilizzati per identificare e risolvere i colli di bottiglia.

\vspace{2em}
\subsection*{Persistenti}
\par Nell'ambito dello sviluppo software, i dati persistenti sono quelli che continuano a esistere anche dopo la chiusura dell'applicazione che li ha creati. Questi dati vengono tipicamente memorizzati su supporti di archiviazione non volatili come dischi rigidi o SSD, utilizzando database, file di log o altri meccanismi di archiviazione. La persistenza costituisce un attributo essenziale per applicazioni che necessitano di conservare informazioni a lungo termine, garantendo che i dati possano essere recuperati e utilizzati in future sessioni di lavoro.

\vspace{2em}
\subsection*{PoC}
\par Vedi \glossario{Proof of Concept}.

\vspace{2em}
\subsection*{Processo}
\par In ambito informatico e gestionale, un processo rappresenta una serie organizzata di attività interconnesse finalizzate alla trasformazione di input in output. Questo può includere procedure standardizzate, strumenti e risorse per raggiungere obiettivi specifici di business o tecnologici. La gestione efficace dei processi è cruciale per migliorare l'efficienza operativa, la qualità del prodotto e la soddisfazione del cliente.

\vspace{2em}
\subsection*{Prodotto software}
\par Un prodotto software è un'applicazione informatica o sistema sviluppato per soddisfare specifiche esigenze utente. Include il ciclo completo di sviluppo software, dalla progettazione alla realizzazione, testing e manutenzione. I prodotti software variano da applicazioni desktop a sistemi embedded e servizi cloud, rivestendo un ruolo cruciale in tutti gli aspetti della vita moderna, dall'automazione industriale alla gestione delle informazioni e all'intrattenimento.

\vspace{2em}
\subsection*{Product backlog}
\par Il product backlog è un elenco ordinato delle attività che il team deve svolgere, non soltanto compiti legati alle funzionalità che il prodotto deve offrire, ma anche task di natura organizzativa, logistica e di redazione dei documenti. Il product backlog viene costantemente rivisto e riordinato in base alle richieste e alle aspettative del cliente. All’interno del framework Scrum, il product backlog è considerato un "artefatto".

\vspace{2em}
\subsection*{Product Baseline}
\par La product baseline è la seconda revisione dello stato di avanzamento del progetto. La presentazione della PB deve illustrare le scelte architetturali del team, in coerenza con la Technology Baseline, riportando i diagrammi delle classi e di sequenza, e i design pattern adottati.

\vspace{2em}
\subsection*{Progettazione}
\par La progettazione è il processo di definizione architetturale delle componenti di un sistema software. Include l'analisi dei requisiti, la creazione di modelli, l'implementazione di strutture dati e algoritmi, assicurando che il sistema soddisfi gli obiettivi funzionali e di performance prefissati. La fase di progettazione è essenziale per garantire la scalabilità, la manutenibilità e la sicurezza del software, contribuendo alla sua efficace implementazione ed evoluzione nel tempo.

\vspace{2em}
\subsection*{Prompt Engineering}
\par Il prompt engineering è un approccio metodologico che si concentra sull'ottimizzazione e progettazione dei prompt utilizzati nelle interfacce utente, nei sistemi di intelligenza artificiale e nelle applicazioni di dialogo uomo-macchina. Questo approccio considera i prompt come un elemento determinante per guidare e influenzare il comportamento degli utenti o dei sistemi, e si propone di progettare prompt efficaci, chiari e coinvolgenti per massimizzare l'interazione e il coinvolgimento degli utenti. Il prompt engineering coinvolge la comprensione del contesto di utilizzo, la progettazione linguistica, l'analisi del feedback degli utenti e l'ottimizzazione continua dei prompt per migliorare l'esperienza complessiva.

\vspace{2em}
\subsection*{Proof of Concept}
\par Il PoC (Proof of Concept) è l’allestimento di una demo prototipale, ovvero la dimostrazione pratica dei funzionamenti di base di un applicativo software, finalizzata ad accertare la fattibilità dei requisiti e la validità delle tecnologie individuate.

\vspace{2em}
\subsection*{Proponente}
\par La Proponente è la parte interessata che propone un progetto, un'idea o un'iniziativa. Nel contesto dell'ingegneria del software, la Proponente è l'attore che avvia il processo di sviluppo, identificando la necessità o l'opportunità di un nuovo prodotto o servizio. La Proponente può essere un individuo, un'organizzazione o un'azienda. La Proponente del capitolato ChatSQL è l'azienda Zucchetti S.p.A.

\vspace{2em}
\subsection*{Pull request}
\par Le pull request sono una caratteristica dei sistemi di controllo di versione come \glossario{Git}; sono utilizzate per richiedere l'integrazione di modifiche da parte di un collaboratore. Una pull request rappresenta una richiesta di revisione del codice: uno sviluppatore propone le sue modifiche e chiede ai membri del team di esaminare, discutere e approvare i cambiamenti. Le pull request sono spesso accompagnate da commenti, descrizioni e discussioni che facilitano la collaborazione e la comunicazione tra i membri del team. Dopo aver ricevuto il consenso e l'approvazione, le modifiche proposte nella pull request vengono integrate (merged) nel ramo base, aggiornando la cronologia del progetto.

\vspace{2em}
\subsection*{Python}
\par Python è un linguaggio di programmazione ad alto livello, interpretato e orientato agli oggetti, noto per la sua sintassi chiara e leggibile. È versatile e può essere utilizzato in una vasta gamma di applicazioni, tra cui sviluppo web, analisi dei dati, automazione di compiti e intelligenza artificiale. Python è apprezzato per la sua facilità d'uso, per la disponibilità di librerie e per la ricca comunità di sviluppatori.
