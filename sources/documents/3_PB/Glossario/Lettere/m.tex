\section{M}
\vspace{2em}
\subsection*{Markup language}
\par Un markup language è un linguaggio di codifica usato per annotare un documento in modo che sia distinguibile sia dal testo normale sia dal software che lo elabora. Esempi comuni includono HTML (HyperText Markup Language), utilizzato per creare pagine web, e XML (eXtensible Markup Language), utilizzato per memorizzare e trasportare dati. I markup language definiscono una struttura e una sintassi specifiche per etichettare il contenuto, semplificando l'organizzazione, la visualizzazione e la manipolazione delle informazioni.

\vspace{2em}
\subsection*{Merge}
\par Il merge è il processo di combinazione di due o più rami di codice sorgente in un singolo ramo. Questa è un'operazione comune nei sistemi di controllo di versione, come Git, dove gli sviluppatori lavorano su branch indipendenti per sviluppare funzionalità o correggere bug in modo isolato. Una volta completati, i cambiamenti vengono uniti nel ramo principale (main o master). Il merge può essere automatico o richiedere intervento manuale in caso di conflitti (quando due modifiche sono state apportate alla stessa porzione di codice).

\vspace{2em}
\subsection*{Metadati}
\par I metadati forniscono informazioni contestuali che facilitano la comprensione, l'organizzazione e la gestione di dati di varia natura (documenti, oggetti, ecc.). Esempi comuni di metadati includono autori, date di creazione, dimensioni dei file e parole chiave. Nei database, i metadati possono descrivere la struttura delle tabelle, i campi e le relazioni tra i dati. Nei sistemi di gestione dei contenuti, i metadati migliorano la ricerca e il recupero delle informazioni, permettendo una gestione più efficiente delle risorse digitali.

\vspace{2em}
\subsection*{Metriche}
\par Le metriche sono indicatori quantitativi utilizzati per valutare determinati aspetti di un sistema, un processo o un prodotto. Nell'ambito dello sviluppo software, le metriche possono essere utilizzate per valutare la qualità del prodotto, le prestazioni del sistema e l'efficacia del processo di sviluppo. Le metriche forniscono informazioni utili per comprendere lo stato attuale, identificare le aree di miglioramento e intraprendere azioni correttive. È importante selezionare e applicare le metriche in modo puntuale, assicurandosi che siano rilevanti per gli obiettivi e i requisiti prefissati.

\vspace{2em}
\subsection*{Mock}
\par Nell'ingegneria del software, un mock è un oggetto, una funzione o un modulo simulato che emula il comportamento di un componente reale in modo controllato. Un mock funge da sostituto del componente reale in fase di test, consentendo agli sviluppatori di isolare le unità software dalle dipendenze esterne.

\vspace{2em}
\subsection*{Modello}
\par Vedi \glossario{LLM}.

\vspace{2em}
\subsection*{MVC}
\par Il Model-view-controller (MVC) è un pattern architetturale molto diffuso nell'ambito della programmazione orientata agli oggetti e dello sviluppo web. Si tratta di un modello di progettazione software in grado di separare la logica di presentazione dalla logica di business. La logica del programma è divisa in tre elementi interconnessi: il modello (rappresentazione interna e gestione dei dati), la vista (esposizione dei dati all'utente) e il controller (collegamento tra model e view).
