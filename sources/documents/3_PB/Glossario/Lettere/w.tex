\section{W}

\vspace{2em}
\subsection*{Way of Working}
\par Il WoW, acronimo di "Way of Working", è un insieme di pratiche, processi e metodologie adottati da un team o un'organizzazione. Il Way of Working definisce le modalità di collaborazione, comunicazione, pianificazione ed esecuzione delle attività. Include procedure per la gestione del progetto, l'organizzazione del lavoro, la comunicazione interna e la risoluzione dei problemi. Un Way of Working coerente e metodico contribuisce a incrementare l'efficienza, la produttività e la qualità del lavoro svolto dal team.

\vspace{2em}
\subsection*{Web App}
\par Una web app, abbreviazione di "applicazione web", è un'applicazione software che viene eseguita su un server web e visitata dagli utenti attraverso un browser web su una rete Internet. Le web app sono progettate per funzionare su diverse piattaforme e dispositivi senza la necessità di essere installate localmente sul dispositivo dell'utente.

\vspace{2em}
\subsection*{Workflow}
\par Il workflow è una sequenza di operazioni, attività o compiti necessari per raggiungere un obiettivo; il worfklow viene spesso gestito tramite software per automatizzare il processo di sviluppo.

\vspace{2em}
\subsection*{WoW}
\par Vedi \glossario{Way of working}.
