\section{D}

\vspace{2em}
\subsection*{Database}
\par Un database è una raccolta organizzata di dati strutturati, che possono essere memorizzati e recuperati elettronicamente da un sistema informatico. I database sono progettati per consentire una gestione efficiente dei dati, semplificando operazioni di inserimento, aggiornamento, eliminazione e interrogazione. Utilizzano sistemi di gestione di database (DBMS) per garantire l'integrità, la sicurezza e la disponibilità dei dati.

\vspace{2em}
\subsection*{DBMS}
\par Un database management system (DBMS) è un sistema software finalizzato alla creazione, manipolazione e interrogazione di una o più basi di dati. Un DBMS fornisce agli utenti servizi come il controllo degli accessi, il mantenimento dell'integrità e la gestione delle transazioni. Tra i DBMS più noti vi sono MySQL, MariaDB, PostgreSQL, Microsoft SQL Server e SQLite.

\vspace{2em}
\subsection*{Database vettoriale}
\par Un database vettoriale è progettato per indicizzare, archiviare e gestire grandi quantità di dati vettoriali altamente dimensionali. Alcuni esempi di database vettoriali sono \glossario{FAISS} (Facebook AI Similarity Search) e Pinecone, utilizzati in applicazioni di intelligenza artificiale e machine learning per effettuare ricerche di similarità e recupero di dati vettoriali, con funzionalità aggiuntive come operazioni \glossario{CRUD} e filtraggio dei metadati.

% TODO: da scommentare quando modificato "debug del prompt" su tutti i documenti
% \vspace{2em}
% \subsection*{Debug}
% Il debug è il processo di individuazione, analisi e risoluzione di errori o problemi all'interno di un software o di un sistema informatico. Consiste nel rilevare e correggere bug, ovvero difetti o comportamenti indesiderati che causano il malfunzionamento del programma. Le attività di debug possono includere l'ispezione del codice sorgente, l'utilizzo di strumenti di debugging per monitorare lo stato del programma durante l'esecuzione e l'implementazione di modifiche per eliminare gli errori e migliorare la stabilità e le prestazioni del software.

\vspace{2em}
\subsection*{Debug}
\par Il debug, nel contesto del capitolato ChatSQL, implica la dimostrazione del processo di generazione del prompt. Il debug aiuta l'operatore a capire come riformulare il dizionario dati, al fine di migliorare la qualità delle risposte generate dall'\glossario{LLM}. Lo scopo del debug è illustrare le modalità di interazione tra il dizionario dati (in particolare le descrizioni in linguaggio naturale in esso riportate) e il Large Language Model.

% \vspace{2em}
% \subsection*{Debug del prompt}
% TODO: da copiare voce DEBUG quando modificato su tutti i documenti

\vspace{2em}
\subsection*{Debugging}
\par Vedi \glossario{Debug}.
%Il debugging è il processo di ricerca e risoluzione dei bug o dei malfunzionamenti in un software o in un sistema informatico. Coinvolge l'analisi dei sintomi dei problemi, la ricerca delle cause sottostanti e l'implementazione di soluzioni per correggere gli errori. Il debugging può richiedere l'uso di strumenti specializzati, la revisione del codice sorgente e il testing approfondito per garantire che il software funzioni correttamente senza errori.

\vspace{2em}
\subsection*{Design Pattern}
\par I design pattern sono soluzioni generiche e riutilizzabili a problemi comuni che si presentano durante la progettazione del software. Sono descrizioni o template che possono essere applicati per risolvere problemi di progettazione in vari contesti. I design pattern contribuiscono a creare software manutenibile, scalabile e comprensibile. Alcuni esempi comuni includono il Singleton, per garantire una sola istanza di una classe, il Factory, per la creazione di oggetti senza specificare l'esatta classe dell'oggetto, e l'Observer, per un sistema di notifica tra oggetti.

\vspace{2em}
\subsection*{Diagramma di Gantt}
\par Il diagramma di Gantt è uno strumento di visualizzazione utilizzato nella gestione dei progetti per rappresentare la pianificazione delle attività nel tempo. È composto da una serie di barre orizzontali che rappresentano le attività del progetto e la loro durata prevista, disposte lungo un asse temporale. Il diagramma di Gantt consente di visualizzare facilmente la sequenza delle attività, le dipendenze tra di esse e lo stato di avanzamento del progetto.

\vspace{2em}
\subsection*{Distribuzioni}
\par Nel contesto dello sviluppo software e dei sistemi operativi, una distribuzione (spesso abbreviata in "distro") si riferisce a una versione preconfezionata di un sistema operativo, tipicamente basata su un kernel comune come quello di Linux. Le distribuzioni includono il kernel, strumenti di sistema, applicazioni e un gestore di pacchetti per facilitare l'installazione e la gestione del software. Ogni distribuzione è ottimizzata per determinati scenari d'uso, come server, desktop o dispositivi embedded. Esempi popolari di distribuzioni Linux includono Ubuntu, Fedora, Debian e CentOS. Le distribuzioni agevolano l'adozione di Linux fornendo un sistema pronto all'uso con configurazioni e software preinstallati.

\vspace{2em}
\subsection*{Dizionario dati}
\par Un dizionario dati è una raccolta di metadati che descrive la struttura, le caratteristiche e le relazioni dei dati all'interno di un \glossario{database}. Include informazioni come: i nomi delle tabelle e le loro descrizioni, i nomi dei campi e le loro descrizioni, i tipi di dati e le relazioni tra tabelle. Questo strumento in ChatSQL serve al sistema per costruire gli \glossario{indici vettoriali}, agli utenti per capire quali richieste formulare e ai tecnici per comprendere il processo di generazione del \glossario{prompt}.

\vspace{2em}
\subsection*{Django}
\par Django è un framework web open source, scritto in linguaggio Python e basato sul paradigma "Model-Template-View".

\vspace{2em}
\subsection*{Docker}
\par Docker è una piattaforma open-source per la creazione, la distribuzione e l'esecuzione di applicazioni in contenitori leggeri e portabili. I contenitori Docker consentono agli sviluppatori di eseguire le proprie applicazioni e le relative dipendenze in un ambiente isolato, garantendo che esse funzionino in modo consistente su qualsiasi sistema operativo o infrastruttura. Docker facilita la distribuzione delle applicazioni, permettendo agli sviluppatori di incorporare il proprio codice insieme a tutte le librerie e le dipendenze necessarie, eliminando così i problemi di compatibilità tra ambienti di sviluppo e produzione.

\vspace{2em}
\subsection*{DTO}
\par DTO è l'acronimo di Data Transfer Object, un design pattern utilizzato per trasferire dati tra sottosistemi di un'applicazione software. I DTO sono oggetti che non dovrebbero contenere alcuna logica di business, in quanto si occupano di archiviare, recuperare, serializzare e deserializzare i dati in rete.

\vspace{2em}
\subsection*{Dump SQL}
\par Struttura di dati che replica nella struttura e talvolta anche nei dati la rappresentazione di un database. Viene utilizzato per l'esportazione e importazione del database associato.
