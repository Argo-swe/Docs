\section{V}

\vspace{2em}
\subsection*{Validazione}
\par Il processo di validazione o convalida del software ha lo scopo di determinare se un prodotto soddisfa i requisiti concordati con il cliente e formalizzati durante l'analisi dei requisiti. La validazione può essere effettuata in presenza della \glossario{Proponente}.

\vspace{2em}
\subsection*{VCS}
\par Vedi \glossario{Version Control System}.

\vspace{2em}
\subsection*{Verifica}
\par Il processo di verifica ha lo scopo di determinare se un prodotto o un processo è conforme alle specifiche e ai vincoli qualitativi prefissati. La verifica si concentra sull'esame delle caratteristiche, degli attributi e delle funzionalità del prodotto per garantire la sua correttezza e completezza. La verifica può coinvolgere attività come il testing del software, il walkthrough e l'ispezione.

\vspace{2em}
\subsection*{Versionamento}
\par Il versionamento è il processo di gestione e tracciamento delle versioni di un documento, file o sistema software nel corso del tempo. Consiste nel mantenere un registro delle modifiche, indicando chi ha apportato un cambiamento, quando è stato effettuato e quale sia la natura della modifica. Il versionamento è essenziale per tenere traccia delle modifiche, garantire la collaborazione tra più autori e ripristinare versioni precedenti in caso di necessità. I sistemi di controllo di versione come \glossario{Git} sono ampiamente utilizzati per implementare il versionamento nel contesto dello sviluppo software.

\vspace{2em}
\subsection*{Version Control System}
\par Gli strumenti per il controllo di versione sono indispensabili per la maggior parte dei progetti di sviluppo software collaborativi. Tali strumenti consentono al team di gestire in modo strutturato versioni multiple di un insieme di informazioni. Facilitano inoltre la risoluzione dei conflitti e l'annullamento delle modifiche, oltre a garantire un backup remoto del codice.
Alcuni esempi sono \glossario{Git}, Mercurial, Subversion, CVS.

\vspace{2em}
\subsection*{Vettore}
\par Nell’ambito della ricerca semantica, un vettore è una rappresentazione numerica di un testo all’interno di uno spazio vettoriale multidimensionale. Ogni dimensione del vettore cattura un aspetto o una caratteristica semantica del testo. Questa codifica consente di misurare similitudini e relazioni tra parole, frasi o documenti attraverso operazioni matematiche.

\vspace{2em}
\subsection*{Vue}
\par Vedi \glossario{Vue.js}.

\vspace{2em}
\subsection*{Vue.js}
\par Vue.js è un framework JavaScript utilizzato per la creazione di interfacce utente reattive e dinamiche. Può essere integrato gradualmente in progetti esistenti senza richiedere una riscrittura completa del codice. Vue.js offre funzionalità per la creazione di componenti riutilizzabili, la gestione dello stato dell'applicazione e l'aggiornamento dinamico dell'interfaccia utente in risposta ai cambiamenti.
