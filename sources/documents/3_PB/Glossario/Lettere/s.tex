\section{S}

\vspace{2em}
\subsection*{Scrum}
\par Scrum è un framework agile utilizzato per gestire progetti complessi, specie nel contesto dello sviluppo software. Si basa su principi di trasparenza, ispezione e adattamento, e promuove il lavoro collaborativo, l'autorganizzazione del team e la consegna graduale di prodotti di valore. In Scrum, il lavoro è organizzato in cicli di sviluppo chiamati sprint, durante i quali il team si impegna a rilasciare un insieme di funzionalità completate e testate.

\vspace{2em}
\subsection*{Scrum Meeting}
\par Il framework Scrum prevede quattro occasioni formali, o meeting, per valutare lo stato di avanzamento delle iterazioni: sprint planning, daily scrum, sprint review e sprint retrospective.

\vspace{2em}
\subsection*{Sentence Similarity}
\par La sentence similarity, o similarità tra frasi, è una misura che quantifica il grado di somiglianza tra due frasi o segmenti di testo. Questo concetto è ampiamente utilizzato nell'ambito del Natural Language Processing (NLP) per valutare quanto due frasi siano semanticamente simili o correlate. Le tecniche per calcolare la similarità tra frasi possono variare a seconda del contesto e degli obiettivi specifici, ma spesso coinvolgono l'uso di modelli di \glossario{embedding} del linguaggio naturale per rappresentare le frasi in spazi vettoriali e misurare la distanza tra i loro vettori rappresentativi.

\vspace{2em}
\subsection*{Sentence Transformers}
\par I sentence transformers sono modelli di elaborazione del linguaggio naturale (NLP) progettati per codificare frasi e paragrafi in spazi vettoriali continui, rappresentando il significato semantico delle frasi. Utilizzano tecniche di apprendimento automatico basate su trasformatori per catturare le relazioni semantiche tra le parole e le frasi. I sentence transformers sono destinati a compiti come la ricerca semantica, il clustering di testo, il riassunto automatico e l'analisi della similarità tra frasi.

\vspace{2em}
\subsection*{SOLID}
\par In ingegneria del software, il termine SOLID si riferisce a cinque principi della programmazione orientata agli oggetti (Single Responsibility Principle, Open-Closed Principle, Liskov Substitution Principle, Interface Segregation Principle e Dependency Inversion Principle).

\vspace{2em}
\subsection*{SPICE}
\par SPICE (Software Process Improvement and Capability dEtermination) è uno standard internazionale per la valutazione e il miglioramento dei processi software. Fornisce un quadro per misurare la maturità e la capacità dei processi, favorendo il miglioramento continuo della qualità e della produttività. SPICE è utilizzato in settori critici come l'aerospaziale, l'automotive e il settore della difesa, dove la qualità e la sicurezza del software sono prioritari per il successo dei prodotti e dei servizi.

\vspace{2em}
\subsection*{Spreadsheet}
\par Uno spreadsheet, o foglio di calcolo, è un'applicazione software utilizzata per organizzare, elaborare e analizzare dati in forma tabellare. Lo spreadsheet è composto da una griglia di celle organizzate in righe e colonne, in cui è possibile inserire numeri, testo e formule matematiche per eseguire calcoli e analisi dei dati. Le celle dello spreadsheet possono essere formattate e personalizzate in base alle esigenze dell'utente, e possono essere utilizzate per creare grafici, report e visualizzazioni dei dati.

\vspace{2em}
\subsection*{Sprint}
\par Lo sprint è un periodo di tempo definito e limitato all'interno di uno sviluppo \glossario{Agile} durante il quale viene svolto un lavoro specifico e concreto. In genere, ha una durata prestabilita che varia da una a quattro settimane, durante le quali il team di sviluppo si impegna a completare un insieme di attività pianificate. Gli sprint sono caratterizzati da obiettivi chiari e misurabili e terminano con la consegna di un incremento di lavoro funzionante. Durante uno sprint, il lavoro viene suddiviso in attività chiamate backlog items, e il progresso viene monitorato attraverso riunioni quotidiane chiamate daily scrum o stand-up.

\vspace{2em}
\subsection*{Sprint Backlog}
\par Uno sprint backlog è un sottoinsieme del \glossario{product backlog} e viene definito in fase di pianificazione delle iterazioni (sprint planning).

\vspace{2em}
\subsection*{SQL}
\par SQL (Structured Query Language) è un linguaggio utilizzato per estrarre e manipolare i dati contenuti all'interno di database basati sul modello relazionale. Non si tratta però di un semplice linguaggio di interrogazione, in quanto alcune sue componenti, come il Data Control Language, permettono di gestire e amministrare basi di dati.

\vspace{2em}
\subsection*{SQLite}
\par SQLite è una libreria di gestione del database relazionale incorporata nella maggior parte dei linguaggi di programmazione. È un database leggero e autosufficiente che non richiede un processo separato per funzionare ed è progettato per essere integrato direttamente nelle applicazioni. SQLite supporta la quasi totalità delle funzionalità dei database SQL standard, tra cui la creazione di tabelle, l'inserimento, l'aggiornamento e l'estrazione dei dati, nonché l'implementazione di vincoli di integrità dei dati. È utilizzato in applicazioni mobile, desktop e web per gestire dati locali o di piccole dimensioni.

\vspace{2em}
\subsection*{Staging Area}
\par In ambito informatico, una staging area è un ambiente separato e controllato dove le modifiche al software sono testate prima della distribuzione. Questo ambiente è essenziale per verificare l'integrità delle modifiche e ridurre il rischio di errori. La staging area semplifica il testing completo delle funzionalità, la risoluzione dei bug e l'ottimizzazione delle performance prima che le modifiche diventino operative, garantendo un'implementazione sicura e affidabile delle nuove funzionalità software. Nel contesto di Git, la staging area si colloca tra la working directory e il repository, e memorizza le modifiche che verranno incluse nel commit successivo.

\vspace{2em}
\subsection*{Stakeholder}
\par Gli stakeholder sono individui, gruppi o organizzazioni che hanno un interesse diretto o indiretto in un progetto o un'iniziativa e possono essere influenzati o influenzare il suo successo o fallimento. Il ruolo di stakeholder può essere rivestito da clienti, utenti finali, investitori, dipendenti, partner commerciali, organizzazioni governative e gruppi di interesse pubblico. È importante coinvolgere gli stakeholder nel processo decisionale e nella pianificazione al fine di garantire che le loro esigenze, aspettative e preoccupazioni siano prese in considerazione e gestite in modo appropriato.

\vspace{2em}
\subsection*{Story}
\par Nella metodologia Agile, una story è un requisito di business espresso in modo chiaro e conciso, focalizzandosi sul valore da fornire all'utente finale. Le story sono brevi descrizioni di funzionalità desiderate, scritte in un linguaggio comprensibile sia agli sviluppatori che agli stakeholder, e sono utilizzate come unità di lavoro nello sviluppo iterativo del software. Definire le story è essenziale per garantire il corretto funzionamento del software e soddisfare le aspettative del cliente.

\vspace{2em}
\subsection*{Streamlit}
\par Streamlit è un framework open-source per la creazione rapida di applicazioni web interattive. Si concentra sulla semplicità e sulla facilità d'uso, consentendo agli sviluppatori di creare e distribuire applicazioni web complesse utilizzando il linguaggio Python. Streamlit offre componenti predefiniti per la visualizzazione dei dati, la gestione degli input e la creazione di interfacce intuitive, permettendo ai programmatori di concentrarsi sullo sviluppo della logica dell'applicazione senza doversi preoccupare della complessità dell'infrastruttura web.

\vspace{2em}
\subsection*{Sviluppatore}
\par Uno sviluppatore software è un professionista esperto nella creazione, implementazione e manutenzione di software e sistemi informatici. Applicando competenze in programmazione, design software e testing, gli sviluppatori trasformano requisiti utente in applicazioni funzionali e performanti. Essi rivestono un ruolo chiave in vari settori, dalla tecnologia all'educazione, lavorando in team per produrre soluzioni innovative.
