\section{Introduzione}
\label{sec:introduzione}

\subsection{Scopo del documento}
\par Il presente documento ha lo scopo di descrivere le scelte tecnologiche e progettuali alla base dello sviluppo del prodotto ChatSQL. Verranno quindi illustrati gli stili e i pattern architetturali adottati dal team.

\subsection{Riferimenti}
\par Il presente documento si basa su normative elaborate dal team, dall'ente proponente o da entità esterne, oltre a includere materiali informativi. Tali riferimenti sono elencati di seguito.

\subsubsection{Riferimenti normativi}
\begin{itemize}
  \item \NormeDiProgetto;
  \item Slide PD2 - Corso di Ingegneria del Software - Regolamento del progetto didattico:\\ \href{https://www.math.unipd.it/~tullio/IS-1/2023/Dispense/PD2.pdf}{https://www.math.unipd.it/\textasciitilde tullio/IS-1/2023/Dispense/PD2.pdf};
  \item Capitolato C9 - ChatSQL:
  \begin{itemize}
    \item \href{https://www.math.unipd.it/~tullio/IS-1/2023/Progetto/C9.pdf}{https://www.math.unipd.it/\textasciitilde tullio/IS-1/2023/Progetto/C9.pdf} \\ (Ultimo accesso: 2024-09-12);
    \item \href{https://www.math.unipd.it/~tullio/IS-1/2023/Progetto/C9p.pdf}{https://www.math.unipd.it/\textasciitilde tullio/IS-1/2023/Progetto/C9p.pdf} \\ (Ultimo accesso: 2024-09-12).
  \end{itemize}
\end{itemize}

\subsubsection{Riferimenti informativi}
\begin{itemize}
  \item \AnalisiDeiRequisiti;
  \item Corso di Ingegneria del Software - Dependency injection:\\ \href{https://www.math.unipd.it/~rcardin/swea/2022/Design%20Pattern%20Architetturali%20-%20Dependency%20Injection.pdf}{https://www.math.unipd.it/\textasciitilde rcardin/swea/2022/Design\_Pattern\_Architetturali \- Dependency\_Injection.pdf};
  \item Corso di Ingegneria del Software - Object-oriented programming (OOP):\\ \href{https://www.math.unipd.it/~rcardin/swea/2023/Object-Oriented%20Progamming%20Principles%20Revised.pdf}{https://www.math.unipd.it/\textasciitilde rcardin/swea/2023/Object-Oriented\_Progamming \- Principles\_Revised.pdf};
  \item Corso di Ingegneria del Software - Diagrammi delle classi:\\ \href{https://www.math.unipd.it/~rcardin/swea/2023/Diagrammi%20delle%20Classi.pdf}{https://www.math.unipd.it/\textasciitilde rcardin/swea/2023/Diagrammi\_delle\_Classi.pdf};
  \item Corso di Ingegneria del Software - Design pattern architetturali:\\ \href{https://www.math.unipd.it/~rcardin/swea/2022/Software%20Architecture%20Patterns.pdf}{https://www.math.unipd.it/\textasciitilde rcardin/swea/2022/Software\_Architecture\_Patterns.pdf};
  \item Corso di Ingegneria del Software - Design pattern creazionali:\\ \href{https://www.math.unipd.it/~rcardin/swea/2022/Design%20Pattern%20Creazionali.pdf}{https://www.math.unipd.it/\textasciitilde rcardin/swea/2022/Design\_Pattern\_Creazionali.pdf};
  \item Corso di Ingegneria del Software - Design pattern strutturali:\\ \href{https://www.math.unipd.it/~rcardin/swea/2022/Design%20Pattern%20Strutturali.pdf}{https://www.math.unipd.it/\textasciitilde rcardin/swea/2022/Design\_Pattern\_Strutturali.pdf};
  \item Corso di Ingegneria del Software - Principi SOLID:\\ \href{https://www.math.unipd.it/~rcardin/swea/2021/SOLID%20Principles%20of%20Object-Oriented%20Design_4x4.pdf}{https://www.math.unipd.it/\textasciitilde rcardin/swea/2021/SOLID\_Principles\_of\_Object-Ori \- ented\_Design\_4x4.pdf};
  \item Repository GitHub Ingegneria del Software:\\ \href{https://github.com/rcardin/swe-imdb}{https://github.com/rcardin/swe-imdb};
  \item Repository GitHub Architettura esagonale:\\ \href{https://github.com/rcardin/hexagonal}{https://github.com/rcardin/hexagonal};
  \item Corso di Ingegneria del Software - T6 - Progettazione software:\\ \href{https://www.math.unipd.it/~tullio/IS-1/2023/Dispense/T6.pdf}{https://www.math.unipd.it/\textasciitilde tullio/IS-1/2023/Dispense/T6.pdf};
  \item Corso di Ingegneria del Software - T7 - Qualità del software:\\ \href{https://www.math.unipd.it/~tullio/IS-1/2023/Dispense/T7.pdf}{https://www.math.unipd.it/\textasciitilde tullio/IS-1/2023/Dispense/T7.pdf};
  \item Pattern Strategy:\\ \href{https://refactoring.guru/design-patterns/strategy}{https://refactoring.guru/design-patterns/strategy} \\ (Ultimo accesso: 2024-09-05);
  \item Struttura del pattern Strategy:\\ \href{https://it.wikipedia.org/wiki/Strategy_pattern}{https://it.wikipedia.org/wiki/Strategy\_pattern} \\ (Ultimo accesso: 2024-09-05);
  \item Abstract factory:\\ \href{https://it.wikipedia.org/wiki/Abstract_factory}{https://it.wikipedia.org/wiki/Abstract\_factory} \\ (Ultimo accesso: 2024-08-25);
  \item Abstract Factory in Python:\\ \href{https://shanenullain.medium.com/abstract-factory-in-python-with-generic-typing-b9ceca2bf89e}{https://shanenullain.medium.com/abstract-factory-in-python-with-generic \- typing-b9ceca2bf89e} \\ (Ultimo accesso: 2024-09-05);
  \item Interfacce e classi astratte in Python:\\ \href{https://medium.com/@shashikantrbl123/interfaces-and-abstract-classes-in-python-understanding-the-differences-3e5889a0746a}{https://medium.com/@shashikantrbl123/interfaces-and-abstract-classes-in \- python-understanding-the-differences-3e5889a0746a} \\ (Ultimo accesso: 2024-08-20);
  \item Diagrammi delle classi - Notazioni:\\ \href{https://www.ionos.it/digitalguide/siti-web/programmazione-del-sito-web/diagrammi-di-classe-con-uml/}{https://www.ionos.it/digitalguide/siti-web/programmazione-del-sito-web/di \- agrammi-di-classe-con-uml/} \\ (Ultimo accesso: 2024-08-20);
  \item Guida alla creazione di diagrammi UML:\\ \href{https://www.drawio.com/blog/uml-class-diagrams}{https://www.drawio.com/blog/uml-class-diagrams} \\ (Ultimo accesso: 2024-08-10);
  \item Service e repository pattern:\\ \href{https://medium.com/@ankitpal181/service-repository-pattern-802540254019}{https://medium.com/@ankitpal181/service-repository-pattern-802540254019} \\ (Ultimo accesso: 2024-09-05);
  \item \Glossario;
  \item Verbali interni:
  \begin{itemize}
    \item 2024-07-26;
    \item 2024-08-01;
    \item 2024-08-08;
    \item 2024-08-14;
    \item 2024-08-19;
    \item 2024-08-27.
  \end{itemize}
  \item Verbali esterni:
  \begin{itemize}
    \item 2024-09-09.
  \end{itemize}
\end{itemize}

\subsection{Glossario} 
\GlossarioIntroduzione

