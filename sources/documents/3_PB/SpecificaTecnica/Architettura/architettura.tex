\section{Architettura}

\par Il prodotto ChatSQL è basato su un'architettura client-server. Il client è l'interfaccia attraverso la quale gli utenti interagiscono con il sistema, come ad esempio un browser web. In altre parole, il client è il componente che richiede risorse o servizi. Il server è l'applicazione che riceve ed elabora le richieste provenienti da uno o più client, fornendo risposte appropriate. Per migliorare lo sviluppo collaborativo, la modularità e la manutenibilità, il sistema è stato suddiviso in due componenti principali:
\begin{itemize}
  \item \textbf{Front-end}: La componente di un sistema che l'utente visualizza e con cui può interagire;
  \item \textbf{Back-end}: La componente che gestisce la logica di business, l'elaborazione dei dati e la comunicazione con i database e altri servizi.
\end{itemize}