\section{Architettura}

\subsection{Introduzione}
\par Il prodotto ChatSQL è basato su un'architettura client-server. Il client è l'interfaccia attraverso la quale gli utenti interagiscono con il sistema, come ad esempio un browser web. In altre parole, il client è il componente che richiede risorse o servizi. Il server è l'applicazione che riceve ed elabora le richieste provenienti da uno o più client, fornendo risposte appropriate.

\begin{figure}[H]
  \centering
  \includegraphics[width=0.95\textwidth]{assets/client_server.png}
  \caption{Architettura client-server}
\end{figure}

\par Per migliorare lo sviluppo collaborativo, la modularità e la manutenibilità, il sistema è stato suddiviso in due componenti principali:
\begin{itemize}
  \item \textbf{Front-end}: è la porzione di un sistema che l'utente visualizza e con cui può interagire. Il front-end è sviluppato utilizzando il framework \glossario{Vue.js} ed è responsabile dell'interfaccia grafica, che deve essere intuitiva, funzionale e accattivante. Trasmette le richieste dell'utente al back-end e visualizza i risultati ottenuti;
  \item \textbf{Back-end}: è il segmento che gestisce la logica di business, l'elaborazione dei dati e la comunicazione con i database e altri servizi. Il back-end è sviluppato utilizzando il framework \glossario{FastAPI}.
\end{itemize}

\vspace{0.5\baselineskip}
\par La comunicazione tra il front-end e il back-end avviene tramite chiamate \glossario{API}. Il team segue le linee guida e i principi definiti da REST (representational state transfer), uno stile architetturale che impone condizioni sul funzionamento di un'API. Le REST API (o RESTful API) sono stateless, il che significa che ogni richiesta HTTP deve includere tutte le informazioni necessarie per elaborarla. Questo riduce il carico sul server e migliora la scalabilità. Inoltre, l’approccio stateless agevola l'implementazione di sistemi di caching, migliorando le prestazioni complessive. Una REST API è simile a un sito web in esecuzione in un browser con funzionalità HTTP integrata. Le operazioni sono basate su metodi HTTP standard come GET, POST, PUT e DELETE.

\vspace{0.5\baselineskip}
\par La persistenza delle informazioni dei dizionari (nome, descrizione, ecc.) è garantita dalla presenza di un database, che memorizza anche gli operatori registrati nel sistema. Il database è implementato utilizzando SQLite. Di seguito è riportata l'architettura ad alto livello della web app.

\begin{figure}[H]
  \centering
  \includegraphics[width=\textwidth]{assets/architettura_web_app.pdf}
  \caption{Architettura web app}
\end{figure}

\subsection{Assemblaggio dei componenti}
\par Docker Compose viene utilizzato per gestire applicazioni multi-container, permettendo di assemblare diversi servizi che compongono un'applicazione. Nel contesto di ChatSQL, il team ha creato i seguenti container Docker:
\begin{itemize}
  \item \textbf{backend}: espone l'interfaccia di backend sulla porta 8000. All'indirizzo \textit{localhost:8000/docs} è possibile consultare la documentazione interattiva delle API. Inoltre, sono disponibili dettagli sui Data Transfer Objects (DTO);
  \item \textbf{frontend}: espone l'interfaccia utente sulla porta 5173.
\end{itemize}

\subsection{Struttura del sistema}

\subsubsection{Back-end}

\subsubsection{Modello architetturale}
\par Il modello architetturale scelto dal gruppo è il modello ad \textbf{architettura esagonale}. Questo modello trova le sue basi nell'architettura a livelli, dalla quale cerca di prenderne le basi e di far fronte ai problemi di dipendenza stretta che si creavano tra i livelli.
\par L'architettura è composta da 3 parti principali:
\begin{itemize}
    \item \textbf{Core}: il core è la sezione centrale dell'architettura, in cui risiede la business logic dell'applicazione. È indipendente da qualsiasi interfaccia utente o servizio esterno. Ciò viene fatto per mantenerla isolata e disaccoppiata dal resto dell'architettura, in modo che possa agire indipendentemente;
    \item \textbf{Ports}: costituiscono dei punti di comunicazione tra il core e altri parti del sistema o servizi esterni. Esistono due tipi principali di porte:
    \begin{itemize}
        \item Inbound port: consentono al nucleo di essere invocato da componenti esterne;
        \item Outbound port: consentono al nucleo di reperire informazioni dall'esterno.
    \end{itemize}
    \item \textbf{Adapters}: sono pattern architetturali che permettono di adattare i dati in modo che possano essere ricevuti dall'esterno e passare per le porte o vice versa. Costituiscono la sezione più esterna dell'architettura e consentono pertanto il passaggio e il monitoraggio dei dati.
\end{itemize}
\par In generale questa architettura è rappresentata a forma esagonale per più motivi: da una parte l'architettura vuole ricordare la cella di un'alveare. Avendo tale forma, la cella si può incastrare con altre celle allo scopo di definire un alveare. Allo stesso modo un'architettura esagonale può interfacciarsi con altre strutture con le stesse caratteristiche per definire un sistema più grande. Un altro motivo della scelta è la visualizzazione della simmetria della struttura, la quale viene visualmente divisa a metà, dove una metà si interfaccia con l'esterno per ricevere gli input, mentre l'altra si interfaccia con l'esterno per inviare informazioni in output.
\par L'architettura esagonale presenta numerosi pregi, tra i quali:
\begin{itemize}
    \item ha la facoltà di essere un'architettura flessibile, in quanto le parti di questa architettura agiscono indipendentemente tra di loro e di conseguenze si possono adattare con facilità;
    \item è scalabile in quanto una cella ad architettura esagonale può essere vista come un microservizio e per tanto più facilmente replicabile;
    \item è facilmente testabile, in quanto il core è isolato e disaccoppiato con il resto dell'architettura.
    \item è sufficientemente menutenibile, in quanto il core è isolato e disaccoppiato dal resto dell'architettura.
\end{itemize}
\par Questo tipo di architettura non è immune a difetti, tra i quali si possono trovare:
\begin{itemize}
    \item la complessità dell'implementazione, in quanto l'architettura richiede una buona strutturazione e una buona progettazione;
    \item la dipendenza che permane tra il core e le porte, in quanto cambiamenti sostanziali del core, possono richiedere una manutenzione delle porte che possono venir coinvolte nel cambiamento;
    \item la difficoltà di implementazione, in quanto l'architettura richiede una buona conoscenza dei pattern architetturali;
    \item la difficoltà di debugging, in quanto essendo il core separato, è possibile introdurre \glossario{bug} nell'attuazione dell'integrazione con servizi esterni;
\end{itemize}
\par Il gruppo ha comunque deciso di adottare questo modello architetturale, poiché ritiene che il core dell'applicazione sia sufficientemente robusto da non dover creare grossi problemi di manutenzione nel tempo. Inoltre la scalabilità e la flessibilità di questo tipo di architettura sono state considerate come un punto a favore per far fronte alla poca conoscenza della richiesta e dell'applicazione del sistema, garantendo una mitigazione di questi rischi. Infine l'architettura si sposa bene con le esigenze dell'applicazione di interfacciarsi con servizi esterni quali \glossario{database} o \glossario{LLM} e \glossario{API}.

\par La struttura organizzativa del back-end segue i principi dello stile architetturale scelto, al fine di semplificare il processo di traduzione della progettazione in codice. Il back-end è suddiviso nelle seguenti cartelle:

\begin{minipage}{\textwidth}
  \dirtree{%
    .1 backend.
    .2 adapter.
    .3 incoming.
    .4 schema\_validator.
    .3 outcoming.
    .4 db\_manager.
    .5 sql\_alchemy.
    .4 embeddings.
    .5 txtai.
    .4 file.
    .2 core.
    .3 port.
    .4 incoming.
    .4 outcoming.
    .5 embeddings.
    .3 service.
    .2 models.
    .3 dictionary\_internal\_structure.
    .3 responses.
    .2 routes.
    .3 auth.
    .2 tools.
  }
\end{minipage}
\subsubsection{Back-end}

\subsubsection{Modello architetturale}
\par Il modello architetturale scelto dal gruppo è il modello ad \textbf{architettura esagonale}. Questo modello trova le sue basi nell'architettura a livelli, dalla quale cerca di prenderne le basi e di far fronte ai problemi di dipendenza stretta che si creavano tra i livelli.
\par L'architettura è composta da 3 parti principali:
\begin{itemize}
    \item \textbf{Core}: il core è la sezione centrale dell'architettura, in cui risiede la business logic dell'applicazione. È indipendente da qualsiasi interfaccia utente o servizio esterno. Ciò viene fatto per mantenerla isolata e disaccoppiata dal resto dell'architettura, in modo che possa agire indipendentemente;
    \item \textbf{Ports}: costituiscono dei punti di comunicazione tra il core e altri parti del sistema o servizi esterni. Esistono due tipi principali di porte:
    \begin{itemize}
        \item Inbound port: consentono al nucleo di essere invocato da componenti esterne;
        \item Outbound port: consentono al nucleo di reperire informazioni dall'esterno.
    \end{itemize}
    \item \textbf{Adapters}: sono pattern architetturali che permettono di adattare i dati in modo che possano essere ricevuti dall'esterno e passare per le porte o vice versa. Costituiscono la sezione più esterna dell'architettura e consentono pertanto il passaggio e il monitoraggio dei dati.
\end{itemize}
\par In generale questa architettura è rappresentata a forma esagonale per più motivi: da una parte l'architettura vuole ricordare la cella di un'alveare. Avendo tale forma, la cella si può incastrare con altre celle allo scopo di definire un alveare. Allo stesso modo un'architettura esagonale può interfacciarsi con altre strutture con le stesse caratteristiche per definire un sistema più grande. Un altro motivo della scelta è la visualizzazione della simmetria della struttura, la quale viene visualmente divisa a metà, dove una metà si interfaccia con l'esterno per ricevere gli input, mentre l'altra si interfaccia con l'esterno per inviare informazioni in output.
\par L'architettura esagonale presenta numerosi pregi, tra i quali:
\begin{itemize}
    \item ha la facoltà di essere un'architettura flessibile, in quanto le parti di questa architettura agiscono indipendentemente tra di loro e di conseguenze si possono adattare con facilità;
    \item è scalabile in quanto una cella ad architettura esagonale può essere vista come un microservizio e per tanto più facilmente replicabile;
    \item è facilmente testabile, in quanto il core è isolato e disaccoppiato con il resto dell'architettura.
    \item è sufficientemente menutenibile, in quanto il core è isolato e disaccoppiato dal resto dell'architettura.
\end{itemize}
\par Questo tipo di architettura non è immune a difetti, tra i quali si possono trovare:
\begin{itemize}
    \item la complessità dell'implementazione, in quanto l'architettura richiede una buona strutturazione e una buona progettazione;
    \item la dipendenza che permane tra il core e le porte, in quanto cambiamenti sostanziali del core, possono richiedere una manutenzione delle porte che possono venir coinvolte nel cambiamento;
    \item la difficoltà di implementazione, in quanto l'architettura richiede una buona conoscenza dei pattern architetturali;
    \item la difficoltà di debugging, in quanto essendo il core separato, è possibile introdurre \glossario{bug} nell'attuazione dell'integrazione con servizi esterni;
\end{itemize}
\par Il gruppo ha comunque deciso di adottare questo modello architetturale, poiché ritiene che il core dell'applicazione sia sufficientemente robusto da non dover creare grossi problemi di manutenzione nel tempo. Inoltre la scalabilità e la flessibilità di questo tipo di architettura sono state considerate come un punto a favore per far fronte alla poca conoscenza della richiesta e dell'applicazione del sistema, garantendo una mitigazione di questi rischi. Infine l'architettura si sposa bene con le esigenze dell'applicazione di interfacciarsi con servizi esterni quali \glossario{database} o \glossario{LLM} e \glossario{API}.

\par La struttura organizzativa del back-end segue i principi dello stile architetturale scelto, al fine di semplificare il processo di traduzione della progettazione in codice. Il back-end è suddiviso nelle seguenti cartelle:

\begin{minipage}{\textwidth}
  \dirtree{%
    .1 backend.
    .2 adapter.
    .3 incoming.
    .4 schema\_validator.
    .3 outcoming.
    .4 db\_manager.
    .5 sql\_alchemy.
    .4 embeddings.
    .5 txtai.
    .4 file.
    .2 core.
    .3 port.
    .4 incoming.
    .4 outcoming.
    .5 embeddings.
    .3 service.
    .2 models.
    .3 dictionary\_internal\_structure.
    .3 responses.
    .2 routes.
    .3 auth.
    .2 tools.
  }
\end{minipage}
\subsubsection{Modello architetturale}
\par Il modello architetturale scelto dal gruppo è il modello ad \textbf{architettura esagonale}. Questo modello trova le sue basi nell'architettura a livelli, dalla quale cerca di prenderne le basi e di far fronte ai problemi di dipendenza stretta che si creavano tra i livelli.
\par L'architettura è composta da 3 parti principali:
\begin{itemize}
    \item \textbf{Core}: il core è la sezione centrale dell'architettura, in cui risiede la business logic dell'applicazione. È indipendente da qualsiasi interfaccia utente o servizio esterno. Ciò viene fatto per mantenerla isolata e disaccoppiata dal resto dell'architettura, in modo che possa agire indipendentemente;
    \item \textbf{Ports}: costituiscono dei punti di comunicazione tra il core e altri parti del sistema o servizi esterni. Esistono due tipi principali di porte:
    \begin{itemize}
        \item Inbound port: consentono al nucleo di essere invocato da componenti esterne;
        \item Outbound port: consentono al nucleo di reperire informazioni dall'esterno.
    \end{itemize}
    \item \textbf{Adapters}: sono pattern architetturali che permettono di adattare i dati in modo che possano essere ricevuti dall'esterno e passare per le porte o vice versa. Costituiscono la sezione più esterna dell'architettura e consentono pertanto il passaggio e il monitoraggio dei dati.
\end{itemize}
\par In generale questa architettura è rappresentata a forma esagonale per più motivi: da una parte l'architettura vuole ricordare la cella di un'alveare. Avendo tale forma, la cella si può incastrare con altre celle allo scopo di definire un alveare. Allo stesso modo un'architettura esagonale può interfacciarsi con altre strutture con le stesse caratteristiche per definire un sistema più grande. Un altro motivo della scelta è la visualizzazione della simmetria della struttura, la quale viene visualmente divisa a metà, dove una metà si interfaccia con l'esterno per ricevere gli input, mentre l'altra si interfaccia con l'esterno per inviare informazioni in output.
\par L'architettura esagonale presenta numerosi pregi, tra i quali:
\begin{itemize}
    \item ha la facoltà di essere un'architettura flessibile, in quanto le parti di questa architettura agiscono indipendentemente tra di loro e di conseguenze si possono adattare con facilità;
    \item è scalabile in quanto una cella ad architettura esagonale può essere vista come un microservizio e per tanto più facilmente replicabile;
    \item è facilmente testabile, in quanto il core è isolato e disaccoppiato con il resto dell'architettura.
    \item è sufficientemente menutenibile, in quanto il core è isolato e disaccoppiato dal resto dell'architettura.
\end{itemize}
\par Questo tipo di architettura non è immune a difetti, tra i quali si possono trovare:
\begin{itemize}
    \item la complessità dell'implementazione, in quanto l'architettura richiede una buona strutturazione e una buona progettazione;
    \item la dipendenza che permane tra il core e le porte, in quanto cambiamenti sostanziali del core, possono richiedere una manutenzione delle porte che possono venir coinvolte nel cambiamento;
    \item la difficoltà di implementazione, in quanto l'architettura richiede una buona conoscenza dei pattern architetturali;
    \item la difficoltà di debugging, in quanto essendo il core separato, è possibile introdurre \glossario{bug} nell'attuazione dell'integrazione con servizi esterni;
\end{itemize}
\par Il gruppo ha comunque deciso di adottare questo modello architetturale, poiché ritiene che il core dell'applicazione sia sufficientemente robusto da non dover creare grossi problemi di manutenzione nel tempo. Inoltre la scalabilità e la flessibilità di questo tipo di architettura sono state considerate come un punto a favore per far fronte alla poca conoscenza della richiesta e dell'applicazione del sistema, garantendo una mitigazione di questi rischi. Infine l'architettura si sposa bene con le esigenze dell'applicazione di interfacciarsi con servizi esterni quali \glossario{database} o \glossario{LLM} e \glossario{API}.
