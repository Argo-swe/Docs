\subsubsection{Incoming - Ports}

\paragraph{AuthenticationUseCase} \label{AuthenticationUseCase}
\begin{itemize}
    \item \textbf{Descrizione:} questa interfaccia permette di gestire l'autenticazione degli utenti. 
    \item \textbf{Implementazione:} questa interfaccia viene implementata dalla classe \hyperref[AuthenticationService]{(\texttt{AuthenticationService})}. 
    \item \textbf{Metodi:} i metodi dell'interfaccia sono visibili alla classe \hyperref[AuthenticationService]{(\texttt{AuthenticationService})}.
\end{itemize}  

\paragraph{DictionaryUseCase} \label{DictionaryUseCase}
\begin{itemize}
    \item \textbf{Descrizione:} questa interfaccia permette di gestire le operazioni sul \glossario{dizionario dati}.
    \item \textbf{Implementazione:} questa interfaccia viene implementata dalla classe \hyperref[DictionaryService]{\texttt{DictionaryService}}.
    \item \textbf{Metodi:} i metodi dell'interfaccia sono visibili alla classe \hyperref[DictionaryService]{(\texttt{DictionaryService})}.
\end{itemize}  

\paragraph{PromptUseCase} \label{PromptUseCase}
\begin{itemize}
    \item \textbf{Descrizione:} questa interfaccia si occupa di gestire le operazioni per la generazione dei \glossario{prompt}.
    \item \textbf{Implementazione:} questa interfaccia viene implementata dalla classe \hyperref[PromptManagementService]{(\texttt{PromptManagementService})}.
    \item \textbf{Metodi:} i metodi dell'interfaccia sono visibili alla classe \hyperref[PromptManagementService]{(\texttt{PromptManagementService})}.
\end{itemize}  

\paragraph{SchemaValidatorUseCase} \label{SchemaValidatorUseCase}
\begin{itemize}
    \item \textbf{Descrizione:} questa interfaccia si occupa della validazione dei \glossario{dizionari dati}.
    \item \textbf{Implementazione:} questa interfaccia viene implementata dalla classe \hyperref[JsonSchemaValidatorAdapter]{(\texttt{JsonSchemaValidatorAdapter})}.
    \item \textbf{Metodi:} i metodi dell'interfaccia sono visibili alla classe \hyperref[JsonSchemaValidatorAdapter]{(\texttt{JsonSchemaValidatorAdapter})}.
    \item \textbf{Dipendenze:}
    \begin{itemize}
        \item \texttt{Dictionary}
    \end{itemize}
\end{itemize}  

\subsubsection{Incoming - Adapters}

\paragraph{JsonSchemaValidatorAdapter} \label{JsonSchemaValidatorAdapter}
\begin{itemize}
    \item \textbf{Descrizione:} questa classe si occupa della validazione dei \glossario{dizionari dati}.
    \item \textbf{Implementazione:} questa classe implementa l'interfaccia \hyperref[SchemaValidatorUseCase]{\texttt{SchemaValidatorUseCase}}.
    \item \textbf{Attributi:}
    \begin{itemize}
        \item \texttt{dictionary\_schema\_file\_path: string} il percorso del file contenente lo schema dei dizionari dati.
    \end{itemize}
    \item \textbf{Metodi:}
    \begin{itemize}
        \item \texttt{JsonSchemaValidatorAdapter(dictionary\_schema\_file\_path: string)}: costruttore della classe;
        \item \texttt{+ validate(dictionary: Dictionary): bool}: valida lo schema del dizionario passato.
    \end{itemize}
    \item \textbf{Dipendenze:}
    \begin{itemize}
        \item \texttt{Utils}
    \end{itemize}
\end{itemize} 

\paragraph{SchemaValidatorFactory} \label{SchemaValidatorFactory}
\begin{itemize}
    \item \textbf{Descrizione:} questa classe si occupa della creazione di istanze della classe \texttt{JsonSchemaValidatorAdapter}.
    \item \textbf{Implementazione:} questa classe non implementa alcuna interfaccia.
    \item \textbf{Attributi:} questa classe non presenta attributi.
    \item \textbf{Metodi:}
    \begin{itemize}
        \item \texttt{+ \underline{create(config: dict)}: SchemaValidatorUseCase}: crea un'istanza della classe \texttt{JsonSchemaValidatorAdapter}.
    \end{itemize}
    \item \textbf{Dipendenze:}
    \begin{itemize}
        \item \texttt{JsonSchemaValidatorAdapter}
    \end{itemize}
\end{itemize}  