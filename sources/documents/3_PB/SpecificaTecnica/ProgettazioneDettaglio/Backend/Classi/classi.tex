\subsection{Descrizione delle classi}

\subsubsection{Introduzione}
\par La sezione seguente serve a descrivere in dettaglio le classi che compongono il \glossario{back-end} dell'applicazione. La descrizione segue classe per classe la composizione dell'architettura, dando una descrizione degli attributi e dei metodi che le caratterizzano, nonché le interfacce che implementano e le dipendenze verso altre classi del sistema.
\par La sezione viene divisa in sottosezioni che separano i servizi dalle classi che indirizzano i dati verso il core (incoming) e quelle che indirizzano i dati verso i servizi (outcoming). Inoltre le sezioni incoming e outcoming sono ulteriormente divise per distinguere le porte dagli adapters: le porte si distinguono per contenere solo le interfacce che vengono poi implementate dagli adapters. Queste non hanno né la sezione sugli attributi della classe né la sezione sulle dipendenze (a meno che queste non siano presenti). 
\input{ProgettazioneDettaglio/Backend/Classi/servizi}
\subsubsection{Incoming - Ports}

\paragraph{AuthenticationUseCase} \label{AuthenticationUseCase}
\begin{itemize}
    \item \textbf{Descrizione:} questa interfaccia permette di gestire l'autenticazione degli utenti. 
    \item \textbf{Implementazione:} questa interfaccia viene implementata dalla classe \hyperref[AuthenticationService]{(\texttt{AuthenticationService})}. 
    \item \textbf{Metodi:} i metodi dell'interfaccia sono visibili alla classe \hyperref[AuthenticationService]{(\texttt{AuthenticationService})}.
\end{itemize}  

\paragraph{DictionaryUseCase} \label{DictionaryUseCase}
\begin{itemize}
    \item \textbf{Descrizione:} questa interfaccia permette di gestire le operazioni sul \glossario{dizionario dati}.
    \item \textbf{Implementazione:} questa interfaccia viene implementata dalla classe \hyperref[DictionaryService]{\texttt{DictionaryService}}.
    \item \textbf{Metodi:} i metodi dell'interfaccia sono visibili alla classe \hyperref[DictionaryService]{(\texttt{DictionaryService})}.
\end{itemize}  

\paragraph{PromptUseCase} \label{PromptUseCase}
\begin{itemize}
    \item \textbf{Descrizione:} questa interfaccia si occupa di gestire le operazioni per la generazione dei \glossario{prompt}.
    \item \textbf{Implementazione:} questa interfaccia viene implementata dalla classe \hyperref[PromptManagementService]{(\texttt{PromptManagementService})}.
    \item \textbf{Metodi:} i metodi dell'interfaccia sono visibili alla classe \hyperref[PromptManagementService]{(\texttt{PromptManagementService})}.
\end{itemize}  

\paragraph{SchemaValidatorUseCase} \label{SchemaValidatorUseCase}
\begin{itemize}
    \item \textbf{Descrizione:} questa interfaccia si occupa della validazione dei \glossario{dizionari dati}.
    \item \textbf{Implementazione:} questa interfaccia viene implementata dalla classe \hyperref[JsonSchemaValidatorAdapter]{(\texttt{JsonSchemaValidatorAdapter})}.
    \item \textbf{Metodi:} i metodi dell'interfaccia sono visibili alla classe \hyperref[JsonSchemaValidatorAdapter]{(\texttt{JsonSchemaValidatorAdapter})}.
    \item \textbf{Dipendenze:}
    \begin{itemize}
        \item \texttt{Dictionary}
    \end{itemize}
\end{itemize}  

\subsubsection{Incoming - Adapters}

\paragraph{JsonSchemaValidatorAdapter} \label{JsonSchemaValidatorAdapter}
\begin{itemize}
    \item \textbf{Descrizione:} questa classe si occupa della validazione dei \glossario{dizionari dati}.
    \item \textbf{Implementazione:} questa classe implementa l'interfaccia \hyperref[SchemaValidatorUseCase]{\texttt{SchemaValidatorUseCase}}.
    \item \textbf{Attributi:}
    \begin{itemize}
        \item \texttt{dictionary\_schema\_file\_path: string} il percorso del file contenente lo schema dei dizionari dati.
    \end{itemize}
    \item \textbf{Metodi:}
    \begin{itemize}
        \item \texttt{JsonSchemaValidatorAdapter(dictionary\_schema\_file\_path: string)}: costruttore della classe;
        \item \texttt{+ validate(dictionary: Dictionary): bool}: valida lo schema del dizionario passato.
    \end{itemize}
    \item \textbf{Dipendenze:}
    \begin{itemize}
        \item \texttt{Utils}
    \end{itemize}
\end{itemize} 

\paragraph{SchemaValidatorFactory} \label{SchemaValidatorFactory}
\begin{itemize}
    \item \textbf{Descrizione:} questa classe si occupa della creazione di istanze della classe \texttt{JsonSchemaValidatorAdapter}.
    \item \textbf{Implementazione:} questa classe non implementa alcuna interfaccia.
    \item \textbf{Attributi:} questa classe non presenta attributi.
    \item \textbf{Metodi:}
    \begin{itemize}
        \item \texttt{+ \underline{create(config: dict)}: SchemaValidatorUseCase}: crea un'istanza della classe \texttt{JsonSchemaValidatorAdapter}.
    \end{itemize}
    \item \textbf{Dipendenze:}
    \begin{itemize}
        \item \texttt{JsonSchemaValidatorAdapter}
    \end{itemize}
\end{itemize}  
\subsubsection{Outcoming - Ports}

\paragraph{DebugManagerPort} \label{DebugManagerPort}
\begin{itemize}
    \item \textbf{Descrizione:} questa interfaccia si occupa della costruzione delle sezioni del documento di \glossario{log} per il \glossario{debug}.
    \item \textbf{Implementazione:} questa interfaccia viene implementata dalla classe \hyperref[TxtaiDebugManagerAdapter]{\texttt{TxtaiDebugManagerAdapter}}.
    \item \textbf{Metodi:} i metodi dell'interfaccia sono visibili alla classe \hyperref[TxtaiDebugManagerAdapter]{\texttt{TxtaiDebugManagerAdapter}}.
\end{itemize}  

\paragraph{EmbeddingsAbstractFactory} \label{EmbeddingsAbstractFactory}
\begin{itemize}
    \item \textbf{Descrizione:} questa interfaccia si occupa della creazione dell'index manager e delle classi per la gestione dei \glossario{prompt}.
    \item \textbf{Implementazione:} questa interfaccia viene implementata dalla classe \hyperref[TxtaiEmbeddingsManagerFactory]{\texttt{TxtaiEmbeddingsManagerFactory}}.
    \item \textbf{Metodi:} i metodi dell'interfaccia sono visibili alla classe \hyperref[TxtaiEmbeddingsManagerFactory]{\texttt{TxtaiEmbeddingsManagerFactory}}.
    \item \textbf{Dipendenze:}
    \begin{itemize}
        \item \texttt{DebugManagerPort};
        \item \texttt{IndexManagerPort};
        \item \texttt{PromptManagerPort};
    \end{itemize}
\end{itemize}  

\paragraph{IndexManagerPort} \label{IndexManagerPort}
\begin{itemize}
    \item \textbf{Descrizione:} questa interfaccia si occupa delle operazioni CRUD per gli indici degli embeddings.
    \item \textbf{Implementazione:} questa interfaccia viene implementata dalla classe \hyperref[TxtaiIndexManagerAdapter]{\texttt{TxtaiIndexManagerAdapter}}.
    \item \textbf{Metodi:} i metodi dell'interfaccia sono visibili alla classe \hyperref[TxtaiIndexManagerAdapter]{\texttt{TxtaiIndexManagerAdapter}}.
\end{itemize} 

\paragraph{PromptManagerPort} \label{PromptManagerPort}
\begin{itemize}
    \item \textbf{Descrizione:} questa interfaccia si occupa del recupero delle informazioni e la generazione del prompt.
    \item \textbf{Implementazione:} questa interfaccia viene implementata dalla classe \hyperref[TxtaiPromptManagerAdapter]{\texttt{TxtaiPromptManagerAdapter}}. 
    \item \textbf{Attributi:} FIXME (ci sono degli attributi, ma è un'interfaccia)
    \item \textbf{Metodi:} i metodi dell'interfaccia sono visibili alla classe \hyperref[TxtaiPromptManagerAdapter]{\texttt{TxtaiPromptManagerAdapter}}.
    \item \textbf{Dipendenze:}
    \begin{itemize}
        \item \texttt{IndexManagerPort};
        \item \texttt{DebugManagerPort}.
    \end{itemize}
\end{itemize} 

\paragraph{AuthenticationRepository} \label{AuthenticationRepository}
\begin{itemize}
    \item \textbf{Descrizione:} questa interfaccia si occupa del recupero degli utenti.
    \item \textbf{Implementazione:} questa interfaccia viene implementata dalla classe \hyperref[SqlAlchemyAuthenticationRepositoryAdapter]{\texttt{SqlAlchemyAuthenticationRepositoryAdapter}}.
    \item \textbf{Metodi:} i metodi dell'interfaccia sono visibili alla classe \hyperref[SqlAlchemyAuthenticationRepositoryAdapter]{\texttt{SqlAlchemyAuthenticationRepositoryAdapter}}.
\end{itemize} 

\paragraph{DbManagerAbstractFactory} \label{DbManagerAbstractFactory}
\begin{itemize}
    \item \textbf{Descrizione:} questa interfaccia si occupa della creazione di istanze delle classi \texttt{AuthenticationRepository} e \texttt{DictionaryRepository}.
    \item \textbf{Implementazione:} questa interfaccia viene implementata dalla classe \hyperref[SqlAlchemyDbManagerFactory]{\texttt{SqlAlchemyDbManagerFactory}}.
    \item \textbf{Metodi:} i metodi dell'interfaccia sono visibili alla classe \hyperref[SqlAlchemyDbManagerFactory]{\texttt{SqlAlchemyDbManagerFactory}}.
    \item \textbf{Dipendenze:}
    \begin{itemize}
        \item \texttt{AuthenticationRepository};
        \item \texttt{DictionaryRepository}.
    \end{itemize}
\end{itemize} 

\paragraph{DictionaryRepository} \label{DictionaryRepository}
\begin{itemize}
    \item \textbf{Descrizione:} questa interfaccia si occupa del recupero dei dizionari dati e delle operazioni CRUD su questi.
    \item \textbf{Implementazione:} questa interfaccia viene implementata dalla classe \hyperref[SqlAlchemyDictionaryRepositoryAdapter]{\texttt{SqlAlchemyDictionaryRepositoryAdapter}}.
    \item \textbf{Metodi:} i metodi dell'interfaccia sono visibili alla classe \hyperref[SqlAlchemyDictionaryRepositoryAdapter]{\texttt{SqlAlchemyDictionaryRepositoryAdapter}}.
\end{itemize} 

\paragraph{FileRepository} \label{FileRepository}
\begin{itemize}
    \item \textbf{Descrizione:} questa interfaccia si occupa delle operazioni CRUD sui file dei dizionari dati.
    \item \textbf{Implementazione:} questa interfaccia viene implementata dalla classe \hyperref[JsonFileAdapter]{\texttt{JsonFileAdapter}}.
    \item \textbf{Metodi:} i metodi dell'interfaccia sono visibili alla classe \hyperref[JsonFileAdapter]{\texttt{JsonFileAdapter}}.
\end{itemize}  

\subsection{Outcoming - Adapters}

\paragraph{SqlAlchemyAuthenticationRepositoryAdapter} \label{SqlAlchemyAuthenticationRepositoryAdapter}
\begin{itemize}
    \item \textbf{Descrizione:} questa classe si occupa del recupero delle informazioni sugli utenti dal database.
    \item \textbf{Implementazione:} questa classe implementa l'interfaccia \hyperref[AuthenticationRepository]{\texttt{AuthenticationRepository}}.
    \item \textbf{Attributi:}
    \begin{itemize}
        \item \texttt{session: Session}: la sessione di connessione al database.
    \end{itemize}
    \item \textbf{Metodi:}
    \begin{itemize}
        \item \texttt{SqlAlchemyAuthenticationRepositoryAdapter(session: Session)}: costruttore della classe;
        \item \texttt{+ get\_user\_by\_username(username: string): AdminDto}: recupera un admin dal database tramite il suo username.
    \end{itemize}
    \item \textbf{Dipendenze:} questa classe non presenta dipendenze.
\end{itemize} 

\paragraph{SqlAlchemyDbManagerFactory} \label{SqlAlchemyDbManagerFactory}
\begin{itemize}
    \item \textbf{Descrizione:} questa classe si occupa della costruzione delle istanze delle classi \texttt{SqlAlchemyAuthenticationRepositoryAdapter} e \texttt{SqlAlchemyDictionaryRepositoryAdapter}.
    \item \textbf{Implementazione:} questa classe implementa l'interfaccia \hyperref[DbManagerAbstractFactory]{\texttt{DbManagerAbstractFactory}}.
    \item \textbf{Attributi:} questa classe non presenta attributi.
    \item \textbf{Metodi:}
    \begin{itemize}
        \item \texttt{+ create\_authentication\_repository(): SqlAlchemyAuthenticationRepositoryAdapter} crea un'istanza della classe \texttt{SqlAlchemyAuthenticationRepositoryAdapter};
        \item \texttt{+ create\_dictionary\_repository(): SqlAlchemyDictionaryRepositoryAdapter} crea un'istanza della classe \texttt{SqlAlchemyDictionaryRepositoryAdapter}.
    \end{itemize}
    \item \textbf{Dipendenze:}
    \begin{itemize}
        \item \texttt{SqlAlchemyAuthenticationRepositoryAdapter};
        \item \texttt{SqlAlchemyDictionaryRepositoryAdapter}.
    \end{itemize}
\end{itemize}

\paragraph{SqlAlchemyDictionaryRepositoryAdapter} \label{SqlAlchemyDictionaryRepositoryAdapter}
\begin{itemize}
    \item \textbf{Descrizione:} questa classe si occupa del recupero dei dizionari dati e delle operazioni CRUD su questi.
    \item \textbf{Implementazione:} questa classe implementa l'interfaccia \hyperref[DictionaryRepository]{\texttt{DictionaryRepository}}.
    \item \textbf{Attributi:}
    \begin{itemize}
        \item \texttt{session: Session}: la sessione di connessione al database.
    \end{itemize}
    \item \textbf{Metodi:}
    \begin{itemize}
        \item \texttt{SqlAlchemyDictionaryRepositoryAdapter(session: Session)}: costruttore della classe;
        \item \texttt{+ get\_all\_dictionaries(): List[DictionaryDto]}: recupera tutti i dizionari dati dal database;
        \item \texttt{+ get\_dictionary\_by\_id(id: integer): DictionaryDto}: recupera un dizionario dato il suo id;
        \item \texttt{+ get\_dictionary\_by\_name(name: string): DictionaryDto}: recupera un dizionario dato il suo nome;
        \item \texttt{+ create\_dictionary(name: string, description: string): DictionaryDto}: crea un nuovo dizionario nel database;
        \item \texttt{+ update\_dictionary(id: integer, name: string, description: string): DictionaryDto}: aggiorna un dizionario esistente nel database, dato il suo id;
        \item \texttt{+ delete\_dictionary(id: integer): void}: elimina un dizionario esistente nel database, dato il suo id.
    \end{itemize}
    \item \textbf{Dipendenze:} questa classe non presenta dipendenze.
\end{itemize} 

\paragraph{DbManagerAbstractFactory} \label{DbManagerAbstractFactory}
\begin{itemize}
    \item \textbf{Descrizione:} questa classe si occupa della creazione di istanze della classe \texttt{DbManagerAbstractFactory}. 
    \item \textbf{Implementazione:} questa classe non implementa alcuna interfaccia. 
    \item \textbf{Attributi:} questa classe non presenta attributi.
    \item \textbf{Metodi:}
    \begin{itemize}
        \item \texttt{+ \underline{create(config: dict)}: DbManagerAbstractFactory}: crea un'istanza della classe \texttt{DbManagerAbstractFactory}.
    \end{itemize}
    \item \textbf{Dipendenze:}
    \begin{itemize}
        \item \texttt{DbManagerAbstractFactory}
    \end{itemize}
\end{itemize} 

\paragraph{TxtaiDebugManagerAdapter} \label{TxtaiDebugManagerAdapter}
\begin{itemize}
    \item \textbf{Descrizione:} questa classe si occupa della costruzione delle sezioni del documento di \glossario{log} per il \glossario{debug}.
    \item \textbf{Implementazione:} questa classe implementa l'interfaccia \hyperref[DebugManagerPort]{\texttt{DebugManagerPort}}.
    \item \textbf{Attributi:}
    \begin{itemize}
        \item \texttt{index\_manager: IndexManagerPort}: l'interfaccia di gestione degli indici;
    \end{itemize}
    \item \textbf{Metodi:}
    \begin{itemize}
        \item \texttt{TxtaiDebugManagerAdapter(index\_manager: IndexManagerPort)}: costruttore della classe;
        \item \texttt{+ semantic\_search\_log(user\_request: string, tuples: list): list} costruisce il primo pezzo documento di log, contenente le tabelle considerate nella ricerca semantica;
        \item \texttt{+ semantic\_search\_log\_custom\_algorithm(relevant\_tuples: list, tuples: list): list} costruisce il secondo pezzo del log, contenente la lista delle tabelle rilevanti;
        \item \texttt{- get\_debug\_header(): string} costruisce l'intestazione del documento di log.
    \end{itemize}
    \item \textbf{Dipendenze:}
    \begin{itemize}
        \item \texttt{IndexManagerPort}
    \end{itemize}
\end{itemize} 

\paragraph{TxtaiEmbeddingsManagerFactory} \label{TxtaiEmbeddingsManagerFactory}
\begin{itemize}
    \item \textbf{Descrizione:} questa classe si occupa della creazione dell'index manager e delle classi per la gestione dei \glossario{prompt}.
    \item \textbf{Implementazione:} questa classe implementa l'interfaccia \hyperref[EmbeddingsAbstractFactory]{\texttt{EmbeddingsAbstractFactory}}. 
    \item \textbf{Attributi:}
    \begin{itemize}
        \item \texttt{config: dict}: la configurazione del sistema;
    \end{itemize}
    \item \textbf{Metodi:}
    \begin{itemize}
        \item \texttt{TxtaiEmbeddingsManagerFactory(config: dict)}: costruttore della classe;
        \item \texttt{create\_index\_manager(file\_repository: FileRepository): TxtaiIndexManagerAdapter}: crea un'istanza della classe \texttt{TxtaiIndexManagerAdapter};
        \item \texttt{create\_prompt\_manager\_with\_dependencies(index\_manager: TxtaiIndexManagerAdapter, file\_repository: FileRepository): TxtaiPromptManagerAdapter}: crea un'istanza della classe \texttt{TxtaiPromptManagerAdapter} dato un indice esistente.
        \item \texttt{create\_prompt\_manager(file\_repository: FileRepository): TxtaiPromptManagerAdapter}: crea un'istanza della classe \texttt{TxtaiPromptManagerAdapter}, creando un nuovo indice.
    \end{itemize}
    \item \textbf{Dipendenze:}
    \begin{itemize}
        \item \texttt{FileRepository};
        \item \texttt{TxtaiIndexManagerAdapter};
    \end{itemize}
\end{itemize} 

\paragraph{TxtaiIndexManagerAdapter} \label{TxtaiIndexManagerAdapter}
\begin{itemize}
    \item \textbf{Descrizione:} questa classe si occupa delle operazioni CRUD per gli indici degli embeddings.
    \item \textbf{Implementazione:} questa classe implementa l'interfaccia \hyperref[IndexManagerPort]{\texttt{IndexManagerPort}}.
    \item \textbf{Attributi:}
    \begin{itemize}
        \item \texttt{file\_repository: FileRepository}: la repository per le operazioni sui file;
        \item \texttt{table\_path: string}: il percorso al modello di generazione delle tabelle;
        \item \texttt{column\_path: string}: il percorso al modello di generazione delle colonne;
    \end{itemize}
    \item \textbf{Metodi:}
    \begin{itemize}
        \item \texttt{TxtaiIndexManagerAdapter(file\_repository: FileRepository, table\_path: string, column\_path: string)}: costruttore della classe. Si occupa anche della costruzione dell'embedding;
    \end{itemize}
    \item \textbf{Dipendenze:}
    \begin{itemize}
        \item \texttt{FileRepository}
    \end{itemize}
\end{itemize} 

\paragraph{TxtaiPromptManagerAdapter} \label{TxtaiPromptManagerAdapter}
\begin{itemize}
    \item \textbf{Descrizione:} questa classe si occupa del recupero delle informazioni e la generazione del prompt.
    \item \textbf{Implementazione:} questa classe implementa l'interfaccia \hyperref[PromptManagerPort]{\texttt{PromptManagerPort}}.
    \item \textbf{Attributi:}
    \begin{itemize}
        \item \texttt{index\_manager: TxtaiIndexManagerAdapter}: la classe di gestione degli indici;
        \item \texttt{file\_repository: FileRepository}: la repository per le operazioni sui file.
    \end{itemize}
    \item \textbf{Metodi:}
    \begin{itemize}
        \item \texttt{+ prompt\_generator(dictionary\_id: integer, user\_request: string, lang: string, dbms: string, activate\_log: bool): tuple}: crea il prompt dato l'id del dizionario dati, una richiesta in linguaggio naturale, la lingua (inglese di default) e il dbms (MariaDB di default) scelti;
        \item \texttt{+ get\_index\_manager(): TxtaiIndexManagerAdapter}: restituisce l'istanza dell'index manager;
        \item \texttt{- create\_debug\_manager(): DebugManagerPort}: crea un'istanza del debug manager.
        \item \texttt{- get\_tuples(user\_request: string, activate\_log: bool): list}: recupera tutte le tabelle considerate nella ricerca semantica;
        \item \texttt{- get\_relevant\_tuples(tuples: list, activate\_log: bool): list}: recupera le tabelle rilevanti, cioè che superano una certa soglia di precisione, dalla ricerca semantica.
    \end{itemize}
    \item \textbf{Dipendenze:}
    \begin{itemize}
        \item \texttt{TxtaiIndexManagerAdapter};
        \item \texttt{FileRepository};
        \item \texttt{DebugManagerPort}.
    \end{itemize}
\end{itemize} 

\paragraph{EmbeddingsManagerFactory} \label{EmbeddingsManagerFactory}
\begin{itemize}
    \item \textbf{Descrizione:} questa classe si occupa della costruzione di istanze della classe \texttt{EmbeddingsAbstractFactory}.
    \item \textbf{Implementazione:} questa classe non implementa alcuna interfaccia.
    \item \textbf{Attributi:} questa classe non presenta attributi.
    \item \textbf{Metodi:}
    \begin{itemize}
        \item \texttt{+ \underline{create(config: dict)}: EmbeddingsAbstractFactory}: crea un'istanza della classe \texttt{EmbeddingsAbstractFactory}.
    \end{itemize}
    \item \textbf{Dipendenze:}
    \begin{itemize}
        \item \texttt{EmbeddingsAbstractFactory}
    \end{itemize}
\end{itemize} 

\paragraph{FileFactory} \label{FileFactory}
\begin{itemize}
    \item \textbf{Descrizione:} questa classe si occupa della creazione di istanze della classe \texttt{FileRepository}.
    \item \textbf{Implementazione:} questa classe non implementa alcuna interfaccia.
    \item \textbf{Attributi:} questa classe non presenta attributi.
    \item \textbf{Metodi:}
    \begin{itemize}
        \item \texttt{+ \underline{create(config: dict)}: FileRepository}: crea un'istanza della classe \texttt{FileRepository}.
    \end{itemize}
    \item \textbf{Dipendenze:}
    \begin{itemize}
        \item \texttt{FileRepository}
    \end{itemize}
\end{itemize} 

\paragraph{JsonFileAdapter} \label{JsonFileAdapter}
\begin{itemize}
    \item \textbf{Descrizione:} questa classe si occupa delle operazioni CRUD sui file dei dizionari dati.
    \item \textbf{Implementazione:} questa classe implementa l'interfaccia \hyperref[FileRepository]{\texttt{FileRepository}}.
    \item \textbf{Attributi:}
    \begin{itemize}
        \item \texttt{file\_path: string}: il percorso al file;
    \end{itemize}
    \item \textbf{Metodi:}
    \begin{itemize}
        \item \texttt{JsonFileAdapter(file\_path: string)}: costruttore della classe;
        \item \texttt{+ save(id: integer, file: string): void}: salva il file;
        \item \texttt{+ load(id: integer): string}: carica il file del dizionario dati dall'id specificato;
        \item \texttt{+ delete(id: integer): void}: elimina il file del dizionario dati dall'id specificato;
        \item \texttt{+ get\_preview(id: integer): Union}: recupera un'anteprima del dizionario dati;
        \item \texttt{+ extract\_index\_metadata(id: integer): list}: estrae i metadati dell'indice.
        \item \texttt{+ extract\_schema\_metadata(id: integer, tuples: list): string}: seleziona le tabelle dal dizionario dati;
        \item \texttt{+ get\_json\_schema(id: integer)}: recupera le tabelle dal dizionario dati.
        \item \texttt{- generate\_schema\_file\_name(id: integer): string}: ritorna il percorso al dizionario dati. 
    \end{itemize}
    \item \textbf{Dipendenze:}
    \begin{itemize}
        \item \texttt{Utils}
    \end{itemize}
\end{itemize} 