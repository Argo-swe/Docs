\subsubsection{Adapters - Incoming}

\paragraph{JsonSchemaValidatorAdapter} \label{JsonSchemaValidatorAdapter}
\begin{figure}[H]
    \centering
    \includegraphics[width=0.95\textwidth]{assets/Backend/json_schema_validator_adapter.png}
    \caption{Rappresentazione della classe JsonSchemaValidatorAdapter}
  \end{figure}
\begin{itemize}
    \item \textbf{Descrizione:} questa classe si occupa della validazione dei \glossario{dizionari dati}.
    \item \textbf{Implementazione:} questa classe implementa l'interfaccia \hyperref[SchemaValidatorUseCase]{\texttt{SchemaValidatorUseCase}}.
    \item \textbf{Attributi:}
    \begin{itemize}
        \item \texttt{dictionary\_schema\_file\_path: string} il percorso del file contenente lo schema dei \glossario{dizionario dati}.
    \end{itemize}
    \item \textbf{Metodi:}
    \begin{itemize}
        \item \texttt{JsonSchemaValidatorAdapter(dictionary\_schema\_file\_path: string)}: costruttore della classe;
        \item \texttt{+ validate(dictionary: Dictionary): bool}: valida lo schema del dizionario passato.
    \end{itemize}
    \item \textbf{Dipendenze:}
    \begin{itemize}
        \item \texttt{Utils}
    \end{itemize}
\end{itemize} 

\paragraph{SchemaValidatorFactory} \label{SchemaValidatorFactory}
\begin{figure}[H]
    \centering
    \includegraphics[width=0.95\textwidth]{assets/Backend/schema_validator_factory.png}
    \caption{Rappresentazione della classe SchemaValidatorFactory}
  \end{figure}
\begin{itemize}
    \item \textbf{Descrizione:} questa classe si occupa della creazione di istanze della classe \texttt{JsonSchemaValidatorAdapter}.
    \item \textbf{Implementazione:} questa classe non implementa alcuna interfaccia.
    \item \textbf{Attributi:} questa classe non presenta attributi.
    \item \textbf{Metodi:}
    \begin{itemize}
        \item \texttt{+ \underline{create(config: dict)}: SchemaValidatorUseCase}: crea un'istanza della classe \texttt{JsonSchemaValidatorAdapter}.
    \end{itemize}
    \item \textbf{Dipendenze:}
    \begin{itemize}
        \item \texttt{JsonSchemaValidatorAdapter}
    \end{itemize}
\end{itemize}  

\subsection{Outcoming - Adapters}

\paragraph{SqlAlchemyAuthenticationRepositoryAdapter} \label{SqlAlchemyAuthenticationRepositoryAdapter}
\begin{figure}[H]
    \centering
    \includegraphics[width=0.95\textwidth]{assets/Backend/sql_alchemy_authentication_repository_adapter.png}
    \caption{Rappresentazione della classe SqlAlchemyAuthenticationRepositoryAdapter}
  \end{figure}
\begin{itemize}
    \item \textbf{Descrizione:} questa classe si occupa del recupero delle informazioni sugli utenti dal database.
    \item \textbf{Implementazione:} questa classe implementa l'interfaccia \hyperref[AuthenticationRepository]{\texttt{AuthenticationRepository}}.
    \item \textbf{Attributi:}
    \begin{itemize}
        \item \texttt{session: Session}: la sessione di connessione al database.
    \end{itemize}
    \item \textbf{Metodi:}
    \begin{itemize}
        \item \texttt{SqlAlchemyAuthenticationRepositoryAdapter(session: Session)}: costruttore della classe;
        \item \texttt{+ get\_user\_by\_username(username: string): AdminDto}: recupera un admin dal database tramite il suo username.
    \end{itemize}
    \item \textbf{Dipendenze:} questa classe non presenta dipendenze.
\end{itemize} 

\paragraph{SqlAlchemyDbManagerFactory} \label{SqlAlchemyDbManagerFactory}
\begin{figure}[H]
    \centering
    \includegraphics[width=0.95\textwidth]{assets/Backend/sql_alchemy_db_manager_factory.png}
    \caption{Rappresentazione della classe SqlAlchemyDbManagerFactory}
  \end{figure}
\begin{itemize}
    \item \textbf{Descrizione:} questa classe si occupa della costruzione delle istanze delle classi \texttt{SqlAlchemyAuthenticationRepositoryAdapter} e \texttt{SqlAlchemyDictionaryRepositoryAdapter}.
    \item \textbf{Implementazione:} questa classe implementa l'interfaccia \hyperref[DbManagerAbstractFactory]{\texttt{DbManagerAbstractFactory}}.
    \item \textbf{Attributi:} questa classe non presenta attributi.
    \item \textbf{Metodi:}
    \begin{itemize}
        \item \texttt{+ create\_authentication\_repository(): AuthenticationRepository} crea un'istanza della classe \texttt{SqlAlchemyAuthenticationRepositoryAdapter};
        \item \texttt{+ create\_dictionary\_repository(): DictionaryRepository} crea un'istanza della classe \texttt{SqlAlchemyDictionaryRepositoryAdapter}.
    \end{itemize}
    \item \textbf{Dipendenze:}
    \begin{itemize}
        \item \texttt{AuthenticationRepository};
        \item \texttt{DictionaryRepository}.
    \end{itemize}
\end{itemize}

\paragraph{SqlAlchemyDictionaryRepositoryAdapter} \label{SqlAlchemyDictionaryRepositoryAdapter}
\begin{figure}[H]
    \centering
    \includegraphics[width=0.95\textwidth]{assets/Backend/sql_alchemy_dictionary_repository_adapter.png}
    \caption{Rappresentazione della classe SqlAlchemyDictionaryRepositoryAdapter}
  \end{figure}
\begin{itemize}
    \item \textbf{Descrizione:} questa classe si occupa del recupero dei dizionari dati e delle operazioni CRUD su questi.
    \item \textbf{Implementazione:} questa classe implementa l'interfaccia \hyperref[DictionaryRepository]{\texttt{DictionaryRepository}}.
    \item \textbf{Attributi:}
    \begin{itemize}
        \item \texttt{session: Session}: la sessione di connessione al database.
    \end{itemize}
    \item \textbf{Metodi:}
    \begin{itemize}
        \item \texttt{SqlAlchemyDictionaryRepositoryAdapter(session: Session)}: costruttore della classe;
        \item \texttt{+ get\_all\_dictionaries(): List[DictionaryDto]}: recupera tutti i dizionari dati dal database;
        \item \texttt{+ get\_dictionary\_by\_id(id: integer): DictionaryDto}: recupera un dizionario dato il suo id;
        \item \texttt{+ get\_dictionary\_by\_name(name: string): DictionaryDto}: recupera un dizionario dato il suo nome;
        \item \texttt{+ create\_dictionary(name: string, description: string): DictionaryDto}: crea un nuovo dizionario nel database;
        \item \texttt{+ update\_dictionary(id: integer, name: string, description: string): DictionaryDto}: aggiorna un dizionario esistente nel database, dato il suo id;
        \item \texttt{+ delete\_dictionary(id: integer): void}: elimina un dizionario esistente nel database, dato il suo id.
    \end{itemize}
    \item \textbf{Dipendenze:} questa classe non presenta dipendenze.
\end{itemize} 

\paragraph{DbManagerFactory} \label{DbManagerFactory}
\begin{figure}[H]
    \centering
    \includegraphics[width=0.95\textwidth]{assets/Backend/db_manager_factory.png}
    \caption{Rappresentazione della classe DbManagerFactory}
  \end{figure}
\begin{itemize}
    \item \textbf{Descrizione:} questa classe si occupa della creazione delle istanze della classe \texttt{DbManagerAbstractFactory} 
    \item Questa classe non implementa alcuna interfaccia.
    \item \textbf{Attributi:} questa classe non presenta attributi.
    \item \textbf{Metodi:}
    \begin{itemize}
        \item \texttt{+ create(config: dict): DbManagerAbstractFactory}: crea un'istanza della classe \texttt{DbManagerAbstractFactory}.
    \end{itemize}
    \item \textbf{Dipendenze:}
    \begin{itemize}
        \item \texttt{DbManagerAbstractFactory}
    \end{itemize}
\end{itemize}

\paragraph{TxtaiDebugManagerAdapter} \label{TxtaiDebugManagerAdapter}
\begin{figure}[H]
    \centering
    \includegraphics[width=0.95\textwidth]{assets/Backend/txtai_debug_manager_adapter.png}
    \caption{Rappresentazione della classe TxtaiDebugManagerAdapter}
  \end{figure}
\begin{itemize}
    \item \textbf{Descrizione:} questa classe si occupa della costruzione delle sezioni del documento di \glossario{log} per il \glossario{debug}.
    \item \textbf{Implementazione:} questa classe implementa l'interfaccia \hyperref[DebugManagerPort]{\texttt{DebugManagerPort}}.
    \item \textbf{Attributi:}
    \begin{itemize}
        \item \texttt{index\_manager: IndexManagerPort}: l'interfaccia di gestione degli indici;
    \end{itemize}
    \item \textbf{Metodi:}
    \begin{itemize}
        \item \texttt{TxtaiDebugManagerAdapter(index\_manager: IndexManagerPort)}: costruttore della classe;
        \item \texttt{+ semantic\_search\_log(user\_request: string, tuples: list): list} costruisce il primo pezzo documento di log, contenente le tabelle considerate nella ricerca semantica;
        \item \texttt{+ semantic\_search\_log\_custom\_algorithm(relevant\_tuples: list, tuples: list): list} costruisce il secondo pezzo del log, contenente la lista delle tabelle rilevanti;
        \item \texttt{- get\_debug\_header(): string} costruisce l'intestazione del documento di log.
    \end{itemize}
    \item \textbf{Dipendenze:}
    \begin{itemize}
        \item \texttt{IndexManagerPort}
    \end{itemize}
\end{itemize} 

\paragraph{TxtaiEmbeddingsManagerFactory} \label{TxtaiEmbeddingsManagerFactory}
\begin{figure}[H]
    \centering
    \includegraphics[width=0.95\textwidth]{assets/Backend/txtai_embeddings_manager_factory.png}
    \caption{Rappresentazione della classe TxtaiEmbeddingsManagerFactory}
  \end{figure}
\begin{itemize}
    \item \textbf{Descrizione:} questa classe si occupa della creazione dell'index manager e delle classi per la gestione dei \glossario{prompt}.
    \item \textbf{Implementazione:} questa classe implementa l'interfaccia \hyperref[EmbeddingsAbstractFactory]{\texttt{EmbeddingsAbstractFactory}}. 
    \item \textbf{Attributi:}
    \begin{itemize}
        \item \texttt{config: dict}: la configurazione del sistema;
    \end{itemize}
    \item \textbf{Metodi:}
    \begin{itemize}
        % Nella spiegazione lascio che l'istanza che viene creata è di tipo TxtaiEmbeddingsManagerFactory, ma in realtà il tipo di ritorno è l'interfaccia
        \item \texttt{TxtaiEmbeddingsManagerFactory(config: dict)}: costruttore della classe;
        \item \texttt{create\_index\_manager(file\_repository: FileRepository): IndexManagerPort}: crea un'istanza della classe \texttt{TxtaiIndexManagerAdapter};
        \item \texttt{create\_prompt\_manager\_with\_dependencies(index\_manager: IndexManagerPort, file\_repository: FileRepository): TxtaiPromptManagerAdapter}: crea un'istanza della classe \texttt{TxtaiPromptManagerAdapter} dato un indice esistente.
        \item \texttt{create\_prompt\_manager(file\_repository: FileRepository): IndexManagerPort}: crea un'istanza della classe \texttt{TxtaiPromptManagerAdapter}, creando un nuovo indice.
    \end{itemize}
    \item \textbf{Dipendenze:}
    \begin{itemize}
        % Come dipendenza la lasio verso l'interfaccia come da diagramma
        \item \texttt{FileRepository};
        \item \texttt{IndexManagerPort};
    \end{itemize}
\end{itemize} 

\paragraph{TxtaiIndexManagerAdapter} \label{TxtaiIndexManagerAdapter}
\begin{figure}[H]
    \centering
    \includegraphics[width=0.95\textwidth]{assets/Backend/txtai_index_manager_adapter.png}
    \caption{Rappresentazione della classe TxtaiIndexManagerAdapter}
  \end{figure}
\begin{itemize}
    \item \textbf{Descrizione:} questa classe si occupa delle operazioni CRUD per gli indici degli embeddings.
    \item \textbf{Implementazione:} questa classe implementa l'interfaccia \hyperref[IndexManagerPort]{\texttt{IndexManagerPort}}.
    \item \textbf{Attributi:}
    \begin{itemize}
        \item \texttt{file\_repository: FileRepository}: la repository per le operazioni sui file;
        \item \texttt{table\_path: string}: il percorso al modello di generazione delle tabelle;
        \item \texttt{column\_path: string}: il percorso al modello di generazione delle colonne;
    \end{itemize}
    \item \textbf{Metodi:}
    \begin{itemize}
        \item \texttt{TxtaiIndexManagerAdapter(file\_repository: FileRepository, table\_path: string, column\_path: string)}: costruttore della classe. Si occupa anche della costruzione dell'embedding;
    \end{itemize}
    \item \textbf{Dipendenze:}
    \begin{itemize}
        \item \texttt{FileRepository}
    \end{itemize}
\end{itemize} 

\paragraph{TxtaiPromptManagerAdapter} \label{TxtaiPromptManagerAdapter}
\begin{figure}[H]
    \centering
    \includegraphics[width=0.95\textwidth]{assets/Backend/txtai_prompt_manager_adapter.png}
    \caption{Rappresentazione della classe TxtaiPromptManagerAdapter}
  \end{figure}
\begin{itemize}
    \item \textbf{Descrizione:} questa classe si occupa del recupero delle informazioni e la generazione del prompt.
    \item \textbf{Implementazione:} questa classe implementa l'interfaccia \hyperref[PromptManagerPort]{\texttt{PromptManagerPort}}.
    \item \textbf{Attributi:}
    \begin{itemize}
        \item \texttt{index\_manager: IndexManagerPort}: la classe di gestione degli indici;
        \item \texttt{file\_repository: FileRepository}: la repository per le operazioni sui file.
    \end{itemize}
    \item \textbf{Metodi:}
    \begin{itemize}
        \item \texttt{+ prompt\_generator(dictionary\_id: integer, user\_request: string, lang: string, dbms: string, activate\_log: bool): tuple}: crea il prompt dato l'id del dizionario dati, una richiesta in linguaggio naturale, la lingua (inglese di default) e il dbms (MariaDB di default) scelti;
        \item \texttt{+ get\_index\_manager(): IndexManagerPort}: restituisce l'istanza dell'index manager;
        \item \texttt{- create\_debug\_manager(): DebugManagerPort}: crea un'istanza del debug manager.
        \item \texttt{- get\_tuples(user\_request: string, activate\_log: bool): list}: recupera tutte le tabelle considerate nella ricerca semantica;
        \item \texttt{- get\_relevant\_tuples(tuples: list, activate\_log: bool): list}: recupera le tabelle rilevanti, cioè che superano una certa soglia di precisione, dalla ricerca semantica.
    \end{itemize}
    \item \textbf{Dipendenze:}
    \begin{itemize}
        \item \texttt{IndexManagerPort};
        \item \texttt{FileRepository};
        \item \texttt{DebugManagerPort}.
    \end{itemize}
\end{itemize} 

\paragraph{EmbeddingsManagerFactory} \label{EmbeddingsManagerFactory}
\begin{figure}[H]
    \centering
    \includegraphics[width=0.95\textwidth]{assets/Backend/embeddings_manager_factory.png}
    \caption{Rappresentazione della classe EmbeddingsManagerFactory}
  \end{figure}
\begin{itemize}
    \item \textbf{Descrizione:} questa classe si occupa della costruzione di istanze della classe \texttt{EmbeddingsAbstractFactory}.
    \item \textbf{Implementazione:} questa classe non implementa alcuna interfaccia.
    \item \textbf{Attributi:} questa classe non presenta attributi.
    \item \textbf{Metodi:}
    \begin{itemize}
        \item \texttt{+ \underline{create(config: dict)}: EmbeddingsAbstractFactory}: crea un'istanza della classe \texttt{EmbeddingsAbstractFactory}.
    \end{itemize}
    \item \textbf{Dipendenze:}
    \begin{itemize}
        \item \texttt{EmbeddingsAbstractFactory}
    \end{itemize}
\end{itemize} 

\paragraph{FileFactory} \label{FileFactory}
\begin{figure}[H]
    \centering
    \includegraphics[width=0.95\textwidth]{assets/Backend/file_factory.png}
    \caption{Rappresentazione della classe FileFactory}
  \end{figure}
\begin{itemize}
    \item \textbf{Descrizione:} questa classe si occupa della creazione di istanze della classe \texttt{FileRepository}.
    \item \textbf{Implementazione:} questa classe non implementa alcuna interfaccia.
    \item \textbf{Attributi:} questa classe non presenta attributi.
    \item \textbf{Metodi:}
    \begin{itemize}
        \item \texttt{+ \underline{create(config: dict)}: FileRepository}: crea un'istanza della classe \texttt{FileRepository}.
    \end{itemize}
    \item \textbf{Dipendenze:}
    \begin{itemize}
        \item \texttt{FileRepository}
    \end{itemize}
\end{itemize} 

\paragraph{JsonFileAdapter} \label{JsonFileAdapter}
\begin{figure}[H]
    \centering
    \includegraphics[width=0.95\textwidth]{assets/Backend/json_file_adapter.png}
    \caption{Rappresentazione della classe JsonFileAdapter}
  \end{figure}
\begin{itemize}
    \item \textbf{Descrizione:} questa classe si occupa delle operazioni CRUD sui file dei dizionari dati.
    \item \textbf{Implementazione:} questa classe implementa l'interfaccia \hyperref[FileRepository]{\texttt{FileRepository}}.
    \item \textbf{Attributi:}
    \begin{itemize}
        \item \texttt{file\_path: string}: il percorso al file;
    \end{itemize}
    \item \textbf{Metodi:}
    \begin{itemize}
        \item \texttt{JsonFileAdapter(file\_path: string)}: costruttore della classe;
        \item \texttt{+ save(id: integer, file: string): void}: salva il file;
        \item \texttt{+ load(id: integer): string}: carica il file del dizionario dati dall'id specificato;
        \item \texttt{+ delete(id: integer): void}: elimina il file del dizionario dati dall'id specificato;
        \item \texttt{+ get\_preview(id: integer): Union}: recupera un'anteprima del dizionario dati;
        \item \texttt{+ extract\_index\_metadata(id: integer): list}: estrae i metadati dell'indice.
        \item \texttt{+ extract\_schema\_metadata(id: integer, tuples: list): string}: seleziona le tabelle dal dizionario dati;
        \item \texttt{+ get\_json\_schema(id: integer)}: recupera le tabelle dal dizionario dati.
        \item \texttt{- generate\_schema\_file\_name(id: integer): string}: ritorna il percorso al dizionario dati. 
    \end{itemize}
    \item \textbf{Dipendenze:}
    \begin{itemize}
        \item \texttt{Utils}
    \end{itemize}
\end{itemize} 