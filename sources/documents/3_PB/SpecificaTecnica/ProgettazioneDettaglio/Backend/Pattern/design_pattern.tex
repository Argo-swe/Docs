\subsubsection{Design patterns}
\paragraph{Introduzione}
\par La seguente sezione serve a elencare i design patterns utilizzati durante lo sviluppo dell'architettura. Si occupa inoltre di dare una loro descrizione e spiegare i motivi che hanno spinto il gruppo a sceglierli per lo sviluppo.

\paragraph{Abstract Factory}
\begin{itemize}
    \item{\textbf{Descrizione:}} il pattern Abstract Factory è un design pattern creazionale che fornisce un'interfaccia per creare famiglie di oggetti correlati o dipendenti senza specificarne le classi concrete.
    \item{\textbf{Utilizzo:}}
    \begin{itemize}
        \item \texttt{DbManagerFactory}
        \item \texttt{FileFactory}
        \item \texttt{SchemaValidatorFactory}
        \item \texttt{EmbeddingsManagerFactory}
    \end{itemize}
\end{itemize}

\paragraph{Dependency Injection}
\begin{itemize}
    \item{\textbf{Descrizione:}} il pattern Dependency Injection è un design pattern comportamentale che permette di creare oggetti con le loro dipendenze, invece di creare oggetti con le dipendenze già istanziate.
    \item{\textbf{Utilizzo:}}
    \begin{itemize}
        \item Classi che utilizzano il pattern
    \end{itemize}
\end{itemize}

\paragraph{Facade}
\begin{itemize}
    \item{\textbf{Descrizione:}} il pattern Facade è un design pattern strutturale che fornisce un'interfaccia unificata per un insieme di interfacce in un sottosistema. Il pattern definisce un'interfaccia di alto livello che semplifica l'uso del sottosistema.
    \item{\textbf{Utilizzo:}} Le classi che utilizzano questo pattern sono tutte le interfacce che sono implementate da una classe concreta. Questo è stato fatto per permettere di dirigere le dipendenze verso interfacce uniche che consentano una manutenibilità più semplice (bozza).
    \begin{itemize}
        \item \texttt{AuthenticationUseCase}
        \item \texttt{DictionaryUseCase}
        \item \texttt{IndexManagerPort}
        \item \texttt{SchemaValidatorUseCase}
        \item \texttt{PromptManagerPort}
        \item \texttt{DebugManagerPort}
        \item \texttt{EmbeddingsAbstractFactory}
    \end{itemize}
\end{itemize}

\paragraph{Strategy}
\begin{itemize}
    \item{\textbf{Descrizione:}} il pattern Strategy è un design pattern comportamentale che definisce una famiglia di algoritmi, incapsula ciascuno di essi e li rende intercambiabili. Il pattern Strategy permette all'algoritmo di variare indipendentemente dai client che lo utilizzano.
    \item{\textbf{Utilizzo:}}
    \begin{itemize}
        \item Classi che utilizzano il pattern
    \end{itemize}
\end{itemize}

\paragraph{Adapters}
\begin{itemize}
    \item{\textbf{Descrizione:}} il pattern Adapters è un design pattern strutturale che consente a oggetti con tipi dati incompatibili di collaborare.
    \item{\textbf{Utilizzo:}}
    \begin{itemize}
        \item Classi che utilizzano il pattern
    \end{itemize}
\end{itemize}

\paragraph{Ports}
\begin{itemize}
    \item{\textbf{Descrizione:}} il pattern Ports è un design pattern architetturale che permette di separare le interfacce di input e output dal core dell'applicazione.
    \item{\textbf{Utilizzo:}}
    \begin{itemize}
        \item \texttt{AuthenticationUseCase}
        \item \texttt{DictionaryUseCase}
        \item \texttt{PromptUseCase}
        \item \texttt{SchemaValidatorUseCase}
        \item \texttt{PromptManagerPort}
        \item \texttt{EmbeddingsAbstractFactory}
        \item \texttt{DebugManagerPort}
        \item \texttt{IndexManagerPort}
        \item \texttt{AuthenticationRepository}
        \item \texttt{DictionaryRepository}
        \item \texttt{FileRepository}
        \item \texttt{DbManagerAbstractFactory}
    \end{itemize}
\end{itemize}

\paragraph{Services}
\begin{itemize}
    \item{\textbf{Descrizione:}} il pattern Services è un design pattern architetturale che permette di separare la logica di business dal resto dell'applicazione.
    \item{\textbf{Utilizzo:}}
    \begin{itemize}
        \item \texttt{DictionaryService}
        \item \texttt{PromptManagerService}
        \item \texttt{AuthenticationService}
    \end{itemize}
\end{itemize}

\paragraph{Repository}
\begin{itemize}
    \item{\textbf{Descrizione:}} il pattern Repository è un design pattern architetturale che permette di separare la logica di accesso ai dati dal resto dell'applicazione.
    \item{\textbf{Utilizzo:}}
    \begin{itemize}
        \item \texttt{DictionaryRepository}
        \item \texttt{AuthenticationRepository}
        \item \texttt{FileRepository}
    \end{itemize}
\end{itemize}