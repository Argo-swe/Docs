\subsubsection{Design patterns}
\paragraph{Introduzione}
\par La seguente sezione serve a elencare i design patterns utilizzati durante lo sviluppo dell'architettura. Si occupa inoltre di dare una loro descrizione e spiegare i motivi che hanno spinto il gruppo a sceglierli per lo sviluppo.

\paragraph{Singleton}
\begin{itemize}
    \item{\textbf{Descrizione:}} il pattern Singleton è un design pattern creazionale che garantisce che una classe abbia una sola istanza e fornisce un punto di accesso globale a tale istanza.
    \item{\textbf{Utilizzo:}}
    \begin{itemize}
        \item Classi che utilizzano il pattern
    \end{itemize}
\end{itemize}

\paragraph{Abstract Factory}
\begin{itemize}
    \item{\textbf{Descrizione:}} il pattern Abstract Factory è un design pattern creazionale che fornisce un'interfaccia per creare famiglie di oggetti correlati o dipendenti senza specificarne le classi concrete.
    \item{\textbf{Utilizzo:}}
    \begin{itemize}
        \item Classi che utilizzano il pattern
    \end{itemize}
\end{itemize}

\paragraph{Dependency Injection}
\begin{itemize}
    \item{\textbf{Descrizione:}} il pattern Dependency Injection è un design pattern comportamentale che permette di creare oggetti con le loro dipendenze, invece di creare oggetti con le dipendenze già istanziate.
    \item{\textbf{Utilizzo:}}
    \begin{itemize}
        \item Classi che utilizzano il pattern
    \end{itemize}
\end{itemize}

\paragraph{Facade}
\begin{itemize}
    \item{\textbf{Descrizione:}} il pattern Facade è un design pattern strutturale che fornisce un'interfaccia unificata per un insieme di interfacce in un sottosistema. Il pattern definisce un'interfaccia di alto livello che semplifica l'uso del sottosistema.
    \item{\textbf{Utilizzo:}}
    \begin{itemize}
        \item Classi che utilizzano il pattern
    \end{itemize}
\end{itemize}

\paragraph{Strategy}
\begin{itemize}
    \item{\textbf{Descrizione:}} il pattern Strategy è un design pattern comportamentale che definisce una famiglia di algoritmi, incapsula ciascuno di essi e li rende intercambiabili. Il pattern Strategy permette all'algoritmo di variare indipendentemente dai client che lo utilizzano.
    \item{\textbf{Utilizzo:}}
    \begin{itemize}
        \item Classi che utilizzano il pattern
    \end{itemize}
\end{itemize}