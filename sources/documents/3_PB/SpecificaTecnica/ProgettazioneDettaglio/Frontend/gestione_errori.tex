\subsection{Gestione degli errori} \label{GestioneErrori}

\par Il gruppo ha introdotto una gestione centralizzata degli errori nel \glossario{front-end}. Il file \textit{status-messages.ts} si occupa della traduzione e della generazione di messaggi di stato personalizzati. Il team ha definito un set di messaggi per operazioni comuni come:
\begin{itemize}
  \item Autenticazione;
  \item Generazione di \glossario{prompt};
  \item Creazione, aggiornamento, cancellazione e recupero dei \glossario{dizionari dati}.
\end{itemize}

\vspace{0.5\baselineskip}
\par Il file \textit{status-messages.ts} utilizza le interfacce e i tipi definiti in \textit{wrapper.ts}, in particolare StatusMessages, che svolge due compiti principali:
\begin{itemize}
  \item Rappresenta un dizionario di messaggi organizzati in base ai codici di stato (analoghi a quelli delle risposte HTTP) definiti nell'enumerazione presente in \textit{openapi.d.ts};
  \item Contiene funzioni che generano messaggi specifici basati sulle opzioni fornite.
\end{itemize}