\subsection{Componenti} \label{Componenti}

% App Layout
\subsubsection{AppLayout}

\paragraph*{Elementi chiave}
\begin{itemize}
  \item Layout Wrapper: aggiorna lo stile dell'applicazione in base alla configurazione del sistema;
  \item Outside Click Listener: i menu vengono chiusi quando viene effettuato un clic al di fuori di essi;
  \item Scroll To Top: pulsante per tornare rapidamente all'inizio della pagina;
  \item Toast Message: messaggio di notifica per fornire feedback all'utente.
\end{itemize}

% Chat View
\subsubsection{ChatView}

\paragraph*{Elementi chiave}
\begin{itemize}
  \item Selected Dictionary Name: nome ed estensione del \glossario{dizionario dati} selezionato dall'utente;
  \item Toggle Select View: pulsante per visualizzare o nascondere il form di selezione del dizionario dati, della lingua e del \glossario{DBMS};
  \item Select Dictionary Dropdown: menu a discesa per selezionare un dizionario dati;
  \item View Details: pulsante per visualizzare l'anteprima di un dizionario dati;
  \item Select DBMS Dropdown: menu a discesa per selezionare un DBMS;
  \item Select Language Dropdown: menu a discesa per selezionare una lingua;
  \item Chat Container: contenitore per i messaggi della chat;
  \item Scroll To Bottom: pulsante per scorrere fino all'ultimo messaggio della chat;
  \item Chat Textarea: campo di testo per inserire una richiesta da inviare al ChatBOT;
  \item Request Button: pulsante per inviare la richiesta;
  \item Loading Spinner: indica che la generazione del \glossario{prompt} è ancora in corso.
\end{itemize}

% Dictionaries List View
\subsubsection{DictionariesListView}

\paragraph*{Elementi chiave}
\begin{itemize}
  \item Create Dictionary Button: pulsante per aprire la finestra di inserimento di un \glossario{dizionario dati};
  \item Search By Name Input: campo di testo per filtrare i dizionari dati in base al nome;
  \item Search By Description Input: campo di testo per filtrare i dizionari dati in base alla descrizione;
  \item Dictionary Data Table: tabella contenente i seguenti campi:
  \begin{itemize}
    \item Dictionary Name: nome del dizionario dati;
    \item Dictionary Description: descrizione del dizionario dati;
    \item Update Metadata Button: pulsante per aprire la finestra di modifica dei metadati (nome e descrizione);
    \item Update File Button: pulsante per aprire la finestra di modifica del file;
    \item Download File Button: pulsante per scaricare il file relativo a un dizionario dati;
    \item Delete Dictionary Button: pulsante per eliminare un dizionario dati.
  \end{itemize}
  \item Status Message: messaggio per notificare all'utente che:
  \begin{itemize}
    \item L'eliminazione del dizionario dati è avvenuta con successo;
    \item L'eliminazione del dizionario dati è fallita.
  \end{itemize}
\end{itemize}

% Login Dialog
\subsubsection{LoginDialog}

\paragraph*{Elementi chiave}
\begin{itemize}
  \item Dialog: finestra per l'autenticazione dell'utente. La finestra interattiva appare sopra il contenuto principale dell'interfaccia (in overlay);
  \item Login Form: modulo per l'inserimento delle credenziali di accesso;
  \item Username Input: campo di testo per l'inserimento dello username;
  \item Password Input: campo di testo per l'inserimento della password;
  \item V-Model: crea un binding bidirezionale tra i campi di input e le rispettive variabili;
  \item Login Button: pulsante per richiedere l'accesso all'area di amministrazione;
  \item Status Message: messaggio per notificare all'utente che:
  \begin{itemize}
    \item Il login è avvenuto con successo;
    \item L'utente non è autorizzato ad accedere all'area di amministrazione;
    \item Le credenziali inserite non sono corrette;
    \item Il server non è raggiungibile.
  \end{itemize}
\end{itemize}

% Chat Message
\subsubsection{ChatMessage}

\paragraph*{Props}
\begin{itemize}
  \item Message: testo del messaggio;
  \item IsSent: flag per indicare se il messaggio è stato inviato o ricevuto dall'utente;
  \item Debug (opzionale): contenuto del messaggio di \glossario{debug}.
\end{itemize}

\paragraph*{Elementi chiave}
\begin{itemize}
  \item Message Card: contenitore per il messaggio. La disposizione e il colore del contenitore variano in base al mittente del messaggio;
  \item Avatar: icona del mittente. L'avatar è posizionato a sinistra o a destra del messaggio in base al mittente;
  \item Message Text: testo del messaggio;
  \item Copy Button: pulsante per copiare il testo del messaggio negli Appunti di sistema;
  \item Debug Button: pulsante per aprire la finestra modale contenente il messaggio di debug. Il pulsante è disponibile solo per il profilo Tecnico.
\end{itemize}

% String Data Modal
\subsubsection{StringDataModal}

\paragraph*{Elementi chiave}
\begin{itemize}
  \item Modal: finestra modale responsive;
  \item Close Dialog: pulsante per chiudere la finestra modale;
  \item String Data: stringa di testo da visualizzare nel dialog.
\end{itemize}

% Debug Message
\subsubsection{DebugMessage}

\paragraph*{Props}
\begin{itemize}
  \item Message: testo del messaggio.
\end{itemize}

\paragraph*{Elementi chiave}
\begin{itemize}
  \item Message Text: testo del messaggio;
  \item Download Button: pulsante per scaricare un file di \glossario{log}.
\end{itemize}

% Dict Preview
\subsubsection{DictPreview}

\paragraph*{Props}
\begin{itemize}
  \item DetailsVisible: flag per visualizzare o nascondere l'anteprima del \glossario{dizionario dati};
  \item DictionaryPreview: anteprima del dizionario dati, contenente le seguenti informazioni:
  \begin{itemize}
    \item Database Name: nome del dizionario dati;
    \item Database Description: descrizione del dizionario dati;
    \item Lista delle tabelle: 
    \begin{itemize}
      \item Table Name: nome della tabella;
      \item Table Description: descrizione della tabella.
    \end{itemize}
  \end{itemize}
\end{itemize}

\paragraph*{Emits}
\begin{itemize}
  \item HideDetails: evento per notificare al componente genitore che l'anteprima del dizionario dati deve essere chiusa.
\end{itemize}

\paragraph*{Elementi chiave}
\begin{itemize}
  \item Preview Card: contenitore per l'anteprima del dizionario dati;
  \item Expand Button: pulsante per aumentare la dimensione della Preview Card. Il pulsante di espansione migliora l'accessibilità e l'usabilità;
  \item Collapse Button: pulsante per ridurre la dimensione della Preview Card;
  \item Close Button: pulsante per chiudere l'anteprima del dizionario dati.
\end{itemize}

% Chat Delete Btn
\subsubsection{ChatDeleteBtn}

\paragraph*{Props}
\begin{itemize}
  \item Messages: contenuto della chat da eliminare;
  \item Loading: flag per indicare se il ChatBOT sta elaborando un nuovo messaggio.
\end{itemize}

\paragraph*{Emits}
\begin{itemize}
  \item ClearMessages: evento per notificare al componente genitore che i messaggi devono essere cancellati.
\end{itemize}

\paragraph*{Elementi chiave}
\begin{itemize}
  \item Toggle: il pulsante di cancellazione deve essere nascosto se:
  \begin{itemize}
    \item Non ci sono messaggi da cancellare;
    \item Il ChatBOT sta elaborando un nuovo messaggio.
  \end{itemize}
\end{itemize}

% Create Update Dictionary Modal 
\subsubsection{CreateUpdateDictionaryModal}

\paragraph*{Elementi chiave}
\begin{itemize}
  \item Close Dialog: pulsante per chiudere la finestra modale;
  \item Dictionary Name Input: campo di testo per inserire o modificare il nome di un \glossario{dizionario dati};
  \item Dictionary Description Input: campo di testo per inserire o modificare la descrizione di un dizionario dati;
  \item File Uploader: componente per caricare il file relativo a un dizionario dati;
  \item Clear File Button: pulsante per annullare il caricamento del file;
  \item Submit Button: pulsante per richiedere il salvataggio dei dati;
  \item Loading Spinner: indica che il salvataggio del dizionario dati e del rispettivo \glossario{indice} è ancora in corso;
  \item Status Message: messaggio per notificare all'utente che:
  \begin{itemize}
    \item L'inserimento del dizionario è avvenuto con successo;
    \item L'inserimento del dizionario è fallito;
    \item La modifica dei metadati del dizionario è avvenuta con successo;
    \item La modifica dei metadati del dizionario è fallita;
    \item La modifica del file associato al dizionario è avvenuta con successo;
    \item La modifica del file associato al dizionario è fallita.
  \end{itemize}
\end{itemize}

% App Topbar
\subsubsection{AppTopbar}

\paragraph*{Elementi chiave}
\begin{itemize}
  \item Logo: logo dell'applicazione, affiancato dal nome del progetto (per migliorare l'accessibilità);
  \item Open Main Nav Button: pulsante per aprire il menu di navigazione principale;
  \item Open Config Menu Button: pulsante per aprire il menu di configurazione del sistema;
  \item Open Settings Button: pulsante per aprire la barra laterale di impostazioni;
  \item Open Login Dialog Button: pulsante per aprire la finestra di autenticazione;
  \item Logout Button: pulsante per effettuare il logout;
  \item Confirmation Dialog: finestra per confermare o rifiutare il logout;
  \item Outside Click Listener: i menu vengono chiusi quando viene effettuato un clic al di fuori di essi.
\end{itemize}

% Menu Sidebar
\subsubsection{MenuSidebar}

\paragraph*{Elementi chiave}
\begin{itemize}
  \item Close Main Nav Button: pulsante per chiudere il menu di navigazione principale;
  \item App Menu: menu di navigazione principale;
  \item App Footer: footer dell'applicazione.
\end{itemize}

% Config Sidebar
\subsubsection{ConfigSidebar}

\paragraph*{Elementi chiave}
\begin{itemize}
  \item Decrement Scale Button: pulsante per diminuire la dimensione degli elementi nell'interfaccia;
  \item Increment Scale Button: pulsante per ingrandire la dimensione degli elementi nell'interfaccia;
  \item Change Theme Switch: interruttore per alternare il tema dell'interfaccia tra la modalità chiara e quella scura;
  \item Change Language Dropdown: menu a discesa per selezionare la lingua dell'interfaccia.
\end{itemize}

% App Menu
\subsubsection{AppMenu}

\paragraph*{Elementi chiave}
\begin{itemize}
  \item Layout Menu: contenitore per il menu di navigazione principale;
  \item Menu Items: voci di menu per accedere alle diverse sezioni dell'applicazione;
  \item Menu Icon: stringa contenente la classe da applicare all'icona della voce di menu;
  \item Menu Label: stringa contenente l'etichetta della voce di menu;
  \item Menu Link: stringa contenente il percorso della voce di menu.
\end{itemize}

% App Menu Item
\subsubsection{AppMenuItem}

\paragraph*{Props}
\begin{itemize}
  \item Item: oggetto che rappresenta una voce di menu;
  \item Index: indice della voce di menu;
  \item Root: flag per indicare se la voce di menu è la radice del menu di navigazione;
  \item Parent Item Key: chiave per identificare la voce di menu genitore.
\end{itemize}

\paragraph*{Elementi chiave}
\begin{itemize}
  \item Menu Item: contenitore per una voce di menu;
  \item Menu Icon: icona per identificare la voce di menu;
  \item Menu Label: etichetta per identificare la voce di menu;
  \item Menu Link: collegamento esterno (non gestito da Vue Router) o interno (gestito da Vue Router). Un link può essere anche un elemento cliccabile per aprire un sotto-menu;
  \item Sub Menu: contenitore per le voci di menu figlie.
\end{itemize}

% App Logo
\subsubsection{AppLogo}

\paragraph*{Props}
\begin{itemize}
  \item Width (opzionale): variabile per impostare la larghezza del logo;
  \item Height (opzionale): variabile per impostare l'altezza del logo;
  \item Path: variabile per impostare il percorso dell'immagine.
\end{itemize}

\paragraph*{Elementi chiave}
\begin{itemize}
  \item Filter: funzionalità per invertire il colore del logo da bianco a nero (in base al tema attivo).
\end{itemize}

% App Footer
\subsubsection{AppFooter}

\paragraph*{Elementi chiave}
\begin{itemize}
  \item Logo: logo dell'applicazione;
  \item Version: numero di versione dell'applicazione;
  \item Copyright section: sezione contenente le seguenti informazioni:
  \begin{itemize}
    \item Year: anno corrente;
    \item Group Name: nome del team di sviluppo;
    \item Copyright: prova di diritto d'autore.
  \end{itemize}
\end{itemize}