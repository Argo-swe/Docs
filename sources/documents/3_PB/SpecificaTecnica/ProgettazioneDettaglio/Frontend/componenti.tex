\subsection{Front-end}

\subsubsection{LoginDialog}

\paragraph*{Elementi chiave}
\begin{itemize}
  \item Dialog: finestra per l'autenticazione dell'utente. La finestra interattiva appare sopra il contenuto principale dell'interfaccia (in overlay);
  \item Login Form: modulo per l'inserimento delle credenziali di accesso;
  \item Username Input: campo di testo per l'inserimento dello username;
  \item Password Input: campo di testo per l'inserimento della password;
  \item V-Model: crea un binding bidirezionale tra i campi di input e le rispettive variabili;
  \item Login Button: pulsante per richiedere l'accesso all'area di amministrazione;
  \item Error Message: messaggio di errore per notificare all'utente che:
  \begin{itemize}
    \item L'utente non è autorizzato ad accedere all'area di amministrazione;
    \item Le credenziali inserite non sono corrette;
    \item Il server non è raggiungibile.
  \end{itemize}
\end{itemize}

\subsubsection{ChatMessage}

\paragraph*{Props}
\begin{itemize}
  \item Message: testo del messaggio;
  \item IsSent: flag per indicare se il messaggio è stato inviato o ricevuto dall'utente.
\end{itemize}

\paragraph*{Elementi chiave}
\begin{itemize}
  \item Message Card: contenitore per il messaggio. La disposizione e il colore del contenitore variano in base al mittente del messaggio;
  \item Avatar: icona del mittente. L'avatar è posizionato a sinistra o a destra del messaggio in base al mittente;
  \item Message Text: testo del messaggio;
  \item Copy Button: pulsante per copiare il testo del messaggio negli Appunti di sistema.
\end{itemize}

\subsubsection{DictPreview}

\paragraph*{Props}
\begin{itemize}
  \item DetailsVisible: flag per visualizzare o nascondere l'anteprima del \glossario{dizionario dati};
  \item DictionaryPreview: anteprima del dizionario dati, contenente le seguenti informazioni:
  \begin{itemize}
    \item Database Name: nome del dizionario dati;
    \item Database Description: descrizione del dizionario dati;
    \item Lista delle tabelle: 
    \begin{itemize}
      \item Table Name: nome della tabella;
      \item Table Description: descrizione della tabella.
    \end{itemize}
  \end{itemize}
\end{itemize}

\paragraph*{Elementi chiave}
\begin{itemize}
  \item Preview Card: contenitore per l'anteprima del dizionario dati;
  \item Expand Button: pulsante per aumentare la dimensione della Preview Card. Il pulsante di espansione migliora l'accessibilità e l'usabilità;
  \item Collapse Button: pulsante per ridurre la dimensione della Preview Card.
\end{itemize}

\subsubsection{ChatDeleteBtn}

\paragraph*{Props}
\begin{itemize}
  \item Messages: contenuto della chat da eliminare;
  \item Loading: flag per indicare se il ChatBOT sta elaborando un nuovo messaggio.
\end{itemize}

\paragraph*{Emits}
\begin{itemize}
  \item ClearMessages: evento per notificare il componente genitore che i messaggi devono essere cancellati.
\end{itemize}

\paragraph*{Elementi chiave}
\begin{itemize}
  \item Toggle: il pulsante di cancellazione deve essere nascosto se:
  \begin{itemize}
    \item Non ci sono messaggi da cancellare;
    \item Il ChatBOT sta elaborando un nuovo messaggio.
  \end{itemize}
\end{itemize}

\subsubsection{AppMenu}

\paragraph*{Elementi chiave}
\begin{itemize}
  \item Layout Menu: contenitore per il menu di navigazione principale;
  \item Menu Items: voci di menu per accedere alle diverse sezioni dell'applicazione;
  \item Menu Icon: stringa contenente la classe da applicare all'icona della voce di menu;
  \item Menu Label: stringa contenente l'etichetta della voce di menu;
  \item Menu Link: stringa contenente il percorso della voce di menu.
\end{itemize}

\subsubsection{AppMenuItem}

\paragraph*{Props}
\begin{itemize}
  \item Item: oggetto che rappresenta una voce di menu;
  \item Index: indice della voce di menu;
  \item Root: flag per indicare se la voce di menu è la radice del menu di navigazione;
  \item Parent Item Key: chiave per identificare la voce di menu genitore.
\end{itemize}

\paragraph*{Elementi chiave}
\begin{itemize}
  \item Menu Item: contenitore per una voce di menu;
  \item Menu Icon: icona per identificare la voce di menu;
  \item Menu Label: etichetta per identificare la voce di menu;
  \item Menu Link: collegamento esterno (non gestito da Vue Router) o interno (gestito da Vue Router). Un link può essere anche un elemento cliccabile per aprire un sotto-menu;
  \item Sub Menu: contenitore per le voci di menu figlie.
\end{itemize}

\subsubsection{AppLogo}

\paragraph*{Props}
\begin{itemize}
  \item Width: variabile per impostare la larghezza del logo;
  \item Height: variabile per impostare l'altezza del logo;
  \item Path: variabile per impostare il percorso dell'immagine.
\end{itemize}

\paragraph*{Elementi chiave}
\begin{itemize}
  \item Filter: funzionalità per invertire il colore del logo da bianco a nero (in base al tema attivo).
\end{itemize}

\subsubsection{AppFooter}

\paragraph*{Elementi chiave}
\begin{itemize}
  \item Logo: logo dell'applicazione;
  \item Version: numero di versione dell'applicazione;
  \item Copyright section: sezione contenente le seguenti informazioni:
  \begin{itemize}
    \item Year: anno corrente;
    \item Group Name: nome del team di sviluppo;
    \item Copyright: prova di diritto d'autore.
  \end{itemize}
\end{itemize}