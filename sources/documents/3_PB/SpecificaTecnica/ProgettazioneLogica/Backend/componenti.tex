\subsection{Back-end}

%SEZIONE ROUTES DI dictionaries_api

\subsubsection{get\_all\_dictionaries}

\subsubsubsection{Descrizione}
\par La funzione \texttt{get\_all\_dictionaries} recupera tutti i dizionari presenti nel sistema e restituisce una risposta strutturata.

\subsubsubsection{Dettagli dell'endpoint}
\begin{itemize}
  \item \textbf{HTTP Method}: \texttt{GET};
  \item \textbf{Endpoint}: \texttt{/};
  \item \textbf{Tags}: \texttt{[tag]};
  \item \textbf{Response Model}: \texttt{DictionariesResponseDto};
  \item \textbf{Nome}: \texttt{getAllDictionaries};
  \item \textbf{Dependency Injection}:
  \begin{itemize}
    \item \textbf{dictionary\_service (DictionaryService)}: dipendenza risolta tramite Depends (get\_dictionary\_service).
  \end{itemize}
\end{itemize}

\subsubsubsection{Implementazione}
\begin{itemize}
  \item Effettua una chiamata al servizio di gestione dei dizionari per recuperare l'elenco completo dei dizionari;
  \item Restituisce un oggetto \texttt{DictionariesResponseDto} con lo stato della risposta e, in caso di successo, i dati richiesti.
\end{itemize}

\subsubsection{get\_dictionary}

\subsubsubsection{Descrizione}
\par La funzione \texttt{get\_dictionary} recupera un dizionario specifico utilizzando l'ID fornito e restituisce una risposta strutturata.

\subsubsubsection{Dettagli dell'endpoint}
\begin{itemize}
  \item \textbf{HTTP Method}: \texttt{GET};
  \item \textbf{Endpoint}: \texttt{/\{id\}};
  \item \textbf{Tags}: \texttt{[tag]};
  \item \textbf{Response Model}: \texttt{DictionaryResponseDto};
  \item \textbf{Nome}: \texttt{getDictionary};
  \item \textbf{Dependency Injection}:
  \begin{itemize}
    \item \textbf{dictionary\_service (DictionaryService)}: dipendenza risolta tramite Depends (get\_dictionary\_service).
  \end{itemize}
  \item \textbf{Parametri}:
  \begin{itemize}
    \item \textbf{id (int)}: ID del dizionario da recuperare.
  \end{itemize}
\end{itemize}

\subsubsubsection{Implementazione}
\begin{itemize}
  \item Effettua una chiamata al servizio di gestione dei dizionari per recuperare i dettagli del dizionario specificato;
  \item Restituisce un oggetto \texttt{DictionaryResponseDto} con lo stato della risposta e, in caso di successo, i dati richiesti.
\end{itemize}

\subsubsection{get\_dictionary\_file}

\subsubsubsection{Descrizione}
\par La funzione \texttt{get\_dictionary\_file} recupera il file associato a un dizionario specifico utilizzando l'ID fornito.

\subsubsubsection{Dettagli dell'endpoint}
\begin{itemize}
  \item \textbf{HTTP Method}: \texttt{GET};
  \item \textbf{Endpoint}: \texttt{/\{id\}/file};
  \item \textbf{Tags}: \texttt{[tag]};
  \item \textbf{Nome}: \texttt{getDictionaryFile};
  \item \textbf{Dependency Injection}:
  \begin{itemize}
    \item \textbf{dictionary\_service (DictionaryService)}: dipendenza risolta tramite Depends (get\_dictionary\_service).
  \end{itemize}
    \item \textbf{Parametri}:
  \begin{itemize}
    \item \textbf{id (int)}: ID del dizionario di cui recuperare il file.
  \end{itemize}
\end{itemize}

\subsubsubsection{Implementazione}
\begin{itemize}
  \item Effettua una chiamata al servizio di gestione dei dizionari per recuperare il file associato al dizionario specificato;
  \item Se il dizionario viene trovato, restituisce il file come parte di una risposta HTTP (\texttt{FileResponse});
  \item Se il dizionario non viene trovato, restituisce un oggetto \texttt{ResponseDto} con lo stato della risposta.
\end{itemize}

\subsubsection{get\_dictionary\_preview}

\subsubsubsection{Descrizione}
\par La funzione \texttt{get\_dictionary\_preview} recupera l'anteprima di un dizionario utilizzando l'ID fornito e restituisce una risposta strutturata.

\subsubsubsection{Dettagli dell'endpoint}
\begin{itemize}
  \item \textbf{HTTP Method}: \texttt{GET};
  \item \textbf{Endpoint}: \texttt{/\{id\}/dictionary-preview};
  \item \textbf{Tags}: \texttt{[tag]};
  \item \textbf{Response Model}: \texttt{DictionaryResponseDto};
  \item \textbf{Nome}: \texttt{getDictionaryPreview};
  \item \textbf{Dependency Injection}:
  \begin{itemize}
    \item \textbf{dictionary\_service (DictionaryService)}: dipendenza risolta tramite Depends (get\_dictionary\_service).
  \end{itemize}
  \item \textbf{Parametri}:
  \begin{itemize}
    \item \textbf{id (int)}: ID del dizionario di cui recuperare l'anteprima.
  \end{itemize}
\end{itemize}

\subsubsubsection{Implementazione}
\begin{itemize}
  \item Effettua una chiamata al servizio di gestione dei dizionari per recuperare l'anteprima del dizionario specificato;
  \item Restituisce un oggetto \texttt{DictionaryResponseDto} con lo stato della risposta e, in caso di successo, i dati richiesti.
\end{itemize}

\subsubsection{create\_dictionary}

\subsubsubsection{Descrizione}
\par La funzione \texttt{create\_dictionary} inserisce un nuovo dizionario nel sistema utilizzando i dati forniti e un file di configurazione.

\subsubsubsection{Dettagli dell'endpoint}
\begin{itemize}
  \item \textbf{HTTP Method}: \texttt{POST};
  \item \textbf{Endpoint}: \texttt{/};
  \item \textbf{Tags}: \texttt{[tag]};
  \item \textbf{Response Model}: \texttt{DictionaryResponseDto};
  \item \textbf{Dependencies}:
  \begin{itemize}
    \item \texttt{Depends(JwtBearer())}: autenticazione JWT necessaria per l'accesso all'endpoint.
  \end{itemize}
  \item \textbf{Nome}: \texttt{createDictionary};
  \item \textbf{Dependency Injection}:
  \begin{itemize}
    \item \textbf{dictionary\_service (DictionaryService)}: dipendenza risolta tramite Depends (get\_dictionary\_service).
  \end{itemize}
  \item \textbf{Parametri}:
  \begin{itemize}
    \item \textbf{file (Annotated[UploadFile, File()])}: il file associato al dizionario;
    \item \textbf{dictionary (DictionaryDto)}: i metadati del dizionario.
  \end{itemize}
\end{itemize}

\subsubsubsection{Implementazione}
\begin{itemize}
  \item Verifica la presenza del file e legge il contenuto, se disponibile;
  \item Effettua una chiamata al servizio di gestione dei dizionari per salvare un nuovo dizionario nel sistema;
  \item Restituisce un oggetto \texttt{DictionaryResponseDto} con lo stato della risposta e, in caso di successo, i dati del dizionario;
\end{itemize}

\subsubsection{update\_dictionary\_file}

\subsubsubsection{Descrizione}
\par La funzione \texttt{update\_dictionary\_file} aggiorna il file associato a un dizionario esistente utilizzando l'ID fornito e un nuovo file di configurazione.

\subsubsubsection{Dettagli dell'endpoint}
\begin{itemize}
  \item \textbf{HTTP Method}: \texttt{PUT};
  \item \textbf{Endpoint}: \texttt{/\{id\}/file};
  \item \textbf{Tags}: \texttt{[tag]};
  \item \textbf{Response Model}: \texttt{DictionaryResponseDto};
  \item \textbf{Dependencies}:
  \begin{itemize}
    \item \texttt{Depends(JwtBearer())}: autenticazione JWT necessaria per l'accesso all'endpoint.
  \end{itemize}
  \item \textbf{Nome}: \texttt{updateDictionaryFile};
  \item \textbf{Dependency Injection}:
  \begin{itemize}
    \item \textbf{dictionary\_service (DictionaryService)}: dipendenza risolta tramite Depends (get\_dictionary\_service).
  \end{itemize}
  \item \textbf{Parametri}:
  \begin{itemize}
    \item \textbf{id (int)}: ID del dizionario da aggiornare;
    \item \textbf{file (Annotated[UploadFile, File()])}: il file associato al dizionario.
  \end{itemize}
\end{itemize}

\subsubsubsection{Implementazione}
\begin{itemize}
  \item Verifica la presenza del file e legge il contenuto, se disponibile;
  \item Effettua una chiamata al servizio di gestione dei dizionari per modificare il file associato al dizionario specificato;
  \item Restituisce un oggetto \texttt{DictionaryResponseDto} con lo stato della risposta e, in caso di successo, i dati del dizionario.
\end{itemize}

\subsubsection{update\_dictionary\_metadata}

\subsubsubsection{Descrizione}
\par La funzione \texttt{update\_dictionary\_metadata} aggiorna i metadati di un dizionario esistente utilizzando l'ID fornito e un oggetto \texttt{DictionaryDto}.

\subsubsubsection{Dettagli dell'endpoint}
\begin{itemize}
  \item \textbf{HTTP Method}: \texttt{PUT};
  \item \textbf{Endpoint}: \texttt{/\{id\}};
  \item \textbf{Tags}: \texttt{[tag]};
  \item \textbf{Response Model}: \texttt{DictionaryResponseDto};
  \item \textbf{Dependencies}:
  \begin{itemize}
    \item \texttt{Depends(JwtBearer())}: autenticazione JWT necessaria per l'accesso all'endpoint.
  \end{itemize}
  \item \textbf{Nome}: \texttt{updateDictionaryMetadata};
  \item \textbf{Dependency Injection}:
  \begin{itemize}
    \item \textbf{dictionary\_service (DictionaryService)}: dipendenza risolta tramite Depends (get\_dictionary\_service).
  \end{itemize}
  \item \textbf{Parametri}:
  \begin{itemize}
    \item \textbf{id (int)}: ID del dizionario da aggiornare;
    \item \textbf{dictionary (DictionaryDto)}: i nuovi metadati del dizionario.
  \end{itemize}
\end{itemize}

\subsubsubsection{Implementazione}
\begin{itemize}
  \item Effettua una chiamata al servizio di gestione dei dizionari per modificare i metadati del dizionario specificato;
  \item Restituisce un oggetto \texttt{DictionaryResponseDto} con lo stato della risposta e, in caso di successo, i dati del dizionario.
\end{itemize}

\subsubsection{delete\_dictionary}

\subsubsubsection{Descrizione}
\par La funzione \texttt{delete\_dictionary} elimina un dizionario dal sistema utilizzando l'ID fornito.

\subsubsubsection{Dettagli dell'endpoint}
  \begin{itemize}
  \item \textbf{HTTP Method}: \texttt{DELETE};
  \item \textbf{Endpoint}: \texttt{/\{id\}};
  \item \textbf{Tags}: \texttt{[tag]};
  \item \textbf{Response Model}: \texttt{ResponseDto};
  \item \textbf{Dependencies}:
  \begin{itemize}
    \item \texttt{Depends(JwtBearer())}: autenticazione JWT necessaria per l'accesso all'endpoint.
  \end{itemize}
  \item \textbf{Nome}: \texttt{deleteDictionary};
  \item \textbf{Dependency Injection}:
  \begin{itemize}
    \item \textbf{dictionary\_service (DictionaryService)}: dipendenza risolta tramite Depends (get\_dictionary\_service).
  \end{itemize}
  \item \textbf{Parametri}:
  \begin{itemize}
    \item \textbf{id (int)}: ID del dizionario da eliminare.
  \end{itemize}
\end{itemize}

\subsubsubsection{Implementazione}
\begin{itemize}
  \item Effettua una chiamata al servizio di gestione dei dizionari per eliminare il dizionario specificato;
  \item Restituisce un oggetto \texttt{ResponseDto} con lo stato della risposta.
\end{itemize}

%SEZIONE ROUTES DI login_api

\subsubsection{login}

\subsubsubsection{Descrizione}
\par La funzione \texttt{login} autentica un utente utilizzando le credenziali fornite (username e password). Se le credenziali sono corrette, restituisce una risposta strutturata contenente un token di autenticazione JWT.

\subsubsubsection{Dettagli dell'endpoint}
\begin{itemize}
  \item \textbf{HTTP Method}: \texttt{POST};
  \item \textbf{Endpoint}: \texttt{/};
  \item \textbf{Tags}: \texttt{[tag]};
  \item \textbf{Response Model}: \texttt{AuthResponseDto};
  \item \textbf{Nome}: \texttt{login};
  \item \textbf{Dependency Injection}:
  \begin{itemize}
  \item \textbf{authentication\_service (AuthenticationService)}: dipendenza risolta tramite Depends (get\_authentication\_service).
  \end{itemize}
  \item \textbf{Parametri}:
  \begin{itemize}
    \item \textbf{data (LoginDto)}: i dati di autenticazione dell'utente.
  \end{itemize}
\end{itemize}

\subsubsubsection{Implementazione}
\begin{itemize}
  \item Effettua una chiamata al servizio di autenticazione per autorizzare l'utente attraverso le credenziali fornite;
  \item Restituisce un oggetto \texttt{AuthResponseDto} con lo stato della risposta e, in caso di successo, il token di autenticazione JWT.
\end{itemize}


%SEZIONE ROUTES DI prompt_api

\subsubsection{generate\_prompt}

\subsubsubsection{Descrizione}
\par La funzione \texttt{generate\_prompt} genera un \glossario{prompt} utilizzando i parametri forniti dall'utente, tra cui un dizionario, una richiesta, una lingua e un \glossario{DBMS}.

\subsubsubsection{Dettagli dell'endpoint}
\begin{itemize}
  \item \textbf{HTTP Method}: \texttt{GET};
  \item \textbf{Endpoint}: \texttt{/};
  \item \textbf{Tags}: \texttt{[tag]};
  \item \textbf{Response Model}: \texttt{StringDataResponseDto};
  \item \textbf{Nome}: \texttt{generatePrompt};
  \item \textbf{Dependency Injection}:
  \begin{itemize}
    \item \textbf{prompt\_manager\_service (PromptManagerService)}: dipendenza risolta tramite Depends (get\_prompt\_manager\_service).
  \end{itemize}
  \item \textbf{Parametri}:
  \begin{itemize}
    \item \textbf{dictionary\_id (Annotated[int, Query(alias="dictionaryId")])}: l'ID del dizionario;
    \item \textbf{query (str)}: la richiesta dell'utente in linguaggio naturale;
    \item \textbf{dbms (str)}: il sistema di gestione del database (DBMS);
    \item \textbf{lang (str)}: la lingua di richiesta.
  \end{itemize}
\end{itemize}

\subsubsubsection{Implementazione}
\begin{itemize}
  \item Effettua una chiamata al servizio di gestione dei prompt per generare un prompt in base ai parametri forniti;
  \item Restituisce un oggetto \texttt{StringDataResponseDto} con lo stato della risposta e, in caso di successo, il prompt generato.
\end{itemize}

\subsubsection{generate\_prompt\_with\_debug}

\subsubsubsection{Descrizione}
\par La funzione \texttt{generate\_prompt\_with\_debug} genera un \glossario{prompt} e include informazioni dettagliate di \glossario{debug}, fornendo dati aggiuntivi per la risoluzione di problemi.

\subsubsubsection{Dettagli dell'endpoint}
\begin{itemize}
  \item \textbf{HTTP Method}: \texttt{GET};
  \item \textbf{Endpoint}: \texttt{/debug};
  \item \textbf{Tags}: \texttt{[tag]};
  \item \textbf{Response Model}: \texttt{PromptResponseDto};
  \item \textbf{Dependencies}:
    \begin{itemize}
      \item \texttt{Depends(JwtBearer())}: autenticazione JWT necessaria per l'accesso all'endpoint.
    \end{itemize}
  \item \textbf{Nome}: \texttt{generatePromptWithDebug};
  \item \textbf{Dependency Injection}:
  \begin{itemize}
    \item \textbf{prompt\_manager\_service (PromptManagerService)}: dipendenza risolta tramite Depends (get\_prompt\_manager\_service);
  \end{itemize}
  \item \textbf{Parametri}:
  \begin{itemize}
    \item \textbf{dictionary\_id (Annotated[int, Query(alias="dictionaryId")])}: l'ID del dizionario;
    \item \textbf{query (str)}: la richiesta dell'utente in linguaggio naturale;
    \item \textbf{dbms (str)}: il sistema di gestione del database (DBMS);
    \item \textbf{lang (str)}: la lingua di richiesta.
  \end{itemize}
\end{itemize}

\subsubsubsection{Implementazione}
\begin{itemize}
  \item Effettua una chiamata al servizio di gestione dei prompt per generare un prompt con informazioni di debug relative ad esso;
  \item Restituisce un oggetto \texttt{PromptResponseDto} con lo stato della risposta e, in caso di successo, i dati richiesti.
\end{itemize}


% TABELLA TRACCIAMENTO REQUISITI

\subsubsection{Tracciamento dei requisiti}
\bgroup
\begin{adjustwidth}{-0.5cm}{-0.5cm}
	\centering
	% MAX 12.5cm
  \begin{longtable}{|c|c|}
		\caption{Tracciamento dei requisiti back-end}
  	\label{tab:tracciamento-requisiti-back-end} \\
    \hline
		\textbf{ID} & \textbf{Route} \\
		\hline
		\endfirsthead

		\caption[]{Tracciamento dei requisiti back-end (continua)} \\
		\hline
		\textbf{ID} & \textbf{Route} \\
		\hline
		\endhead

		\hline
		\multicolumn{2}{|r|}{{Continua nella prossima pagina}} \\
		\hline
		\endfoot

		\hline
		\endlastfoot

    RF.O.1 & \texttt{login} \\
	  \hline RF.O.2 & \texttt{login} \\
    \hline RF.O.3 & \texttt{generate\_prompt}\\
    \hline RF.O.9 & \texttt{get\_all\_dictionaries} \\
    \hline RF.O.9.1 & \texttt{get\_all\_dictionaries} \\
    \hline RF.O.10 & \texttt{get\_all\_dictionaries}, \texttt{get\_dictionary} \\
    \hline RF.O.10.1 & \texttt{get\_all\_dictionaries}, \texttt{get\_dictionary} \\
    \hline RF.O.10.2 & \texttt{get\_all\_dictionaries}, \texttt{get\_dictionary} \\
    \hline RF.O.10.3 & \texttt{get\_all\_dictionaries}, \texttt{get\_dictionary} \\
    \hline RF.O.11 & \texttt{generate\_prompt}, \texttt{generate\_prompt\_with\_debug} \\
    \hline RF.O.13 & \texttt{create\_dictionary} \\
    \hline RF.O.14 & \texttt{get\_dictionary\_preview} \\
    \hline RF.O.14.1 & \texttt{get\_dictionary\_preview} \\
    \hline RF.O.14.2 & \texttt{get\_dictionary\_preview} \\
    \hline RF.O.14.3 & \texttt{get\_dictionary\_preview} \\
    \hline RF.O.14.3.1 & \texttt{get\_dictionary\_preview} \\
    \hline RF.O.14.3.1.1 & \texttt{get\_dictionary\_preview} \\
    \hline RF.O.14.3.1.2 & \texttt{get\_dictionary\_preview} \\
    \hline RF.O.15 & \texttt{create\_dictionary} \\
    \hline RF.O.16 & \texttt{create\_dictionary} \\
    \hline RF.O.17 & \texttt{create\_dictionary} \\
    \hline RF.O.18 & \texttt{delete\_dictionary} \\
    \hline RF.O.19 & \texttt{create\_dictionary}, \texttt{update\_dictionary\_file} \\
    \hline RF.O.20 & \texttt{update\_dictionary\_file} \\
    \hline RF.O.21 & \texttt{delete\_dictionary} \\
    \hline RF.D.22 & \texttt{generate\_prompt\_with\_debug} \\
    \hline RF.O.28 & \texttt{create\_dictionary}, \texttt{update\_dictionary\_file} \\
    \hline RF.O.29 & \texttt{update\_dictionary\_metadata} \\
    \hline RF.O.30 & \texttt{update\_dictionary\_metadata} \\
    \hline RF.O.31 & \texttt{create\_dictionary}, \texttt{update\_dictionary\_metadata} \\
    \hline RF.O.31.1 & \texttt{create\_dictionary}, \texttt{update\_dictionary\_metadata} \\
    \hline RF.O.31.2 & \texttt{create\_dictionary}, \texttt{update\_dictionary\_metadata} \\
    \hline RF.O.32 & \texttt{create\_dictionary}, \texttt{update\_dictionary\_metadata} \\
    \hline RF.O.33 & \texttt{create\_dictionary}, \texttt{update\_dictionary\_file} \\
    \hline RF.O.34 & \texttt{create\_dictionary}, \texttt{update\_dictionary\_file} \\
    \hline RF.O.38 & \texttt{get\_dictionary\_file} \\
    \hline RF.O.39 & \texttt{create\_dictionary}, \texttt{update\_dictionary\_file} \\
    \hline RF.D.40 & \texttt{create\_dictionary}, \texttt{update\_dictionary\_file} \\
    \hline RF.OP.42 & \texttt{generate\_prompt}, \texttt{generate\_prompt\_with\_debug} \\
    \hline RF.O.45 & \texttt{generate\_prompt}, \texttt{generate\_prompt\_with\_debug} \\
    \hline RF.D.46 & \texttt{generate\_prompt\_with\_debug} \\
    \hline RF.O.47 & \texttt{generate\_prompt}, \texttt{generate\_prompt\_with\_debug} \\
    \hline RF.OP.52 & \texttt{generate\_prompt}, \texttt{generate\_prompt\_with\_debug} \\
    \hline RF.O.55 & \texttt{create\_dictionary}, \texttt{update\_dictionary\_file} \\
    \hline RF.O.56 & \texttt{create\_dictionary}, \texttt{update\_dictionary\_file} \\
    \hline RF.O.57 & \texttt{create\_dictionary}, \texttt{update\_dictionary\_file} \\
  \end{longtable}
\end{adjustwidth}
\egroup
