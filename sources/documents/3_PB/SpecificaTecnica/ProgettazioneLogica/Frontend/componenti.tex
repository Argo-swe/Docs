\subsection{Front-end}

\subsubsection{AppLayout}

\subsubsubsection{Descrizione}
\par AppLayout è il componente che definisce la struttura generale dell'applicazione.

\subsubsubsection{Sottocomponenti}
\par AppLayout è composto dai seguenti sottocomponenti:
\begin{itemize}
  \item \textbf{AppTopbar}: componente condiviso da tutte le pagine dell'interfaccia. La topbar permette di accedere ai menu dell'applicazione. È situata nella zona superiore dello schermo e include il seguente componente (puramente visivo):
  \begin{itemize}
    \item \textbf{AppLogo}: logo dell'applicazione.
  \end{itemize}
  \item \textbf{LoginDialog}: finestra per l'autenticazione dell'utente;
  \item \textbf{MenuSidebar}: barra laterale che contiene il menu di navigazione principale;
  \item \textbf{ConfigSidebar}: barra laterale che contiene il menu per la configurazione del sistema e l'apertura del dialog di login.
\end{itemize}

\subsubsubsection{Tracciamento dei requisiti}
\bgroup
\begin{adjustwidth}{-0.5cm}{-0.5cm}
	\centering
	% MAX 12.5cm
  \begin{longtable}{|c|c|}
		\caption{Tracciamento dei requisiti per il componente AppLayout}
  	\label{tab:tracciamento-requisiti-layout} \\
    \hline
		\textbf{ID} & \textbf{Componente} \\
		\hline
		\endfirsthead

		\caption[]{Tracciamento dei requisiti per il componente AppLayout (continua)} \\
		\hline
		\textbf{ID} & \textbf{Componente} \\
		\hline
		\endhead

		\hline
		\multicolumn{2}{|r|}{{Continua nella prossima pagina}} \\
		\hline
		\endfoot

		\hline
		\endlastfoot

    RF.O.1 & LoginDialog \\
		\hline RF.O.1.1 & LoginDialog \\
    \hline RF.O.1.2 & LoginDialog \\
    \hline RF.O.2 & LoginDialog \\
    \hline RF.O.60 & ConfigSidebar \\
    \hline RF.D.61 & ConfigSidebar \\
    \hline RF.OP.62 & ConfigSidebar \\
  \end{longtable}
\end{adjustwidth}
\egroup

\subsubsection{ChatView}

\subsubsubsection{Descrizione}
\par ChatView è la pagina di generazione del \glossario{prompt}. Espone le seguenti funzionalità:
\begin{itemize}
  \item Selezione di un \glossario{dizionario dati};
  \item Visualizzazione di un'anteprima del dizionario dati;
  \item Selezione della lingua di richiesta;
  \item Selezione di un \glossario{DBMS};
  \item Invio di una richiesta al ChatBOT;
  \item Visualizzazione della risposta del ChatBOT;
  \item Copia del prompt generato;
  \item Visualizzazione del messaggio di \glossario{debug} associato a ciascun prompt;
  \item Download di un file di \glossario{log};
  \item Eliminazione del contenuto della chat.
\end{itemize}

\subsubsubsection{Sottocomponenti}
\par ChatView è composto dai seguenti sottocomponenti:
\begin{itemize}
  \item \textbf{ChatMessage}: rappresenta i messaggi inviati e/o restituiti all'interno della chat;
  \item \textbf{ChatDeleteBtn}: pulsante per eliminare il contenuto della chat;
  \item \textbf{DictPreview}: mostra un'anteprima del dizionario dati selezionato.
\end{itemize}

\subsubsubsection{Tracciamento dei requisiti}
\bgroup
\begin{adjustwidth}{-0.5cm}{-0.5cm}
	\centering
	% MAX 12.5cm
  \begin{longtable}{|c|c|}
		\caption{Tracciamento dei requisiti per il componente ChatView}
  	\label{tab:tracciamento-requisiti-chat} \\
    \hline
		\textbf{ID} & \textbf{Componente} \\
		\hline
		\endfirsthead

		\caption[]{Tracciamento dei requisiti per il componente ChatView (continua)} \\
		\hline
		\textbf{ID} & \textbf{Componente} \\
		\hline
		\endhead

		\hline
		\multicolumn{2}{|r|}{{Continua nella prossima pagina}} \\
		\hline
		\endfoot

		\hline
		\endlastfoot

    RF.O.3 & ChatView \\
		\hline RF.O.4 & ChatView \\
    \hline RF.O.5 & ChatView \\
    \hline RF.O.6 & ChatMessage \\
    \hline RF.D.7 & ChatView \\
    \hline RF.O.8 & ChatMessage \\
    \hline RF.O.9 & ChatView \\
    \hline RF.O.9.1 & ChatView \\
    \hline RF.O.10 & ChatView \\
    \hline RF.O.10.1 & ChatView \\
    \hline RF.O.11 & ChatView \\
    \hline RF.O.14 & DictPreview \\
    \hline RF.O.14.1 & DictPreview \\
    \hline RF.O.14.2 & DictPreview \\
    \hline RF.O.14.3 & DictPreview \\
    \hline RF.O.14.3.1 & DictPreview \\
    \hline RF.O.14.3.1.1 & DictPreview \\
    \hline RF.O.14.3.1.2 & DictPreview \\
    \hline RF.O.25 & ChatView \\
    \hline RF.O.25.1 & ChatMessage \\
    \hline RF.O.25.1.1 & ChatMessage \\
    \hline RF.O.25.1.2 & ChatMessage \\
    \hline RF.O.26 & ChatView \\
    \hline RF.O.27 & ChatDeleteBtn \\
    \hline RF.O.35 & ChatMessage \\
    \hline RF.O.36 & ChatView \\
    \hline RF.O.41 & ChatMessage \\
    \hline RF.O.45 & ChatMessage \\
    \hline RF.O.47 & ChatView \\
    \hline RF.O.48 & ChatView \\
    \hline RF.O.49 & ChatView \\
    \hline RF.O.50 & ChatView \\
    \hline RF.O.51 & ChatView \\
    \hline RF.D.52 & ChatView \\
  \end{longtable}
\end{adjustwidth}
\egroup

\subsubsection{DictionariesListView}

\subsubsubsection{Descrizione}
\par DictionariesListView è la pagina di gestione \glossario{CRUD} dei dizionari dati. Espone le seguenti funzionalità:
\begin{itemize}
  \item Visualizzazione della lista dei \glossario{dizionari dati};
  \item Creazione, modifica e cancellazione di un dizionario dati;
  \item Download di un dizionario dati;
  \item Persistenza degli indici associati ai dizionari dati.
\end{itemize}

\subsubsubsection{Sottocomponenti}
\par DictionariesListView è composto dai seguenti sottocomponenti:
\begin{itemize}
  \item \textbf{CreateUpdateDictionaryModal}: finestra per la creazione e modifica di un dizionario dati.
\end{itemize}

\subsubsubsection{Tracciamento dei requisiti}
\bgroup
\begin{adjustwidth}{-0.5cm}{-0.5cm}
	\centering
	% MAX 12.5cm
  \begin{longtable}{|c|c|}
		\caption{Tracciamento dei requisiti per il componente DictionariesListView}
  	\label{tab:tracciamento-requisiti-dict} \\
    \hline
		\textbf{ID} & \textbf{Componente} \\
		\hline
		\endfirsthead

		\caption[]{Tracciamento dei requisiti per il componente DictionariesListView (continua)} \\
		\hline
		\textbf{ID} & \textbf{Componente} \\
		\hline
		\endhead

		\hline
		\multicolumn{2}{|r|}{{Continua nella prossima pagina}} \\
		\hline
		\endfoot

		\hline
		\endlastfoot

    RF.O.9 & DictionariesListView \\
		\hline RF.O.9.1 & DictionariesListView \\
    \hline RF.O.10 & DictionariesListView \\
    \hline RF.O.10.1 & DictionariesListView \\
    \hline RF.O.10.3 & DictionariesListView \\
    \hline RF.O.13 & CreateUpdateDictionaryModal \\
    \hline RF.O.15 & CreateUpdateDictionaryModal \\
    \hline RF.O.16 & CreateUpdateDictionaryModal \\
    \hline RF.O.17 & CreateUpdateDictionaryModal \\
    \hline RF.O.18 & DictionariesListView \\
    \hline RF.O.19 & CreateUpdateDictionaryModal \\
    \hline RF.O.20 & CreateUpdateDictionaryModal \\
    \hline RF.O.21 & DictionariesListView \\
    \hline RF.O.24 & DictionariesListView \\
    \hline RF.O.28 & CreateUpdateDictionaryModal \\
    \hline RF.O.29 & CreateUpdateDictionaryModal \\
    \hline RF.O.30 & CreateUpdateDictionaryModal \\
    \hline RF.O.31 & CreateUpdateDictionaryModal \\
    \hline RF.O.31.1 & CreateUpdateDictionaryModal \\
    \hline RF.O.31.2 & CreateUpdateDictionaryModal \\
    \hline RF.O.32 & CreateUpdateDictionaryModal \\
    \hline RF.O.33 & CreateUpdateDictionaryModal \\
    \hline RF.O.34 & CreateUpdateDictionaryModal \\
    \hline RF.O.36 & DictionariesListView \\
    \hline RF.O.37 & DictionariesListView \\
    \hline RF.O.39 & CreateUpdateDictionaryModal \\
    \hline RF.D.40 & CreateUpdateDictionaryModal \\
    \hline RF.O.55 & CreateUpdateDictionaryModal \\
    \hline RF.O.56 & CreateUpdateDictionaryModal \\
    \hline RF.O.57 & CreateUpdateDictionaryModal \\
  \end{longtable}
\end{adjustwidth}
\egroup

\subsubsection{MenuSidebar}

\subsubsubsection{Descrizione}
\par MenuSidebar è la barra laterale che consente di navigare tra le seguenti pagine:
\begin{itemize}
  \item ChatView;
  \item DictionariesListView.
\end{itemize}

\subsubsubsection{Sottocomponenti}
\par MenuSidebar è composto dai seguenti sottocomponenti:
\begin{itemize}
  \item \textbf{AppMenu}: rappresenta il menu di navigazione principale;
  \item \textbf{AppFooter}: rappresenta il footer dell'applicazione. Include il seguente componente (puramente visivo):
  \begin{itemize}
    \item \textbf{AppLogo}: logo dell'applicazione.
  \end{itemize}
\end{itemize}

\subsubsection{AppMenu}

\subsubsubsection{Descrizione}
\par AppMenu è il menu di navigazione principale dell'applicazione.

\subsubsubsection{Sottocomponenti}
\par AppMenu è composto dai seguenti sottocomponenti:
\begin{itemize}
  \item \textbf{AppMenuItem}: rappresenta un singolo elemento del menu di navigazione.
\end{itemize}

\subsubsection{ChatMessage}

\subsubsubsection{Descrizione}
\par ChatMessage è il componente che rappresenta i messaggi inviati e/o restituiti all'interno della chat.

\subsubsubsection{Sottocomponenti}
\par AppMenu è composto dai seguenti sottocomponenti:
\begin{itemize}
  \item \textbf{StringDataModal}: componente per visualizzare stringhe di testo all'interno di una finestra modale.
\end{itemize}

\subsubsection{StringDataModal}

\subsubsubsection{Descrizione}
\par StringDataModal consente di visualizzare stringhe di testo all'interno di una finestra modale.

\subsubsubsection{Sottocomponenti}
\par StringDataModal è composto dai seguenti sottocomponenti:
\begin{itemize}
  \item \textbf{DebugMessage}: rappresenta un messaggio di \glossario{debug}.
\end{itemize}

\subsubsubsection{Tracciamento dei requisiti}
\bgroup
\begin{adjustwidth}{-0.5cm}{-0.5cm}
	\centering
	% MAX 12.5cm
  \begin{longtable}{|c|c|}
		\caption{Tracciamento dei requisiti per il componente DebugMessage}
  	\label{tab:tracciamento-requisiti-debug} \\
    \hline
		\textbf{ID} & \textbf{Componente} \\
		\hline
		\endfirsthead

		\caption[]{Tracciamento dei requisiti per il componente DebugMessage (continua)} \\
		\hline
		\textbf{ID} & \textbf{Componente} \\
		\hline
		\endhead

		\hline
		\multicolumn{2}{|r|}{{Continua nella prossima pagina}} \\
		\hline
		\endfoot

		\hline
		\endlastfoot

    RF.O.22 & DebugMessage \\
    \hline RF.O.23 & DebugMessage \\
    \hline RF.O.24 & DebugMessage \\
  \end{longtable}
\end{adjustwidth}
\egroup
