\section{Introduzione}
\label{sec:introduzione}

\subsection{Scopo del documento}
\par Il presente documento è concepito per fornire una panoramica dettagliata delle funzionalità del prodotto ChatSQL. Attraverso questa documentazione, l'utente avrà l'opportunità di comprendere:
\begin{itemize}
  \item Se il prodotto è adeguato alle sue esigenze;
  \item Come utilizzare il prodotto per determinati scopi.
\end{itemize}

\subsection{Riferimenti}
\par Il presente documento si basa su normative elaborate dal team, dall'ente proponente o da entità esterne, oltre a includere materiali informativi. Tali riferimenti sono elencati di seguito.

\subsubsection{Riferimenti normativi}
\begin{itemize}
  \item \NormeDiProgetto;
  \item Slide PD2 - Corso di Ingegneria del Software - Regolamento del progetto didattico:\\ \href{https://www.math.unipd.it/~tullio/IS-1/2023/Dispense/PD2.pdf}{https://www.math.unipd.it/\textasciitilde tullio/IS-1/2023/Dispense/PD2.pdf}.
\end{itemize}

\subsubsection{Riferimenti informativi}
\begin{itemize}
  \item Capitolato C9 - ChatSQL:
  \begin{itemize}
    \item \href{https://www.math.unipd.it/~tullio/IS-1/2023/Progetto/C9.pdf}{https://www.math.unipd.it/\textasciitilde tullio/IS-1/2023/Progetto/C9.pdf} \\ (Ultimo accesso: 2024-07-02);
    \item \href{https://www.math.unipd.it/~tullio/IS-1/2023/Progetto/C9p.pdf}{https://www.math.unipd.it/\textasciitilde tullio/IS-1/2023/Progetto/C9p.pdf} \\ (Ultimo accesso: 2024-07-02).
  \end{itemize}
  \item \AnalisiDeiRequisiti;
  \item \Glossario;
  \item Verbali interni:
  \begin{itemize}
    \item 2024-07-26;
    \item 2024-08-01;
    \item 2024-08-08;
    \item 2024-08-14;
    \item 2024-08-19;
    \item 2024-08-27.
  \end{itemize}
  \item Verbali esterni:
  \begin{itemize}
    \item 2024-09-09.
  \end{itemize}
\end{itemize}

\subsection{Glossario} 
\GlossarioIntroduzione

