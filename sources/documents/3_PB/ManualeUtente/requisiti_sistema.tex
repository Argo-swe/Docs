\section{Requisiti di sistema}

\subsection{Requisiti software}
\par Per eseguire correttamente l'applicazione, è necessario avere \glossario{Docker} installato. I requisiti specifici dipendono dal sistema operativo in uso:
\begin{itemize}
  \item \textbf{Windows}
  \begin{enumerate}
    \item \textbf{Docker Desktop}
    \begin{itemize}
      \item Docker Desktop per Windows include Docker Engine, Docker Compose e tutte le componenti necessarie per eseguire i container;
      \item Maggiori informazioni sono disponibili qui: \href{https://docs.docker.com/desktop/install/windows-install}{https://docs.docker.com/des\- ktop/install/windows-install}.
    \end{itemize}
    \item \textbf{WSL 2 (consigliato)}
    \begin{itemize}
      \item Docker Desktop si integra con Windows Subsystem for Linux 2 (WSL 2) per eseguire container Linux in modo più efficiente;
      \item Maggiori informazioni sono disponibili qui: \href{https://docs.microsoft.com/en-us/windows/wsl/install}{https://docs.microsoft.com/en-us/windows/wsl/install}.
    \end{itemize}
  \end{enumerate}
  \item \textbf{macOS}
  \begin{enumerate}
    \item \textbf{Docker Desktop}
    \begin{itemize}
      \item Docker Desktop per macOS include Docker Engine, Docker Compose e tutte le componenti necessarie per eseguire i container;
      \item Maggiori informazioni sono disponibili qui: \href{https://docs.docker.com/desktop/install/mac-install}{https://docs.docker.com/des\- ktop/install/mac-install}.
    \end{itemize}
  \end{enumerate}
  \item \textbf{Linux}
  \begin{enumerate}
    \item \textbf{Docker Desktop}
    \begin{itemize}
      \item Docker Desktop per Linux include Docker Engine, Docker Compose e tutte le componenti necessarie per eseguire i container;
      \item Maggiori informazioni sono disponibili qui: \href{https://docs.docker.com/desktop/install/linux}{https://docs.docker.com/desktop/install/linux}.
    \end{itemize}
    \item \textbf{Docker Engine} (alternativa a Docker Desktop)
    \begin{itemize}
      \item Su Linux, è generalmente preferibile installare direttamente Docker Engine piuttosto che Docker Desktop, poiché offre un'integrazione più semplice, leggera e nativa con il sistema operativo;
      \item Maggiori informazioni sono disponibili qui: \href{https://docs.docker.com/engine/install}{https://docs.docker.com/engine/install}.
    \end{itemize}
    \item \textbf{Docker Compose} (alternativa a Docker Desktop)
    \begin{itemize}
      \item Se Docker Engine viene installato separatamente su Linux, è necessario installare anche Docker Compose per orchestrare più container;
      \item Maggiori informazioni sono disponibili qui: \href{https://docs.docker.com/compose/install}{https://docs.docker.com/compose/install}.
    \end{itemize}
  \end{enumerate}
\end{itemize}

\subsection{Requisiti hardware}
\begin{itemize}
  \item \textbf{RAM}
  \begin{itemize}
    \item \textbf{Minimo}: 4 GB;
    \item \textbf{Consigliato}: 8 GB o più (per un'esperienza ottimale).
  \end{itemize}
\end{itemize}

\subsection{Browser supportati}
\par Di seguito sono elencati i browser in cui l'applicazione è accessibile e fruibile.

\begin{table}[H]
  \centering
  \begin{tabular}{|c|c|}
      \hline
      \textbf{Browser} & \textbf{Versione} \\
      \hline
      Google Chrome & 110 e successive \\
      \hline
      Mozilla Firefox & 109 e successive \\
      \hline
      Safari & 15 e successive \\
      \hline
      Opera & 94 e successive \\
      \hline
      Microsoft Edge & 110 e successive \\
      \hline
  \end{tabular}
  \caption{Tabella dei browser supportati}
\end{table}