\section{Installazione}

\par Per utilizzare l'applicazione, il primo passo è copiare il codice sorgente in una cartella dedicata. Il \glossario{repository} è disponibile al seguente indirizzo:

\quad \href{https://github.com/Argo-swe/ChatSQL/tree/v2.0.0}{ChatSQL - v2.0.0} \newline

\par Per scaricare il codice sorgente vi sono due modalità principali:
\begin{enumerate}
  \item Scaricare il repository in formato zip e decomprimerlo in una cartella dedicata;
  \item Clonare il repository tramite il seguente comando:
  \begin{itemize} 
    \item \texttt{git clone https://github.com/Argo-swe/ChatSQL.git}
  \end{itemize}
\end{enumerate}

\vspace{0.5\baselineskip}
\par Una volta scaricato il codice dell'applicazione, è necessario navigare nelle cartelle \textit{backend} e \textit{frontend}, e aggiungere il file \textit{.env.local} in entrambe le cartelle. Questo file consente all'utente di impostare le variabili d'ambiente.