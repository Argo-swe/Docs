\subsubsection{RO6: Risorse disponibili ma non impiegate}
\begin{itemize}
    \item \textbf{Probabilità}: Alta;
    \item \textbf{Grado di criticità}: Alto;
    \item \textbf{Descrizione}: Alcuni ruoli, specialmente il \Verificatore{} e l'\Amministratore{}, potrebbero attraversare dei periodi in cui le attività a loro assegnate non richiedono uno sforzo produttivo e un'intensità di lavoro equiparabili ad altri incarichi. Perciò, il rischio è che uno o più membri del gruppo disperdano delle ore potenzialmente produttive.
    \item \textbf{Stragie di rilevamento}: Il team ha creato per ogni \glossario{sprint} una tabella di rendicontazione delle ore, da compilare quotidianamente con le attività svolte, il ruolo assunto e il numero di ore produttive. Consultando la tabella, il \Responsabile{} può osservare quante attività sono state completate per ogni ruolo. Alla luce del lavoro svolto rispetto allo \glossario{sprint backlog}, il \Responsabile{} valuta se il carico è adeguato o se c'è una disponibilità di risorse non sfruttata;
    \item \textbf{Contromisure}: La lista delle attività da svolgere è disponibile a tutti i membri del team, i quali possono prendere in carico task non ancora assegnati, anche in ruoli differenti. Perciò, su richiesta del \Responsabile{} o di propria iniziativa, ciascun componente è chiamato a impiegare produttivamente il proprio "tempo libero". Ciò non significa soltanto essere assegnati ad attività non ancora designate, ma anche suddividere task onerosi in sotto-attività. 
\end{itemize}
