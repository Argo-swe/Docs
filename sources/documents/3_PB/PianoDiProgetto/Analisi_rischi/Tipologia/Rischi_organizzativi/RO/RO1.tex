\subsubsection{RO1: Periodi di rallentamento}
\begin{itemize}
    \item \textbf{Probabilità}: Media;
    \item \textbf{Grado di criticità}: Medio;
    \item \textbf{Descrizione}: Il team ha previsto dei periodi di rallentamento dovuti a fattori esterni (festività, sessione di esami). In questi periodi, è prevista una diminuzione dell'intensità lavorativa che, se non gestita adeguatamente, potrebbe interferire con il rispetto delle scadenze;
    \item \textbf{Stragie di rilevamento}: Monitoraggio continuo, da parte del \Responsabile, delle ore impiegate da ciascun membro del team. Inoltre, il \Responsabile{} deve controllare il calendario per verificare la vicinanza a date particolari (incluse quelle indicate in tabella);
    \item \textbf{Contromisure}: La stima della data di consegna è stata formulata considerando questo rischio. Tuttavia, in caso di rallentamenti, il \Responsabile{} dovrà procedere prontamente alla riallocazione delle risorse, al fine di prevenire sovrapposizioni con rischi tecnologici che potrebbero avere un impatto significativo sul progetto.
\end{itemize}

\noindent\begin{minipage}{\textwidth}
\begin{table}[H]
    \centering
    \begin{tabular}{|P{4cm}|P{4cm}|>{\arraybackslash}P{4cm}|}
        \hline
        \textbf{Periodo} & \textbf{Da} & \textbf{A} \\
        \hline
        Pasquale & 2024-03-29 & 2024-04-02 \\
        \hline
        Ponte 25 aprile & 2024-04-25 & 2024-04-28 \\
        \hline
        Sessione estiva & 2024-06-17 & 2024-07-20 \\
        \hline 
        Sessione autunnale & 2024-08-19 & Termine progetto \\
        \hline
    \end{tabular}
    \caption{Tabella dei periodi di rallentamento}\label{tab:periodi-rallentamento}
\end{table}
\end{minipage}
