\subsubsection{RO5: Sovraccarico di risorse}
\begin{itemize}
    \item \textbf{Probabilità}: Media;
    \item \textbf{Grado di criticità}: Medio;
    \item \textbf{Descrizione}: Le risorse allocate superano la loro capacità ottimale. Questo può derivare da una pianificazione inadeguata, da un cambiamento tecnologico o da variazioni impreviste nei requisiti del progetto. Il rischio è che si verifichi un aumento non giustificato dei costi, sia diretti che indiretti, per far fronte al sovraccarico. Inoltre, un'eccessiva pressione sul team può portare a burnout, riducendo la produttività e aumentando il tasso di errori;
    \item \textbf{Stragie di rilevamento}: Il \Responsabile{} esegue un controllo regolare sullo stato di avanzamento del progetto, consultando i diagrammi di Gantt e la tabella di rendicontazione delle ore. Inoltre, ha il compito di raccogliere feedback dai membri del team riguardo la loro percezione del carico di lavoro;
    \item \textbf{Contromisure}: Ciascun membro del gruppo si impegna a compilare quotidianamente la tabella di rendicontazione delle ore, al fine di agevolare il monitoraggio delle risorse. Inoltre, la pianificazione delle attività deve considerare le capacità del team e includere margini di flessibilità, evitando di sovraccaricare le risorse. Per migliorare le competenze del team, viene attuato il piano di formazione definito nelle \NormeDiProgetto.
\end{itemize}
