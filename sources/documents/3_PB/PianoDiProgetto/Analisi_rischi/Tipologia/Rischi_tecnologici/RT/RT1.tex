\subsubsection{RT1: Scarso know-how tecnologico}
\begin{itemize}
    \item \textbf{Probabilità}: Alta;
    \item \textbf{Grado di criticità}: Alto;
    \item \textbf{Descrizione}: La maggior parte dei componenti del gruppo non ha mai lavorato con \glossario{Python} e \glossario{LaTeX}, i due linguaggi scelti rispettivamente per lo sviluppo del back-end e la stesura dei documenti. Inoltre, nessun membro del team ha esperienza con gli strumenti e le librerie suggerite dalla \glossario{Proponente}. Pertanto, l'avanzamento del progetto rischia di subire rallentamenti dovuti all'apprendimento delle nuove tecnologie. Con questo si intendono anche fasi di esplorazione di tecnologie che possono risultare più vantaggiose rispetto a quelle già utilizzate;
    \item \textbf{Stragie di rilevamento}: Il primo passo consiste in una serie di incontri, di breve durata, atti a valutare le competenze tecniche e l'esperienza del team relativamente a ciascuna tecnologia. Inoltre, è prevista un'analisi collaborativa per stimare la curva di apprendimento. A questo si aggiunge poi una valutazione delle risorse disponibili, specialmente quelle temporali.
    \begin{itemize}
        \item Nel corso dello \glossario{sprint}, il \Responsabile{} si impegna a monitorare constantemente le attività e raccogliere feedback dai singoli membri;
        \item Come strategia di rilevamento, il team ha introdotto anche la \glossario{continuous integration}. Tale pratica prevede un allineamento frequente con l'ambiente condiviso (\glossario{GitHub}) e consente al gruppo di individuare eventuali difficoltà nell'uso delle tecnolgoie e prevenire la propagazione degli errori. 
    \end{itemize}
    \item \textbf{Contromisure}: Considerando l'inesperienza del gruppo e la continua evoluzione delle tecnologie proposte, il rischio tecnologico non può essere totalmente scongiurato. Tuttavia, il team intende lavorare per mitigare i problemi ed evitare rallentamenti sfavorevoli. La prima contromisura prevede lo studio individuale da parte di un gruppo ristretto di risorse, così da non dover sospendere le attività in corso e non pregiudicare l'avanzamento dello \glossario{sprint}. I materiali di studio spaziano dalla documentazione ufficiale delle tecnologie a tutti gli strumenti collaterali. Una volta ultimato l'apprendimento di una determinata tecnologia, i membri interessati terranno un workshop per uniformare le conoscenze del gruppo. Le misure di mitigazione comprendono anche:
    \begin{itemize}
        \item Incontri di formazione con la \glossario{Proponente}, qualora dovessero sorgere dei dubbi sulle tecnologie proposte;
        \item Workshop preventivi, nel caso in cui le tecnologie siano già note ad alcuni membri del team.
    \end{itemize}
    \par Se un componente del gruppo dovesse riscontrare comunque delle difficoltà, quest'ultimo verrà affiancato nelle prime fasi da un membro più esperto del team. Il gruppo, inoltre, non esclude la possibilità di impiegare tecnologie alternative, o di supporto, che possano incrementare l'\glossario{efficienza} del progetto.
\end{itemize}
