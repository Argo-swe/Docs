\section{Analisi dei rischi}
\label{sec:analisi_rischi}

\subsection{Ultimo aggiornamento: 2024-07-25}

\par In questa sezione vengono esaminate le situazioni problematiche che possono verificarsi durante lo svolgimento del progetto. Ciascun rischio è corredato da:
\begin{itemize}
  \item Probabilità di occorrenza;
  \item Grado di criticità;
  \item Descrizione;
  \item Strategie di rilevamento;
  \item Contromisure.
\end{itemize}

\vspace{0.5\baselineskip}
\par La notazione utilizzata per identificare i rischi è riportata nelle \NormeDiProgetto\ (§Gestione dei rischi). L'analisi dei rischi influenza la pratica di gestione dei rischi, che è documentata nel consuntivo di periodo (\sezione{sec:consuntivo}). La gestione dei rischi fornisce un riscontro in merito a:
\begin{itemize}
  \item Occorrenza effettiva di un rischio;
  \item Attuazione delle misure di mitigazione previste;
  \item Valutazione dell'impatto del rischio. 
\end{itemize}

\vspace{0.5\baselineskip}
\par Dalla gestione dei rischi deriva una valutazione critica dell’efficacia delle misure di mitigazione previste e attuate, che porta all’aggiornamento e al miglioramento dell’analisi iniziale.

\subsection{Rischi tecnologici}

\subsubsection{RT1: Scarso know-how tecnologico}
\begin{itemize}
    \item \textbf{Probabilità:} Alta;
    \item \textbf{Grado di criticità:} Alto;
    \item \textbf{Descrizione:} Utilizzo di tecnologie sconosciute. Nessun membro del gruppo ha
    esperienze pregresse con gli strumenti e le librerie suggerite dalla Proponente; di conseguenza, l’avanzamento del progetto rischia di subire rallentamenti dovuti a fasi di apprendimento delle nuove tecnologie.
    Con questo si intendono anche fasi di esplorazione di tecnologie che possono risultare più vantaggiose rispetto a quelle utilizzate al momento;
    \item \textbf{Stragie di rilevamento:} Il primo passo consiste in una serie di incontri, di breve durata, atti a valutare le competenze tecniche e l'esperienza del team su ciascuna tecnologia. Inoltre, è necessaria un'analisi collaborativa per stimare la curva di apprendimento. A questo si aggiunge poi una valutazione delle risorse disponibili, specialmente quelle temporali.
    \begin{itemize}
        \item Nel corso dello \glossario{sprint}, il responsabile si impegna a monitorare constantemente le attività e raccogliere feedback dai singoli membri;
        \item Come strategia di rilevamento, il team ha introdotto anche la \glossario{continuous integration}. Tale pratica prevede un allineamento frequente con l'ambiente condiviso (\glossario{GitHub}) e consente al gruppo di individuare eventuali difficoltà nell'uso delle tecnolgoie e prevenire la propagazione degli errori. 
    \end{itemize}
    \item \textbf{Contromisure:} Considerando l'inesperienza del gruppo e la continua evoluzione delle tecnologie proposte, il rischio tecnologico non può essere totalmente scongiurato. Tuttavia, il team intende lavorare per mitigare i problemi ed evitare rallentamenti sfavorevoli. La prima contromisura prevede lo studio individuale da parte di un gruppo ristretto di risorse, così da non dover sospendere le attività in corso e non pregiudicare l'avanzamento dello \glossario{sprint}. Una volta ultimato l'approfondimento di una determinata tecnologia, i membri interessati terranno un workshop per uniformare le conoscenze del gruppo. Di seguito sono elencate altre misure di mitigazione:
    \begin{itemize}
        \item 
    \end{itemize}
    \par In caso di difficoltà, il team valuterà la possibilità di impiegare tecnologie alternative, o di supporto, che possano incrementare l'\glossario{efficienza} del progetto.
\end{itemize}

\noindent\begin{minipage}{\textwidth}
\subsubsection{RT2: Malfunzionamenti hardware}

\bgroup
\begin{adjustwidth}{-0.5cm}{-0.5cm}
	% MAX 12.5cm
 	\begin{longtable}{P{4.5cm}|>{\justifying \arraybackslash}P{9cm}}

		\textbf{Probabilità} & Bassa \\
        \hline
        \textbf{Grado di criticità} & Basso \\
        \hline
        \textbf{Descrizione} & Possibili malfunzionamenti hardware imprevedibili delle macchine dei membri del gruppo \\
        \hline
        \textbf{Strategie di rilevamento} & Controllo ripetuto da parte di ciascun componente del gruppo sulla propria strumentazione e segnalazione tempestiva in casi di guasto. \\
        \hline
        \textbf{Contromisure} & Svolgimento del lavoro con ripetuti aggiornamenti su sistemi di versionamento, al fine di avere un ambiente condiviso e limitare la perdita di informazioni.  
	\end{longtable}
\end{adjustwidth}
\egroup
\end{minipage}
\subsubsection{RT3: Malfunzionamenti software}
\begin{itemize}
    \item \textbf{Probabilità}: Alta;
    \item \textbf{Grado di criticità}: Medio;
    \item \textbf{Descrizione}: L'utilizzo di tecnologie di terze parti, come ad esempio \glossario{Docker}, potrebbe comportare delle interruzioni impreviste dovute a malfunzionamenti software (bug, problemi di prestazioni, incompatibilità di versioni, aggiornamenti non riusciti). Questo rischio, se non gestito correttamente, può avere un impatto negativo sulla produttività a causa di tempi di fermo prolungati;
    \item \textbf{Stragie di rilevamento}: Segnalazione immediata di errori o comportamenti anomali;
    \item \textbf{Contromisure}: Durante le riunioni interne, il team riserva del tempo per affrontare e risolvere eventuali criticità di natura software. Qualora un problema dovesse persistere, verrà organizzato un incontro più breve mirato alla sua risoluzione.
\end{itemize}
\subsubsection{RT4: Connettività limitata}
\begin{itemize}
    \item \textbf{Probabilità}: Media;
    \item \textbf{Grado di criticità}: Medio;
    \item \textbf{Descrizione}: Difficoltà o impossibilità di accedere a Internet. Questo rischio può complicare la partecipazione dei membri interessati alle riunioni, sia interne che esterne;
    \item \textbf{Stragie di rilevamento}: Comunicazione chiara e tempestiva attraverso strumenti di messaggistica;
    \item \textbf{Contromisure}: Organizzazione di incontri in presenza con frequenza variabile, per usufruire della connessione offerta dai laboratori universitari. In aggiunta, il team ha configurato tre piattaforme di comunicazione (Discord, Zoom e Google Meet), in modo tale da disporre di un fallback in caso di necessità.
\end{itemize}

\subsubsection{RT5: Cambio di tecnologie}
\begin{itemize}
    \item \textbf{Probabilità}: Bassa;
    \item \textbf{Grado di criticità}: Alto;
    \item \textbf{Descrizione}: Il cambiamento tecnologico può introdurre una serie di sfide e ostacoli, come la curva di apprendimento e la necessità di allocare risorse aggiuntive per l'implementazione. Nonostante la bassa probabilità di tale rischio, il suo grado di criticità è alto, poiché può provocare interruzioni operative e ritardi nella consegna. Vi è inoltre il rischio di incompatibilità con le soluzioni esistenti, portando a malfunzionamenti;
    \item \textbf{Stragie di rilevamento}: Il responsabile, insieme ad altre figure preposte, raccoglie costantemente feedback dai membri coinvolti nello sviluppo;
    \item \textbf{Contromisure}: Pianificazione dettagliata del processo di cambiamento, analizzandone gli impatti e includendo fallback. Inoltre, viene formata una squadra dedicata al supporto durante il periodo di transizione.
\end{itemize}

\subsection{Rischi organizzativi}

\noindent\begin{minipage}{\textwidth}
\subsubsection{RO1: Rischi relativi a rallentamenti}

\bgroup
\begin{adjustwidth}{-0.5cm}{-0.5cm}
    % MAX 12.5cm
    \begin{longtable}{P{4.5cm}|>{\justifying \arraybackslash}P{9cm}}

        \textbf{Probabilità} & Alta \\
        \hline
        \textbf{Grado di criticità} & Basso \\
        \hline
        \textbf{Descrizione} & Durante il corso del progetto saranno presenti periodi di rallentamento dovuti a fattori esterni (giorni festivi, impegni studenteschi) o altri al
        momento non prevedibili. \\
        \hline
        \textbf{Strategie di rilevamento} & Monitoraggio continuo delle ore dedicate dai membri
        del gruppo; controllo del calendario per verificare la vicinanza a date particolari
        (tra cui quelle indicate in tabella). \\
        \hline
        \textbf{Contromisure} &  La stima della data di consegna è stata adattata tenendo conto
        di questo rischio.  
    \end{longtable}
\end{adjustwidth}
\egroup
\end{minipage}

\noindent\begin{minipage}{\textwidth}
\bgroup
\begin{adjustwidth}{-0.5cm}{-0.5cm}
    \begin{longtable}{|P{4cm}|P{4cm}|>{\arraybackslash}P{4cm}|}
        \hline
        \textbf{Periodo} & \textbf{Da} & \textbf{A} \\
        \hline
        Pasquale & 2024-03-29 & 2024-04-02 \\
        \hline
        Ponte 25 Aprile & 2024-04-25 & 2024-04-28 \\
        \hline
        Sessione estiva & 2024-06-17 & 2024-07-20 \\
        \hline 
        Sessione autunnale & 2024-08-19 & Termine progetto \\
        \hline
    \end{longtable}
\end{adjustwidth}
\egroup
\end{minipage}

\noindent\begin{minipage}{\textwidth}
\subsubsection{RO2: Rischi relativi all'inesperienza}
    
\bgroup
\begin{adjustwidth}{-0.5cm}{-0.5cm}
    % MAX 12.5cm
    \begin{longtable}{P{4.5cm}|>{\justifying \arraybackslash}P{9cm}}

        \textbf{Probabilità} & Bassa \\
        \hline
        \textbf{Grado di criticità} & Alto \\
        \hline
        \textbf{Descrizione} & Idee, metodologie e tempistiche di lavoro potrebbero essere diversi tra i vari membri del gruppo, con conseguenti divergenze da sanare per un ottimale proseguimento \\
        \hline
        \textbf{Strategie di rilevamento} &  Il responsabile in carica ha il compito di monitorare costantemente le attività e le relazioni tra le attività. Ha inoltre il compito di coordinare il gruppo al fine di rilevare potenziali discrepanze o divergenze. \\
        \hline
        \textbf{Contromisure} & La suddivisione dei compiti e una stesura ottimale di un way of working condiviso dai membri, possono prevenire questo tipo di rischio.
    \end{longtable}
\end{adjustwidth}
\egroup
\end{minipage}
\subsubsection{RO3: Sottostima delle risorse necessarie per un'attività}
\begin{itemize}
    \item \textbf{Probabilità:} Alta;
    \item \textbf{Grado di criticità:} Alto;
    \item \textbf{Descrizione:} La complessità del progetto e l'inesperienza del gruppo potrebbero portare a una sottostima delle risorse, economiche e temporali, necessarie per completare determinate attività;
    \item \textbf{Stragie di rilevamento:} In caso di valutazioni errate o difficoltà nella gestione del carico di lavoro, ciascun componente deve informare tempestivamente il resto del gruppo. La discussione delle criticità può avvenire sia nella chat Telegram che durante le riunioni interne. Con questa procedura, il team ritiene di poter rilevare i problemi celermente e di poter mitigare gli impatti negativi sui task successivi;
    \item \textbf{Contromisure:} Se lo sforzo produttivo necessario per completare un task risulta sottostimato, l'attività viene suddivisa in sotto-task. Ciascuna sotto-attività può essere assegnata a membri diversi del team, al fine di portare a termine l’incarico col minor ritardo possibile. Inoltre, chi ha esperienza si impegna a fornire assistenza per minimizzare l'impatto sui task successivi. In alternativa, un’attività non particolarmente urgente può essere suddivisa in sotto-task con priorità diversa. Così facendo il team può prolungare la finestra temporale di esecuzione del task, concentrandosi sulle sotto-attività a priorità più alta.
\end{itemize}

\subsubsection{RO4: Rotazione dei ruoli}
\begin{itemize}
    \item \textbf{Probabilità}: Alta;
    \item \textbf{Grado di criticità}: Alto;
    \item \textbf{Descrizione}: Ciascun componente del gruppo è tenuto ad assumere, a rotazione, tutti i ruoli previsti dal regolamento del progetto. Questa pratica differisce dall'approccio abitualmente adottato nella realizzazione di progetti didattici. Pertanto, esiste la possibilità che i membri del team riscontrino delle difficoltà all’inizio di ogni sprint. L’impatto di tale rischio può risultare negativo sul lungo periodo, quando ciascun componente dovrà prendere in carico il lavoro (sempre più voluminoso) svolto da altri membri del gruppo;
    \item \textbf{Stragie di rilevamento}: Confronto costante attraverso i canali di comunicazione stabiliti nelle \NdP. Per rilevare eventuali rallentamenti dovuti alla rotazione dei ruoli, il responsabile può consultare la tabella di rendicontazione delle ore;
    \item \textbf{Contromisure}: Nelle \NdP\ sono elencati i compiti assegnati a ciascun ruolo in base al regolamento del progetto. Durante le riunioni di pianificazione e retrospettiva, le mansioni generali (definite dal \WoW) vengono integrate con le attività specifiche relative allo sprint corrente. In aggiunta, il team organizza brevi riunioni dedicate alla rotazione dei ruoli, alle quali partecipano unicamente i membri interessati.
\end{itemize}

\subsubsection{RO5: Rischi relativi al preventivo}
\begin{itemize}
    \item \textbf{Probabilità:} Media;
    \item \textbf{Grado di criticità:} Alto;
    \item \textbf{Descrizione:} Forte variazione tra preventivo e consuntivo, con relativo aumento
    dei costi;
    \item \textbf{Stragie di rilevamento:} Controllo periodico dello stato di avanzamento delle
    attività, rendicontazione delle ore tramite una tabella condivisa che riesca ad indicare le ore produttive svolte nell'attività;
    \item \textbf{Contromisure:} Preventivare le attività tenendo conto di una fase di allentamento. Facendo ciò si cercano di anticipare e prevenire eventuali imprevisti, rimanendo più laschi sul preventivo iniziale. 
\end{itemize}


\subsection{Rischi di natura personale}

\subsubsection{RP1: Questioni personali}
\begin{itemize}
    \item \textbf{Probabilità:} Bassa;
    \item \textbf{Grado di criticità:} Alta;
    \item \textbf{Descrizione:} La necessità di un periodo di fermo per questioni personali, seppur giustificate, potrebbe condurre il team a una fase di stallo;
    \item \textbf{Stragie di rilevamento:} Comunicazione tempestiva da parte dei soggetti interessati;
    \item \textbf{Contromisure:} Il team si impegna a ridistribuire i task assegnati ai membri indisponbili previa individuazione e proroga di attività ritenute meno urgenti. In caso di necessità, il gruppo attuerà un'opportuna riallocazione delle risorse temporali. I soggetti interessati saranno tenuti ad allineare le proprie ore produttive con il resto del gruppo.
\end{itemize}