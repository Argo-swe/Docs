\section{Riunione}
\subsection{Ordine del Giorno}
\begin{itemize}
	\item Miglioramento del documento "Stima dei costi e assunzione impegni";
	\item Definizione di una timeline con la pianificazione dei macro-periodi;
	\item Modifica del formato delle date e ordinamento dei file sul sito in ordine decrescente, in modo che compaiano in cima i documenti più recenti.
\end{itemize}

\subsection{Discussione e decisioni}

\subsubsection{Operazioni sul documento "Stima dei costi e assunzione impegni"}
Allo scopo di integrare una timeline relativa ai periodi di consegna, è stata definita una sotto-sezione apposita all’interno del documento. Viene inoltre stilata una lista dei documenti che si sostiene verranno realizzati nei diversi periodi.

\subsubsection{Definizione di una timeline con la pianificazione dei periodi}
La timeline definita al punto precedente include i due periodi di consegna più rilevanti, con una suddivisione più dettagliata del periodo relativo alla RTB. L’intervallo individuato è quello tra il 27 maggio e il 7 giugno, mentre la stima per la realizzazione della PB è fissata per il 13 settembre. \\
Viene discussa, inoltre, una pianificazione sul primo periodo, riguardante uno sprint di studio delle tecnologie e la produzione del POC.

\subsubsection{Formato delle date e collocazione documenti}
A seguito delle indicazioni ricevute dal Committente riguardo l’ordinamento lessicografico delle date e la loro organizzazione, sono state apportate le seguenti modifiche:
\begin{itemize}
	\item Adottato il formato AAAA-MM-GG;
	\item I documenti più recenti vengono inseriti in cima alla lista nel sito \href{https://argo-swe.github.io}{argo-swe.github.io}.
\end{itemize}
\clearpage
