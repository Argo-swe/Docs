%%%%%%%%%%%%%%
%  COSTANTI  %
%%%%%%%%%%%%%%

% In questa prima parte vanno definite le 'costanti' utilizzate soltanto da questo documento.
% Devono iniziare con una lettera maiuscola per distinguersi dalle funzioni.

\newcommand{\DocTitle}{Verbale Riunione 2024-03-06}
\newcommand{\DocVersion}{1.0.0}
\newcommand{\DocRedazione}{\riccardo}
\newcommand{\DocVerifica}{\raul}
\newcommand{\DocApprovazione}{\sebastiano}
\newcommand{\DocUso}{Interno}
\newcommand{\DocDistribuzione}{
	\Committente{} \\
	Gruppo \GroupName{}
}

% La descrizione del documento
\newcommand{\DocDescription}{
Durante il primo incontro, per lo più conoscitivo, il gruppo ha discusso gli impegni universitari e la disponibilità dei singoli membri. Questo per stabilire le fasce orarie in cui è generalmente garantita la presenza completa del team. In seguito, si è cominciato a stilare il profilo del gruppo (nome, logo, ecc.), a formulare il way of working e a stabilire un indice di gradimento dei capitolati.
}

%%%%%%%%%%%%%%
%  FUNZIONI  %
%%%%%%%%%%%%%%

% In questa seconda parte vanno definite le 'funzioni' utilizzate soltanto da questo documento.
