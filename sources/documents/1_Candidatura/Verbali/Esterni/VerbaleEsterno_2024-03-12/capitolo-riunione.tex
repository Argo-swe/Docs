\section{Riunione}
\subsubsection{Ritrovo iniziale}
Circa 30 minuti prima dell’incontro, il gruppo ha avviato un breve meeting privato per discutere la correttezza e l’ordine dei quesiti da porre al proponente. Dopo aver formulato una scaletta coerente con gli argomenti, il gruppo ha partecipato alla riunione avviata dai referenti di Sync Lab. Il meeting è iniziato con una rapida presentazione dell’azienda e del capitolato, di cui sono stati forniti esempi concreti utili alla comprensione del problema. In seguito, la Proponente ha chiarito i dubbi esposti dal responsabile del gruppo Argo.

\subsection{Argomenti e temi dell'incontro}

\subsubsection{Panoramica dell’azienda e del capitolato}
\textbf{Domanda:} Da dove è nata l’idea del capitolato e, più precisamente, quali sono stati gli spunti o le sfide che hanno portato alla definizione del capitolato?

\textbf{Risposta:} [Breve presentazione dell’azienda] Dopo vent'anni di attività, Sync Lab S.r.l. conta, ad oggi, 300 risorse e vanta una distribuzione su tutto il territorio. Sono state citate spesso le parole chiave monitoraggio e integrazione. Il capitolato fa infatti riferimento a un ambito su cui l’azienda lavora da un po' di tempo: il monitoraggio, che nel nostro caso riguarda l’ambiente. Si tratta fondamentalmente di acquisire dati da dispositivi IoT e sensori (ad esempio sensori di temperatura), per poi organizzarli e renderli disponibili agli utenti mediante interfacce grafiche (dashboard).
La Proponente suggerisce anche di simulare sensori di parcheggio. In ambito smart city, è fondamentale introdurre il concetto di smart parking, di cui si menzionano i seguenti aspetti:
\begin{itemize}
	\item Gestione di entrate e uscite dal parcheggio;
	\item Acquisizione di parametri ambientali;
	\item Transazioni per sosta;
	\item Controllo real-time della correttezza del pagamento sulla base del tempo di parcheggio.
\end{itemize}
L’azienda promuove la ricerca e la sperimentazione di nuove soluzioni rispetto a quelle indicate in capitolato nel rispetto dei requisiti funzionali previsti.

\subsubsection{IoT e sensori}
Il focus del capitolato è principalmente sul lato back-end: l’informazione si trova nella “parte bassa” (sensori, sistemi di controllo, smart parking). I sensori vengono solamente simulati, in quanto la parte fisica è demandata ad altre entità. Il sistema deve essere in grado di elaborare le informazioni indipendentemente dal sensore, senza che sorgano problemi sul formato o sul tipo di dato. Tra tutti i possibili linguaggi, Python è stato indicato come scelta ottimale per la generazione dei dati emessi dal sensore. Trattandosi di un linguaggio piuttosto diffuso e popolare nella comunità informatica, l’impiego di librerie volte a generare dati realistici non dovrebbe risultare complesso. 
Inoltre, Python permette di inserire facilmente le informazioni in una coda Kafka (un broker-applicativo che acquisisce i dati e li accoda finché non vengono elaborati). Kafka dovrebbe accettare anche il protocollo MQTT, già sperimentato in passato da alcuni componenti del team Argo. Se il gruppo volesse utilizzare il protocollo di cui sopra, la Proponente si dichiara aperta a soddisfare la richiesta.

\subsubsection{ClickHouse}
\textbf{Domanda:} Da cosa deriva la scelta di utilizzare ClickHouse?

\textbf{Risposta:} Innanzitutto, i DBMS si suddividono in due macro-categorie: relazionali (SQL) e non relazionali (NoSQL). ClickHouse rientra nella seconda definizione: si tratta di un DBMS colonnare, scalabile, capace di gestire grandi moli di dati ad alta velocità (anche a costo di aggiungere ridondanza). Se tradizionalmente si fa riferimento a tabelle in forma normale, in questo caso si fa l’esatto opposto, dato che all’interno di una tupla vanno inserite più informazioni possibili. La scelta di ClickHouse dipende quindi dalle prestazioni, oltre al fatto che le interrogazioni sono SQL like. Inoltre, per quanto complessa, si tratta di una tecnologia all’avanguardia e in continua crescita.

\subsubsection{Big data}
\textbf{Domanda:} Parlando di big data, qual è l’ordine di grandezza dei dati?

\textbf{Risposta:} In ambienti di produzione importanti (per esempio un sistema di smart parking), l’ordine di grandezza dei dati ruota attorno a 100 dati al secondo. Nell'ambito di SyncCity, invece, il simulatore può emettere qualche dato al secondo, molto dipende anche dai casi d'uso e dai limiti delle macchine locali. La Proponente menziona la possibilità di affrontare il problema dello smart parking, con annessi eventi di occupazione e uscita dalla piazzola, collegamento della targa al pagamento e verifica della transazione.
Bisogna capire se questo requisito non si discosta troppo dal focus del capitolato, in quanto sarebbe una verticalizzazione del capitolato stesso. Una soluzione avanzata dalla Proponente prevederebbe l’acquisizione di informazioni da più sensori (temperatura, umidità, polveri sottili), incluso ovviamente il sensore di parcheggio che emette dati sotto forma di bit 0/1. La verifica della transazione potrebbe quindi essere implementata durante un’eventuale finalizzazione del prodotto.

\subsubsection{Materiali di studio}
\textbf{Domanda:} Dove si potrebbero trovare dei materiali per studiare e approfondire le tecnologie consigliate?

\textbf{Risposta:} Nel documento di presentazione del capitolato sono riportati alcuni ottimi spunti di partenza. Qualora dovessero sorgere dubbi sulla configurazione degli strumenti, la Proponente mette a disposizione un canale dedicato su Discord. Un referente dell’azienda, specializzato nelle tecnologie richieste da SyncCity, si rende disponibile per chiarire dubbi o approfondire l’utilizzo di determinati strumenti.

\subsubsection{Rapporto fornitore-cliente}\label{rapporto fornitore-cliente}
Il rapporto tra fornitore e cliente deve essere costante, pertanto le riunioni ufficiali si terranno ogni due settimane o, in caso di necessità, addirittura settimanalmente, per verificare lo stato di avanzamento dei lavori (SAL). Gli incontri, da svolgere per via telematica o in presenza, vengono fissati in un calendario condiviso. Se il fornitore non riuscisse a rispettare il “To-Do” concordato nella riunione precedente, la Proponente chiede di essere informata con anticipo via mail. Nonostante la distanza tra la sede dell’azienda e l’Università di Padova, la partecipazione in presenza è consigliata.

\subsubsection{Riflessioni sul capitolato}
\textbf{Domanda:} Dopo aver seguito lo sviluppo di altri gruppi, la loro idea è rimasta la stessa? Hanno visto un’evoluzione nelle tecnologie utili o nella fattibilità?

\textbf{Risposta:} Le considerazioni dell’azienda sulla fattibilità e la validità del capitolato non sono cambiate. Anzi, i gruppi del primo lotto hanno realizzato un prodotto abbastanza aderente alle richieste, per questo la Proponente si chiedeva se cambiare un po' rotta, provando ad andare in verticale sullo smart parking. Il mondo sta virando verso una continua integrazione con dispositivi di tipo IoT, quindi è necessario avere strumenti per raccogliere la mole enorme di informazioni in circolazione e poterle elaborare, cosicché possano essere usufruite da aziende con determinate esigenze. Il problema dello smart parking si inserisce perfettamente all’interno di una visione concreta e reale della smart city.

\subsubsection{Ostacoli da affrontare}
\textbf{Domanda:} Quale parte del progetto potrebbe risultare più complessa?

\textbf{Risposta:} La simulazione dei sensori non dovrebbe risultare particolarmente dispendiosa. Lo stesso non si può dire per il passaggio Kafka-ClickHouse, che potrebbe richiedere una curva di apprendimento notevole. Altri aspetti che la Proponente ritiene potenzialmente complessi sono i seguenti:
La configurazione di ClickHouse;
la containerizzazione.
L’utilizzo di Docker è fortemente consigliato.

\subsubsection{Formazione}
\textbf{Domanda:} Che cosa si intende con sessioni deep-dive, e su quali tecnologie si concentra la formazione?

\textbf{Risposta:} Se dovessero servire dei chiarimenti su Kafka, per esempio, l'azienda si rende disponibile a fornire supporto telematico, o in sede, per la configurazione degli strumenti e l’approfondimento delle tecnologie.

\subsubsection{GitHub}
\textbf{Domanda:} GitHub è sufficiente come sistema di versionamento?

\textbf{Risposta:} GitHub è più che sufficiente ed è assolutamente lo strumento migliore per affrontare un progetto didattico di questo tipo.
